%
% Copyright � 2018 Peeter Joot.  All Rights Reserved.
% Licenced as described in the file LICENSE under the root directory of this GIT repository.
%
\makeproblem{Playing with the non-relativistic limit}{qft:problemSet2:4}{
Consider a real scalar relativistic field theory of mass m with \( \lambda \phi^4 \) interaction. Let there be \( N \) particles of momenta labeled by \( p_1,\cdots, p_N\), whose energies are such that they are insufficient to create any new particles. Nevertheless, the particles can scatter and exchange momenta. In what follows you will study this N-particle nonrelativistic limit in some detail.
\makesubproblem{}{qft:problemSet2:4a}
Write down the Hamiltonian of the field theory, including the interaction term, restricted to the N-particle sector of Hilbert space. (Use the creation and annihilation operator representation, i.e. write the result as sums of products of creation and annihilation operators of particles of various momenta.)
\makesubproblem{}{qft:problemSet2:4b}
Does the resulting Hamiltonian preserve particle number? Is there an associated symmetry? What is the operator that generates it?
\makesubproblem{}{qft:problemSet2:4c}
Consider now the interaction term in your reduced (to the N-particle sector of Hilbert space) Hamiltonian. How does a typical interaction term (for given configurations of momenta) act on an N-particle state? What kinds of scattering processes does it describe?
\makesubproblem{}{qft:problemSet2:4d}
What do you think is the potential, in x-space, that allows the various particles to scatter and exchange momentum? How would you describe the resulting nonrelativistic quantum system to friends who never took QFT but are well-versed in quantum mechanics?
} % makeproblem

\makeanswer{qft:problemSet2:4}{
\makeSubAnswer{}{qft:problemSet2:4a}
TODO.
\makeSubAnswer{}{qft:problemSet2:4a}
TODO.
\makeSubAnswer{}{qft:problemSet2:4b}
TODO.
\makeSubAnswer{}{qft:problemSet2:4c}
TODO.
\makeSubAnswer{}{qft:problemSet2:4d}
TODO.
}
