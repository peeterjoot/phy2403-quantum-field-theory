%
% Copyright � 2018 Peeter Joot.  All Rights Reserved.
% Licenced as described in the file LICENSE under the root directory of this GIT repository.
%
\makeproblem{Playing with the non-relativistic limit}{qft:problemSet2:4}{
Consider a real scalar relativistic field theory of mass m with \( \lambda \phi^4 \) interaction. Let there be \( N \) particles of momenta labeled by \( p_1,\cdots, p_N\), whose energies are such that they are insufficient to create any new particles. Nevertheless, the particles can scatter and exchange momenta. In what follows you will study this N-particle nonrelativistic limit in some detail.
\makesubproblem{}{qft:problemSet2:4a}
Write down the Hamiltonian of the field theory, including the interaction term, restricted to the N-particle sector of Hilbert space. (Use the creation and annihilation operator representation, i.e. write the result as sums of products of creation and annihilation operators of particles of various momenta.)
\makesubproblem{}{qft:problemSet2:4b}
Does the resulting Hamiltonian preserve particle number? Is there an associated symmetry? What is the operator that generates it?
\makesubproblem{}{qft:problemSet2:4c}
Consider now the interaction term in your reduced (to the N-particle sector of Hilbert space) Hamiltonian. How does a typical interaction term (for given configurations of momenta) act on an N-particle state? What kinds of scattering processes does it describe?
\makesubproblem{}{qft:problemSet2:4d}
What do you think is the potential, in x-space, that allows the various particles to scatter and exchange momentum? How would you describe the resulting nonrelativistic quantum system to friends who never took QFT but are well-versed in quantum mechanics?
} % makeproblem

\makeanswer{qft:problemSet2:4}{
\makeSubAnswer{}{qft:problemSet2:4a}
The Lagrangian density of a massive scalar field with a \( \lambda \phi^4 \) interaction has the form
\begin{dmath}\label{eqn:ProblemSet2Problem4:20}
\LL = \inv{2} \partial_\mu \phi \partial^\mu \phi - \inv{2} m^2 \phi^2 - \lambda \phi^4.
\end{dmath}
The corresponding Hamiltonian is
\begin{dmath}\label{eqn:ProblemSet2Problem4:40}
H = \inv{2} \int d^3x \lr{ \pi^2 + \frac{m^2}{2} (\spacegrad \phi)^2 + m^2 \phi^2 } + \lambda \int d^3 x \phi^4.
\end{dmath}
In terms of creation and annihilation operators, we know the form of the non-interaction portion of the Hamiltonian, which in normal order is
\begin{dmath}\label{eqn:ProblemSet2Problem4:60}
H_0 = \int \frac{d^3 p}{(2 \pi)^3} \omega_\Bp a_\Bp^\dagger a_\Bp,
\end{dmath}
but the interaction contribution is much messier
\begin{dmath}\label{eqn:ProblemSet2Problem4:80}
H_{\text{int}}
=
\lambda \int d^3 x \frac{ d^3 p d^3 k d^3 q d^3 s}{4 (2 \pi)^{12} \sqrt{
\omega_\Bp  \omega_\Bk  \omega_\Bq  \omega_\Bs
} }
\lr{ a_\Bp e^{-i p \cdot x} + a_\Bp e^{i p \cdot x} }
\lr{ a_\Bk e^{-i k \cdot x} + a_\Bk e^{i k \cdot x} }
\lr{ a_\Bq e^{-i q \cdot x} + a_\Bq e^{i q \cdot x} }
\lr{ a_\Bs e^{-i s \cdot x} + a_\Bs e^{i s \cdot x} }
=
\lambda \int d^3 x \frac{ d^3 p d^3 k d^3 q d^3 s}{4 (2 \pi)^{12} \sqrt{
\omega_\Bp  \omega_\Bk  \omega_\Bq  \omega_\Bs
} }
\lr{ a_\Bp e^{-i \omega_\Bp t + i \Bp \cdot \Bx} + a_\Bp e^{i \omega_\Bp t - i \Bp \cdot \Bx} }
\lr{ a_\Bk e^{-i \omega_\Bk t + i \Bk \cdot \Bx} + a_\Bk e^{i \omega_\Bk t - i \Bk \cdot \Bx} }
\lr{ a_\Bq e^{-i \omega_\Bq t + i \Bq \cdot \Bx} + a_\Bq e^{i \omega_\Bq t - i \Bq \cdot \Bx} }
\lr{ a_\Bs e^{-i \omega_\Bs t + i \Bs \cdot \Bx} + a_\Bs e^{i \omega_\Bs t - i \Bs \cdot \Bx} }.
=
\lambda \int \frac{ d^3 p d^3 k d^3 q d^3 s}{4 (2 \pi)^{9} \sqrt{
\omega_\Bp  \omega_\Bk  \omega_\Bq  \omega_\Bs
} }
\lr{
   a_\Bp a_\Bk a_\Bq a_\Bs e^{-i (\omega_\Bp + \omega_\Bk + \omega_\Bq + \omega_\Bs)t} \delta^3( \Bp + \Bk + \Bq + \Bs )
   +
   a_\Bp a_\Bk a_\Bq a_\Bs^\dagger e^{-i (\omega_\Bp + \omega_\Bk + \omega_\Bq - \omega_\Bs)t} \delta^3( \Bp + \Bk + \Bq - \Bs )
   +
   \cdots
   +
   a_\Bp^\dagger a_\Bk^\dagger a_\Bq^\dagger a_\Bs^\dagger e^{-i (-\omega_\Bp - \omega_\Bk - \omega_\Bq - \omega_\Bs)t} \delta^3( -\Bp - \Bk - \Bq - \Bs )
}
=
\lambda \int \frac{ d^3 p d^3 k d^3 q }{4 (2 \pi)^{9}
}
\lr{
   \inv{\sqrt{
      \omega_\Bp  \omega_\Bk  \omega_\Bq  \omega_{-\Bp - \Bk - \Bq}
   }}
   a_\Bp a_\Bk a_\Bq a_{-\Bp -\Bk - \Bq} e^{-i (\omega_\Bp + \omega_\Bk + \omega_\Bq + \omega_{-\Bp -\Bk -\Bq})t}
+
   \inv{\sqrt{
      \omega_\Bp  \omega_\Bk  \omega_\Bq  \omega_{\Bp + \Bk + \Bq}
   }}
   a_\Bp a_\Bk a_\Bq a_{\Bp + \Bk + \Bq}^\dagger e^{-i (\omega_\Bp + \omega_\Bk + \omega_\Bq - \omega_{\Bp + \Bk + \Bq})t}
+
   \cdots
+
   \inv{\sqrt{
      \omega_\Bp  \omega_\Bk  \omega_\Bq  \omega_{-\Bp - \Bk - \Bq}
   }}
   a_\Bp^\dagger a_\Bk^\dagger a_\Bq^\dagger a_{-\Bp -\Bk -\Bq}^\dagger e^{-i (-\omega_\Bp - \omega_\Bk - \omega_\Bq - \omega_{-\Bp -\Bk -\Bq})t}
}.
\end{dmath}
Assuming we can normal order these terms as in \( H_0 \), we can rewrite the interaction as
\begin{dmath}\label{eqn:ProblemSet2Problem4:100}
H_{\text{int}}
=
\lambda \int \frac{ d^3 p d^3 k d^3 q }{4 (2 \pi)^{9} }
\lr{
   \binom{4}{0}
      \inv{\sqrt{
         \omega_\Bp  \omega_\Bk  \omega_\Bq  \omega_{-\Bp - \Bk - \Bq}
      }}
      a_\Bp a_\Bk a_\Bq a_{-\Bp -\Bk - \Bq} e^{-i (\omega_\Bp + \omega_\Bk + \omega_\Bq + \omega_{-\Bp -\Bk -\Bq})t}
   +
   \binom{4}{1}
      \inv{\sqrt{
         \omega_\Bp  \omega_\Bk  \omega_\Bq  \omega_{\Bp - \Bk - \Bq}
      }}
      a_\Bp^\dagger a_\Bk a_\Bq a_{\Bp - \Bk - \Bq} e^{-i (-\omega_\Bp + \omega_\Bk + \omega_\Bq + \omega_{\Bp - \Bk - \Bq})t}
   +
   \binom{4}{2}
      \inv{\sqrt{
         \omega_\Bp  \omega_\Bk  \omega_\Bq  \omega_{\Bp + \Bk - \Bq}
      }}
      a_\Bp^\dagger a_\Bk^\dagger a_\Bq a_{\Bp + \Bk - \Bq} e^{-i (-\omega_\Bp - \omega_\Bk + \omega_\Bq + \omega_{\Bp + \Bk - \Bq})t}
   +
   \binom{4}{3}
      \inv{\sqrt{
         \omega_\Bp  \omega_\Bk  \omega_\Bq  \omega_{\Bp + \Bk _ \Bq}
      }}
      a_\Bp^\dagger a_\Bk^\dagger a_\Bq^\dagger a_{\Bp + \Bk + \Bq} e^{-i (-\omega_\Bp - \omega_\Bk - \omega_\Bq + \omega_{\Bp + \Bk + \Bq})t}
   +
   \binom{4}{4}
      \inv{\sqrt{
         \omega_\Bp  \omega_\Bk  \omega_\Bq  \omega_{-\Bp - \Bk - \Bq}
      }}
      a_\Bp^\dagger a_\Bk^\dagger a_\Bq^\dagger a_{-\Bp -\Bk -\Bq}^\dagger e^{-i (-\omega_\Bp - \omega_\Bk - \omega_\Bq - \omega_{-\Bp -\Bk -\Bq})t}
}
\end{dmath}
If we restrict the allowed momenta to the discrete set \( \Bp \in \setlr{ \Bp_1, \Bp_2, \cdots \Bp_N} \), the total Hamiltonian including the interaction term
takes the form
\begin{dmath}\label{eqn:ProblemSet2Problem4:120}
\text{\(:H:\)} =
\sum_{i = 1}^N \omega_{\Bp_i} a_{\Bp_i}^\dagger a_{\Bp_i}
+
\frac{
\lambda
}{4 }
\sum_{j,m,n = 1}^N
\lr{
   \binom{4}{0}
      \inv{\sqrt{
         \omega_{\Bp_j}  \omega_{\Bp_m}  \omega_{\Bp_n}  \omega_{-{\Bp_j} - {\Bp_m} - {\Bp_n}}
      }}
      a_{\Bp_j} a_{\Bp_m} a_{\Bp_n} a_{-\Bp -\Bk - \Bq} e^{-i (\omega_{\Bp_j} + \omega_{\Bp_m} + \omega_{\Bp_n} + \omega_{-{\Bp_j} -{\Bp_m} -{\Bp_n}})t}
   +
   \binom{4}{1}
      \inv{\sqrt{
         \omega_{\Bp_j}  \omega_{\Bp_m}  \omega_{\Bp_n}  \omega_{{\Bp_j} - {\Bp_m} - {\Bp_n}}
      }}
      a_{\Bp_j}^\dagger a_{\Bp_m} a_{\Bp_n} a_{{\Bp_j} - {\Bp_m} - {\Bp_n}} e^{-i (-\omega_{\Bp_j} + \omega_{\Bp_m} + \omega_{\Bp_n} + \omega_{{\Bp_j} - {\Bp_m} - {\Bp_n}})t}
   +
   \binom{4}{2}
      \inv{\sqrt{
         \omega_{\Bp_j}  \omega_{\Bp_m}  \omega_{\Bp_n}  \omega_{{\Bp_j} + {\Bp_m} - {\Bp_n}}
      }}
      a_{\Bp_j}^\dagger a_{\Bp_m}^\dagger a_{\Bp_n} a_{{\Bp_j} + {\Bp_m} - {\Bp_n}} e^{-i (-\omega_{\Bp_j} - \omega_{\Bp_m} + \omega_{\Bp_n} + \omega_{{\Bp_j} + {\Bp_m} - {\Bp_n}})t}
   +
   \binom{4}{3}
      \inv{\sqrt{
         \omega_{\Bp_j}  \omega_{\Bp_m}  \omega_{\Bp_n}  \omega_{{\Bp_j} + {\Bp_m} - {\Bp_n}}
      }}
      a_{\Bp_j}^\dagger a_{\Bp_m}^\dagger a_{\Bp_n}^\dagger a_{{\Bp_j} + {\Bp_m} + {\Bp_n}} e^{-i (-\omega_{\Bp_j} - \omega_{\Bp_m} - \omega_{\Bp_n} + \omega_{{\Bp_j} + {\Bp_m} + {\Bp_n}})t}
   +
   \binom{4}{4}
      \inv{\sqrt{
         \omega_{\Bp_j}  \omega_{\Bp_m}  \omega_{\Bp_n}  \omega_{-{\Bp_j} - {\Bp_m} - {\Bp_n}}
      }}
      a_{\Bp_j}^\dagger a_{\Bp_m}^\dagger a_{\Bp_n}^\dagger a_{-{\Bp_j} -{\Bp_m} -{\Bp_n}}^\dagger e^{-i (-\omega_{\Bp_j} - \omega_{\Bp_m} - \omega_{\Bp_n} - \omega_{-{\Bp_j} -{\Bp_m} -{\Bp_n}})t}
}.
\end{dmath}
When we did the same sort of calculation for \( (\spacegrad \phi)^2 + m^2 \phi^2 \) all the time dependent terms cancelled nicely, but that isn't obviously the case here.
However, we haven't used the non-relativistic (low energy) constraint.  That constraint can be expressed as \( \Bp^2 \ll m^2 \), in which case \( \omega_\Bp = \sqrt{ \Bp^2 + m^2 } \sim m \), the mass of each of the particles.  Incorporating that into our N-particle Hamiltonian, we have
\begin{dmath}\label{eqn:ProblemSet2Problem4:140}
\text{\(:H:\)} =
\sum_{i = 1}^N \omega_{\Bp_i} a_{\Bp_i}^\dagger a_{\Bp_i}
+
\frac{
\lambda
}{4 m^2 }
\sum_{j,m,n = 1}^N
\lr{
   \binom{4}{0}
      a_{\Bp_j} a_{\Bp_m} a_{\Bp_n} a_{-\Bp -\Bk - \Bq} e^{- 4 i m t}
   +
   \binom{4}{1}
      a_{\Bp_j}^\dagger a_{\Bp_m} a_{\Bp_n} a_{{\Bp_j} - {\Bp_m} - {\Bp_n}} e^{-3 i m t}
   +
   \binom{4}{2}
      a_{\Bp_j}^\dagger a_{\Bp_m}^\dagger a_{\Bp_n} a_{{\Bp_j} + {\Bp_m} - {\Bp_n}}
   +
   \binom{4}{3}
      a_{\Bp_j}^\dagger a_{\Bp_m}^\dagger a_{\Bp_n}^\dagger a_{{\Bp_j} + {\Bp_m} + {\Bp_n}} e^{ 3 i m t }
   +
   \binom{4}{4}
      a_{\Bp_j}^\dagger a_{\Bp_m}^\dagger a_{\Bp_n}^\dagger a_{-{\Bp_j} -{\Bp_m} -{\Bp_n}}^\dagger e^{4 i m t}
}.
\end{dmath}
Presuming there's a good argument to kill off the time dependent terms, the N-sector Hamiltonian is reduced to just
\begin{dmath}\label{eqn:ProblemSet2Problem4:160}
\text{\(:H:\)} =
\sum_{i = 1}^N \omega_{\Bp_i} a_{\Bp_i}^\dagger a_{\Bp_i}
+
\frac{
3 \lambda
}{2 m^2 }
\sum_{j,m,n = 1}^N
      a_{\Bp_j}^\dagger a_{\Bp_m}^\dagger a_{\Bp_n} a_{{\Bp_j} + {\Bp_m} - {\Bp_n}}.
\end{dmath}

The only annoying aspect to this Hamiltonian is the \( a_{{\Bp_j} + {\Bp_m} - {\Bp_n}} \) operator in the interaction term, which is not clear to me how to interpret.  That seems to imply that it is possible to create particles with linear combinations of momentum that may not be in the original set of \( N \) particle momenta.  I think that this can be further fudged by invoking the non-relativisitic constraint again, and decreeing that each of the uniquely indexed creation and anhillation operators are distinguishable only by index, so we can write the N-particle non-relativisitic sector Hamiltonian as
\begin{dmath}\label{eqn:ProblemSet2Problem4:170}
\text{\(:H:\)} =
\sum_{i = 1}^N \omega_{\Bp_i}
a_{i}^\dagger a_{i}
+
\frac{
3 \lambda
}{2 m^2 }
\sum_{r,s,t,u = 1}^N
      a_{r}^\dagger a_{s}^\dagger a_{t} a_{u}.
\end{dmath}

\makeSubAnswer{}{qft:problemSet2:4b}
Yes, with the number of creation and anhillation operators matched, this Hamiltonian preserves particle number (one particle is created for each particle destroyed).
The symmetry appears to be one associated with a permutation operation in the interaction.

%The hint seems to suggest that particle number is conserved, even though the interaction term does not have the structure of a number operator.  I have to conclude (too late for problem set submission) that I don't really understand what is meant by preservation of particle number in this case, and will need to see the problem set solution or discuss this in office hours to understand what is being asked for.
\makeSubAnswer{}{qft:problemSet2:4c}
Continued freehand, time allowing.
%If we designate an N-particle momentum state by
%\begin{dmath}\label{eqn:ProblemSet2Problem4:180}
%\ket{\Bp_1, \Bp_2, \cdots \Bp_N} =
%a_{\Bp_1}^\dagger
%a_{\Bp_2}^\dagger
%\cdots
%a_{\Bp_N}^\dagger \ket{0, 0, \cdots, 0},
%\end{dmath}
%then the interaction terms action on such a state is
%\begin{dmath}\label{eqn:ProblemSet2Problem4:200}
%      a_{\Bp_j}^\dagger a_{\Bp_m}^\dagger a_{\Bp_n} a_{{\Bp_j} + {\Bp_m} - {\Bp_n}}
%\ket{\Bp_1, \Bp_2, \cdots \Bp_N}.
%\end{dmath}
%I'm not sure if this is meaningful, or how to interpret it, and think that I'm going to have to get explanation about what this abstraction means.  I'm also not sure what is meant by the question ``What kinds of scattering processes does it describe.''
\makeSubAnswer{}{qft:problemSet2:4d}
Also continued freehand, time allowing.
%Not attempted.
}
