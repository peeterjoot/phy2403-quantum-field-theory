%
% Copyright © 2018 Peeter Joot.  All Rights Reserved.
% Licenced as described in the file LICENSE under the root directory of this GIT repository.
%
%{
\section{Additional resources.}

The video \citep{tobiasQFTL12Smatrix} does an excellent job explaining these concepts, covering the same material, but doing so in a very structured fashion.  He also nicely highlights which parts we are basically taking on faith in order to gain some calculation experience.

\section{Definitions and motivation.}
In QM we did lots of scattering problems as sketched in \cref{fig:qftLecture15b:qftLecture15bFig8},
and were able to compute the reflected and transmitted wave functions and quantities such as the reflection and transmission coefficients
\imageFigure{../figures/phy2403-quantum-field-theory/qftLecture15bFig8}{Reflection and transmission of wave packets.}{fig:qftLecture15b:qftLecture15bFig8}{0.3}
\begin{dmath}\label{eqn:qftLecture15b:440}
\begin{aligned}
R &= \frac
{\Abs{ \Psi_{\text{ref}}}^2}
{\Abs{ \Psi_{\text{in}}}^2} \\
T &= \frac
{\Abs{ \Psi_{\text{trans}}}^2}
{\Abs{ \Psi_{\text{in}}}^2}.
\end{aligned}
\end{dmath}
We'd like to consider scattering in some region of space with a non-zero potential, such as the scattering of a plane wave with known electron flux rate as sketched in
\cref{fig:qftLecture15b:qftLecture15bFig9}.
We can imagine that we have a detector capable of measuring the number of electrons with momentum \( \Bp_{\text{out}} \) per unit time.
\imageFigure{../figures/phy2403-quantum-field-theory/qftLecture15bFig9}{Plane wave scattering off a potential.}{fig:qftLecture15b:qftLecture15bFig9}{0.4}

\makedefinition{Total cross section (X-section).}{dfn:qftLecture15b:460}{
\begin{equation*}
\sigma_{\text{total}}
=
\frac{
\text{number of scattering events with \( \Bp_{\text{out}} \ne \Bk_{\text{in}} \) per unit time}
}
{
\text{Flux of incoming particles}
},
\end{equation*}
where the flux is the number of particles crossing a unit area in unit time.
} % definition

Units of the x-section are (with \( \hbar = c = 1 \))
\begin{equation}\label{eqn:qftLecture15b:480}
[\sigma] = \text{area} = \inv{M^2}.
\end{equation}

The concept of scattering cross section may not be new, as it can even be encountered in classical mechanics.  One such scenario is sketched in \cref{fig:qftLecture15b:qftLecture15bFig10} where the cross section is just the area
\begin{dmath}\label{eqn:qftLecture15b:500}
\sigma = \pi R^2.
\end{dmath}
\imageFigure{../figures/phy2403-quantum-field-theory/qftLecture15bFig10}{Classical scattering.}{fig:qftLecture15b:qftLecture15bFig10}{0.3}
Other classical fields where cross section is encountered includes antenna theory (radar scattering profiles, ...).

\makedefinition{Differential cross section.}{dfn:qftLecture15b:520}{
\begin{equation*}
\frac{d^3 \sigma}{dp_x dp_y dp_z} = \frac{
\text{number of scattering events with \( \Bp_{\text{out}} \) between \( (\Bp, \Bp + \Delta \Bp )\)}
}
{
\text{flux}
}.
\end{equation*}
} % definition

%}
