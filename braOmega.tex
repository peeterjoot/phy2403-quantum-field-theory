%
% Copyright � 2018 Peeter Joot.  All Rights Reserved.
% Licenced as described in the file LICENSE under the root directory of this GIT repository.
%
%{
%\input{../latex/blogpost.tex}
%\renewcommand{\basename}{braOmega}
%%\renewcommand{\dirname}{notes/phy1520/}
%\renewcommand{\dirname}{notes/ece1228-electromagnetic-theory/}
%%\newcommand{\dateintitle}{}
%%\newcommand{\keywords}{}
%
%\input{../latex/peeter_prologue_print2.tex}
%
%\usepackage{peeters_layout_exercise}
%\usepackage{peeters_braket}
%\usepackage{peeters_figures}
%\usepackage{siunitx}
%\usepackage{verbatim}
%%\usepackage{mhchem} % \ce{}
%%\usepackage{macros_bm} % \bcM
%%\usepackage{macros_qed} % \qedmarker
%%\usepackage{txfonts} % \ointclockwise
%
%% qftLecture14:
%% https://tex.stackexchange.com/a/68357/15
%\DeclareMathOperator*{\SumInt}{%
%\mathchoice%
%  {\ooalign{$\displaystyle\sum$\cr\hidewidth$\displaystyle\int$\hidewidth\cr}}
%  {\ooalign{\raisebox{.14\height}{\scalebox{.7}{$\textstyle\sum$}}\cr\hidewidth$\textstyle\int$\hidewidth\cr}}
%  {\ooalign{\raisebox{.2\height}{\scalebox{.6}{$\scriptstyle\sum$}}\cr$\scriptstyle\int$\cr}}
%  {\ooalign{\raisebox{.2\height}{\scalebox{.6}{$\scriptstyle\sum$}}\cr$\scriptstyle\int$\cr}}
%}
%
%\beginArtNoToc
%
%\generatetitle{PHY2403 (QFT I).  Pondering the ground state bra formula.}
%%\chapter{XXX}
%%\label{chap:braOmega}
%

\makeproblem{Pondering the ground state bra formula.}{problem:braOmega:1}{
Prove \cref{eqn:qftLecture15:80}.  What is wrong with conjugating 
\cref{eqn:qftLecture15:60} to find
\begin{equation*}
\bra{\Omega}
=
\evalbar{
   \frac{ \bra{0} U(-T, t_0) }
   {
   e^{+i E_0(T - t_0)} \braket{0}{\Omega}
   }
}
{
T \rightarrow \infty( 1 - i \epsilon )
}.
\end{equation*}
} % problem

\makeanswer{problem:braOmega:1}{
While there is nothing wrong with stating
\begin{dmath}\label{eqn:braOmega:100}
\lr{
   \frac{ U(t_0, -T) \ket{0} }
   {
   e^{-i E_0(T - t_0)} \braket{\Omega}{0}
   }
}^\dagger
=
   \frac{ \bra{0} U(-T, t_0) }
   {
   e^{+i E_0(T - t_0)} \braket{0}{\Omega}
   },
\end{dmath}
the limit point \( \infty(1 - i \epsilon) \) also needs to be changed with this conjugation.  So \cref{eqn:braOmega:100} is correct, but it is only part of the story, and should really be stated as
\begin{dmath}\label{eqn:braOmega:120}
\bra{\Omega}
=
\evalbar{
   \frac{ \bra{0} U(-T, t_0) }
   {
   e^{+i E_0(T - t_0)} \braket{0}{\Omega}
   }
}{T \rightarrow \infty(1 + i \epsilon)}.
\end{dmath}
This is awkward because now our expressions for \( \bra{\Omega} \) and \( \ket{\Omega} \) approach \( T \) from different directions, and we want to evaluate both with a single limiting argument.

To resolve this, we really have to start back with the identity expansion we used in lecture 14, and write
\begin{dmath}\label{eqn:braOmega:140}
\bra{0} e^{-i H T}
=
\lr{
\braket{0}{\Omega}\bra{\Omega}
 + \SumInt_n \braket{0}{n} \bra{n}
}
e^{-i H T}
=
\braket{0}{\Omega}\bra{\Omega}
e^{-i E_0 T}
 + \SumInt_n \braket{0}{n} \bra{n} e^{-i E_n T}.
\end{dmath}
We argued (as does the text) that approaching to as \( T( 1 - i \epsilon) \) kills off the energetic states since
\begin{dmath}\label{eqn:braOmega:160}
\bra{n} e^{-i E_n T}
\rightarrow
\bra{n} e^{-i E_n T} e^{-E_n T \epsilon}
\end{dmath}
and the exponential damping factor is smaller for each \( E_n > E_0 \), so it can be neglected in the large \( T \) limit, leaving
\begin{dmath}\label{eqn:braOmega:180}
\bra{0} e^{-i H T}
=
\lim_{T \rightarrow \infty(1 - i \epsilon)}
\braket{0}{\Omega}\bra{\Omega}.
\end{dmath}
As we did for \( \ket{\Omega} \) we can shift the large time \( T \) by a small constant (this time \( -t_0 \) instead of \( t_0 \)), to give
\begin{dmath}\label{eqn:braOmega:200}
\bra{\Omega}
=
\lim_{T \rightarrow \infty(1 - i \epsilon)}
\frac{ \bra{0} e^{-i H T} }
{
\braket{0}{\Omega} e^{-i E_0 T}
}
\approx
\lim_{T \rightarrow \infty(1 - i \epsilon)}
\frac{ \bra{0} e^{-i H (T - t_0)} }
{
\braket{0}{\Omega} e^{-i E_0 (T - t_0)}
}
=
\lim_{T \rightarrow \infty(1 - i \epsilon)}
\frac{ \bra{0} e^{i H_0( T - t_0)} e^{-i H (T - t_0)} }
{
\braket{0}{\Omega} e^{-i E_0 (T - t_0)}
}
=
\lim_{T \rightarrow \infty(1 - i \epsilon)}
\frac{ \bra{0} U(T, t_0) }
{
\braket{0}{\Omega} e^{-i E_0 (T - t_0)}
},
\end{dmath}
where the projective property \( \bra{0} e^{i H_0 \alpha} = \bra{0} \) has been used to insert a no-op (i.e. \( \bra{0} H_0 = 0 \)).  This recovers the result stated in class (also: 
\citep{peskin1995introduction}
eq. (4.29).)
} % answer

%}
%\EndArticle
%\EndNoBibArticle
