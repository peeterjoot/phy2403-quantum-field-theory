%
% Copyright © 2018 Peeter Joot.  All Rights Reserved.
% Licenced as described in the file LICENSE under the root directory of this GIT repository.
%
%\section{Problems.}

%\makeproblem{Matrix elements of Lorentz/metric product.}{problem:qftLecture3:520}{
%Justify \cref{eqn:qftLecture3:220} explicitly.
%} % problem
%
%\makeanswer{problem:qftLecture3:520}{
%Fixme.
%} % answer
%
\makeproblem{Four vector form of the Maxwell gauge transformation.}{problem:qftLecture3:540}{
Show that the transformation
\begin{equation}\label{eqn:qftLecture3:580}
A^\mu \rightarrow A^\mu + \partial^\mu \chi
\end{equation}
is the desired four-vector form of the gauge transformation \cref{eqn:qftLecture3:520}, that is
\begin{equation}\label{eqn:qftLecture3:540}
\begin{aligned}
j^\nu
&= \partial_\mu {F'}^{\mu\nu}
\\&= \partial_\mu F^{\mu\nu}.
\end{aligned}
\end{equation}
Also relate this four-vector gauge transformation to the spacetime split.
} % problem

\makeanswer{problem:qftLecture3:540}{
\begin{equation}\label{eqn:qftLecture3:560}
\begin{aligned}
\partial_\mu {F'}^{\mu\nu}
&=
\partial_\mu \lr{ \partial^\mu {A'}^\nu - \partial_\nu {A'}^\mu }
\\&=
\partial_\mu \lr{
  \partial^\mu \lr{ A^\nu + \partial^\nu \chi }
- \partial_\nu \lr{ A^\mu + \partial^\mu \chi }
}
\\&=
\partial_\mu {F}^{\mu\nu}
+
\partial_\mu \partial^\mu \partial^\nu \chi
-
\partial_\mu \partial^\nu \partial^\mu \chi
\\&=
\partial_\mu {F}^{\mu\nu},
\end{aligned}
\end{equation}
by equality of mixed partials.
Expanding \cref{eqn:qftLecture3:580} explicitly we find
\begin{equation}\label{eqn:qftLecture3:600}
{A'}^\mu = A^\mu + \partial^\mu \chi,
\end{equation}
which is
\begin{equation}\label{eqn:qftLecture3:620}
\begin{aligned}
\phi' = {A'}^0 &= A^0 + \partial^0 \chi = \phi + \dot{\chi} \\
\BA' \cdot \Be_k = {A'}^k &= A^k + \partial^k \chi = \lr{ \BA - \spacegrad \chi } \cdot \Be_k.
\end{aligned}
\end{equation}
The last of which can be written in vector notation as \( \BA' = \BA - \spacegrad \chi \).
} % answer

