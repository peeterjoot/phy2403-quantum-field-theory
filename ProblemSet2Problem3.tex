%
% Copyright � 2018 Peeter Joot.  All Rights Reserved.
% Licenced as described in the file LICENSE under the root directory of this GIT repository.
%
\makeoproblem{\( SU(2)_L \times SU(2)_R\), realized in the Wigner and Nambu-Goldstone modes.}{qft:problemSet2:3}{2018 HW2.III}{
Consider now our Lagrangian
\cref{eqn:ProblemSet2Problem2:50}
and imagine that \( m^2 < 0\), for whatever reason (nobody knows, really), while \( \lambda\) is still positive. This now becomes the Higgs Lagrangian of the Standard Model.

\makesubproblem{}{qft:problemSet2:3a}
Show that the classical potential in
\cref{eqn:ProblemSet2Problem2:50}
now becomes:
\begin{dmath}\label{eqn:ProblemSet2Problem3:20}
V = -\Abs{m^2}
\trace{H^\dagger H}
+ \lambda
\lr{ \trace{H^\dagger H} }^2
= \lambda \lr{
   \Abs{\phi_1}^2
   +
   \Abs{\phi_2}^2
   -
   \frac{\Abs{m^2}}{2 \lambda}
}^2
+ \text{const}.
\end{dmath}
\makesubproblem{}{qft:problemSet2:3b}
Clearly, there are extrema of the potential when
\(
   \Abs{\phi_1}^2
   +
   \Abs{\phi_2}^2
= 0 \)
and when
\(
   \Abs{\phi_1}^2
   +
   \Abs{\phi_2}^2
=
   \frac{\Abs{m^2}}{2 \lambda}
 \)
The second one has, clearly, smaller energy density. To quantize the theory, we now have to choose which classical minimum to expand around. Show that, if we expand around
\(
   \Abs{\phi_1}^2
   +
   \Abs{\phi_2}^2
= 0 \)
, we will find that the \( \phi _{1,2} \) excitations are tachyons, even classically. This signals an instability, rather than a faster-than-light propagation and shows that we have chosen the wrong value of \( \Phi \) to build our quantum theory.
\makesubproblem{}{qft:problemSet2:3c}
Thus, consider the
\(
   \Abs{\phi_1}^2
   +
   \Abs{\phi_2}^2
=
   \frac{\Abs{m^2}}{2 \lambda}
 \)
minimum of \( V \). This is really a set of minima. In fact
the set parameterized by
\(
   \Abs{\phi_1}^2
   +
   \Abs{\phi_2}^2
=
   \text{const}
 \)
is also known as a three sphere (\(S^3\), embedded in a four-dimensional space parameterized by
\(\psi^{1\cdots4}\) - not the spacetime!). To build the quantum theory, we will choose a point on this three sphere (a.k.a. the ``vacuum manifold'' - the set of field values that minimize the potential). We will now study the small fluctuations around the chosen point and the spectrum of the theory in this vacuum. There is an infinite number of parameterizations that can be used to do this, but I will suggest one that makes the symmetries the clearest.
 Thus, use the \(H\)-representation and take
\begin{dmath}\label{eqn:ProblemSet2Problem3:40}
H(x) = \frac{\Abs{m}}{2\sqrt{\lambda}} ( 1 + h(x) ) e^{i \phi^a(x) \sigma^a }
\end{dmath}
The logic here is as follows. When \( h(x)\) and \( \phi^a(x) \) vanish (i.e. there are no excitations), the
parameterization
\cref{eqn:ProblemSet2Problem3:40}
is equivalent, by
\cref{eqn:ProblemSet2Problem2:40}
, to taking a specific point on the vacuum manifold,
i.e. the one where \( \phi_1 = 0 \) and \( \phi_2 = \Abs{m}/\sqrt{2\lambda} \).
The fields \( h(x) \) and \( \phi^a(x) \) parameterize the
fluctuations around this ground state (for sure, they can be mapped - the map is nonlinear - to the fluctuations of the fields
\( \phi_{1,2} \) around the chosen vacuum value for \( \phi_2\).\footnote{As in classical mechanics, which variables one uses to describe physics is a matter of choice and convenience. The Euler-Lagrange equations have the property that they are invariant under changes of variables, so long as no singularity occurs in the process. In fact, one of the main motivations of using Lagrangians in classical mechanics is that the change of variables is much easier to do. In other words, it is much easier to first transform the Lagrangian to spherical coordinates and then find the Euler-Lagrange equations then to transform the equations found in Cartesian coordinates to spherical coordinates (in the latter case you need to differentiate twice...). Invariance of physics under nonsingular changes of variables in the Lagrangian is, of course, inherited in field theory.}
What you will do now is take the form
\cref{eqn:ProblemSet2Problem3:40}
, plug it into the Lagrangian
\cref{eqn:ProblemSet2Problem2:50}
with
\( m^2 = -\Abs{m^2}\),
and expand what you find to second order in the fields \( h(x) \) and \( \phi^a(x)\).
Show that the field \( h(x) \) has a mass and find an expression for it.
Show that the fields \( \phi^a(x) \) remain massless and that their Lagrangian (not just to quadratic order) only contains derivatives.

The latter point can be seen pretty simply by noting that \( H(x) \) from
\cref{eqn:ProblemSet2Problem3:40}
can be written as
\begin{dmath}\label{eqn:ProblemSet2Problem3:60}
H(x) = \frac{\Abs{m}}{2 \sqrt{ \lambda } }\Omega(x) ( 1 + h(x) ),
\end{dmath}
with \( \Omega^\dagger \Omega = 1 \) and \( \det(\Omega(x)) = 1 \).
In this parameterization \( \Omega(x) \) fluctuations
  correspond to going around the vacuum manifold \( S^3 \), while the \( h(x) \) fluctuations are along the ``radial'' directions away from the minimum.
The latter cost energy, hence \( h \) is massive (the Higgs field!), while the \( \Omega(x) \) only cost energy if the x-dependence is nontrivial.
The \( \phi^a(x)\) (or \( \Omega(x) \)) are equivalent parameterizations of the Goldstone fields. What you found here is an example of a general story: if a theory has a continuous symmetry, which is not a symmetry of the ground state, there is a number of massless Goldstone (or Nambu-Goldstone) modes. For internal symmetries like the ones we are considering here, their number is equal to the number of broken generators.

In the Standard Model, \(h(x)\) is indeed the Higgs field. The fields \(\phi^a(x)\) actually become the longitudinal components of the W and Z-bosons (one usually says that they are ``eaten'', a manifestation of the Landau-Anderson-Higgs-Brout-Englert-Guralnik-Hagen-... mechanism).

\makesubproblem{}{qft:problemSet2:3d}
One question that was not discussed and remained a bit obscure is that of the unbroken part of the symmetry. The original Lagrangian has \( SU(2)_L \times SU(2)_R \) symmetry. The value of
\( H(x) \) in the vacuum, denoted by \( \expectation{H}\), is given by
\cref{eqn:ProblemSet2Problem3:40}
with \( h = \phi^a = 0 \) and is
 \( \expectation{H} \sim\)
unit matrix.
Show that, while  \( \expectation{H} \) is not invariant under \( SU(2)_L \times SU(2)_R \) for arbitrary \( SU(2)_L \) and \( SU(2)_R \) transformations, it is invariant under
\cref{eqn:ProblemSet2Problem2:60}
with
\( U_L = U_R\). Such \( SU(2)_L \times SU(2)_R \) transformations with \( U_L = U_R \) are called ``diagonal'' or ``vector'' \( SU(2)_V \) transformations.
These remain unbroken in the vacuum. In the electroweak theory, the third component of \( SU(2)_V \) is identified with electromagnetic \( U(1)\).
Show that the current associated with \( SU(2)_V \) transformations has the form:
\begin{dmath}\label{eqn:ProblemSet2Problem3:80}
j_\mu^{V,a} = \frac{i}{2} \trace{\lr{
\partial_\mu H^\dagger \antisymmetric{\sigma^a}{H}
+
\partial_\mu H \antisymmetric{\sigma^a}{H^\dagger}
}}
\end{dmath}
Show also that the other ``linear'' combination of \( SU(2)_L \) and \( SU(2)_R \),
\cref{eqn:ProblemSet2Problem2:60}
with \( U_R = U_L^\dagger \) corresponds to the current (not conserved!) usually called the ``axial current''
\begin{dmath}\label{eqn:ProblemSet2Problem3:120}
j_\mu^{A,a} = \frac{i}{2} \trace{
\partial_\mu H^\dagger \symmetric{\sigma^a}{H} - \partial_\mu H \symmetric{\sigma^a}{H^\dagger},
}
\end{dmath}
where \( \symmetric{A}{B} = AB + BA \) denotes the anticommutator.

\makesubproblem{}{qft:problemSet2:3e}
Show that to linear order in the fields \( h(x),\phi^a(x) \), the a-th axial current is simply
\begin{dmath}\label{eqn:ProblemSet2Problem3:100}
j^{A,a} \sim \expectation{H} \partial_\mu \phi^a,
\end{dmath}
and find the constant in front. Thus, when the quantum operator corresponding to
\cref{eqn:ProblemSet2Problem3:100} % (12)
acts on the vacuum, it creates a quantum of the Goldstone boson (times the momentum and the ``Goldstone boson decay constant'' which is really equal to  \( \expectation{H} \)).

Show also that, to leading nontrivial order in the fields, the conserved vector current \( j^{V,a} \) is
quadratic in the fields \( \phi^a\).

In QCD, the relation
\cref{eqn:ProblemSet2Problem3:100} % (12)
and the algebra of the currents \( j^{V,A} \) constitute the basis of an approach to
soft-pion physics (soft means low energy) known as ``current algebra''.

Here, we studied the Nambu-Goldstone mode. In the Wigner mode, when \( m^2 > 0\), there are no massless particles, as is easy to convince yourselves.
\index{Nambu-Goldstone}
\index{Wigner mode}
\index{Higgs}
\index{W-boson}
\index{Z-boson}
} % makeproblem

\makeanswer{qft:problemSet2:3}{
\withproblemsetsParagraph{
\makeSubAnswer{}{qft:problemSet2:3a}
To expand the potential note that
\begin{dmath}\label{eqn:ProblemSet2Problem3:140}
\trace{\lr{
H^\dagger H
}}
=
\inv{2}
\trace{\lr{
\begin{bmatrix}
-i \Phi^\T \sigma^2 \\
\Phi^\dagger
\end{bmatrix}
\begin{bmatrix}
i \sigma^2 \Phi^\conj & \Phi
\end{bmatrix}
}}
=
\inv{2}\lr{ \Phi^\T \Phi^\dagger + \Phi^\dagger \Phi }
=
\inv{2}\lr{
   \phi_1 \phi_1^\conj + \phi_2 \phi_2^\conj + \phi_1^\conj \phi_1 + \phi_2^\conj \phi^2
}
=
\Abs{\phi_1}^2 + \Abs{\phi_2}^2,
\end{dmath}
so we have
\begin{dmath}\label{eqn:ProblemSet2Problem3:160}
V = -\Abs{m}^2
\trace{\lr{
H^\dagger H
}}
+ \lambda\lr{
\trace{\lr{
H^\dagger H
}}}^2
=
-
\Abs{m}^2
\lr{
   \Abs{\phi_1}^2 + \Abs{\phi_2}^2
}
+ \lambda
\lr{
   \Abs{\phi_1}^2 + \Abs{\phi_2}^2
}^2
=
\lambda\lr{
   \lr{
      \Abs{\phi_1}^2 + \Abs{\phi_2}^2
   }^2
   -
   \frac{\Abs{m}^2 }{\lambda}
   \lr{
      \Abs{\phi_1}^2 + \Abs{\phi_2}^2
   }
}.
\end{dmath}
Completing the square gives
\begin{dmath}\label{eqn:ProblemSet2Problem3:180}
V
=
\lambda\lr{
      \Abs{\phi_1}^2 + \Abs{\phi_2}^2
   -
   \frac{\Abs{m}^2 }{2\lambda}
}^2
   -
\lambda
\lr{
   \frac{\Abs{m}^2 }{2\lambda}
}^2,
\end{dmath}
which proves the result and shows that the constant is \( - \frac{\Abs{m}^4 }{4\lambda} \).
\makeSubAnswer{}{qft:problemSet2:3b}
From \cref{eqn:ProblemSet2Problem3:160} the first order expansion, ignoring constant terms, around \( \Abs{\phi_1}^2 + \Abs{\phi_2}^2 = 0 \) is
\begin{equation}\label{eqn:ProblemSet2Problem3:360}
V = -\Abs{m^2} \lr{ \Abs{\phi_1}^2 + \Abs{\phi_2}^2 } = -\Abs{m^2} \Phi^\dagger \Phi.
\end{equation}
The Lagrangian density, to first order, may be written in the compact form
\begin{dmath}\label{eqn:ProblemSet2Problem3:200}
\LL = \partial_\mu \Phi^\dagger \partial^\mu \Phi + \Abs{m}^2 \Phi^\dagger \Phi.
\end{dmath}
The equations of motion are
\begin{dmath}\label{eqn:ProblemSet2Problem3:220}
\begin{aligned}
\partial_\mu \partial^\mu \Phi &= \Abs{m}^2 \Phi \\
\partial_\mu \partial^\mu \Phi^\dagger &= \Abs{m}^2 \Phi^\dagger
\end{aligned},
\end{dmath}
or, \( \partial_\mu \partial^\mu \psi = \Abs{m}^2 \psi \) for any \( \psi \in \phi_1, \phi_2, \phi_1^\conj, \phi_2^\conj \).

Suppose that one of these wave functions has a Fourier transform representation
\begin{dmath}\label{eqn:ProblemSet2Problem3:240}
\psi(x) = \int \frac{d^4 p}{(2\pi)^4} e^{i p \cdot x} \tilde{\psi}.
\end{dmath}
Such a solution must satisfy the equations of motion
\begin{dmath}\label{eqn:ProblemSet2Problem3:260}
0
=
\lr{
   \partial_{tt} - \spacegrad^2 - \Abs{m^2}
}
\psi
=
\lr{
   \partial_{tt} - \spacegrad^2 - \Abs{m^2}
}
\int \frac{d^4 p}{(2\pi)^4} e^{i \omega t - i \Bp \cdot \Bx} \tilde{\psi}.
=
\int \frac{d^4 p}{(2\pi)^4}
\lr{
   (i\omega)^2 - (-i \Bp)^2 -\Abs{m}^2
}
e^{i \omega t - i \Bp \cdot \Bx} \tilde{\psi},
\end{dmath}
so
\begin{dmath}\label{eqn:ProblemSet2Problem3:280}
0 = -\omega^2 + \Bp^2 - \Abs{m}^2,
\end{dmath}
or
\begin{dmath}\label{eqn:ProblemSet2Problem3:300}
\omega = \sqrt{\Bp^2 - \Abs{m}^2}.
\end{dmath}
Any \( \Norm{\Bp} < \Abs{m} \) results in an imaginary angular frequency.  For example, at \( \Bp = 0 \), we have
\begin{dmath}\label{eqn:ProblemSet2Problem3:320}
\omega = \pm i \Abs{m}.
\end{dmath}
In particular
\begin{dmath}\label{eqn:ProblemSet2Problem3:340}
p_0 x^0
=
\omega t
= \pm i \Abs{m} t
= \pm \Abs{m} ( i t ).
\end{dmath}
We see that the angular momentum constraint on the system \cref{eqn:ProblemSet2Problem3:280} results in the imaginary time that is characteristic of tachonic solutions.

\makeSubAnswer{}{qft:problemSet2:3c}
It seems reasonable that we can assume that \( h(x) \) and \( \phi^a(x) \) in
\cref{eqn:ProblemSet2Problem3:40} are all real valued scalar (non-matrix) functions.  That is \( h(x) \) has the role of radial extension or compression of the field magnitude, and the exponential is of the form \( e^{i \Bsigma \cdot \Bphi(x) } \), a matrix valued rotation operator, where \( \Bphi = (\phi^1, \phi^2, \phi^3)\).  Given that assumption, \( H^\dagger H \) can be computed with relative ease, and has only radial dependence
\begin{dmath}\label{eqn:ProblemSet2Problem3:380}
\trace{\lr{H^\dagger H}}
=
\frac{\Abs{m}^2}{4 \lambda} (1 + h(x))^2 \trace{\lr{ e^{-i \Bsigma \cdot \Bphi} e^{i \Bsigma \cdot \Bphi} }}
=
\frac{\Abs{m}^2}{4 \lambda} (1 + h(x))^2 \trace{\BOne}
=
\frac{\Abs{m}^2}{2 \lambda} (1 + h)^2.
\end{dmath}
For the derivative quadratic form, it is expedient to use the form \cref{eqn:ProblemSet2Problem3:60}, which gives
\begin{dmath}\label{eqn:ProblemSet2Problem3:400}
\partial_\mu H^\dagger \partial^\mu H
=
\frac{\Abs{m}^2}{4 \lambda}
\lr{
   \partial_\mu h \Omega^\dagger
   + (1 + h) \partial_\mu \Omega^\dagger
}
\lr{
   \partial^\mu h \Omega
   + (1 + h)
\partial^\mu \Omega
}
=
\frac{\Abs{m}^2}{4 \lambda}
\lr{
   \partial_\mu h \Omega^\dagger \partial^\mu h \Omega
   + (1 + h)
      \lr{
         \partial_\mu h
         \Omega^\dagger (\partial^\mu \Omega)
       +
         \partial^\mu h
         (\partial_\mu \Omega^\dagger) \Omega
      }
   + (1 + h)^2 \partial_\mu \Omega^\dagger \partial^\mu \Omega
}
\end{dmath}
where we have made the usual assumptions that the independent fields \((h, \Omega)\) commute.
Because \( \Omega^\dagger \Omega = 1 \), we have
\begin{dmath}\label{eqn:ProblemSet2Problem3:480}
\partial_\mu h
\Omega^\dagger (\partial^\mu \Omega)
 +
\partial^\mu h
(\partial_\mu \Omega^\dagger) \Omega
=
\partial_\mu h
\lr{
   \Omega^\dagger (\partial^\mu \Omega)
    +
   (\partial^\mu \Omega^\dagger) \Omega
}
=
\partial_\mu h
\lr{
   \partial^\mu (\Omega^\dagger \Omega) - (\partial^\mu \Omega^\dagger) \Omega
    +
   (\partial^\mu \Omega^\dagger) \Omega
}
=
   \partial^\mu (1)
= 0.
\end{dmath}
All the cross terms with both \( h \) and \( \Omega \) derivatives are zero (to all orders, not just quadratic).

Taking traces (and using cyclic permutation of the matrices in the trace operations),
the Lagrangian density is now determined to quadratic order
\begin{dmath}\label{eqn:ProblemSet2Problem3:500}
\LL =
\frac{\Abs{m}^2}{2 \lambda}
   \partial_\mu h \partial^\mu h
+
\frac{\Abs{m}^2}{4 \lambda}
   \trace{\lr{
      \partial_\mu \Omega^\dagger \partial^\mu \Omega
   }}
+ \Abs{m}^2
\frac{\Abs{m}^2}{2 \lambda} \lr{ 1 + h }^2
- \lambda
\lr{\frac{\Abs{m}^2}{2 \lambda}}^2
\lr{ 1 + h }^4.
\end{dmath}
Observe that the Lagrangian density can be split into two independent parts, one for the radial field \( h \), and another for the rotation field \( \Omega \).  Rescaling to drop the common constant factor \( \Abs{m}^2/2\lambda \), the radial Lagrangian is
\begin{dmath}\label{eqn:ProblemSet2Problem3:560}
\LL_h
=
\partial_\mu h \partial^\mu h
+ \Abs{m}^2
\lr{ 1 + h }^2
-
\frac{\Abs{m}^2}{2}
\lr{ 1 + h }^4
=
\partial_\mu h \partial^\mu h
-\frac{\Abs{m}^2}{2}
\lr{
   \lr{ 1 + h }^4
   -2 \lr{ 1 + h }^2
}
=
\partial_\mu h \partial^\mu h
-\frac{\Abs{m}^2}{2}
\lr{
   \lr{ 1 + h }^2 - 1
}^2
+ \cancel{\text{const.}}
=
\partial_\mu h \partial^\mu h
-\frac{\Abs{m}^2}{2}
\lr{ 2 h + h^2
}^2
=
\partial_\mu h \partial^\mu h
- \frac{\Abs{m}^2}{2} h^2
\lr{ 2 + h }^2
=
\partial_\mu h \partial^\mu h
- 2 \Abs{m}^2 h^2
+ O(h^3).
\end{dmath}
This shows that the mass of the \( h \) field is \( \sqrt{2} \Abs{m} \).

The only remaining task is to express the Lagrangian density for \( \phi^a \) in terms of those field instead of \( \Omega \).  To evaluate those derivatives, we can utilize a first order Taylor expansion

\begin{dmath}\label{eqn:ProblemSet2Problem3:580}
\partial_\mu \Omega
=
\partial_\mu \lr{ \BOne + i \Bsigma \cdot \Bphi }
=
i \Bsigma \cdot \partial_\mu \Bphi,
\end{dmath}
so the rotation Lagrangian density is
\begin{dmath}\label{eqn:ProblemSet2Problem3:600}
\LL_\Bphi =
\inv{2} \trace{\lr{
(-i \Bsigma \cdot \partial_\mu \Bphi)
(i \Bsigma \cdot \partial^\mu \Bphi)
}}
=
(\partial_\mu \Bphi) \cdot (\partial^\mu \Bphi)
=
(\partial_\mu \phi^a) (\partial^\mu \phi^a),
\end{dmath}
where we use the fact that \( \trace{\lr{(\Bsigma \cdot \Bx)(\Bsigma \cdot \By)}} = 2 \Bx \cdot \By \).

The full Lagrangian density, to quadratic order, is
\boxedEquation{eqn:ProblemSet2Problem3:620}{
\LL
= \LL_h + \LL_\Bphi
=
\partial_\mu h \partial^\mu h
- 2 \Abs{m}^2 h^2
+
\partial_\mu \phi^a \partial^\mu \phi^a.
}

\makeSubAnswer{}{qft:problemSet2:3d}
\paragraph{Problem statement inconsistency.}
In the problem statement \( \expectation{H} \) is defined as a \( 2 \times 2 \) unit matrix scaled by \( \Abs{m}/2 \sqrt{\lambda} \), but later when used in the statement of the axial current, it appears as a number (since the current is a number, and not a matrix).  In this solution I've used \( \expectation{H} \) as just the numeric factor, and dropped the identity matrix factor.

\paragraph{Setup.}
This problem is easiest if we can work directly with in matrix notation, but first need to know how to express the current.  Given matrix elements \( H_{ab}, H^\conj_{ab} \), that current is
\begin{dmath}\label{eqn:ProblemSet2Problem3:640}
j^\mu
=
\PD{(\partial_\mu H_{ij})}{\LL} \delta H_{ij}
+
\PD{(\partial_\mu H^\conj_{ij})}{\LL} \delta H^\conj_{ij}.
\end{dmath}
The trace of a matrix product in terms of the respective matrix elements is
\begin{equation}\label{eqn:ProblemSet2Problem3:660}
\trace{\lr{ A B }}
=
A_{ik} B_{kj} \delta_{ij}
=
A_{ij} B_{ji},
\end{equation}
so the Kinetic portion of the Lagrangian density expands as
\begin{equation}\label{eqn:ProblemSet2Problem3:680}
\trace{\lr{
\partial_\mu H^\dagger \partial^\mu H
}}
=
\partial_\mu (H^\dagger)_{ji} \partial^\mu H_{ij}
=
\partial_\mu H^\conj_{ij} \partial^\mu H_{ij}.
\end{equation}
We can now put the current \cref{eqn:ProblemSet2Problem3:640} into matrix form
\begin{dmath}\label{eqn:ProblemSet2Problem3:700}
j^\mu
=
\partial^\mu H^\conj_{ij} \delta H_{ij}
+
\delta H^\conj_{ij} \partial^\mu H_{ij}
=
\partial^\mu (H^\dagger)_{ji} \delta H_{ij}
+
\delta (H^\dagger)_{ji} \partial^\mu H_{ij}
=
\trace{\lr{
\partial^\mu H^\dagger \delta H
+
\delta H^\dagger \partial^\mu H
}}.
\end{dmath}

\paragraph{Vector current.}

With \( H \rightarrow U_L H U_L^\dagger \), the \( H \) variation is
\begin{dmath}\label{eqn:ProblemSet2Problem3:720}
\delta H
=
H' - H
\approx
\lr{ 1 + \frac{i}{2} \Bsigma \cdot \Bomega } H
\lr{ 1 - \frac{i}{2} \Bsigma \cdot \Bomega }
- H
=
\frac{i}{2} (\Bsigma \cdot \Bomega) H
 - \frac{i}{2} H (\Bsigma \cdot \Bomega)
+ O(\Bomega^2)
=
\frac{i}{2} \antisymmetric{\Bsigma \cdot \Bomega}{H},
\end{dmath}
and its conjugate is
\begin{equation}\label{eqn:ProblemSet2Problem3:740}
\delta H^\dagger
=
-\frac{i}{2} \antisymmetric{H^\dagger}{\Bsigma \cdot \Bomega}
=
\frac{i}{2} \antisymmetric{\Bsigma \cdot \Bomega}{H^\dagger}.
\end{equation}

Putting the pieces together gives
\begin{dmath}\label{eqn:ProblemSet2Problem3:860}
j_\mu^{V, \Bomega}
= \frac{i}{2}
\trace{\lr{
\partial_\mu H^\dagger
\antisymmetric{\Bsigma \cdot \Bomega}{H}
+
\antisymmetric{\Bsigma \cdot \Bomega}{H^\dagger}
\partial_\mu H
}}
= \frac{i \omega^a}{2}
\trace{\lr{
\partial_\mu H^\dagger
\antisymmetric{\sigma^a}{H}
+
\partial_\mu H
\antisymmetric{\sigma^a}{H^\dagger}
}},
\end{dmath}
so setting \( j_\mu^{V, \Bomega} = \omega^a j_\mu^{V,a} \) to factor out the \( \omega^a \)'s, provides the desired result.

\paragraph{Axial current.}
This is only cosmetically different from the Vector current.

With \( H \rightarrow U_L H U_L \), the \( H \) variation is
\begin{dmath}\label{eqn:ProblemSet2Problem3:780}
\delta H
=
H' - H
\approx
\lr{ 1 + \frac{i}{2} \Bsigma \cdot \Bomega } H
\lr{ 1 + \frac{i}{2} \Bsigma \cdot \Bomega }
- H
=
\frac{i}{2} (\Bsigma \cdot \Bomega) H
 + \frac{i}{2} H (\Bsigma \cdot \Bomega)
+ O(\Bomega^2)
=
\frac{i}{2} \symmetric{\Bsigma \cdot \Bomega}{H},
\end{dmath}
and its conjugate is
\begin{equation}\label{eqn:ProblemSet2Problem3:800}
\delta H^\dagger
=
-\frac{i}{2} \symmetric{\Bsigma \cdot \Bomega}{H^\dagger}.
\end{equation}

Putting the pieces together gives
\begin{dmath}\label{eqn:ProblemSet2Problem3:880}
j_\mu^{A, \Bomega}
= \frac{i}{2}
\trace{\lr{
\partial_\mu H^\dagger
\symmetric{\Bsigma \cdot \Bomega}{H}
-
\symmetric{\Bsigma \cdot \Bomega}{H^\dagger}
\partial_\mu H
}}
= \frac{i \omega^a}{2}
\trace{\lr{
\partial_\mu H^\dagger
\symmetric{\sigma^a}{H}
-
\partial_\mu H
\symmetric{\sigma^a}{H^\dagger}
}},
\end{dmath}
so setting \( j_\mu^{A, \Bomega} = \omega^a j_\mu^{A,a} \) to factor out the \( \omega^a \)'s, provides the desired result.

\makeSubAnswer{}{qft:problemSet2:3e}
\paragraph{Axial current to first order.}

To first order the \( H \) partial is
\begin{dmath}\label{eqn:ProblemSet2Problem3:820}
\partial_\mu H
=
\expectation{H}
\lr{
   \partial_\mu h \lr{ 1 + i \Bsigma \cdot \Bphi }
+
   (1 + h) i \Bsigma \cdot \partial_\mu \Bphi
}
=
\expectation{H}
\lr{
   \partial_\mu h
+
   i \Bsigma \cdot \partial_\mu \Bphi
}
+ O(2).
\end{dmath}
Because this has no zero order terms, we need only the zeroth order parts of the anticommutators
\begin{dmath}\label{eqn:ProblemSet2Problem3:840}
\symmetric{\sigma^a}{H}
=
\expectation{H}
(1 + h) \symmetric{\sigma^a}{1 + i \Bsigma \cdot \Bphi}
=
\expectation{H} \symmetric{\sigma^a}{1}
+ O(1)
=
2 \expectation{H} \sigma^a.
\end{dmath}
To first order
\begin{dmath}\label{eqn:ProblemSet2Problem3:900}
j^{A,a}_\mu
=
i \expectation{H}^2 \trace{\lr{
   \lr{
      \partial_\mu h
   -
      i \Bsigma \cdot \partial_\mu \Bphi
   }
   \sigma^a
-
   \lr{
      \partial_\mu h
   +
      i \Bsigma \cdot \partial_\mu \Bphi
   } \sigma^a
}}
=
2 \expectation{H}^2 \trace{\lr{ \sigma^b \partial_\mu \phi^b \sigma^a }}.
\end{dmath}
Since \( \trace{\lr{ \sigma^a \sigma^b }} = 2 \delta_{ab} \), this reduces to
\begin{dmath}\label{eqn:ProblemSet2Problem3:920}
j^{A,a}_\mu
=
\expectation{H} \lr{ 4 \expectation{H} } \partial_\mu \phi^a,
\end{dmath}
so the ``constant in front'' is \( 4 \expectation{H} = 2 \Abs{m}/\sqrt{\lambda} \).

\paragraph{Vector current to second order.}
%%For this part of the problem, we are to show that, to leading order in the fields, the conserved vector current is quadratic in \( \phi^a \).  Let's assume that we can drop any \( O(3) \) terms in the current, and look at what terms are left over of lesser order.
%%
To make life less messy, let's write
\begin{dmath}\label{eqn:ProblemSet2Problem3:940}
H = \expectation{H} (1 + h) \Omega,
\end{dmath}
so that
\begin{equation}\label{eqn:ProblemSet2Problem3:960}
\antisymmetric{\sigma^a}{H}
=
\expectation{H}
\antisymmetric{\sigma^a}{(1 + h)\Omega}
=
\expectation{H} h \antisymmetric{\sigma^a}{\Omega}.
\end{equation}
We also have, also to all orders,
\begin{dmath}\label{eqn:ProblemSet2Problem3:980}
\partial_\mu H
=
\expectation{H} \lr{ \partial_\mu h \Omega + (1 + h)\partial_\mu \Omega }
%=
%\expectation{H} \lr{
%   \partial_\mu h \Omega + \partial_\mu \Omega + h \partial_\mu \Omega
%}
\end{dmath}
The current is
\begin{dmath}\label{eqn:ProblemSet2Problem3:1000}
j^{V,a}_\mu
=
\frac{i}{2}
\trace{\lr{
   \partial_\mu H^\dagger \antisymmetric{\sigma^a}{H}
   +
   \partial_\mu H \antisymmetric{\sigma^a}{H^\dagger}
}}
=
\frac{i}{2}
\expectation{H}^2
\trace{\lr{
   \lr{
      \partial_\mu h \Omega^\dagger
         + (1 + h) \partial_\mu \Omega^\dagger
   }
   h \antisymmetric{\sigma^a}{\Omega}
+
   \lr{
      \partial_\mu h \Omega
      + (1 + h )\partial_\mu \Omega
   }
   h \antisymmetric{\sigma^a}{\Omega^\dagger}
}}
=
\frac{i}{2}
\expectation{H}^2
\lr{
   (\partial_\mu h) h
   \trace{\lr{
         \Omega^\dagger \antisymmetric{\sigma^a }{\Omega}
         +
         \Omega \antisymmetric{\sigma^a }{\Omega^\dagger}
   }}
   +
   h(1+h)
   \trace{\lr{
         \partial_\mu \Omega^\dagger \antisymmetric{\sigma^a}{\Omega}
      +
         \partial_\mu \Omega \antisymmetric{\sigma^a}{\Omega^\dagger}
   }}
}
=
\frac{i}{2}
\expectation{H}^2
\lr{
   (\partial_\mu h) h A
   +
   h(1+h) B
},
\end{dmath}
where
\begin{dmath}\label{eqn:ProblemSet2Problem3:1100}
\begin{aligned}
A &=
   \trace{\lr{
         \Omega^\dagger \antisymmetric{\sigma^a }{\Omega}
         +
         \Omega \antisymmetric{\sigma^a }{\Omega^\dagger}
   }} \\
B &=
   \trace{\lr{
         \partial_\mu \Omega^\dagger \antisymmetric{\sigma^a}{\Omega}
      +
         \partial_\mu \Omega \antisymmetric{\sigma^a}{\Omega^\dagger}
   }}.
\end{aligned}
\end{dmath}
The first trace \( A \) is easily shown to be zero
\begin{dmath}\label{eqn:ProblemSet2Problem3:1020}
A
=
\trace{\lr{
      \Omega^\dagger \sigma^a \Omega
      -\Omega^\dagger \Omega \sigma^a
      +
      \Omega \sigma^a \Omega^\dagger
      -
      \Omega \Omega^\dagger \sigma^a
}}
=
\trace{\lr{
   \lr{
        \Omega \Omega^\dagger
      - \Omega^\dagger \Omega
      + \Omega^\dagger \Omega
      - \Omega \Omega^\dagger
   }  \sigma^a
}}
= 0,
\end{dmath}
where cyclic permutation within the trace was used to arrange the terms for easy cancellation \(1 - 1 + 1 - 1 = 0\).

Expanding commutators, and using cyclic permutation in the trace, we have for \( B \)
\begin{dmath}\label{eqn:ProblemSet2Problem3:1040}
B
=
\trace{\lr{
      \partial_\mu \Omega^\dagger \antisymmetric{\sigma^a}{\Omega}
   +
      \partial_\mu \Omega \antisymmetric{\sigma^a}{\Omega^\dagger}
}}
=
\trace{\lr{
     (\partial_\mu \Omega^\dagger) \sigma^a \Omega
   -
     (\partial_\mu \Omega^\dagger) \Omega \sigma^a
   +
     (\partial_\mu \Omega) \sigma^a \Omega^\dagger
   -
     (\partial_\mu \Omega) \Omega^\dagger \sigma^a
}}
=
\trace{\lr{
   \lr{
        \Omega (\partial_\mu \Omega^\dagger )
      - (\partial_\mu \Omega^\dagger) \Omega
      + \Omega^\dagger (\partial_\mu \Omega )
      - (\partial_\mu \Omega) \Omega^\dagger
   } \sigma^a
}}
\end{dmath}
This can be simplified using
\begin{dmath}\label{eqn:ProblemSet2Problem3:1060}
\begin{aligned}
\Omega (\partial_\mu \Omega^\dagger) &= -(\partial_\mu \Omega) \Omega^\dagger \\
\Omega^\dagger (\partial_\mu \Omega) &= -(\partial_\mu \Omega^\dagger) \Omega
\end{aligned},
\end{dmath}
so
\begin{dmath}\label{eqn:ProblemSet2Problem3:1080}
B
=
2
\trace{\lr{
   \lr{
        \Omega^\dagger (\partial_\mu \Omega )
      - (\partial_\mu \Omega) \Omega^\dagger
   } \sigma^a
}}.
\end{dmath}
The derivative \( \partial_\mu \Omega \) has no \( O(0) \) terms, so let's expand the rotation matrix only to \( O(1) \), and then drop any \( O(2) \) terms from \( \partial_\mu \Omega \).  This gives
\begin{dmath}\label{eqn:ProblemSet2Problem3:1120}
B
=
2
\trace{\lr{
   \lr{
        (1 - i \Bsigma \cdot \Bphi) (\partial_\mu \Omega )
      - (\partial_\mu \Omega) (1 - i \Bsigma \cdot \Bphi)
   } \sigma^a
}}
+ O(3)
=
- 2 i
\trace{\lr{
   \lr{
        (\Bsigma \cdot \Bphi) (\partial_\mu \Omega )
      - (\partial_\mu \Omega) (\Bsigma \cdot \Bphi)
   } \sigma^a
}}
=
- 2 i^2
\trace{\lr{
   \lr{
      \sigma^c \phi^c \sigma^b \partial_\mu \phi^b
      -
      \sigma^b \partial_\mu \phi^b \sigma^c \phi^c
   } \sigma^a
}}
+ O(3)
=
2 \phi^c (\partial_\mu \phi^b) \trace{\lr{ \sigma^c \sigma^b \sigma^a }}
-2 (\partial_\mu \phi^b) \phi^c \trace{\lr{ \sigma^b \sigma^c \sigma^a }}
=
-2
\lr{ \phi^c (\partial_\mu \phi^b) + (\partial_\mu \phi^b) \phi^c }
\trace{\lr{ \sigma^1 \sigma^2 \sigma^3 }} \epsilon^{abc}
= - 4 i
\symmetric{ \phi^c}{\partial_\mu \phi^b} \epsilon^{abc},
\end{dmath}
to quadratic order in \( \phi^a \).  The final steps above used the fact that the trace of three Pauli matrices is zero unless they are all different, and \( \trace{\lr{ \sigma^1 \sigma^2 \sigma^3 }} = 2 i \).

The current, to lowest order in \( \phi^a \), and all orders in \( h \), is
\begin{dmath}\label{eqn:ProblemSet2Problem3:1140}
j^{V,a}_\mu
=
2
\expectation{H}^2 h(1 + h) \symmetric{ \phi^c}{\partial_\mu \phi^b} \epsilon^{abc},
\end{dmath}
which is quadratic in \( \phi^a \) as claimed.
}}
