%
% Copyright � 2018 Peeter Joot.  All Rights Reserved.
% Licenced as described in the file LICENSE under the root directory of this GIT repository.
%
%{
\input{../latex/blogpost.tex}
\renewcommand{\basename}{qftLecture21b}
\renewcommand{\dirname}{notes/phy2403/}
\newcommand{\keywords}{PHY2403H}
\input{../latex/peeter_prologue_print2.tex}

%\usepackage{phy2403}
\usepackage{peeters_braket}
\usepackage{peeters_layout_exercise}
\usepackage{peeters_figures}
\usepackage{mathtools}
\usepackage{siunitx}
\usepackage{macros_cal} % LL

\newcommand{\ultensor}[3]{{{#1}^{#2}}_{#3}}
\newcommand{\oPsi}[0]{\overbar{\Psi}}
\newcommand{\osigma}[0]{\overbar{\sigma}}
\newcommand{\ubar}[0]{\overbar{u}}
\newcommand{\vbar}[0]{\overbar{v}}
\newcommand{\deltathree}[0]{\delta^{(3)}}
\newcommand{\deltafour}[0]{\delta^{(4)}}
\newcommand{\ITwo}[0]{{\begin{bmatrix} 1 & 0 \\ 0 & 1 \end{bmatrix}}}
\newcommand{\DiracGammaZero}[0]{{\begin{bmatrix} 0 & 1 \\ 1 & 0 \end{bmatrix}}}
\newcommand{\DiracGammaK}[1]{{\begin{bmatrix} 0 & \sigma^{#1} \\ -\sigma^{#1} & 0 \end{bmatrix}}}

\beginArtNoToc
\generatetitle{PHY2403H Quantum Field Theory.  Lecture 21, Part II: XXX.  Taught by Prof.\ Erich Poppitz}
%\chapter{XXX}
\label{chap:qftLecture21b}

%%Peeter's lecture notes from class.  These may be incoherent and rough.
%%
%%These are notes for the UofT course PHY2403H, Quantum Field Theory, taught by Prof. Erich Poppitz, covering \textchapref{{1}} \citep{peskin1995introduction} content.

\paragraph{DISCLAIMER: Very rough notes from class, with some additional side notes.}

These are notes for the UofT course PHY2403H, Quantum Field Theory, taught by Prof. Erich Poppitz, fall 2018.
%, covering \textchapref{{1}} \citep{peskin1995introduction} content.

\section{Lagrangian.}

\begin{dmath}\label{eqn:qftLecture21b:500}
\LL_{\text{Dirac}}
= \oPsi i \gamma^0 \partial_0 \Psi + i \oPsi \gamma^j \partial_j \Psi - m \oPsi \Psi
= \Psi^\dagger \gamma^0 i \gamma^0 \partial_0 \Psi + i \Psi \gamma^0 \gamma^j \partial_j \Psi - m \Psi^\dagger \gamma^0 \Psi
= \Psi^\dagger i \dot{\Psi} + i \Psi^\dagger \gamma^0 \gamma^j \partial_j \Psi - m \Psi^\dagger \gamma^0 \Psi
\end{dmath}

\begin{dmath}\label{eqn:qftLecture21b:520}
\pi_\psi = \PD{\dot{\Psi}}{\LL} = i \Psi^\dagger
\end{dmath}

Our Hamiltonian density is
\begin{dmath}\label{eqn:qftLecture21b:540}
H_{\text{Dirac}}
= \pi_\Psi \dot{\Psi} - \LL
= i \Psi^\dagger \dot{\Psi} - \lr{
\Psi^\dagger i \dot{\Psi} + i \Psi^\dagger \gamma^0 \gamma^j \partial_j \Psi - m \Psi^\dagger \gamma^0 \Psi
}
\end{dmath}

so the Dirac Hamiltonian is
\begin{dmath}\label{eqn:qftLecture21b:560}
H
=
\int d^3 x
\lr{
- i \Psi^\dagger \gamma^0 \gamma^j \partial_j \Psi + m \Psi^\dagger \gamma^0 \Psi
}
=
\int d^3 x
\Psi^\dagger
\lr{
- i \gamma^0 \gamma^j \partial_j \Psi + m \gamma^0
}
\Psi.
\end{dmath}

Now we use linear combinations of the solutions we have found
\begin{dmath}\label{eqn:qftLecture21b:580}
\Psi(\Bx,0) = \int \frac{d^3 p}{(2 \pi)^3 \sqrt{2 \omega_\Bp} } e^{i \Bp \cdot \Bx} \sum_{s = 1,2} \lr{
   u^s(\Bp) a^s_\Bp
   +
   v^s(-\Bp) b^s_{-\Bp}
},
\end{dmath}
where \( t = 0 \) for convience.

Given the Dirac equation
\begin{dmath}\label{eqn:qftLecture21b:600}
\lr{ i \gamma^\mu \partial_\mu - m } \Psi = 0.
\end{dmath}
When we plug in \( \Psi_u = u(p) e^{-i p \cdot x} \) we get \( (\gamma^\mu p_\mu - m) u = 0 \), so
\begin{dmath}\label{eqn:qftLecture21b:640}
\gamma^i p_i u = (-\gamma^0 p^0 + m)u
\end{dmath}
whereas when we plug in
\( \Psi_v = v(p) e^{i p \cdot x} \) we get \( (-\gamma^\mu p_\mu - m) u = 0 \), so
\begin{dmath}\label{eqn:qftLecture21b:660}
\gamma^j p_j v = -(m + \gamma^0 p_0) v
\end{dmath}

What is the action of \( -i \gamma^0 \gamma^j \partial_j + m \gamma^0 \) on \(\Psi_u \)?  We get

\begin{dmath}\label{eqn:qftLecture21b:620}
(-i \gamma^0 \gamma^j \partial_j + m \gamma^0 ) \Psi_u
=
- \gamma^0 \lr{ - \gamma^0 p^0 + m }
u e^{-i p \cdot x}
+ m \gamma^0
u e^{-i p \cdot x}
=
(\gamma^0)^2 p_0 u^{-i p \cdot x}
\end{dmath}

\begin{dmath}\label{eqn:qftLecture21b:630}
(-i \gamma^0 \gamma^j \partial_j + m \gamma^0 ) \Psi_u
=
 \gamma^0 \lr{ m + \gamma^0 p_0 }
v e^{i p \cdot x}
+ m \gamma^0
v e^{i p \cdot x}
=
(\gamma^0)^2 p_0 u^{-i p \cdot x}
\end{dmath}

restart
\begin{dmath}\label{eqn:qftLecture21b:680}
\lr{ -i \gamma^0 \gamma^j \partial_j + m \gamma^0 } \Psi = - \gamma^0
\lr{ i \gamma^j \partial_j - m } \Psi
\end{dmath}

acting on
\begin{dmath}\label{eqn:qftLecture21b:700}
\Psi_u = u(p)
e^{-i p_\mu x^\mu }
\end{dmath}

we get

\begin{dmath}\label{eqn:qftLecture21b:720}
\lr{ i \gamma^j \partial_j - m }
e^{-i p_\mu x^\mu }
=
\lr{ i \gamma^j \partial_j - m }
e^{-i p_\mu x^\mu }
=
\lr{ \gamma^j p_j - m } e^{-i p\cdot x}
\end{dmath}

used EOM for u

\begin{dmath}\label{eqn:qftLecture21b:740}
\Psi_v = v(p)
e^{i p_\mu x^\mu }
\end{dmath}

\begin{dmath}\label{eqn:qftLecture21b:880}
\lr{ i \gamma^j \partial_j - m }
e^{-i p_k x^k }
=
\lr{ -i \gamma^j p_j - m } e^{i p \cdot x}
\end{dmath}

We find

\begin{dmath}\label{eqn:qftLecture21b:760}
-\gamma^0 \lr{ i \gamma^j \partial_j - m } \Psi_v = -\gamma^0 \gamma^0 p_0 \Psi_v = -p_0 \Psi_v
\end{dmath}


\boxedEquation{eqn:qftLecture21b:780}{
\begin{aligned}
-\gamma^0 \lr{ i \gamma^j \partial_j - m } \Psi_u &= p_0 \Psi_u \\
-\gamma^0 \lr{ i \gamma^j \partial_j - m } \Psi_v &= -p_0 \Psi_v
\end{aligned}
}

Adding time back in the mix the most general wave function is
\begin{dmath}\label{eqn:qftLecture21b:800}
\Psi(\Bx, t)
=
\int \frac{d^3 p}{(2 \pi)^3 \sqrt{ 2 \omega_\Bp } }
\lr{
   \sum_s e^{-i p \cdot x} u^s_\Bp a_\Bp^s
+
   \sum_s e^{i p \cdot x} v^s_\Bp b_\Bp^s
},
\end{dmath}

\begin{dmath}\label{eqn:qftLecture21b:820}
\Psi(\Bx,0)
=
\int \frac{d^3 p}{(2 \pi)^3 \sqrt{2 \omega_\Bp} } e^{i \Bp \cdot x} \sum_{s = 1,2} \lr{
   u^s(\Bp) a^s_\Bp
   +
   v^s(-\Bp) b^s_{-\Bp}
}
\end{dmath}

\begin{dmath}\label{eqn:qftLecture21b:840}
\Psi^\dagger(\Bx,0)
= \int \frac{d^3 q}{(2 \pi)^3 \sqrt{2 \omega_\Bq} } e^{-i \Bq \cdot x} \sum_{r = 1,2} \lr{
   u^{r \dagger}(\Bq) a^{r \dagger}_\Bq
   +
   v^{r \dagger}(-\Bq) b^{r \dagger}_{-\Bq}
}
\end{dmath}

We'll get the following

\begin{dmath}\label{eqn:qftLecture21b:860}
H_{\text{Dirac}} =
\int \frac{d^3 p}{(2 \pi)^3 2 \omega_\Bp }
\sum_{r = 1,2} \lr{
   u^r(\Bp) a^r_\Bp
   +
   v^r(-\Bp) b^r_{-\Bp}
}
\times
\sum_{s = 1,2} \lr{
   \omega_\Bp
   u^s(\Bp) a^s_\Bp
   +
   \omega_\Bp
   v^s(-\Bp) b^s_{-\Bp}
}
=
\int \frac{d^3 p}{(2 \pi)^3 2 \omega_\Bp }
\omega_\Bp
\sum_{r,s = 1,2} \lr{
a^{r \dagger}_\Bp a^s_\Bp \underbrace{u^{r \dagger}(\Bp) u^s(\Bp) }_{\delta^{rs} 2 \omega_\Bp}
-
\cancel{a^{r \dagger}_\Bp b^s_{-\Bp} u^{r \dagger}(\Bp) v^s(\Bp) }
+
\cancel{b^{r \dagger}_{-\Bp} a^s_\Bp v^{r \dagger}(-\Bp) u^s(\Bp) }
-
b^{r \dagger}_{-\Bp} b^s_{-\Bp} \underbrace{v^{r \dagger}(\Bp) v^s(\Bp) }_{\delta^{rs} 2 \omega_\Bp}
}
=
\int \frac{d^3 p}{(2 \pi)^3 }
\omega_\Bp
\sum_{r = 1}
\lr{
(a_\Bp^{r \dagger} a_\Bp^s - b_\Bp^{r \dagger} b_\Bp^s
}.
\end{dmath}
Note that we have a minus sign in the Hamiltonian, so there is no bound to the energy from below!  This makes interpretation of the \( a_p \)'s and \( b_p \)'s as the familiar raising and lowering operators that we know, we end up in trouble.

We can save the day, making the ``Dirac sea'' argument\footnote{There was a long discussion of this topic that I was not able to capture in my notes.}.

It will turn out that our operators are Fermions.
Let
\begin{dmath}\label{eqn:qftLecture21b:900}
\begin{aligned}
b^{s \dagger}_p &= tb_p \\
b^s &= \tilde{b^{s \dagger}}_p
\end{aligned}
\end{dmath}

some properties are
\begin{dmath}\label{eqn:qftLecture21b:920}
\begin{aligned}
(a^s_p)^2 &= 0 \\
(a^{s \dagger}_p)^2 &= 0 \\
(b^s_p)^2 &= 0 \\
(b^{s \dagger}_p)^2 &= 0
\end{aligned}
\end{dmath}

\begin{dmath}\label{eqn:qftLecture21b:940}
\begin{aligned}
\symmetric{a^s_\Bp}{a^{r \dagger}_\Bq} &= \delta^{sr} \deltathree(\Bp - \Bq) \\
\symmetric{b^s_\Bp}{b^{r \dagger}_\Bq} &= \delta^{sr} \deltathree(\Bp - \Bq)
\end{aligned}
\end{dmath}

and all other anticommutators are zero
\begin{dmath}\label{eqn:qftLecture21b:960}
\symmetric{a^r}{b^s} =
\symmetric{a^r}{b^{s\dagger}} =
\symmetric{a^{r\dagger}}{b^s} =
\symmetric{a^{r\dagger}}{b^{s\dagger}} = 0.
\end{dmath}

\begin{dmath}\label{eqn:qftLecture21b:980}
H_{\text{Dirac}}
=
\int \frac{d^3 p}{(2 \pi)^3 }
\omega_\Bp
\sum_{r = 1}
\lr{
\tilde{a}_\Bp^r a_\Bp^s - \tilde{b}_\Bp^r b_\Bp^s
}
=
\int \frac{d^3 p}{(2 \pi)^3 }
\omega_\Bp
\sum_{s = 1}
\lr{
a_\Bp^{r \dagger} a_\Bp^s
+ \tilde{b}_\Bp^s \tilde{b}^{s\dagger}_\Bp + \delta^{ss} \deltathree( \Bp - \Bp )
}
=
\int \frac{d^3 p}{(2 \pi)^3 }
\lr{
\omega_\Bp
\sum_{s = 1}
\lr{
a_\Bp^{ r \dagger} a_\Bp^s
+ \tilde{b}_\Bp^s \tilde{b}^{s\dagger}_\Bp
}
- 4 V_3 \frac{\omega_\Bp}{2}
}
\end{dmath}

We'll end up dropping the vacuum energy term.  We'll end up labelling the \( a \)'s as the operators associated with electrons, and the \( b \)'s with antielectrons.

%}
%\EndArticle
\EndNoBibArticle
