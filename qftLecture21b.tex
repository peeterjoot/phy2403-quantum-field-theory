%
% Copyright � 2018 Peeter Joot.  All Rights Reserved.
% Licenced as described in the file LICENSE under the root directory of this GIT repository.
%
%{
%\input{../latex/blogpost.tex}
%\renewcommand{\basename}{qftLecture21b}
%\renewcommand{\dirname}{notes/phy2403/}
%\newcommand{\keywords}{PHY2403H}
%\input{../latex/peeter_prologue_print2.tex}
%
%%\usepackage{phy2403}
%\usepackage{peeters_braket}
%\usepackage{peeters_layout_exercise}
%\usepackage{peeters_figures}
%\usepackage{mathtools}
%\usepackage{siunitx}
%\usepackage{macros_cal} % LL
%
%\newcommand{\ultensor}[3]{{{#1}^{#2}}_{#3}}
%\newcommand{\oPsi}[0]{\overbar{\Psi}}
%\newcommand{\osigma}[0]{\overbar{\sigma}}
%\newcommand{\ubar}[0]{\overbar{u}}
%\newcommand{\vbar}[0]{\overbar{v}}
%\newcommand{\deltathree}[0]{\delta^{(3)}}
%\newcommand{\deltafour}[0]{\delta^{(4)}}
%\newcommand{\ITwo}[0]{{\begin{bmatrix} 1 & 0 \\ 0 & 1 \end{bmatrix}}}
%\newcommand{\DiracGammaZero}[0]{{\begin{bmatrix} 0 & 1 \\ 1 & 0 \end{bmatrix}}}
%\newcommand{\DiracGammaK}[1]{{\begin{bmatrix} 0 & \sigma^{#1} \\ -\sigma^{#1} & 0 \end{bmatrix}}}
%
%\beginArtNoToc
%\generatetitle{PHY2403H Quantum Field Theory.  Lecture 21, Part II: Dirac Hamiltonian, Hamiltonian eigenvalues, general solution, creation and annihilation operators, Dirac Sea, anti-electrons.  Taught by Prof.\ Erich Poppitz}
%\chapter{Dirac Hamiltonian, Hamiltonian eigenvalues, general solution, creation and anhillation operators, Dirac Sea, antielectrons}
\index{Dirac Hamiltonian!eigenvalues}
\index{Dirac creation operator}
\index{Dirac anhillation operator}
\index{Dirac sea}
\index{anti-electrons}
\label{chap:qftLecture21b}

%%Peeter's lecture notes from class.  These may be incoherent and rough.
%%
%%These are notes for the UofT course PHY2403H, Quantum Field Theory, taught by Prof. Erich Poppitz, covering \textchapref{{1}} \citep{peskin1995introduction} content.
%\paragraph{DISCLAIMER: Very rough notes from class, with some additional side notes.}
%
%These are notes for the UofT course PHY2403H, Quantum Field Theory, taught by Prof. Erich Poppitz, fall 2018.
%%, covering \textchapref{{1}} \citep{peskin1995introduction} content.
%
\section{Lagrangian.}

\maketheorem{Dirac Hamiltonian.}{thm:qftLecture21b:1}{
\index{Dirac Hamiltonian}
The Dirac Hamiltonian is
\begin{equation*}
H
%=
%\int d^3 x
%\lr{
%- i \Psi^\dagger \gamma^0 \gamma^j \partial_j \Psi + m \Psi^\dagger \gamma^0 \Psi
%}
=
\int d^3 x
\Psi^\dagger
\lr{
- i \gamma^0 \gamma^j \partial_j \Psi + m \gamma^0
}
\Psi.
\end{equation*}
} % theorem

\begin{proof}
To prove \cref{thm:qftLecture21b:1}, we start with the spacetime expansion of the Dirac Lagrangian density
\begin{equation}\label{eqn:qftLecture21b:500}
\begin{aligned}
\LL_{\text{Dirac}}
&= \oPsi i \gamma^0 \partial_0 \Psi + i \oPsi \gamma^j \partial_j \Psi - m \oPsi \Psi \\
&= \Psi^\dagger \gamma^0 i \gamma^0 \partial_0 \Psi + i \Psi \gamma^0 \gamma^j \partial_j \Psi - m \Psi^\dagger \gamma^0 \Psi \\
&= \Psi^\dagger i \dot{\Psi} + i \Psi^\dagger \gamma^0 \gamma^j \partial_j \Psi - m \Psi^\dagger \gamma^0 \Psi.
\end{aligned}
\end{equation}
We see that the momentum conjugate to \( \Psi \) is
\begin{equation}\label{eqn:qftLecture21b:520}
\pi_\psi = \PD{\dot{\Psi}}{\LL} = i \Psi^\dagger.
\end{equation}
Computing the
Hamiltonian density in the usual way, we have
\begin{equation}\label{eqn:qftLecture21b:540}
\begin{aligned}
\calH_{\text{Dirac}}
&= \pi_\Psi \dot{\Psi} - \LL \\
&= i \Psi^\dagger \dot{\Psi} - \lr{
\Psi^\dagger i \dot{\Psi} + i \Psi^\dagger \gamma^0 \gamma^j \partial_j \Psi - m \Psi^\dagger \gamma^0 \Psi
} \\
&=
- i \Psi^\dagger \gamma^0 \gamma^j \partial_j \Psi + m \Psi^\dagger \gamma^0 \Psi.
\end{aligned}
\end{equation}
Integrating over a 3-volume provides
the Dirac Hamiltonian of
\cref{thm:qftLecture21b:1}.
\end{proof}

Now we want to examine the
action of \( -i \gamma^0 \gamma^j \partial_j + m \gamma^0 = \gamma^0 \lr{ -i \gamma^j \partial_j + m }\) on the plane wave solutions we have found.
\maketheorem{Hamiltonian action on Dirac plane wave solutions.}{thm:qftLecture21b:2}{
For \(\Psi_u = u(p) e^{-i p \cdot x}\), and \( \Psi_v = v(p) e^{i p \cdot x} \), we have
\begin{equation*}
\begin{aligned}
-\gamma^0 \lr{ i \gamma^j \partial_j - m } \Psi_u &=  p_0 \Psi_u \\
-\gamma^0 \lr{ i \gamma^j \partial_j - m } \Psi_v &= -p_0 \Psi_v.
\end{aligned}
\end{equation*}
} % theorem
\Cref{thm:qftLecture21b:2} shows that \( \Psi_u, \Psi_v \) are eigenvectors of the operator
\begin{equation}\label{eqn:qftLecture21b:n}
\gamma^0 \lr{ -i \gamma^j \partial_j + m },
\end{equation}
with eigenvalues \( \pm \omega_\Bp \).
\begin{proof}
These eigenvalue equations follow from the Dirac equation for \( \Psi_u, \Psi_v \).  These are
\begin{equation}\label{eqn:qftLecture21b:1000}
\begin{aligned}
\lr{ i \gamma^\mu \partial_\mu - m } u e^{-i p \cdot x}
&=
\lr{ i \gamma^j \partial_j + i \gamma^0 \partial_0 - m } u e^{-i p \cdot x} \\
&=
\lr{ i \gamma^j \partial_j + i(-i) \gamma^0 p_0 - m } u e^{-i p \cdot x}
\end{aligned}
\end{equation}
and
\begin{equation}\label{eqn:qftLecture21b:1020}
\begin{aligned}
\lr{ i \gamma^\mu \partial_\mu - m } v e^{i p \cdot x}
&=
\lr{ i \gamma^j \partial_j + i \gamma^0 \partial_0 - m } v e^{i p \cdot x} \\
&=
\lr{ i \gamma^j \partial_j + i(i) \gamma^0 p_0 - m } v e^{i p \cdot x}.
\end{aligned}
\end{equation}
Rearranging gives
\begin{equation}\label{eqn:qftLecture21b:1040}
\begin{aligned}
\lr{ i \gamma^j \partial_j - m } u e^{-i p \cdot x} &= - \gamma^0 p_0 u e^{-i p \cdot x} \\
\lr{ i \gamma^j \partial_j - m } v e^{ i p \cdot x} &= + \gamma^0 p_0 u e^{-i p \cdot x},
\end{aligned}
\end{equation}
and \cref{thm:qftLecture21b:2} follows immediately.
\end{proof}

\section{General solution and Hamiltonian.}
As with the Klein-Gordon equation, let's introduce a generic solution formed from linear combinations of our specific \( u^s(p) = u^s_\Bp, v^s(p) = v^s_\Bp \) solutions
%.  Such a zero time solution might look like
%Adding time back in the mix the most general wave function is
\begin{equation}\label{eqn:qftLecture21b:800}
\Psi(\Bx, t)
=
\sum_{s = 1}^2
\int \frac{d^3 p}{(2 \pi)^3 \sqrt{ 2 \omega_\Bp } }
\lr{
   e^{-i p \cdot x} u^s_\Bp a_\Bp^s
+
   e^{i p \cdot x} v^s_\Bp b_\Bp^s
}.
\end{equation}
\maketheorem{Ladder representation of Dirac Hamiltonian.}{thm:qftLecture21b:7}{
Substitution of the superposition \cref{eqn:qftLecture21b:800} into
the Dirac Hamiltonian of
\cref{thm:qftLecture21b:1} results in
\begin{equation*}
H_{\text{Dirac}}
=
\sum_{r = 1}^2
\int \frac{d^3 p }{(2\pi)^3 }
\omega_\Bp
\lr{
   a^{r \dagger}_\Bp
   a^r_\Bp
-
   b^{r \dagger}_{-\Bp}
   b^r_{-\Bp}
}.
\end{equation*}
} % theorem

\begin{proof}
Deferring interpretation slightly, we first
prove \cref{thm:qftLecture21b:7}, making the somewhat lazy guess that
all the time dependent terms will be wiped out.  This assumption allows us to use the zero time fields of our superposition solution
\begin{subequations}
\label{eqn:qftLecture21b:1060}
%\begin{equation}\label{eqn:qftLecture21b:580}
%\Psi(\Bx,0)
%%=
%%\sum_{s = 1}^2
%%\int \frac{d^3 p}{(2 \pi)^3 \sqrt{2 \omega_\Bp} }
%%\lr{
%%e^{i \Bp \cdot \Bx}
%%   u^s(\Bp) a^s_\Bp
%%   +
%%e^{-i \Bp \cdot \Bx}
%%   v^s(\Bp) b^s_{\Bp}
%%}
%=
%\sum_{s = 1}^2
%\int \frac{d^3 p}{(2 \pi)^3 \sqrt{2 \omega_\Bp} } e^{i \Bp \cdot \Bx}
%\lr{
%   u^s_\Bp a^s_\Bp
%   +
%   v^s_{-\Bp} b^s_{-\Bp}
%}.
%\end{equation}
%
\begin{equation}\label{eqn:qftLecture21b:820}
\Psi(\Bx,0)
=
\sum_{s = 1}^2
\int \frac{d^3 p}{(2 \pi)^3 \sqrt{2 \omega_\Bp} } e^{i \Bp \cdot \Bx}
\lr{
   u^s_\Bp a^s_\Bp
   +
   v^s_{-\Bp} b^s_{-\Bp}
}
\end{equation}
\begin{equation}\label{eqn:qftLecture21b:840}
\Psi^\dagger(\Bx,0)
=
\sum_{r = 1}^2
\int \frac{d^3 q}{(2 \pi)^3 \sqrt{2 \omega_\Bq} }
e^{-i \Bq \cdot \Bx}
\lr{
   u^{r \dagger}_\Bq a^{r \dagger}_\Bq
   +
   v^{r \dagger}_{-\Bq} b^{r \dagger}_{-\Bq}
}.
\end{equation}
\end{subequations}

Making use of the eigenvalue equations \cref{thm:qftLecture21b:2} the Hamiltonian is reduced to
\begin{equation}\label{eqn:qftLecture21b:1080}
\begin{aligned}
H_{\text{Dirac}}
&=
\sum_{r,s = 1}^2
\int \frac{d^3 x d^3 p d^3 q}{(2\pi)^6 2 \sqrt{ \omega_\Bp \omega_\Bq } }
e^{i (\Bp - \Bq ) \cdot \Bx }
\lr{
   u^{r \dagger}_\Bq a^{r \dagger}_\Bq
   +
   v^{r \dagger}_{-\Bq} b^{r \dagger}_{-\Bq}
}
\omega_\Bp
\lr{
   u^s_\Bp a^s_\Bp
   -
   v^s_{-\Bp} b^s_{-\Bp}
} \\
&=
\sum_{r,s = 1}^2
\int \frac{d^3 p }{(2\pi)^3 2 \cancel{\omega_\Bp} }
\lr{
   u^{r \dagger}_\Bp a^{r \dagger}_\Bp
   +
   v^{r \dagger}_{-\Bp} b^{r \dagger}_{-\Bp}
}
\cancel{\omega_\Bp}
\lr{
   u^s_\Bp a^s_\Bp
   -
   v^s_{-\Bp} b^s_{-\Bp}
} \\
&=
\inv{2} \sum_{r,s = 1}^2
\int \frac{d^3 p }{(2\pi)^3 }
\Biglr{
   u^{r \dagger}_\Bp
   u^s_\Bp
   a^{r \dagger}_\Bp
   a^s_\Bp
-
   u^{r \dagger}_\Bp
   v^s_{-\Bp}
   a^{r \dagger}_\Bp
   b^s_{-\Bp}
\\&\quad
+
   v^{r \dagger}_{-\Bp}
   u^s_\Bp
   b^{r \dagger}_{-\Bp}
   a^s_\Bp
-
   v^{r \dagger}_{-\Bp}
   v^s_{-\Bp}
   b^{r \dagger}_{-\Bp}
   b^s_{-\Bp}
}
,
\end{aligned}
\end{equation}
where care was taken not to commute any \( a, b\)'s.  Recall that
\begin{subequations}
\label{eqn:qftLecture21b:1140}
\begin{equation}\label{eqn:qftLecture21b:1100}
u^{r\dagger}_\Bp u^s_\Bp = v^{r\dagger}_\Bp v^s_\Bp = 2 \omega_\Bp \delta^{rs}
\end{equation}
\begin{equation}\label{eqn:qftLecture21b:1160}
u^{r\dagger}_\Bp v^s_{-\Bp} = v^{r\dagger}_{-\Bp} u^s_{\Bp} = 0.
\end{equation}
\end{subequations}
\Cref{eqn:qftLecture21b:1160} kills off our cross terms, and \cref{eqn:qftLecture21b:1100} wipes out one of the summation indexes
\begin{equation}\label{eqn:qftLecture21b:1120}
\begin{aligned}
H_{\text{Dirac}}
&=
\inv{2} \sum_{r,s = 1}^2
\int \frac{d^3 p }{(2\pi)^3 }
\lr{
   u^{r \dagger}_\Bp
   u^s_\Bp
   a^{r \dagger}_\Bp
   a^s_\Bp
-
   \cancel{u^{r \dagger}_\Bp
   v^s_{-\Bp}
}
   a^{r \dagger}_\Bp
   b^s_{-\Bp}
+
   \cancel{v^{r \dagger}_{-\Bp}
   u^s_\Bp
}
   b^{r \dagger}_{-\Bp}
   a^s_\Bp
-
   v^{r \dagger}_{-\Bp}
   v^s_{-\Bp}
   b^{r \dagger}_{-\Bp}
   b^s_{-\Bp}
} \\
&=
\sum_{r = 1}^2
\int \frac{d^3 p }{(2\pi)^3 }
\omega_\Bp
\lr{
   a^{r \dagger}_\Bp
   a^r_\Bp
-
   b^{r \dagger}_{-\Bp}
   b^r_{-\Bp}
}.
\end{aligned}
\end{equation}
We see above how the mixed terms were killed off nicely by \cref{eqn:qftLecture21b:1160}.  That also justifies the use of the zero-time fields in this derivation, which can also be seen explicitly without use of the zero-time fields \cref{problem:qftLecture21b:1}.
\end{proof}

\paragraph{Interpretation.}
With a minus sign in the Hamiltonian, there is no bound to the energy from below!  This makes it troublesome to interpret the \( a_\Bp \)'s and \( b_\Bp \)'s as the familiar raising and lowering operators that we know.

We can save the day, making the ``Dirac sea'' argument, roughly speaking that we can consider a set of completely full negative energy states, where creation of a particle makes a hole in one of those states\footnote{There was a long discussion of this topic in class that I was not able to capture in my notes.}, as sketched roughly in
\cref{fig:diracSea:diracSeaFig1}.
\imageFigure{../figures/phy2403-quantum-field-theory/diracSeaFig1}{Dirac Sea.}{fig:diracSea:diracSeaFig1}{0.3}
Such an argument does not work for bosons (photons, ...) since an arbitrary number of such particles can be stuffed into any given state.  It will turn out that our operators are fermions, which gets us out of this trouble.

We can also get out of this hole algebraically.  For \( X = a, b \), let
\begin{equation}\label{eqn:qftLecture21b:900}
\begin{aligned}
X^{s \dagger}_\Bp &= \tilde{X}^s_\Bp \\
X^s_\Bp &= \tilde{X}^{s \dagger}_\Bp.
\end{aligned}
\end{equation}
It turns out that some properties of our creation and annihilation operators are
\begin{equation}\label{eqn:qftLecture21b:920}
\begin{aligned}
(a^s_p)^2 &= 0 \\
(a^{s \dagger}_p)^2 &= 0 \\
(b^s_p)^2 &= 0 \\
(b^{s \dagger}_p)^2 &= 0,
\end{aligned}
\end{equation}
and
\begin{equation}\label{eqn:qftLecture21b:940}
\begin{aligned}
\symmetric{a^s_\Bp}{a^{r \dagger}_\Bq} &= \delta^{sr} \deltathree(\Bp - \Bq) \\
\symmetric{b^s_\Bp}{b^{r \dagger}_\Bq} &= \delta^{sr} \deltathree(\Bp - \Bq),
\end{aligned}
\end{equation}
where all other anticommutators are zero
\begin{equation}\label{eqn:qftLecture21b:960}
\symmetric{a^r}{b^s} =
\symmetric{a^r}{b^{s\dagger}} =
\symmetric{a^{r\dagger}}{b^s} =
\symmetric{a^{r\dagger}}{b^{s\dagger}} = 0.
\end{equation}

Such a substitution gives
\begin{equation}\label{eqn:qftLecture21b:980}
\begin{aligned}
H_{\text{Dirac}}
&=
\sum_{s = 1}^2
\int \frac{d^3 p}{(2 \pi)^3 }
\omega_\Bp
\lr{
\tilde{a}_\Bp^s a_\Bp^s - \tilde{b}_\Bp^s b_\Bp^s
} \\
&=
\sum_{s = 1}^2
\int \frac{d^3 p}{(2 \pi)^3 }
\omega_\Bp
\lr{
\tilde{a}_\Bp^{s} a_\Bp^s
+ b_\Bp^s \tilde{b}^{s}_\Bp + \delta^{ss} \deltathree( \Bp - \Bp )
} \\
&=
\sum_{s = 1}^2
\int \frac{d^3 p}{(2 \pi)^3 }
\lr{
\omega_\Bp
\lr{
\tilde{a}_\Bp^{ s } a_\Bp^s
+ b_\Bp^s \tilde{b}^{s}_\Bp
}
- 4 V_3 \frac{\omega_\Bp}{2}
}.
\end{aligned}
\end{equation}

We'll end up dropping the vacuum energy term.  We'll end up labeling the \( a \)'s as the operators associated with electrons, and the \( b \)'s with anti-electrons.

%}
%\EndNoBibArticle
