%
% Copyright � 2018 Peeter Joot.  All Rights Reserved.
% Licenced as described in the file LICENSE under the root directory of this GIT repository.
%
%{
\makeoproblem{Electromagnetic theory in 2+1 dimensions.}{qft:problemSet5:2}{2018 Hw5.II}{
Consider the theory in three space-time dimensions described by the Lagrangian \cref{eqn:ProblemSet5Problem2:100}.
This theory provides an
example of an internal symmetry, where the equations of motion are invariant, but the Lagrangian is not. In other
words, this an example of the general case considered in class, with the Lagrangian changing into a divergence under
a symmetry transformation (usually this was the case for spacetime symmetries). Terms in the Lagrangian that
change under an internal symmetry into a total derivative are often called ``Wess-Zumino terms.''
The theory in question is electrodynamics in 3d (hence the indices \( \mu, \nu = 0, 1, 2 \)), with the following Lagrangian:
\begin{equation}\label{eqn:ProblemSet5Problem2:100}
\LL = -\inv{4} F_{\alpha\beta} F^{\alpha\beta} + c \epsilon^{\mu\alpha\beta} A_\mu F_{\alpha\beta}.
\end{equation}
where \( \epsilon^{\mu\alpha\beta} \) is the totally antisymmetric tensor in 3d, with, say \( \epsilon^{012} = +1.\)

The Lagrangian is invariant under gauge transformations, \( \delta A_\mu = \partial_\mu \omega \), up to surface term.

The internal symmetry in question\footnote{Hint: The following identity might come handy, \( x \PD{x}{} \delta(x) = -\delta(x) \) (in words: the delta function is homogeneous of order \(-1\)). Also, if you think that helps,
feel free to normal order (does it help?).} is a shift of the vector field by a constant, \( \delta A_\mu =  \eta_\mu = \text{const}.\)
\makesubproblem{}{qft:problemSet5:2a}
Find the Noether current of this symmetry.
\makesubproblem{}{qft:problemSet5:2b}
Does the conservation of this current bring any new insight, compared to what you already know from the
equations of motion?
\makesubproblem{}{qft:problemSet5:2c}
What becomes of this Noether current if one only keeps the most relevant term in the Lagrangian? Is the
corresponding charge, constructed as we usually do, gauge invariant?
} % makeproblem

\makeanswer{qft:problemSet5:2}{

Let's explicitly calculate a few things before moving on to the specific problem questions.
\paragraph{EOM.}

The first order field derivative is
\begin{equation}\label{eqn:ProblemSet5Problem2:20}
\PD{A_\mu}{\LL} = c \epsilon^{\mu\alpha\beta} F_{\alpha\beta},
\end{equation}
and the second order derivative is
\begin{equation}\label{eqn:ProblemSet5Problem2:40}
\begin{aligned}
\PD{(\partial_\nu A_\mu)}{\LL}
&=
\lr{
   -\inv{2} F^{\alpha\beta}
   + c \epsilon^{\sigma\alpha\beta} A_\sigma
}
\PD{(\partial_\nu A_\mu)}{
F_{\alpha\beta}
} \\
&=
   -\inv{2} F^{\nu\mu}
   + c \epsilon^{\sigma\nu\mu} A_\sigma
   +\inv{2} F^{\mu\nu}
   - c \epsilon^{\sigma\mu\nu} A_\sigma \\
&=
   F^{\mu\nu}
   + 2 c \epsilon^{\sigma\nu\mu} A_\sigma,
\end{aligned}
\end{equation}
so the EOM is
\begin{equation}\label{eqn:ProblemSet5Problem2:120}
\begin{aligned}
0
&=
\PD{A_\mu}{\LL}
-
\partial_\nu
\PD{(\partial_\nu A_\mu)}{\LL} \\
&=
c \epsilon^{\mu\alpha\beta} F_{\alpha\beta}
-
   \partial_\nu F^{\mu\nu}
   - 2 c \epsilon^{\sigma\nu\mu} \partial_\nu A_\sigma \\
&=
c \epsilon^{\mu\alpha\beta} F_{\alpha\beta}
-
   \partial_\nu F^{\mu\nu}
   - c \epsilon^{\sigma\nu\mu} F_{\nu\sigma} \\
&=
c \epsilon^{\mu\alpha\beta} F_{\alpha\beta}
-
   \partial_\nu F^{\mu\nu}
   - c \epsilon^{\mu \beta\alpha} F_{\alpha\beta}
\end{aligned}
\end{equation}
or
\begin{equation}\label{eqn:ProblemSet5Problem2:180}
\partial_\nu F^{\mu\nu} = 2 c \epsilon^{\mu\alpha\beta} F_{\alpha\beta}.
\end{equation}
The given coupling from the Lagrangian results in a effective-current (not Noether current) in the equations of motion (just as the normal \( A_\mu j^\mu \) gives rise to equations of motion of the form \( \partial_\nu F^{\mu\nu} \sim j^\mu \).
\paragraph{Gauge invariance.}
Let's verify the invariance under the gauge transformation \( A_\mu \rightarrow A_\mu + \partial_\mu \omega \).

The field transforms as
\begin{equation}\label{eqn:ProblemSet5Problem2:60}
\begin{aligned}
F_{\mu\nu} \rightarrow
\partial_\mu\lr{
A_\nu + \partial_\nu \omega
}
-
\partial_\nu\lr{
A_\mu + \partial_\mu \omega
}
&=
F_{\mu\nu} + \lr{ \partial_{\mu\nu} - \partial_{\nu\mu} } \omega \\
&=
F_{\mu\nu}.
\end{aligned}
\end{equation}
Only the coupling term has any remnant after the gauge transformation
\begin{equation}\label{eqn:ProblemSet5Problem2:80}
\begin{aligned}
\LL
&\rightarrow
\LL
+ c \epsilon^{\mu\alpha\beta} \partial_\mu \omega F_{\alpha\beta} \\
&=
\LL
+ \partial_\mu \lr{ c \omega \epsilon^{\mu\alpha\beta} F_{\alpha\beta} }
- c \omega \epsilon^{\mu\alpha\beta} \partial_\mu F_{\alpha\beta} \\
&=
\LL
+ \partial_\mu \lr{ c \omega \epsilon^{\mu\alpha\beta} F_{\alpha\beta} } \\
&=
\LL
+ \partial_\mu J^\mu,
\end{aligned}
\end{equation}
where the divergence of
\begin{equation}\label{eqn:ProblemSet5Problem2:340}
J^\mu = c \omega \epsilon^{\mu\alpha\beta} F_{\alpha\beta}
\end{equation}
only contributes a boundary term to the action.  Note that
%Integrating that remainder by parts gives
%\begin{equation}\label{eqn:ProblemSet5Problem2:160}
%c \epsilon^{\mu\alpha\beta} \partial_\mu \omega F_{\alpha\beta}
%=
%c \epsilon^{\mu\alpha\beta}
%\lr{
%   \partial_\mu \lr{ \omega F_{\alpha\beta} }
%   -
%   \omega \partial_\mu F_{\alpha\beta}
%}
%=
%c \epsilon^{\mu\alpha\beta}
%   \partial_\mu \lr{ \omega F_{\alpha\beta} }
%-c
%\omega
%\epsilon^{\mu\alpha\beta}
%   \partial_\mu F_{\alpha\beta}
%=
%\partial_\mu \lr{ c \epsilon^{\mu\alpha\beta} \omega F_{\alpha\beta} },
%\end{equation}
we've identified and dropped a ``Bianchi identity'' sum along the way%
\footnote{
The proof that this slightly different (from usual) form of the Bianchi identity is zero is essentially the same as the usual one from standard electromagnetism.  In fact, the argument from Hw1 can be used as-is, simply removing one index from the \( \epsilon \)
\begin{equation*}%\label{eqn:ProblemSet5Problem2:200}
\begin{aligned}
\epsilon^{\mu\alpha\beta} \partial_\mu F_{\alpha\beta}
&=
\epsilon^{\mu\alpha\beta} \partial_\mu \lr{ \partial_\alpha A_\beta - \partial_\beta A_\alpha} \\
&=
\epsilon^{\mu\alpha\beta} \partial_\mu \partial_\alpha A_\beta
-\epsilon^{\mu\beta\alpha} \partial_\mu \partial_\beta A_\alpha \\
&=
2 \epsilon^{\mu\alpha\beta} \partial_\mu \partial_\alpha A_\beta.
\end{aligned}
\end{equation*}
This is zero since it's a contraction of an antisymmetric and symmetric tensor.
}. % digression

Summarizing, so far
%\begin{equation}\label{eqn:ProblemSet5Problem2:260}
\boxedEquation{eqn:ProblemSet5Problem2:260}{
\begin{aligned}
\partial_\nu F^{\mu\nu} &= 2 c \epsilon^{\mu\alpha\beta} F_{\alpha\beta} \\
\LL &\rightarrow \LL + \partial_\mu J^\mu \\
J^\mu &=
c \omega \epsilon^{\mu\alpha\beta} F_{\alpha\beta}.
\end{aligned}
}
%\end{equation}
\makeSubAnswer{}{qft:problemSet5:2a}
We've done some of the work to compute the Noether current, from \cref{eqn:ProblemSet5Problem2:40} we have
\begin{equation}\label{eqn:ProblemSet5Problem2:360}
\PD{(\partial_\mu A_\nu)}{\LL} = F^{\nu\mu} + 2 c \epsilon^{\sigma\mu\nu} A_\sigma.
\end{equation}
If \( \partial_\mu \omega = \eta_\mu \) then
\begin{equation}\label{eqn:ProblemSet5Problem2:280}
\omega = \eta_\mu x^\mu,
\end{equation}
and we can now compute the Noether current associated with the symmetry \( \delta A_\mu = \eta_\mu \)
\begin{equation}\label{eqn:ProblemSet5Problem2:220}
\begin{aligned}
j^\mu
&=
\PD{(\partial_\mu A_\nu)}{\LL} \delta A_\nu -J^\mu \\
&=
\lr{
   F^{\nu\mu}
   + 2 c \epsilon^{\sigma\mu\nu} A_\sigma
} \eta_\nu
-
c x^\sigma \eta_\sigma \epsilon^{\mu\alpha\beta} F_{\alpha\beta} \\
&=
\lr{
   F^{\nu\mu}
   + 2 c \epsilon^{\sigma\mu\nu} A_\sigma
   - c x^\sigma \ulDelta{\nu}{\sigma} \epsilon^{\mu\alpha\beta} F_{\alpha\beta}
} \eta_\nu,
\end{aligned}
\end{equation}
or
\boxedEquation{eqn:ProblemSet5Problem2:320}{
\begin{aligned}
   j^\mu
   &= T^{\mu\nu} \eta_\nu \\
T^{\mu\nu} &=
   F^{\nu\mu}
   + 2 c \epsilon^{\sigma\mu\nu} A_\sigma
   - c x^\nu \epsilon^{\mu\alpha\beta} F_{\alpha\beta}.
\end{aligned}
}
Introducing an energy momentum tensor, allows us to factor the uninteresting \( \eta_\nu \) constants out of the mix, and rewrite the current conservation equation \( 0 = \partial_\mu j^\mu \) as \( 0 = \partial_\mu T^{\mu\nu} \).

\makeSubAnswer{}{qft:problemSet5:2b}
Let's compute the divergence to see if we learn anything new
\begin{equation}\label{eqn:ProblemSet5Problem2:240}
\begin{aligned}
\partial_\mu T^{\mu\nu}
=
%\partial_\sigma F^{\mu\sigma} &=
\partial_\mu
\lr{
   F^{\nu\mu}
   + 2 c \epsilon^{\sigma\mu\nu} A_\sigma
   - c x^\nu \epsilon^{\mu\alpha\beta} F_{\alpha\beta}.
}
&=
2 c \epsilon^{\nu\alpha\beta} F_{\alpha\beta}
   + c \epsilon^{\sigma\mu\nu} F_{\mu\sigma}
   - c \partial_\mu x^\nu \epsilon^{\mu\alpha\beta} F_{\alpha\beta}
   - c x^\nu \epsilon^{\mu\alpha\beta} \partial_\mu F_{\alpha\beta} \\
&=
2 c \epsilon^{\nu\alpha\beta} F_{\alpha\beta}
   + c \epsilon^{\beta\alpha\nu} F_{\alpha\beta}
   - c \epsilon^{\nu\alpha\beta} F_{\alpha\beta}
   - c x^\nu \epsilon^{\mu\alpha\beta} \partial_\mu F_{\alpha\beta} \\
&=
2 c \epsilon^{\nu\alpha\beta} F_{\alpha\beta}
   - c \epsilon^{\nu\alpha\beta} F_{\alpha\beta}
   - c \epsilon^{\nu\alpha\beta} F_{\alpha\beta}
   - c x^\nu \epsilon^{\mu\alpha\beta} \partial_\mu F_{\alpha\beta},
\end{aligned}
\end{equation}
leaving just
\begin{equation}\label{eqn:ProblemSet5Problem2:380}
0 = - c x^\nu \epsilon^{\mu\alpha\beta} \partial_\mu F_{\alpha\beta},
\end{equation}
which holds for any values \( x^\nu \), so the current conservation equation only tells us that
\begin{equation}\label{eqn:ProblemSet5Problem2:400}
\epsilon^{\mu\alpha\beta} \partial_\mu F_{\alpha\beta} = 0,
\end{equation}
which is just the Bianchi identity that we already knew.

\makeSubAnswer{}{qft:problemSet5:2c}
I had some trouble figuring out what was meant by keeping only the most relevant coupling, given that we only have one coupling to start with?  Additionally
\begin{enumerate}
\item The current for the \( c = 0 \) case would clearly be gauge invariant since it only has only a \( F^{\mu\nu} \) term.  That current and the associated charge is not terribly interesting, and it doesn't make sense to ask about it.
\item It doesn't make sense to consider a Lagrangian with only the coupling term.  The EOM for that case doesn't even have derivative terms.
\end{enumerate}

In class, \underline{relevant} was defined as a coupling for which the coupling constant had positive mass dimension.  For our system, in natural units the action \( \int d^3 x \LL \) is dimensionless, which allows us to assign dimensions to some of our terms.  Those are
\begin{equation}\label{eqn:ProblemSet5Problem2:420}
\begin{aligned}
1
&= [ L^3 \LL ] \\
&= M^{-3} [ \LL ] \\
&\implies [ \LL ] = M^3,
\end{aligned}
\end{equation}
\begin{equation}\label{eqn:ProblemSet5Problem2:440}
[F]^2 = M^3 \implies [F] = M^{3/2},
\end{equation}
\begin{equation}\label{eqn:ProblemSet5Problem2:460}
[F] = L^{-1} [A] = M [A] = M^{3/2} \implies [A] = M^{1/2},
\end{equation}
\begin{equation}\label{eqn:ProblemSet5Problem2:480}
M^3 = [c A F] = [c] M^{1/2} M^{3/2} = [c] M^2 \implies [c] = M.
\end{equation}
By this criteria the \( c \) coupling is relevant, since it has a positive mass dimension.  Since it is also the only coupling, it is also the most relevant coupling, and my conclusion is that we are being asked whether the charge associated with the current calculated above is gauge invariant.

With that conclusion in place, let's continue to calculate the charge, which is
\begin{equation}\label{eqn:ProblemSet5Problem2:500}
\begin{aligned}
Q
&= \int d^3 x j^0 \\
&= \int d^3 x T{0\nu} \eta_\nu \\
&=
\int d^3 x
\lr{
   F^{\nu 0}
   + 2 c \epsilon^{\sigma 0 \nu} A_\sigma
   - c x^\nu \epsilon^{0 \alpha\beta} F_{\alpha\beta}
}.
\end{aligned}
\end{equation}
Under gauge transformation, this becomes
\begin{equation}\label{eqn:ProblemSet5Problem2:520}
Q \rightarrow Q
+
\int d^3 x
   2 c \epsilon^{\sigma 0 \nu} \eta_\sigma \eta_\nu,
\end{equation}
but
\begin{equation}\label{eqn:ProblemSet5Problem2:540}
\begin{aligned}
\epsilon^{\sigma 0 \nu} \eta_\sigma \eta_\nu
&=
\epsilon^{1 0 2} \eta_1 \eta_2
+
\epsilon^{2 0 1} \eta_2 \eta_1 \\
&=
\lr{ -1 + 1 } \eta_1 \eta_2
= 0,
\end{aligned}
\end{equation}
so, \underline{yes}, the charge for this symmetry is gauge invariant.

} % answer
%}
