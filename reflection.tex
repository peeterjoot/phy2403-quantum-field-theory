%
% Copyright � 2018 Peeter Joot.  All Rights Reserved.
% Licenced as described in the file LICENSE under the root directory of this GIT repository.
%
%{
\input{../latex/blogpost.tex}
\renewcommand{\basename}{reflection}
%\renewcommand{\dirname}{notes/phy1520/}
\renewcommand{\dirname}{notes/ece1228-electromagnetic-theory/}
%\newcommand{\dateintitle}{}
%\newcommand{\keywords}{}

\input{../latex/peeter_prologue_print2.tex}

\usepackage{peeters_layout_exercise}
\usepackage{peeters_braket}
\usepackage{peeters_figures}
\usepackage{siunitx}
\usepackage{verbatim}
%\usepackage{mhchem} % \ce{}
%\usepackage{macros_bm} % \bcM
%\usepackage{macros_qed} % \qedmarker
%\usepackage{txfonts} % \ointclockwise

\beginArtNoToc

\generatetitle{Reflection using Pauli matrices.}
%\chapter{Reflection using Pauli matrices.}
%\label{chap:reflection}

In class yesterday (lecture 19, notes not yet posted) we used \( \Bsigma^\T = -\sigma_2 \Bsigma \sigma_2 \), which implicitly shows that \( (\Bsigma \cdot \Bx)^\T \) is a reflection about the y-axis.
This form of reflection will be familiar to a student of geometric algebra (see \citep{doran2003gap}).  I can't recall any mention of the geometrical reflection identity from when I took QM.  It's a fun exercise to demonstrate the reflection identity when constrained to the Pauli matrix notation.

\maketheorem{Reflection about a normal.}{thm:reflection:1}{
Given a unit vector \( \ncap \in \bbR^3 \) and a vector \( \Bx \in \bbR^3 \) the reflection of \( \Bx \) about a plane with normal \( \ncap \) can be represented in Pauli notation as
\begin{equation*}
-\Bsigma \cdot \ncap \Bsigma \cdot \Bx \Bsigma \cdot \ncap.
\end{equation*}
} % theorem

In standard vector notation, we can decompose a vector into its projective and rejective components
\begin{dmath}\label{eqn:reflection:20}
\Bx = (\Bx \cdot \ncap) \ncap + \lr{ \Bx - (\Bx \cdot \ncap) \ncap }.
\end{dmath}
A reflection about the plane normal to \( \ncap \) just flips the component in the direction of \( \ncap \), leaving the rest unchanged.  That is
\begin{dmath}\label{eqn:reflection:40}
-(\Bx \cdot \ncap) \ncap + \lr{ \Bx - (\Bx \cdot \ncap) \ncap }
=
\Bx - 2 (\Bx \cdot \ncap) \ncap.
\end{dmath}
We may write this in \( \Bsigma \) notation as
\begin{dmath}\label{eqn:reflection:60}
\Bsigma \cdot \Bx - 2 \Bx \cdot \ncap \Bsigma \cdot \ncap.
\end{dmath}
We also know that
\begin{dmath}\label{eqn:reflection:80}
\begin{aligned}
\Bsigma \cdot \Ba \Bsigma \cdot \Bb &= a \cdot b + i \Bsigma \cdot (\Ba \cross \Bb) \\
\Bsigma \cdot \Bb \Bsigma \cdot \Ba &= a \cdot b - i \Bsigma \cdot (\Ba \cross \Bb),
\end{aligned}
\end{dmath}
or
\begin{dmath}\label{eqn:reflection:100}
a \cdot b = \inv{2} \symmetric{\Bsigma \cdot \Ba}{\Bsigma \cdot \Bb},
\end{dmath}
where \( \symmetric{\Ba}{\Bb} \) is the anticommutator of \( \Ba, \Bb \).
Inserting \cref{eqn:reflection:100} into \cref{eqn:reflection:60} we find that the reflection is
\begin{dmath}\label{eqn:reflection:120}
\Bsigma \cdot \Bx -
\symmetric{\Bsigma \cdot \ncap}{\Bsigma \cdot \Bx}
\Bsigma \cdot \ncap
=
\Bsigma \cdot \Bx -
{\Bsigma \cdot \ncap}{\Bsigma \cdot \Bx}
\Bsigma \cdot \ncap
-
{\Bsigma \cdot \Bx}{\Bsigma \cdot \ncap}
\Bsigma \cdot \ncap
=
\Bsigma \cdot \Bx -
{\Bsigma \cdot \ncap}{\Bsigma \cdot \Bx}
\Bsigma \cdot \ncap
-
{\Bsigma \cdot \Bx}
=
-
{\Bsigma \cdot \ncap}{\Bsigma \cdot \Bx}
\Bsigma \cdot \ncap,
\end{dmath}
which completes the proof.

When we expand \( (\Bsigma \cdot \Bx)^\T \) and find
\begin{dmath}\label{eqn:reflection:140}
(\Bsigma \cdot \Bx)^\T
=
\sigma^1 x^1 - \sigma^2 x^2 + \sigma^3 x^3,
\end{dmath}
it is clear that this coordinate expansion is a reflection about the y-axis.  Knowing the reflection formula above provides a rationale for why we might want to write this in the compact form \( -\sigma^2 (\Bsigma \cdot \Bx) \sigma^2 \), which might not be obvious otherwise.

%}
\EndArticle
%\EndNoBibArticle
