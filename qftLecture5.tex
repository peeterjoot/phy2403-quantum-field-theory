%
% Copyright � 2017 Peeter Joot.  All Rights Reserved.
% Licenced as described in the file LICENSE under the root directory of this GIT repository.
%
\input{../latex/blogpost.tex}
\renewcommand{\basename}{qft5}
\renewcommand{\dirname}{notes/phy2403/}
\newcommand{\keywords}{PHY2403H}
\input{../latex/peeter_prologue_print2.tex}

%\usepackage{phy2403}
\usepackage{peeters_braket}
%\usepackage{peeters_layout_exercise}
\usepackage{peeters_figures}
\usepackage{mathtools}
\usepackage{siunitx}
\usepackage{macros_cal} % LL

\beginArtNoToc
\generatetitle{PHY2403H Quantum Field Theory.  Lecture 5: XXX.  Taught by Prof.\ Erich Poppitz}
%\chapter{XXX}
\label{chap:qft5}

\paragraph{Disclaimer}

\paragraph{DISCLAIMER: Very rough notes from class.  Some additional side notes, but otherwise barely edited.}

These are notes for the UofT course PHY2403H, Quantum Field Theory I, taught by Prof. Erich Poppitz fall 2018.
%, covering \textchapref{{1}} \citep{peskin1995introduction} content.

\section{Canonical quantization}

\begin{dmath}\label{eqn:qftLecture5:20}
L = \int d^3 x
\lr{
\inv{2} \lr{\partial_0 \phi}^2 - \inv{2} \lr{\spacegrad \phi}^2 - \frac{m^2}{2} \phi^2  - \frac{\lambda}{4} \phi^4
}
\end{dmath}

\begin{dmath}\label{eqn:qftLecture5:40}
S = \int dt L = \int dt d^3 x \LL
\end{dmath}

\begin{dmath}\label{eqn:qftLecture5:60}
\Pi(\Bx, t) = \frac{\delta \LL}{\delta \phidot(\Bx, t) } = \PD{\phidot(\Bx, t)}{\LL}
\end{dmath}

\begin{dmath}\label{eqn:qftLecture5:80}
H = \int d^3 x \lr{ \Pi(\Bx, t) \phidot(\Bx, t) - \LL }
= \int d^3 x
\lr{ \Pi^2 /2 + (\spacegrad phi)^2 + \inv{2} m^2 \phi^2 + \frac{\lambda}{4} \phi^4 }
\end{dmath}

\begin{dmath}\label{eqn:qftLecture5:100}
H = \int d^3 x \calH(\Bx, t)
\end{dmath}
\begin{dmath}\label{eqn:qftLecture5:120}
\calH(\Bx, t) =
\lr{ \Pi^2 /2 + (\spacegrad phi)^2 + \inv{2} m^2 \phi^2 + \frac{\lambda}{4} \phi^4 }
\end{dmath}

\paragraph{Canonical Commutation Relations (CCR)}:

\begin{dmath}\label{eqn:qftLecture5:140}
\antisymmetric{\hat{\Pi}(\Bx, t)}{\hat{\phi}(\By, t)} = -i \delta^3 (\Bx - \By)
\end{dmath}

This is in analogy to

\begin{dmath}\label{eqn:qftLecture5:160}
\antisymmetric{\hat{p}_i}{\hat{q}_j} = -i \delta_{ij},
\end{dmath}

To choose a representation, we may map the \( \Psi \) of QM \( \rightarrow \) to a wave functional \( \Psi[\phi] \)

\begin{dmath}\label{eqn:qftLecture5:180}
\hat{\phi}(\By, t) \Psi[\phi] = \phi(\By, t) \Psi[\phi]
\end{dmath}

This is similar to the QM wave functions

\begin{dmath}\label{eqn:qftLecture5:200}
\hat{q}_i \Psi(\setlr{q}) = q_i \Psi(q)
\hat{p}_i \Psi(\setlr{q}) =
-i \PD{q_i}{}
\Psi(p)
\end{dmath}

\begin{dmath}\label{eqn:qftLecture5:220}
\hat{\Pi}(\Bx, t) = -i \frac{\delta}{\delta \phi(\Bx, t)}
\end{dmath}

To quantize the Hamiltonian we just add hats

\begin{dmath}\label{eqn:qftLecture5:240}
\calH(\Bx, t)
=
\lr{ \hat{\Pi}^2 /2 + (\spacegrad \hat{\phi})^2 + \inv{2} m^2 \hat{\phi}^2 + \frac{\lambda}{4} \hat{\phi}^4 }
\end{dmath}

Recall the QM SHO
\begin{dmath}\label{eqn:qftLecture5:260}
\hat{H} = \inv{2} \hat{p}^2 + \inv{2} \omega^2 \hat{q}^2,
\end{dmath}
where
\begin{dmath}\label{eqn:qftLecture5:280}
\antisymmetric{\hat{p}}{\hat{q}} = -i
\end{dmath}

Recall the Heisenberg picture time evoluation operators

\begin{dmath}\label{eqn:qftLecture5:300}
\ddt{\hat{p}}
= i \antisymmetric{\hat{H}}{\hatp}
= i \frac{\omega^2}{2} \antisymmetric{\hatq^2}{\hatp} = i \omega^2 \hatq i = -i \omega^2 \hatq
\end{dmath}

\begin{dmath}\label{eqn:qftLecture5:320}
\ddt{\hat{q}}
= i \antisymmetric{\hat{H}}{\hatq} = i \inv{2} \antisymmetric{\hatp^2}{\hatq} = i(-i)\hatp = \hatp
\end{dmath}

so
\begin{dmath}\label{eqn:qftLecture5:340}
\ddot{\hat{q}} = \dot{\hat{p}} = - \omega^2 \hatq
\end{dmath}

We see that the Heisenberg operators obey the classical equations of motion.
Now we want to try this with the quantized QFT we've started with

\begin{dmath}\label{eqn:qftLecture5:360}
\dot{\hat{\Pi}}(\Bx, t)
= i \antisymmetric{\hatH}{\hat{\Pi}(\Bx, t)}
=
i \int d^3 \lr{ \inv{2} \antisymmetric{ \spacegrad \phi(\By) }^2 }{\Pihat{\Bx) }
+
i \int d^3 \lr{ \frac{m^2}{2} \antisymmetric{ \hatphi(\By)^2 }{\Pihat{\Bx) }
+
i \frac{\lambda}{4} \int d^3 \antisymmetric{ \hatphi(\By)^4 }{\Pihat{\Bx) }
=
i \int d^3 y \spacegrad \phi(\By) \antisymmetric{\spacegrad \phi(\By)}{\Pi(\Bx)}
+
i m^2 \int d^3 y \spacegrad \phi(\By) (+i) \delta^3(\Bx - \By)
+
i \frac{\lambda}{4} \int d^3 y 4 \phi(\By)^3 (+i) \delta^3(\Bx - \By)
=
i \int d^3 y \spacegrad \phi(\By) (+i) \spacegrad_\By \delta^3(\Bx - \By)
+
i m^2 \int d^3 y \spacegrad \phi(\By) (+i) \delta^3(\Bx - \By)
+
i \frac{\lambda}{4} \int d^3 y 4 \phi(\By)^3 (+i) \delta^3(\Bx - \By)
=
\lr{ -m^2 \phi - \lambda \phihat^3 + \spacegrad^2 \hatphi }(\Bx).
\end{dmath}
\begin{dmath}\label{eqn:qftLecture5:380}
\dot{\hat{\Pi}}(\Bx, t)
= i \antisymmetric{\hatH}{\hat{\phi}(\Bx, t)}
= i \inv{2} \int d^3 y \antisymmetric{\Pihat^2(\By)}{\hat{\phi}(\Bx)}
= \Pihat(\Bx)
\end{dmath}
\begin{dmath}\label{eqn:qftLecture5:400}
\ddot{\hat{\Pi}}(\Bx, t)
%= i \antisymmetric{\hatH}{\dot{\hat{\phi}}(\Bx, t)}
= \ddot{\Pihat}
=
\spacegrad^2 \hatphi
-m^2 \phi - \lambda \phihat^3.
\end{dmath}
That is
\begin{dmath}\label{eqn:qftLecture5:420}
\ddot{\phicap} - \spacegrad^2 \phihat + m^2 \phihat + \lambda \phihat^3 = 0,
\end{dmath}
which is the classical Euler-Lagrange equation, also obeyed by the
Heisenburg operator \( \phi(\Bx, t) \).  When \( \lambda = 0 \) this is the Klein-Gordon equation.

Dropping hats, let's consider the momentum space representation of our operators

\begin{dmath}\label{eqn:qftLecture5:440}
\phi(\Bx, t) = \int \frac{d^3 p}{(2\pi)^3} e^{i \Bp \cdot \Bx} \tilde{\phi}(\Bp, t)
\end{dmath}

\begin{dmath}\label{eqn:qftLecture5:460}
\phi^\conj\Bx, t) = \phi(\Bx, t) \leftrightarrow \tilde{phi}(\Bp, t) = \tilde{\phi}^\conj(-\Bp, t)
\end{dmath}

\begin{dmath}\label{eqn:qftLecture5:480}
\tilde{phi}(\Bp, t)
= \int d^3 x e^{-i \Bp \cdot \Bx} \phi(\Bx, t)
= \int d^3 x e^{-i \Bp \cdot \Bx} \int \frac{d^3 q}{(2 \pi)^3} e^{i \Bq \cdot \Bx} \tilde{\phi}(\Bq, t)
\end{dmath}

so

\boxedEquation{eqn:qftLecture5:500}{
\int d^3 x e^{i \BA \cdot \Bx} = (2 \pi)^3 \delta^3(\BA)
}

Wnt the EOM for \( \tilde{\phi}(\Bp, t) \) where the operator obeys the KG equation

\begin{dmath}\label{eqn:qftLecture5:520}
\lr{ \partial_t^2 - \spacegrad^2 + m^2 } \phi(\Bx, t) = 0
\end{dmath}

Inserting the transform relation \cref{eqn:qftLecture5:440} we get

Fixme: WORK OUT:
\begin{dmath}\label{eqn:qftLecture5:n}
\int \frac{d^3 p}{(2 \pi)^3} e^{i \Bp \cdot \Bx} \lr{
\ddot{\tphi}(\Bp, t) + \lr( \Bp^2 + m^2 } \tphi(\Bp, t) } = 0
\end{dmath}

With
\begin{dmath}\label{eqn:qftLecture5:n}
\omega_\Bq = \sqrt{ \Bq^2 + m^2 }
\end{dmath}
we find
\boxedEquation{eqn:qftLecture5:n}{
\ddot{\tphi}(\Bq, t) = - \omega_\Bq^2 \tphi(\Bq, t).
}
The Fourier components of the HP operators are SHOs!

As we have SHO's and know how to deal with these in QM, we use the same strategy, introducing raising and lowering operators
\begin{dmath}\label{eqn:qftLecture5:n}
\tphi(\Bq, t) = \inv{\sqrt{2 \omega_q}} \lr{ e^{-i \omega_\Bq t } a_\Bq + e^{i \omega_\Bq t} a^\conj_{-\Bq}
}
\end{dmath}

CHECK!! :
\begin{dmath}\label{eqn:qftLecture5:n}
\tphi^\conj(\Bq, t) = \tphi(-\Bq, t).
\end{dmath}

We will find (Wednesday) that
\begin{dmath}\label{eqn:qftLecture5:n}
\antisymmetric{\hata_\Bq}{\hata^+_\Bp} = \delta^3(\Bp - \Bq) (2 \pi)^3.
\end{dmath}

These are equivalent to 
\begin{dmath}\label{eqn:qftLecture5:n}
\antisymmetric{\hatPi(\By, t)}{\hatphi(\Bx, t)} = -i \delta^3(\Bx - \By)
\end{dmath}

\EndArticle
%\EndNoBibArticle
