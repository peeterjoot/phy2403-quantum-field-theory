%
% Copyright � 2018 Peeter Joot.  All Rights Reserved.
% Licenced as described in the file LICENSE under the root directory of this GIT repository.
%
%{
%%%\input{../latex/blogpost.tex}
%%%\renewcommand{\basename}{qftLecture8}
%%%\renewcommand{\dirname}{notes/phy2403/}
%%%\newcommand{\keywords}{PHY2403H}
%%%\input{../latex/peeter_prologue_print2.tex}
%%%
%%%%\usepackage{phy2403}
%%%\usepackage{peeters_braket}
%%%\usepackage{peeters_layout_exercise}
%%%\usepackage{peeters_figures}
%%%\usepackage{mathtools}
%%%\usepackage{siunitx}
%%%\usepackage{macros_cal} % LL
%%%
%%%\beginArtNoToc
%%%\generatetitle{PHY2403H Quantum Field Theory.  Lecture 8: 1st Noether theorem, spacetime translation current, energy momentum tensor, dilatation current.  Taught by Prof.\ Erich Poppitz}
%\chapter{1st Noether theorem, spacetime translation current, energy momentum tensor, dilatation current.}
\index{spacetime translation current}
\index{energy momentum tensor}
\index{dilatation current}
\label{chap:qftLecture8}

%%%\paragraph{DISCLAIMER: Very rough notes from class, with some additional side notes.}
%%%
%%%These are notes for the UofT course PHY2403H, Quantum Field Theory I, taught by Prof. Erich Poppitz fall 2018.
%%%%, covering \textchapref{{1}} \citep{peskin1995introduction} content.
%%%
\section{1st Noether theorem.}
\index{Noether's theorem!1st}

Recall that, given a transformation
\begin{equation}\label{eqn:qftLecture8:20}
\phi(x) \rightarrow \phi(x) + \delta \phi(x),
\end{equation}
such that the transformation of the Lagrangian is only changed by a total derivative
\begin{equation}\label{eqn:qftLecture8:40}
\LL(\phi, \partial_\mu \phi) \rightarrow
\LL(\phi, \partial_\mu \phi)
+ \partial_\mu J_\epsilon^\mu,
\end{equation}
then there is a conserved current
\begin{equation}\label{eqn:qftLecture8:60}
j^\mu = \PD{(\partial_\mu \phi)}{\LL} \delta_\epsilon \phi - J_\epsilon^\mu.
\end{equation}
Here \( \epsilon \) is an x-independent quantity (i.e. a \underline{global symmetry}).
This is in contrast to ``gauge symmetries'', which can be more accurately be categorized as a redundancy in the description.

As an example, for \( \LL = (\partial_\mu \phi \partial^\mu \phi - m^2 \phi^2)/2 \), let
\begin{equation}\label{eqn:qftLecture8:80}
\phi(x) \rightarrow \phi(x) - a^\mu \partial_\mu \phi.
\end{equation}
The Lagrangian density transforms as
\begin{equation}\label{eqn:qftLecture8:100}
\begin{aligned}
\LL(\phi, \partial_\mu \phi)
&\rightarrow
\LL(\phi, \partial_\mu \phi)
- a^\mu \partial_\mu \LL \\
&=
\LL(\phi, \partial_\mu \phi)
+ \partial_\mu \lr{ -{\delta^\mu}_\nu a^\nu \LL }.
\end{aligned}
\end{equation}
Here \( J^\mu_\epsilon = \evalbar{J^\mu_\epsilon}{\epsilon = a^\nu} \), and the current is
\begin{equation}\label{eqn:qftLecture8:120}
J^\mu = (\partial^\mu \phi)(-a^\nu \partial_\nu \phi) + {\delta^{\mu}}_\nu a^\nu \LL.
\end{equation}
In particular, we have one such current for each \( \nu \), and we write
\begin{equation}\label{eqn:qftLecture8:140}
{T^\mu}_\nu =
-(\partial^\mu \phi)(\partial_\nu \phi) + {\delta^{\mu}}_\nu \LL.
\end{equation}
By Noether's theorem, we must have
\begin{equation}\label{eqn:qftLecture8:160}
\partial_\mu
{T^\mu}_\nu = 0, \quad \forall \nu.
\end{equation}

\paragraph{Check:}
\begin{equation}\label{eqn:qftLecture8:1380}
\begin{aligned}
\partial_\mu {T^\mu}_\nu
&=
-(\partial_\mu \partial^\mu \phi)(\partial_\nu \phi)
-(\partial^\mu \phi)(\partial_\mu \partial_\nu \phi)
+ {\delta^{\mu}}_\nu
\partial_\mu \lr{
\inv{2} \partial_\alpha \phi \partial^\alpha \phi - \frac{m^2}{2} \phi^2
} \\
&=
-(\partial_\mu \partial^\mu \phi)(\partial_\nu \phi)
-(\partial^\mu \phi)(\partial_\mu \partial_\nu \phi)
+
\inv{2} (\partial_\nu \partial_\mu \phi) (\partial^\mu \phi )
\\&\quad
+
\inv{2} (\partial_\mu \phi) (\partial_\nu \partial^\mu \phi )
- m^2 (\partial_\nu \phi) \phi \\
&=
-\lr{ \partial_\mu \partial^\mu \phi + m^2 \phi }(\partial_\nu \phi)
-(\partial_\mu \phi)(\partial^\mu \partial_\nu \phi)
+
\inv{2} (\partial_\nu \partial^\mu \phi) (\partial_\mu \phi )
\\&\quad
+
\inv{2} (\partial_\mu \phi) (\partial_\nu \partial^\mu \phi )
&= 0.
\end{aligned}
\end{equation}

\paragraph{Example: our potential Lagrangian}
\begin{equation}\label{eqn:qftLecture8:180}
\LL = \inv{2} \partial^\mu \phi \partial_\nu \phi - \frac{m^2}{2} \phi^2 - \frac{\lambda}{4} \phi^4
\end{equation}
Written with upper indexes
\begin{equation}\label{eqn:qftLecture8:200}
\begin{aligned}
T^{\mu\nu}
&= -(\partial^\mu \phi)(\partial^\nu \phi) + g^{\mu\nu} \LL \\
&= -(\partial^\mu \phi)(\partial^\nu \phi) + g^{\mu\nu} \lr{
\inv{2} \partial^\alpha \phi \partial_\alpha \phi - \frac{m^2}{2} \phi^2 - \frac{\lambda}{4} \phi^4
}
\end{aligned}
\end{equation}

There are 4 conserved currents \( J^{\mu(\nu)} = T^{\mu\nu} \).  Observe that this is symmetric (\( T^{\mu\nu} = T^{\nu\mu} \)).

We have four associated charges
\begin{equation}\label{eqn:qftLecture8:220}
Q^\nu = \int d^3 x T^{0 \nu}.
\end{equation}
We call
\begin{equation}\label{eqn:qftLecture8:240}
Q^0 = \int d^3 x T^{0 0},
\end{equation}
the energy density, and call
\begin{equation}\label{eqn:qftLecture8:260}
P^i = \int d^3 x T^{0 i},
\end{equation}
(i = 1,2,3) the momentum density.

writing this out explicitly the energy density is
\begin{equation}\label{eqn:qftLecture8:280}
\begin{aligned}
T^{00}
&= - \dot{\phi}^2 + \inv{2} \lr{ \dot{\phi}^2 - (\spacegrad \phi)^2 - \frac{m^2}{2}\phi^2 - \frac{\lambda}{4} \phi^4} \\
&= -\lr{
\inv{2} \dot{\phi}^2 + \inv{2} (\spacegrad \phi)^2 + \frac{m^2}{2}\phi^2 + \frac{\lambda}{4} \phi^4
},
\end{aligned}
\end{equation}
and
\begin{equation}\label{eqn:qftLecture8:300}
T^{0i} = \partial^0 \phi \partial^i \phi,
\end{equation}
\begin{equation}\label{eqn:qftLecture8:320}
P^{i} = -\int d^3 x\partial^0 \phi \partial^i \phi.
\end{equation}
Since the energy density is negative definite (due to an arbitrary choice of translation sign), let's redefine \( T^{\mu\nu} \) to have a positive sign
\begin{equation}\label{eqn:qftLecture8:340}
T^{00}
\equiv
\inv{2} \dot{\phi}^2 + \inv{2} (\spacegrad \phi)^2 + \frac{m^2}{2} \phi^2 + \frac{\lambda}{4} \phi^4,
\end{equation}
and
\begin{equation}\label{eqn:qftLecture8:360}
P^{i} = \int d^3 x\partial^0 \phi \partial^i \phi
\end{equation}

As an operator the charge is
\begin{equation}\label{eqn:qftLecture8:380}
\hatQ = \int d^3 x \hatT^{00} =
\int d^3 x
\lr{
\inv{2} \pihat^2 + \inv{2} (\spacegrad \phihat)^2 + \frac{m^2}{2} \phihat^2 + \frac{\lambda}{4} \phihat^4
},
\end{equation}
and the momenta are
\begin{equation}\label{eqn:qftLecture8:400}
\hatP^{i} = \int d^3 x \pihat \partial^i \phi.
\end{equation}

We showed that
\begin{equation}\label{eqn:qftLecture8:420}
\ddt \hatO = i \antisymmetric{\hatH}{\hatO}.
\end{equation}
This implied that \( \phihat, \pihat \) obey the classical equations of motion
\begin{equation}\label{eqn:qftLecture8:440}
\ddt \phihat = i \antisymmetric{\hatH}{\phihat} = \ddt{\pihat}
\end{equation}
\begin{equation}\label{eqn:qftLecture8:460}
\ddt \pihat = i \antisymmetric{\hatH}{\pihat} = ...
\end{equation}

In terms of creation and annihilation operators (for the \( \lambda = 0 \) free field), up to a constant
\begin{equation}\label{eqn:qftLecture8:480}
\begin{aligned}
\hatH
&= \int d^3 x \hatT^{00} \\
&= \int \frac{d^3 p}{(2 \pi)^3} \omega_\Bp \hata_\Bp^\dagger \hata_\Bp.
\end{aligned}
\end{equation}
It can be shown (\cref{chap:momentum}) that the operator form of the field momentum is
\begin{equation}\label{eqn:qftLecture8:500}
\begin{aligned}
\hatP^i
&= \int d^3 x \pihat \partial^i \phihat \\
&= \int \frac{d^3 p}{(2 \pi)^3} p^i \hata_\Bp^\dagger \hata_\Bp.
\end{aligned}
\end{equation}
Now we see the energy and momentum as conserved quantities associated with spacetime translation.
\section{Unitary operators.}
\index{unitary operator}
\index{translation operator}
In QM we say that \( \hat{\BP} \) ``generates translations''.
With \( \hat{\BP} \equiv -i \Hbar \spacegrad \) that translation is
\begin{equation}\label{eqn:qftLecture8:520}
\hatU(\Ba) = e^{i \Ba \cdot \hat{\BP}} = e^{-\Ba \cdot \spacegrad}.
\end{equation}
In particular
\begin{equation}\label{eqn:qftLecture8:540}
\begin{aligned}
\bra{\Bx} \hatU(\Ba) \ket{\psi}
&=
\int d^3 p
\bra{\Bx} \hatU(\Ba) \ket{\Bp}\braket{\Bp}{\psi} \\
&=
\int d^3 p
\bra{\Bx} e^{i \Ba \cdot \hat{\BP}} \ket{\Bp}\braket{\Bp}{\psi} \\
&=
\int d^3 p
e^{i \Ba \cdot \hat{\Bp}} \braket{\Bx}{\Bp} \tilde{\psi}(\Bp) \\
&=
\int \frac{d^3 p}{(2 \pi)^3}
e^{i \Ba \cdot \hat{\Bp}} e^{i \Bx \cdot \Bp} \tilde{\psi}(\Bp) \\
&= \psi(\Bx + \Ba).
%e^{i \Ba \cdot \hat{\BP} } \psi(\Bx) = \psi(\Bx + \Ba),
\end{aligned}
\end{equation}
Implicitly, this shows that the action of the translation operator on just a bra is
\begin{equation}\label{eqn:qftLecture8:1560}
\bra{\Bx} \hatU(\Ba) = \bra{\Bx + \Ba},
\end{equation}
or
\begin{equation}\label{eqn:qftLecture8:1580}
\begin{aligned}
\hatU(-\Ba) \ket{\Bx}
&=
\hatU^\dagger(\Ba) \ket{\Bx} \\
&=
\ket{\Bx + \Ba}.
\end{aligned}
\end{equation}
This is a different sign convention for the translation operator than is found in some other texts\footnote{In particular \citep{desai2009quantum} uses \( D(\Ba) = e^{-i \Ba \cdot \hat{\BP}/\Hbar} \) defined by the property \( D(\Ba) \ket{\Bx} = \ket{\Bx + \Ba} \).}.

In one dimension, we can compute
\begin{equation}\label{eqn:qftLecture8:560}
\begin{aligned}
\hatU(a) \hat{X} \hatU^\dagger(a)
&=
e^{i a \hat{P} } \hatX
e^{- i a \hat{P} } \\
&= \hat{X} + a \hat{1},
\end{aligned}
\end{equation}
which is a consequence of the Baker-Campbell-Hausdorff theorem.
\index{Baker-Campbell-Hausdorff theorem}
\index{BCH theorem}
\maketheorem{Baker-Campbell-Hausdorff.}{thm:qftLecture8:580}{
\begin{equation}\label{eqn:qftLecture8:600}
e^{B} A e^{-B} = \sum_{n = 0}^\infty \inv{n!} \antisymmetric{B \cdots}{\antisymmetric{B}{A}},
\end{equation}
where the n-th commutator is denoted above

\begin{itemize}
\item \( n = 0 \) : \( A \)
\item \( n = 1 \) : \( \antisymmetric{B}{A} \)
\item \( n = 2 \) : \( \antisymmetric{B}{\antisymmetric{B}{A}} \)
\item \( n = 3 \) : \( \antisymmetric{B}{\antisymmetric{B}{\antisymmetric{B}{A}}} \)
\end{itemize}
} % theorem

\begin{proof}
\begin{equation}\label{eqn:qftLecture8:620}
\begin{aligned}
f(t)
&= e^{tB} A e^{-tB} \\
&= f(0) + t f'(0) + \frac{t^2}{2} f''(0) + \cdots \frac{t^n}{n!} f^{(n)}(0)
\end{aligned}
\end{equation}
\begin{equation}\label{eqn:qftLecture8:640}
f(0) = A
\end{equation}
\begin{equation}\label{eqn:qftLecture8:660}
\begin{aligned}
f'(t)
&=
e^{tB} B A e^{-tB}
+
e^{tB} A (-B) e^{-tB} \\
&=
e^{tB} \antisymmetric{B}{A} e^{-tB}
\end{aligned}
\end{equation}
\begin{equation}\label{eqn:qftLecture8:680}
\begin{aligned}
f''(t)
&=
e^{tB} B \antisymmetric{B}{A} e^{-tB}
+
e^{tB} \antisymmetric{B}{A} (-B) e^{-tB} \\
&=
e^{tB} \antisymmetric{B}{\antisymmetric{B}{A}} e^{-tB}.
\end{aligned}
\end{equation}
From
\begin{equation}\label{eqn:qftLecture8:700}
f(1) = f(0) + f'(0) + \inv{2} f''(0) + \cdots \inv{n!} f^{(n)}(0)
\end{equation}
we have
\begin{equation}\label{eqn:qftLecture8:720}
e^{B} A e^{-B} = A +
\antisymmetric{B}{A} + \inv{2} \antisymmetric{B}{\antisymmetric{B}{A}} + \cdots
\end{equation}
\end{proof}

Example (as claimed above) :
\begin{equation}\label{eqn:qftLecture8:740}
\begin{aligned}
%e^{a \partial_x} x e^{-a \partial_x } = x + a \antisymmetric{\partial_x}{x} + \cdots = x + a.
e^{i a \hatP} \hatX e^{-i a \hatP}
&= \hatX + \antisymmetric{i a \hatP}{\hatX} + \cdots \\
&= \hatX + i a (-i) \hat{1} \\
&= \hatX + a \hat{1}.
\end{aligned}
\end{equation}
\paragraph{Application:}
\begin{equation}\label{eqn:qftLecture8:760}
e^{i \text{Hermitian} } = \text{unitary}
\end{equation}
\begin{equation}\label{eqn:qftLecture8:860}
e^{i \text{Hermitian} } \times
e^{-i \text{Hermitian} }
= 1
\end{equation}
So
\begin{equation}\label{eqn:qftLecture8:780}
\hatU(\Ba) =
e^{i a^j \hat{p}^j }
\end{equation}
is a unitary operator representing finite translations in a Hilbert space.

In particular, we can apply the BCH theorem to a field operator
\begin{equation}\label{eqn:qftLecture8:800}
\begin{aligned}
\hatU(\Ba) \phihat(\Bx) \hatU^\dagger(\Ba)
&=
e^{i a^j \hat{P}^j }
\phihat(\Bx)
e^{-i a^k \hat{P}^k } \\
&=
\phihat(\Bx)
+ i a^j \antisymmetric{\hatP^j}{\phihat(\Bx)} + \frac{-a^{j_1} a^{j_2}}{2} \antisymmetric{\hatP^{j_1}}{\antisymmetric{\hatP^{j_2}}{\phihat(\Bx)}},
\end{aligned}
\end{equation}
where the first order commutator is
\begin{equation}\label{eqn:qftLecture8:820}
\begin{aligned}
\antisymmetric{\hatP^j}{\phihat(\Bx)}
&=
\int d^3 y \antisymmetric{\pihat(\By) \partial^j \phihat(\By)}{\phihat(\Bx)} \\
&=
\int d^3 y \antisymmetric{\pihat(\By)}{\phihat(\Bx)} \partial^j \phihat(\By) \\
&=
\int d^3 y (-i ) \deltathree(\By - \Bx) \partial^j \phihat(\By) \\
&=
-i \partial^j \phihat(\Bx),
\end{aligned}
\end{equation}
and any higher order commutator is zero
\begin{equation}\label{eqn:qftLecture8:1600}
\begin{aligned}
\antisymmetric{P^k}{
\antisymmetric{P^j}{\phi(x)}}
&=
\int d^3 y
\antisymmetric{ \pi(y) \partial^k \phi(y) }{-i \partial^j \phi(x)} \\
&= 0.
\end{aligned}
\end{equation}

This gives
\begin{equation}\label{eqn:qftLecture8:840}
\begin{aligned}
\hatU(\Ba) \phihat(\Bx) \hatU^\dagger(\Ba)
&= \phihat(\Bx) + i a^j (-i) \partial^j \phihat(\Bx) + \cdots \\
&= \phihat(\Bx) + a^j \partial^j \phihat(\Bx) + \cdots \\
&= \phihat(\Bx) + a^j \PD{x_j}{} \phihat(\Bx) + \cdots \\
&= \phihat(\Bx) - a^j \PD{x^j}{} \phihat(\Bx) + \cdots \\
&= \phihat(\Bx - \Ba).
\end{aligned}
\end{equation}
\index{translation operator}

\section{Continuous symmetries.}
\index{continuous symmetries}

For all infinitesimal transformations, continuous symmetries lead to conserved charges \( Q \).  In QFT we map these charges to Hermitian operators \( Q \rightarrow \hatQ \).  We say that these charges are ``generators of the corresponding symmetry'' through unitary operators
\begin{equation}\label{eqn:qftLecture8:880}
\hatU = e^{i \text{parameter} \hatQ}.
\end{equation}
These represent the action of the symmetry in the Hilbert space.

\paragraph{Example: spatial translation}
\index{spatial translation}
\begin{equation}\label{eqn:qftLecture8:900}
\hatU(\Ba) = e^{i \Ba \cdot \hat{\BP}}
\end{equation}
\paragraph{Example: time translation}
\index{time translation}
\begin{equation}\label{eqn:qftLecture8:920}
\hatU(t) = e^{i t \hat{H}}.
\end{equation}

\section{Classical scalar theory.}
\index{classical scalar theory}

For \( d > 2 \) let's look at
\begin{equation}\label{eqn:qftLecture8:940}
S =
\int d^d x \lr{
\inv{2} \partial^\mu \phi \partial_\mu \phi - \frac{m^2}{2} \phi^2 - \lambda \phi^{d-2}
}
\end{equation}
\paragraph{Take \( m^2, \lambda \rightarrow 0 \), the free massless scalar field.}

We have a shift symmetry in this case since \( \phi(x) \rightarrow \phi(x) + \text{constant} \).
The current is just
\begin{equation}\label{eqn:qftLecture8:960}
\begin{aligned}
j^\mu
&= \PD{(\partial_\mu \phi)}{\phi} \delta \phi - \cancel{J^\mu} \\
&= \text{constant} \times \partial^\mu \phi \\
&= \partial^\mu \phi,
\end{aligned}
\end{equation}
where the constant factor has been set to one.
This current is clearly conserved since \( \partial_\mu J^\mu = \partial_\mu \partial^\mu \phi = 0\) (the equation of motion).
These are called ``Goldstone bosons'', or ``Nambu-Goldstone bosons''.
\index{Goldstone boson}
\index{Nambu-Goldstone bosons}

\paragraph{With \( m = \lambda = 0, d = 4 \) we have}

NOTE: We did this in class differently with \( d \ne 4, m, \lambda \ne 0\), and then switched to \( m = \lambda = 0, d = 4\), which was confusing.  I've reworked my notes to \( d = 4 \) like the supplemental handout that did the same.
%%%\begin{equation}\label{eqn:qftLecture8:980}
%%%S =
%%%\int d^d x \lr{
%%%\inv{2} \partial
%%%\inv{2} \partial^\mu \phi \partial_\nu \phi - \tilde{\lambda} \phi^{d-2}
%%%}
%%%\end{equation}
%%%Here we have a scale or dilatation invariance
%%%\begin{equation}\label{eqn:qftLecture8:1000}
%%%x \rightarrow e^{\lambda} x
%%%\end{equation}
%%%\begin{equation}\label{eqn:qftLecture8:1020}
%%%\phi(x) \rightarrow \phi'(x') = e^{-(d-2) \lambda/2} \phi
%%%\end{equation}
%%%
%%%\begin{equation}\label{eqn:qftLecture8:1040}
%%%d^3 x = e^{d\lambda} d^3 x
%%%\end{equation}
%%%
%%%\begin{equation}\label{eqn:qftLecture8:1060}
%%%(\partial_\mu \phi)^2 \rightarrow e^{-(d-2)\lambda} e^{-2 \lambda} \lr{ \partial_\mu \phi(x) }^2
%%%\end{equation}
%%%
%%%\begin{equation}\label{eqn:qftLecture8:1080}
%%%\phi^{2 d/(d -2)} \rightarrow \lr{\phi'(x')}^{2d/(d-2)}
%%%=
%%%e^{-(d-2)/2 \times 2d/(d-2) \lambda} \lr{ \phi }^{2d/(d-2)}
%%%\end{equation}
%%%
\begin{equation}\label{eqn:qftLecture8:980}
S =
\int d^4 x \lr{
\inv{2} \partial^\mu \phi \partial_\mu \phi
}
\end{equation}
Here we have a scale or dilatation invariance
\begin{equation}\label{eqn:qftLecture8:1000}
x \rightarrow x' = e^{\lambda} x,
\end{equation}
\begin{equation}\label{eqn:qftLecture8:1020}
\phi(x) \rightarrow \phi'(x') = e^{-\lambda} \phi,
\end{equation}
\begin{equation}\label{eqn:qftLecture8:1040}
d^4 x \rightarrow d^4 x' = e^{4\lambda} d^4 x.
\end{equation}

The partials transform as
\begin{equation}\label{eqn:qftLecture8:1400}
\begin{aligned}
\partial^\mu \rightarrow
\PD{x'_\mu}{}
&=
\PD{x'_\mu}{x_\mu}
\PD{x_\mu}{} \\
&=
e^{-\lambda}
\PD{x_\mu}{},
\end{aligned}
\end{equation}
% x'_\mu = e^\lambda x_\mu
% x_\mu = e^-\lambda x'_\mu
so the partial of the field transforms as
\begin{equation}\label{eqn:qftLecture8:1420}
\partial^\mu \phi(x) \rightarrow \PD{x'_\mu}{\phi'(x')} = e^{-2\lambda} \partial^\mu \phi(x),
\end{equation}
and finally
\begin{equation}\label{eqn:qftLecture8:1060}
(\partial_\mu \phi)^2 \rightarrow e^{-4\lambda} \lr{ \partial_\mu \phi(x) }^2.
\end{equation}

With a \( -4 \lambda \) power in the transformed quadratic term, and \( 4 \lambda \) in the volume element, we see that the action is invariant.
%%%\begin{equation}\label{eqn:qftLecture8:1080}
%%%\phi^{4} \rightarrow \lr{\phi'(x')}^{4}
%%%=
%%%e^{-3 \times 4 \lambda} \lr{ \phi }^{4}
%%%\end{equation}
To find Noether current, we need to vary the field and it's derivatives
\begin{equation}\label{eqn:qftLecture8:1100}
\begin{aligned}
\delta_\lambda \phi
&= \phi'(x) - \phi(x) \\
&= \phi'(e^{-\lambda} x') - \phi(x) \\
&\approx \phi'(x' -\lambda x') - \phi(x) \\
&\approx \phi'(x') - \lambda {x'}^\alpha \partial_\alpha \phi'(x') - \phi(x)
&\approx (1 - \lambda) \phi(x) - \lambda {x'}^\alpha \partial_\alpha \phi'(x') - \phi(x) \\
&= - \lambda(1 + x^\alpha \partial_\alpha ) \phi,
\end{aligned}
\end{equation}
where the last step assumes that \( x' \rightarrow x, \phi' \rightarrow \phi \), effectively weeding out any terms that are quadratic or higher in \( \lambda \).
%%%%In free field with \( \tilde{\lambda} = 0 \)
%%%\begin{equation}\label{eqn:qftLecture8:1120}
%%%\delta_\lambda \lr{ \inv{2} \partial_\mu \phi \partial^\mu \phi}
%%%= \partial^\mu \delta_\lambda (\partial_\mu \phi)
%%%= \partial^\mu \phi \partial_\mu ( - \lambda( 1 + x^\nu \partial_\nu )\phi )
%%%= -\lambda \lr{ \partial^\mu \phi \partial_\mu \phi + \partial^\mu \phi \partial_\mu \lr{ x^\nu \partial_\nu \phi } }
%%%= -\lambda \lr{ 2 \partial^\mu \phi \partial_\mu \phi + \partial^\mu \phi x^\nu \partial_\nu \phi }
%%%=
%%%\end{equation}
%%%
%%%Wrong.  Starting over.
%%%
Now we need the variation of the derivatives of \( \phi \)
\begin{equation}\label{eqn:qftLecture8:1440}
\delta \partial_\mu \phi(x) = \partial_\mu' \phi'(x) - \partial_\mu \phi(x),
\end{equation}
By \cref{eqn:qftLecture8:1420}
\begin{equation}\label{eqn:qftLecture8:1460}
\begin{aligned}
\partial_\mu' \phi'(x')
&=
e^{-2\lambda} \partial_\mu \phi(x) \\
&=
e^{-2\lambda} \partial_\mu \phi(e^{-\lambda} x') \\
&\approx
e^{-2\lambda} \partial_\mu
\lr{
   \phi(x') - \lambda {x'}^\alpha \partial_\alpha \phi(x')
} \\
&\approx
\lr{
   1 - 2 \lambda
}
\partial_\mu
\lr{
   \phi(x') - \lambda {x'}^\alpha \partial_\alpha \phi(x')
},
\end{aligned}
\end{equation}
so
\begin{equation}\label{eqn:qftLecture8:1480}
\begin{aligned}
\delta \partial_\mu \phi
&=
- \lambda {x}^\alpha \partial_\alpha \partial_\mu \phi(x)
- 2 \lambda \partial_\mu \phi(x) + O(\lambda^2) \\
&=
- \lambda \lr{
   {x}^\alpha \partial_\alpha + 2
}
\partial_\mu \phi(x).
\end{aligned}
\end{equation}
%%%\begin{equation}\label{eqn:qftLecture8:1140}
%%%\inv{2} \partial^\mu \phi \partial_\mu \phi
%%%\rightarrow
%%%\inv{2} {\partial'}^\mu \phi' {\partial'}_\mu \phi'
%%%=
%%%\inv{2} e^{-4\lambda}
%%%\partial^\mu \phi \partial_\mu \phi
%%%\end{equation}
%%%Let
%%%\begin{equation}\label{eqn:qftLecture8:1160}
%%%\begin{aligned}
%%%\Phi'(x')
%%%&=
%%%{\partial'}^\mu \phi' {\partial'}_\mu \phi' \\
%%%\Phi(x) &=
%%%\partial^\mu \phi \partial_\mu \phi,
%%%\end{aligned}
%%%\end{equation}
%%%
%%%so
%%%\begin{equation}\label{eqn:qftLecture8:1180}
%%%\begin{aligned}
%%%\Phi'(x') &= e^{-4 \lambda} \Phi(x) \\
%%%\Phi'(e^{\lambda} x) &= e^{-4 \lambda} \Phi(x)
%%%\end{aligned}
%%%\end{equation}
%%%...
%%%so
%%%
\begin{equation}\label{eqn:qftLecture8:1200}
\begin{aligned}
\delta \LL
&=
(\partial^\mu \phi) \delta (\partial_\mu \phi) \\
&= - \lambda \lr{ 2
\partial_\mu \phi
+ x^\alpha \partial_\alpha
\partial_\mu \phi
}
\partial^\mu \phi,
\end{aligned}
\end{equation}
or
\begin{equation}\label{eqn:qftLecture8:1500}
\begin{aligned}
\frac{\delta \LL }{-\lambda}
&=
4 \LL + x^\alpha \lr{ \partial_\alpha \partial_\mu \phi } \partial^\mu \phi \\
&=
4 \LL + x^\alpha \partial_\alpha \lr{ \LL } \\
&=
\cancel{4 \LL} + \partial_\alpha \lr{ x^\alpha \LL } - \cancel{\LL \partial_\alpha x^\alpha}.
\end{aligned}
\end{equation}
The variation in the Lagrangian density is thus
\begin{equation}\label{eqn:qftLecture8:1520}
\delta \LL = \partial_\mu J^\mu_\lambda = \partial_\mu \lr{ -\lambda x^\mu \LL },
\end{equation}
and the current is
\begin{equation}\label{eqn:qftLecture8:1540}
J^\mu_\lambda = -\lambda x^\mu \LL.
\end{equation}

The Noether current is
\begin{equation}\label{eqn:qftLecture8:1240}
\begin{aligned}
j^\mu
&= \PD{(\partial_\mu \phi)}{\LL} \delta \phi - J^\mu \\
&= -\partial^\mu \phi \lr{ 1 + x^\nu \partial_\nu } \phi + \inv{2} x^\mu \partial_\nu \phi \partial^\nu \phi,
\end{aligned}
\end{equation}
or after flipping signs
\begin{equation}\label{eqn:qftLecture8:1280}
\begin{aligned}
j^\mu_{\text{dil}}
&= \partial^\mu \phi \lr{ 1 + x^\nu \partial_\nu } \phi - \inv{2} x^\mu
\partial_\nu \phi \partial^\nu \phi \\
&= x_\nu \lr{ \partial^\mu \phi \partial^\nu \phi - \inv{2} g^{\mu\nu} \partial_\lambda \phi \partial^\lambda \phi }
+ \inv{2} \partial^\mu (\phi^2),
\end{aligned}
\end{equation}
%
\begin{equation}\label{eqn:qftLecture8:1300}
j^\mu_{\text{dil}} = -x_\nu T^{\nu \mu} + \inv{2} \partial^\mu (\phi^2).
\end{equation}
\begin{equation}\label{eqn:qftLecture8:1620}
T^{\mu\nu} =
\partial^\mu \phi \partial^\nu \phi - g^{\mu\nu} \LL
\end{equation}

The current and \( T^{\mu\nu} \) can both be redefined \( j^{\mu'} = j^\mu + \partial_\nu C^{\nu\mu} \) adding an antisymmetric \( C^{\mu\nu} = -C^{\nu\mu} \)
\begin{equation}\label{eqn:qftLecture8:1320}
j^\mu_{\text{dil conformal}} = - x_\nu T^{\nu\mu}_{\text{conformal}}
\end{equation}
\begin{equation}\label{eqn:qftLecture8:1340}
\partial_\mu
j^\mu_{\text{dil conformal}} = - {{T_{\text{conformal}}}^\mu}_\mu.
\end{equation}

Consequence: \( 0 = T^{00} - T^{11} - T^{22} - T^{33} \), which is essentially
\begin{equation}\label{eqn:qftLecture8:1360}
0 = \rho - 3 p = 0.
\end{equation}
%}
%\EndNoBibArticle
