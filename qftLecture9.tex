%
% Copyright � 2017 Peeter Joot.  All Rights Reserved.
% Licenced as described in the file LICENSE under the root directory of this GIT repository.
%
\input{../latex/blogpost.tex}
\renewcommand{\basename}{qftLecture9}
\renewcommand{\dirname}{notes/phy2403/}
\newcommand{\keywords}{PHY2403H}
\input{../latex/peeter_prologue_print2.tex}

%\usepackage{phy2403}
\usepackage{peeters_braket}
%\usepackage{peeters_layout_exercise}
\usepackage{peeters_figures}
\usepackage{mathtools}
\usepackage{siunitx}
\usepackage{enumerate}
\usepackage{macros_cal} % LL

\beginArtNoToc
\generatetitle{PHY2403H Quantum Field Theory.  Lecture 9: XXX.  Taught by Prof.\ Erich Poppitz}
%\chapter{XXX}
\label{chap:qftLecture9}

\paragraph{DISCLAIMER: Very rough notes from class, with some additional side notes.}

These are notes for the UofT course PHY2403H, Quantum Field Theory I, taught by Prof. Erich Poppitz fall 2018.
%, covering \textchapref{{1}} \citep{peskin1995introduction} content.

\section{Last time}

We followed a sequence of operations
\begin{enumerate}
\item
Noether's theorem
\item \( \rightarrow \)
conserved currents
\item \( \rightarrow \)
charges (classical)
\item \( \rightarrow \)
``correspondence principle''
\item \( \rightarrow \hatQ \)
\end{enumerate}

\begin{itemize}
\item Hermitian operators
\item ``generators of symmetry"
\begin{dmath}\label{eqn:qftLecture9:20}
\hatU(\alpha) = e^{i \alpha \hatQ}
\end{dmath}
We found
\begin{dmath}\label{eqn:qftLecture9:40}
\hatU(\alpha) \phihat \hatU^\dagger(\alpha) = \phihat + i \alpha \antisymmetric{\hatQ}{\phihat} + \cdots
\end{dmath}
\end{itemize}

\paragraph{Example: internal symmetries:}
(non-spacetime), such as \( O(N)\) or \( U(1) \).

In QFT internal symmetries can have different ``\underline{modes of realization}''.

\begin{enumerate}[I]
\item ``Wigner mode''.  These are also called ``unbroken symmetries''.
\begin{dmath}\label{eqn:qftLecture9:60}
\hatQ \ket{0} = 0
\end{dmath}
i.e. \( \hatU(\alpha) \ket{0} = 0 \).
Ground state invariant.  Formally \( :\hatQ: \) annihilates \( \ket{0} \).
\( \antisymmetric{\hatQ}{\hatH} = 0 \) implies that all eigenstates are eigenstates of \( \hatQ \) in \( U(1) \).  Example from HW 1
\begin{dmath}\label{eqn:qftLecture9:80}
\hatQ = \text{``charge'' under \( U(1) \)}.
\end{dmath}
All states have definite charge, just live in QU.
\item ``Nambu-Goldstone mode'' (Landau-ginsburg).  This is also called a ``spontaneously broken symmetry''\footnote{
First encounter example (HWII, \( SU(2) \times SU(2) \rightarrow SU(2) \)).  Here a \( U(1) \) spontaneous broken symmetry.}.
\( H \) or \( L \) is invariant under symmetry, but ground state is not.
\end{enumerate}

Example:
\begin{dmath}\label{eqn:qftLecture9:100}
\LL = \partial_\mu \phi^\conj \partial^\mu \phi - V(\Abs{\phi}),
\end{dmath}
where
\begin{dmath}\label{eqn:qftLecture9:120}
V(\Abs{\phi}) = m^2 \phi^\conj \phi + \frac{\lambda}{4} \lr{ \phi^\conj \phi }^2.
\end{dmath}
When \( m^2 > 0 \) we have a Wigner mode, but when \( m^2 < 0 \) we have an issue: \( \phi = 0 \) is not a minimum of potential.
When \( m^2 < 0 \) we write
\begin{dmath}\label{eqn:qftLecture9:140}
V(\phi)
= - m^2 \phi^\conj \phi + \frac{\lambda}{4} \lr{ \phi^\conj \phi}^2
=
\frac{\lambda}{4} \lr{ \phi^\conj \phi - v^2 }^2 + \text{const}.
\end{dmath}
The potential is illustrated in
F1.
Try plotting this ``Mexican Hat'' potential in Mathematica, using a Manipulate varying \( m \).
%Here we mean \( \phi^\conj \sim \phi_1 +
We choose to expand around some point on the minimum, and it doesn't matter which one (P in the figure).
When there is no potential, we call the field massless (i.e. if we are in the minimum ring).
We expand as
\begin{dmath}\label{eqn:qftLecture9:160}
\phi(x) = v \lr{ 1 + \frac{\rho(x)}{v} } e^{i \alpha(x)/v },
\end{dmath}
so
\begin{dmath}\label{eqn:qftLecture9:180}
\frac{\lambda}{4}
\lr{\phi^\conj \phi - v^2}^2 =
\lr{
v^2 \lr{ 1 + \frac{\rho(x)}{v} }^2
- v^2
}^2
=
\frac{\lambda}{4}
v^4 \lr{ \lr{ 1 + \frac{\rho(x)}{v} }^2 - 1 }
=
\frac{\lambda}{4}
v^4
\lr{
   \frac{2 \rho}{v} + \frac{\rho^2}{v^2}
}^2.
\end{dmath}

\begin{dmath}\label{eqn:qftLecture9:200}
\partial_\mu \phi =
\lr{
v \lr{ 1 + \frac{\rho(x)}{v} } \frac{i}{v} \partial_\mu \alpha
+ \partial_\mu \rho
} e^{i \alpha}
\end{dmath}

so
\begin{dmath}\label{eqn:qftLecture9:220}
\LL
= \Abs{\partial \phi^\conj}^2 - \frac{\lambda}{4} \lr{ \Abs{\phi^\conj}^2 - v^2 }^2
=
\partial_\mu \rho \partial^\mu \rho + \partial_\mu \alpha \partial^\mu \alpha \lr{ 1 + \frac{\rho}{v} }
-
\frac{\lambda v^4}{4} \frac{ 4\rho^2}{v^2} + O(\rho^3)
=
\partial_\mu \rho \partial^\mu \rho
- \lambda v^2\rho^2
+
\partial_\mu \alpha \partial^\mu \alpha \lr{ 1 + \frac{\rho}{v} }.
\end{dmath}
We have two fields, \( \rho \) : a massive scalar field, the ``Higgs'', and a massless field \( \alpha \) (the Goldstone boson).

\( U(1) \) symmetry acts on \( \phi(x) \rightarrow e^{i \omega } \phi(x) \) i.t.o \( \alpha(x) \rightarrow \alpha(x) + v \omega \).
\( U(1) \) global symmetry (broken) acts on the Goldstone field \( \alpha(x) \) by a constant shift.  (\(U(1)\) is still a symmetry of the Lagrangian.)

The current of the \( U(1) \) symmetry is:
\begin{dmath}\label{eqn:qftLecture9:240}
j_\mu = \partial_\mu \alpha \lr{ 1 + \text{higher dimensional \( \rho \) terms} }.
\end{dmath}

When we quantize
\begin{dmath}\label{eqn:qftLecture9:260}
\alpha(x) =
\int \frac{d^3p}{(2\pi)^3 \sqrt{ 2 \omega_p }} e^{i \omega_p t - i \Bp \cdot \Bx} \hata_\Bp^\dagger +
\int \frac{d^3p}{(2\pi)^3 \sqrt{ 2 \omega_p }} e^{-i \omega_p t + i \Bp \cdot \Bx} \hata_\Bp
\end{dmath}
\begin{dmath}\label{eqn:qftLecture9:280}
j^\mu(x) = \partial^\mu \alpha(x) =
\int \frac{d^3p}{(2\pi)^3 \sqrt{ 2 \omega_p }} \lr{ i \omega_\Bp - i \Bp } e^{i \omega_p t - i \Bp \cdot \Bx} \hata_\Bp^\dagger +
\int \frac{d^3p}{(2\pi)^3 \sqrt{ 2 \omega_p }} \lr{ -i \omega_\Bp + i \Bp } e^{-i \omega_p t + i \Bp \cdot \Bx} \hata_\Bp.
\end{dmath}

\begin{dmath}\label{eqn:qftLecture9:300}
j^\mu(x) \ket{0} \ne 0,
\end{dmath}
instead it creates a single particle state.

\section{Examples of symmetries}
In particle physics, examples of Wigner vs Nambu-Goldstone, ignoring gravity the only exact intenral symmetry in the standard module is
\( (B\# - L\#) \), believed to be a \( U(1) \) symmetry in Wigner mode.

Here \(B\#\) is the Baryon number, and \( L\# \) is the Lepton number.  Examples:

\begin{itemize}
\item \( B(p) = 1 \), proton.
\item \( B(q) = 1/3 \), quark
\item \( B(e) = 1 \), electron
\item \( B(n) = 1 \), neutron.
\item \( L(p) = 1 \), proton.
\item \( L(q) = 0 \), quark.
\item \( L(e) = 0 \), electron.
\end{itemize}

The major use of global internal symmetries in the standard model is as ``approximate'' ones.  They become symmetries when one neglects some effect( ``terms in \( \LL \)'').
There are other approximate symmetries (use of group theory to find the Balmer series).
\paragraph{Example from HW2:}
QCD in limit
\begin{equation}\label{eqn:qftLecture9:320}
m_u = m_d = 0.
\end{equation}
\( m_u m_d \ll m_p \) (the products of the up-quark mass and the down-quark mass are much less than a composite one (name?)).
\( SU(2)_L \times SU(2)_R \rightarrow SU(2)_V \)
\paragraph{EWSB (Electro-Weak-Symmetry-Breaking) sector}
When the couplings \( g_2, g_1 = 0 \).  (\( g_2 \in SU(2), g_1 \in U(1) \)).

\section{Scale invariance}

\begin{dmath}\label{eqn:qftLecture9:340}
\begin{aligned}
x &\rightarrow e^{\lambda} x \\
\phi &\rightarrow e^{-\lambda} \phi \\
A_\mu &\rightarrow e^{-\lambda} A_\mu
\end{aligned}
\end{dmath}
Any unitary theory which is scale invariant is also \underline{conformal} invariant.  Conformal invarance means that angles are preserved.
The point here is that there is more than scale invariance.

We have classical internal global continuous symmetries.
These can be either
\begin{enumerate}
\item
``unbroken'' (Wigner mode)
\begin{dmath}\label{eqn:qftLecture9:360}
\hatQ\ket{0} = 0.
\end{dmath}
\item
``spontaneously broken''
\begin{dmath}\label{eqn:qftLecture9:380}
j^\mu(x) \ket{0} \ne 0
\end{dmath}
(creates Goldstone modes).
\item ``anomalous''.  Classical symmetries are not a symmetry of QFT.
Examples:
\begin{itemize}
\item Scale symmetry (to be studied in QFT II), although this is not truly internal.
\item In QCD again when \( \omega_\Bq = 0 \), a \( U(1\) symmetry (chiral symmetry) becomes exact, and cannot be preserved in QFT.
\item In the standard model (E.W sector), the Baryon number and Lepton numbers are not symmetries, but their difference \( B\# - L\# \) is a symmetry.
\end{itemize}
\end{enumerate}

\section{Lorentz invariance.}
We'd like to study the action of Lorentz symmetries on quantum states.  We are going to ``go by the book'', finding symmetries, currents, quantize, find generators, and so forth.

Under a Lorentz transformation
\begin{dmath}\label{eqn:qftLecture9:400}
x^\mu \rightarrow {x'}^\mu = {\Lambda^\mu}_\nu x^\nu,
\end{dmath}
We are going to consider infitesimal Lorentz transformations
\begin{dmath}\label{eqn:qftLecture9:420}
{\Lambda^\mu}_\nu \approx
{\delta^\mu}_\nu
+
{\omega^\mu}_\nu,
\end{dmath}
where \( {\omega^\mu}_\nu \) is small.

{\Lambda^\mu}_\nu
g^{\nu\lambda}
{\Lambda^\kappa}_\lambda
=
g^{\nu\kappa}

\lr{
   {\delta^\mu}_\nu
   +
   {\omega^\mu}_\nu
}
g^{\nu\lambda}
\lr{
   {\delta^\kappa}_\lambda
   +
   {\omega^\kappa}_\lambda
}

(paper)

so the infinitesimal transformation of the coordinates is
\begin{dmath}\label{eqn:qftLecture9:440}
x^\mu \rightarrow {x'}^\mu \approx x^\mu + \omega^{\mu\nu} x_\nu,
\end{dmath}
the field transforms as
\begin{dmath}\label{eqn:qftLecture9:460}
\phi(x) \rightarrow \phi'(x') = \phi(x)
\end{dmath}
or
\phi'(x^\mu + \omega^{\mu\nu} x_\nu) =
\phi'(x) + \omega^{\mu\nu} x_\nu \partial_\mu\phi(x) = \phi(x),
so
\begin{dmath}\label{eqn:qftLecture9:480}
\delta \phi = \phi'(x) - \phi(x) =
-\omega^{\mu\nu} x_\nu \partial_\nu \phi.
\end{dmath}

Since \( \LL \) is a scalar
\begin{dmath}\label{eqn:qftLecture9:500}
\delta \LL
=
-\omega^{\mu\nu} x_\nu \partial_\nu \LL
=
-
\partial_\nu \lr{
   \omega^{\mu\nu} x_\mu \LL
}
+
(\partial_\nu x_\mu) \omega^{\mu\nu} \LL
=
\partial_\nu \lr{
-
   \omega^{\mu\nu} x_\mu \LL
},
\end{dmath}
since \( \partial_\nu x_\mu = g_{\nu\mu} \) is symmetric, and \( \omega \) is antisymmetric.
Our current is
\begin{dmath}\label{eqn:qftLecture9:520}
J^\mu_\omega =
-
   \omega^{\mu\nu} x_\mu \LL.
\end{dmath}
Our Noether current is
\begin{dmath}\label{eqn:qftLecture9:540}
j^\nu_{\omega_{\mu\rho}}
= \PD{\phi_{,\nu}}{\delta \phi} - J^\mu_\omega
=
\partial^\nu \phi\lr{ - \omega^{\mu\rho} x_\rho \partial_\mu \phi } + \omega^{\nu \rho} x_\rho \LL
=
\omega^{\mu\rho}
\lr{
   \partial^\nu \phi\lr{ - x_\rho \partial_\mu \phi } + {\delta^{\nu}}_\mu x_\rho \LL
}
=
\omega^{\mu\rho} x_\rho
\lr{
   -\partial^\nu \phi \partial_\mu \phi + {\delta^{\nu}}_\mu \LL
}
\end{dmath}
We identify
\begin{dmath}\label{eqn:qftLecture9:560}
-
{T^\nu}_\mu =
   -\partial^\nu \phi \partial_\mu \phi + {\delta^{\nu}}_\mu \LL,
\end{dmath}
so the current is
\begin{dmath}\label{eqn:qftLecture9:580}
j^\nu_{\omega_{\mu\rho}}
=
-\omega^{\mu\rho} x_\rho
{T^\nu}_\mu.
\end{dmath}
Define
\begin{dmath}\label{eqn:qftLecture9:600}
j^{\nu\mu\rho} = \inv{2} \lr{ x^\rho T^{\nu\mu} - x^{\mu} T^{\mu\nu} },
\end{dmath}
which retains the antisymmetry in \( \mu \rho \) yet still drops the parameter \( \omega^{\mu\rho} \).

Example.  Rotations \( \mu\rho = ij \)
\begin{dmath}\label{eqn:qftLecture9:620}
J^{0 i j} \epsilon_{ijk}
=
\inv{2} \lr{ x^i T^{0j} - x^{j} T^{0i} } \epsilon_{ijk}
x^i T^{0j} \epsilon_{ijk}.
\end{dmath}
Observe that this has the structure of \( (\Bx \cross \Bp)_k \), where \( \Bp \) is the momentum density of the field.
Let
\begin{dmath}\label{eqn:qftLecture9:640}
L_k \equiv Q_k =
\int d^3 x J^{0ij} \epsilon_{ijk}
\end{dmath}
We can now build a generator
\begin{dmath}\label{eqn:qftLecture9:660}
\hatU(\Balpha)
= e^{i \Balpha \cdot \hat{\BL}}
= \exp\lr{i \alpha_k
\int d^3 x x^i \hat{T}^{0j}
}
=
\exp\lr{i \alpha_k
\int d^3 x x^i \hatpi \partial^j \hatphi
}
=
\exp\lr{i \Balpha
\int d^3 x \hatpi \spacegrad \phi \cross \Bx
}
\end{dmath}
(up to a sign in the exponent which doesn't matter)
\begin{dmath}\label{eqn:qftLecture9:680}
\phihat(\By) \rightarrow \hatU(\alpha) \hatphi(\By) \hatU^\dagger(\alpha)
\approx
\phihat(\By) +
i \Balpha \cdot \antisymmmetric
{
   \int d^3 x \pihat(\Bx) \spacegrad \phihat(\Bx) \cross \Bx
}
{
   \phihat(\By)
}
=
\phihat(\By) +
\Balpha \cdot \lr{ \spacegrad \phihat(\By ) \cross \By)
\end{dmath}
Explicitly, in coordinates, this is
\begin{dmath}\label{eqn:qftLecture9:700}
\phihat(\By) \rightarrow
\phihat(\By) +
\alpha^i
\lr{
   \partial^j \phihat(\By) y^k \epsilon_{jki}
}
=
\phihat(\By) -
\epsilon_{ikj} \alpha^i y^k \partial^j \phihat
=
\phihat( y^j - \epsilon^{jij} a^i y^k ).
\end{dmath}
This is a rotation.  Example: pick \( \Balpha = (0, 0, 1) \)
\begin{dmath}\label{eqn:qftLecture9:720}
\begin{bmatrix}
y^1 \\
y^2 \\
y^3 \\
\end{bmatrix}
\rightarrow
\begin{bmatrix}
1 & \alpha & 0 \\
-\alpha & 1 & 0 \\
0 & 0 & 1
\end{bmatrix}
\begin{bmatrix}
y^1 \\
y^2 \\
y^3 \\
\end{bmatrix}
\end{dmath}

%\EndArticle
\EndNoBibArticle
