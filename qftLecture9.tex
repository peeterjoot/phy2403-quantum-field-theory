%
% Copyright � 2018 Peeter Joot.  All Rights Reserved.
% Licenced as described in the file LICENSE under the root directory of this GIT repository.
%
%%%\input{../latex/blogpost.tex}
%%%\renewcommand{\basename}{qftLecture9}
%%%\renewcommand{\dirname}{notes/phy2403/}
%%%\newcommand{\keywords}{PHY2403H}
%%%\input{../latex/peeter_prologue_print2.tex}
%%%
%%%%\usepackage{phy2403}
%%%\usepackage{peeters_braket}
%%%%\usepackage{peeters_layout_exercise}
%%%\usepackage{peeters_figures}
%%%\usepackage{mathtools}
%%%\usepackage{siunitx}
%%%\usepackage{enumerate}
%%%\usepackage{macros_cal} % LL
%%%\usepackage{mmacells}
%%%
%%%%\newcommand{\munu}[0]{\mu\nu}
%%%\newcommand{\ultensor}[3]{{{#1}^{#2}}_{#3}}
%%%%s/\\ultensor{\([^}]\+\)}{\([^}]\+\)}{\([^}]\+\)}/{{\1}^\2}_\3/g
%%%%s/ulLambda/ultensor{\\Lambda}/g
%%%\newcommand{\ulLambda}[2]{\ultensor{\Lambda}{#1}{#2}}
%%%\newcommand{\ulDelta}[2]{\ultensor{\delta}{#1}{#2}}
%%%
%%%\beginArtNoToc
%%%\generatetitle{PHY2403H Quantum Field Theory.  Lecture 9: Unbroken and spontaneously broken symmetries, Higgs Lagrangian, scale invariance, Lorentz invariance, angular momentum quantization.  Taught by Prof.\ Erich Poppitz}
%\chapter{Unbroken and spontaneously broken symmetries, Higgs Lagrangian, scale invariance, Lorentz invariance, angular momentum quantization}
\index{unbroken symmetries}
\index{spontaneous symmetry breaking}
\index{symmetries!unbroken}
\index{symmetries!spontaneously broken}
\index{Higgs Lagrangian}
\index{scale invariance}
\index{Lorentz invariance}
\index{angular momentum quantization}
\label{chap:qftLecture9}

%\paragraph{DISCLAIMER: Very rough notes from class, with some additional side notes.}
%
%These are notes for the UofT course PHY2403H, Quantum Field Theory I, taught by Prof. Erich Poppitz fall 2018.
%%, covering \textchapref{{1}} \citep{peskin1995introduction} content.
%
\section{Last time.}

We followed a sequence of operations
\begin{enumerate}
\item
Noether's theorem
\item \( \rightarrow \)
conserved currents
\item \( \rightarrow \)
charges (classical)
\item \( \rightarrow \)
``correspondence principle''
\item \( \rightarrow \hatQ \)
\end{enumerate}

\begin{itemize}
\item Hermitian operators
\item ``generators of symmetry"
\begin{dmath}\label{eqn:qftLecture9:20}
\hatU(\alpha) = e^{i \alpha \hatQ}
\end{dmath}
We found
\begin{dmath}\label{eqn:qftLecture9:40}
\hatU(\alpha) \phihat \hatU^\dagger(\alpha) = \phihat + i \alpha \antisymmetric{\hatQ}{\phihat} + \cdots
\end{dmath}
\end{itemize}

\paragraph{Example: internal symmetries:}
\index{internal symmetries}
(non-spacetime), such as \( O(N)\) or \( U(1) \).

In QFT internal symmetries can have different ``\underline{modes of realization}''.

\begin{enumerate}[I]
\item ``Wigner mode''.  These are also called ``unbroken symmetries''.
\begin{dmath}\label{eqn:qftLecture9:60}
\hatQ \ket{0} = 0
\end{dmath}
i.e. \( \hatU(\alpha) \ket{0} = 0 \).
Ground state invariant.  Formally \( :\hatQ: \) annihilates \( \ket{0} \).
\( \antisymmetric{\hatQ}{\hatH} = 0 \) implies that all eigenstates are eigenstates of \( \hatQ \) in \( U(1) \).  Example from HW 1
\begin{dmath}\label{eqn:qftLecture9:80}
\hatQ = \text{``charge'' under \( U(1) \)}.
\end{dmath}
All states have definite charge, just live in QU.
\item ``Nambu-Goldstone mode'' (Landau-Ginsburg).  This is also called a ``spontaneously broken symmetry''\footnote{
First encounter example (HWII, \( SU(2) \times SU(2) \rightarrow SU(2) \)).  Here a \( U(1) \) spontaneous broken symmetry.}.
\( H \) or \( L \) is invariant under symmetry, but ground state is not.
\end{enumerate}

Example:
\begin{dmath}\label{eqn:qftLecture9:100}
\LL = \partial_\mu \phi^\conj \partial^\mu \phi - V(\Abs{\phi}),
\end{dmath}
where
\begin{dmath}\label{eqn:qftLecture9:120}
V(\Abs{\phi}) = m^2 \phi^\conj \phi + \frac{\lambda}{4} \lr{ \phi^\conj \phi }^2.
\end{dmath}
When \( m^2 > 0 \) we have a Wigner mode, but when \( m^2 < 0 \) we have an issue: \( \phi = 0 \) is not a minimum of potential.
When \( m^2 < 0 \) we write
\begin{dmath}\label{eqn:qftLecture9:140}
V(\phi)
= - m^2 \phi^\conj \phi + \frac{\lambda}{4} \lr{ \phi^\conj \phi}^2
= \frac{\lambda}{4} \lr{
\lr{ \phi^\conj \phi}^2 - \frac{4}{\lambda} m^2 }
= \frac{\lambda}{4} \lr{
\phi^\conj \phi - \frac{2}{\lambda} m^2 }^2 - \frac{4 m^4}{\lambda^2},
\end{dmath}
or simply
\begin{dmath}\label{eqn:qftLecture9:780}
V(\phi)
=
\frac{\lambda}{4} \lr{ \phi^\conj \phi - v^2 }^2 + \text{const}.
\end{dmath}
The potential (called the Mexican hat potential) is illustrated in \cref{fig:mexicanHatPotential:mexicanHatPotentialFig1} for non-zero \( v \), and in
\cref{fig:mexicanHatPotential:mexicanHatPotentialFig2} for \( v = 0 \).
The following is a Mathematica code listing that can be used to play with this shape
\begin{mmaCell}[moredefined={potential, v},morepattern={x_, y_, \
v_},morefunctionlocal={x, y}]{Input}
  ClearAll[potential]
  potential[x_, y_, v_] := (\mmaPat{x}^2 + \mmaPat{y}^2 - v^2)^2

  Manipulate[
  Plot3D[ potential[x, y, v], \{x, -5, 5\}, \{y, -5, 5\}, PlotRange\
\(\pmb{\to}\)Full],
  \{\{v,4\}, 0, 10\}
  ]
\end{mmaCell}
\mathImageFigure{../figures/phy2403-quantum-field-theory/mexicanHatPotentialFig1}{Mexican hat potential.}{fig:mexicanHatPotential:mexicanHatPotentialFig1}{0.3}{mexicanHatPotentialManipulate.nb}
\mathImageFigure{../figures/phy2403-quantum-field-theory/mexicanHatPotentialFig2}{Degenerate Mexican hat potential \( v = 0 \).}{fig:mexicanHatPotential:mexicanHatPotentialFig2}{0.3}{mexicanHatPotentialManipulate.nb}
We choose to expand around some point on the minimum ring (it doesn't matter which one). % (P in the figure).
When there is no potential, we call the field massless (i.e. if we are in the minimum ring).
We expand as
\begin{dmath}\label{eqn:qftLecture9:160}
\phi(x) = v \lr{ 1 + \frac{\rho(x)}{v} } e^{i \alpha(x)/v },
\end{dmath}
so
\begin{dmath}\label{eqn:qftLecture9:180}
\frac{\lambda}{4}
\lr{\phi^\conj \phi - v^2}^2 =
\lr{
v^2 \lr{ 1 + \frac{\rho(x)}{v} }^2
- v^2
}^2
=
\frac{\lambda}{4}
v^4 \lr{ \lr{ 1 + \frac{\rho(x)}{v} }^2 - 1 }
=
\frac{\lambda}{4}
v^4
\lr{
   \frac{2 \rho}{v} + \frac{\rho^2}{v^2}
}^2,
\end{dmath}
and
\begin{dmath}\label{eqn:qftLecture9:200}
\partial_\mu \phi =
\lr{
v \lr{ 1 + \frac{\rho(x)}{v} } \frac{i}{v} \partial_\mu \alpha
+ \partial_\mu \rho
} e^{i \alpha}.
\end{dmath}
The Lagrangian takes the form
\begin{dmath}\label{eqn:qftLecture9:220}
\LL
= \Abs{\partial \phi^\conj}^2 - \frac{\lambda}{4} \lr{ \Abs{\phi^\conj}^2 - v^2 }^2
=
\partial_\mu \rho \partial^\mu \rho + \partial_\mu \alpha \partial^\mu \alpha \lr{ 1 + \frac{\rho}{v} }
-
\frac{\lambda v^4}{4} \frac{ 4\rho^2}{v^2} + O(\rho^3)
=
\partial_\mu \rho \partial^\mu \rho
- \lambda v^2\rho^2
+
\partial_\mu \alpha \partial^\mu \alpha \lr{ 1 + \frac{\rho}{v} }.
\end{dmath}
We have two fields, \( \rho \) : a massive scalar field, the ``Higgs'', and a massless field \( \alpha \) (the Goldstone boson).

\( U(1) \) symmetry acts on \( \phi(x) \rightarrow e^{i \omega } \phi(x) \) i.t.o \( \alpha(x) \rightarrow \alpha(x) + v \omega \).
\( U(1) \) global symmetry (broken) acts on the Goldstone field \( \alpha(x) \) by a constant shift.  (\(U(1)\) is still a symmetry of the Lagrangian.)

The current of the \( U(1) \) symmetry is:
\begin{dmath}\label{eqn:qftLecture9:240}
j_\mu = \partial_\mu \alpha \lr{ 1 + \text{higher dimensional \( \rho \) terms} }.
\end{dmath}

When we quantize
\begin{dmath}\label{eqn:qftLecture9:260}
\alpha(x) =
\int \frac{d^3p}{(2\pi)^3 \sqrt{ 2 \omega_p }} e^{i \omega_p t - i \Bp \cdot \Bx} \hata_\Bp^\dagger +
\int \frac{d^3p}{(2\pi)^3 \sqrt{ 2 \omega_p }} e^{-i \omega_p t + i \Bp \cdot \Bx} \hata_\Bp
\end{dmath}
\begin{dmath}\label{eqn:qftLecture9:280}
\begin{aligned}
j^\mu(x)
&= \partial^\mu \alpha(x) \\
&=
\int \frac{d^3p}{(2\pi)^3 \sqrt{ 2 \omega_p }} \lr{ i \omega_\Bp - i \Bp } e^{i \omega_p t - i \Bp \cdot \Bx} \hata_\Bp^\dagger  \\
&\quad +
\int \frac{d^3p}{(2\pi)^3 \sqrt{ 2 \omega_p }} \lr{ -i \omega_\Bp + i \Bp } e^{-i \omega_p t + i \Bp \cdot \Bx} \hata_\Bp.
\end{aligned}
\end{dmath}

\begin{dmath}\label{eqn:qftLecture9:300}
j^\mu(x) \ket{0} \ne 0,
\end{dmath}
instead it creates a single particle state.

\section{Examples of symmetries.}
In particle physics, examples of Wigner vs Nambu-Goldstone, ignoring gravity the only exact internal symmetry in the standard module is
\( (B\# - L\#) \), believed to be a \( U(1) \) symmetry in Wigner mode.

Here \(B\#\) is the Baryon number, and \( L\# \) is the Lepton number.  Examples:
\index{Baryon number}
\index{Lepton number}
\index{proton}
\index{quark}
\index{electron}
\index{neutron}
\begin{itemize}
\item \( B(p) = 1 \), proton.
\item \( B(q) = 1/3 \), quark
\item \( B(e) = 1 \), electron
\item \( B(n) = 1 \), neutron.
\item \( L(p) = 1 \), proton.
\item \( L(q) = 0 \), quark.
\item \( L(e) = 0 \), electron.
\end{itemize}

The major use of global internal symmetries in the standard model is as ``approximate'' ones.  They become symmetries when one neglects some effect( ``terms in \( \LL \)'').
There are other approximate symmetries (use of group theory to find the Balmer series).
\paragraph{Example from \cref{qft:problemSet2:2} (Hw2):}
QCD in limit
\begin{equation}\label{eqn:qftLecture9:320}
m_u = m_d = 0.
\end{equation}
\( m_u m_d \ll m_p \) (the products of the up-quark mass and the down-quark mass are much less than a composite one (name?)).
\( SU(2)_L \times SU(2)_R \rightarrow SU(2)_V \)
\paragraph{EWSB (Electro-Weak-Symmetry-Breaking) sector}
\index{EWSB}
\index{Electro-Weak-Symmetry-Breaking}
When the couplings \( g_2, g_1 = 0 \).  (\( g_2 \in SU(2), g_1 \in U(1) \)).
\section{Scale invariance.}
\index{scale invariance}

\begin{dmath}\label{eqn:qftLecture9:340}
\begin{aligned}
x &\rightarrow e^{\lambda} x \\
\phi &\rightarrow e^{-\lambda} \phi \\
A_\mu &\rightarrow e^{-\lambda} A_\mu
\end{aligned}.
\end{dmath}
\index{unitary}
\index{conformal invariance}
Any unitary theory which is scale invariant is also \underline{conformal} invariant.  Conformal invariance means that angles are preserved.
The point here is that there is more than scale invariance.

We have classical internal global continuous symmetries.
These can be either
\begin{enumerate}
\item
``unbroken'' (Wigner mode)
\begin{dmath}\label{eqn:qftLecture9:360}
\hatQ\ket{0} = 0.
\end{dmath}
\item
``spontaneously broken''
\begin{dmath}\label{eqn:qftLecture9:380}
j^\mu(x) \ket{0} \ne 0
\end{dmath}
(creates Goldstone modes).
\item ``anomalous''.  Classical symmetries are not a symmetry of QFT.
\index{anomalous symmetry}
Examples:
\begin{itemize}
\item Scale symmetry (to be studied in QFT II), although this is not truly internal.
\item In QCD again when \( \omega_\Bq = 0 \), a \( U(1\) symmetry (chiral symmetry) becomes exact, and cannot be preserved in QFT.
\item In the standard model (E.W sector), the Baryon number and Lepton numbers are not symmetries, but their difference \( B\# - L\# \) is a symmetry.
\end{itemize}
\end{enumerate}

\section{Lorentz invariance.}
\index{Lorentz invariance}
We'd like to study the action of Lorentz symmetries on quantum states.  We are going to ``go by the book'', finding symmetries, currents, quantize, find generators, and so forth.

Under a Lorentz transformation
\index{Lorentz transformation!infinitesimal}
\begin{equation}\label{eqn:qftLecture9:400}
x^\mu \rightarrow {x'}^\mu = {\Lambda^\mu}_\nu x^\nu,
\end{equation}
We are going to consider infinitesimal Lorentz transformations
\begin{dmath}\label{eqn:qftLecture9:420}
{\Lambda^\mu}_\nu \approx
{\delta^\mu}_\nu + {\omega^\mu}_\nu
,
\end{dmath}
where \( {\omega^\mu}_\nu \) is small.
A Lorentz transformation \( \Lambda \) must satisfy \( \Lambda^\T G \Lambda = G \), or
\begin{dmath}\label{eqn:qftLecture9:800}
g_{\mu\nu} = \ulLambda{\alpha}{\mu} g_{\alpha \beta} \ulLambda{\beta}{\nu},
\end{dmath}
into which we insert the infinitesimal transformation representation
\begin{dmath}\label{eqn:qftLecture9:820}
0 =
- g_{\mu\nu} +
\lr{ {\delta^\alpha}_\mu + {\omega^\alpha}_\mu }
g_{\alpha \beta}
\lr{ {\delta^\beta}_\nu + {\omega^\beta}_\nu }
=
- g_{\mu\nu} +
\lr{
   g_{\mu \beta}
   +
   \omega_{\beta\mu}
}
\lr{ {\delta^\beta}_\nu + {\omega^\beta}_\nu }
=
- g_{\mu\nu} +
   g_{\mu \nu}
   +
   \omega_{\nu\mu}
+
\omega_{\mu\nu}
+
   \omega_{\beta\mu}
{\omega^\beta}_\nu.
\end{dmath}
The quadratic term can be ignored, leaving just
\begin{dmath}\label{eqn:qftLecture9:840}
0 =
   \omega_{\nu\mu}
+
\omega_{\mu\nu},
\end{dmath}
or
\begin{dmath}\label{eqn:qftLecture9:860}
   \omega_{\nu\mu} = - \omega_{\mu\nu}.
\end{dmath}
Note that \( \omega \) is a completely antisymmetric tensor, and like \( F_{\mu\nu} \) this has only 6 elements.
This means that the
infinitesimal transformation of the coordinates is
\begin{equation}\label{eqn:qftLecture9:440}
x^\mu \rightarrow {x'}^\mu \approx x^\mu + \omega^{\mu\nu} x_\nu,
\end{equation}
the field transforms as
\begin{equation}\label{eqn:qftLecture9:460}
\phi(x) \rightarrow \phi'(x') = \phi(x)
\end{equation}
or
\begin{dmath}\label{eqn:qftLecture9:760}
\phi'(x^\mu + \omega^{\mu\nu} x_\nu) =
\phi'(x) + \omega^{\mu\nu} x_\nu \partial_\mu\phi(x) = \phi(x),
\end{dmath}
so
\begin{dmath}\label{eqn:qftLecture9:480}
\delta \phi = \phi'(x) - \phi(x) =
-\omega^{\mu\nu} x_\nu \partial_\mu \phi.
\end{dmath}

Since \( \LL \) is a scalar
\begin{dmath}\label{eqn:qftLecture9:500}
\delta \LL
=
-\omega^{\mu\nu} x_\nu \partial_\mu \LL
=
-
\partial_\mu \lr{
   \omega^{\mu\nu} x_\nu \LL
}
+
(\partial_\mu x_\nu) \omega^{\mu\nu} \LL
=
\partial_\mu \lr{
-
   \omega^{\mu\nu} x_\nu \LL
},
\end{dmath}
since \( \partial_\nu x_\mu = g_{\nu\mu} \) is symmetric, and \( \omega \) is antisymmetric.
Our current is
\begin{dmath}\label{eqn:qftLecture9:520}
J^\mu_\omega
=
-
   \omega^{\mu\nu} x_\mu \LL
.
\end{dmath}
\paragraph{FIXME: index mismatch above!}.

Our Noether current is
\begin{dmath}\label{eqn:qftLecture9:540}
j^\nu_{\omega^{\mu\rho}}
= \PD{\phi_{,\nu}}{\LL} \delta \phi - J^\mu_\omega
=
\partial^\nu \phi\lr{ - \omega^{\mu\rho} x_\rho \partial_\mu \phi } + \omega^{\nu \rho} x_\rho \LL
=
\omega^{\mu\rho}
\lr{
   \partial^\nu \phi\lr{ - x_\rho \partial_\mu \phi } + {\delta^{\nu}}_\mu x_\rho \LL
}
=
\omega^{\mu\rho} x_\rho
\lr{
   -\partial^\nu \phi \partial_\mu \phi + {\delta^{\nu}}_\mu \LL
}.
\end{dmath}
We identify
\begin{dmath}\label{eqn:qftLecture9:560}
-
{T^\nu}_\mu =
   -\partial^\nu \phi \partial_\mu \phi + {\delta^{\nu}}_\mu \LL,
\end{dmath}
so the current is
\begin{equation}\label{eqn:qftLecture9:580}
j^\nu_{\omega_{\mu\rho}}
=
-\omega^{\mu\rho} x_\rho
{T^\nu}_\mu
=
-\omega_{\mu\rho} x^\rho
T^{\nu\mu}
.
\end{equation}
Define
\begin{dmath}\label{eqn:qftLecture9:600}
j^{\nu\mu\rho} = \inv{2} \lr{ x^\rho T^{\nu\mu} - x^{\mu} T^{\nu\rho} },
\end{dmath}
which retains the antisymmetry in \( \mu \rho \) yet still drops the parameter \( \omega^{\mu\rho} \).
To check that this makes sense, we can contract
\( j^{\nu\mu\rho} \) with \( \omega_{\rho\mu} \)
\begin{dmath}\label{eqn:qftLecture9:880}
j^{\nu\mu\rho} \omega_{\rho\mu}
= -\inv{2} \lr{ x^\rho T^{\nu\mu} - x^{\mu} T^{\nu\rho} }
\omega_{\mu\rho}
=
-\inv{2} x^\rho T^{\nu\mu}
\omega_{\mu\rho}
- \inv{2} x^{\mu} T^{\nu\rho}
\omega_{\rho\mu}
=
-\inv{2} x^\rho T^{\nu\mu}
\omega_{\mu\rho}
- \inv{2} x^{\rho} T^{\nu\mu}
\omega_{\mu\rho}
=
- x^{\rho} T^{\nu\mu}
\omega_{\mu\rho},
\end{dmath}
which matches \cref{eqn:qftLecture9:580} as desired.

\paragraph{Example.  Rotations \( \mu\rho = ij \)}
\index{rotations}

\begin{dmath}\label{eqn:qftLecture9:620}
J^{0 i j} \epsilon_{ijk}
=
\inv{2} \lr{ x^i T^{0j} - x^{j} T^{0i} } \epsilon_{ijk}
=
x^i T^{0j} \epsilon_{ijk}.
\end{dmath}
Observe that this has the structure of \( (\Bx \cross \Bp)_k \), where \( \Bp \) is the momentum density of the field.
Let
\begin{equation}\label{eqn:qftLecture9:640}
L_k \equiv Q_k = \int d^3 x J^{0ij} \epsilon_{ijk}.
\end{equation}
We can now quantize and build a generator
\begin{dmath}\label{eqn:qftLecture9:660}
\hatU(\Balpha)
= e^{i \Balpha \cdot \hat{\BL}}
= \exp\lr{i \alpha_k
\int d^3 x x^i \hat{T}^{0j} \epsilon_{ijk}
}
\end{dmath}
From \cref{eqn:qftLecture9:560} we can quantize with \( T^{0j} = \partial^0 \phi \partial^j \phi \rightarrow \pihat \lr{\spacegrad \phihat}_j\), or
\begin{dmath}\label{eqn:qftLecture9:900}
\hatU(\Balpha)
=
\exp\lr{i \alpha_k
\int d^3 x x^i \pihat (\spacegrad \phihat)_j \epsilon_{ijk}
}
=
\exp\lr{i \Balpha \cdot
\int d^3 x \pihat \spacegrad \phihat \cross \Bx
}
\end{dmath}
(up to a sign in the exponent which doesn't matter)
\begin{dmath}\label{eqn:qftLecture9:680}
\phihat(\By) \rightarrow \hatU(\alpha) \phihat(\By) \hatU^\dagger(\alpha)
\approx
\phihat(\By) +
i \Balpha \cdot
\antisymmetric{
   \int d^3 x \pihat(\Bx) \spacegrad \phihat(\Bx) \cross \Bx
}
{
   \phihat(\By)
}
=
\phihat(\By) +
i \Balpha \cdot
   \int d^3 x
(-i) \deltathree(\Bx - \By)
\spacegrad \phihat(\Bx) \cross \Bx
=
\phihat(\By) +
\Balpha \cdot \lr{ \spacegrad \phihat(\By ) \cross \By}.
\end{dmath}
Explicitly, in coordinates, this is
\begin{dmath}\label{eqn:qftLecture9:700}
\phihat(\By) \rightarrow
\phihat(\By) +
\alpha^i
\lr{
   \partial^j \phihat(\By) y^k \epsilon_{jki}
}
=
\phihat(\By) -
\epsilon_{ikj} \alpha^i y^k \partial^j \phihat
=
\phihat( y^j - \epsilon^{ikj} \alpha^i y^k ).
\end{dmath}
This is a rotation.  To illustrate, pick \( \Balpha = (0, 0, \alpha) \), so \( y^j \rightarrow y^j - \epsilon^{ikj} \alpha y^k \delta_{i3} = y^j - \epsilon^{3kj} \alpha y^k \), or
\begin{dmath}\label{eqn:qftLecture9:920}
\begin{aligned}
y^1 &\rightarrow y^1 - \epsilon^{3k1} \alpha y^k = y^1 + \alpha y^2 \\
y^2 &\rightarrow y^2 - \epsilon^{3k2} \alpha y^k = y^2 - \alpha y^1 \\
y^3 &\rightarrow y^3 - \epsilon^{3k3} \alpha y^k = y^3,
\end{aligned}
\end{dmath}
or in matrix form
\begin{dmath}\label{eqn:qftLecture9:720}
\begin{bmatrix}
y^1 \\
y^2 \\
y^3 \\
\end{bmatrix}
\rightarrow
\begin{bmatrix}
1 & \alpha & 0 \\
-\alpha & 1 & 0 \\
0 & 0 & 1
\end{bmatrix}
\begin{bmatrix}
y^1 \\
y^2 \\
y^3 \\
\end{bmatrix}.
\end{dmath}

%\EndNoBibArticle
