%
% Copyright � 2018 Peeter Joot.  All Rights Reserved.
% Licenced as described in the file LICENSE under the root directory of this GIT repository.
%
%{
%%%\input{../latex/blogpost.tex}
%%%\renewcommand{\basename}{qftLecture12}
%%%\renewcommand{\dirname}{notes/phy2403/}
%%%\newcommand{\keywords}{PHY2403H}
%%%\input{../latex/peeter_prologue_print2.tex}
%%%
%%%%\usepackage{phy2403}
%%%\usepackage{peeters_braket}
%%%\usepackage{peeters_layout_exercise}
%%%\usepackage{peeters_figures}
%%%\usepackage{mathtools}
%%%\usepackage{siunitx}
%%%\usepackage{macros_cal} % LL
%%%% :%s/\\tG/\\tilde{G}/g
%%%
%%%\newcommand{\ultensor}[3]{{{#1}^{#2}}_{#3}}
%%%
%%%\beginArtNoToc
%%%\generatetitle{PHY2403H Quantum Field Theory.  Lecture 12: Klein-Gordon Green's function, Feynman propagator path deformation, Wightman function, Retarded Green's function.  Taught by Prof.\ Erich Poppitz}
%\chapter{Klein-Gordon Green's function, Feynman propagator path deformation, Wightman function, Retarded Green's function.}
\label{chap:qftLecture12}
\index{Green's function!Klein-Gordon}
\index{Feynman propagator}
\index{path deformation}
\index{Green's function!retarded time}

%%%\paragraph{DISCLAIMER: Very rough notes from class, with some additional side notes.}
%%%
%%%These are notes for the UofT course PHY2403H, Quantum Field Theory, taught by Prof. Erich Poppitz, fall 2018.
%%%%, covering \textchapref{{1}} \citep{peskin1995introduction} content.
%%%
\section{Green's functions for the forced Klein-Gordon equation.}

The problem were were preparing to do was to study the problem of ``particle creation by external classical source''.

We continue with a real scalar field, free, massive, but with an interaction with a source
\begin{dmath}\label{eqn:qftLecture12:20}
S_{\text{int}} = \int d^4 x j(x) \phi(x).
\end{dmath}

\paragraph{Modern application:} think of \( \phi \) has some SM field and think of \( j \) as due to inflaton (i.e. cosmological inflation interaction) oscillation.  In the inflationary model, the process of ``reheating'' creates all the matter in the universe.  We won't be talking about inflation, but will be considering a toy model that has some similar characteristics to the inflationary theory.
\index{inflation}

The equation of motion that we end up with is
\begin{dmath}\label{eqn:qftLecture12:40}
\lr{ \partial_\mu \partial^\mu + m^2} \phi = j,
\end{dmath}
and we wish to solve this using Green's function techniques.

\makedefinition{Klein-Gordon Green's function.}{dfn:qftLecture12:60}{
The QFT conventions for the Klein-Gordon Green's function is
\begin{equation*}
\lr{ \partial_\mu \partial^\mu + m^2} G(x - y) = -i \deltafour(x - y).
\end{equation*}
} % definition

As usual, we assume that it is possible to find a solution \( \phi \) by convolution
\begin{dmath}\label{eqn:qftLecture12:80}
\phi(x) = i \int d^4 y G(x - y) j(y).
\end{dmath}
\paragraph{Check:}
\begin{dmath}\label{eqn:qftLecture12:100}
\lr{ \partial_\mu \partial^\mu + m^2} \phi(x)
=
i
\lr{ \partial_\mu \partial^\mu + m^2}
\int d^4 y G(x - y) j(y)
=
i \int d^4 y (-i) \deltafour(x - y) j(y)
= j(x).
\end{dmath}

Also, as usual, we take out our Fourier transforms, the power tool of physics, and determine the structure of the Green's function by inverting the transform equation
\begin{dmath}\label{eqn:qftLecture12:120}
G(x - y) = \int \frac{d^4 p}{(2 \pi)^4} e^{-i p \cdot (x-y) } \tilde{G}(p).
\end{dmath}
Operating with Klein-Gordon gives
\begin{dmath}\label{eqn:qftLecture12:520}
\lr{ \partial_\mu \partial^\mu + m^2}
G(x)
=
\int \frac{d^4 p}{(2 \pi)^4}
\lr{ (-i p_\mu)(-i p^\mu) + m^2 }
e^{-i p \cdot (x-y) } \tilde{G}(p).
\end{dmath}
This must equal
\begin{dmath}\label{eqn:qftLecture12:140}
-i \deltafour(x - y) =
-i \int \frac{d^4 p}{(2 \pi)^4} e^{-i p \cdot (x -y)},
\end{dmath}
or
\begin{dmath}\label{eqn:qftLecture12:160}
\lr{ m^2 - p_\mu p^\mu } \tilde{G}(p) = -i.
\end{dmath}
The Green's function in the momentum domain is
\begin{dmath}\label{eqn:qftLecture12:180}
\tilde{G}(p) = \frac{i}{p^2 - m^2}.
\end{dmath}

The inverse transform provides the spatial domain representation of the Green's function
\begin{dmath}\label{eqn:qftLecture12:200}
G(x)
=
\int \frac{d^4 p}{(2 \pi)^4} e^{-i p \cdot x }
\frac{i}{(p^0)^2 - \Bp^2 - m^2}
=
\int \frac{d^3 p}{(2\pi)^3} e^{i \Bp \cdot \Bx}
\int \frac{d p_0}{2 \pi} e^{-i p_0 x^0 }
\frac{i}{(p_0 - \omega_\Bp)(p_0 + \omega_\Bp)}.
\end{dmath}

In the \( p_0 \) plane, we have two poles at \( p_0 = \pm \omega_\Bp \).
There are 4 ways to go around the poles, the retarded time deformation that we used to derive the Green's function for the harmonic oscillator, as sketched in \cref{fig:retardedTimeContours:retardedTimeContoursFig1}, the advanced time deformation sketched in \cref{fig:deformationForAdvancedTime:deformationForAdvancedTimeFig2}, and mixed deformations.
\imageFigure{../figures/phy2403-quantum-field-theory/retardedTimeContoursFig1}{Retarded time deformations and contours.}{fig:retardedTimeContours:retardedTimeContoursFig1}{0.3}
\imageFigure{../figures/phy2403-quantum-field-theory/deformationForAdvancedTimeFig2}{Advanced time deformation.}{fig:deformationForAdvancedTime:deformationForAdvancedTimeFig2}{0.1}

We will evaluate the integral using the ``Feynman propagator'' contour
sketched in \cref{fig:feynmanDeformation:feynmanDeformationFig3}.
Why we use the Feynman contour, and not the retarded contour can be justified by how well this works for the perturbation methods that will be developed later.
\imageFigure{../figures/phy2403-quantum-field-theory/feynmanDeformationFig3}{Feynman propagator deformation path.}{fig:feynmanDeformation:feynmanDeformationFig3}{0.1}

Consider each contour in turn.
\paragraph{Case I.  \( x^0 > 0 \)}

For this case, we use the lower half plane contour sketched in \cref{fig:feynmanContour:feynmanContourFig4}, which vanishes for \( \Im(p_0) < 0, x_0 > 0 \), where \( -i (i \Im(p_0) x_0) < 0 \).
\imageFigure{../figures/phy2403-quantum-field-theory/feynmanContourFig4}{Feynman propagator contour for \( t > 0 \).}{fig:feynmanContour:feynmanContourFig4}{0.3}

Here we pick up just the pole at \( p_0 = \omega_\Bp \), and take a negatively oriented path
\begin{dmath}\label{eqn:qftLecture12:220}
G_\txtF
=
\int \frac{d^3 p}{(2\pi)^3} e^{i \Bp \cdot \Bx}
\int \frac{d p_0}{2 \pi} e^{-i p_0 x^0 }
\frac{i}{(p_0 - \omega_\Bp)(p_0 + \omega_\Bp)}
=
\int \frac{d^3 p}{(2\pi)^3} e^{i \Bp \cdot \Bx}
(-2 \pi i)
\evalbar{\lr{ \frac{e^{-i p_0 x^0 }}{2 \pi}
\frac{i}{p_0 + \omega_\Bp} }}{p_0 = \omega_p}
=
\int \frac{d^3 p}{(2\pi)^3} e^{i \Bp \cdot \Bx}
\frac{-2 \pi i}{2 \pi} \frac{i e^{-i p_0 x^0 } }{2 \omega_\Bp}
=
\int \frac{d^3 p}{(2\pi)^3} e^{i \Bp \cdot \Bx}
\frac{ e^{-i \omega_\Bp x^0 } }{2 \omega_\Bp}.
\end{dmath}

\paragraph{Case II.  \( x^0 < 0 \)}

For \( x^0 < 0 \) we use an upper half plane contour with the same deformation around the poles.  This time
\begin{dmath}\label{eqn:qftLecture12:240}
G_\txtF
=
\int \frac{d^3 p}{(2\pi)^3} e^{i \Bp \cdot \Bx}
\int \frac{d p_0}{2 \pi} e^{-i p_0 x^0 }
\frac{i}{(p_0 - \omega_\Bp)(p_0 + \omega_\Bp)}
=
\int \frac{d^3 p}{(2\pi)^3} e^{i \Bp \cdot \Bx}
(+ 2 \pi i)
\evalbar{\lr{\frac{e^{-i p_0 x^0 }}{2 \pi}
\frac{i}{p_0 - \omega_\Bp}}}{p_0 = -\omega_\Bp}
=
\int \frac{d^3 p}{(2\pi)^3} e^{i \Bp \cdot \Bx}
\frac{+2 \pi i}{2 \pi} \frac{i e^{-i p_0 x^0 } }{-2 \omega_\Bp}
=
\int \frac{d^3 p}{(2\pi)^3} e^{i \Bp \cdot \Bx}
\frac{ e^{i \omega_\Bp x^0 } }{2 \omega_\Bp}.
\end{dmath}
We've obtained
a piecewise representation of the Green's function, where the only difference is the sign of the \( i \omega_\Bp x^0 \) exponential.

We can combine
\cref{eqn:qftLecture12:220}
\cref{eqn:qftLecture12:240} by using \( \Theta \) functions
\begin{dmath}\label{eqn:qftLecture12:260}
\int \frac{d^3 p}{(2\pi)^3 2 \omega_\Bp} e^{i \Bp \cdot \Bx}
\lr{
e^{-i \omega_\Bp x^0 } \Theta(x_0)
+
e^{i \omega_\Bp x^0 } \Theta(-x_0)
}.
\end{dmath}
The first integral (without the \(\Theta\) factor) is the Wightman function
\index{Wightman function}
%\begin{dmath}\label{eqn:qftLecture12:280}
\boxedEquation{eqn:qftLecture12:280}{
D(x)
=
\int \frac{d^3 p}{(2\pi)^3 2 \omega_\Bp} \evalbar{e^{-i p \cdot x}}{p^0 = \omega_\Bp}.
}
%\end{dmath}

For the second integral, we make a change of variables \( \Bp \rightarrow -\Bp \) leaving
\begin{dmath}\label{eqn:qftLecture12:300}
\int \frac{d^3 p}{(2\pi)^3 2 \omega_\Bp} e^{i \Bp \cdot \Bx + i \omega_\Bp x^0}
\rightarrow
\int \frac{d^3 p}{(2\pi)^3 2 \omega_\Bp} e^{-i \Bp \cdot \Bx + i \omega_\Bp x^0}
=
\int \frac{d^3 p}{(2\pi)^3 2 \omega_\Bp} e^{-i p \cdot x}
= D(-x),
\end{dmath}
so
%\begin{dmath}\label{eqn:qftLecture12:320}
\index{Feynman propagator}
\boxedEquation{eqn:qftLecture12:340}{
G_\txtF (x) = \Theta(x^0) D(x) + \Theta(-x^0) D(-x).
}
%\end{dmath}

%
% Copyright © 2018 Peeter Joot.  All Rights Reserved.
% Licenced as described in the file LICENSE under the root directory of this GIT repository.
%
\section{Pole shifting.}
\index{pole shifting}

Recall that the four dimensional form of the Green's function was
\begin{equation}\label{eqn:qftLecture13:400}
D_F = i \int \frac{d^4 p}{(2 \pi)^4} e^{-i p \cdot x} \inv{ p^2 - m^2 }.
\end{equation}
For the Feynman case, the contour that we were taking around the poles can also be accomplished by shifting the poles strategically, as sketched in \cref{fig:feynmanDeformationTwoWays:feynmanDeformationTwoWaysFig1}.

\imageFigure{../figures/phy2403-quantum-field-theory/feynmanDeformationTwoWaysFig1}{Feynman deformation or equivalent shift of the poles.}{fig:feynmanDeformationTwoWays:feynmanDeformationTwoWaysFig1}{0.2}

This shift can be expressed explicit algebraically by introducing an offset
%\begin{equation}\label{eqn:qftLecture13:420}
\boxedEquation{eqn:qftLecture13:420}{
D_F = i \int \frac{d^4 p}{(2 \pi)^4} e^{-i p \cdot x} \inv{ p^2 - m^2 + i \epsilon },
}
%\end{equation}
which puts the poles at
\begin{equation}\label{eqn:qftLecture13:440}
\begin{aligned}
p^0
&= \pm \sqrt{ \omega_\Bp^2 - i \epsilon }
\\&= \pm \omega_\Bp \lr{ 1 - \frac{i \epsilon}{\omega_\Bp^2} }^{1/2}
\\&= \pm \omega_\Bp \lr{ 1 - \inv{2} \frac{i \epsilon}{\omega_\Bp^2} }
\\&=
\left\{
\begin{array}{l}
+\omega_\Bp - \inv{2} i \frac{\epsilon}{\omega_\Bp} \\
-\omega_\Bp + \inv{2} i \frac{\epsilon}{\omega_\Bp} \\
\end{array}
\right.
\end{aligned}
\end{equation}



\section{Matrix element representation of the Wightman function.}
\index{Wightman function}

Recall that the Wightman function \cref{eqn:qftLecture12:280} also had a matrix element representation
\begin{dmath}\label{eqn:qftLecture12:360}
D(x) = \bra{0} \phi(x) \phi(0) \ket{0}.
\end{dmath}
This can be shown by expansion.
\begin{dmath}\label{eqn:qftLecture12:380}
\bra{0} \phi(x) \phi(0) \ket{0}
=
\bra{0}
\int \frac{d^3 p}{(2 \pi)^3} \inv{\sqrt{2 \omega_\Bp}} \evalbar{\lr{ a_\Bp e^{-i p \cdot x} + a_\Bp^\dagger e^{i p \cdot x} }}{p_0 = \omega_\Bp}
\int \frac{d^3 q}{(2 \pi)^3} \inv{\sqrt{2 \omega_\Bq}} \lr{ a_\Bq^\dagger + a_\Bq }
\ket{0}.
\end{dmath}
Since \( a_\Bq \ket{0} = 0 = \bra{0} a_\Bp^\dagger \), \cref{eqn:qftLecture12:380}
reduces to
\begin{dmath}\label{eqn:qftLecture12:540}
\bra{0} \phi(x) \phi(0) \ket{0}
=
\bra{0}
\int
\frac{d^3 p}{(2 \pi)^3}
\frac{d^3 q}{(2 \pi)^3}
\inv{\sqrt{2 \omega_\Bp}}
\inv{\sqrt{2 \omega_\Bq}}
\evalbar{\lr{ a_\Bp a_\Bq^\dagger e^{-i p \cdot x} }}{p_0 = \omega_\Bp}
\ket{0}
=
\bra{0}
\int
\frac{d^3 p}{(2 \pi)^3}
\frac{d^3 q}{(2 \pi)^3}
\inv{\sqrt{2 \omega_\Bp}}
\inv{\sqrt{2 \omega_\Bq}}
\evalbar{\lr{
\lr{
   a_\Bp
   a_\Bq^\dagger
   +
   \antisymmetric{
   a_\Bp
   }{
   a_\Bq^\dagger
   }
}
e^{-i p \cdot x} }}{p_0 = \omega_\Bp}
\ket{0}
=
\bra{0}
\int
\frac{d^3 p}{(2 \pi)^3}
\frac{d^3 q}{(2 \pi)^3}
\inv{\sqrt{2 \omega_\Bp}}
\inv{\sqrt{2 \omega_\Bq}}
\evalbar{\lr{
\lr{
   (2 \pi)^3 \deltathree(\Bp - \Bq)
}
e^{-i p \cdot x} }}{p_0 = \omega_\Bp}
=
\int \frac{d^3 p}{(2 \pi)^3} \evalbar{ \frac{e^{-i p \cdot x}}{2 \omega_\Bp} }
{p_0 = \omega_\Bp}.
\end{dmath}

\section{Retarded Green's function.}

Claim: Retarded Green's function (bumps up contour) can be written
\begin{dmath}\label{eqn:qftLecture12:400}
D_R(x) = \theta(x_0) (D(x) - D(-x)),
\end{dmath}
where \( D(x) \) is given by \cref{eqn:qftLecture12:280}.
Proof: The upper half plane contour (\(x_0 < 0\)) is zero since it encloses no poles.  For the
lower half plane contour we have
\begin{dmath}\label{eqn:qftLecture12:420}
\evalbar{D_R(x)}{x_0 > 0}
=
i \int \frac{d^3 p}{(2 \pi)^3} e^{i \Bp \cdot \Bx} \int \frac{dp_0}{2 \pi} e^{-i p_0 x^0 }
\frac{i}{(p_0 - \omega_\Bp)(p_0 + \omega_\Bp)}
=
i \int \frac{d^3 p}{(2 \pi)^3} e^{i \Bp \cdot \Bx} \frac{(-2 \pi i)}{2 \pi}
\lr{
e^{-i \omega_\Bp x^0 }
\frac{i}{2 \omega_\Bp}
+
e^{i \omega_\Bp x^0 }
\frac{i}{-2 \omega_\Bp}
}
=
\int \frac{d^3 p}{(2 \pi)^3} e^{i \Bp \cdot \Bx}
\frac{1}{2 \omega_\Bp}
\lr{
e^{-i \omega_\Bp x^0 }
-
e^{i \omega_\Bp x^0 }
}
=
D(x) - D(-x).
\end{dmath}

What does the field look like in terms of the propagator?
Assuming that \( \phi_0 \) satisfies the homogeneous equation, we have
\begin{dmath}\label{eqn:qftLecture12:440}
\phi(x)
= \phi_0(x) + i \int d^4 y D_R(x - y) j(y)
= \phi_0(x) + i \int d^3 y d y_0 \Theta(x_0 - y_0) \lr{ D(x - y) - D(y - x) } j(y).
\end{dmath}

Imagine that we have a windowed source function \( j(y^0, \By) \), as sketched in \cref{fig:windowedImpulse:windowedImpulseFig5}.

\imageFigure{../figures/phy2403-quantum-field-theory/windowedImpulseFig5}{Finite window impulse response.}{fig:windowedImpulse:windowedImpulseFig5}{0.2}

\begin{dmath}\label{eqn:qftLecture12:460}
\begin{aligned}
&\evalbar{\phi(x)}{x^0 > t_{\text{after}}}
= \phi_0(x) \\
&\qquad + i \int d^4 y
\Biglr{
   \int \frac{d^3 p}{(2\pi)^3 2 \omega_\Bp } e^{-i p \cdot (x - y)} j(y)
   -
   \int \frac{d^3 p}{(2\pi)^3 2 \omega_\Bp } e^{i p \cdot (x - y)} j(y)
}.
\end{aligned}
\end{dmath}
Define
\begin{dmath}\label{eqn:qftLecture12:480}
\tilde{j}(p) = \int d^4 y e^{i p \cdot y} j(y),
\end{dmath}
which gives
\begin{dmath}\label{eqn:qftLecture12:500}
\evalbar{\phi(x)}{x^0 > t_{\text{after}}}
= \phi_0(x)
+ i \int \frac{d^3 p }{(2 \pi)^3}
\inv{
2 \omega_\Bp }
\evalbar{
\lr{
   e^{-i p \cdot x} \tilde{j}(p)
   - e^{i p \cdot x} \tilde{j}(-p)
}
}{p_0 = \omega_\Bp}.
\end{dmath}
We will interpret this in the next lecture, and start in on Feynman diagrams.

%}
%\EndNoBibArticle
