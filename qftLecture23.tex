%
% Copyright � 2017 Peeter Joot.  All Rights Reserved.
% Licenced as described in the file LICENSE under the root directory of this GIT repository.
%
%{
%%\input{../latex/blogpost.tex}
%%\renewcommand{\basename}{qftLecture23}
%%\renewcommand{\dirname}{notes/phy2403/}
%%\newcommand{\keywords}{PHY2403H}
%%\input{../latex/peeter_prologue_print2.tex}
%%
%%%\usepackage{phy2403}
%%\usepackage{peeters_braket}
%%\usepackage{peeters_layout_exercise}
%%\usepackage{peeters_figures}
%%\usepackage{mathtools}
%%\usepackage{siunitx}
%%\usepackage{macros_cal} % LL
%%\usepackage{simplewick}
%%\usepackage{slashed}
%%
%%\newcommand{\ultensor}[3]{{{#1}^{#2}}_{#3}}
%%\newcommand{\oPsi}[0]{\overbar{\Psi}}
%%\newcommand{\osigma}[0]{\overbar{\sigma}}
%%\newcommand{\ubar}[0]{\overbar{u}}
%%\newcommand{\vbar}[0]{\overbar{v}}
%%\newcommand{\deltathree}[0]{\delta^{(3)}}
%%\newcommand{\deltafour}[0]{\delta^{(4)}}
%%\newcommand{\ITwo}[0]{{\begin{bmatrix} 1 & 0 \\ 0 & 1 \end{bmatrix}}}
%%\newcommand{\DiracGammaZero}[0]{{\begin{bmatrix} 0 & 1 \\ 1 & 0 \end{bmatrix}}}
%%\newcommand{\DiracGammaK}[1]{{\begin{bmatrix} 0 & \sigma^{#1} \\ -\sigma^{#1} & 0 \end{bmatrix}}}
%%\newcommand{\nbref}[1]{#1}
%%
%%\beginArtNoToc
%%\generatetitle{PHY2403H Quantum Field Theory.  Lecture 23: QED and QCD interaction Lagrangian, Feynman propagator and rules for fermions, hadron pair production, scattering cross section, quark pair production.  Taught by Prof.\ Erich Poppitz}
%%%\chapter{QED and QCD interaction Lagrangian, Feynman propagator and rules for fermions, hadron pair production, scattering cross section, quark pair production}
%%\label{chap:qftLecture23}
%%
%%\paragraph{DISCLAIMER: Notes from class, with auxillary details.}
%%
%%These are notes for the UofT course PHY2403H, Quantum Field Theory, taught by Prof. Erich Poppitz, fall 2018.
%%
%%These notes cover the final lecture of the course, which followed \textchapref{{1}} \citep{peskin1995introduction} \S 5.1 fairly closely (filling in some details, leaving out some others.)
%%
\section{Review.}

Our Lagrangian is
\begin{dmath}\label{eqn:qftLecture23:140}
\LL_{\text{Dirac}} = \oPsi \lr{ i \gamma^\mu \partial_\mu -m } \Psi,
\end{dmath}
which can be consider solved by fields \( \Psi(x), \oPsi(x) = \Psi^\dagger(x) \gamma^0 \)
\begin{subequations}
\label{eqn:qftLecture23:420}
\begin{dmath}\label{eqn:qftLecture23:160}
\Psi(x) =
\sum_{s = 1}^2
\int \frac{d^3 p}{(2 \pi)^3 \sqrt{2 \omega_\Bp}}
\lr{
   e^{-i p \cdot x} u^s(p) a_\Bp^s
   +
   e^{i p \cdot x} v^s(p) a_\Bp^{s \dagger}
}
\end{dmath}
\begin{dmath}\label{eqn:qftLecture23:180}
\oPsi(x)
=
\sum_{s = 1}^2
\int \frac{d^3 p}{(2 \pi)^3 \sqrt{2 \omega_\Bp}}
\lr{
   e^{i p \cdot x} \ubar^s(p) a_\Bp^{s\dagger}
   +
   e^{-i p \cdot x} \vbar^s(p) a_\Bp^{s}
}
\end{dmath}
\end{subequations}
where the creation and annihilation operators satisfy
\begin{subequations}
\label{eqn:qftLecture23:400}
\begin{dmath}\label{eqn:qftLecture23:200}
\symmetric{a_\Bp^s}{a_\Bq^{r\dagger}} = (2 \pi)^3 \delta^{sr} \deltathree( \Bp - \Bq),
\end{dmath}
\begin{dmath}\label{eqn:qftLecture23:201}
\symmetric{b_\Bp^s}{b_\Bq^{r\dagger}} = (2 \pi)^3 \delta^{sr} \deltathree( \Bp - \Bq),
\end{dmath}
\end{subequations}
(plus various relations for the \( u,v\)'s.)

\section{Photon.}
Recall that we identified a number of symmetries
\begin{itemize}
\item \( SO(1,3) \)
\item \( P, C, T \) : DIY
\item \( U(1)_V \) : \( \Psi \rightarrow e^{i \alpha} \Psi \)
\item \( U(1)_A \) : If \( m = 0 \), then \( U(1)_A : \Psi \rightarrow e^{i \alpha \gamma_5 } \Psi \).  If \( m \ne 0 \) only for \( \alpha = \pi \) : \( \Psi \rightarrow -\Psi \).
\end{itemize}

Photon interaction can be introduced by utilizing a \( U(1) \) gauge field, demanding invariance under \( U(1)_V \) with \( \alpha = \alpha(x) \).  That is
\begin{dmath}\label{eqn:qftLecture23:20}
\Psi(x) \rightarrow e^{i \alpha(x) } \Psi(x),
\end{dmath}
which has derivatives
\begin{dmath}\label{eqn:qftLecture23:280}
\partial_\mu \Psi(x) \rightarrow
e^{i \alpha(x) } \lr{ \partial_\mu \Psi(x) + i \partial_\mu \alpha(x) \Psi(x) }.
\end{dmath}

Solution.  Introduce \( A_\mu(x) \), such that under \( U(1)_V \) we have

\begin{dmath}\label{eqn:qftLecture23:300}
A_\mu(x) \rightarrow A_\mu(x) - \inv{e} \partial_\mu \alpha(x)
\end{dmath}
where ``e'' is a dimensionless coupling constant
\begin{dmath}\label{eqn:qftLecture23:320}
\partial_\mu \Psi(x) \rightarrow
\lr{ \partial_\mu + i e A_\mu } \Psi
\rightarrow
e^{i \alpha(x)} \lr{ \partial_\mu \Psi + \cancel{ i \partial_\mu \alpha} \Psi - \cancel{ i \partial_\mu \alpha } \Psi }
\end{dmath}

We've now constructed the QED Lagrangian density
\begin{dmath}\label{eqn:qftLecture23:40}
\LL_{\text{QED}} = \oPsi \lr{ i \gamma^\mu \lr{ \partial_\mu + i e A_\mu } - m } \Psi - \inv{4} F_{\mu\nu} F^{\mu\nu}.
\end{dmath}
%This is Gauged \( U(1)_V \) symmetry.

We may write this as
\begin{dmath}\label{eqn:qftLecture23:60}
\LL_{\text{QED}} =
\mathLabelBox[ labelstyle={yshift=1.2em}, linestyle={} ]
{
- \inv{4} F_{\mu\nu} F^{\mu\nu}
+
\oPsi \lr{ i \gamma^\mu \partial_\mu - m } \Psi
}
{
Free Lagrangian
}
-
\mathLabelBox[ labelstyle={below of=m\themathLableNode, below of=m\themathLableNode} ]
{
e \oPsi \gamma_\mu \Psi A^\mu
}
{
interaction Lagrangian
}
\end{dmath}

We introduce spinor fields \( \Psi_e \) and muon fields \( \Psi_\mu \), so that the total Lagrangian is now
\begin{dmath}\label{eqn:qftLecture23:80}
\LL_{\text{QED}}
=
- \inv{4} F_{\mu\nu} F^{\mu\nu}
+
\oPsi_e \lr{ i \gamma^\mu \partial_\mu - m } \Psi_e
-
e \oPsi_e \gamma_\mu \Psi_e A^\mu
+
\oPsi_\mu \lr{ i \gamma^\mu \partial_\mu - m } \Psi_\mu
-
e \oPsi_\mu \gamma_\mu \Psi_\mu A^\mu
\end{dmath}

\begin{itemize}
\item \( m_e \sim 0.5 \,\si{MeV} \)
\item \( m_\mu \sim 105 \,\si{MeV} \)
\end{itemize}

There are also quark fields that we can add into the mix
\begin{dmath}\label{eqn:qftLecture23:100}
\LL_{\text{quarks}} = \sum_q \oPsi_q \lr{ i \gamma^\mu - m_q } \Psi_q + e Q_q \oPsi_q \gamma^\nu \Psi_q A_\nu
\end{dmath}
Quark charges are \( Q_q = (2/3, -1/3) \).
It turns out that the only way to produce quarks is through (electron?) interaction?

Can also introduce a Fermi interaction

\begin{dmath}\label{eqn:qftLecture23:120}
\LL_{4-Fermi} = \frac{c}{v^2} \oPsi_\mu \gamma^\nu \lr{ 1 - \gamma_5 } \Psi_{\nu, \mu} - \oPsi_e \lr{ 1 - \gamma_5 } ....
\end{dmath}

We now want to do some calculations with the photon interactions from \cref{eqn:qftLecture23:80}.  In particular, we will study the effects of the \( -e \oPsi_e \gamma_\mu \Psi_e A^\mu \) interaction Lagrangian.  
%This can result in creation and annihilation processes, including pair production (sketched in \cref{fig:l23:l23Fig1}), and
%\imageFigure{../figures/phy2403-quantum-field-theory/l23Fig1}{Pair production.}{fig:l23:l23Fig1}{0.3}
%photon absorbtion and emmission (sketched in \cref{fig:l23:l23Fig2}).
%\imageFigure{../figures/phy2403-quantum-field-theory/l23Fig3}{Photon absorbtion and emission.}{fig:l23:l23Fig3}{0.3}
%For the pair production process, we have a couple allowable diagrams, which are sketched in \cref{fig:l23:l23Fig3}.
%\imageFigure{../figures/phy2403-quantum-field-theory/l23Fig2}{Allowable diagrams}{fig:l23:l23Fig2}{0.3}

\section{Propagator.}

Before we can study the interaction, we need to determine the structure of the propagator.  For Grassman (anti-commuting) operators
\begin{dmath}\label{eqn:qftLecture23:220}
T( O_f(x) O'_f(x)) =
\Theta ( x_0 - x_0' ) O_f(x) O_f(x')
+
\Theta ( x_0' - x_0 ) O_f(x') O_f(x)
\end{dmath}
The propagator can be determined from
\begin{dmath}\label{eqn:qftLecture23:240}
\expectation{ T( \Psi_\alpha(x) \Psi_\beta(x) }_0 = D_{F_{\alpha \beta}}(x - y),
\end{dmath}
where \( \alpha, \beta = 1, 2, 3, 4 \).

Referring back to \cref{eqn:qftLecture23:160}, \cref{eqn:qftLecture23:180}, that propagator is
\begin{dmath}\label{eqn:qftLecture23:260}
\begin{aligned}
\expectation{ T( \Psi_\alpha(x) \Psi_\beta(x) }_0
&=
\int \frac{d^3 p}{(2\pi)^3 \sqrt{ 2 \omega_\Bp } }
\int \frac{d^3 q}{(2\pi)^3 \sqrt{ 2 \omega_\Bq } }
\biglr{
   e^{-i p \cdot x} e^{+ i q \cdot y}
   \Theta ( x_0 - y_0 ) u^s_\alpha(p) \ubar^r_\beta(q) \expectation{ a_\Bp^s a_\Bq^{r\dagger} } \\
\qquad +
   e^{i p \cdot x} e^{- i q \cdot y}
   \Theta ( y_0 - x_0 ) \vbar^s_\beta(p) v^r_\beta(q) \expectation{ b_\Bq^s a_\Bp^{r\dagger} }
} \\
&=
\int \frac{d^3 p}{(2\pi)^3 2 \omega_\Bp }
\lr{
   e^{-i p \cdot (x-y)}
   \Theta ( x_0 - y_0 ) u^s_\alpha(p) \ubar^r_\beta(p)
+
   e^{i p \cdot (x- y)}
   \Theta ( y_0 - x_0 ) \vbar^s_\beta(p) v^r_\beta(p)
} \\
&=
\int \frac{d^3 p}{(2\pi)^3 2 \omega_\Bp }
\lr{
   e^{-i p \cdot x}
   \Theta ( x_0 - y_0 )
\lr{ \gamma^\mu_{\alpha\beta} p_\mu + m}
+
   e^{i p \cdot x}
   \Theta ( y_0 - x_0 )
\lr{ \gamma^\mu_{\alpha\beta} p_\mu - m}
}
\end{aligned}
\end{dmath}
where \( \gamma^\mu_{\alpha\beta} \) are the \( \alpha, \beta \) components of the gamma matrices.
Now we can replace the \( p_\mu \)'s with derivatives acting on the exponentials

\begin{dmath}\label{eqn:qftLecture23:340}
\begin{aligned}
\expectation{ T( \Psi_\alpha(x) \Psi_\beta(x) }_0
&=
\Theta ( x_0 - y_0 )
\biglr{ i \gamma^\mu_{\alpha\beta} \partial_\mu + m }
\int \frac{d^3 p}{(2\pi)^3 2 \omega_\Bp } e^{-i p \cdot (x - y)} \\
&\qquad -
\Theta ( y_0 - x_0 )
\lr{ -i \gamma^\mu_{\alpha\beta} \partial_\mu - m }
\int \frac{d^3 p}{(2\pi)^3 2 \omega_\Bp } e^{-i p \cdot (x - y)} \\
&=
\Theta ( x_0 - y_0 )
\lr{ i \gamma^\mu_{\alpha\beta} \partial_\mu + m }
D(x - y)
-
\Theta ( y_0 - x_0 )
\lr{ -i \gamma^\mu_{\alpha\beta} \partial_\mu - m }
D(y - x) \\
&=
\lr{ \gamma^\mu_{\alpha\beta} \partial_\mu^{(x)} + m }
\biglr{
   \Theta ( x_0 - y_0 )
   D(x - y) 
   +
   \Theta ( y_0 - x_0 )
   D(y - x)
} \\
&-
i \gamma^0 \delta(x^0 - y^0) \cancel{\lr{ D(x - y) - D(y - x) }},
\end{aligned}
\end{dmath}
where we've killed off a factor that is zero (off the light cone?)

We are left with just an action on the Feynman propagator
\begin{dmath}\label{eqn:qftLecture23:360}
\expectation{ T( \Psi_\alpha(x) \Psi_\beta(x) }_0
=
\lr{ \gamma^\mu_{\alpha\beta} \partial_\mu^{(x)} + m } D_F(x - y)
=
\int \frac{d^4 p}{(2 \pi)^4 } \frac{ i ( \gamma^\mu_{\alpha\beta} p_\mu + m ) }{p^2 - m^2 + i \epsilon} e^{-i p \cdot (x - y)}
\end{dmath}

Now that we have a propagator, let's try
\begin{dmath}\label{eqn:qftLecture23:380}
\LL_{\text{int}} = \int dt d^3 x \lr{ e \oPsi \gamma_\mu \Psi A^\mu }.
\end{dmath}

\section{Feynman rules.}
We can consider various scattering processes, such as \( e^{+} e^{-} \rightarrow \mu^{+} \mu^{-} \) as sketched in
\cref{fig:l23:l23Fig1}, or
\( e^{+} e^{-} \rightarrow e^{+} e^{-} \) as sketched in
\cref{fig:l23:l23Fig2}, or
Compton scattering \( e^{-} \gamma \rightarrow e^{-} \gamma \) as sketched in
\cref{fig:l23:l23Fig3}.
\imageFigure{../figures/phy2403-quantum-field-theory/l23Fig1}{Electron, positron decay to muon pairs.}{fig:l23:l23Fig1}{0.2}
\imageFigure{../figures/phy2403-quantum-field-theory/l23Fig2}{Electron, positron collision.}{fig:l23:l23Fig2}{0.3}
\imageFigure{../figures/phy2403-quantum-field-theory/l23Fig3}{Compton scattering.}{fig:l23:l23Fig3}{0.2}

To do so we need to determine the Feynman rules for fermions.  For fermions \( \Psi \) and anti-fermions \( \oPsi \) we have
\begin{dmath}\label{eqn:qftLecture23:440}
\begin{aligned}
\contraction{}{\Psi}{\lvert}{\Bp} \Psi \ket{\Bp, s} &= u^s(p) \\
\contraction{}{\oPsi}{\lvert}{\Bp} \oPsi \ket{\Bp, s} &= \vbar^s(p) \\
\contraction{\langle}{\Bp}{, s \rvert}{\Psi} \langle \Bp, s \rvert \Psi &= \ubar^s(p) \\
\contraction{\langle}{\Bp}{, s \rvert}{\oPsi} \langle \Bp, s \rvert  \oPsi &= v^s(p),
\end{aligned}
\end{dmath}
where we mean
\begin{dmath}\label{eqn:qftLecture23:460}
\ket{\Bp, s} = a_\Bp^{s\dagger} \ket{0} \sqrt{ 2 \omega_\Bp },
\end{dmath}
for fermions, and
\begin{dmath}\label{eqn:qftLecture23:480}
\ket{\Bp, s} = b_\Bp^{s\dagger} \ket{0} \sqrt{ 2 \omega_\Bp },
\end{dmath}
for anti-fermions.

The flow of fermion and anti-fermion number charge is designated by arrow direction in the diagram, as in the respective diagrams of \cref{fig:l23:l23Fig4}.
\imageFigure{../figures/phy2403-quantum-field-theory/l23Fig4}{Flow of \# charge.}{fig:l23:l23Fig4}{0.3}

The Feynman propagator for fermions is
\begin{dmath}\label{eqn:qftLecture23:500}
\frac{ i \lr{ \slashed{p} + m } }{p^2 - m^2 + i \epsilon},
\end{dmath}
whereas the photon propagator is
\begin{dmath}\label{eqn:qftLecture23:520}
\expectation{ A_\mu A_\nu } = -i \frac{g_{\mu\nu}}{q^2 + i \epsilon}.
\end{dmath}

\section{Example: \( e^{-} e^{+} \rightarrow \mu^{-} \mu^{+} \).}
As an example, consider the process sketched in \cref{fig:l23:l23Fig5}.
\imageFigure{../figures/phy2403-quantum-field-theory/l23Fig5}{\( e^{-} e^{+} \rightarrow \mu^{-} \mu^{+} \) process.}{fig:l23:l23Fig5}{0.3}
Such a process is ``ultra-relativistic'', in that the electron and positron pair must be moving \underline{very fast} to create muons.

The matrix element is
\begin{dmath}\label{eqn:qftLecture23:540}
\contraction{\langle}{\mu^{+}}{\mu^{-} \rvert \oPsi \gamma^\sigma}{\Psi}
\bcontraction{\langle \mu^{+}}{\mu^{-}}{\rvert}{\oPsi}
\contraction{\langle \mu^{+} \mu^{-} \rvert \oPsi \gamma^\sigma \Psi }{A^\sigma}{}{A^\rho}
\contraction{\langle \mu^{+} \mu^{-} \rvert \oPsi \gamma^\sigma \Psi A^\sigma A^\rho}{\oPsi}{\gamma^\rho \Psi \lvert}{e^{+}}
\bcontraction{\langle \mu^{+} \mu^{-} \rvert \oPsi \gamma^\sigma \Psi A^\sigma A^\rho \oPsi \gamma^\rho}{\Psi}{\lvert e^{+}}{e^{-}}
\langle \mu^{+} \mu^{-} \rvert \oPsi \gamma^\sigma \Psi A^\sigma A^\rho \oPsi \gamma^\rho \Psi \lvert e^{+} e^{-} \rangle
=
\mathLabelBox{
\vbar^{s'}(p')
}
{
incoming anti-electron
}
(-i e \gamma^\rho )
\mathLabelBox[ labelstyle={below of=m\themathLableNode, below of=m\themathLableNode} ]
{
u^s(p)
}
{
incoming electron
}
\mathLabelBox[ labelstyle={yshift=1.2em}, linestyle={} ]
{
\lr{ \frac{-i g_{\rho\sigma}}{q^2} }
}
{
ignoring \( i \epsilon \).
}
\ubar^r(k) (-i e \gamma^\sigma ) v^{r'}(k')
\end{dmath}
\paragraph{Question:} Why are we writing the factors of the matrix element from left to right, corresponding to the right to left reading of the matrix element?

\Cref{eqn:qftLecture23:540} reduces to
\begin{dmath}\label{eqn:qftLecture23:560}
i M
=
i \frac{e^2}{q^2}
\vbar^{s'}(p') \gamma^\rho u^s(p)
\ubar^r(k) \gamma_\rho v^{r'}(k'),
\end{dmath}
where the \( (2 \pi)^4 \deltafour(...) \) term hasn't been made explicit.

We'd like to compute the absolute square of \cref{eqn:qftLecture23:560}, and use the following lemma to do so.
\makelemma{Some conjugates.}{lemma:qftLecture23:1}{
\begin{equation*}
\begin{aligned}
(\vbar \gamma^\mu u)^\dagger &= \ubar \gamma^\mu v \\
(\ubar \gamma^\mu v)^\dagger &= \vbar \gamma^\mu u.
\end{aligned}
\end{equation*}
} % lemma
The proof is left to \cref{problem:qftLecture23:1}.  Employing this, we have
\begin{dmath}\label{eqn:qftLecture23:600}
\Abs{M}^2
=
\frac{e^4}{q^4}
\lr{
\vbar^{r'}(k')
\gamma_\rho
u^r(k)
\ubar^s(p)
\gamma^\rho
v^{s'}(p')
}
\times
\lr{
\vbar^{s'}(p') \gamma^\mu u^s(p)
\ubar^r(k) \gamma_\mu v^{r'}(k')
}.
\end{dmath}

The problem can be simplified by computing the cross section that sums over all spins, assuming that the states are not polarized (i.e. average over all the up, down states).
\makedigression{
Such an average is related to the density matrix
\begin{dmath}\label{eqn:qftLecture23:620}
\rho_{\text{in}} = \sum_{s s'} \ket{s s'} \inv{4} \bra{ s s'}.
\end{dmath}
\begin{dmath}\label{eqn:qftLecture23:640}
\trace{
\lr{
   e^{i H t} \rho_{\text{in}} e^{i H t} \rho_\txtf
   \ket{r r'} \bra{r r'}
}
}
\end{dmath}
} % digression
That is, We want to sum over all the initial and final state polarizations \( \inv{4} \sum_{ss'} \sum_{rr'} \Abs{M}^2 \)

\begin{dmath}\label{eqn:qftLecture23:610}
\inv{4} \sum_{ss', rr'} \Abs{M}^2
=
\sum_{s s' r r'}
\frac{e^4}{4 q^4}
\vbar^{r'}(k')
\gamma_\rho
u^r(k)
\ubar^r(k)
\gamma_\mu
v^{r'}(k')
\ubar^s(p)
\gamma^\rho
v^{s'}(p')
\vbar^{s'}(p')
\gamma^\mu
u^s(p)
=
\frac{e^4}{4 q^4}
\sum_{r'}
\vbar^{r'}(k')
\gamma_\rho
\lr{ \slashed{k} + m_\mu }
\gamma_\mu
v^{r'}(k')
\times
\sum_{s}
\ubar^s(p)
\gamma^\rho
\lr{ \slashed{p'} - m_e }
\gamma^\mu
u^s(p),
\end{dmath}
where we first used the freedom to move the \( \ubar \gamma v, \vbar \gamma u \) terms, which are scalars, and then used
\cref{thm:qftLecture21:17} to eliminate the sum over \( s', r \) indexes.

Temporarily expressing the remaining factors in coordinates exposes a trace structure.  For example
\begin{dmath}\label{eqn:qftLecture23:660}
\sum_{r'}
\vbar^{r'}(k')
\gamma_\rho
\lr{ \slashed{k} + m_\mu }
\gamma_\mu
v^{r'}(k')
=
\sum_{r'}
(\vbar^{r'}(k'))_a
(\gamma_\rho)_{ab}
\lr{ \slashed{k} + m_\mu }_{bc}
(\gamma_\mu)_{cd}
(v^{r'}(k'))_d
=
\sum_{r'}
(v^{r'}(k'))_d
(\vbar^{r'}(k'))_a
(\gamma_\rho)_{ab}
\lr{ \slashed{k} + m_\mu }_{bc}
(\gamma_\mu)_{cd}
=
\lr{ \slashed{k'} - m_\mu }_{da}
(\gamma_\rho)_{ab}
\lr{ \slashed{k} + m_\mu }_{bc}
(\gamma_\mu)_{cd}
=
\trace{
\lr{
\lr{ \slashed{k'} - m_\mu }
\gamma_\rho
\lr{ \slashed{k} + m_\mu }
\gamma_\mu
}},
\end{dmath}
since the cyclic sum of matrix coordinates can be expressed as a trace, namely \( \trace{A B C} = A_{ab} B_{bc} C_{ca} \).  We are left with
\begin{dmath}\label{eqn:qftLecture23:680}
\inv{4} \sum_{ss', rr'} \Abs{M}^2
=
\frac{e^4}{4 q^4}
\trace{ \lr{
\lr{ \slashed{k'} - m_\mu }
\gamma_\nu
\lr{ \slashed{k} + m_\mu }
\gamma_\mu
}}
\times
\trace{ \lr{
\lr{ \slashed{p} + m_e }
\gamma^\nu
\lr{ \slashed{p'} - m_e }
\gamma^\mu
}}.
\end{dmath}
Each trace is now a product of two, three, or four gamma matrices, which can be reduced using the identities:
\makelemma{Dirac matrix product traces.}{lemma:qftLecture23:700}{
\begin{equation*}
\begin{aligned}
\trace{\lr{ \gamma_\mu \gamma_\nu }} &= 4 g_{\mu\nu} \\
\trace{\lr{ \gamma_\mu \gamma_\nu \gamma_\alpha }} &= 0 \\
\trace{\lr{ \gamma_\mu \gamma_\nu \gamma_\alpha \gamma_\beta }} &= 4 \lr{
   g_{\mu\nu} g_{\alpha\beta}
   -
   g_{\mu\alpha} g_{\nu\beta}
   +
   g_{\mu\beta} g_{\alpha\nu}
}
\end{aligned}
\end{equation*}
} % lemma
The proof is left to \cref{problem:qftLecture23:2}.

Utilizing the above, and setting \( m_e = 0 \) (compared to \(m_\mu\))
the \( p, p' \) dependent trace reduces to
\begin{dmath}\label{eqn:qftLecture23:780}
\trace{\lr{
(\slashed{p} + m_e) \gamma^\nu (\slashed{p'} - m_e) \gamma^\mu
}}
=
\trace{\lr{
\slashed{p} \gamma^\nu \slashed{p'} \gamma^\mu
}}
=
p_\alpha {p'}_\beta
\trace{\lr{
\gamma^\alpha
\gamma^\nu
\gamma^\beta
\gamma^\mu
}}
=
4 p_\alpha {p'}_\beta
\lr{
g^{\alpha\nu}
g^{\beta\mu}
-
g^{\alpha\beta}
g^{\nu\mu}
+
g^{\alpha\mu}
g^{\nu\beta}
}
=
4 \lr{
-
p \cdot p'
g^{\nu\mu}
+
p^\nu {p'}^\mu
+
p^\mu {p'}^\nu
},
\end{dmath}
and the \( k, k' \) dependent trace reduces to
\begin{dmath}\label{eqn:qftLecture23:800}
\trace{ \lr{
\lr{ \slashed{k'} - m_\mu }
\gamma_\nu
\lr{ \slashed{k} + m_\mu }
\gamma_\mu
}}
=
\trace{ \lr{
\slashed{k'}
\gamma_\nu
\slashed{k}
\gamma_\mu
}}
-
m_\mu^2
\trace{ \lr{
\gamma_\nu
\gamma_\mu
}}
+
m_\mu
\cancel{
\trace{ \lr{
\slashed{k'}
\gamma_\nu
\gamma_\mu
}}}
-
m_\mu
\cancel{
\trace{ \lr{
\gamma_\nu
\slashed{k}
\gamma_\mu
}}}
=
4\lr{
   {k'}_\alpha
   {k}_\beta
   \lr{
      g_{\alpha\nu} g_{\beta\mu}
      -
      g_{\alpha\beta} g_{\nu\mu}
      +
      g_{\alpha\mu} g_{\nu\beta}
   }
   -
   m_\mu^2 g_{\nu\mu}
}
=
4\lr{
   {k'}_\nu
   {k}_\mu
   +
   {k'}_\mu
   {k}_\nu
   -
   \lr{
   k \cdot k'
   +
      m_\mu^2
   }
   g_{\nu\mu}
}.
\end{dmath}

We can now multiply out the traces and simplify (\cref{problem:psq:1}) to get
\begin{dmath}\label{eqn:qftLecture23:820}
\inv{4} \sum_{\text{spins}} \Abs{M}^2
=
\frac{8 e^4}{q^4}
\lr{
    p \cdot k' p' \cdot k
+   p \cdot k p' \cdot k'
+   p \cdot p' m_\mu^2
}.
\end{dmath}
The next task is to consider these four vector dot products from the center of mass frame for the electrons, as sketched in
\cref{fig:l23:l23Fig6}.
\imageFigure{../figures/phy2403-quantum-field-theory/l23Fig6}{Electron center of mass frame.}{fig:l23:l23Fig6}{0.3}
Let \( q \) represent the total rest frame four momentum
\begin{dmath}\label{eqn:qftLecture23:840}
q = p + p' = (2 E, \Bzero),
\end{dmath}
where \( q^2 = 4 E^2 \).
We also have
\begin{subequations}
\label{eqn:qftLecture23:960}
\begin{dmath}\label{eqn:qftLecture23:860}
p \cdot p'
= \lr{ E, E \zcap} \cdot \lr{ E, -E \zcap }
=
E^2 - E^2 (\zcap \cdot (-\zcap))
=
2 E^2.
\end{dmath}
\begin{dmath}\label{eqn:qftLecture23:880}
p \cdot k
= (E, E \zcap) \cdot (E, \Bk)
=
E^2 - E \Norm{\Bk} \cos\theta,
\end{dmath}
\begin{dmath}\label{eqn:qftLecture23:900}
p \cdot k'
= (E, E \zcap) \cdot (E, -\Bk)
=
E^2 - (E \zcap) \cdot( -\Bk)
=
E^2 + E \Norm{\Bk} \cos\theta
\end{dmath}
\begin{dmath}\label{eqn:qftLecture23:920}
p' \cdot k'
=
(E, -E \zcap) \cdot (E, -\Bk)
=
E^2 - (-E \zcap) \cdot( -\Bk)
=
E^2 - E \Norm{\Bk} \cos\theta
\end{dmath}
\begin{dmath}\label{eqn:qftLecture23:940}
p' \cdot k
=
(E, -E \zcap) \cdot (E, \Bk)
=
E^2 -(-E \zcap) \cdot \Bk
=
E^2 + E \Norm{\Bk} \cos\theta,
\end{dmath}
\end{subequations}
but
\begin{dmath}\label{eqn:qftLecture23:980}
\Bk^2 = E^2 - m_\mu^2,
\end{dmath}
or
\begin{dmath}\label{eqn:qftLecture23:1000}
\Norm{\Bk} = E \sqrt{ 1 - \frac{m_\mu^2}{E^2} }.
\end{dmath}
We can now put the pieces back together and almost have the non-polarized cross section
\begin{dmath}\label{eqn:qftLecture23:1020}
\inv{4} \sum_{\text{spins}} \Abs{M}^2
=
\frac{ 8 e^4}{ (4 E^2)^2 }
\lr{
   \lr{
      E^2 + E \Norm{\Bk} \cos\theta
   }^2
   +
   \lr{
      E^2 - E \Norm{\Bk} \cos\theta
   }^2
   +
   m_\mu^2
   2 E^2
}
=
\frac{e^4}{2}
\lr{
   \lr{
      1 + \sqrt{ 1 - \frac{m_\mu^2}{E^2} } \cos\theta
   }^2
   +
   \lr{
      1 - \sqrt{ 1 - \frac{m_\mu^2}{E^2} } \cos\theta
   }^2
   +
   2 \frac{m_\mu^2}{E^2}
}
=
\frac{e^4}{2}
\lr{
   2 + 2 \lr{ 1 - \frac{m_\mu^2}{E^2} } \cos^2\theta
   +
   2 \frac{m_\mu^2}{E^2}
},
\end{dmath}
or
\boxedEquation{eqn:qftLecture23:1040}{
\inv{4} \sum_{\text{spins}} \Abs{M}^2
=
e^4 \lr{
   1 + \frac{m_\mu^2}{E^2}
   + \lr{ 1 - \frac{m_\mu^2}{E^2} } \cos^2\theta
}.
}
The total (average polarization) differential cross section (\citep{peskin1995introduction} eq. 4.84), is
\begin{dmath}\label{eqn:qftLecture23:1060}
{\frac{d\sigma}{d\Omega}}_{\text{CM}}
=
\inv{ 2 E_A 2 E_B \Abs{v_A - v_B} } \frac{\Abs{\Bk}}{(2 \pi)^2 4 E_{\text{CM}} } \inv{4} \sum_{\text{spins}} \Abs{M}^2.
\end{dmath}
Plug in \( E_A = E_B = 2 E_{\text{CM}} \), \( v_A - v_B \sim 2 c = 2 \), \( e^2 = 4 \pi \alpha \), and \cref{eqn:qftLecture23:1040} for
\begin{dmath}\label{eqn:qftLecture23:1080}
{\frac{d\sigma}{d\Omega}}_{\text{CM}}
=
\inv{ E_{\text{CM}}^2 (2) } \inv{(4 \pi)^2 E_{\text{CM}}} \frac{E_{\text{CM}}}{2}
\sqrt{ 1 -  \frac{m_\mu^2}{E^2} }
(4 \pi \alpha)^2
\lr{
   1 + \frac{m_\mu^2}{E^2}
   + \lr{ 1 - \frac{m_\mu^2}{E^2} } \cos^2\theta
}
=
\frac{\alpha^2}{4 E_{\text{CM}}^2 }
\sqrt{ 1 -  \frac{m_\mu^2}{E^2} }
\lr{
   1 + \frac{m_\mu^2}{E^2}
   + \lr{ 1 - \frac{m_\mu^2}{E^2} } \cos^2\theta
}.
\end{dmath}
Integrating to find the total cross section we have
\begin{dmath}\label{eqn:qftLecture23:1100}
\sigma_{\text{total}}
= \int d \Omega \frac{d\sigma}{d\Omega}
= 2 \pi \int_{-1}^1 d \cos\theta
\frac{\alpha^2}{4 E_{\text{CM}}^2 }
\sqrt{ 1 -  \frac{m_\mu^2}{E^2} }
\lr{
   1 + \frac{m_\mu^2}{E^2}
   + \lr{ 1 - \frac{m_\mu^2}{E^2} } \cos^2\theta
}
=
\frac{2 \pi \alpha^2}{4 E_{\text{CM}}^2 }
\sqrt{ 1 -  \frac{m_\mu^2}{E^2} }
\lr{
   2\lr{ 1 + \frac{m_\mu^2}{E^2} }
+ \lr{ 1 - \frac{m_\mu^2}{E^2} } \int_{-1}^1 u^2 du
}
=
\frac{4 \pi \alpha^2}{4 E_{\text{CM}}^2 }
\sqrt{ 1 -  \frac{m_\mu^2}{E^2} }
\lr{
   1 + \frac{m_\mu^2}{E^2}
+\inv{3} \lr{ 1 - \frac{m_\mu^2}{E^2} }
},
\end{dmath}
or
\boxedEquation{eqn:qftLecture23:1120}{
\sigma_{\text{total}}
=
\frac{4 \pi \alpha^2}{3 E_{\text{CM}}^2 }
\sqrt{ 1 -  \frac{m_\mu^2}{E^2} }
\lr{
   1 + \inv{2} \frac{m_\mu^2}{E^2}
},
}
where \( E_{\text{CM}} = 2 E \).

At the start of the year dimensional analysis was used to state the total cross section, which was determined to have the form
\begin{dmath}\label{eqn:qftLecture23:1140}
\sigma_{\text{total}} \sim \frac{\alpha^2}{s},
\end{dmath}
whereas for \( E \gg m_\mu \) we've now found
\begin{dmath}\label{eqn:qftLecture23:1160}
\sigma_{\text{total}}
=
\frac{4 \pi \alpha^2}{3 E_{\text{CM}}^2 }.
\end{dmath}
Three months of work has gained us an additional factor of \( 4/3 \)!

\section{Measurement of intermediate quark scattering processes.}
In the diagram that we are working from for the \( e^{-} e^{+} \rightarrow \mu^{-} \mu^{+} \) process, we can replace the muon half of the interaction (\cref{fig:l23:l23Fig7})
\imageFigure{../figures/phy2403-quantum-field-theory/l23Fig7}{Electron and muon halves of the diagram}{fig:l23:l23Fig7}{0.3}
with anything else that is charged, as sketched in
\cref{fig:l23:l23Fig8}.
\imageFigure{../figures/phy2403-quantum-field-theory/l23Fig8}{Alternate charged pair production.}{fig:l23:l23Fig8}{0.3}
In particular, quark pairs from QCD are possible at high energies (\(m_\mu \sim 105 \,\si{MeV} \)) and such products can be measured indirectly.
Quarks were the theorized to be strong force carriers, an intermediate stage similar to the photon propagators of QED, connecting two branches of a diagram, as sketched in
\cref{fig:l23:l23Fig9}.
\imageFigure{../figures/phy2403-quantum-field-theory/l23Fig9}{Quark pair production.}{fig:l23:l23Fig9}{0.3}
If one hypothesizes a proportionality relationship between the hadron (i.e. muon) and quark scattering cross sections
\begin{equation}\label{eqn:qftLecture23:1180}
\sigma_{\text{total}}(  e^{-} e^{+} \rightarrow \text{hadrons} ) \propto
\sigma_{\text{total}}(  e^{-} e^{+} \rightarrow \text{quarks} ),
\end{equation}
the ratio between the two
\begin{dmath}\label{eqn:qftLecture23:1200}
R =
\frac{
\sigma_{\text{total}}(  e^{-} e^{+} \rightarrow \text{quarks} ) }
{
\sigma_{\text{total}}(  e^{-} e^{+} \rightarrow \text{hadrons} )
}
= 3 \sum_q (Q_q)^4,
\end{dmath}
can be measured, and such measurement was deemed to be one of the validations of the QCD theory.  The \( 3 \sum_q (Q_q)^4 \) expression includes a 3 that is related to quark ``color'', and a sum over only the quark charges \( q \) that are light enough to be produced.  \citep{peskin1995introduction} fig. 5.3 includes an experimental depiction of such a measurement, which has a step function form roughly like
\cref{fig:l23:l23Fig10}, where the steps occur at the energy levels that are sufficient to produce new quarks.
\imageFigure{../figures/phy2403-quantum-field-theory/l23Fig10}{\( R \) quark step function.}{fig:l23:l23Fig10}{0.3}

%}
%\EndArticle
