%
% Copyright � 2017 Peeter Joot.  All Rights Reserved.
% Licenced as described in the file LICENSE under the root directory of this GIT repository.
%
%{
\input{../latex/blogpost.tex}
\renewcommand{\basename}{qftLecture23}
\renewcommand{\dirname}{notes/phy2403/}
\newcommand{\keywords}{PHY2403H}
\input{../latex/peeter_prologue_print2.tex}

%\usepackage{phy2403}
\usepackage{peeters_braket}
%\usepackage{peeters_layout_exercise}
\usepackage{peeters_figures}
\usepackage{mathtools}
\usepackage{siunitx}
\usepackage{macros_cal} % LL

\newcommand{\ultensor}[3]{{{#1}^{#2}}_{#3}}
\newcommand{\oPsi}[0]{\overbar{\Psi}}
\newcommand{\osigma}[0]{\overbar{\sigma}}
\newcommand{\ubar}[0]{\overbar{u}}
\newcommand{\vbar}[0]{\overbar{v}}
\newcommand{\deltathree}[0]{\delta^{(3)}}
\newcommand{\deltafour}[0]{\delta^{(4)}}
\newcommand{\ITwo}[0]{{\begin{bmatrix} 1 & 0 \\ 0 & 1 \end{bmatrix}}}
\newcommand{\DiracGammaZero}[0]{{\begin{bmatrix} 0 & 1 \\ 1 & 0 \end{bmatrix}}}
\newcommand{\DiracGammaK}[1]{{\begin{bmatrix} 0 & \sigma^{#1} \\ -\sigma^{#1} & 0 \end{bmatrix}}}

\beginArtNoToc
\generatetitle{PHY2403H Quantum Field Theory.  Lecture 23: XXX.  Taught by Prof.\ Erich Poppitz}
%\chapter{XXX}
\label{chap:qftLecture23}

%%Peeter's lecture notes from class.  These may be incoherent and rough.
%%
%%These are notes for the UofT course PHY2403H, Quantum Field Theory, taught by Prof. Erich Poppitz, covering \textchapref{{1}} \citep{peskin1995introduction} content.

\paragraph{DISCLAIMER: Very rough notes from class, with some additional side notes.}

These are notes for the UofT course PHY2403H, Quantum Field Theory, taught by Prof. Erich Poppitz, fall 2018.
%, covering \textchapref{{1}} \citep{peskin1995introduction} content.

\section{Review}

\begin{dmath}\label{eqn:qftLecture23:140}
\LL_{\text{Dirac}} = \oPsi \lr{ i \gamma^\mu \partial_\mu -m } \Psi,
\end{dmath}
\begin{dmath}\label{eqn:qftLecture23:160}
\Psi(x) =
\sum_{s = 1}^2
\int \frac{d^3 p}{(2 \pi)^3 \sqrt{2 \omega_\Bp}}
\lr{
   e^{-i p \cdot x} u^s(p) a_\Bp^s
   +
   e^{i p \cdot x} v^s(p) a_\Bp^{s \dagger}
}
\end{dmath}
\begin{dmath}\label{eqn:qftLecture23:180}
\oPsi(x) = \Psi^\dagger(x) \gamma^0 =
\sum_{s = 1}^2
\int \frac{d^3 p}{(2 \pi)^3 \sqrt{2 \omega_\Bp}}
\lr{
   e^{i p \cdot x} \ubar^s(p) a_\Bp^{s\dagger}
   +
   e^{-i p \cdot x} \vbar^s(p) a_\Bp^{s}
}
\end{dmath}

\begin{dmath}\label{eqn:qftLecture23:200}
\symmetric{a_\Bp^s}{a_\Bq^{r\dagger}} = (2 \pi)^3 \delta^{sr} \deltathree( \Bp - \Bq),
\end{dmath}

There are some symmetries
\begin{itemize}
\item \( SO(1,3) \)
\item \( P, C, T \) : DIY
\item \( U(1)_V \) : \( \Psi \rightarrow e^{i \alpha} \Psi \)
\item \( U(1)_A \) : If \( m = 0 \), then \( U(1)_A : \Psi \rightarrow e^{i \alpha \gamma_5 } \Psi \).  If \( m \ne 0 \) only for \( \alpha = \pi \) : \( \Psi \rightarrow -\Psi \).
\end{itemize}

Now introduce interaction with photon by using a \( U(1) \) gauge field, and demand invariance under \( U(1)_V \) with \( \alpha = \alpha(x) \).  Now
\begin{dmath}\label{eqn:qftLecture23:20}
\Psi(x) \rightarrow e^{i \alpha(x) } \Psi(x)
\end{dmath}

\begin{dmath}\label{eqn:qftLecture23:280}
\partial_\mu \Psi(x) \rightarrow
e^{i \alpha(x) } \lr{ \partial_\mu \Psi(x) + i \partial_\mu \alpha(x) } \Psi
\end{dmath}

Solution.  Introduce \( A_\mu(x) \), such that under \( U(1)_V \) we have

\begin{dmath}\label{eqn:qftLecture23:300}
A_\mu(x) \rightarrow A_\mu(x) - \inv{e} \partial_\mu \alpha(x)
\end{dmath}

where ``e'' is a dimensionless coupling constant

\begin{dmath}\label{eqn:qftLecture23:320}
\partial_\mu \Psi(x) \rightarrow
\lr{ \partial_\mu + i e A_\mu } \Psi
\rightarrow
e^{i \alpha(x} \lr{ \partial_\mu \Psi + \cancel{ i \partial_\mu \alpha} - \cancel{ i \partial_\mu \alpha } ... }
\end{dmath}

We've now constructed the QED Lagrangian density
\begin{dmath}\label{eqn:qftLecture23:40}
\LL_{\text{QED}} = \oPsi \lr{ i \gamma^\mu \lr{ \partial_\mu + i e A_\mu } - m } \Psi - \inv{4} F_{\mu\nu} F^{\mu\nu}.
\end{dmath}
%This is Gauged \( U(1)_V \) symmetry.

We may write this as
\begin{dmath}\label{eqn:qftLecture23:60}
\LL_{\text{QED}} =
\underbrace{- \inv{4} F_{\mu\nu} F^{\mu\nu}
+
\oPsi \lr{ i \gamma^\mu \partial_\mu - m } \Psi
}_{\text{Free Lagrangian}}
-
\underbrace{e \oPsi \gamma_\mu \Psi A^\mu }_{\text{interaction Lagrangian}}
\end{dmath}

We introduce spinor fields \( \Psi_e \) and muon fields \( \Psi_\mu \), so that the total Lagrangian is now
\begin{dmath}\label{eqn:qftLecture23:80}
\LL_{\text{QED}}
=
- \inv{4} F_{\mu\nu} F^{\mu\nu}
+
\oPsi_e \lr{ i \gamma^\mu \partial_\mu - m } \Psi_e
-
e \oPsi_e \gamma_\mu \Psi_e A^\mu
+
\oPsi_\mu \lr{ i \gamma^\mu \partial_\mu - m } \Psi_\mu
-
e \oPsi_\mu \gamma_\mu \Psi_\mu A^\mu
\end{dmath}

\begin{itemize}
\item \( m_e \sim 0.5 \,\si{MeV} \)
\item \( m_\mu \sim 105 \,\si{MeV} \)
\end{itemize}

There are also quark fields that we can add into the mix
\begin{dmath}\label{eqn:qftLecture23:100}
\LL_{\text{quarks}} = \sum_q \oPsi_q \lr{ i \gamma^\mu - m_q } \Psi_q + e Q_q \oPsi_q \gamma^\nu \Psi_q A_\nu
\end{dmath}
Quark charges are \( Q_q = (2/3, -1/3) \).
It turns out that the only way to produce quarks is through (electron?) interaction?

Can also introduce a Fermi interaction

\begin{dmath}\label{eqn:qftLecture23:120}
\LL_{4-Fermi} = \frac{c}{v^2} \oPsi_\mu \gamma^\nu \lr{ 1 - \gamma_5 } \Psi_{\nu, \mu} - \oPsi_e \lr{ 1 - \gamma_5 } ....
\end{dmath}

We are now going to do some calculations with \cref{eqn:qftLecture23:80}

In that interaction the \( -e \oPsi_e \gamma_\mu \Psi_e A^\mu \) can create and anhillate (what?)

F1
F2
F3

For Grassman (anti-commuting) operators

\begin{dmath}\label{eqn:qftLecture23:220}
T( O_f(x) O'_f(x)) =
\Theta ( x_0 - x_0' ) O_f(x) O_f(x')
+
\Theta ( x_0' - x_0 ) O_f(x') O_f(x)
\end{dmath}

\begin{dmath}\label{eqn:qftLecture23:240}
\expectation{ T( \Psi_\alpha(x) \Psi_\beta(x) }_0 = D_{F_{\alpha \beta}}(x - y),
\end{dmath}
where \( \alpha, \beta = 1, 2, 3, 4 \).

Referring back to
%\begin{dmath}\label{eqn:qftLecture23:160}
%\begin{dmath}\label{eqn:qftLecture23:180}

\begin{dmath}\label{eqn:qftLecture23:260}
\expectation{ T( \Psi_\alpha(x) \Psi_\beta(x) }_0
=
\int \frac{d^3 p}{(2\pi)^3 \sqrt{ 2 \omega_\Bp } }
\int \frac{d^3 q}{(2\pi)^3 \sqrt{ 2 \omega_\Bq } }
\lr{
   e^{-i p \cdot x} e^{+ i q \cdot y}
   \Theta ( x_0 - y_0 ) u^s_\alpha(p) \ubar^r_\beta(q) \expectation{ a_\Bp^s a_\Bq^{r\dagger} }
+
   e^{i p \cdot x} e^{- i q \cdot y}
   \Theta ( y_0 - x_0 ) \vbar^s_\beta(p) v^r_\beta(q) \expectation{ b_\Bq^s a_\Bp^{r\dagger} }
}
=
\int \frac{d^3 p}{(2\pi)^3 2 \omega_\Bp }
\lr{
   e^{-i p \cdot (x-y)}
   \Theta ( x_0 - y_0 ) u^s_\alpha(p) \ubar^r_\beta(p)
+
   e^{i p \cdot (x- y)}
   \Theta ( y_0 - x_0 ) \vbar^s_\beta(p) v^r_\beta(p)
}
=
\int \frac{d^3 p}{(2\pi)^3 2 \omega_\Bp }
\lr{
   e^{-i p \cdot x}
   \Theta ( x_0 - y_0 )
\lr{ \gamma^\mu_{\alpha\beta} p_\mu + m}
+
   e^{i p \cdot x}
   \Theta ( y_0 - x_0 )
\lr{ \gamma^\mu_{\alpha\beta} p_\mu - m}
}
=
\end{dmath}
where \( \gamma^\mu_{\alpha\beta} \) are the \( \alpha, \beta \) components of the gamma matrices.
Now we can replace the \( p_\mu \)'s with derivatives acting on the exponentials

\begin{dmath}\label{eqn:qftLecture23:340}
\expectation{ T( \Psi_\alpha(x) \Psi_\beta(x) }_0
=
\Theta ( x_0 - y_0 )
\lr{ i \gamma^\mu_{\alpha\beta} \partial_\mu + m }
\int \frac{d^3 p}{(2\pi)^3 2 \omega_\Bp } e^{-i p \cdot (x - y)}
-
\Theta ( y_0 - x_0 )
\lr{ -i \gamma^\mu_{\alpha\beta} \partial_\mu - m }
\int \frac{d^3 p}{(2\pi)^3 2 \omega_\Bp } e^{-i p \cdot (x - y)}
=
\Theta ( x_0 - y_0 )
\lr{ i \gamma^\mu_{\alpha\beta} \partial_\mu + m }
D(x - y)
-
\Theta ( y_0 - x_0 )
\lr{ -i \gamma^\mu_{\alpha\beta} \partial_\mu - m }
D(y - x)
=
\lr{ \gamma^\mu_{\alpha\beta} \partial_\mu^{(x)} + m }
\lr{
   \Theta ( x_0 - y_0 )
   D(x - y)
   +
   \Theta ( y_0 - x_0 )
   D(y - x)
}
-
i \gamma^0 \delta(x^0 - y^0) \cancel{\lr{ D(x - y) - D(y - x) }},
\end{dmath}
where we've killed off a factor that is zero (off the light cone?)

We are left with just an action on the Feynman propagator
\begin{dmath}\label{eqn:qftLecture23:360}
\expectation{ T( \Psi_\alpha(x) \Psi_\beta(x) }_0
=
\lr{ \gamma^\mu_{\alpha\beta} \partial_\mu^{(x)} + m } D_F(x - y)
=
\int \frac{d^4 p}{(2 \pi)^4 } \frac{ i ( \gamma^\mu_{\alpha\beta} p_\mu + m ) }{p^2 - m^2 + i \epsilon} e^{-i p \cdot (x - y)}
\end{dmath}

Now that we have a propagator, let's try

\begin{dmath}\label{eqn:qftLecture23:380}
\LL_{\text{int}} = \int dt d^3 x \lr{ e \oPsi \gamma_\mu \Psi A^\mu }
\end{dmath}

%}
%\EndArticle
\EndNoBibArticle
