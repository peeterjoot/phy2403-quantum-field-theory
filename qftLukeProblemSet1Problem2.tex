%
% Copyright � 2015 Peeter Joot.  All Rights Reserved.
% Licenced as described in the file LICENSE under the root directory of this GIT repository.
%
\makeoproblem{Coherent states.}
{qft:LukeProblemSet1:2}
{2015 ps1.2}
{

In a theory of a single harmonic oscillator, define the coherent state \( \ket{ z } \) by

\begin{dmath}\label{eqn:LukeProblemSet1Problem2:20}
\ket{z} = N e^{z a^\dagger} \ket{0}
\end{dmath}

where \( z \) is a complex number and \( N \) is a real positive constant, chosen such that \( \braket{ z}{z} = 1\).
Coherent
states of the SHO are interesting because they smoothly interpolate between the classical and quantum
worlds: for large z they become indistinguishable from classical oscillators. (Similarly, coherent states
of photons correspond to electromagnetic waves in the limit of large numbers of photons). They also
give you good practice at manipulating creation and annihilation operators.
As usual, \( H = \omega (p^2 +q^2 )/2 \) and the raising and lowering operators \( a \) and \( a^\dagger \)are defined as \( a = (q + i p)/\sqrt{2} \), \( a^\dagger = (q - i p)/\sqrt{2}\),
where the usual momentum \( P \) and position \( X \) are \( P = \sqrt{ \mu \omega } p \), \( X = q/\sqrt{\mu \omega} \).
\makesubproblem{}{qft:LukeProblemSet1:2a}
Find N.
\makesubproblem{}{qft:LukeProblemSet1:2b}
Compute \( \braket{z'}{z} \), and \( \bra{ z } H \ket{z} \).
\makesubproblem{}{qft:LukeProblemSet1:2c}
Show that \( \ket{z} \) is an eigenstate of the annihilation operator \( a \) and find its eigenvalue. (Don't be disturbed by finding non-orthogonal eigenstates with complex eigenvalues; \( a \) is not a Hermitian
operator.)
\makesubproblem{}{qft:LukeProblemSet1:2d}
The statement that \( \ket{ z } \) is an eigenstate of a with well-known eigenvalue is, in the \(q\)-representation, a first-order differential equation for \( \braket{q}{z} \), the position-space wave-function of \( \ket{z} \). Solve this equation and find and sketch the wave-function. (Don't bother with normalization factors here).
\makesubproblem{}{qft:LukeProblemSet1:2e}
Consider the time evolution of the system (work in the Heisenberg representation). Show that for
real \( z \) (this just sets the initial conditions) the expectation values of the position and momentum
of the coherent state satisfy

\begin{dmath}\label{eqn:LukeProblemSet1Problem2:40}
\bra{z} X \ket{z} = \sqrt{\frac{2}{\mu\omega}} z \cos\omega t
\end{dmath}
\begin{dmath}\label{eqn:LukeProblemSet1Problem2:60}
\bra{z} P \ket{z} = -\sqrt{2 \mu\omega} z \sin\omega t
\end{dmath}

By contrast, what are the expectation values of \( X \) and \( P \) for an oscillator in any state of definite excitation number \( n \)? Using a sketch, describe the behavior of the wavepacket as a function of
time.
} % makeproblem

\makeanswer{qft:LukeProblemSet1:2}{
\withproblemsetsParagraph{
\makeSubAnswer{}{qft:LukeProblemSet1:2a}

Expanding this definition of \( \ket{z} \) in power series

\begin{dmath}\label{eqn:qftProblemSet1Problem2:80}
\ket{z} = N \sum_{k= 0}^\infty \inv{k!} \lr{ z a^\dagger }^k \ket{k}
\end{dmath}

but

\begin{equation}\label{eqn:qftProblemSet1Problem2:100}
\begin{aligned}
a^\dagger \ket{0} &= \sqrt{1} \ket{1} \\
(a^\dagger)^2 \ket{0} &= \sqrt{2 \times 1} \ket{2} \\
(a^\dagger)^3 \ket{0} &= \sqrt{3 \times 2 \times 1} \ket{3},
\end{aligned}
\end{equation}

or
\begin{equation}\label{eqn:qftProblemSet1Problem2:120}
(a^\dagger)^k = \sqrt{k!} \ket{k}.
\end{equation}

This gives
\begin{equation}\label{eqn:qftProblemSet1Problem2:140}
\ket{z} = N^2 \sum_{k= 0}^\infty \inv{\sqrt{k!}} z^k \ket{k},
\end{equation}

from which the braket can be computed
\begin{dmath}\label{eqn:qftProblemSet1Problem2:160}
1
= \braket{z}{z}
= N^2 \sum_{k,m = 0}^\infty \inv{\sqrt{k!}} (z^\conj)^k \bra{k} \inv{\sqrt{m!}} z^m \ket{m}
= N^2 \sum_{k = 0}^\infty \inv{k!} (z^\conj z)^k
= N^2 e^{\Abs{z}^2}.
\end{dmath}

This gives
\boxedEquation{eqn:qftProblemSet1Problem2:180}{
N = e^{-\Abs{z}^2/2}.
}

\makeSubAnswer{}{qft:LukeProblemSet1:2c}

\begin{dmath}\label{eqn:qftProblemSet1Problem2:260}
a \ket{z}
=
a N
\sum_{k = 0}^\infty \inv{\sqrt{k!}} z^k \ket{k}
=
N
\sum_{k = 1}^\infty \inv{\sqrt{k!}} z^k a \ket{k}
=
N
\sum_{k = 1}^\infty \inv{\sqrt{k!}} z^k \sqrt{k} \ket{k-1}
=
z N
\sum_{k = 1}^\infty \inv{\sqrt{(k-1)!}} z^{k-1} \ket{k-1}
=
z \ket{z}
\end{dmath}

\makeSubAnswer{}{qft:LukeProblemSet1:2b}

\begin{dmath}\label{eqn:qftProblemSet1Problem2:200}
\braket{z}{z'}
=
e^{-\Abs{z}^2/2 - \Abs{z'}^2/2 }
\sum_{k,m = 0}^\infty \inv{\sqrt{k!}} (z^\conj)^k \bra{k} \inv{\sqrt{m!}} {z'}^m \ket{m}
=
\exp\lr{
-\Abs{z}^2/2 - \Abs{z'}^2/2 + z^\conj z
}.
\end{dmath}

We also want to put the Hamiltonian into its number operator form by factoring it

\begin{dmath}\label{eqn:qftProblemSet1Problem2:220}
H
= \omega\lr{ p^2 + q^2 }/2
= \omega\lr{ \inv{2} \lr{ q - i p } \lr{ q + i p } - i \antisymmetric{q}{p}/2 }
= \omega\lr{ a^\dagger a + \inv{2} }
\end{dmath}

Having found that \( a \ket{z} = z \ket{z} \), we also have

\begin{dmath}\label{eqn:qftProblemSet1Problem2:280}
\bra{z} a^\dagger
=
\lr{ a \ket{z} }^\dagger
=
\lr{ z \ket{z} }^\dagger
=
\bra{z} z^\conj,
\end{dmath}

so
\begin{dmath}\label{eqn:qftProblemSet1Problem2:240}
\bra{z} H \ket{z}
=
\omega
\bra{z} a^\dagger a + \inv{2} \ket{z}
=
\omega
\lr{ \Abs{z}^2 + \inv{2} }.
\end{dmath}

\makeSubAnswer{}{qft:LukeProblemSet1:2d}

\begin{dmath}\label{eqn:qftProblemSet1Problem2:300}
\bra{q} a \ket{z}
=
\inv{\sqrt{2}} \bra{q} q + i p \ket{z}
=
\inv{\sqrt{2}} \lr{ q + \PD{q}{} } \braket{q}{z},
\end{dmath}

with \( \psi(q) = \braket{q}{z} \), this is

\begin{dmath}\label{eqn:qftProblemSet1Problem2:320}
\lr{ z - \frac{q}{\sqrt{2}} } \psi = \inv{\sqrt{2}} \PD{q}{\psi},
\end{dmath}

which separates into

\begin{dmath}\label{eqn:qftProblemSet1Problem2:340}
\lr{ \sqrt{2} z - q} dq = \frac{d\psi}{\psi}.
\end{dmath}

The solution is of the form

\begin{dmath}\label{eqn:qftProblemSet1Problem2:360}
\psi
\propto \exp\lr{ \sqrt{2} z q - q^2/2 }
= \exp\lr{ -(q^2 - 2 \sqrt{2} z q )/2 }
\propto \exp\lr{ -(q - \sqrt{2} z )^2/2 }.
\end{dmath}

SKETCH: This is a Gaussian, and when \( z \) is real is centred at \( \sqrt{2} z \).

\makeSubAnswer{}{qft:LukeProblemSet1:2e}

Noting that \( 2 q = \sqrt{2} \lr{ a + a^\dagger } \), and \( 2 i p = \sqrt{2}\lr{ a - a^\dagger } \)

\begin{dmath}\label{eqn:qftProblemSet1Problem2:380}
\bra{z} X \ket{z}
=
\inv{\sqrt{\mu \omega}}
\bra{z} q \ket{z}
=
\inv{\sqrt{2 \mu \omega}}
\bra{z}
a e^{-i\omega t} + a^\dagger e^{i \omega t}
\ket{z}
=
\inv{\sqrt{2 \mu \omega}} \lr{
z e^{-i\omega t} + z^\conj e^{i \omega t}
}
=
\sqrt{\frac{2}{\mu \omega}} z \cos(\omega t).
\end{dmath}

\begin{dmath}\label{eqn:qftProblemSet1Problem2:400}
\bra{z} P \ket{z}
=
\sqrt{\mu \omega}
\bra{z} p \ket{z}
=
i\frac{\sqrt{\mu \omega}}{2}
\bra{z}
a^\dagger e^{i\omega t} - a e^{-i \omega t}
\ket{z}
=
i\frac{\sqrt{\mu \omega}}{2} z
\lr{ e^{i\omega t} - e^{-i \omega t}}
=
-\sqrt{2 \mu \omega} z \sin(\omega t).
\end{dmath}

SKETCH: particle expectation values trace an ellipse in phase space.
}
}
