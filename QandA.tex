%
% Copyright � 2018 Peeter Joot.  All Rights Reserved.
% Licenced as described in the file LICENSE under the root directory of this GIT repository.
%
%{
\input{../latex/blogpost.tex}
\renewcommand{\basename}{facebook}
%\renewcommand{\dirname}{notes/phy1520/}
\renewcommand{\dirname}{notes/ece1228-electromagnetic-theory/}
%\newcommand{\dateintitle}{}
%\newcommand{\keywords}{}

\input{../latex/peeter_prologue_print2.tex}

\usepackage{peeters_layout_exercise}
\usepackage{peeters_braket}
\usepackage{peeters_figures}
\usepackage{siunitx}
\usepackage{verbatim}
%\usepackage{mhchem} % \ce{}
%\usepackage{macros_bm} % \bcM
%\usepackage{macros_qed} % \qedmarker
%\usepackage{txfonts} % \ointclockwise

\newcommand{\deltathree}[0]{\delta^{(3)}}
%\newcommand{\br3}[1]{{(3)}}
%\newcommand{\br4}[1]{{(4)}}

\beginArtNoToc

\generatetitle{Images for the group chat}
%\chapter{Images for the group chat}
%\label{chap:facebook}

%\begin{equation*}
%j^\mu = \trace{\lr{ \delta H^\dagger \partial^\mu H + \partial^\mu H^\dagger \delta H }}
%\end{equation*}
%
i.e.
\begin{equation*}
\begin{aligned}
H &= \inv{\sqrt{2}}
\begin{bmatrix}
i \sigma^2 \Phi^\conj & \Phi
\end{bmatrix} \\
H^\dagger &= \inv{\sqrt{2}}
\begin{bmatrix}
-i \Phi^\T \sigma^2 \\
\Phi^\dagger
\end{bmatrix}
\end{aligned}
\end{equation*}
so
\begin{dmath}\label{eqn:facebook:20}
j^{\mu, a}_L
=
\frac{i}{2}
   \trace{\lr{
- H^\dagger \sigma^a
\partial^\mu H
   +
   \partial^\mu H^\dagger
\sigma^a H
   }}
=
\frac{i}{4}
   \trace{\lr{
-
\begin{bmatrix}
-i \Phi^\T \sigma^2 \\
\Phi^\dagger
\end{bmatrix}
\sigma^a
\begin{bmatrix}
i \sigma^2 \partial^\mu \Phi^\conj & \partial^\mu \Phi
\end{bmatrix}
   +
\begin{bmatrix}
-i \partial^\mu \Phi^\T \sigma^2 \\
\partial^\mu \Phi^\dagger
\end{bmatrix}
\sigma^a
\begin{bmatrix}
i \sigma^2 \Phi^\conj & \Phi
\end{bmatrix}
  }}
=
\cdots
=
\frac{i}{2} \lr{ \partial^\mu \Phi^\dagger \sigma^a \Phi - \Phi^\dagger \sigma^a \partial^\mu \Phi}
\end{dmath}

Actually, it looks like I have one cross term after the expansion to second order

\begin{equation*}
i \lr{ \partial_\mu h \sigma^a \partial^\mu \phi^a
-
\sigma^a \partial^\mu \phi^a \partial_\mu h }
\end{equation*}

\section{blah}

The hint seems to suggest that particle number is conserved, even though the interaction term does not have the structure of a number operator.  I have to conclude (too late for problem set submission) that I don't really understand what is meant by preservation of particle number in this case, and will need to see the problem set solution or discuss this in office hours to understand what is being asked for.
%\makeSubAnswer{}{qft:problemSet2:4c}
If we designate an N-particle momentum state by
\begin{dmath}\label{eqn:ProblemSet2Problem4:180}
\ket{\Bp_1, \Bp_2, \cdots \Bp_N} =
a_{\Bp_1}^\dagger
a_{\Bp_2}^\dagger
\cdots
a_{\Bp_N}^\dagger \ket{0, 0, \cdots, 0},
\end{dmath}
then the interaction terms action on such a state is
\begin{dmath}\label{eqn:ProblemSet2Problem4:200}
      a_{\Bp_j}^\dagger a_{\Bp_m}^\dagger a_{\Bp_n} a_{{\Bp_j} + {\Bp_m} - {\Bp_n}}
\ket{\Bp_1, \Bp_2, \cdots \Bp_N}.
\end{dmath}
I'm not sure if this is meaningful, or how to interpret it, and think that I'm going to have to get explanation about what this abstraction means.  I'm also not sure what is meant by the question ``What kinds of scattering processes does it describe.''
%\makeSubAnswer{}{qft:problemSet2:4d}
Not attempted.

%\section{piazza q: Nov 4}
%
%In lecture 13, we evaluated the retarded time convolution for the Klein-Gordon equation, and found the non-homogeneous solution included terms like
%\begin{equation}\label{eqn:qftLecture12:460}
%i \int d^4 y
%   \int \frac{d^3 p}{(2\pi)^3 2 \omega_\Bp } e^{-i p \cdot (x - y)} j(y)
%=
%i \int \frac{d^3 p }{(2 \pi)^3}
%\inv{
%2 \omega_\Bp }
%\evalbar{
%   e^{-i p \cdot x} \tilde{j}(p)
%}{p_0 = \omega_\Bp},
%\end{equation}
%where
%\begin{equation}\label{eqn:qftLecture12:480}
%\tilde{j}(p) = \int d^4 y e^{i p \cdot y} j(y).
%\end{equation}
%
%Peskin and Schroeder also does this at the end of section 2.4 but they make a point of saying.
%
%}

\section{Q: l8}

I'm having sign troubles with problem IV.1, and think that's it's due to a sign error in the following derivation from lecture 8:
\begin{equation}\label{eqn:facebook:500}
\begin{aligned}
\hat{U}(\Ba) \hat{\phi}(\Bx) \hat{U}^\dagger(\Ba)
&= \hat{\phi}(\Bx) + i a^j (-i) \partial^j \hat{\phi}(\Bx) + \cdots \\
&= \hat{\phi}(\Bx) + a^j \partial^j \hat{\phi}(\Bx) + \cdots \\
&= \hat{\phi}(\Bx + \Ba).
\end{aligned}
\end{equation}
I think that this should be (adding a couple extra steps for clarity) :
\begin{equation}\label{eqn:facebook:520}
\begin{aligned}
\hat{U}(\Ba) \hat{\phi}(\Bx) \hat{U}^\dagger(\Ba)
&= \hat{\phi}(\Bx) + i a^j (-i) \partial^j \hat{\phi}(\Bx) + \cdots \\
&= \hat{\phi}(\Bx) + a^j \PD{x_j}{} \hat{\phi}(\Bx) + \cdots \\
&= \hat{\phi}(\Bx) - a^j \PD{x^j}{} \hat{\phi}(\Bx) + \cdots \\
&= \hat{\phi}(\Bx - \Ba).
\end{aligned}
\end{equation}
I'm not sure if I wrote this down wrong in during the lecture, but I think this correction is right.

Further confounding me was the fact that my old QM book used a different sign convention for the translation operator (
\citep{desai2009quantum} uses \( D(\Ba) = e^{-i \Ba \cdot \hat{\BP}/\Hbar} \) defined by the property \( D(\Ba) \ket{\Bx} = \ket{\Bx + \Ba} \), instead of \( \bra{\Bx + \Ba} = \bra{\Bx} U(\Ba) \) as we found.)

\section{blah}

\begin{dmath}\label{eqn:facebook:540}
%\delta^\brn{3}(x - y)
\deltathree(x - y)
\end{dmath}

\section{GA q.}

\begin{dmath}\label{eqn:QandA:560}
(\Ba\cross \Bb)\cdot(\Ba\cross \Bb)
=[-I(\Ba\wedge \Bb)]\cdot[-I(\Ba\wedge \Bb)]
=[I(\Ba\wedge \Bb)]\cdot[I(\Ba\wedge \Bb)],
\end{dmath}
\begin{dmath}\label{eqn:QandA:580}
(Ia)\cdot \Bb=I(\Ba\wedge \Bb)
=I[(\Ba\wedge \Bb)\wedge(I(\Ba\wedge \Bb))]
\Ba\wedge(Ib)=I(\Ba\cdot \Bb)
=I^2(\Ba\wedge \Bb)\cdot(\Ba\wedge \Bb)
=I^2(\Ba^2 \Bb^2-(\Ba\cdot \Bb)^2)
=-(\Ba^2 \Bb^2-(\Ba\cdot \Bb)^2)
\end{dmath}

I find it hard to follow what you have done.  This is how I'd show it.
\begin{dmath*}
\lr{ \Ba \cross \Bb } \cdot \lr{ \Ba \cross \Bb }
=
\gpgradezero{
\lr{ \Ba \cross \Bb } \lr{ \Ba \cross \Bb }
}
=
\gpgradezero{ -I(\Ba \wedge \Bb) (-I) (\Ba \wedge \Bb) }
=
-\gpgradezero{
(\Ba \wedge \Bb) (\Ba \wedge \Bb)
}
=
- \Ba \cdot \lr{ \Bb \cdot \lr{ \Ba \wedge \Bb } }
=
- \Ba \cdot \lr{ (\Bb \cdot \Ba) \Bb - \Bb^2 \Ba }
=
\Ba^2 \Bb^2 - (\Bb \cdot \Ba)^2.
\end{dmath*}

\section{l20}

\begin{equation*}
\begin{aligned}
p \cdot \sigma &= p^0 \sigma^0 - \Bp \cdot \Bsigma \\
p \cdot \overbar{\sigma} &= p^0 \sigma^0 + \Bp \cdot \Bsigma.
\end{aligned}
\end{equation*}

This makes sense to me, as it follows the usual pattern \( x \cdot p = x^\mu p_\mu = x^0 p^0 - \Bx \cdot \Bp \).

I ask because I got confused looking at Peskin and Schroeder, where they seemingly use an opposite sign convention where \( \sigma \cdot x \) was first defined, namely eq. 3.41 and its neighbours where they write
\begin{equation*}
\sigma \cdot \partial = \partial_0 + \Bsigma \cdot \spacegrad,
\end{equation*}
not
\begin{equation*}
\sigma \cdot \partial = \partial_0 - \Bsigma \cdot \spacegrad.
\end{equation*}

It took a bit of thought, but I think the explaination for this difference is because the coordinates of the four-gradient (\(\partial_\mu\)) are a lower index quantity, so in the hybrid scalar+vector tuple form, we should write \( \partial^\mu = (\partial^0, -\spacegrad) \), or \( \partial_\mu = (\partial_0, \spacegrad) \).

So, when Peskin and Schroeder apply this to \( \sigma \cdot \partial \), it really means
\begin{equation*}
\begin{aligned}
\sigma \cdot \partial
&= \sigma^0 \partial^0 - \Bsigma \cdot (-\spacegrad) \\
&= \partial_0 + \Bsigma \cdot \spacegrad
\end{aligned}
\end{equation*}

(Having a tuple notation that can be used to represent either lower or upper index quantities is very confusing.)

\section{normalizationQ}

I'd thought of this relativistic normalization as a mapping between three-vector and four-vector momentum states
\begin{dmath}\label{eqn:QandA:600}
\ket{p} = \sqrt{2 \Omega_\Bp} \ket{\Bp}.
\end{dmath}
This means that the non-relativisitic normalization of three-momentum states
\begin{dmath}\label{eqn:QandA:620}
\braket{\Bp}{\Bq} = (2 \pi)^3 \delta^{(3)}(\Bp - \Bq)
\end{dmath}
results in the desired Lorentz-invariant normalization of four-momentum states
\begin{dmath}\label{eqn:QandA:640}
\braket{p}{q} = (2 \pi)^3 2 \omega_\Bp \delta^{(3)}(\Bp - \Bq).
\end{dmath}

\section{scattering}

\begin{equation}\label{eqn:QandA:660}
\frac{\text{probability for \( i \rightarrow f\)}}{\text{unit time}}
\sim
(2 \pi)^4 \delta^{(4)}( p_{\text{in}} - \sum p_f )
V_3
\times \Abs{ M_{fi} }^2
\end{equation}

\begin{equation}\label{eqn:QandA:680}
\frac{\text{probability for \(i \rightarrow f\)}}{\text{unit time}}
\sim
\frac{
(2 \pi)^4 \delta^{(4)}( p_{\text{in}} - \sum p_f )
\Abs{ M_{fi} }^2 V_3
}
{
2 \omega_\Bk V_3
2 \omega_{\Bp_1}
\cdots
2 \omega_{\Bp_n} V_3^n
}
\end{equation}

\begin{dmath}\label{eqn:qftLecture17:380}
\braket{p}{p} = 2 \omega_\Bp V_3
\end{dmath}

\section{Connecting the algebra to interpretation: two point correlation function.}

This is going to sound like a really dumb question, because I didn't ask early enough, and now
my confusion is building on itself.

We came up with expressions for the two point correlation function
\begin{equation}\label{eqn:QandA:700}
 \bra{\Omega} T(\phi(x_1) \phi(x_2)) \ket{\Omega}
\end{equation}
and generalized that to many fields
\begin{equation}\label{eqn:QandA:720}
\begin{aligned}
 \bra{\Omega} & T(\phi(x_1) \cdots \phi(x_n)) \ket{\Omega} \\
&=
 \bra{0} T(\phi(x_1) \cdots \phi(x_n)) \ket{0}  \\
&-i \lambda
 \bra{0} T(\phi(x_1) \cdots \phi(x_n)) \int dt H_I \ket{0} \\
& + \cdots
\end{aligned}
\end{equation}
So far so good.  This was just algebra.  We then developed Wick's theorem so that we could evaluate these expressions.

I think that I was okay up to that point (especially after doing the HW4 problem on Wick's theorem).

However, it wasn't clear to me why we were looking at something like \( \bra{\Omega} T(\phi(x_1) \phi(x_2)) \ket{\Omega} \) in the first place.  Somehow, we seem to have ended up interpretting these time ordered matrix elements (or their diagramatic representations) as representations of processes that took an initial configuration of \( N' \) particles and mapped it to a final configuration that had \( M' \) particles.  Looking in retrospect, I really don't understand what magic hat that was pulled from.

For example, in (4.42), suddenly \( \bra{0} T(\phi(x_1) \phi(x_2) \phi(x_3) \phi(x_4)) \ket{0} \) has a diagramatic representation where they claim: "the diagrams do suggest an interpretation: Particles are create at two spacetime points, each propagates to one of the other points, and then they are annihilated."

\EndArticle
%\EndNoBibArticle
