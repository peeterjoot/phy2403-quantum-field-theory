%
% Copyright � 2018 Peeter Joot.  All Rights Reserved.
% Licenced as described in the file LICENSE under the root directory of this GIT repository.
%
%{
%%%\input{../latex/blogpost.tex}
%%%\renewcommand{\basename}{qftLecture11}
%%%\renewcommand{\dirname}{notes/phy2403/}
%%%\newcommand{\keywords}{PHY2403H}
%%%\input{../latex/peeter_prologue_print2.tex}
%%%
%%%%\usepackage{phy2403}
%%%\usepackage{peeters_braket}
%%%\usepackage{peeters_layout_exercise}
%%%\usepackage{peeters_figures}
%%%\usepackage{mathtools}
%%%\usepackage{siunitx}
%%%\usepackage{macros_cal} % LL
%%%
%%%\newcommand{\ultensor}[3]{{{#1}^{#2}}_{#3}}
%%%
%%%\beginArtNoToc
%%%\generatetitle{PHY2403H Quantum Field Theory.  Lecture 11: Momentum matrix elements, spacelike surfaces, microcausality, Lorentz invariant measure, wave function Green's function, retarded time contour, advanced time contour.  Taught by Prof.\ Erich Poppitz}
%\chapter{Momentum matrix elements, spacelike surfaces, microcausality, Lorentz invariant measure, wave function Green's function, retarded time contour, advanced time contour.}
%\chapter{Microcausality, Lorentz invariant measure, retarded time SHO Green's function.}
\index{microcausality}
\index{Lorentz invariant measure}
\index{Green's function!retarded time}
\label{chap:qftLecture11}

%%\paragraph{DISCLAIMER: Very rough notes from class, with some additional side notes.}
%%
%%These are notes for the UofT course PHY2403H, Quantum Field Theory I, taught by Prof. Erich Poppitz fall 2018.
%%%, covering \textchapref{{1}} \citep{peskin1995introduction} content.
%%
\section{Relativistic normalization.}

We will continue looking at the generator of spacetime translation \( \hatU(\Lambda) \), which has the property
\begin{equation}\label{eqn:qftLecture11:40}
\hatU(\Lambda) \ket{0} = \ket{0},
\end{equation}
%\underline{Lorentz invariance unbroken}
That is
\begin{equation}\label{eqn:qftLecture11:760}
\hatU(\Lambda) = \BOne + \text{operators that annihilate the vacuum state}.
\end{equation}

The action on a field was
\begin{equation}\label{eqn:qftLecture11:60}
\hatU(\Lambda)
\phihat(x) \hatU^\dagger(\Lambda)
= \phihat(\Lambda x),
\end{equation}
and the action on the annihilation operator was
\begin{equation}\label{eqn:qftLecture11:300}
\hatU(\Lambda)
\sqrt{ 2 \omega_\Bp } \hata_\Bp
\hatU^\dagger(\Lambda)
=
\sqrt{ 2 \omega_{\Lambda \Bp} } \hata_{\Lambda \Bp}.
\end{equation}

If \( \ket{\Bp_1} \) is the one particle state with momentum \( \Bp_1 \), then that momentum state can be generated from the ground state with the following normalized creation operation
\begin{equation}\label{eqn:qftLecture11:780}
\ket{\Bp_1} = \sqrt{ 2 \omega_{\Bp_1} } \hata_{\Bp_1}^\dagger \ket{0}.
\end{equation}

We can compute the matrix element between two matrix states using the creation operator representation
\begin{equation}\label{eqn:qftLecture11:80}
\begin{aligned}
\braket{\Bp_2}{\Bp_1}
&=
\sqrt{ 2 \omega_{\Bp_1} }
\sqrt{ 2 \omega_{\Bp_2} }
\bra{0}
\hata_{\Bp_2}
\hata_{\Bp_1}^\dagger
\ket{0} \\
&=
\sqrt{ 2 \omega_{\Bp_1} }
\sqrt{ 2 \omega_{\Bp_2} }
\bra{0}
\lr{
   \hata_{\Bp_1}^\dagger
   \hata_{\Bp_2}
   +
   i (2 \pi)^3 \deltathree(\Bp - \Bq)
} \\
&=
\sqrt{ 2 \omega_{\Bp_1} }
\sqrt{ 2 \omega_{\Bp_2} }
(2 \pi)^3 \deltathree(\Bp_1 - \Bp_2) \\
&=
2 \omega_{\Bp_1}
(2 \pi)^3 \deltathree(\Bp_1 - \Bp_2).
\end{aligned}
\end{equation}

\section{Spacelike surfaces.}
\index{spacelike surface}

If \( x^\mu, p^\mu \) are four vectors, then \( p^\mu x_\mu = \text{invariant} = {p'}^\mu x'_\mu \).  The light cone is the surface \( p_0^2 = \Bp^2  \), whereas timelike four-momentum form a paraboloid surface \( p_0^2 - \Bp^2 = m^2 \) (i.e. \( E = \sqrt{ m^2 c^4 + \Bp^2 c^2 } \)).
The surface for constant spacelike points (i.e. all related by a Lorentz transformation) is illustrated in \cref{fig:spaceLikeAndLightCone:spaceLikeAndLightConeFig1}.  A boost moves a point up or down that surface along the energy axis.  It is therefore possible to use a sequence of boost and rotation to transform a point \( (E, \Bp) \rightarrow (-E, \Bp) \rightarrow (-E, -\Bp) \).  That is, any spacelike four-vector \( x \) may be transformed to \( -x \) using a Lorentz transformation.
\mathImageFigure{../figures/phy2403-quantum-field-theory/spaceLikeAndLightConeFig1}{Constant spacelike surface.}{fig:spaceLikeAndLightCone:spaceLikeAndLightConeFig1}{0.3}{hyperboliodAndLightCone.nb}

\section{Condition on microcausality.}

We defined operators \( \phihat(\Bx) \), which was a Hermitian operator for the real scalar field.  For the complex scalar field we used \( \phihat(\Bx) = (\phihat_1 + \phihat_2)/\sqrt{2} \), where each of \( \phihat_1, \phihat_2 \) were Hermitian operators.  i.e. we can think of these operators as ``observables'', that is \( \phihat(\Bx) = \phihat^\dagger(\Bx) \).

We now want to show that these operators commute at spacelike separations, and see how this relates to the question of causality.  In particular, we want to see that an observation of one operator, will not effect the measurement of the other.

The condition of microcausality is
\begin{equation*}
\antisymmetric{\phihat(x)}{\phihat(y)} = 0
\end{equation*}
if \( x \sim y \), that is \( (x - y)^2 < 0 \).  That is, \( x, y \) are spacelike separated.

We wrote
\begin{equation}\label{eqn:qftLecture11:160}
\phihat(x)
=
\int \frac{d^3 p}{(2 \pi)^3 \sqrt{2 \omega_\Bp}}
\evalbar{
e^{-i p \cdot x} }{p^0 = \omega_\Bp} \hata_\Bp
+
\int \frac{d^3 p}{(2 \pi)^3 \sqrt{2 \omega_\Bp}}
\evalbar{
e^{i p \cdot x} }{p^0 = \omega_\Bp} \hata^\dagger_\Bp
,
\end{equation}
or \( \phihat(x) = \phihat_{-}(x) + \phihat_{+}(x) \), where
\begin{equation}\label{eqn:qftLecture11:180}
\begin{aligned}
\phihat_{-}(x) &=
\int \frac{d^3 p}{(2 \pi)^3 \sqrt{2 \omega_\Bp}}
\evalbar{
e^{-i p \cdot x} }{p^0 = \omega_\Bp} \hata_\Bp \\
\phihat_{+}(x) &=
\int \frac{d^3 p}{(2 \pi)^3 \sqrt{2 \omega_\Bp}}
\evalbar{
e^{i p \cdot x} }{p^0 = \omega_\Bp} \hata^\dagger_\Bp.
\end{aligned}
\end{equation}

Compute the commutator
\begin{equation}\label{eqn:qftLecture11:200}
\begin{aligned}
D(x)
&= \antisymmetric{\phihat_{-}(x)}{\phihat_{+}(0)}
\\&=
\int \frac{d^3 p}{(2 \pi)^3 \sqrt{2 \omega_\Bp}}
\evalbar{ e^{-i p \cdot x} }{p^0 = \omega_\Bp}
\int \frac{d^3 k}{(2 \pi)^3 \sqrt{2 \omega_\Bk}}
\evalbar{ e^{i k \cdot 0} }{k^0 = \omega_\Bk}
\antisymmetric{\hata_\Bp }{\hata_\Bk^\dagger }
\\&=
\int \frac{d^3 p}{(2 \pi)^3 \sqrt{2 \omega_\Bp}}
\evalbar{ e^{-i p \cdot x} }{p^0 = \omega_\Bp}
\int \frac{d^3 k}{(2 \pi)^3 \sqrt{2 \omega_\Bk}}
(2 \pi)^3 \deltathree(\Bp - \Bk),
\end{aligned}
\end{equation}
\boxedEquation{eqn:qftLecture11:800}{
D(x)
=
\int \frac{d^3 p}{(2 \pi)^3 2 \omega_\Bp}
\evalbar{ e^{-i p \cdot x} }{p^0 = \omega_\Bp}.
}

Now about the commutator at two spacetime points
\begin{equation}\label{eqn:qftLecture11:220}
\begin{aligned}
\antisymmetric{\phihat(x)}{\phihat(y)}
&=
\antisymmetric{\phihat_{-}(x) + \phihat_{+}(x)}{\phihat_{-}(y) + \phihat_{+}(y)}
\\&=
\antisymmetric{\phihat_{-}(x)}{\phihat_{+}(y)}
+
\antisymmetric{\phihat_{+}(x)}{\phihat_{-}(y)}
\\&=
-D(y - x) + D(x - y).
\end{aligned}
\end{equation}

Find
\begin{equation}\label{eqn:qftLecture11:240}
\begin{aligned}
\antisymmetric{\phihat(x)}{\phihat(y)} &= D(x - y) - D(y - x) \\
\antisymmetric{\phihat(x)}{\phihat(0)} &= D(x) - D(- x).
\end{aligned}
\end{equation}

Let's look at \( D(x) \), \cref{eqn:qftLecture11:800}, a bit more closely.

\paragraph{Claim:}
\( D(x) \) is Lorentz invariant (has the same value for all \( x^\mu, {x'}^\mu \)

We can see this by writing this out as
\begin{equation}\label{eqn:qftLecture11:280}
D(x)
=
\int \frac{d^3 p}{(2 \pi)^3 } dp^0
\delta( p_0^2 - \Bp^2 - m^2) \Theta(p^0)
e^{-i p \cdot x}.
\end{equation}

The exponential is Lorentz invariant, and the delta function has been put into a Lorentz invariant form.

\paragraph{Claim 1:}
\( D(x) = D(x') \) where \( x^2 = {x'}^2 \).

\paragraph{Claim 2:}
\( x^\mu, -x^\mu \) are related by Lorentz transformations if \( x^2 < 0 \).

From the figure, we see that \( D(x) = D(-x) \) for a spacelike point, which implies that
\(
\antisymmetric{\phihat(x)}{\phihat(0)} = 0 \) for a spacelike point \( x \).

We've shown this for free fields, but later we will see that this is the case for interacting fields too.

