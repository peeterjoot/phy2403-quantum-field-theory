%
% Copyright � 2018 Peeter Joot.  All Rights Reserved.
% Licenced as described in the file LICENSE under the root directory of this GIT repository.
%
%{
%\input{../latex/blogpost.tex}
%\renewcommand{\basename}{qftLecture18}
%\renewcommand{\dirname}{notes/phy2403/}
%\newcommand{\keywords}{PHY2403H}
%\input{../latex/peeter_prologue_print2.tex}
%
%%\usepackage{phy2403}
%\usepackage{peeters_braket}
%%\usepackage{peeters_layout_exercise}
%\usepackage{peeters_figures}
%\usepackage{mathtools}
%\usepackage{siunitx}
%\usepackage{macros_cal} % LL
%
%\newcommand{\ultensor}[3]{{{#1}^{#2}}_{#3}}
%\newcommand{\deltathree}[0]{\delta^{(3)}}
%\newcommand{\deltafour}[0]{\delta^{(4)}}
%
%\beginArtNoToc
%\generatetitle{PHY2403H Quantum Field Theory.  Lecture 18: XXX.  Taught by Prof.\ Erich Poppitz}
%%\chapter{XXX}
\label{chap:qftLecture18}

%%Peeter's lecture notes from class.  These may be incoherent and rough.
%%
%%These are notes for the UofT course PHY2403H, Quantum Field Theory, taught by Prof. Erich Poppitz, covering \textchapref{{1}} \citep{peskin1995introduction} content.

%\paragraph{DISCLAIMER: Very rough notes from class, with some additional side notes.}
%
%These are notes for the UofT course PHY2403H, Quantum Field Theory, taught by Prof. Erich Poppitz, fall 2018.
%%, covering \textchapref{{1}} \citep{peskin1995introduction} content.
%
\section{More on cross section.}

\begin{equation}\label{eqn:qftLecture18:20}
d\sigma( AB \rightarrow 1 \cdots n)
=
\inv {
4 \sqrt{ (k_A k_B)^2 - m_A^2 m_B^2 }
}
\Abs{M_{fi}}^2 \times
(2 \pi)^4 \deltafour( \sum p_i - \sum_f p^f )
\prod_{f} \frac{ d^3 p}{(2 \pi)^3 2 \omega_{\Bp^f} }.
\end{equation}

\paragraph{Digression: One to three muon decay}
I can't recall what prompted this, but an example of muon (a negative charged ``heavy electron'') decay was given, sketched in
\cref{fig:L18TwoToTwoScattering:L18TwoToTwoScatteringFig2}.
In such decay lepton number and charge both have to be conserved.  This is why such a decay must have a mu-neutrino (\(\nu_\mu\)), as well as an antineutrino (\(\overbar{\nu}_e\)) along with the electron (\(e^-\)).
\imageFigure{../figures/phy2403-quantum-field-theory/L18TwoToTwoScatteringFig2}{Muon decay: One to three scattering.}{fig:L18TwoToTwoScattering:L18TwoToTwoScatteringFig2}{0.3}

\paragraph{Back to the scattering:}
Example:
\begin{dmath}\label{eqn:qftLecture18:40}
\phi \phi \rightarrow \phi \phi
\end{dmath}

%\cref{fig:L18OnetoThreeDecay:L18OnetoThreeDecayFig1}.
\imageFigure{../figures/phy2403-quantum-field-theory/L18OnetoThreeDecayFig1}{Two to two scattering.}{fig:L18OnetoThreeDecay:L18OnetoThreeDecayFig1}{0.3}

1) Do flux factor in CM frame as sketched in \cref{fig:L18TwoToTwoScatteringCMFrame:L18TwoToTwoScatteringCMFrameFig3}.

\imageFigure{../figures/phy2403-quantum-field-theory/L18TwoToTwoScatteringCMFrameFig3}{Two to two scattering in CM frame.}{fig:L18TwoToTwoScatteringCMFrame:L18TwoToTwoScatteringCMFrameFig3}{0.3}

\begin{dmath}\label{eqn:qftLecture18:60}
\begin{aligned}
k_A^2 &= \frac{s}{4} - \Bk^2 = m^2 \\
k_B^2 &= \frac{s}{4} - \Bk^2 = m^2
\end{aligned}
\end{dmath}

(initial particles are on shell)

\begin{dmath}\label{eqn:qftLecture18:80}
\begin{aligned}
k_A &= \lr{ \frac{\sqrt{s}}{2}, -\Bk } \\
k_B &= \lr{ \frac{\sqrt{s}}{2}, +\Bk }
\end{aligned}
\end{dmath}

In CM:

\begin{dmath}\label{eqn:qftLecture18:100}
\sqrt{s} = E_A + E_B
\end{dmath}

\begin{dmath}\label{eqn:qftLecture18:120}
4 \sqrt{ (k_A k_B)^2 - m_A^2 m_B^2 }
=
4 \sqrt{ \lr{ \frac{s}{4} + \Bk^2 }^2 - (m^2)^2 }
=
4 \sqrt{
\lr{ \frac{s}{4} + \Bk^2 - m^2 }
\lr{ \frac{s}{4} + \Bk^2 + m^2 }
}
\end{dmath}

but
\begin{dmath}\label{eqn:qftLecture18:140}
\frac{s}{4} - m^2 = \Bk^2
\end{dmath}

\begin{dmath}\label{eqn:qftLecture18:160}
4 \sqrt{ (k_A k_B)^2 - m_A^2 m_B^2 }
=
4 \sqrt{ \cancel{2} \Bk^2 \frac{s}{\cancel{2}} }
=
4 \Abs{\Bk} \sqrt{s}
\end{dmath}

\paragraph{Note:} \( s \) is known as a Mandelstam variable (see for example: \citep{wiki:Mandelstam}).
\index{Mandelstam variable}

We have two particles

In the particle data book this factor
\begin{dmath}\label{eqn:qftLecture18:180}
(2 \pi)^4 \deltafour( \sum p_i - \sum_f p^f )
\prod_{f} \frac{ d^3 p}{(2 \pi)^3 2 \omega_{\Bp^f} },
\end{dmath}
is called \( d(LIPS)_2 \).  In the CM frame the delta function simplifies and we have

\begin{dmath}\label{eqn:qftLecture18:200}
d(LIPS)_2
=
\frac{ d^3 p_1}{(2 \pi)^3}
\frac{ d^3 p_2}{(2 \pi)^3} \inv{2 \omega_1 2 \omega_2 }
2 \pi (2 \pi)^3 \deltathree( \Bp_1 + \Bp_2 ) \delta( 2 \omega_1 - \sqrt{s} )
=
\frac{ d^3 p_1}{(2 \pi)^3} \inv{4 \omega_1^2 }
2 \pi \delta( 2 \omega_1 - \sqrt{s} )
=
\frac{ d^3 p_1}{(2 \pi)^3} \inv{ 4 \omega_1^2}
2 \pi \delta( 2 \sqrt{\Bp_1^2 + m^2 } - \sqrt{s} )
\end{dmath}

\begin{dmath}\label{eqn:qftLecture18:220}
p_1 = \lr{ \frac{\sqrt{s}}{2}, \Bp_1 }
\end{dmath}

\begin{dmath}\label{eqn:qftLecture18:240}
p_1^2 = m^2 = \frac{s}{4} - \Bp_1^2,
\end{dmath}

so

\begin{dmath}\label{eqn:qftLecture18:260}
\Bp_2 = \frac{s}{4} - m^2
\end{dmath}

With
\begin{dmath}\label{eqn:qftLecture18:280}
\delta(f(x)) = \frac{\delta(x^\conj)}{f'(x^\conj)},
\end{dmath}
where \( f(x^\conj) = 0 \).

giving
\begin{dmath}\label{eqn:qftLecture18:300}
d(LIPS)_2
=
\frac{ d^3 p_1}{(2 \pi)^3} \inv{ 4 \omega_1^2}
2 \pi \delta( p_1 - \sqrt{s/4 - m^2} )
=
\frac{d^2 \Omega p_1^2 dp_1}{(2 \pi)^2 4 \omega_1^2 } \frac{\delta(p_1 - \sqrt{s/4 - m^2}}{2 p_1/\omega_1}
=
\frac{d^2 \Omega p_1^2 dp_1}{(2 \pi)^2 8 \omega_1 }
=
\frac{d^2 \Omega p_1}{16 \pi^2 \sqrt{s} },
\end{dmath}
since \( \omega_1 = \sqrt{s}/2 \).

This gives
\begin{dmath}\label{eqn:qftLecture18:320}
\frac{d\sigma}{d^2 \Omega}
=
\frac{ p_1 }{16 \pi^2 \sqrt{s} }
\frac{ \lambda^2 }{ \Abs{p_1} \sqrt{s} } 4
=
\frac{\lambda^2}{64 \pi^2 s}.
\end{dmath}

Since
\begin{dmath}\label{eqn:qftLecture18:340}
\int d^2 \Omega = 4 \pi,
\end{dmath}
the total cross section is
\begin{dmath}\label{eqn:qftLecture18:360}
\sigma_{\text{total}}
=
\frac{\lambda^2}{s \pi^2} \frac{ 4 \pi}{2 64/16}
=
\frac{\lambda^2}{32 s}.
\end{dmath}
There was a counting adjustment made here that I didn't quite catch.

%}
%\EndArticle
