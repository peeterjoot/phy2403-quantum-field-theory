%
% Copyright � 2017 Peeter Joot.  All Rights Reserved.
% Licenced as described in the file LICENSE under the root directory of this GIT repository.
%
%{
\input{../latex/blogpost.tex}
\renewcommand{\basename}{qftLecture18}
\renewcommand{\dirname}{notes/phy2403/}
\newcommand{\keywords}{PHY2403H}
\input{../latex/peeter_prologue_print2.tex}

%\usepackage{phy2403}
\usepackage{peeters_braket}
%\usepackage{peeters_layout_exercise}
\usepackage{peeters_figures}
\usepackage{mathtools}
\usepackage{siunitx}
\usepackage{macros_cal} % LL

\newcommand{\ultensor}[3]{{{#1}^{#2}}_{#3}}
\newcommand{\deltathree}[0]{\delta^{(3)}}
\newcommand{\deltafour}[0]{\delta^{(4)}}

\beginArtNoToc
\generatetitle{PHY2403H Quantum Field Theory.  Lecture 18: XXX.  Taught by Prof.\ Erich Poppitz}
%\chapter{XXX}
\label{chap:qftLecture18}

%%Peeter's lecture notes from class.  These may be incoherent and rough.
%%
%%These are notes for the UofT course PHY2403H, Quantum Field Theory, taught by Prof. Erich Poppitz, covering \textchapref{{1}} \citep{peskin1995introduction} content.

\paragraph{DISCLAIMER: Very rough notes from class, with some additional side notes.}

These are notes for the UofT course PHY2403H, Quantum Field Theory, taught by Prof. Erich Poppitz, fall 2018.
%, covering \textchapref{{1}} \citep{peskin1995introduction} content.

\section{More on cross section.}

\begin{equation}\label{eqn:qftLecture18:20}
d\sigma( AB \rightarrow 1 \cdots n)
=
\inv {
4 \sqrt{ (k_A - k_B)^2 - m_A^2 m_B^2 }
}
\Abs{M_{fi}}^2 \times
(2 \pi)^4 \deltafour( \sum p_i - \sum_f p^f )
\prod_{f} \frac{ d^3 p}{(2 \pi)^3 2 \omega_{\Bp^f} }.
\end{equation}

digression

F1

Lepton number has to be conserved, so the muon (a negative charged ``heavy electron''

Example:
\begin{dmath}\label{eqn:qftLecture18:40}
\phi \phi \rightarrow \phi \phi
\end{dmath}

1) Do flux factor in CM frame

F2

\begin{dmath}\label{eqn:qftLecture18:60}
\begin{aligned}
k_A^2 = \frac{s}{4} - \Bk^2 = m^2
k_B^2 = \frac{s}{4} - \Bk^2 = m^2
\end{aligned}
\end{dmath}

(initial particles are on shell)

\begin{dmath}\label{eqn:qftLecture18:80}
\begin{aligned}
k_A &= \lr{ \sqrt{s}{2}, -\Bk } \\
k_B &= \lr{ \sqrt{s}{2}, +\Bk }
\end{aligned}
\end{dmath}

In CM:

\begin{dmath}\label{eqn:qftLecture18:100}
\sqrt{s} = E_A + E_B
\end{dmath}

\begin{dmath}\label{eqn:qftLecture18:120}
4 \sqrt{ (k_A - k_B)^2 - m_A^2 m_B^2 }
=
4 \sqrt{ \lr{ \frac{s}{4} + \Bk^2 }^2 - (m^2)^2 }
=
4 \sqrt{
\lr{ \frac{s}{4} + \Bk^2 - m^2 }
\lr{ \frac{s}{4} + \Bk^2 + m^2 }
}
\end{dmath}

but
\begin{dmath}\label{eqn:qftLecture18:140}
\frac{s}{4} - m^2 = \Bk^2
\end{dmath}

\begin{dmath}\label{eqn:qftLecture18:160}
4 \sqrt{ (k_A - k_B)^2 - m_A^2 m_B^2 }
=
4 \sqrt{ \cancel{2} \Bk^2 \frac{s}{2} }
=
4 \Abs{\Bk} \sqrt{s}
\end{dmath}

Note: \( s \) is known as a Mandelshen variable (sp?)

We have two particles

In the particle data book this factor
\begin{dmath}\label{eqn:qftLecture18:180}
(2 \pi)^4 \deltafour( \sum p_i - \sum_f p^f )
\prod_{f} \frac{ d^3 p}{(2 \pi)^3 2 \omega_{\Bp^f} },
\end{dmath}
is called \( d(LIPS)_2 \).  In the CM frame the delta function simplifies and we have

\begin{dmath}\label{eqn:qftLecture18:200}
d(LIPS)_2
=
\frac{ d^3 p_1}{(2 \pi)^3}
\frac{ d^3 p_2}{(2 \pi)^3} \inv{2 \omega_1 2 \omega_2 }
2 \pi (2 \pi)^3 \deltathree( \Bp_1 + \Bp_2 ) \delta( 2 \omega_1 - \sqrt{s} )
=
\frac{ d^3 p_1}{(2 \pi)^3} \inv{4 \omega_1^2 }
2 \pi \delta( 2 \omega_1 - \sqrt{s} )
=
\frac{ d^3 p_1}{(2 \pi)^3} \inv{ 4 \omega_1^2}
2 \pi \delta( 2 \sqrt{\Bp_1^2 + m^2 } - \sqrt{s} )
\end{dmath}

\begin{dmath}\label{eqn:qftLecture18:220}
p_1 = \lr{ \frac{\sqrt{s}}{2}, \Bp_1 }
\end{dmath}

\begin{dmath}\label{eqn:qftLecture18:240}
p_1^2 = m^2 = \frac{s}{4} - \Bp_1^2,
\end{dmath}

so

\begin{dmath}\label{eqn:qftLecture18:260}
\Bp_2 = \frac{s}{4} - m^2
\end{dmath}

With
\begin{dmath}\label{eqn:qftLecture18:280}
\delta(f(x)) = \frac{\delta(x^\conj)}{f'(x^\conj)},
\end{dmath}
where \( f(x^\conj) = 0 \).

giving
\begin{dmath}\label{eqn:qftLecture18:300}
d(LIPS)_2
=
\frac{ d^3 p_1}{(2 \pi)^3} \inv{ 4 \omega_1^2}
2 \pi \delta( p_1 - \sqrt{s/4 - m^2} )
=
\frac{d^2 \Omega p_1^2 dp_1}{(2 \pi)^2 4 \omega_1^2 } \frac{\delta(p_1 - \sqrt{s/4 - m^2}}{2 p_1/\omega_1}
=
\frac{d^2 \Omega p_1^2 dp_1}{(2 \pi)^2 8 \omega_1 }
=
\frac{d^2 \Omega p_1}{16 \pi^2 \sqrt{s} },
\end{dmath}
since \( \omega_1 = \sqrt{s}/2 \).

This gives
\begin{dmath}\label{eqn:qftLecture18:320}
\frac{d\sigma}{d^2 \Omega}
=
\frac{ p_1 }{16 \pi^2 \sqrt{s} }
\frac{ \lambda^2 }{ \Abs{p_1} \sqrt{s} } 4
=
\frac{\lambda^2}{64 \pi^2 s}.
\end{dmath}

Since
\begin{dmath}\label{eqn:qftLecture18:340}
\int d^2 \Omega = 4 \pi,
\end{dmath}
the total cross section is
\begin{dmath}\label{eqn:qftLecture18:360}
\sigma_{\text{total}}
=
\frac{\lambda^2}{s \pi^2} \frac{ 4 \pi}{2 64/16}
=
\frac{\lambda^2}{32 s}.
\end{dmath}
There was a counting adjustment made here that I didn't quite catch.

\section{Fermions: \R{3} rotations.}

Given a real vector \( \Bx \) and the Pauli matrices
\begin{equation}\label{eqn:qftLecture18:400}
\sigma^1 = \PauliX, \qquad
\sigma^2 = \PauliY, \qquad
\sigma^3 = \PauliZ.
\end{equation}
We may form a Pauli matrix representation of a vector
\begin{dmath}\label{eqn:qftLecture18:420}
\Bsigma \cdot \Bx
=
\begin{bmatrix}
x_3 & x_1 - i x_2 \\
x_1 + i x_2 & -x_3
\end{bmatrix},
\end{dmath}
where \( \Bsigma = \lr{ \sigma^1, \sigma^2, \sigma^3 } \).
This matrix, like the Pauli matrixes, is a \( 2 \times 2 \) Hermitian traceless matrix.
We find that the determinant is
\begin{dmath}\label{eqn:qftLecture18:480}
\det \Bx \cdot \Bsigma
=
-x_3^2 - x_1^2 - x_2^2
=
-\Bx^2.
\end{dmath}

We may form
\begin{dmath}\label{eqn:qftLecture18:440}
U \Bx \cdot \Bsigma U^\dagger,
\end{dmath}
where \( U \) is a unitary \( 2 \times 2 \) unit determinant matrix, satisfying
\begin{dmath}\label{eqn:qftLecture18:460}
\begin{aligned}
U^\dagger U &= 1 \\
\det U &= 1.
\end{aligned}
\end{dmath}
Further
\begin{dmath}\label{eqn:qftLecture18:500}
\det (U \Bx \cdot \Bsigma) U^\dagger
=
\det U \det (\Bx \cdot \Bsigma) \det U^\dagger
=
\det (\Bx \cdot \Bsigma).
\end{dmath}

Moral: \( U \Bx \cdot \Bsigma U^\dagger = \Bx' \cdot \Bsigma \), where \( \Bx' \) has the same length of \( \Bx \).

We may use this to represent an arbitrary rotation
\begin{dmath}\label{eqn:qftLecture18:520}
U \Bx \cdot \Bsigma U^\dagger
= {R^{i}}_j x^j \sigma^i
\end{dmath}

We say that \( U \in SU(2) \) and \( R \in SU(3) \), and \( SU(2) \) is called the ``universal cover of \(SO(3)\)''.

Pauli figured out that, in non-relativistic QM, that this type of transformation also applies to (spin) wave functions (spinors)
\index{spinor}
\begin{equation}\label{eqn:qftLecture18:540}
\Psi(\Bx) \rightarrow \Psi'(\Bx') = U \Psi(\Bx)
\end{equation}
where
\begin{equation}\label{eqn:qftLecture18:560}
\Bx \rightarrow \Bx' = R \Bx,
\end{equation}
and \( R^\T R = 1 \), and where
\begin{dmath}\label{eqn:qftLecture18:580}
\Psi(\Bx) =
\begin{bmatrix}
\Psi_\uparrow(\Bx) \\
\Psi_\downarrow(\Bx)
\end{bmatrix}.
\end{dmath}

Since such representations are allowed and required in NRQM, how do these generalize to relativistic.

\section{Lorentz group}
Let
\begin{dmath}\label{eqn:qftLecture18:600}
(x^0, \Bx) =
x^0 \sigma^0 + \Bx \cdot \Bsigma
=
\begin{bmatrix}
x^0 + x_3 & x_1 - i x_2 \\
x_1 + i x_2 & x^0 -x_3.
\end{bmatrix}
\end{dmath}
This has determinant
\begin{dmath}\label{eqn:qftLecture18:620}
\det
(x^0, \Bx) =
(x^0)^2
-
(x^1)^2
-
(x^2)^2
-
(x^3)^2
= x^\mu x_\mu.
\end{dmath}
We therefore identify \( (x^0, \Bx) \) as a four vector
\begin{dmath}\label{eqn:qftLecture18:640}
(x^0, \Bx) = x^\mu \sigma_\mu
\end{dmath}
We say that \( SL(2, \bbC)\) is a double cover of \( SO(1,3)\).

%}
%\EndArticle
\EndNoBibArticle
