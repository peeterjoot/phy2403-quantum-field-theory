%
% Copyright � 2018 Peeter Joot.  All Rights Reserved.
% Licenced as described in the file LICENSE under the root directory of this GIT repository.
%
%{
%%%\input{../latex/blogpost.tex}
%%%\renewcommand{\basename}{nonhomoKGhamiltonian}
%%%%\renewcommand{\dirname}{notes/phy1520/}
%%%\renewcommand{\dirname}{notes/ece1228-electromagnetic-theory/}
%%%%\newcommand{\dateintitle}{}
%%%%\newcommand{\keywords}{}
%%%
%%%\input{../latex/peeter_prologue_print2.tex}
%%%
%%%\usepackage{peeters_layout_exercise}
%%%\usepackage{peeters_braket}
%%%\usepackage{peeters_figures}
%%%\usepackage{siunitx}
%%%\usepackage{verbatim}
%%%%\usepackage{mhchem} % \ce{}
%%%%\usepackage{macros_bm} % \bcM
%%%%\usepackage{macros_qed} % \qedmarker
%%%%\usepackage{txfonts} % \ointclockwise
%%%
%%%\beginArtNoToc
%%%
%%%\generatetitle{Hamiltonian for the non-homogeneous Klein-Gordon equation}
%\chapter{Hamiltonian for the non-homogeneous Klein-Gordon equation}
%\label{chap:nonhomoKGhamiltonian}
\section{Problems.}

\makeproblem{Hamiltonian with forcing term.}{problem:qftLecture13:1280}{
Prove \cref{eqn:qftLecture13:140}.
} % problem

\makeanswer{problem:qftLecture13:1280}{
In class we derived the field for the non-homogeneous Klein-Gordon equation
\begin{dmath}\label{eqn:nonhomoKGhamiltonian:20}
\phi(x)
= \int \frac{d^3 p}{(2\pi)^3} \inv{\sqrt{2 \omega_\Bp}}
\evalbar{
\lr{
   e^{-i p \cdot x} \lr{ a_\Bp + \frac{ i \tilde{j}(p) }{\sqrt{2 \omega_\Bp}} }
   +
   e^{i p \cdot x} \lr{ a_\Bp^\dagger - \frac{ i \tilde{j}^\conj(p) }{\sqrt{2 \omega_\Bp}} }
}
}
{
p^0 = \omega_\Bp
}
= \int \frac{d^3 p}{(2\pi)^3} \inv{\sqrt{2 \omega_\Bp}}
\lr{
   e^{-i \omega_\Bp t + i \Bp \cdot \Bx} \lr{ a_\Bp + \frac{ i \tilde{j}(p) }{\sqrt{2 \omega_\Bp}} }
   +
   e^{i \omega_\Bp t - i \Bp \cdot \Bx} \lr{ a_\Bp^\dagger - \frac{ i \tilde{j}^\conj(p) }{\sqrt{2 \omega_\Bp}} }
}.
\end{dmath}
This means that we have
\begin{equation}\label{eqn:nonhomoKGhamiltonian:40}
\begin{aligned}
\pi = \dot{\phi}
&= \int \frac{d^3 p}{(2\pi)^3} \frac{i \omega_\Bp}{\sqrt{2 \omega_\Bp}}
\lr{
   - e^{-i \omega_\Bp t + i \Bp \cdot \Bx} \lr{ a_\Bp + \frac{ i \tilde{j}(p) }{\sqrt{2 \omega_\Bp}} }
   +
   e^{i \omega_\Bp t - i \Bp \cdot \Bx} \lr{ a_\Bp^\dagger - \frac{ i \tilde{j}^\conj(p) }{\sqrt{2 \omega_\Bp}} }
} \\
(\spacegrad \phi)_k =
&= \int \frac{d^3 p}{(2\pi)^3} \frac{i p_k}{\sqrt{2 \omega_\Bp}}
\lr{
     e^{-i \omega_\Bp t + i \Bp \cdot \Bx} \lr{ a_\Bp + \frac{ i \tilde{j}(p) }{\sqrt{2 \omega_\Bp}} }
   -
   e^{i \omega_\Bp t - i \Bp \cdot \Bx} \lr{ a_\Bp^\dagger - \frac{ i \tilde{j}^\conj(p) }{\sqrt{2 \omega_\Bp}} }
},
\end{aligned}
\end{equation}
and could plug these into the Hamiltonian
\begin{dmath}\label{eqn:nonhomoKGhamiltonian:60}
H = \int d^3 p \lr{ \inv{2} \pi^2 + \inv{2} \lr{ \spacegrad \phi}^2 + \frac{m^2}{2} \phi^2 },
\end{dmath}
to find \( H \) in terms of \( \tilde{j} \) and \( a_\Bp^\dagger, a_\Bp \).  The result was mentioned in class, and it was left as an exercise to verify.

There's an easy way and a dumb way to do this exercise.  I did it the dumb way, and then after suffering through two long pages, where the equations were so long that I had to write on the paper sideways, I realized the way I should have done it.

The easy way is to observe that we've already done exactly this for the case \( \tilde{j} = 0 \), which had the answer
\begin{dmath}\label{eqn:nonhomoKGhamiltonian:80}
H = \inv{2} \int \frac{d^3 p}{(2 \pi)^3} \omega_\Bp \lr{ a_\Bp^\dagger a_\Bp + a_\Bp a_\Bp^\dagger }.
\end{dmath}
To handle this more general case, all we have to do is apply a transformation
\begin{dmath}\label{eqn:nonhomoKGhamiltonian:100}
a_\Bp \rightarrow
a_\Bp + \frac{i \tilde{j}(p)}{\sqrt{2 \omega_\Bp}},
\end{dmath}
to \cref{eqn:nonhomoKGhamiltonian:80}, which gives
\begin{dmath}\label{eqn:nonhomoKGhamiltonian:120}
H
=
\inv{2} \int \frac{d^3 p}{(2 \pi)^3} \omega_\Bp \lr{\lr{ a_\Bp + \frac{i \tilde{j}(p)}{\sqrt{2 \omega_\Bp}} }^\dagger\lr{ a_\Bp + \frac{i \tilde{j}(p)}{\sqrt{2 \omega_\Bp}} } +\lr{ a_\Bp + \frac{i \tilde{j}(p)}{\sqrt{2 \omega_\Bp}} }\lr{ a_\Bp + \frac{i \tilde{j}(p)}{\sqrt{2 \omega_\Bp}} }^\dagger }
=
\inv{2} \int \frac{d^3 p}{(2 \pi)^3} \omega_\Bp \lr{\lr{ a_\Bp^\dagger - \frac{i \tilde{j}^\conj(p)}{\sqrt{2 \omega_\Bp}} } \lr{ a_\Bp + \frac{i \tilde{j}(p)}{\sqrt{2 \omega_\Bp}} } +\lr{ a_\Bp + \frac{i \tilde{j}(p)}{\sqrt{2 \omega_\Bp}} }\lr{ a_\Bp^\dagger - \frac{i \tilde{j}^\conj(p)}{\sqrt{2 \omega_\Bp}} }
}.
\end{dmath}

Like the \( \tilde{j} = 0 \) case, we can use normal ordering.  This is easily seen by direct expansion:
\begin{dmath}\label{eqn:nonhomoKGhamiltonian:140}
\begin{aligned}
\lr{ a_\Bp^\dagger - \frac{i \tilde{j}^\conj(p)}{\sqrt{2 \omega_\Bp}} } \lr{ a_\Bp + \frac{i \tilde{j}(p)}{\sqrt{2 \omega_\Bp}} }
&=
a_\Bp^\dagger a_\Bp
- \frac{i \tilde{j}^\conj(p) a_\Bp}{\sqrt{2 \omega_\Bp}}
+ \frac{ a_\Bp^\dagger i \tilde{j}^\conj(p)}{\sqrt{2 \omega_\Bp}}
+ \frac{\Abs{j}^2}{2 \omega_\Bp} \\
\lr{ a_\Bp + \frac{i \tilde{j}(p)}{\sqrt{2 \omega_\Bp}} }\lr{ a_\Bp^\dagger - \frac{i \tilde{j}^\conj(p)}{\sqrt{2 \omega_\Bp}} }
&=
a_\Bp^\dagger a_\Bp
+ \frac{i \tilde{j}^\conj(p) a_\Bp^\dagger}{\sqrt{2 \omega_\Bp}}
- \frac{ a_\Bp i \tilde{j}^\conj(p)}{\sqrt{2 \omega_\Bp}}
+ \frac{\Abs{j}^2}{2 \omega_\Bp}.
\end{aligned}
\end{dmath}
Because \( \tilde{j} \) is just a complex valued function, it commutes with \( a_\Bp, a_\Bp^\dagger \), and these are equal up to the normal ordering, allowing us to write
\begin{equation}\label{eqn:nonhomoKGhamiltonian:160}
:H: =
\int \frac{d^3 p}{(2 \pi)^3} \omega_\Bp \lr{ a_\Bp^\dagger - \frac{i \tilde{j}^\conj(p)}{\sqrt{2 \omega_\Bp}}} \lr{ a_\Bp + \frac{i \tilde{j}(p)}{\sqrt{2 \omega_\Bp}} },
\end{equation}
which is the result mentioned in class (albeit without the explicit normal ordering syntax.)
} % answer

%}
%\EndNoBibArticle
