%
% Copyright � 2018 Peeter Joot.  All Rights Reserved.
% Licenced as described in the file LICENSE under the root directory of this GIT repository.
%
%{
%\input{../latex/blogpost.tex}
%\renewcommand{\basename}{qftLecture19}
%\renewcommand{\dirname}{notes/phy2403/}
%\newcommand{\keywords}{PHY2403H}
%\input{../latex/peeter_prologue_print2.tex}
%
%%\usepackage{phy2403}
%\usepackage{peeters_braket}
%\usepackage{peeters_layout_exercise}
%\usepackage{peeters_figures}
%\usepackage{mathtools}
%\usepackage{siunitx}
%\usepackage{macros_cal} % LL
%\usepackage{simplewick}
%
%\newcommand{\ultensor}[3]{{{#1}^{#2}}_{#3}}
%\newcommand{\deltathree}[0]{\delta^{(3)}}
%\newcommand{\deltafour}[0]{\delta^{(4)}}
%
%\beginArtNoToc
%\generatetitle{PHY2403H Quantum Field Theory.  Lecture 19: Pauli matrices, Weyl spinors, SL(2,c), Weyl action, Weyl equation, Dirac matrix, Dirac action, Dirac Lagrangian.  Taught by Prof.\ Erich Poppitz}
%%\chapter{Pauli matrices, Weyl spinors, SL(2,c), Weyl action, Weyl equation, Dirac matrix, Dirac action, Dirac Lagrangian}
\label{chap:qftLecture19}

%%Peeter's lecture notes from class.  These may be incoherent and rough.
%%
%%These are notes for the UofT course PHY2403H, Quantum Field Theory, taught by Prof. Erich Poppitz, covering \textchapref{{1}} \citep{peskin1995introduction} content.

%%\paragraph{DISCLAIMER: Very rough notes from class, with some additional side notes.}
%%
%%These are notes for the UofT course PHY2403H, Quantum Field Theory, taught by Prof. Erich Poppitz, fall 2018.
%%%, covering \textchapref{{1}} \citep{peskin1995introduction} content.
%%
\section{Fermions: \R{3} rotations.}
%(Tail end of Lecture 18)

\index{Pauli matrix}
Given a real vector \( \Bx \) and the Pauli matrices
\begin{equation}\label{eqn:qftLecture19:1794}
\sigma^1 = \PauliX, \qquad
\sigma^2 = \PauliY, \qquad
\sigma^3 = \PauliZ.
\end{equation}
We may form a Pauli matrix representation of a vector
\begin{dmath}\label{eqn:qftLecture19:1814}
\Bsigma \cdot \Bx
=
\begin{bmatrix}
x^3 & x^1 - i x^2 \\
x^1 + i x^2 & -x^3
\end{bmatrix},
\end{dmath}
where \( \Bsigma = \lr{ \sigma^1, \sigma^2, \sigma^3 } \).
This matrix, like the Pauli matrixes, is a \( 2 \times 2 \) Hermitian traceless matrix.
We find that the determinant is
\begin{dmath}\label{eqn:qftLecture19:1834}
\det (\Bsigma \cdot \Bx)
=
-(x^3)^2 - (x^1)^2 - (x^2)^2
=
-\Bx^2.
\end{dmath}

We may form
\begin{dmath}\label{eqn:qftLecture19:1854}
U (\Bsigma \cdot \Bx) U^\dagger,
\end{dmath}
where \( U \) is a unitary \( 2 \times 2 \) unit determinant matrix, satisfying
\begin{dmath}\label{eqn:qftLecture19:1874}
\begin{aligned}
U^\dagger U &= 1 \\
\det U &= 1.
\end{aligned}
\end{dmath}
Further
\begin{dmath}\label{eqn:qftLecture19:1894}
\det (U \Bsigma \cdot \Bx) U^\dagger
=
\det U \det (\Bsigma \cdot \Bx) \det U^\dagger
=
\det (\Bsigma \cdot \Bx).
\end{dmath}

Moral: \( U (\Bsigma \cdot \Bx) U^\dagger = \Bsigma \cdot \Bx' \), where \( \Bx' \) has the same length of \( \Bx \).

We may use this to represent an arbitrary rotation
\begin{dmath}\label{eqn:qftLecture19:1914}
U (\Bsigma \cdot \Bx) U^\dagger
= {R^{i}}_j x^j \sigma^i
\end{dmath}

We say that \( U \in SU(2) \) and \( R \in SU(3) \), and \( SU(2) \) is called the ``universal cover of \(SO(3)\)''.

Pauli figured out that, in non-relativistic QM, that this type of transformation also applies to (spin) wave functions (spinors)
\index{spinor}
\begin{equation}\label{eqn:qftLecture19:1934}
\Psi(\Bx) \rightarrow \Psi'(\Bx') = U \Psi(\Bx)
\end{equation}
where
\begin{equation}\label{eqn:qftLecture19:1954}
\Bx \rightarrow \Bx' = R \Bx,
\end{equation}
and \( R^\T R = 1 \).  Here \( \Psi \) is a two element vector
\begin{dmath}\label{eqn:qftLecture19:1974}
\Psi(\Bx) =
\begin{bmatrix}
\Psi_\uparrow(\Bx) \\
\Psi_\downarrow(\Bx)
\end{bmatrix},
\end{dmath}
so the transformation should be thought of as a matrix operation
\begin{equation}\label{eqn:qftLecture19:80}
\begin{bmatrix}
\Psi_\uparrow(\Bx) \\
\Psi_\downarrow(\Bx)
\end{bmatrix}
\rightarrow
\begin{bmatrix}
\Psi'_\uparrow(\Bx') \\
\Psi'_\downarrow(\Bx')
\end{bmatrix}
=
U
\begin{bmatrix}
\Psi_\uparrow(\Bx) \\
\Psi_\downarrow(\Bx)
\end{bmatrix}.
\end{equation}
Having seen such representations and their \( SU(2) \) transformations in NRQM, we want to know what the relativistic generalization is.

\section{Lorentz group}
Let
\begin{dmath}\label{eqn:qftLecture19:1994}
(x^0, \Bx) =
x^0 \sigma^0 - \Bsigma \cdot \Bx
=
\begin{bmatrix}
x^0 - x^3 & -x^1 + i x^2 \\
-x^1 - i x^2 & x^0 + x^3
\end{bmatrix}.
\end{dmath}
This has determinant
\begin{dmath}\label{eqn:qftLecture19:2014}
\det
(x^0, \Bx) =
(x^0)^2
-
(x^1)^2
-
(x^2)^2
-
(x^3)^2
= x^\mu x_\mu.
\end{dmath}
We therefore identify \( (x^0, \Bx) \) as a four vector
\begin{dmath}\label{eqn:qftLecture19:2034}
(x^0, \Bx) = x_\mu \sigma^\mu = x^\mu \sigma^\nu g_{\mu\nu}.
\end{dmath}
We say that \( SL(2, \bbC)\) is a double cover of \( SO(1,3)\).

%\section{Rotations \( SU(2) \rightarrow SO(3) \).}
%%We will map the traditional vector basis to one with matrix basis elements using
%%\begin{dmath}\label{eqn:qftLecture19:20}
%%\Bx \rightarrow \Bsigma \cdot \Bx,
%%\end{dmath}
%%where \( \Bsigma = (\sigma^1, \sigma^2, \sigma^3) \), and \( \sigma^k \) are the Pauli matrices.
%%
%%A transformation of the following form
%%\begin{dmath}\label{eqn:qftLecture19:40}
%%U^\dagger (\Bsigma \cdot \Bx) U = \Bx' \cdot \Bsigma,
%%\end{dmath}
%%where the matrix \( U \) has \( \det U = 1 \) and is unitary \( U^\dagger U = \BOne \),
%%is length preserving.  We saw that was the case since
%%\begin{dmath}\label{eqn:qftLecture19:60}
%%\det \Bsigma \cdot \Bx \sim \Bx^2.
%%\end{dmath}
%%
%%We say that there is a double cover, as there are multiple representations of the rotation operator.  In particular \( U \) and \( -U \) of \( SU(2) \) correspond to the same \( SO(3) \) element.
Note that the matrix \( U \) can be built explicitly.  For example, it may be built up using Euler angles as sketched in \cref{fig:l19:l19Fig1}.
\imageFigure{../figures/phy2403-quantum-field-theory/l19Fig1}{Euler angle rotations.}{fig:l19:l19Fig1}{0.3}
or algebraically
\begin{dmath}\label{eqn:qftLecture19:100}
U = e^{i \psi \sigma_3/2 } e^{i \theta \sigma_1/2} e^{i \phi \sigma_3/2}.
\end{dmath}

\section{Weyl spinors.}
We will see that there is generalization of Pauli spinors, called Weyl spinors, but we will have to introduce 4 component objects.

We'd like to argue that there is a correspondence (also \( 2 \rightarrow 1 \)) between \( SL(2, \bbC) \rightarrow SO(1,3) \).
Here:
\begin{itemize}
\item \( S \) : special
\item \( L \) : linear
\item \( 2 \) : \( 2 \times 2 \)
\item \( \bbC \) : complex.
\end{itemize}
and we say that \( M \in SL(2,\bbC) \) if \( \det M = 1 \), where \( M \) is a complex \( 2 \times 2 \), but not necessarily unitary.
The \( SU(2) \) group is a subset of
\( SL(2,\bbC) \).  In this representation
\( SU(2) \) matrices are \( SL(2,\bbC) \) matrices, but not necessarily the opposite.

We introduce a special notation for the identity matrix
\begin{dmath}\label{eqn:qftLecture19:120}
\sigma^0 \equiv
\begin{bmatrix}
1 & 0 \\
0 & 1
\end{bmatrix}
\end{dmath}
and can now form four vectors in a matrix representation
\begin{dmath}\label{eqn:qftLecture19:140}
x \cdot \sigma
\equiv
x^\mu \sigma_\mu
=
x^0 \sigma^0 - \Bsigma \cdot \Bx
=
\begin{bmatrix}
x^0 - x^3    & -x^1 + i x^2 \\
-x^1 - i x^2 & x^0 + x^3
\end{bmatrix}.
\end{dmath}
Such \( 2 \times 2 \) matrices are Hermitian.  Notice that the space of \( 2 \times 2 \) Hermitian matrices is 4 dimensional.

We found that
\begin{dmath}\label{eqn:qftLecture19:160}
\det( x^\mu \sigma_\mu )
=
(x^0)^2 - \Bx^2.
\end{dmath}

The transformation
\begin{dmath}\label{eqn:qftLecture19:180}
x^\mu \sigma_\mu \rightarrow M \lr{ x^\mu \sigma_\mu } M^\dagger,
\end{dmath}
maps \( 2 \times 2 \) Hermitian matrices to \( 2 \times 2 \) Hermitian matrices using a unit determinant transformation \( M \).  Note that \( M \) is not unitary, as it is an arbitrary (Hermitian) matrix.  In particular \( M M^\dagger \ne 1 \)!
Also note that the determinant of the transformed object is
\begin{dmath}\label{eqn:qftLecture19:200}
\det \lr{ M \lr{ x^\mu \sigma_\mu } M^\dagger }
=
1 \times \det \lr{ x^\mu \sigma_\mu } \times 1,
\end{dmath}
since \( \det M = 1 \), so that we see that the Lorentz invariant length is preserved by such a transformation.  This can be expressed as
\begin{equation}\label{eqn:qftLecture19:220}
x \cdot \sigma \rightarrow M x \cdot \sigma M^\dagger = x' \cdot \sigma,
\end{equation}
where \( (x')^2 = x^2 \).

Motivated by this \( SL(2,\bbC) \rightarrow SO(1,3) \) correspondence, postulate that we study two component objects
\begin{dmath}\label{eqn:qftLecture19:240}
U(x) =
\begin{bmatrix}
U_1(x) \\
U_2(x) \\
\end{bmatrix},
\end{dmath}
where \( x = (x^0, x^1, x^2, x^3) \) is a four-vector, and assume that such objects transform as follows in \( SO(1,3) \)

\begin{dmath}\label{eqn:qftLecture19:260}
\begin{aligned}
U(x) \rightarrow U'(x') &= M^\dagger U(x) \\
x^\mu \rightarrow {x'}^\mu &= \ultensor{\Lambda}{\mu}{\nu} x^\nu,
\end{aligned}
\end{dmath}
where \( M^\dagger \) is the one giving rise to \( \Lambda \).  To understand what is meant by ``giving rise to'', consider
\begin{dmath}\label{eqn:qftLecture19:280}
M x^\mu \sigma_\mu M^\dagger = {x'}^\nu \sigma_\nu = \sigma_\nu \ultensor{\Lambda}{\nu}{\mu} x^\mu,
\end{dmath}
and this holds for all \( x^\mu \), we must have
\boxedEquation{eqn:qftLecture19:300}{
M \sigma_\mu M^\dagger = \sigma_\nu \ultensor{\Lambda}{\nu}{\mu}.
}

\maketheorem{Transformation of \( U^\dagger(x) \sigma_\mu U(x) \)}{thm:qftLecture19:1121}{
\( U^\dagger(x) \sigma_\mu U(x) \) transforms as a four vector.
} % theorem
Proof:
\begin{dmath}\label{eqn:qftLecture19:320}
U^\dagger(x) \sigma_\mu U(x)
\rightarrow
{U'}^\dagger(x') \sigma_\mu U'(x')
=
(M^\dagger U(x))^\dagger \sigma_\mu M^\dagger U(x)
=
U^\dagger(x) \lr{ M \sigma_\mu M^\dagger } U(x)
=
U^\dagger(x) \sigma_\nu U(x) \ultensor{\Lambda}{\nu}{\mu}
\end{dmath}
so we find that \( U^\dagger(x) \sigma_\mu U(x) \) transforms as a four vector as claimed.

\maketheorem{Transformation of partials.}{thm:qftLecture19:1120}{
The four-gradient coordinates transform as a four vector
\begin{equation*}
(\partial_\mu)' =
\ultensor{(\Lambda^{-1})}{\sigma}{\mu}
\partial_\sigma.
\end{equation*}
} % theorem

Proof\footnote{In class we proved this by considering the transformation properties of a direction derivative \( dx^\mu \cdot \partial_\mu \), but that isn't the method that seems most intuitive to me.}:
Inverting the transformation relation
\begin{dmath}\label{eqn:qftLecture19:1694}
{x'}^\mu =
\ultensor{\Lambda}{\mu}{\nu} x^\nu,
\end{dmath}
gives
\begin{equation}\label{eqn:qftLecture19:1714}
x^\sigma
=
\ultensor{(\Lambda^{-1})}{\sigma}{\mu}
\ultensor{\Lambda}{\mu}{\nu} x^\nu
=
\ultensor{(\Lambda^{-1})}{\sigma}{\mu}
{x'}^\mu,
\end{equation}
so
\begin{dmath}\label{eqn:qftLecture19:1734}
\partial_\mu
\rightarrow
(\partial_\mu)'
=
\PD{{x'}^\mu}{}
=
\PD{{x'}^\mu}{x^\sigma}
\PD{x^\sigma}{}
=
\ultensor{(\Lambda^{-1})}{\sigma}{\mu}
\PD{x^\sigma}{}
=
\ultensor{(\Lambda^{-1})}{\sigma}{\mu} \partial_\sigma.
\end{dmath}
%Because a four vector dot product is Lorentz invariant, including a directional derivative \( a^\mu \cdot \partial_\mu \), we have
%\begin{dmath}\label{eqn:qftLecture19:340}
%a^\mu \PD{x^\mu} {}
%= {a'}^\nu \PD{{x'}^\mu} {}
%= \ultensor{\Lambda}{\nu}{\mu} a^\mu \PD{{x'}^\nu}{}.
%\end{dmath}
%As this holds for all \( a^\mu \) we must have
%\begin{dmath}\label{eqn:qftLecture19:360}
%\PD{x^\mu}{} =
%\PD{{x'}^\nu}{}
% \ultensor{\Lambda}{\nu}{\mu},
%\end{dmath}
%and can multiply by \( \ultensor{(\Lambda^{-1})}{\mu}{\sigma} \) to find
%\begin{dmath}\label{eqn:qftLecture19:380}
%\PD{{x'}^\sigma}{} =
%\PD{x^\mu}{}
% \ultensor{(\Lambda^{-1})}{\mu}{\sigma},
%\end{dmath}
%as stated.

\maketheorem{Transformation of \( U^\dagger \sigma^\mu \partial_\mu U \)}{thm:qftLecture19:1122}{
\( U^\dagger \sigma^\mu \partial_\mu U \) transforms as a four vector.
} % theorem

Proof:
\begin{dmath}\label{eqn:qftLecture19:400}
U^\dagger(x) \sigma^\mu \PD{x^\mu}{} U(x)
\rightarrow
{U'}^\dagger(x') \sigma^\mu \PD{{x'}^\mu}{} U'(x')
=
\ultensor{\Lambda}{\mu}{\nu}
U^\dagger(x) \sigma^\nu
\PD{{x'}^\mu}{}
 \ultensor{(\Lambda^{-1})}{\mu'}{\mu}
U(x)
=
U^\dagger(x) \sigma^\nu
\PD{{x'}^\mu}{}
 \ultensor{\delta}{\mu'}{\nu}
U(x)
=
U^\dagger(x) \sigma^\nu
\PD{{x}^\nu}{}
U(x)
\end{dmath}

We can now define
\makedefinition{Weyl action (name?)}{dfn:qftLecture19:1234}{
We may construct the following Lorentz invariant action
\begin{equation*}
S_{\text{Weyl}} = \int d^4 x i U^\dagger(x) \sigma^\mu \partial_\mu U(x),
\end{equation*}
where \( U(x) \) is a Weyl spinor.
} % definition

The \( i \) factor here is so that the action is real.  This can be seen by noting that \( (i\sigma^\mu)^\dagger = -i \sigma^\mu \) and integrating the Hermitian conjugate by parts

\begin{subequations}
\label{eqn:qftLecture19:1314}
\begin{equation}\label{eqn:qftLecture19:1334}
\lr{ i \sigma^0 }^\dagger = {\begin{bmatrix}
0 & i \\
i & 0
\end{bmatrix}}^\dagger = -i \sigma^0
\end{equation}
\begin{equation}\label{eqn:qftLecture19:1354}
\lr{ i \sigma^1 }^\dagger = {\begin{bmatrix}
0 & i \\
i & 0
\end{bmatrix}}^\dagger = -i \sigma^1
\end{equation}
\begin{equation}\label{eqn:qftLecture19:1374}
\lr{ i \sigma^2 }^\dagger =
{\begin{bmatrix}
0 & 1 \\
-1 & 0
\end{bmatrix}}^\dagger = -i \sigma^2
\end{equation}
\begin{equation}\label{eqn:qftLecture19:1394}
\lr{ i \sigma^3 }^\dagger =
{\begin{bmatrix}
i & 0 \\
0 & -i
\end{bmatrix}}^\dagger = -i \sigma^3
\end{equation}
\end{subequations}

\begin{dmath}\label{eqn:qftLecture19:1414}
S_{\text{Weyl}}^\dagger
= \int d^4 x
\partial_\mu U^\dagger(x) (i\sigma^\mu)^\dagger U(x)
=
-\int d^4 x
\partial_\mu U^\dagger(x) i \sigma^\mu U(x)
=
-
\int d^4 x
\partial_\mu \lr{ U^\dagger(x) i \sigma^\mu U(x) }
+\int d^4 x
U^\dagger(x) i \sigma^\mu \partial_\mu U(x)
=
\int d^4 x
U^\dagger(x) i \sigma^\mu \partial_\mu U(x)
=
S_{\text{Weyl}},
\end{dmath}
where it was assumed that any boundary terms vanish.

\maketheorem{Weyl equation.}{dfn:qftLecture19:1254}{
Variation of the action \cref{dfn:qftLecture19:1234} gives rise to the equations of motion
\begin{equation*}
\sigma^\mu \PD{x^\mu}{} U = 0
%\PD{x^\mu}{} U^\dagger \sigma^\mu &= 0
\end{equation*}
which is called the Weyl equation.
} % theorem
Proof:
\begin{dmath}\label{eqn:qftLecture19:1294}
\delta S
=
i \int d^4 x
\lr{
   \delta U^\dagger \sigma^\mu \partial_\mu U
   + U^\dagger \sigma^\mu \partial_\mu \delta U
}
=
i \int d^4 x
\lr{
   \delta U^\dagger \sigma^\mu \partial_\mu U
   +
   \partial_\mu \lr{ U^\dagger \sigma^\mu \delta U }
   -
   (\partial_\mu U^\dagger) \sigma^\mu \delta U
}
=
i \int d^4 x
\lr{
   \delta U^\dagger \lr{ \sigma^\mu \partial_\mu U }
   -
   \lr{ (\partial_\mu U^\dagger) \sigma^\mu } \delta U
}
=
\int d^4 x
\lr{
   \delta U^\dagger \lr{ i \sigma^\mu \partial_\mu U }
+
   \lr{ \delta U^\dagger \lr{ i \sigma^\mu \partial_\mu U } }^\dagger
}.
\end{dmath}
Requiring this to vanish for all variations \( \delta U^\dagger \) proves the result.

Written out explicitly in matrix form, the Weyl equation is
\begin{dmath}\label{eqn:qftLecture19:440}
\begin{bmatrix}
\partial_0 + \partial_3 & \partial_1 - i \partial_2  \\
\partial_1 + i \partial_2 & \partial_0 - \partial_3
\end{bmatrix}
\begin{bmatrix}
U_1 \\
U_2
\end{bmatrix}
= 0,
\end{dmath}
or
\begin{subequations}
\label{eqn:qftLecture19:460}
\begin{dmath}\label{eqn:qftLecture19:480}
(\partial_0 + \partial_3)U_1 + (\partial_1 - i \partial_2) U_2 = 0
\end{dmath}
\begin{dmath}\label{eqn:qftLecture19:500}
(\partial_1 + i \partial_2)U_1 + (\partial_0 - \partial_3) U_2 = 0.
\end{dmath}
\end{subequations}

\maketheorem{Weyl equation relation to the massless Klein-Gordon equation.}{thm:qftLecture19:1274}{
The Weyl equation is equivalent to a set of massless Klein-Gordon equations.
\begin{equation*}
\partial_\mu \partial^\mu U_k = 0,
\end{equation*}
for \( k = 1, 2\).
} % theorem
Proof:

Multiplying \cref{eqn:qftLecture19:480} by \( \partial_1 + i \partial_2 \) gives
\begin{dmath}\label{eqn:qftLecture19:520}
\lr{ \partial_1 + i \partial_2 }
\Biglr{
   (\partial_0 + \partial_3)U_1 + (\partial_1 - i \partial_2) U_2
}
=
(\partial_0 + \partial_3)
\lr{ \partial_1 + i \partial_2 }
U_1 +
\lr{ \partial_1 + i \partial_2 }
(\partial_1 - i \partial_2)
U_2
=
-(\partial_0 + \partial_3)
(\partial_0 - \partial_3) U_2
+
\lr{ \partial_1 + i \partial_2 }
(\partial_1 - i \partial_2)
U_2
=
\lr{
   -\partial_{00}
   + \partial_{33}
   + \partial_{11}
   + \partial_{22}
} U_2
=
\lr{
   -\partial_{0} \partial^0
   - \partial_{3} \partial^3
   - \partial_{1} \partial^1
   - \partial_{2} \partial^2
} U_2
=
- \partial_\mu \partial^\mu U_2.
\end{dmath}
Similarly, multiplying \cref{eqn:qftLecture19:500} by \( \partial_1 - i \partial_2 \) we find
\begin{dmath}\label{eqn:qftLecture19:1274}
0
=
\lr{ \partial_1 - i \partial_2 }
\Biglr{
   (\partial_1 + i \partial_2)U_1
+ (\partial_0 - \partial_3) U_2
}
=
\lr{ \partial_{11} + \partial_{22} } U_1
+
(\partial_0 - \partial_3)
\underbrace{\lr{ \partial_1 - i \partial_2 }
U_2
}_{= -(\partial_0 + \partial_3) U_1}
=
\lr{
\partial_{11} + \partial_{22} - \partial_{00} + \partial_{33}
} U_1
=
- \partial_\mu \partial^\mu U_1.
\end{dmath}

Because \( S_{\text{Weyl}} \) results in a massless Klein-Gordon equation, this is no good for electrons, and we have to look for a different action.

%\paragraph{Q:} Can we write bilinear inut w/ no derivatives.
%
%\paragraph{A:} Yes but, ...
%
\paragraph{Claim:} \( U^\T \sigma_2 U \) is the only bilinear Lorentz invariant that we can add to the action.

An action like:
\begin{dmath}\label{eqn:qftLecture19:560}
\LL_{\text{mass}} = \inv{2} m U^\T \sigma_2 U + \inv{2} m^\conj U^\dagger \sigma_2 (U^\dagger)^\T,
\end{dmath}
may exist in nature (we don't know), and are called Majorana neutrino masses.  The problem with such a Lagrangian density is that it breaks \( U(1) \) symmetry.  In particular \( U \rightarrow e^{i \alpha} U \) symmetry of the kinetic term.  This means that the particle associated with such a Lagrangian cannot be charged.

Recall that we introduced electromagnetic potentials into NRQM with
\begin{dmath}\label{eqn:qftLecture19:580}
i \Hbar \PD{t}{} \Psi = \inv{2m} \lr{ \spacegrad - e \BA }^2 \Psi
\end{dmath}
which is a gauge transformation.  We'd like to have this capability.

What we can do instead and maintain \( U(1) \) symmetries, is to introduce two U's, like
\begin{dmath}\label{eqn:qftLecture19:600}
\LL_{\text{mass}} = \inv{2} m U_1^\T \sigma_2 U_2 + \inv{2} m^\conj U_2^\dagger \sigma_2 (U_1^\dagger)^\T
\end{dmath}
What we are really doing is assembling a four component spinor out of the two U's.

\section{Lorentz symmetry.}

We want to examine the Lorentz invariance of \(U^\T \sigma_2 U\), but need an intermediate result first.
\makelemma{Transpose of Pauli vector representation}{lemma:qftLecture19:21}{
For any \( \Bx \in \bbR^3 \)
\begin{equation*}
(\Bsigma \cdot \Bx)^\T = -\sigma^2 (\Bsigma \cdot \Bx) \sigma^2,
\end{equation*}
or more compactly
\begin{equation*}
\Bsigma^\T = -\sigma^2 \Bsigma \sigma^2.
\end{equation*}
Geometrically, this transposition operation reflects \( \Bx \) about the y-axis.
} % lemma
Proving
\cref{lemma:qftLecture19:21}
is well suited to software (FIXME: link: diracWeylMatrixRepresentationAndIdentities.nb), but can also be done algebraically with ease.  First note that
\begin{dmath}\label{eqn:qftLecture19:740}
\begin{aligned}
\sigma_1^\T &= \sigma_1 \\
\sigma_2^\T &= -\sigma_2 \\
\sigma_3^\T &= \sigma_3
\end{aligned}
\end{dmath}
which means that
\begin{dmath}\label{eqn:qftLecture19:1454}
(\Bsigma \cdot \Bx)^\T
=
\sigma^1 x^1 - \sigma^2 x^2 + \sigma^3 x^3
=
\sigma^2 \sigma^2
\lr{
\sigma^1 x^1 - \sigma^2 x^2 + \sigma^3 x^3
}
=
\sigma^2
\lr{
-\sigma^1 x^1 - \sigma^2 x^2 - \sigma^3 x^3
} \sigma^2
=
- \sigma^2 (\Bsigma \cdot \Bx)^\T \sigma^2.
\end{dmath}

Now we are ready to proceed.
\maketheorem{\(U^\T \sigma_2 U\) invariance}{thm:qftLecture19:10}{
\(U^\T \sigma_2 U\) is Lorentz invariant.
} % theorem

Proof:
\begin{dmath}\label{eqn:qftLecture19:620}
U^\T \sigma_2 U
\rightarrow
{U'}^\T \sigma_2 U'
=
U^\T {M^\dagger}^\T \sigma_2 M^\dagger U,
\end{dmath}
where \( U' = M^\dagger U \) and \( {U'}^\T = U^\T {M^\dagger}^T \).

Note that if we can show that \( {M^\dagger}^\T \sigma_2 M^\dagger = \sigma_2 \), then we are done.

It is simple to show that any
\begin{dmath}\label{eqn:qftLecture19:640}
U = e^{i \Bsigma \cdot \Ba},
\end{dmath}
for \( \Ba \in \bbR^{3} \), has eigenvalues \( \pm i \Abs{\Ba} \).  The determinant of such a matrix is
\begin{equation}\label{eqn:qftLecture19:1434}
\det U = \begin{vmatrix}
e^{i \Abs{\Ba}} & 0 \\
0 & e^{-i \Abs{\Ba}}
\end{vmatrix}
= 1,
\end{equation}
so we see that such a matrix has the \( U^\dagger U = 1 \) and \( \det U = 1 \) properties that we desire for elements of \( SU(2) \)\footnote{In class the suitability of \( e^{i \Bsigma \cdot \Ba} \) as an element of \( SU(2) \) was demonstrated with an argument that diagonalizable matrices satisfy \( \det e^A = e^{\trace{A}} \)}.
We haven't shown that all matrices \( U \in SU(2) \) can be written in this form, but let's assume that's the case.

\paragraph{Claim:}  Generalizing from the exponential form of \( SU(2) \) elements seen above, we assume that any \( SL(2,\bbC) \) matrix \( M \) can be written as
\begin{dmath}\label{eqn:qftLecture19:680}
M^\dagger =
e^{i \Bsigma \cdot (\Ba + i \Bb) },
\end{dmath}
for \( \Ba, \Bb \in \bbR^3 \).

The transpose of an exponential of a sigma matrix goes like
\begin{dmath}\label{eqn:qftLecture19:1474}
(e^{\Bsigma \cdot \Bu})^\T
=
\sum_{k = 0}^\infty \inv{k!}\lr{(\Bsigma \cdot \Bu)^k}^\T
=
\sum_{k = 0}^\infty \inv{k!}\lr{- \sigma_2 (\Bsigma \cdot \Bu) \sigma_2}^k
=
\sigma_2 \lr{
   \sum_{k = 0}^\infty \inv{k!}\lr{- \Bsigma \cdot \Bu }^k
}
\sigma_2
=
\sigma_2
e^{-\Bsigma \cdot \Bu}
\sigma_2,
\end{dmath}
so
\begin{dmath}\label{eqn:qftLecture19:700}
{M^\dagger}^\T \sigma_2 M^\dagger
=
\lr{ e^{i \Bsigma \cdot (\Ba + i \Bb) } }^\T
\sigma_2
e^{i \Bsigma \cdot (\Ba + i \Bb) }
=
\lr{
   \sigma_2
   e^{-i \Bsigma \cdot (\Ba + i \Bb) }
   \sigma_2
}
\sigma_2
e^{i \Bsigma \cdot (\Ba + i \Bb) }
=
\sigma_2,
\end{dmath}
which is the result required to finish the proof of \cref{thm:qftLecture19:10}\footnote{A slightly different derivation was done in class, but this one makes more sense to me.}.

%%%\begin{dmath}\label{eqn:qftLecture19:760}
%%%{M^\dagger}^\T \sigma_2 M^\dagger
%%%=
%%%\sigma_2
%%%\lr{
%%%   \sigma_2 e^{-i (\Ba + i \Bb) \cdot (\sigma_2 \Bsigma \sigma_2) } \sigma_2
%%%}
%%%e^{i (\Ba + i \Bb) \cdot \Bsigma }
%%%=
%%%\sigma_2
%%%\lr{
%%%   \sigma_2 e^{-i (\Ba + i \Bb) \cdot (\sigma_2 \sigma_2 \Bsigma \sigma_2 \sigma_2) }
%%%}
%%%e^{i (\Ba + i \Bb) \cdot \Bsigma }
%%%=
%%%\sigma_2
%%%\lr{
%%%   \sigma_2 e^{-i (\Ba + i \Bb) \cdot \Bsigma }
%%%}
%%%e^{i (\Ba + i \Bb) \cdot \Bsigma }
%%%=
%%%\sigma_2
%%%\end{dmath}
%%%
%%%where we used an argument like
%%%\begin{dmath}\label{eqn:qftLecture19:780}
%%%e^{A} \sigma_2
%%%= \sigma_2( 1 + A + \inv{2} A^2 ...)
%%%= ?
%%%\end{dmath}
%%%to move the \( \sigma_2 \) into the exponential argument.
%%%
\section{Dirac matrices.}
\makedefinition{Dirac matrices.}{dfn:qftLecture19:1754}{
The Dirac matrices \( \gamma^\mu, \mu \in \setlr{0,1,2,3} \) are matrices that satisfy
\begin{equation*}
\symmetric{\gamma^\mu}{\gamma^\nu} = 2 g^{\mu\nu},
\end{equation*}
that is
\begin{equation*}
\gamma^\mu \gamma^\nu + \gamma^\nu \gamma^\mu = 2 g^{\mu\nu},
\end{equation*}
We will use the explicit \( 4 \times 4 \) matrix representation
\begin{equation*}%\label{eqn:qftLecture19:880}
\gamma^0 =
\begin{bmatrix}
0 & 1 \\
1 & 0
\end{bmatrix},
\end{equation*}
and
\begin{equation*}%\label{eqn:qftLecture19:900}
\gamma^i=
\begin{bmatrix}
0 & \sigma^i \\
-\sigma^i & 0
\end{bmatrix}.
\end{equation*}
} % definition
The metric relations can also be written explicitly in the handy form
\begin{dmath}\label{eqn:qftLecture19:860}
\begin{aligned}
\lr{\gamma^0}^2 &= 1 \\
\lr{\gamma^i}^2 &= -1.
\end{aligned}
\end{dmath}
Written out explicitly, these matrices are
\begin{dmath}\label{eqn:qftLecture19:1774}
\begin{aligned}
\gamma^0
&=
\begin{bmatrix}
 0 & 0 & 1 & 0 \\
 0 & 0 & 0 & 1 \\
 1 & 0 & 0 & 0 \\
 0 & 1 & 0 & 0 \\
\end{bmatrix}
,\qquad
\gamma^1 =
\begin{bmatrix}
 0 & 0 & 0 & 1 \\
 0 & 0 & 1 & 0 \\
 0 & -1 & 0 & 0 \\
 -1 & 0 & 0 & 0 \\
\end{bmatrix},
\\
\gamma^2 &=
\begin{bmatrix}
 0 & 0 & 0 & -i \\
 0 & 0 & i & 0 \\
 0 & i & 0 & 0 \\
 -i & 0 & 0 & 0 \\
\end{bmatrix},
\qquad
\gamma^3 =
\begin{bmatrix}
 0 & 0 & 1 & 0 \\
 0 & 0 & 0 & -1 \\
 -1 & 0 & 0 & 0 \\
 0 & 1 & 0 & 0 \\
\end{bmatrix}.
\end{aligned}
\end{dmath}

We will see (HW4) that Lorentz transformations take the form
\begin{dmath}\label{eqn:qftLecture19:1494}
x' \cdot \gamma = \Lambda_{1/2}^{-1} \lr{ x \cdot \gamma } \Lambda_{1/2},
\end{dmath}
where
\begin{dmath}\label{eqn:qftLecture19:1514}
\Lambda_{1/2}
=
e^{-\frac{i}{2} \omega_{\mu\nu} S^{\mu\nu}},
\end{dmath}
where
\begin{dmath}\label{eqn:qftLecture19:920}
S^{\mu\nu} = \frac{i}{4} \antisymmetric{\gamma^\mu}{\gamma^\nu}.
\end{dmath}
In particular
\begin{dmath}\label{eqn:qftLecture19:940}
S^{0k}
= \frac{i}{4} \antisymmetric{\gamma^0}{\gamma^k}
= \frac{i}{2} \gamma^0 \gamma^k
= \frac{i}{2}
\begin{bmatrix}
0 & 1 \\
1 & 0
\end{bmatrix}
\begin{bmatrix}
0 & \sigma^k \\
-\sigma^k & 0
\end{bmatrix}
=
\frac{i}{2}
\begin{bmatrix}
-\sigma^k & 0 \\
0 & \sigma^k
\end{bmatrix}
\end{dmath}
will generate boosts, whereas (for \( j \ne k \))
\begin{dmath}\label{eqn:qftLecture19:960}
S^{jk}
= \frac{i}{4} \antisymmetric{\gamma^j}{\gamma^k}
= \frac{i}{2} \gamma^j \gamma^k
=
\frac{i}{2}
\begin{bmatrix}
0 & \sigma^j \\
-\sigma^j & 0
\end{bmatrix}
\begin{bmatrix}
0 & \sigma^k \\
-\sigma^k & 0
\end{bmatrix}
=
-\frac{i}{2}
\begin{bmatrix}
\sigma^k \sigma^j & 0 \\
0 & \sigma^k \sigma^j
\end{bmatrix}
= \inv{2} \epsilon^{jkl}
\begin{bmatrix}
\sigma^l & 0 \\
0 & \sigma^l
\end{bmatrix},
\end{dmath}
are rotations (and in this case, are Hermitian).

The explicit expansion of the half Lorentz transformation operator is
\begin{dmath}\label{eqn:qftLecture19:1000}
\begin{aligned}
\Lambda_{1/2}
&=
e^{-\frac{i}{2} \omega_{\mu\nu} S^{\mu\nu} } \\
&=
e^{
-i
\omega_{0k} S^{0k}
-\frac{i}{2}
\omega_{jk} S^{jk}
} \\
&=
\exp\lr{
-\inv{2}
\begin{bmatrix}
\omega_{0k} \sigma^k & 0 \\
0 & -\omega_{0k} \sigma^k
\end{bmatrix}
- \frac{i}{4}
\begin{bmatrix}
\omega_{jk} \epsilon^{jkl} \sigma^l & 0 \\
0 & \omega_{jk} \epsilon^{jkl} \sigma^l
\end{bmatrix}
} \\
&=
\begin{bmatrix}
e^{-\lr{ \inv{2} \omega_{0k} \sigma^k + \frac{i}{4} \omega_{jk} e^{jkl}  \sigma^l }} & 0 \\
0 & e^{-\lr{ -\inv{2} \omega_{0k} \sigma^k + \frac{i}{4} \omega_{jk} e^{jkl} \sigma^l }}  \\
\end{bmatrix}
\end{aligned}
\end{dmath}
where the \( 1/2\) factor of \( \omega_{0i} \) vanished because we had a sum over \( 0i \) and \( i0 \) which have been grouped.

\makelemma{Some Dirac matrix identities.}{thm:qftLecture19:51}{
\begin{equation*}
(\gamma^0)^\dagger = \gamma^0
\end{equation*}
\begin{equation*}
(\gamma^k)^\dagger = - \gamma^k
\end{equation*}
\begin{equation*}
\gamma^0 (i \gamma^\mu)^\dagger \gamma^0 = -i \gamma^\mu.
\end{equation*}
} % theorem

The first two are clear from inspection of
\cref{eqn:qftLecture19:1774}.
For the last, for \( \mu = 0 \)
\begin{dmath}\label{eqn:qftLecture19:1554}
\gamma^0 (i \gamma^0)^\dagger \gamma^0
=
\gamma^0 (\gamma^0)^\dagger (-i) \gamma^0
=
- i \gamma^0 \gamma^0\gamma^0
=
- i \gamma^0,
\end{dmath}
and for \( \mu = k \ne 0 \)
\begin{dmath}\label{eqn:qftLecture19:1574}
\gamma^0 (i \gamma^k)^\dagger \gamma^0
=
\gamma^0 (-i) (-\gamma^k) \gamma^0
=
+ i \gamma^0 \gamma^k \gamma^0
=
- i \gamma^0 \gamma^0 \gamma^k
=
- i \gamma^k,
\end{dmath}
which completes the proof.

\section{Dirac Lagrangian.}

We postulate that there is a four-component object
\begin{equation}\label{eqn:qftLecture19:800}
\Psi =
\begin{bmatrix}
\psi_1 \\
\psi_2 \\
\psi_3 \\
\psi_4
\end{bmatrix}
\qquad
\Psi^\dagger = \lr{ \psi_1^\conj, \psi_2^\conj, \psi_3^\conj, \psi_4^\conj },
\end{equation}
where \( \psi_\mu \)'s are all complex fields, and assume that the fields transform as
\begin{equation}\label{eqn:qftLecture19:1040}
\Psi(x) \rightarrow \Psi'(x') = \Lambda_{1/2} \Psi(x),
\end{equation}
where our vectors transform in the usual \( x \rightarrow x' = \Lambda x \) fashion, where the incremental form of the Lorentz transformation is the usual
\begin{dmath}\label{eqn:qftLecture19:980}
\ultensor{\Lambda}{\mu}{\nu} =
\ultensor{\delta}{\mu}{\nu}
+
\ultensor{\omega}{\mu}{\nu}
+ O(\omega^2).
\end{dmath}

\makedefinition{Overbar operator (name?).}{dfn:qftLecture19:1534}{
\begin{equation*}
\overbar{\Psi} = \Psi^\dagger \gamma^0.
\end{equation*}
} % definition

\makedefinition{Dirac Lagrangian.}{dfn:qftLecture19:31}{
\begin{equation*}
\LL_{\text{Dirac}} = \overbar{\Psi}(x)\lr{ i \gamma^\mu \partial_\mu - m } \Psi(x).
\end{equation*}
} % definition

Armed with \cref{thm:qftLecture19:51} we can now show the following.
\maketheorem{The Dirac action is a real Lorentz scalar.}{thm:qftLecture19:42}{
The action
\begin{equation*}
S = \int d^4 x \overbar{\Psi} \lr{ i \gamma^\mu \partial_\mu - m } \Psi,
\end{equation*}
is a real scalar and is Lorentz invariant.
} % theorem

\paragraph{Real:}
To show that the action is real, we compute it's Hermitian conjugate, apply \cref{thm:qftLecture19:51} and integrate by parts
\begin{dmath}\label{eqn:qftLecture19:1594}
S^\dagger
=
\int d^4 x
\Psi^\dagger
\lr{ -i (\gamma^\mu)^\dagger \lpartial_\mu - m }
(\gamma^0)^\dagger
\Psi
=
\int d^4 x
\Psi^\dagger
\lr{ (i\gamma^\mu)^\dagger \lpartial_\mu - m }
\gamma^0
\Psi
=
\int d^4 x
\Psi^\dagger
\lr{ -\gamma^0 (i\gamma^\mu) \gamma^0 \lpartial_\mu - m }
\gamma^0
\Psi
=
\int d^4 x
\overbar{\Psi}
\lr{ - i\gamma^\mu \lpartial_\mu - m }
\Psi
=
- \int d^4 x
\partial_\mu \lr{
   \overbar{\Psi}
    i\gamma^\mu
   \Psi
}
+ \int d^4 x
   \overbar{\Psi}
    i\gamma^\mu
   \partial_\mu \Psi
-
\int d^4 x
\overbar{\Psi}
m
\Psi
=
\int d^4 x
   \overbar{\Psi}
\lr{
    i\gamma^\mu
   \partial_\mu
-m
}
\Psi
=
S,
\end{dmath}
where \( \partial_\mu \) without an overarrow means the traditional right acting operator, and assuming that the boundary terms vanish.

To show the Lorentz invariance, we will consider just the transformation of the Dirac Lagrangian density.  We need a couple additional pieces of information to do so, the first of which is the transformation property\footnote{Not proven here, but there's an argument for that in \citep{peskin1995introduction} (eq. 3.33).}
\begin{dmath}\label{eqn:qftLecture19:1614}
\overbar{\Psi} \rightarrow \Psi \Lambda^{-1}_{1/2},
\end{dmath}
and (from HW4)
\begin{dmath}\label{eqn:qftLecture19:1634}
\Lambda^{-1}_{1/2} \gamma^\mu \Lambda_{1/2} = \ultensor{\Lambda}{\mu}{\alpha}\gamma^\alpha.
\end{dmath}
The Lagrangian transforms as
\begin{dmath}\label{eqn:qftLecture19:1020}
\overbar{\Psi}(x) \lr{ i \gamma^\mu \partial_\mu - m } \Psi(x)
\rightarrow
\overbar{\Psi'}(x') \gamma^0 \lr{ i \gamma^\mu \PD{{x'}^\mu}{} - m } \Psi'(x')
=
\overbar{\Psi}(x) \Lambda^{-1}_{1/2} \lr{ i \gamma^\mu \ultensor{(\Lambda^{-1})}{\alpha}{\mu} \partial_\alpha - m } \Lambda_{1/2} \Psi(x)
=
\overbar{\Psi}(x)
\lr{ i
\Lambda^{-1}_{1/2}
\gamma^\mu
\Lambda_{1/2}
\ultensor{(\Lambda^{-1})}{\alpha}{\mu} \partial_\alpha - m }
\Psi(x)
%%=
%%\lr{
%%   \Psi^\dagger
%%   \Lambda_{1/2}^{\dagger}
%%   \gamma^0
%%}
%%\lr{
%%      \Lambda_{1/2}
%%   i
%%      {\Lambda^{-1}}_{1/2}
%%      \gamma^\mu
%%      \Lambda_{1/2}
%%   \partial_\alpha
%%   \ultensor{
%%   {\Lambda^{-1}}_{1/2}
%%   }{\alpha}{\mu}
%%}
%%\Psi
%%-
%%m \Psi^\dagger
%%\gamma_0
%%\Lambda_{1/2}
%%\Psi
%%=
%%\Psi^\dagger
%%\lr{
%%      \underbrace{
%%      \Lambda_{1/2}^{\dagger}
%%      \gamma^0
%%      \Lambda_{1/2}
%%      }_{= \gamma^0}
%%   i
%%      \underbrace{
%%      {\Lambda^{-1}}_{1/2}
%%      \gamma^\mu
%%      \Lambda_{1/2}
%%      }_{= \ultensor{\Lambda}{\mu}{\nu} \gamma^\nu}
%%   \partial_\alpha
%%   \ultensor{
%%   {\Lambda^{-1}}_{1/2}
%%   }{\alpha}{\mu}
%%}
%%\Psi
%%-
%%m \Psi^\dagger
%%\underbrace{\Lambda_{1/2}^\dagger
%%\gamma_0
%%\Lambda_{1/2}
%%}_{= \gamma_0}
%%\Psi
%%=
%%\Psi^\dagger \gamma^0 i \gamma^\alpha \partial_\alpha \Psi - m \Psi^\dagger \gamma_0 \Psi
=
\overbar{\Psi} \lr{ i \gamma^\mu \partial_\mu - m } \Psi
\end{dmath}

We find that \( \overbar{\Psi} \Psi = \Psi^\dagger \gamma^0 \Psi \) is a Lorentz scalar, whereas
\( \overbar{\Psi} \gamma^\mu \Psi \) is a 4 vector.

%Can create couplings like \( \overbar{\Psi}_1 \Psi_2 \overbar{\Psi}_3 \Psi_4 \)
%or \( \overbar{\Psi} \gamma_\mu \partial^\mu \Psi \).
%%\cref{fig:l19:l19Fig2}.
%\imageFigure{../figures/phy2403-quantum-field-theory/l19Fig2}{Muon decay.}{fig:l19:l19Fig2}{0.3}
%}
%\EndArticle
