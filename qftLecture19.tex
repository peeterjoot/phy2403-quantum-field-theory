%
% Copyright � 2017 Peeter Joot.  All Rights Reserved.
% Licenced as described in the file LICENSE under the root directory of this GIT repository.
%
%{
\input{../latex/blogpost.tex}
\renewcommand{\basename}{qftLecture19}
\renewcommand{\dirname}{notes/phy2403/}
\newcommand{\keywords}{PHY2403H}
\input{../latex/peeter_prologue_print2.tex}

%\usepackage{phy2403}
\usepackage{peeters_braket}
\usepackage{peeters_layout_exercise}
\usepackage{peeters_figures}
\usepackage{mathtools}
\usepackage{siunitx}
\usepackage{macros_cal} % LL
\usepackage{simplewick}

\newcommand{\ultensor}[3]{{{#1}^{#2}}_{#3}}
\newcommand{\deltathree}[0]{\delta^{(3)}}
\newcommand{\deltafour}[0]{\delta^{(4)}}

\beginArtNoToc
\generatetitle{PHY2403H Quantum Field Theory.  Lecture 19: XXX.  Taught by Prof.\ Erich Poppitz}
%\chapter{XXX}
\label{chap:qftLecture19}

%%Peeter's lecture notes from class.  These may be incoherent and rough.
%%
%%These are notes for the UofT course PHY2403H, Quantum Field Theory, taught by Prof. Erich Poppitz, covering \textchapref{{1}} \citep{peskin1995introduction} content.

\paragraph{DISCLAIMER: Very rough notes from class, with some additional side notes.}

These are notes for the UofT course PHY2403H, Quantum Field Theory, taught by Prof. Erich Poppitz, fall 2018.
%, covering \textchapref{{1}} \citep{peskin1995introduction} content.

\section{Rotations \( SU(2) \rightarrow SO(3) \).}

\index{spinor}
\index{Pauli matrix}
We will map the traditional vector basis to one with matrix basis elements using
\begin{dmath}\label{eqn:qftLecture19:20}
\Bx \rightarrow \Bx \cdot \Bsigma,
\end{dmath}
where \( \Bsigma = (\sigma^1, \sigma^2, \sigma^3) \), and \( \sigma^k \) are the Pauli matrices.

A transformation of the following form
\begin{dmath}\label{eqn:qftLecture19:40}
U^\dagger (\Bx \cdot \Bsigma) U = \Bx' \cdot \Bsigma,
\end{dmath}
where the matrix \( U \) has \( \det U = 1 \) and is unitary \( U^\dagger U = \BOne \),
is length preserving.  We saw that was the case since
\begin{dmath}\label{eqn:qftLecture19:60}
\det \Bx \cdot \Bsigma \sim \Bx^2.
\end{dmath}

We say that there is a double cover, as there are multiple representations of the rotation operator.  In particular \( U \) and \( -U \) of \( SU(2) \) correspond to the same \( SO(3) \) element.

In NRQM, Pauli spinors (or non-relativistic spinors) that were transformed by \( SU(2) \) rotations
\begin{equation}\label{eqn:qftLecture19:80}
\begin{bmatrix}
\Psi_\uparrow(\Bx) \\
\Psi_\downarrow(\Bx)
\end{bmatrix}
\rightarrow
\begin{bmatrix}
\Psi'_\uparrow(\Bx') \\
\Psi'_\downarrow(\Bx')
\end{bmatrix}
=
U
\begin{bmatrix}
\Psi_\uparrow(\Bx) \\
\Psi_\downarrow(\Bx)
\end{bmatrix}.
\end{equation}

Note that the matrix \( U \) can be built explicitly.  For example, it may be built up using Euler angles as sketched in \cref{fig:l19:l19Fig1}.
\imageFigure{../figures/phy2403-quantum-field-theory/l19Fig1}{Euler angle rotations.}{fig:l19:l19Fig1}{0.3}
or algebraically
\begin{dmath}\label{eqn:qftLecture19:100}
U = e^{i \psi \sigma_3/2 } e^{i \theta \sigma_1/2} e^{i \phi \sigma_3/2}.
\end{dmath}

\section{Weyl spinors.}
We will see that there is generalization of Pauli spinors, called Weyl spinors, but we will have to introduce 4 component objects.

We'd like to argue that there is a correspondence (also \( 2 \rightarrow 1 \)) between \( SL(2, \bbC) \rightarrow SO(1,3) \).
Here:
\begin{itemize}
\item \( S \) : special
\item \( L \) : linear
\item \( 2 \) : \( 2 \times 2 \)
\item \( \bbC \) : complex.
\end{itemize}
and we say that \( M \in SL(2,\bbC) \) if \( \det M = 1 \), where \( M \) is a complex \( 2 \times 2 \), but not necessarily unitary.
The \( SU(2) \) group is a subset of
\( SL(2,\bbC) \).  In this representation
\( SU(2) \) matrices are \( SL(2,\bbC) \) matrices, but not neccessarily the opposite.

We introduce a special notation for the identity matrix
\begin{dmath}\label{eqn:qftLecture19:120}
\sigma^0 \equiv
\begin{bmatrix}
1 & 0 \\
0 & 1
\end{bmatrix}
\end{dmath}
and can now form four vectors in a matrix representation
\begin{dmath}\label{eqn:qftLecture19:140}
x \cdot \sigma
\equiv
x^\mu \sigma_\mu \equiv
x^0 \sigma^0 + \Bx \cdot \Bsigma
=
\begin{bmatrix}
x^0 + x^3 & x^1 - i x^2 \\
x^1 + i x^2 & x^0 - x^3
\end{bmatrix}.
\end{dmath}
Such \( 2 \times 2 \) matrices are Hermitian.  Notice that the space of \( 2 \times 2 \) Hermitian matrices is 4 dimensional.

We found that
\begin{dmath}\label{eqn:qftLecture19:160}
\det( x^\mu \sigma_\mu )
=
(x^0)^2 - \Bx^2.
\end{dmath}

The transformation
\begin{dmath}\label{eqn:qftLecture19:180}
x^\mu \sigma_\mu \rightarrow M \lr{ x^\mu \sigma_\mu } M^\dagger,
\end{dmath}
maps \( 2 \times 2 \) Hermitian matrices to \( 2 \times 2 \) Hermitian matrices using a unit determinant transformation \( M \).  Note that \( M \) is not unitary, as it is an arbitrary (Hermitian) matrix.  In particular \( M M^\dagger \ne 1 \)!
Also note that the determinant of the transformed object is
\begin{dmath}\label{eqn:qftLecture19:200}
\det \lr{ M \lr{ x^\mu \sigma_\mu } M^\dagger }
=
1 \times \det \lr{ x^\mu \sigma_\mu } \times 1,
\end{dmath}
since \( \det M = 1 \), so that we see that the Lorentz invariant length is preserved by such a transformation.  This can be expressed as
\begin{equation}\label{eqn:qftLecture19:220}
x \cdot \sigma \rightarrow M x \cdot \sigma M^\dagger = x' \cdot \sigma,
\end{equation}
where \( (x')^2 = x^2 \).

Motivated by this \( SL(2,\bbC) \rightarrow SO(1,3) \) correspondence, postulate that we study two component objects
\begin{dmath}\label{eqn:qftLecture19:240}
U(x) =
\begin{bmatrix}
U_1(x) \\
U_2(x) \\
\end{bmatrix},
\end{dmath}
where \( x = (x^0, x^1, x^2, x^3) \) is a four-vector, and assume that such objects transform as follows in \( SO(1,3) \)

\begin{dmath}\label{eqn:qftLecture19:260}
\begin{aligned}
U(x) \rightarrow U'(x') &= M^\dagger U(x) \\
x^\mu \rightarrow {x'}^\mu &= \ultensor{\Lambda}{\mu}{\nu} x^\nu,
\end{aligned}
\end{dmath}
where \( M^\dagger \) is the one giving rise to \( \Lambda \).  To understand what is meant by ``giving rise to'', consider
\begin{dmath}\label{eqn:qftLecture19:280}
M x^\mu \sigma_\mu M^\dagger = {x'}^\nu \sigma_\nu = \sigma_\nu \ultensor{\Lambda}{\nu}{\mu} x^\mu,
\end{dmath}
and this holds for all \( x^\mu \), we must have
\boxedEquation{eqn:qftLecture19:300}{
M \sigma_\mu M^\dagger = \sigma_\nu \ultensor{\Lambda}{\nu}{\mu}.
}

\maketheorem{Transformation of \( U^\dagger(x) \sigma_\mu U(x) \)}{thm:qftLecture19:1121}{
\( U^\dagger(x) \sigma_\mu U(x) \) transforms as a four vector.
} % theorem
Proof:
\begin{dmath}\label{eqn:qftLecture19:320}
U^\dagger(x) \sigma_\mu U(x)
\rightarrow
{U'}^\dagger(x') \sigma_\mu U'(x')
=
(M^\dagger U(x))^\dagger \sigma_\mu M^\dagger U(x)
=
U^\dagger(x) \lr{ M \sigma_\mu M^\dagger } U(x)
=
U^\dagger(x) \sigma_\nu U(x) \ultensor{\Lambda}{\nu}{\mu}
\end{dmath}
so we find that \( U^\dagger(x) \sigma_\mu U(x) \) transforms as a four vector as claimed.

\maketheorem{Transformation of partials.}{thm:qftLecture19:1120}{
The four-gradient coordinates transform as a four vector
\begin{equation*}
\partial'_\sigma = \partial_\mu \ultensor{(\Lambda^{-1})}{\mu}{\sigma}.
\end{equation*}
} % theorem

Proof.  Because a four vector dot product is Lorentz invariant, including a directional derivative \( a^\mu \cdot \partial_\mu \), we have
\begin{dmath}\label{eqn:qftLecture19:340}
a^\mu \PD{x^\mu} {}
= {a'}^\nu \PD{{x'}^\mu} {}
= \ultensor{\Lambda}{\nu}{\mu} a^\mu \PD{{x'}^\nu}{}.
\end{dmath}
As this holds for all \( a^\mu \) we must have
\begin{dmath}\label{eqn:qftLecture19:360}
\PD{x^\mu}{} =
\PD{{x'}^\nu}{}
 \ultensor{\Lambda}{\nu}{\mu},
\end{dmath}
and can multiply by \( \ultensor{(\Lambda^{-1})}{\mu}{\sigma} \) to find
\begin{dmath}\label{eqn:qftLecture19:380}
\PD{{x'}^\sigma}{} =
\PD{x^\mu}{}
 \ultensor{(\Lambda^{-1})}{\mu}{\sigma},
\end{dmath}
as stated.

\maketheorem{Transformation of \( U^\dagger \sigma^\mu \partial_\mu U \)}{thm:qftLecture19:1122}{
\( U^\dagger \sigma^\mu \partial_\mu U \) transforms as a four vector.
} % theorem

Proof:
\begin{dmath}\label{eqn:qftLecture19:400}
U^\dagger(x) \sigma^\mu \PD{x^\mu}{} U(x)
\rightarrow
{U'}^\dagger(x') \sigma^\mu \PD{{x'}^\mu}{} U'(x')
=
\ultensor{\Lambda}{\mu}{\nu}
U^\dagger(x) \sigma^\nu
\PD{{x'}^\mu}{}
 \ultensor{(\Lambda^{-1})}{\mu'}{\mu}
U(x)
=
U^\dagger(x) \sigma^\nu
\PD{{x'}^\mu}{}
 \ultensor{\delta}{\mu'}{\nu}
U(x)
=
U^\dagger(x) \sigma^\nu
\PD{{x}^\nu}{}
U(x)
\end{dmath}

We can now define
\makedefinition{Weyl action (name?)}{dfn:qftLecture19:1234}{
We may construct the following Lorentz invarinant action
\begin{equation*}
S_{\text{Weyl}} = \int d^4 x i U^\dagger(x) \sigma^\mu \partial_\mu U(x),
\end{equation*}
where \( U(x) \) is a Weyl spinor.
} % definition

The \( i \) factor here is so that the action is real.  This can be seen by noting that \( (i\sigma^\mu)^\dagger = -i \sigma^\mu \) and integrating the Hermitian conjugate by parts

\begin{subequations}
\label{eqn:qftLecture19:1314}
\begin{equation}\label{eqn:qftLecture19:1334}
\lr{ i \sigma^0 }^\dagger = \lr{\begin{bmatrix}
0 & i \\
i & 0
\end{bmatrix}}^\dagger = -i \sigma^0
\end{equation}
\begin{equation}\label{eqn:qftLecture19:1354}
\lr{ i \sigma^1 }^\dagger = {\begin{bmatrix}
0 & i \\
i & 0
\end{bmatrix}}^\dagger = -i \sigma^1
\end{equation}
\begin{equation}\label{eqn:qftLecture19:1374}
\lr{ i \sigma^2 }^\dagger =
{\begin{bmatrix}
0 & 1 \\
-1 & 0
\end{bmatrix}}^\dagger = -i \sigma^2
\end{equation}
\begin{equation}\label{eqn:qftLecture19:1394}
\lr{ i \sigma^3 }^\dagger =
{\begin{bmatrix}
i & 0 \\
0 & -i
\end{bmatrix}}^\dagger = -i \sigma^3
\end{equation}
\end{subequations}

\begin{dmath}\label{eqn:qftLecture19:1414}
S_{\text{Weyl}}^\dagger
= \int d^4 x
\partial_\mu U^\dagger(x) (i\sigma^\mu)^\dagger U(x)
=
-\int d^4 x
\partial_\mu U^\dagger(x) i \sigma^\mu U(x)
=
-
\int d^4 x
\partial_\mu \lr{ U^\dagger(x) i \sigma^\mu U(x) }
+\int d^4 x
U^\dagger(x) i \sigma^\mu \partial_\mu U(x)
=
\int d^4 x
U^\dagger(x) i \sigma^\mu \partial_\mu U(x)
=
S_{\text{Weyl}},
\end{dmath}
where it was assumed that any boundary terms vanish.

\maketheorem{Weyl equation.}{dfn:qftLecture19:1254}{
Variation of the action \cref{dfn:qftLecture19:1234} gives rise to the equations of motion
\begin{equation*}
\sigma^\mu \PD{x^\mu}{} U = 0
%\PD{x^\mu}{} U^\dagger \sigma^\mu &= 0
\end{equation*}
which is called the Weyl equation.
} % theorem
Proof:
\begin{dmath}\label{eqn:qftLecture19:1294}
\delta S
=
i \int d^4 x
\lr{
   \delta U^\dagger \sigma^\mu \partial_\mu U
   + U^\dagger \sigma^\mu \partial_\mu \delta U
}
=
i \int d^4 x
\lr{
   \delta U^\dagger \sigma^\mu \partial_\mu U
   +
   \partial_\mu \lr{ U^\dagger \sigma^\mu \delta U }
   -
   (\partial_\mu U^\dagger) \sigma^\mu \delta U
}
=
i \int d^4 x
\lr{
   \delta U^\dagger \lr{ \sigma^\mu \partial_\mu U }
   -
   \lr{ (\partial_\mu U^\dagger) \sigma^\mu } \delta U
}
=
\int d^4 x
\lr{
   \delta U^\dagger \lr{ i \sigma^\mu \partial_\mu U }
+
   \lr{ \delta U^\dagger \lr{ i \sigma^\mu \partial_\mu U } }^\dagger
}.
\end{dmath}
Requiring this to vanish for all variations \( \delta U^\dagger \) proves the result.

Written out explicitly in matrix form, the Weyl equation is
\begin{dmath}\label{eqn:qftLecture19:440}
\begin{bmatrix}
\partial_0 + \partial_3 & \partial_1 - i \partial_2  \\
\partial_1 + i \partial_2 & \partial_0 - \partial_3
\end{bmatrix}
\begin{bmatrix}
U_1 \\
U_2
\end{bmatrix}
= 0,
\end{dmath}
or
\begin{subequations}
\label{eqn:qftLecture19:460}
\begin{dmath}\label{eqn:qftLecture19:480}
(\partial_0 + \partial_3)U_1 + (\partial_1 - i \partial_2) U_2 = 0
\end{dmath}
\begin{dmath}\label{eqn:qftLecture19:500}
(\partial_1 + i \partial_2)U_1 + (\partial_0 - \partial_3) U_2 = 0.
\end{dmath}
\end{subequations}

\maketheorem{Weyl equation relation to the massless KG equation.}{thm:qftLecture19:1274}{
The Weyl equation is equivalent to a set of massless KG equations.
\begin{equation*}
\partial_\mu \partial^\mu U_k = 0,
\end{equation*}
for \( k = 1, 2\).
} % theorem
Proof:

Multipying \cref{eqn:qftLecture19:480} by \( \partial_1 + i \partial_2 \) gives
\begin{dmath}\label{eqn:qftLecture19:520}
\lr{ \partial_1 + i \partial_2 }
\Biglr{
   (\partial_0 + \partial_3)U_1 + (\partial_1 - i \partial_2) U_2
}
=
(\partial_0 + \partial_3)
\lr{ \partial_1 + i \partial_2 }
U_1 +
\lr{ \partial_1 + i \partial_2 }
(\partial_1 - i \partial_2)
U_2
=
-(\partial_0 + \partial_3)
(\partial_0 - \partial_3) U_2
+
\lr{ \partial_1 + i \partial_2 }
(\partial_1 - i \partial_2)
U_2
=
\lr{
   -\partial_{00}
   + \partial_{33}
   + \partial_{11}
   + \partial_{22}
} U_2
=
\lr{
   -\partial_{0} \partial^0
   - \partial_{3} \partial^3
   - \partial_{1} \partial^1
   - \partial_{2} \partial^2
} U_2
=
- \partial_\mu \partial^\mu U_2.
\end{dmath}
Similarly, multiplying \cref{eqn:qftLecture19:500} by \( \partial_1 - i \partial_2 \) we find
\begin{dmath}\label{eqn:qftLecture19:1274}
0
=
\lr{ \partial_1 - i \partial_2 }
\Biglr{
   (\partial_1 + i \partial_2)U_1
+ (\partial_0 - \partial_3) U_2
}
=
\lr{ \partial_{11} + \partial_{22} } U_1
+
(\partial_0 - \partial_3)
\underbrace{\lr{ \partial_1 - i \partial_2 }
U_2
}_{= -(\partial_0 + \partial_3) U_1}
=
\lr{
\partial_{11} + \partial_{22} - \partial_{00} + \partial_{33}
} U_1
=
- \partial_\mu \partial^\mu U_1.
\end{dmath}

Because \( S_{\text{Weyl}} \) results in a massless KG equation, this is no good for electrons, and we have to look for a different action.

%\paragraph{Q:} Can we write bilinear inut w/ no derivatives.
%
%\paragraph{A:} Yes but, ...
%
\paragraph{Claim:} \( U^\T \sigma_2 U \) is the only bilinear Lorentz invariant that we can add to the action.

An action like:
\begin{dmath}\label{eqn:qftLecture19:560}
\LL_{\text{mass}} = \inv{2} m U^\T \sigma_2 U + \inv{2} m^\conj U^\dagger \sigma_2 (U^\dagger)^\T,
\end{dmath}
may exist in nature (we don't know), and are called Majorana neutrino masses.  The problem with such a Lagrangian density is that it breaks \( U(1) \) symmetry.  In particlar \( U \rightarrow e^{i \alpha} U \) symmetry of the kinetic term.  This means that the particle associated with such a Lagrangian cannot be charged.

Recall that we introduced electromagnetic potentials into NRQM with
\begin{dmath}\label{eqn:qftLecture19:580}
i \Hbar \PD{t}{} \Psi = \inv{2m} \lr{ \spacegrad - e \BA }^2 \Psi
\end{dmath}
which is a gauge transformation.  We'd like to have this capability.

What we can do instead and maintain \( U(1) \) symmetries, is to introduce two U's, like
\begin{dmath}\label{eqn:qftLecture19:600}
\LL_{\text{mass}} = \inv{2} m U_1^\T \sigma_2 U_2 + \inv{2} m^\conj U_2^\dagger \sigma_2 (U_1^\dagger)^\T
\end{dmath}
What we are really doing is assemblying a four component spinor out of the two U's.

\section{Lorentz symmetry.}

We want to examine the Lorentz invariance of \(U^\T \sigma_2 U\), but need an intermediate result first.
\makelemma{Transpose of Pauli vector representation}{lemma:qftLecture19:21}{
For any \( \Bx \in \bbR^3 \)
\begin{equation*}
(\Bsigma \cdot \Bx)^\T = -\sigma^2 (\Bsigma \cdot \Bx) \sigma^2,
\end{equation*}
or more compactly
\begin{equation*}
\Bsigma^\T = -\sigma^2 \Bsigma \sigma^2.
\end{equation*}
Geometrically, this transposition operation reflects \( \Bx \) about the y-axis.
} % lemma

To prove \cref{lemma:qftLecture19:21} first note that
\begin{dmath}\label{eqn:qftLecture19:740}
\begin{aligned}
\sigma_1^\T &= \sigma_1 \\
\sigma_2^\T &= -\sigma_2 \\
\sigma_3^\T &= \sigma_3
\end{aligned}
\end{dmath}
which means that
\begin{dmath}\label{eqn:qftLecture19:1454}
(\Bsigma \cdot \Bx)^\T
=
\sigma^1 x^1 - \sigma^2 x^2 + \sigma^3 x^3
=
\sigma^2 \sigma^2
\lr{
\sigma^1 x^1 - \sigma^2 x^2 + \sigma^3 x^3
}
=
\sigma^2
\lr{
-\sigma^1 x^1 - \sigma^2 x^2 - \sigma^3 x^3
} \sigma^2
=
- \sigma^2 (\Bsigma \cdot \Bx)^\T \sigma^2.
\end{dmath}

Now we are ready to proceed.
\maketheorem{\(U^\T \sigma_2 U\) invariance}{thm:qftLecture19:10}{
\(U^\T \sigma_2 U\) is Lorentz invariant.
} % theorem

Proof:
\begin{dmath}\label{eqn:qftLecture19:620}
U^\T \sigma_2 U
\rightarrow
{U'}^\T \sigma_2 U'
=
U^\T {M^\dagger}^\T \sigma_2 M^\dagger U,
\end{dmath}
where \( U' = M^\dagger U \) and \( {U'}^\T = U^\T {M^\dagger}^T \).

Note that if we can show that \( {M^\dagger}^\T \sigma_2 M^\dagger = \sigma_2 \), then we are done.

It is simple to show that any
\begin{dmath}\label{eqn:qftLecture19:640}
U = e^{i \Ba \cdot \Bsigma },
\end{dmath}
for \( \Ba \in \bbR^{3} \), has eigenvalues \( \pm i \Abs{\Ba} \).  The determinant of such a matrix is
\begin{equation}\label{eqn:qftLecture19:1434}
\det U = \begin{vmatrix}
e^{i \Abs{\Ba}} & 0 \\
0 & e^{-i \Abs{\Ba}}
\end{vmatrix}
= 1,
\end{equation}
so we see that such a matrix has the \( U^\dagger U = 1 \) and \( \det U = 1 \) properties that we desire for elements of \( SU(2) \)\footnote{In class the suitablity of \( e^{i \Bsigma \cdot \Ba} \) as an element of \( SU(2) \) was demonstrated with an argument that diagonalizable matrices satisfy \( \det e^A = e^{\trace{A}} \)}.
We haven't shown that all matrices \( U \in SU(2) \) can be written in this form, but let's assume that's the case.

\paragraph{Claim:}  Generalizing from the exponential form of \( SU(2) \) elements seen above, we assume that any \( SL(2,\bbC) \) matrix \( M \) can be written as
\begin{dmath}\label{eqn:qftLecture19:680}
M^\dagger =
e^{i (\Ba + i \Bb) \cdot \Bsigma},
\end{dmath}
for \( \Ba, \Bb \in \bbR^3 \).

The transpose of an exponential of a sigma matrix goes like
\begin{dmath}\label{eqn:qftLecture19:1474}
(e^{\Bu \cdot \Bsigma})^\T
=
\sum_{k = 0}^\infty \inv{k!}\lr{(\Bu \cdot \Bsigma)^k}^\T
=
\sum_{k = 0}^\infty \inv{k!}\lr{- \sigma_2 \Bu \cdot \Bsigma \sigma_2}^k
=
\sigma_2 \lr{
   \sum_{k = 0}^\infty \inv{k!}\lr{- \Bu \cdot \Bsigma }^k
}
\sigma_2
=
\sigma_2
e^{-\Bu \cdot \Bsigma}
\sigma_2,
\end{dmath}
so
\begin{dmath}\label{eqn:qftLecture19:700}
{M^\dagger}^\T \sigma_2 M^\dagger
=
\lr{ e^{i (\Ba + i \Bb) \cdot \Bsigma} }^\T
\sigma_2
e^{i (\Ba + i \Bb) \cdot \Bsigma}
=
\lr{
   \sigma_2
   e^{-i (\Ba + i \Bb) \cdot \Bsigma}
   \sigma_2
}
\sigma_2
e^{i (\Ba + i \Bb) \cdot \Bsigma}
=
\sigma_2,
\end{dmath}
which is the result required to finish the proof of \cref{thm:qftLecture19:10}\footnote{A slightly different derivation was done in class, but this one makes more sense to me}.

%%%\begin{dmath}\label{eqn:qftLecture19:760}
%%%{M^\dagger}^\T \sigma_2 M^\dagger
%%%=
%%%\sigma_2
%%%\lr{
%%%   \sigma_2 e^{-i (\Ba + i \Bb) \cdot (\sigma_2 \Bsigma \sigma_2) } \sigma_2
%%%}
%%%e^{i (\Ba + i \Bb) \cdot \Bsigma }
%%%=
%%%\sigma_2
%%%\lr{
%%%   \sigma_2 e^{-i (\Ba + i \Bb) \cdot (\sigma_2 \sigma_2 \Bsigma \sigma_2 \sigma_2) }
%%%}
%%%e^{i (\Ba + i \Bb) \cdot \Bsigma }
%%%=
%%%\sigma_2
%%%\lr{
%%%   \sigma_2 e^{-i (\Ba + i \Bb) \cdot \Bsigma }
%%%}
%%%e^{i (\Ba + i \Bb) \cdot \Bsigma }
%%%=
%%%\sigma_2
%%%\end{dmath}
%%%
%%%where we used an argument like
%%%\begin{dmath}\label{eqn:qftLecture19:780}
%%%e^{A} \sigma_2
%%%= \sigma_2( 1 + A + \inv{2} A^2 ...)
%%%= ?
%%%\end{dmath}
%%%to move the \( \sigma_2 \) into the exponential argument.
%%%
\section{Dirac Lagrangian.}

We postulate that there is a four-component object
\begin{dmath}\label{eqn:qftLecture19:800}
\Psi =
\begin{bmatrix}
\psi_1 \\
\psi_2 \\
\psi_3 \\
\psi_4
\end{bmatrix},
\end{dmath}
where \( \psi_\mu \)'s are all complex fields.

Define \( \gamma^\mu \), for \( \mu = 0, 1, 2, 3 \) are \( 4 \times 4 \) matrixes that satisfy

\begin{dmath}\label{eqn:qftLecture19:820}
\symmetric{\gamma^\mu}{\gamma^\nu} = 2 g^{\mu\nu},
\end{dmath}
that is
\begin{dmath}\label{eqn:qftLecture19:840}
\gamma^\mu \gamma^\nu + \gamma^\nu \gamma^\mu = 2 g^{\mu\nu},
\end{dmath}
or
\begin{dmath}\label{eqn:qftLecture19:860}
\begin{aligned}
\lr{\gamma^0}^2 &= 1 \\
\lr{\gamma^i}^2 &= -1.
\end{aligned}
\end{dmath}
We will use the explicit form
\begin{dmath}\label{eqn:qftLecture19:880}
\gamma^0 =
\begin{bmatrix}
0 & 1 \\
1 & 0
\end{bmatrix},
\end{dmath}
and
\begin{dmath}\label{eqn:qftLecture19:900}
\gamma^i=
\begin{bmatrix}
0 & \sigma^i \\
-\sigma^i & 0
\end{bmatrix}.
\end{dmath}

We can form rotations or boosts:
\begin{dmath}\label{eqn:qftLecture19:920}
S^{\mu\nu} = \frac{i}{4} \antisymmetric{\gamma^\mu}{\gamma^\nu}
\end{dmath}
In particular
\begin{dmath}\label{eqn:qftLecture19:940}
S^{0i} = \frac{i}{4} \antisymmetric{\gamma^0}{\gamma^i}
=
\frac{i}{2}
\begin{bmatrix}
-\sigma^i & 0 \\
0 & \sigma^i
\end{bmatrix}
\end{dmath}
are boosts, whereas
\begin{dmath}\label{eqn:qftLecture19:960}
S^{ij} = \frac{i}{4} \antisymmetric{\gamma^i}{\gamma^j}
= \inv{2} \epsilon^{ijk}
\begin{bmatrix}
\sigma^k & 0 \\
0 & \sigma^k
\end{bmatrix},
\end{dmath}
are rotations (and in this case, are Hermitian).

With

\begin{dmath}\label{eqn:qftLecture19:1040}
\begin{aligned}
x \rightarrow x' &= \Lambda x \\
\Psi(x) \rightarrow \Psi'(x') &= \Lambda_{1/2} \Psi(x).
\end{aligned}
\end{dmath}

\begin{dmath}\label{eqn:qftLecture19:980}
\ultensor{\Lambda}{\mu}{\nu} =
\ultensor{\delta}{\mu}{\nu}
+
\ultensor{\omega}{\mu}{\nu}
+ \cdots
\end{dmath}

\begin{dmath}\label{eqn:qftLecture19:1060}
\Psi^\dagger = \lr{ \psi_1^\conj, \psi_2^\conj, \psi_3^\conj, \psi_4^\conj },
\end{dmath}
and
\begin{dmath}\label{eqn:qftLecture19:1080}
\overbar{\Psi} = \Psi^\dagger \gamma^0.
\end{dmath}

We form
\boxedEquation{eqn:qftLecture19:1100}{
\LL_{\text{Dirac}} = \overbar{\Psi}(x)\lr{ i \gamma^\mu \partial_\mu - m } \Psi(x).
}

\begin{dmath}\label{eqn:qftLecture19:1000}
\begin{aligned}
\Lambda_{1/2}
&=
e^{-\frac{i}{2} \omega_{\mu\nu} S^{\mu\nu} } \\
&=
e^{
-i
\omega_{0i} S^{0i}
-\frac{i}{2}
\omega_{ij} S^{ij}
} \\
&=
\exp\lr{
-\inv{2}
\begin{bmatrix}
\omega_{0i} \sigma^i & 0 \\
0 & -\omega_{0i} \sigma^i
\end{bmatrix}
- \frac{i}{4}
\begin{bmatrix}
\omega_{ij} \epsilon^{ijk} \sigma^k & 0 \\
0 & \omega_{ij} \epsilon^{ijk} \sigma^k
\end{bmatrix}
} \\
&=
\begin{bmatrix}
e^{-\lr{ \inv{2} \omega_{0k} \sigma_0 + \frac{i}{4} \omega_{ij} e^{ijk}  \sigma^k }} & 0 \\
0 & e^{-\lr{ -\inv{2} \omega_{0k} \sigma_0 + \frac{i}{4} \omega_{ij} e^{ijk} \sigma^k }} & 0 \\
\end{bmatrix}
\end{aligned}
\end{dmath}
where the \( 1/2\) factor of \( \omega_{0i} \) vanished because we had a sum over \( 0i \) and \( i0 \) which have been grouped.

\paragraph{Claim:} \( \overbar{\Psi} \lr{ i \gamma^\mu \partial_\mu - m } \Psi \) is a Lorentz scalar.  Real follows from definitions and \( (\gamma^0)^\dagger = \gamma^0, {\gamma^i}^\dagger = - \gamma^i \).

\begin{dmath}\label{eqn:qftLecture19:1020}
\overbar{\Psi}(x) \lr{ i \gamma^\mu \partial_\mu - m } \Psi(x)
\rightarrow
\overbar{\Psi'}(x') \gamma^0 \lr{ i \gamma^\mu \PD{{x'}^\mu}{} - m } \Psi'(x')
=
\Psi^\dagger
\lr{
      \underbrace{
      \Lambda_{1/2}
      \gamma^0
      \Lambda_{1/2}
      }_{= \gamma^0}
   i
      \underbrace{
      {\Lambda^{-1}}_{1/2}
      \gamma^\mu
      \Lambda_{1/2}
      }_{= \ultensor{\Lambda}{\mu}{\nu} \gamma^\nu}
   \partial_\alpha
   \ultensor{
   {\Lambda^{-1}}_{1/2}
   }{\alpha}{\mu}
}
\Psi
-
m \Psi^\dagger
\underbrace{\Lambda_{1/2}^\dagger
\gamma_0
\Lambda_{1/2}
}_{= \gamma_0}
\Psi
=
\Psi^\dagger \gamma^0 i \gamma^\alpha \partial_\alpha \Psi - m \Psi^\dagger \gamma_0 \Psi
=
\overbar{\Psi} \lr{ i \gamma^\mu \partial_\mu - m } \Psi
\end{dmath}

We find that \( \overbar{\Psi} \Psi = \Psi^\dagger \gamma^0 \Psi \) is a Lorentz scalar, whereas
\( \overbar{\Psi} \gamma^\mu \Psi \) is a 4 vector.

%Can create couplings like \( \overbar{\Psi}_1 \Psi_2 \overbar{\Psi}_3 \Psi_4 \)
%or \( \overbar{\Psi} \gamma_\mu \partial^\mu \Psi \).
%%\cref{fig:l19:l19Fig2}.
%\imageFigure{../figures/phy2403-quantum-field-theory/l19Fig2}{Muon decay.}{fig:l19:l19Fig2}{0.3}

%}
%\EndArticle
\EndNoBibArticle
