%
% Copyright � 2018 Peeter Joot.  All Rights Reserved.
% Licenced as described in the file LICENSE under the root directory of this GIT repository.
%
\makeoproblem{The ``$h \rightarrow WW, ZZ$" Higgs-decay width.}{qft:problemSet4:2}{2018 Hw4.II}{
\index{Higgs decay width}
\index{Goldstone boson}
From the $\SU{2}_L \times \SU{2}_R$ model of Homework 2---really, the Higgs Lagrangian of the Standard Model, find the coupling of the $h$-particle (the Higgs boson) to the $\phi^a$ particles (these are now Goldstone bosons, in the electroweak theory, they become the longitudinal components of the $W$ and $Z$ particles). Canonically normalizing $h$ and $\phi^a$, this coupling has the form
\begin{equation}
\label{gg1}
\LL_{\text{int}}
=
const. \; h \; \partial_\mu \phi^a \partial^\mu \phi^a~.
\end{equation}

\makesubproblem{}{qft:problemSet4:2a}
 Determine the value of $const.$ for canonically normalized $h$ and $\phi^a$.
\makesubproblem{}{qft:problemSet4:2b}
Use this coupling to compute the width  $\Gamma(h \rightarrow \phi^3 \phi^3)$ of the Higgs particle to decay to two longitudinal (say) $Z$-bosons (hence the index $3$).
\makesubproblem{}{qft:problemSet4:2c}
 Plug in some numbers. Use the fact that the vacuum expectation value $|m|/\sqrt{\lambda}  = 246$ GeV  and the fact that $m_h = 125$GeV to get a number for the lifetime. Compare to the total width of the Higgs from \citep{bernardi2012higgs}%
%from \url{https://pdg.lbl.gov/2012/reviews/rpp2012-rev-higgs-boson.pdf}
, see figure 5 there, as well to the partial width to $WW$ given in Figure 4 there.
\makesubproblem{}{qft:problemSet4:2d}
 At the same time, determine the values of $|m|$ and $\lambda$ separately. Is $\lambda \ll 1$ (i.e. perturbative)?

{\flushleft {\small Notice that this calculation would have been physically relevant had the Higgs been heavy, $m_h \gg m_W \sim 100$ GeV. This is because the $h\rightarrow WW$ decay then is dominated (in this limit) by the decay into the longitudinal component, which is really the Goldstone boson field $\phi^a$ (in this limit, the result is independent of the gauge couplings $g_{1,2}$ of the Standard Model). Nonetheless, having some real numbers in this class is good.}}
} % makeproblem

\makeanswer{qft:problemSet4:2}{
\withproblemsetsParagraph{
\makeSubAnswer{}{qft:problemSet4:2a}
Here's a reminder and summary of the Higgs Lagrangian we will be working with in this problem
\begin{dmath}\label{eqn:ProblemSet4Problem2:640}
\LL = \trace{
   \lr{
      \partial_\mu H^\dagger \partial^\mu H
   }
}
- V,
\end{dmath}
where
\begin{dmath}\label{eqn:ProblemSet4Problem2:660}
V =
-\Abs{m}^2 \trace{
   \lr{
      H^\dagger H
   }
}
+ \lambda
\lr{
   \trace{
      H^\dagger H
   }
}^2.
\end{dmath}
It was postulated that the field had a radial component \( h \), the Higgs field, and an rotational component \( \Omega \), where the total field was given by
\begin{dmath}\label{eqn:ProblemSet4Problem2:680}
H(x) = \frac{\Abs{m}}{2 \sqrt{ \lambda } }\Omega(x) ( 1 + h(x) ),
\end{dmath}
where
\begin{equation}\label{eqn:ProblemSet4Problem2:700}
\Omega = e^{ i \Bsigma \cdot \Bphi } = e^{i \phi^a(x) \sigma^a }.
\end{equation}

Assuming that \( h(x) \) and \( \phi^a(x) \) commute, \( H^\dagger H \) can be computed with relative ease, and has only radial dependence
\begin{dmath}\label{eqn:ProblemSet4Problem2:380}
\trace{\lr{H^\dagger H}}
=
\frac{\Abs{m}^2}{4 \lambda} (1 + h(x))^2 \trace{\lr{ e^{-i \Bsigma \cdot \Bphi} e^{i \Bsigma \cdot \Bphi} }}
=
\frac{\Abs{m}^2}{4 \lambda} (1 + h(x))^2 \trace{\BOne}
=
\frac{\Abs{m}^2}{2 \lambda} (1 + h)^2.
\end{dmath}
For the derivative quadratic form, we find
\begin{dmath}\label{eqn:ProblemSet4Problem2:400}
\partial_\mu H^\dagger \partial^\mu H
=
\frac{\Abs{m}^2}{4 \lambda}
\lr{
   \partial_\mu h \Omega^\dagger
   + (1 + h) \partial_\mu \Omega^\dagger
}
\lr{
   \partial^\mu h \Omega
   + (1 + h)
\partial^\mu \Omega
}
=
\frac{\Abs{m}^2}{4 \lambda}
\lr{
   \partial_\mu h \Omega^\dagger \partial^\mu h \Omega
   + (1 + h)
      \lr{
         \partial_\mu h
         \Omega^\dagger (\partial^\mu \Omega)
       +
         \partial^\mu h
         (\partial_\mu \Omega^\dagger) \Omega
      }
   + (1 + h)^2 \partial_\mu \Omega^\dagger \partial^\mu \Omega
}.
\end{dmath}
Because \( \Omega^\dagger \Omega = 1 \), we have
\begin{dmath}\label{eqn:ProblemSet4Problem2:480}
\partial_\mu h
\Omega^\dagger (\partial^\mu \Omega)
 +
\partial^\mu h
(\partial_\mu \Omega^\dagger) \Omega
=
\partial_\mu h
\lr{
   \Omega^\dagger (\partial^\mu \Omega)
    +
   (\partial^\mu \Omega^\dagger) \Omega
}
=
\partial_\mu h
\lr{
   \partial^\mu (\Omega^\dagger \Omega) - (\partial^\mu \Omega^\dagger) \Omega
    +
   (\partial^\mu \Omega^\dagger) \Omega
}
=
   \partial^\mu (1)
= 0.
\end{dmath}
All the cross terms with both \( h \) and \( \Omega \) derivatives are zero (to all orders, not just quadratic).

Taking traces (and using cyclic permutation of the matrices in the trace operations),
the Lagrangian density is now determined
\begin{dmath}\label{eqn:ProblemSet4Problem2:500}
\LL =
\frac{\Abs{m}^2}{2 \lambda}
   \partial_\mu h \partial^\mu h
+
\frac{\Abs{m}^2}{4 \lambda}
( 1 + h )^2
   \trace{\lr{
      \partial_\mu \Omega^\dagger \partial^\mu \Omega
   }}
+ \Abs{m}^2
\frac{\Abs{m}^2}{2 \lambda} \lr{ 1 + h }^2
- \lambda
\lr{\frac{\Abs{m}^2}{2 \lambda}}^2
\lr{ 1 + h }^4.
%=
%\frac{\Abs{m}^2}{\lambda} \LL',
\end{dmath}
%
Now let's expand the \( \Omega \) derivatives.  To first order, we have
\begin{dmath}\label{eqn:ProblemSet4Problem2:580}
\partial_\mu \Omega
=
\partial_\mu \lr{ \BOne + i \Bsigma \cdot \Bphi }
=
i \Bsigma \cdot \partial_\mu \Bphi,
\end{dmath}
so
\begin{dmath}\label{eqn:ProblemSet4Problem2:760}
   \trace{\lr{
      \partial_\mu \Omega^\dagger \partial^\mu \Omega
   }}
=
\trace{\lr{
   (-i \Bsigma \cdot \partial_\mu \Bphi^\dagger)
   (i \Bsigma \cdot \partial^\mu \Bphi)
}}
=
\trace{\lr{
   (\Bsigma \cdot \partial_\mu \Bphi)
   (\Bsigma \cdot \partial^\mu \Bphi)
}},
\end{dmath}
where the real nature of each of the \( \phi^a \)'s has been used to eliminate the \( \dagger\)'s.
The structure of this trace is that of
\begin{dmath}\label{eqn:ProblemSet4Problem2:780}
\trace{\lr{
   (\Bsigma \cdot \Bx)
   (\Bsigma \cdot \By)
}}
=
x^a y^b
\trace{\lr{
   \sigma^a \sigma^b
}}
=
x^a y^b
\left\{
\begin{array}{l l}
2 & \quad \mbox{\( a = b \)} \\
0 & \quad \mbox{\( a \ne b \)} \\
\end{array}
\right.
=
2 \Bx \cdot \By,
\end{dmath}
%
The Lagrangian density, including just the kinetic term, and the first order \( h \) interaction is
\begin{dmath}\label{eqn:ProblemSet4Problem2:501}
\LL
=
\frac{\Abs{m}^2}{2 \lambda}
   \partial_\mu h \partial^\mu h
+
\frac{\Abs{m}^2}{4 \lambda} (2 h) 2 \partial_\mu \phi^a \partial^\mu \phi^a
+ \cdots
=
\frac{\Abs{m}^2}{2 \lambda}
   \partial_\mu h \partial^\mu h
+
\frac{\Abs{m}^2}{\lambda} h \partial_\mu \phi^a \partial^\mu \phi^a.
\end{dmath}
Imposing a transformation of the fields
\begin{dmath}\label{eqn:ProblemSet4Problem2:1180}
\begin{aligned}
h &\rightarrow \frac{\sqrt{\lambda}}{\Abs{m}} h \\
\phi^a &\rightarrow \frac{\sqrt{\lambda}}{\Abs{m}} \phi^a,
\end{aligned}
\end{dmath}
we find that the portion of the Lagrangian including just the kinetic and interaction terms is transformed to
\begin{dmath}\label{eqn:ProblemSet4Problem2:1200}
\LL \rightarrow
\inv{2}
   \partial_\mu h \partial^\mu h
+
\frac{\sqrt{\lambda}}{\Abs{m}} h \partial_\mu \phi^a \partial^\mu \phi^a.
\end{dmath}
This is now ``canonically normalized'' \footnote{Canonically normalized is assumed to mean that there's a one-half factor on the kinetic terms}.
%\begin{dmath}\label{eqn:ProblemSet4Problem2:720}
%\LL' =
%+
%\inv{4}
%( 1 + h )^2
%   \trace{\lr{
%      \partial_\mu \Omega^\dagger \partial^\mu \Omega
%   }}
%+
%\inv{2}
%\Abs{m}^2
%\lr{ 1 + h }^2
%-
%\frac{\Abs{m}^2}{4}
%\lr{ 1 + h }^4.
%\end{dmath}
%
The coupling to first order in \( h \) is
\begin{dmath}\label{eqn:ProblemSet4Problem2:740}
\LL_{\text{int}}
=
\frac{\sqrt{\lambda}}{\Abs{m}}
h
\partial_\mu \phi^a \partial^\mu \phi^a.
\end{dmath}
The constant in question (called \( 1/v \) in problem 3) has been found to be
\boxedEquation{eqn:ProblemSet4Problem2:1220}{
\inv{v} =
\frac{\sqrt{\lambda}}{\Abs{m}}.
}

\makeSubAnswer{}{qft:problemSet4:2b}

The scattering calculation machine that was presented in class, assumes that the final states scattering process can be related to a scattering matrix with the following structure
\begin{equation}\label{eqn:ProblemSet4Problem2:820}
\prescript{}{\text{out}}{\braket{\Bp_1, \Bp_2}{\Bk}}_{\text{in}}
=
\bra{ \Bp_1, \Bp_2 } \hat{S} \ket{ \Bk}
=
\bra{ \Bp_1, \Bp_2 } T\lr{ e^{-i\int dt H_{\text{int}}(t)} } \ket{ \Bk}
\end{equation}
where \( \Bk \) is the momentum of the initial Higgs particle, \( \Bp_i \) are the momenta of the Z-boson disintegration products, and we evaluate the amplitude by summing ``connected amputated diagrams''.

With the plus-minus decomposition of the field \( h(z) = h^{+}(z) + h^{-}(z) \), contracting the field with the initial momentum state gives
\begin{dmath}\label{eqn:ProblemSet4Problem2:840}
h(z) \ket{\Bk}
=
h^{+}(z) \ket{\Bk}
=
\int \frac{d^3 q}{(2 \pi)^3 \sqrt{2 \omega_\Bq} } e^{-i q \cdot z} a_\Bq \ket{\Bk}
=
\int \frac{d^3 q}{(2 \pi)^3 \sqrt{2 \omega_\Bq} } e^{-i q \cdot z} a_\Bq \sqrt{ 2 \omega_\Bk } a_\Bk^\dagger \ket{0}
=
\int d^3 q e^{-i q \cdot z} \deltathree( \Bq - \Bk ) \ket{0}
=
e^{-i k \cdot z}.
\end{dmath}
Similarly,
\begin{dmath}\label{eqn:ProblemSet4Problem2:860}
\bra{\Bp} \phi^3(z)
=
e^{i p \cdot z}.
\end{dmath}

Apparently\footnote{According to a ``trust-me, it's a long story'' kind of statement related to a classmate from Professor Poppitz.}, the interaction Hamiltonian density that we want to use for this problem is \( H_{\text{int}} = -\LL_{\text{int}} \).  Given that, the exponential argument expands to
\begin{dmath}\label{eqn:ProblemSet4Problem2:880}
-i \int dt H_{\text{int}}(t)
=
i \inv{v} \int dt \int d^3 x h(x) \partial_\mu \phi^a(x) \partial^\mu \phi^a(x).
=
i \inv{v} \int d^4 x h(x) \partial_\mu \phi^a(x) \partial^\mu \phi^a(x).
\end{dmath}
so the
first order expansion of the scattering amplitude is
\begin{dmath}\label{eqn:ProblemSet4Problem2:900}
i \inv{v} \int d^4 x \bra{ \Bp_1, \Bp_2 } T\lr{
   h(x) \partial_\mu \phi^a(x) \partial^\mu \phi^a(x)
} \ket{ \Bk}.
\end{dmath}
There are two possible diagrams associated with this amplitude, sketched in \cref{fig:hw4p2}, but only the first qualifies as ``connected amputated''.
\imageTwoFigures
{../figures/phy2403-quantum-field-theory/hw4p2decayFig1}
{../figures/phy2403-quantum-field-theory/hw4p2VirtualProcessFig2}
{Possible figures}{fig:hw4p2}{scale=0.2}

Algebraically, in terms of contractions the first diagram is
\begin{dmath}\label{eqn:ProblemSet4Problem2:920}
i \inv{v} \int d^4 x
\contraction{\langle }{\Bp}{{}_1, \Bp_2 \rvert h(x) \partial_\mu }{\phi}
\contraction[2ex]{\langle \Bp_1, }{\Bp}{{}_2 \rvert h(x) \partial_\mu \phi^a(x) \partial^\mu }{\phi}
\bcontraction{\langle \Bp_1, \Bp_2 \rvert }{h}{(x) \partial_\mu \phi^a(x) \partial^\mu \phi^a(x) \lvert}{\Bk}
\langle \Bp_1, \Bp_2 \rvert h(x) \partial_\mu \phi^a(x) \partial^\mu \phi^a(x) \lvert \Bk \rangle,
\end{dmath}
however, since \( \Bp_i \) are the momenta for \( \phi^3 \) particles, only the \( a = 3 \) terms above contribute, leaving
\begin{dmath}\label{eqn:ProblemSet4Problem2:940}
i \inv{v} \int d^4 x
\contraction{\langle }{\Bp}{{}_1, \Bp_2 \rvert h(x) \partial_\mu }{\phi}
\contraction[2ex]{\langle \Bp_1, }{\Bp}{{}_2 \rvert h(x) \partial_\mu \phi^3(x) \partial^\mu }{\phi}
\bcontraction{\langle \Bp_1, \Bp_2 \rvert }{h}{(x) \partial_\mu \phi^3(x) \partial^\mu \phi^3(x) \lvert}{\Bk}
\langle \Bp_1, \Bp_2 \rvert h(x) \partial_\mu \phi^3(x) \partial^\mu \phi^3(x) \lvert \Bk \rangle
=
i \inv{v} \int d^4 x \bra{0} \partial_\mu e^{i p_1 \cdot x} \partial^\mu e^{i p_2 \cdot x} e^{-i k \cdot x} \ket{0}
=
i \inv{v} \int d^4 x \bra{0} (i (p_1)_\mu)(i (p_2)^\mu) e^{i (p_1 + p_2 - k) \cdot x} \ket{0}
=
-i \inv{v} \int d^4 x \bra{0} p_1 \cdot p_2 e^{i (p_1 + p_2 - k) \cdot x} \ket{0}
=
-i \inv{v} (p_1 \cdot p_2) (2 \pi)^4 \deltafour(p_1 + p_2 - k).
\end{dmath}
This equals \( i M_{fi} (2 \pi)^4 \deltafour(p_1 + p_2 - k) \), so
\begin{dmath}\label{eqn:ProblemSet4Problem2:960}
M_{fi} = - \inv{v} (p_1 \cdot p_2).
\end{dmath}

We can now start plugging this into our decay rate formula
\begin{dmath}\label{eqn:ProblemSet4Problem2:980}
\Gamma = \inv{2 \omega_\Bk} \int d(LIPS)_2 \Abs{M_{fi}}^2,
\end{dmath}
where\footnote{\citep{LukeQFT} uses \( D \) for the \( d(LIPS)_2 \) symbol we used in class.}
\begin{dmath}\label{eqn:ProblemSet4Problem2:1000}
d(LIPS)_2
=
(2 \pi)^4 \deltafour(p_1 + p_2 - k)
\frac{d^3 p_1}{(2 \pi)^3 2 \omega_{\Bp_1} }
\frac{d^3 p_2}{(2 \pi)^3 2 \omega_{\Bp_2} }
=
(2 \pi)^4 \deltathree(\Bp_1 + \Bp_2 - \Bk) \delta(\omega_1 + \omega_2 - \omega_\Bk)
\frac{d^3 p_1}{(2 \pi)^3 2 \omega_{\Bp_1} }
\frac{d^3 p_2}{(2 \pi)^3 2 \omega_{\Bp_2} }
\end{dmath}
\begin{dmath}\label{eqn:ProblemSet4Problem2:1020}
\Gamma
=
\inv{v^2}
\int
\frac{d^3 p_1}{(2 \pi)^3 2 \omega_{\Bp_1} }
\frac{d^3 p_2}{(2 \pi)^3 2 \omega_{\Bp_2} }
\inv{ 2\omega_\Bk}
(2 \pi)^4 \deltathree(\Bp_1 + \Bp_2 - \Bk) \delta(\omega_{\Bp_1} + \omega_{\Bp_2} - \omega_\Bk)
(-p_1 \cdot p_2)^2.
\end{dmath}
This is simplest to evaluate in the center of mass frame, as sketched in \cref{fig:hw4p2CenterOfMassFrame:hw4p2CenterOfMassFrameFig3}, where \( \Bk = 0 \), and \( \omega_\Bk = m_h \), the Higgs mass.  This leaves
\imageFigure{../figures/phy2403-quantum-field-theory/hw4p2CenterOfMassFrameFig3}{Center of mass frame.}{fig:hw4p2CenterOfMassFrame:hw4p2CenterOfMassFrameFig3}{0.2}
\begin{dmath}\label{eqn:ProblemSet4Problem2:1040}
\Gamma
=
\inv{v^2}
\int
\frac{d^3 p_1}{(2 \pi)^2 4 \omega_{\Bp_1}^2}
\delta(2 \omega_{\Bp_1} - \omega_I)
\evalbar{(p_1 \cdot p_2)^2}{\Bp_2 = -\Bp_1}.
\end{dmath}
If
\begin{dmath}\label{eqn:ProblemSet4Problem2:1060}
\begin{aligned}
p_1 &= (\omega_{\Bp_1}, \Bp_1) \\
p_2 &= (\omega_{\Bp_2}, \Bp_2),
\end{aligned}
\end{dmath}
then
\begin{dmath}\label{eqn:ProblemSet4Problem2:1080}
p_1 \cdot p_2
= \omega_{\Bp_1} \omega_{\Bp_2} - \Bp_1 \cdot \Bp_2,
\end{dmath}
and
\begin{equation}\label{eqn:ProblemSet4Problem2:1100}
\evalbar{p_1 \cdot p_2}{\Bp_2 = -\Bp_1}
= \omega_{\Bp_1} \omega_{\Bp_1} + \Bp_1^2 = m_1^2 + 2 \Bp_1^2,
\end{equation}
however, the \( \phi^3 \) particles are bosons (no mass!), so this is just
\begin{dmath}\label{eqn:ProblemSet4Problem2:1120}
\evalbar{p_1 \cdot p_2}{\Bp_2 = -\Bp_1}
= 2 \Bp_1^2.
\end{dmath}
so \cref{eqn:ProblemSet4Problem2:1020} becomes
\begin{dmath}\label{eqn:ProblemSet4Problem2:1140}
\Gamma
=
\inv{v^2}
\int
\frac{d^3 p_1}{(2 \pi)^3 2 \omega_{\Bp_1} }
\frac{d^3 p_2}{(2 \pi)^3 2 \omega_{\Bp_2} }
\inv{ 2 m_h}
(2 \pi)^4 \deltathree(\Bp_1 + \Bp_2 - \Bk) \delta(\omega_{\Bp_1} + \omega_{\Bp_2} - m_h)
4 \Norm{\Bp_1}^4
=
\inv{v^2}
\int
\frac{d^3 p_1}{(2 \pi)^3 \omega_{\Bp_1}^2 }
\inv{ 2 m_h}
(2 \pi) \delta( 2 \omega_{\Bp_1} - m_h)
\Norm{\Bp_1}^4
=
\inv{v^2}
\int
\frac{d^3 p_1}{(2 \pi)^2 }
\inv{ 2 m_h}
\delta( 2 \omega_{\Bp_1} - m_h)
\Norm{\Bp_1}^2.
\end{dmath}
Evaluating in spherical coordinates with \( \Norm{\Bp_1} = \calp \), we are left with
\begin{dmath}\label{eqn:ProblemSet4Problem2:1160}
\Gamma
=
\inv{v^2}
\frac{4 \pi}{8 m_h \pi^2}
\int_0^\infty
d\calp
\calp^4
\delta( 2 \calp - m_h)
=
\inv{v^2}
\frac{1}{2 m_h \pi}
\int_0^\infty
d\calp
\calp^4
\frac{\delta( \calp - m_h/2)}{2}
=
\inv{v^2} \frac{m_h^3}{64 \pi},
\end{dmath}
where no adjustment of the integration range \( \int_0^\infty p^4 \rightarrow \inv{2} \int_{-\infty}^\infty p^4 \) transformation was made before evaluating the delta function.  That was done on purpose since the
zero of our delta function sits at \( m_h/2 > 0 \), a point already in the \( [0, \infty] \) range of the delta function integral above.  The final result, putting in our constant factor \( \chi \) from \cref{eqn:ProblemSet4Problem2:1220} is
\boxedEquation{eqn:ProblemSet4Problem2:1240}{
\Gamma( h \rightarrow \phi^3 \phi^3 )
=
\frac{\lambda}{\Abs{m}^2}
\frac{m_h^3}{64 \pi}.
}

\paragraph{Commentary on possible errors.}
I'm not entirely convinced that this answer is not off by some \( 2^n \) factor, even assuming no plain old algebraic errors.  The easiest place I can imagine messing this up, is by double counting our indistinguishable bosons.  I think that I've implicitly accounted for that indistinguishably by not separately labelling \( \phi^3_A, \phi^3_B \) end points in the first diagram of \cref{fig:hw4p2}, therefore counting that diagram only once.

\makeSubAnswer{}{qft:problemSet4:2c}

The results of plugging the numbers can be found in \cref{fig:hw4p2Mathematica:hw4p2MathematicaFig1}.
\mathImageFigure{../figures/phy2403-quantum-field-theory/hw4p2MathematicaFig1}{Plugging in the numbers.}{fig:hw4p2Mathematica:hw4p2MathematicaFig1}{0.5}{hw4p2numbers.nb}
, we have
%\begin{dmath}\label{eqn:ProblemSet4Problem2:1260}
%\Gamma
%=
%\frac{\lambda}{\Abs{m}^2}
%\frac{m_h^3}{64 \pi}
%=
%\inv{ (246 \,\si{GeV})^2 } \frac{ (125 \,\si{GeV})^3 }{ 64 \pi }
%=
%0.161 \,\si{GeV}
%\end{dmath}

I was initially unsure how to compare this meaningfully to figure 5 of the referenced document, since the rest mass of the Higgs is 125GeV, yet \( \Gamma \) was plotted at a range of \( \si{GeV} \) values.  However, this document appears to roughly coincide with the date of the Higgs discovery.  We see in the figures that the decay rate (on a logarithmic scale!) is much smaller for values of the Higgs mass roughly below the threshold mass at which the Higgs was discovered.  In a sense, they allow for a determination of the mass, by looking at the energy ranges for which there are scattering events of the desired types.

\makeSubAnswer{}{qft:problemSet4:2d}
From \cref{fig:hw4p2Mathematica:hw4p2MathematicaFig1}, we see that \( \lambda = 127 \,\si{GeV} \), which isn't small by any typical measure!
}
}
