%
% Copyright � 2018 Peeter Joot.  All Rights Reserved.
% Licenced as described in the file LICENSE under the root directory of this GIT repository.
%
\makeproblem{The ``$h \rightarrow WW, ZZ$" Higgs-decay width.}{qft:problemSet4:2}{
From the $SU(2)_L \times SU(2)_R$ model of Homework 2---really, the Higgs Lagrangian of the Standard Model, find the coupling of the $h$-particle (the Higgs boson) to the $\phi^a$ particles (these are now Goldstone bosons, in the electroweak theory, they become the longitudinal components of the $W$ and $Z$ particles). Canonically normalizing $h$ and $\phi^a$, this coupling has the form
\begin{equation}
\label{gg1}
const. \; h \; \partial_\mu \phi^a \partial^\mu \phi^a~.
\end{equation}

\makesubproblem{}{qft:problemSet4:2a}
 Determine the value of $const.$ for canonically normalized $h$ and $\phi^a$.
\makesubproblem{}{qft:problemSet4:2b}
Use this coupling to compute the width  $\Gamma(h \rightarrow \phi^3 \phi^3)$ of the Higgs particle to decay to two longitudinal (say) $Z$-bosons (hence the index $3$).
\makesubproblem{}{qft:problemSet4:2c}
 Plug in some numbers. Use the fact that the vacuum expectation value $|m|/\sqrt{\lambda}  = 246$ GeV  and the fact that $m_h = 125$GeV to get a number for the lifetime. Compare to the total width of the Higgs from \url{http://pdg.lbl.gov/2012/reviews/rpp2012-rev-higgs-boson.pdf}, see figure 5 there, as well to the partial width to $WW$ given in Figure 4 there.
\makesubproblem{}{qft:problemSet4:2d}
 At the same time, determine the values of $|m|$ and $\lambda$ separately. Is $\lambda \ll 1$ (i.e. perturbative)?

{\flushleft {\small Notice that this calculation would have been physically relevant had the Higgs been heavy, $m_h \gg m_W \sim 100$ GeV. This is because the $h\rightarrow WW$ decay then is dominated (in this limit) by the decay into the longitudinal component, which is really the Goldstone boson field $\phi^a$ (in this limit, the result is independent of the gauge couplings $g_{1,2}$ of the Standard Model). Nonetheless, having some real numbers in this class is good.}}
} % makeproblem

\makeanswer{qft:problemSet4:2}{
\makeSubAnswer{}{qft:problemSet4:2a}
Here's a reminder and summary of the Higgs Lagrangian we will be working with in this problem
\begin{dmath}\label{eqn:ProblemSet4Problem2:640}
\LL = \trace{
   \lr{
      \partial_\mu H^\dagger \partial^\mu H
   }
}
- V,
\end{dmath}
where
\begin{dmath}\label{eqn:ProblemSet4Problem2:660}
V =
-\Abs{m}^2 \trace{
   \lr{
      H^\dagger H
   }
}
+ \lambda
\lr{
   \trace{
      H^\dagger H
   }
}^2.
\end{dmath}
It was postulated that the field had a radial component \( h \), the Higgs field, and an rotational component \( \Omega \), where the total field was given by
\begin{dmath}\label{eqn:ProblemSet4Problem2:680}
H(x) = \frac{\Abs{m}}{2 \sqrt{ \lambda } }\Omega(x) ( 1 + h(x) ),
\end{dmath}
where
\begin{equation}\label{eqn:ProblemSet4Problem2:700}
\Omega = e^{ i \Bsigma \cdot \Bphi } = e^{i \phi^a(x) \sigma^a }.
\end{equation}

Assuming that \( h(x) \) and \( \phi^a(x) \) commute, \( H^\dagger H \) can be computed with relative ease, and has only radial dependence
\begin{dmath}\label{eqn:ProblemSet4Problem2:380}
\trace{\lr{H^\dagger H}}
=
\frac{\Abs{m}^2}{4 \lambda} (1 + h(x))^2 \trace{\lr{ e^{-i \Bsigma \cdot \Bphi} e^{i \Bsigma \cdot \Bphi} }}
=
\frac{\Abs{m}^2}{4 \lambda} (1 + h(x))^2 \trace{\BOne}
=
\frac{\Abs{m}^2}{2 \lambda} (1 + h)^2.
\end{dmath}
For the derivative quadratic form, we find
\begin{dmath}\label{eqn:ProblemSet4Problem2:400}
\partial_\mu H^\dagger \partial^\mu H
=
\frac{\Abs{m}^2}{4 \lambda}
\lr{
   \partial_\mu h \Omega^\dagger
   + (1 + h) \partial_\mu \Omega^\dagger
}
\lr{
   \partial^\mu h \Omega
   + (1 + h)
\partial^\mu \Omega
}
=
\frac{\Abs{m}^2}{4 \lambda}
\lr{
   \partial_\mu h \Omega^\dagger \partial^\mu h \Omega
   + (1 + h)
      \lr{
         \partial_\mu h
         \Omega^\dagger (\partial^\mu \Omega)
       +
         \partial^\mu h
         (\partial_\mu \Omega^\dagger) \Omega
      }
   + (1 + h)^2 \partial_\mu \Omega^\dagger \partial^\mu \Omega
}.
\end{dmath}
Because \( \Omega^\dagger \Omega = 1 \), we have
\begin{dmath}\label{eqn:ProblemSet4Problem2:480}
\partial_\mu h
\Omega^\dagger (\partial^\mu \Omega)
 +
\partial^\mu h
(\partial_\mu \Omega^\dagger) \Omega
=
\partial_\mu h
\lr{
   \Omega^\dagger (\partial^\mu \Omega)
    +
   (\partial^\mu \Omega^\dagger) \Omega
}
=
\partial_\mu h
\lr{
   \partial^\mu (\Omega^\dagger \Omega) - (\partial^\mu \Omega^\dagger) \Omega
    +
   (\partial^\mu \Omega^\dagger) \Omega
}
=
   \partial^\mu (1)
= 0.
\end{dmath}
All the cross terms with both \( h \) and \( \Omega \) derivatives are zero (to all orders, not just quadratic).

Taking traces (and using cyclic permutation of the matrices in the trace operations),
the Lagrangian density is now determined
\begin{dmath}\label{eqn:ProblemSet4Problem2:500}
\LL =
\frac{\Abs{m}^2}{2 \lambda}
   \partial_\mu h \partial^\mu h
+
\frac{\Abs{m}^2}{4 \lambda} ( 1 + h )^2
   \trace{\lr{
      \partial_\mu \Omega^\dagger \partial^\mu \Omega
   }}
+ \Abs{m}^2
\frac{\Abs{m}^2}{2 \lambda} \lr{ 1 + h }^2
- \lambda
\lr{\frac{\Abs{m}^2}{2 \lambda}}^2
\lr{ 1 + h }^4
=
\frac{\Abs{m}^2}{\lambda} \LL',
\end{dmath}
where \( \LL' \) is the ``canonically normalized'' \footnote{Canonically normalized is assumed to mean that there's a one-half factor on the kinetic terms} Lagrangian
\begin{dmath}\label{eqn:ProblemSet4Problem2:720}
\LL' =
   \inv{2} \partial_\mu h \partial^\mu h
+
\inv{4}
( 1 + h )^2
   \trace{\lr{
      \partial_\mu \Omega^\dagger \partial^\mu \Omega
   }}
+
\inv{2}
\Abs{m}^2
\lr{ 1 + h }^2
-
\frac{\Abs{m}^2}{4}
\lr{ 1 + h }^4.
\end{dmath}

The coupling, let's call it \( c \), is to first order in \( h \) is
\begin{dmath}\label{eqn:ProblemSet4Problem2:740}
c = \inv{4} 2 h
   \trace{\lr{
      \partial_\mu \Omega^\dagger \partial^\mu \Omega
   }}.
\end{dmath}
Looking at these derivatives, to first order, we have
\begin{dmath}\label{eqn:ProblemSet4Problem2:580}
\partial_\mu \Omega
=
\partial_\mu \lr{ \BOne + i \Bsigma \cdot \Bphi }
=
i \Bsigma \cdot \partial_\mu \Bphi,
\end{dmath}
so
\begin{dmath}\label{eqn:ProblemSet4Problem2:760}
c
=
\inv{2} h
\trace{\lr{
   (-i \Bsigma \cdot \partial_\mu \Bphi^\dagger)
   (i \Bsigma \cdot \partial^\mu \Bphi)
}}
=
\inv{2}
h
\trace{\lr{
   (\Bsigma \cdot \partial_\mu \Bphi)
   (\Bsigma \cdot \partial^\mu \Bphi)
}},
\end{dmath}
where the real nature of each of the \( \phi^a \)'s has been used to eliminate the \( \dagger\).
The structure of this trace is that of
\begin{dmath}\label{eqn:ProblemSet4Problem2:780}
\trace{\lr{
   (\Bsigma \cdot \Bx)
   (\Bsigma \cdot \By)
}}
=
x^a y^b
\trace{\lr{
   \sigma^a \sigma^b
}}
=
x^a y^b
\left\{
\begin{array}{l l}
2 & \quad \mbox{\( a = b \)} \\
0 & \quad \mbox{\( a \ne b \)} \\
\end{array}
\right.
=
2 \Bx \cdot \By,
\end{dmath}
so the coupling is
\begin{dmath}\label{eqn:ProblemSet4Problem2:800}
c = h
\partial_\mu \phi^a \partial^\mu \phi^a.
\end{dmath}
This answers the question of the constant, which we find is just \( 1 \) after canonical normalization.

\makeSubAnswer{}{qft:problemSet4:2b}
TODO.
\makeSubAnswer{}{qft:problemSet4:2c}
TODO.
\makeSubAnswer{}{qft:problemSet4:2d}
TODO.
}
