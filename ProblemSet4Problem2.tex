%
% Copyright � 2018 Peeter Joot.  All Rights Reserved.
% Licenced as described in the file LICENSE under the root directory of this GIT repository.
%
\makeproblem{The ``$h \rightarrow WW, ZZ$" Higgs-decay width.}{qft:problemSet4:2}{
From the $SU(2)_L \times SU(2)_R$ model of Homework 2---really, the Higgs Lagrangian of the Standard Model, find the coupling of the $h$-particle (the Higgs boson) to the $\phi^a$ particles (these are now Goldstone bosons, in the electroweak theory, they become the longitudinal components of the $W$ and $Z$ particles). Canonically normalizing $h$ and $\phi^a$, this coupling has the form
\begin{equation}
\label{gg1}
const. \; h \; \partial_\mu \phi^a \partial^\mu \phi^a~.
\end{equation}

\makesubproblem{}{qft:problemSet4:2a}
 Determine the value of $const.$ for canonically normalized $h$ and $\phi^a$.
\makesubproblem{}{qft:problemSet4:2b}
Use this coupling to compute the width  $\Gamma(h \rightarrow \phi^3 \phi^3)$ of the Higgs particle to decay to two longitudinal (say) $Z$-bosons (hence the index $3$). 
\makesubproblem{}{qft:problemSet4:2c}
 Plug in some numbers. Use the fact that the vacuum expectation value $|m|/\sqrt{\lambda}  = 246$ GeV  and the fact that $m_h = 125$GeV to get a number for the lifetime. Compare to the total width of the Higgs from \url{http://pdg.lbl.gov/2012/reviews/rpp2012-rev-higgs-boson.pdf}, see figure 5 there, as well to the partial width to $WW$ given in Figure 4 there.
\makesubproblem{}{qft:problemSet4:2d}
 At the same time, determine the values of $|m|$ and $\lambda$ separately. Is $\lambda \ll 1$ (i.e. perturbative)?  

{\flushleft {\small Notice that this calculation would have been physically relevant had the Higgs been heavy, $m_h \gg m_W \sim 100$ GeV. This is because the $h\rightarrow WW$ decay then is dominated (in this limit) by the decay into the longitudinal component, which is really the Goldstone boson field $\phi^a$ (in this limit, the result is independent of the gauge couplings $g_{1,2}$ of the Standard Model). Nonetheless, having some real numbers in this class is good.}}
} % makeproblem

\makeanswer{qft:problemSet4:2}{
\makeSubAnswer{}{qft:problemSet4:2a}
TODO.
\makeSubAnswer{}{qft:problemSet4:2b}
TODO.
\makeSubAnswer{}{qft:problemSet4:2c}
TODO.
\makeSubAnswer{}{qft:problemSet4:2d}
TODO.
}
