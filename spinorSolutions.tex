%
% Copyright � 2018 Peeter Joot.  All Rights Reserved.
% Licenced as described in the file LICENSE under the root directory of this GIT repository.
%
%{
%%\input{../latex/blogpost.tex}
%%\renewcommand{\basename}{spinorSolutions}
%%%\renewcommand{\dirname}{notes/phy1520/}
%%\renewcommand{\dirname}{notes/ece1228-electromagnetic-theory/}
%%%\newcommand{\dateintitle}{}
%%%\newcommand{\keywords}{}
%%
%%\input{../latex/peeter_prologue_print2.tex}
%%
%%\usepackage{peeters_layout_exercise}
%%\usepackage{peeters_braket}
%%\usepackage{peeters_figures}
%%\usepackage{siunitx}
%%\usepackage{verbatim}
%%\usepackage{slashed}
%%%\usepackage{mhchem} % \ce{}
%%%\usepackage{macros_bm} % \bcM
%%%\usepackage{macros_qed} % \qedmarker
%%%\usepackage{txfonts} % \ointclockwise
%%\newcommand{\osigma}[0]{\overbar{\sigma}}
%%
%%\beginArtNoToc
%%
%%\generatetitle{Spinor solutions with alternate \( \gamma^0 \) representation.}
%\chapter{Spinor solutions with alternate \( \gamma^0 \) representation.}
\label{chap:spinorSolutions}

This follows an interesting derivation of the \( u, v \) spinors \citep{itzykson1980quantum}, adding some details.

In class (QFT I) and \citep{peskin1995introduction} we used a non-diagonal \( \gamma^0 \) representation
\begin{equation}\label{eqn:spinorSolutions:20}
\gamma^0 =
\begin{bmatrix}
0 & 1 \\
1 & 0
\end{bmatrix},
\end{equation}
whereas in \citep{itzykson1980quantum} a diagonal representation is used
\begin{equation}\label{eqn:spinorSolutions:40}
\gamma^0 =
\begin{bmatrix}
1 & 0 \\
0 & -1
\end{bmatrix}.
\end{equation}
This representation makes it particularly simple to determine the form of the \( u, v \) spinors.  We seek solutions of the Dirac equation
\begin{equation}\label{eqn:spinorSolutions:60}
\begin{aligned}
0 &= \lr{ i \gamma^\mu \partial_\mu - m } u(p) e^{-i p \cdot x}  \\
0 &= \lr{ i \gamma^\mu \partial_\mu - m } v(p) e^{i p \cdot x},
\end{aligned}
\end{equation}
or
\begin{equation}\label{eqn:spinorSolutions:80}
\begin{aligned}
0 &= \lr{ \slashed{p} - m } u(p) e^{-i p \cdot x}  \\
0 &= -\lr{ \slashed{p} + m } v(p) e^{i p \cdot x}.
\end{aligned}
\end{equation}
In the rest frame where \( \slashed{p} = E \gamma^0 \), where \( E = m = \omega_\Bp \), these take the particularly simple form
\begin{equation}\label{eqn:spinorSolutions:100}
\begin{aligned}
0 &= \lr{ \gamma^0 - 1 } u(E, \Bzero) \\
0 &= \lr{ \gamma^0 + 1 } v(E, \Bzero).
\end{aligned}
\end{equation}
This is a nice relation, as we can determine a portion of the structure of the rest frame \( u, v \) that is independent of the Dirac matrix representation
\begin{equation}\label{eqn:spinorSolutions:120}
\begin{aligned}
u(E, \Bzero) &= (\gamma^0 + 1) \psi \\
v(E, \Bzero) &= (\gamma^0 - 1) \psi.
\end{aligned}
\end{equation}
Similarly, and more generally, we have
\begin{equation}\label{eqn:spinorSolutions:140}
\begin{aligned}
u(p) &= (\slashed{p} + m) \psi \\
v(p) &= (\slashed{p} - m) \psi,
\end{aligned}
\end{equation}
also independent of the representation of \( \gamma^\mu \).  Looking forward to non-matrix representations of the Dirac equation (\citep{doran2003gap}) note that we have not yet imposed a spinorial structure on the solution
\begin{equation}\label{eqn:spinorSolutions:260}
\psi
=
\begin{bmatrix}
\phi \\
\chi
\end{bmatrix},
\end{equation}
where \( \phi, \chi \) are two component matrices.

The particular choice of the diagonal representation \cref{eqn:spinorSolutions:40} for \( \gamma^0 \) makes it simple to determine additional structure for \( u, v \).  Consider the rest frame first, where
\begin{equation}\label{eqn:spinorSolutions:160}
\begin{aligned}
\gamma^0 - 1 &=
\begin{bmatrix}
1 & 0 \\
0 & -1
\end{bmatrix}
-
\begin{bmatrix}
1 & 0 \\
0 & -1
\end{bmatrix}
=
\begin{bmatrix}
0 & 0 \\
0 & 2
\end{bmatrix} \\
\gamma^0 + 1 &=
\begin{bmatrix}
1 & 0 \\
0 & -1
\end{bmatrix}
+
\begin{bmatrix}
1 & 0 \\
0 & -1
\end{bmatrix}
=
\begin{bmatrix}
2 & 0 \\
0 & 0
\end{bmatrix},
\end{aligned}
\end{equation}
so we have
\begin{equation}\label{eqn:spinorSolutions:280}
\begin{aligned}
u(E, \Bzero) &=
\begin{bmatrix}
2 & 0 \\
0 & 0
\end{bmatrix}
\begin{bmatrix}
\phi \\
\chi
\end{bmatrix} \\
v(E, \Bzero) &=
\begin{bmatrix}
0 & 0 \\
0 & 2
\end{bmatrix}
\begin{bmatrix}
\phi \\
\chi
\end{bmatrix}.
\end{aligned}
\end{equation}
Therefore a basis for the spinors \( u \) (in the rest frame), is
\begin{equation}\label{eqn:spinorSolutions:180}
u(E, \Bzero) \in \setlr{
\begin{bmatrix}
1 \\
0 \\
0 \\
0
\end{bmatrix},
\begin{bmatrix}
0 \\
1 \\
0 \\
0
\end{bmatrix}
},
\end{equation}
and a basis for the rest frame spinors \( v \) is
\begin{equation}\label{eqn:spinorSolutions:200}
v(E, \Bzero) \in \setlr{
\begin{bmatrix}
0 \\
0 \\
1 \\
0
\end{bmatrix},
\begin{bmatrix}
0 \\
0 \\
0 \\
1
\end{bmatrix}
}.
\end{equation}
Using the two spinor bases \( \zeta^a, \eta^a \) notation from class, we can write these
\begin{equation}\label{eqn:spinorSolutions:220}
\begin{aligned}
u^a(E, \Bzero) &=
\begin{bmatrix}
\zeta^a \\
0
\end{bmatrix},
\qquad
v^a(E, \Bzero) &=
\begin{bmatrix}
0 \\
\eta^a \\
\end{bmatrix}.
\end{aligned}
\end{equation}

For the non-rest frame solutions, \citep{itzykson1980quantum} opts not to boost, as in \citep{peskin1995introduction}, but to use the geometry of \( \slashed{p} \pm m \).  With their diagonal representation of \( \gamma^0 \) those are
\begin{equation}\label{eqn:spinorSolutions:240}
\begin{aligned}
\slashed{p} - m
&=
p_0
\begin{bmatrix}
1 & 0 \\
0 & -1
\end{bmatrix}
+
p_k
\begin{bmatrix}
0 & \sigma^k \\
- \sigma^k & 0
\end{bmatrix}
-
m
\begin{bmatrix}
1 & 0 \\
0 & 1
\end{bmatrix}
=
\begin{bmatrix}
E - m & - \Bsigma \cdot \Bp \\
\Bsigma \cdot \Bp & -E - m
\end{bmatrix} \\
\slashed{p} + m
&=
p_0
\begin{bmatrix}
1 & 0 \\
0 & -1
\end{bmatrix}
+
p_k
\begin{bmatrix}
0 & \sigma^k \\
- \sigma^k & 0
\end{bmatrix}
+
m
\begin{bmatrix}
1 & 0 \\
0 & 1
\end{bmatrix}
=
\begin{bmatrix}
E + m & - \Bsigma \cdot \Bp \\
\Bsigma \cdot \Bp & -E + m
\end{bmatrix}.
\end{aligned}
\end{equation}

Let's assume that the arbitrary momentum solutions \cref{eqn:spinorSolutions:140} are each proportional to the rest frame solutions
\begin{equation}\label{eqn:spinorSolutions:300}
\begin{aligned}
u^a(p) &= (\slashed{p} + m) u^a(E, \Bzero) \\
v^a(p) &= (\slashed{p} - m) u^a(E, \Bzero).
\end{aligned}
\end{equation}
Plugging in \cref{eqn:spinorSolutions:240} gives
\begin{equation}\label{eqn:spinorSolutions:320}
\begin{aligned}
u^a(p) &=
\begin{bmatrix}
(E + m) \zeta^a \\
(\Bsigma \cdot \Bp ) \zeta^a
\end{bmatrix} \\
v^a(p) &=
\begin{bmatrix}
(\Bsigma \cdot \Bp) \eta^a \\
(E + m) \eta^a
\end{bmatrix},
\end{aligned}
\end{equation}
where an overall sign on \( v^a(p) \) has been dropped.  Let's check the assumption that the rest frame and general solutions are so simply related
\begin{equation}\label{eqn:spinorSolutions:340}
\begin{aligned}
\lr{ \slashed{p} - m } u^a(p)
&=
\begin{bmatrix}
E - m & - \Bsigma \cdot \Bp \\
\Bsigma \cdot \Bp & -E - m
\end{bmatrix}
\begin{bmatrix}
(E + m) \zeta^a \\
(\Bsigma \cdot \Bp ) \zeta^a
\end{bmatrix}
\\&=
\begin{bmatrix}
(E^2 - m^2 - \Bp^2) \zeta^a \\
0
\end{bmatrix}
\\&= 0,
\end{aligned}
\end{equation}
and
\begin{equation}\label{eqn:spinorSolutions:360}
\begin{aligned}
\lr{ \slashed{p} + m } v^a(p)
&=
\begin{bmatrix}
E + m & - \Bsigma \cdot \Bp \\
\Bsigma \cdot \Bp & -E + m
\end{bmatrix}
\begin{bmatrix}
(\Bsigma \cdot \Bp ) \eta^a \\
(E + m) \eta^a \\
\end{bmatrix}
\\&=
\begin{bmatrix}
0 \\
\Bp^2 + m^2 - E^2
\end{bmatrix}
\\&= 0.
\end{aligned}
\end{equation}
Everything works out nicely.  The form of the solution for this representation of \( \gamma^0 \) is much simpler than the Chiral solution that we found in class.  We end up with an explicit split of energy and spatial momentum components in the spinor solutions, instead of factors involving \( p \cdot \sigma \) and \( p \cdot \osigma \), which are arguably nicer from a Lorentz invariance point of view.

%}
%\EndArticle
