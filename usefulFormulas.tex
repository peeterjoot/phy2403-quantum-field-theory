%
% Copyright © 2018 Peeter Joot.  All Rights Reserved.
% Licenced as described in the file LICENSE under the root directory of this GIT repository.
%
%\paragraph{Useful identities.}

\begin{itemize}
\item Gaussian
\begin{dmath}\label{eqn:usefulFormulas:20}
\int_{-\infty}^\infty e^{a x^2} dx = \sqrt{\frac{-\pi}{a}}.
\end{dmath}
Here \( a \) may be real or imaginary, but must be less than 0 if real.
\item Our Fourier transform sign and \( \pi \) placement convention (\( 2 \pi\)'s with momentum elements and negative exponential sign for the inverse transform)
\begin{dmath}\label{eqn:usefulFormulas:40}
\begin{aligned}
f(x) &= \int \frac{d^n k}{(2 \pi)^n} f(k) e^{i k \cdot x} \\
f(k) &= \int d^n x f(x) e^{-i k \cdot x}.
\end{aligned}
\end{dmath}
\item Delta function representation.

Setting \( f(x) = \delta^n(x) \) implies \( f(k) = 1 \) and so
\begin{dmath}\label{eqn:usefulFormulas:60}
\delta(x) = \int \frac{d^n k}{(2 \pi)^n} e^{i k \cdot x}.
\end{dmath}

\item Correct sign for the commutator
\begin{dmath}\label{eqn:usefulFormulas:80}
\antisymmetric{x_r}{p_s} = i \delta_{rs}.
\end{dmath}
\item Hamilton's equations
\begin{dmath}\label{eqn:usefulFormulas:100}
\begin{aligned}
-dH &= d(\LL - p \dot{q}) = \PD{q}{\LL} dq + \cancel{\PD{\dot{q}}{\LL} d\dot{q}} + \PD{t}{\LL} dt - dp \dot{q} - \cancel{p d\dot{q}} \\
dH &=
\PD{q}{H} dq
+
\PD{p}{H} dp
+
\PD{t}{H} dt,
\end{aligned}
\end{dmath}
so
\begin{equation}\label{eqn:usefulFormulas:120}
\PD{q}{H} = -\dot{p},\qquad
\PD{p}{H} = \dot{q},\qquad
\PD{t}{H} = -\PD{t}{\LL}.
\end{equation}
\item Matrix element for the momentum and position operators
\begin{dmath}\label{eqn:usefulFormulas:140}
\begin{aligned}
\bra{x} P \ket{x'} &= -i \delta(x- x') \frac{d}{dx} \\
\bra{p} X \ket{p'} &= i \delta(p- p') \frac{d}{dp}.
\end{aligned}
\end{dmath}
\item Eigenstates
\begin{dmath}\label{eqn:usefulFormulas:160}
\begin{aligned}
p \braket{x}{p}
&= \bra{x} P \ket{p} \\
&= \int dx' \bra{x} P \ket{x'} \braket{x'}{p} \\
&= \int dx' (-i) \delta(x - x') \frac{d}{dx} \braket{x'}{p} \\
&= -i \frac{d}{dx} \braket{x}{p},
\end{aligned}
\end{dmath}
so
\begin{dmath}\label{eqn:usefulFormulas:180}
\braket{x}{p} \propto e^{i p x}.
\end{dmath}
Normalized over all space in \( d \) dimensions
\begin{dmath}\label{eqn:usefulFormulas:200}
\braket{x}{p} = \frac{e^{i p \cdot x}}{(2 \pi)^{d/2}}.
\end{dmath}
\item Time evolution in the Heisenberg picture
\begin{dmath}\label{eqn:usefulFormulas:220}
\ddt{O} = i \antisymmetric{H}{O}.
\end{dmath}
\item Commutators of powers of position and momentum operators
\begin{dmath}\label{eqn:usefulFormulas:240}
\begin{aligned}
\antisymmetric{\hatq^n}{\hatp} &= n i \hatq^{n-1} \\
\antisymmetric{\hatp^n}{\hatq} &= -n i \hatp^{n-1}.
\end{aligned}
\end{dmath}
More generally, for any function with a power series representation \( F(x) = \sum_{k = 0}^\infty a_k x^k \), we have
\begin{dmath}\label{eqn:usefulFormulas:260}
\begin{aligned}
\antisymmetric{F(\hatq)}{\hatp} &= i \frac{dF}{d\hatq} \\
\antisymmetric{F(\hatp)}{\hatq} &= -i \frac{dF}{d\hatp}.
\end{aligned}
\end{dmath}
\item Pauli matrices
\begin{equation}\label{eqn:usefulFormulas:280}
\sigma^1 = \PauliX, \quad
\sigma^2 = \PauliY, \quad
\sigma^3 = \PauliZ.
\end{equation}
\begin{dmath}\label{eqn:usefulFormulas:300}
\antisymmetric{\sigma^a}{\sigma^b} = 2 i \epsilon^{a b c} \sigma^c.
\end{dmath}
\end{itemize}
