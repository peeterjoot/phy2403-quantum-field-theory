%
% Copyright � 2018 Peeter Joot.  All Rights Reserved.
% Licenced as described in the file LICENSE under the root directory of this GIT repository.
%
\makeproblem{Part of Problem 2.2 from Peskin and Schroeder (reproduced below).}{qft:problemSet1:2}{
Consider a complex scalar field with action \( S = \int d^4x\lr{\partial_\mu \phi^\conj \partial^\mu \phi - m^2 \phi^\conj \phi}\).  When doing the variational principle consider \( \phi \) and \(\phi^\conj \) as independent, rather than their real and imaginary parts (this is equivalent, but more convenient).

\makesubproblem{}{qft:problemSet1:2a}
Show that \( H = \int d^3x \lr{ \pi^\conj \pi + \spacegrad \phi^\conj \cdot \spacegrad \phi + m^2 \phi^\conj \phi } \) and that the Klein-Gordon equation is obeyed by \( \phi \) and \( \phi^\conj\).
\makesubproblem{}{qft:problemSet1:2b}
Introduce complex amplitudes, diagonalize the Hamiltonian, and quantize the theory. Show that the theory has now two sets of particles.
\makesubproblem{}{qft:problemSet1:2c}
Write the charge conserved due to the global \( U(1) \) symmetry, \( Q = \int d^3 x \frac{i}{2} \lr{ \phi^\conj \pi^\conj - \pi \phi } \),
in
terms of creation and annihilation operators and find the charge of the particles of each type.
} % makeproblem

\makeanswer{qft:problemSet1:2}{
Before proceeding to the question as stated, let's compute the Euler-Lagrange equations to verify that we find the Klein-Gordon equation.
\begin{dmath}\label{eqn:ProblemSet1Problem2:20}
\begin{aligned}
\PD{\phi}{\LL} &= -m^2 \phi^\conj \\
\PD{(\partial_\mu \phi)}{\LL} &= \partial^\mu \phi^\conj \\
\PD{\phi^\conj}{\LL} &= -m^2 \phi       \\
\PD{(\partial_\mu \phi^\conj)}{\LL} &= \partial^\mu \phi,
\end{aligned}
\end{dmath}
so the equations of the field are respectively
\begin{dmath}\label{eqn:ProblemSet1Problem2:40}
\begin{aligned}
\partial_\mu \partial^\mu \phi^\conj &=  -m^2 \phi^\conj \\
\partial_\mu \partial^\mu \phi &=  -m^2 \phi,
\end{aligned}
\end{dmath}
as expected.
\makeSubAnswer{}{qft:problemSet1:2a}
The Lagrangian density when written out explicitly is
\begin{dmath}\label{eqn:ProblemSet1Problem2:60}
\LL = \partial_0 \phi^\conj \partial_0 \phi - (\spacegrad \phi^\conj) \cdot (\spacegrad \phi) - m^2 \phi^\conj \phi,
\end{dmath}
so the conjugate momentum densities are
\begin{dmath}\label{eqn:ProblemSet1Problem2:80}
\begin{aligned}
\pi(\Bx, t) &= \PD{(\partial_0 \phi)}{\LL} = \partial_0 \phi^\conj \\
\pi^\conj(\Bx, t) &= \PD{(\partial_0 \phi^\conj)}{\LL} = \partial_0 \phi \\
\end{aligned}
\end{dmath}

The Hamiltonian is
\begin{dmath}\label{eqn:ProblemSet1Problem2:100}
H
= \int d^3 x \lr{ \pi \partial_0 \phi + \pi^\conj \partial_0 \phi^\conj - \LL }
=
\int d^3 x \lr{ \pi \pi^\conj + \pi^\conj \pi - \pi \pi^\conj +
(\spacegrad \phi^\conj) \cdot (\spacegrad \phi) + m^2 \phi^\conj \phi
 }
=
\int d^3 x \lr{ \pi^\conj \pi +
(\spacegrad \phi^\conj) \cdot (\spacegrad \phi) + m^2 \phi^\conj \phi
 }
\end{dmath}
TODO.
\makeSubAnswer{}{qft:problemSet1:2b}

TODO.
\makeSubAnswer{}{qft:problemSet1:2c}

TODO.
}
