%
% Copyright � 2018 Peeter Joot.  All Rights Reserved.
% Licenced as described in the file LICENSE under the root directory of this GIT repository.
%
%{
\makeproblem{Part of Problem 2.2 from Peskin and Schroeder (reproduced below).}{qft:problemSet1:2}{
Consider a complex scalar field with action \( S = \int d^4x\lr{\partial_\mu \phi^\dagger \partial^\mu \phi - m^2 \phi^\dagger \phi}\).  When doing the variational principle consider \( \phi \) and \(\phi^\dagger \) as independent, rather than their real and imaginary parts (this is equivalent, but more convenient).

\makesubproblem{}{qft:problemSet1:2a}
Show that \( H = \int d^3x \lr{ \pi^\dagger \pi + \spacegrad \phi^\dagger \cdot \spacegrad \phi + m^2 \phi^\dagger \phi } \) and that the Klein-Gordon equation is obeyed by \( \phi \) and \( \phi^\dagger\).
\makesubproblem{}{qft:problemSet1:2b}
Introduce complex amplitudes, diagonalize the Hamiltonian, and quantize the theory. Show that the theory has now two sets of particles.
\makesubproblem{}{qft:problemSet1:2c}
Write the charge conserved due to the global \( U(1) \) symmetry, \( Q = \int d^3 x \frac{i}{2} \lr{ \phi^\dagger \pi^\dagger - \pi \phi } \),
in
terms of creation and annihilation operators and find the charge of the particles of each type.
} % makeproblem

\makeanswer{qft:problemSet1:2}{
\makeSubAnswer{}{qft:problemSet1:2a}
Classically, evaluating the Euler-Lagrange equations gives us
\begin{dmath}\label{eqn:ProblemSet1Problem2:20}
\begin{aligned}
\PD{\phi}{\LL} &= -m^2 \phi^\dagger \\
\PD{(\partial_\mu \phi)}{\LL} &= \partial^\mu \phi^\dagger \\
\PD{\phi^\dagger}{\LL} &= -m^2 \phi       \\
\PD{(\partial_\mu \phi^\dagger)}{\LL} &= \partial^\mu \phi,
\end{aligned}
\end{dmath}
so the equations of the field are respectively
\begin{dmath}\label{eqn:ProblemSet1Problem2:40}
%\boxedEquation{eqn:ProblemSet1Problem2:40}{
\begin{aligned}
\partial_\mu \partial^\mu \phi^\dagger &=  -m^2 \phi^\dagger \\
\partial_\mu \partial^\mu \phi &=  -m^2 \phi.
\end{aligned}
%}
\end{dmath}
These are Klein-Gordon equations for each field variable \( \phi, \phi^\dagger \) as expected, although this can be made more explicit written out explicitly in the stationary observer frame
\boxedEquation{eqn:ProblemSet1Problem2:340}{
\begin{aligned}
\lr{ \partial_{tt} - \spacegrad^2 + m^2 } \phi^\dagger &= 0 \\
\lr{ \partial_{tt} - \spacegrad^2 + m^2 } \phi &= 0 \\
\end{aligned}
}
To find the Hamiltonian, note that the Lagrangian density written out explicitly is
\begin{dmath}\label{eqn:ProblemSet1Problem2:60}
\LL = \partial_0 \phi^\dagger \partial_0 \phi - (\spacegrad \phi^\dagger) \cdot (\spacegrad \phi) - m^2 \phi^\dagger \phi,
\end{dmath}
so the conjugate momentum densities are
\begin{dmath}\label{eqn:ProblemSet1Problem2:80}
\begin{aligned}
\pi(\Bx, t) &= \PD{(\partial_0 \phi)}{\LL} = \partial_0 \phi^\dagger \\
\pi^\dagger(\Bx, t) &= \PD{(\partial_0 \phi^\dagger)}{\LL} = \partial_0 \phi \\
\end{aligned}
\end{dmath}

The Hamiltonian (including a \( p \dot{q} \) term for each of \( \phi, \phi^\dagger \)) is
\begin{dmath}\label{eqn:ProblemSet1Problem2:100}
H
= \int d^3 x \lr{ \pi \partial_0 \phi + \pi^\dagger \partial_0 \phi^\dagger - \LL }
=
\int d^3 x \lr{ \pi \pi^\dagger + \pi^\dagger \pi - \pi \pi^\dagger +
(\spacegrad \phi^\dagger) \cdot (\spacegrad \phi) + m^2 \phi^\dagger \phi
 }
=
\int d^3 x \lr{ \pi^\dagger \pi +
(\spacegrad \phi^\dagger) \cdot (\spacegrad \phi) + m^2 \phi^\dagger \phi
 }
\end{dmath}
\makeSubAnswer{}{qft:problemSet1:2b}
To canonically quantize the fields, we promote the fields to operators, demand that we have commutators for conjugate pairs of operators
\begin{equation}\label{eqn:ProblemSet1Problem2:120}
\antisymmetric{\phi(\Bx)}{\pi^\dagger(\By)} =
\antisymmetric{\phi^\dagger(\Bx)}{\pi(\By)} = i \delta^3(\Bx - \By),
\end{equation}
and requiring that all the other operator pairs \( \phi \phi^\dagger, \pi \pi^\dagger, \phi\pi, \phi^\dagger \pi^\dagger \) commute\footnote{As I discovered the hard way doing this assignment is it also possible to find the KG equation by demanding \(
\antisymmetric{\phi(\Bx)}{\pi(\By)} =
\antisymmetric{\phi^\dagger(\Bx)}{\pi^\dagger(\By)} = i \delta^3(\Bx - \By) \), however, doing so leads to trouble when attempting to find the pairs of properly behaving creation and annilation operators.  For a nice discussion that motivates the ``proper choice'', requiring commutators for conjugate pairs of position-momentum operators, see \citep{DavidMayrhofer} where the author starts with separate real and imaginary fields and builds the complex representation systematically.}.
If we compute the time evolution of such quantized \( \phi, \phi^\dagger \) field operators using Hamiltonian time evolution operators, the result differs from the classical case by a conjugatation operation
\begin{dmath}\label{eqn:ProblemSet1Problem2:140}
\ddt{\phi(\Bx)}
= i \antisymmetric{H(\By)}{\phi(\Bx)}
= i \int d^3 y \antisymmetric{\pi^\dagger(\By) \pi(\By)}{\phi(\Bx)}
= i \int d^3 y \,\pi(\By) \antisymmetric{\pi^\dagger(\By)}{\phi(\Bx)}
= \int d^3 y \,\pi(\By) \delta^3(\Bx - \By)
= \pi(\Bx),
\end{dmath}
\begin{dmath}\label{eqn:ProblemSet1Problem2:160}
\ddt{\phi^\dagger(\Bx)}
= i \antisymmetric{H(\By)}{\phi^\dagger(\Bx)}
= i \int d^3 y \antisymmetric{\pi^\dagger(\By) \pi(\By)}{\phi^\dagger(\Bx)}
= i \int d^3 y \pi^\dagger(\By) \antisymmetric{\pi(\By)}{\phi(\Bx)}
= \int d^3 y \pi^\dagger(\By) \delta^3(\Bx - \By)
= \pi^\dagger(\Bx),
\end{dmath}
which differs from \cref{eqn:ProblemSet1Problem2:80}.  However, should we compute the time evolution of the momentum operators,
this conjugation difference is ``cancelled'' and we end up with the Klein-Gordon equations in the end.
\begin{dmath}\label{eqn:ProblemSet1Problem2:180}
\ddt{\pi(\Bx)}
= i \antisymmetric{H(\By)}{\pi(\Bx)}
=
i \int d^3 y \antisymmetric{
\spacegrad_\By \phi^\dagger(\By) \cdot \spacegrad_\By \phi(\By)
}{ \pi(\Bx) }
+
i m^2 \int d^3 y \antisymmetric{ \phi^\dagger(\By) \phi(\By) }{ \pi(\Bx) }.
\end{dmath}
The second integral is easy
\begin{dmath}\label{eqn:ProblemSet1Problem2:200}
\int d^3 y \antisymmetric{ \phi^\dagger(\By) \phi(\By) }{ \pi(\Bx) }
=
\int d^3 y\,
\phi(\By)
\antisymmetric{
\phi(\By) }{ \pi(\Bx) }
=
i
\int d^3 y\,
\phi(\By)
\delta^3(\Bx - \By)
=
i
\phi(\Bx).
\end{dmath}
To evaluate the first integral in \cref{eqn:ProblemSet1Problem2:180} we can make use of the linearity to find
\begin{dmath}\label{eqn:ProblemSet1Problem2:220}
i \int d^3 y \antisymmetric{ \spacegrad_\By \phi^\dagger(\By) \cdot \spacegrad_\By \phi(\By) }{ \pi(\Bx) }
=
i \int d^3 y \, \spacegrad_\By \antisymmetric{\phi^\dagger(\By)}{\pi(\Bx)} \cdot \spacegrad_\By \phi(\By)
=
- \int d^3 y \, \spacegrad_\By \delta(\Bx - \By) \cdot \spacegrad_\By \phi(\By)
=
- \int d^3 y \lr{
\spacegrad_\By \cdot \lr{ \delta(\Bx - \By) \spacegrad_\By \phi(\By) }
-
\delta(\Bx - \By) \spacegrad_\By^2 \phi(\By)
}
=
- \int dA_y 
\delta(\Bx - \By) \PD{n}{\phi(\By)}
+
\spacegrad^2 \phi(\Bx).
\end{dmath}
Provided we have some justification for declaring the first integral zero (I'm not sure what that was in class, as the delta function isn't really well behaved unless integrated over a volume), we are left with
\begin{dmath}\label{eqn:ProblemSet1Problem2:260}
\ddt{\pi(\Bx)}
=
\spacegrad^2 \phi(\Bx) - m^2 \phi(\Bx),
\end{dmath}
or
\begin{dmath}\label{eqn:ProblemSet1Problem2:280}
\frac{d^2 \phi}{dt^2}
=
\spacegrad^2 \phi - m^2 \phi.
\end{dmath}
which is a KG equation for \( \phi\).
As the Hamiltonian is symmetric in \( \pi, \pi^\dagger \) and \( \phi, \phi^\dagger \) repeating this calculation for \( \dot{\pi}^\dagger \) gives
\begin{dmath}\label{eqn:ProblemSet1Problem2:300}
\ddt{\pi^\dagger(\Bx)}
=
\spacegrad^2 \phi^\dagger(\Bx) - m^2 \phi^\dagger(\Bx),
\end{dmath}
which is also a KG equation for the quantized field \( \phi \)
\begin{dmath}\label{eqn:ProblemSet1Problem2:320}
\frac{d^2 \phi^\dagger}{dt^2}
=
\spacegrad^2 \phi^\dagger - m^2 \phi^\dagger,
\end{dmath}
as expected.

\makeSubAnswer{}{qft:problemSet1:2c}

TODO.
}
%}
