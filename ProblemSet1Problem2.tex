%
% Copyright � 2018 Peeter Joot.  All Rights Reserved.
% Licenced as described in the file LICENSE under the root directory of this GIT repository.
%
%{
\makeproblem{Part of Problem 2.2 from Peskin and Schroeder (reproduced below).}{qft:problemSet1:2}{
Consider a complex scalar field with action \( S = \int d^4x\lr{\partial_\mu \phi^\dagger \partial^\mu \phi - m^2 \phi^\dagger \phi}\).  When doing the variational principle consider \( \phi \) and \(\phi^\dagger \) as independent, rather than their real and imaginary parts (this is equivalent, but more convenient).

\makesubproblem{}{qft:problemSet1:2a}
Show that \( H = \int d^3x \lr{ \pi^\dagger \pi + \spacegrad \phi^\dagger \cdot \spacegrad \phi + m^2 \phi^\dagger \phi } \) and that the Klein-Gordon equation is obeyed by \( \phi \) and \( \phi^\dagger\).
\makesubproblem{}{qft:problemSet1:2b}
Introduce complex amplitudes, diagonalize the Hamiltonian, and quantize the theory. Show that the theory has now two sets of particles.
\makesubproblem{}{qft:problemSet1:2c}
Write the charge conserved due to the global \( U(1) \) symmetry,
\begin{dmath}\label{eqn:ProblemSet1Problem2:580}
Q = \int d^3 x \frac{i}{2} \lr{ \phi^\dagger \pi - \pi^\dagger \phi },
\end{dmath}
in
terms of creation and annihilation operators and find the charge of the particles of each type.
} % makeproblem

\makeanswer{qft:problemSet1:2}{
\makeSubAnswer{}{qft:problemSet1:2a}
Classically, evaluating the Euler-Lagrange equations gives us
\begin{dmath}\label{eqn:ProblemSet1Problem2:20}
\begin{aligned}
\PD{\phi}{\LL} &= -m^2 \phi^\dagger \\
\PD{(\partial_\mu \phi)}{\LL} &= \partial^\mu \phi^\dagger \\
\PD{\phi^\dagger}{\LL} &= -m^2 \phi       \\
\PD{(\partial_\mu \phi^\dagger)}{\LL} &= \partial^\mu \phi,
\end{aligned}
\end{dmath}
so the equations of the field are respectively
\begin{dmath}\label{eqn:ProblemSet1Problem2:40}
%\boxedEquation{eqn:ProblemSet1Problem2:40}{
\begin{aligned}
\partial_\mu \partial^\mu \phi^\dagger &=  -m^2 \phi^\dagger \\
\partial_\mu \partial^\mu \phi &=  -m^2 \phi.
\end{aligned}
%}
\end{dmath}
These are Klein-Gordon equations for each field variable \( \phi, \phi^\dagger \) as expected, although this can be made more explicit written out explicitly in the stationary observer frame
\boxedEquation{eqn:ProblemSet1Problem2:340}{
\begin{aligned}
\lr{ \partial_{tt} - \spacegrad^2 + m^2 } \phi^\dagger &= 0 \\
\lr{ \partial_{tt} - \spacegrad^2 + m^2 } \phi &= 0 \\
\end{aligned}
}
To find the Hamiltonian, note that the Lagrangian density written out explicitly is
\begin{dmath}\label{eqn:ProblemSet1Problem2:60}
\LL = \partial_0 \phi^\dagger \partial_0 \phi - (\spacegrad \phi^\dagger) \cdot (\spacegrad \phi) - m^2 \phi^\dagger \phi,
\end{dmath}
so the conjugate momentum densities are
\begin{dmath}\label{eqn:ProblemSet1Problem2:80}
\begin{aligned}
\pi(\Bx, t) &= \PD{(\partial_0 \phi)}{\LL} = \partial_0 \phi^\dagger \\
\pi^\dagger(\Bx, t) &= \PD{(\partial_0 \phi^\dagger)}{\LL} = \partial_0 \phi \\
\end{aligned}
\end{dmath}

The Hamiltonian (including a \( p \dot{q} \) term for each of \( \phi, \phi^\dagger \)) is
\begin{dmath}\label{eqn:ProblemSet1Problem2:100}
H
= \int d^3 x \lr{ \pi \partial_0 \phi + \pi^\dagger \partial_0 \phi^\dagger - \LL }
=
\int d^3 x \lr{ \pi \pi^\dagger + \pi^\dagger \pi - \pi \pi^\dagger +
(\spacegrad \phi^\dagger) \cdot (\spacegrad \phi) + m^2 \phi^\dagger \phi
 }
=
\int d^3 x \lr{ \pi^\dagger \pi +
(\spacegrad \phi^\dagger) \cdot (\spacegrad \phi) + m^2 \phi^\dagger \phi
 }
\end{dmath}
\makeSubAnswer{}{qft:problemSet1:2b}
To canonically quantize the fields, we promote the fields to operators, demand that we have commutators for conjugate pairs of operators
\begin{equation}\label{eqn:ProblemSet1Problem2:120}
\antisymmetric{\phi(\Bx)}{\pi^\dagger(\By)} =
\antisymmetric{\phi^\dagger(\Bx)}{\pi(\By)} = i \delta^3(\Bx - \By),
\end{equation}
and requiring that all the other operator pairs \( \phi \phi^\dagger, \pi \pi^\dagger, \phi\pi, \phi^\dagger \pi^\dagger \) commute\footnote{As I discovered the hard way doing this assignment is it also possible to find the KG equation by demanding \(
\antisymmetric{\phi(\Bx)}{\pi(\By)} =
\antisymmetric{\phi^\dagger(\Bx)}{\pi^\dagger(\By)} = i \delta^3(\Bx - \By) \), however, doing so leads to trouble when attempting to find the pairs of properly behaving creation and annihilation operators.  For a nice discussion that motivates the ``proper choice'', requiring commutators for conjugate pairs of position-momentum operators, see \citep{DavidMayrhofer} where the author starts with separate real and imaginary fields and builds the complex representation systematically.}.
If we compute the time evolution of such quantized \( \phi, \phi^\dagger \) field operators using Hamiltonian time evolution operators, the result differs from the classical case by a conjugation operation
\begin{dmath}\label{eqn:ProblemSet1Problem2:140}
\ddt{\phi(\Bx)}
= i \antisymmetric{H(\By)}{\phi(\Bx)}
= i \int d^3 y \antisymmetric{\pi^\dagger(\By) \pi(\By)}{\phi(\Bx)}
= i \int d^3 y \,\pi(\By) \antisymmetric{\pi^\dagger(\By)}{\phi(\Bx)}
= \int d^3 y \,\pi(\By) \delta^3(\Bx - \By)
= \pi(\Bx),
\end{dmath}
\begin{dmath}\label{eqn:ProblemSet1Problem2:160}
\ddt{\phi^\dagger(\Bx)}
= i \antisymmetric{H(\By)}{\phi^\dagger(\Bx)}
= i \int d^3 y \antisymmetric{\pi^\dagger(\By) \pi(\By)}{\phi^\dagger(\Bx)}
= i \int d^3 y \pi^\dagger(\By) \antisymmetric{\pi(\By)}{\phi(\Bx)}
= \int d^3 y \pi^\dagger(\By) \delta^3(\Bx - \By)
= \pi^\dagger(\Bx),
\end{dmath}
which differs from \cref{eqn:ProblemSet1Problem2:80}.  However, should we compute the time evolution of the momentum operators,
this conjugation difference is ``cancelled'' and we end up with the Klein-Gordon equations in the end.
\begin{dmath}\label{eqn:ProblemSet1Problem2:180}
\ddt{\pi(\Bx)}
= i \antisymmetric{H(\By)}{\pi(\Bx)}
=
i \int d^3 y \antisymmetric{
\spacegrad_\By \phi^\dagger(\By) \cdot \spacegrad_\By \phi(\By)
}{ \pi(\Bx) }
+
i m^2 \int d^3 y \antisymmetric{ \phi^\dagger(\By) \phi(\By) }{ \pi(\Bx) }.
\end{dmath}
The second integral is easy
\begin{dmath}\label{eqn:ProblemSet1Problem2:200}
\int d^3 y \antisymmetric{ \phi^\dagger(\By) \phi(\By) }{ \pi(\Bx) }
=
\int d^3 y\,
\phi(\By)
\antisymmetric{
\phi(\By) }{ \pi(\Bx) }
=
i
\int d^3 y\,
\phi(\By)
\delta^3(\Bx - \By)
=
i
\phi(\Bx).
\end{dmath}
To evaluate the first integral in \cref{eqn:ProblemSet1Problem2:180} we can make use of the linearity to find
\begin{dmath}\label{eqn:ProblemSet1Problem2:220}
i \int d^3 y \antisymmetric{ \spacegrad_\By \phi^\dagger(\By) \cdot \spacegrad_\By \phi(\By) }{ \pi(\Bx) }
=
i \int d^3 y \, \spacegrad_\By \antisymmetric{\phi^\dagger(\By)}{\pi(\Bx)} \cdot \spacegrad_\By \phi(\By)
=
- \int d^3 y \, \spacegrad_\By \delta(\Bx - \By) \cdot \spacegrad_\By \phi(\By)
=
- \int d^3 y \lr{
\spacegrad_\By \cdot \lr{ \delta(\Bx - \By) \spacegrad_\By \phi(\By) }
-
\delta(\Bx - \By) \spacegrad_\By^2 \phi(\By)
}
=
- \int dA_y
\delta(\Bx - \By) \PD{n}{\phi(\By)}
+
\spacegrad^2 \phi(\Bx).
\end{dmath}
Provided we have some justification for declaring the first integral zero (I'm not sure what that was in class, as the delta function isn't really well behaved unless integrated over a volume), we are left with
\begin{dmath}\label{eqn:ProblemSet1Problem2:260}
\ddt{\pi(\Bx)}
=
\spacegrad^2 \phi(\Bx) - m^2 \phi(\Bx),
\end{dmath}
or
\begin{dmath}\label{eqn:ProblemSet1Problem2:280}
\frac{d^2 \phi}{dt^2}
=
\spacegrad^2 \phi - m^2 \phi.
\end{dmath}
which is a KG equation for \( \phi\).
As the Hamiltonian is symmetric in \( \pi, \pi^\dagger \) and \( \phi, \phi^\dagger \) repeating this calculation for \( \dot{\pi}^\dagger \) gives
\begin{dmath}\label{eqn:ProblemSet1Problem2:300}
\ddt{\pi^\dagger(\Bx)}
=
\spacegrad^2 \phi^\dagger(\Bx) - m^2 \phi^\dagger(\Bx),
\end{dmath}
which is also a KG equation for the quantized field \( \phi \).
%%%\begin{dmath}\label{eqn:ProblemSet1Problem2:320}
%%%\frac{d^2 \phi^\dagger}{dt^2}
%%%=
%%%\spacegrad^2 \phi^\dagger - m^2 \phi^\dagger,
%%%\end{dmath}
%%%as expected.

Unlike the single particle scalar theory that we examined in class, we won't make any a-priori assumption that the quantized field operator \( \phi \) is Hermitian, but seek a (momentum-space) representation that is slightly more general
\begin{dmath}\label{eqn:ProblemSet1Problem2:360}
\begin{aligned}
\tilde{\phi}(\Bp, t) &= \inv{\sqrt{2 \omega_\Bp}} \lr{ e^{-i \omega_\Bp t} a_\Bp + e^{i \omega_\Bp t} b^\dagger_\Bp } \\
\tilde{\pi}(\Bp, t) &= \frac{i\omega_\Bp}{\sqrt{2 \omega_\Bp}} \lr{ -e^{-i \omega_\Bp t} a_\Bp + e^{i \omega_\Bp t} b^\dagger_\Bp },
\end{aligned}
\end{dmath}
where \( \omega_\Bp^2 = \Bp^2 + m^2 \).  These equations are invertible for \( a_\Bp, b^\dagger_\Bp \)
\begin{dmath}\label{eqn:ProblemSet1Problem2:380}
\begin{aligned}
a_\Bp &= \sqrt{\frac{\omega_\Bp}{2}} e^{i\omega_\Bp t} \lr{ \tilde{\phi}(\Bp, t) - \inv{i\omega_\Bp} \tilde{\pi}(\Bp, t) } \\
b_\Bp^\dagger &= \sqrt{\frac{\omega_\Bp}{2}} e^{-i\omega_\Bp t} \lr{ \tilde{\phi}(\Bp, t) + \inv{i\omega_\Bp} \tilde{\pi}(\Bp, t) },
\end{aligned}
\end{dmath}
or in terms of a spatial representation
\begin{dmath}\label{eqn:ProblemSet1Problem2:440}
\begin{aligned}
a_\Bp &= \int d^3 x e^{-i \Bp \cdot \Bx} \sqrt{\frac{\omega_\Bp}{2}} e^{i\omega_\Bp t} \lr{ \phi(\Bx, t) - \inv{i\omega_\Bp} \pi(\Bx, t) } \\
a_\Bq^\dagger &= \int d^3 y e^{i \Bq \cdot \By} \sqrt{\frac{\omega_\Bq}{2}} e^{-i\omega_\Bq t} \lr{ \phi^\dagger(\By, t) + \inv{i\omega_\Bq} \pi^\dagger(\By, t) } \\
b_\Bp &= \int d^3 x e^{i \Bp \cdot \Bx} \sqrt{\frac{\omega_\Bp}{2}} e^{i\omega_\Bp t} \lr{ \phi^\dagger(\Bx, t) - \inv{i\omega_\Bp} \pi^\dagger(\Bx, t) } \\
b_\Bq^\dagger &= \int d^3 y e^{-i \Bq \cdot \By} \sqrt{\frac{\omega_\Bq}{2}} e^{-i\omega_\Bq t} \lr{ \phi(\By, t) + \inv{i\omega_\Bq} \pi(\By, t) }.
\end{aligned}
\end{dmath}
We seek the commutators of all the \cref{eqn:ProblemSet1Problem2:440}
Fourier coefficient operators.  Since conjugate pairs \( \phi, \pi^\dagger \) or \( \phi^\dagger, \pi \) are required for any non-zero commutator, we have by inspection \(
\antisymmetric{a_\Bp}{b_\Bq^\dagger}
=
\antisymmetric{a_\Bp^\dagger}{b_\Bq}
= 0 \).
For the rest some expansion is required
\begin{dmath}\label{eqn:ProblemSet1Problem2:460}
\antisymmetric{a_\Bp}{b_\Bq}
=
\int d^3 x d^3 y e^{-i \Bp \cdot \Bx} e^{i \Bq \cdot \By}
\sqrt{\frac{\omega_\Bp}{2}} \sqrt{\frac{\omega_\Bq}{2}}
e^{i \omega_\Bp t + i \omega_\Bq t}
\antisymmetric{  \phi(\Bx, t) - \inv{i\omega_\Bp}  \pi(\Bx, t) }
{  \phi^\dagger(\By, t) - \inv{i\omega_\Bq}  \pi^\dagger(\By, t) }
=
\int d^3 x d^3 y e^{-i \Bp \cdot \Bx} e^{i \Bq \cdot \By}
\sqrt{\frac{\omega_\Bp}{2}} \sqrt{\frac{\omega_\Bq}{2}}
e^{i \omega_\Bp t + i \omega_\Bq t}
\lr{
   -\inv{i \omega_\Bq} i \delta^3(\Bx - \By)
   -
   \inv{i \omega_\Bp} (-i) \delta^3(\Bx - \By)
}
=
\int d^3 x e^{i (\Bq -\Bp) \cdot \Bx}
\sqrt{\frac{\omega_\Bp}{2}} \sqrt{\frac{\omega_\Bq}{2}}
e^{i \omega_\Bp t + i \omega_\Bq t}
\lr{
   -\inv{\omega_\Bq}
   +
   \inv{\omega_\Bp}
}
=
(2\pi)^3 \delta^3(\Bq - \Bp)
e^{i \omega_\Bp t + i \omega_\Bq t}
\sqrt{\frac{\omega_\Bp}{2}} \sqrt{\frac{\omega_\Bq}{2}}
\lr{
   -\inv{\omega_\Bq}
   +
   \inv{\omega_\Bp}
}
=
\inv{2} (2\pi)^3 \delta^3(\Bq - \Bp)
e^{2 i \omega_\Bp t }
\times 0
= 0,
\end{dmath}
so the commutators \( \antisymmetric{a_\Bp}{b_\Bq} = \antisymmetric{a_\Bp^\dagger}{b_\Bq^\dagger} = 0 \) as well.  Finally,
\begin{dmath}\label{eqn:ProblemSet1Problem2:480}
\antisymmetric{a_\Bp}{a_\Bq^\dagger}
=
\int d^3 x d^3 y e^{-i \Bp \cdot \Bx} e^{i \Bq \cdot \By}
\sqrt{\frac{\omega_\Bp}{2}} \sqrt{\frac{\omega_\Bq}{2}}
e^{i \omega_\Bp t - i \omega_\Bq t}
\antisymmetric
{  \phi(\Bx, t) - \inv{i\omega_\Bp} \pi(\Bx, t) }
{  \phi^\dagger(\Bx, t) + \inv{i\omega_\Bq} \pi^\dagger(\Bx, t) }
=
\int d^3 x d^3 y e^{-i \Bp \cdot \Bx} e^{i \Bq \cdot \By}
\sqrt{\frac{\omega_\Bp}{2}} \sqrt{\frac{\omega_\Bq}{2}}
e^{i \omega_\Bp t - i \omega_\Bq t}
\lr{
   \inv{i\omega_\Bq} i \delta^3(\Bx - \By)
-
   \inv{i\omega_\Bp} (-i) \delta^3(\Bx - \By)
}
=
\int d^3 x e^{i (\Bq - \Bp) \cdot \Bx}
\sqrt{\frac{\omega_\Bp}{2}} \sqrt{\frac{\omega_\Bq}{2}}
e^{i \omega_\Bp t - i \omega_\Bq t}
\lr{
   \inv{\omega_\Bq} + \inv{\omega_\Bp}
}
=
(2\pi)^3 \delta^3(\Bq - \Bp),
\end{dmath}
and
\begin{dmath}\label{eqn:ProblemSet1Problem2:500}
\antisymmetric{b_\Bp}{b_\Bq^\dagger}
=
\int d^3 x d^3 y e^{i \Bp \cdot \Bx} e^{-i \Bq \cdot \By}
\sqrt{\frac{\omega_\Bp}{2}} \sqrt{\frac{\omega_\Bq}{2}}
e^{i \omega_\Bp t - i \omega_\Bq t}
\antisymmetric
{  \phi^\dagger(\Bx, t) - \inv{i\omega_\Bp} \pi^\dagger(\Bx, t) }
{  \phi(\Bx, t) + \inv{i\omega_\Bq} \pi(\Bx, t) }
=
\int d^3 x d^3 y e^{i \Bp \cdot \Bx} e^{-i \Bq \cdot \By}
\sqrt{\frac{\omega_\Bp}{2}} \sqrt{\frac{\omega_\Bq}{2}}
e^{i \omega_\Bp t - i \omega_\Bq t}
\lr{
   \inv{i\omega_\Bq} i \delta^3(\Bx - \By)
-
   \inv{i\omega_\Bp} (-i) \delta^3(\Bx - \By)
}
=
\int d^3 x e^{i (\Bp - \Bq) \cdot \Bx}
\sqrt{\frac{\omega_\Bp}{2}} \sqrt{\frac{\omega_\Bq}{2}}
e^{i \omega_\Bp t - i \omega_\Bq t}
\lr{
   \inv{\omega_\Bq} + \inv{\omega_\Bp}
}
=
(2\pi)^3 \delta^3(\Bp - \Bq),
\end{dmath}
so the only non-zero commutators are \(
\antisymmetric{a_\Bp}{a_\Bq^\dagger} =
\antisymmetric{b_\Bp}{b_\Bq^\dagger} =
-\antisymmetric{a_\Bp^\dagger}{a_\Bq} =
-\antisymmetric{b_\Bp^\dagger}{b_\Bq} = (2 \pi)^3 \delta^3(\Bp - \Bq)
\).
Except for a scale factor, we have commutators associated with a pair of independent creation and annihilation operators.  Let's compute the Hamiltonian representation next to verify that it diagonalizes nicely with this representation.  We need the spatial representation of our operators
\begin{dmath}\label{eqn:ProblemSet1Problem2:520}
\begin{aligned}
\phi(\Bx, t) &=
\int \frac{d^3 p}{(2 \pi)^3} e^{i \Bp \cdot \Bx}
\inv{\sqrt{2 \omega_\Bp}} \lr{ e^{-i \omega_\Bp t} a_\Bp + e^{i \omega_\Bp t} b^\dagger_\Bp } \\
\spacegrad \phi(\Bx, t) &=
\int \frac{d^3 p}{(2 \pi)^3} e^{i \Bp \cdot \Bx}
\frac{i \Bp}{\sqrt{2 \omega_\Bp}} \lr{ e^{-i \omega_\Bp t} a_\Bp + e^{i \omega_\Bp t} b^\dagger_\Bp } \\
\pi(\Bx, t) &=
\int \frac{d^3 p}{(2 \pi)^3} e^{i \Bp \cdot \Bx}
\frac{i \omega_\Bp}{\sqrt{2 \omega_\Bp}} \lr{ -e^{-i \omega_\Bp t} a_\Bp + e^{i \omega_\Bp t} b^\dagger_\Bp } \\
\end{aligned}
\end{dmath}
so the Hamiltonian is
\begin{dmath}\label{eqn:ProblemSet1Problem2:540}
H = \int d^3 x \lr{
\pi^\dagger(\Bx, t)
\pi(\Bx, t)
+
\spacegrad \phi^\dagger(\Bx, t) \cdot
\spacegrad \phi(\Bx, t)
+
m^2
\phi^\dagger(\Bx, t)
\phi(\Bx, t)
}
=
\int d^3 x
\frac{d^3 p}{(2 \pi)^3}
\frac{d^3 q}{(2 \pi)^3}
e^{i (\Bq-\Bp) \cdot \Bx}
\inv{\sqrt{2 \omega_\Bp}}
\inv{\sqrt{2 \omega_\Bq}}
\lr{
   \omega_\Bp
   \omega_\Bq
   \lr{ -e^{i \omega_\Bp t} a_\Bp^\dagger + e^{-i \omega_\Bp t} b_\Bp }
   \lr{ -e^{-i \omega_\Bq t} a_\Bq + e^{i \omega_\Bq t} b^\dagger_\Bq }
+
   \lr{ \Bp \cdot \Bq + m^2 }
   \lr{ e^{i \omega_\Bp t} a_\Bp^\dagger + e^{-i \omega_\Bp t} b_\Bp }
   \lr{ e^{-i \omega_\Bq t} a_\Bq + e^{i \omega_\Bq t} b^\dagger_\Bq }
}
=
\inv{2}
\int
\frac{d^3 p}{(2 \pi)^3}
\omega_\Bp
\lr{
   \lr{ -e^{i \omega_\Bp t} a_\Bp^\dagger + e^{-i \omega_\Bp t} b_\Bp }
   \lr{ -e^{-i \omega_\Bp t} a_\Bp + e^{i \omega_\Bp t} b^\dagger_\Bp }
+
   \lr{ e^{i \omega_\Bp t} a_\Bp^\dagger + e^{-i \omega_\Bp t} b_\Bp }
   \lr{ e^{-i \omega_\Bp t} a_\Bp + e^{i \omega_\Bp t} b^\dagger_\Bp }
}
=
\inv{2}
\int
\frac{d^3 p}{(2 \pi)^3}
\omega_\Bp
\lr{
   e^{2 i \omega_\Bp t} \lr{ -b_\Bp a_\Bp + b_\Bp a_\Bp }
+
   e^{-2 i \omega_\Bp t} \lr{ -a_\Bp^\dagger b_\Bp^\dagger + a_\Bp^\dagger b_\Bp^\dagger }
+
   2 a_\Bp^\dagger a_\Bp + 2 b_\Bp b_\Bp^\dagger +
},
\end{dmath}
so our diagonalized Hamiltonian is just
\boxedEquation{eqn:ProblemSet1Problem2:560}{
H
=
\int
\frac{d^3 p}{(2 \pi)^3}
\omega_\Bp
\lr{
   a_\Bp^\dagger a_\Bp + b_\Bp b_\Bp^\dagger
}.
}

\makeSubAnswer{}{qft:problemSet1:2c}

Inserting
\cref{eqn:ProblemSet1Problem2:520} into
\cref{eqn:ProblemSet1Problem2:580} (which has been corrected from the original problem statement), we find

\begin{dmath}\label{eqn:ProblemSet1Problem2:600}
Q
= \frac{i}{2} \int d^3 x \lr{ \phi^\dagger \pi - \pi^\dagger \phi }
= \frac{i}{4} \int d^3 x
\frac{d^3 p}{(2\pi)^3}
\frac{d^3 q}{(2\pi)^3}
\frac{ e^{i (\Bq-\Bp) \cdot \Bx} }{\sqrt{\omega_\Bp \omega_\Bq}}
\lr{
   i \omega_\Bq
   \lr{ e^{i \omega_\Bp t} a_\Bp^\dagger + e^{-i \omega_\Bp t} b_\Bp }
   \lr{ -e^{-i \omega_\Bq t} a_\Bq + e^{i \omega_\Bq t} b^\dagger_\Bq }
      -
   (-i) \omega_\Bp
   \lr{ -e^{i \omega_\Bp t} a_\Bp^\dagger + e^{-i \omega_\Bp t} b_\Bp }
   \lr{ e^{-i \omega_\Bq t} a_\Bq + e^{i \omega_\Bq t} b^\dagger_\Bq }
}
= -\frac{1}{4} \int
\frac{d^3 p}{(2\pi)^3}
\lr{
   \lr{ e^{i \omega_\Bp t} a_\Bp^\dagger + e^{-i \omega_\Bp t} b_\Bp }
   \lr{ -e^{-i \omega_\Bp t} a_\Bp + e^{i \omega_\Bp t} b^\dagger_\Bp }
   +
   \lr{ -e^{i \omega_\Bp t} a_\Bp^\dagger + e^{-i \omega_\Bp t} b_\Bp }
   \lr{ e^{-i \omega_\Bp t} a_\Bp + e^{i \omega_\Bp t} b^\dagger_\Bp }
}
=
-\frac{1}{4} \int
\frac{d^3 p}{(2\pi)^3}
\lr{
   e^{2 i \omega_\Bp t} \lr{ a_\Bp^\dagger b_\Bp^\dagger - a_\Bp^\dagger b_\Bp^\dagger }
+
   e^{-2 i \omega_\Bp t} \lr{ -b_\Bp a_\Bp + b_\Bp a_\Bp }
- a_\Bp^\dagger a_\Bp + b_\Bp b_\Bp^\dagger - a_\Bp^\dagger a_\Bp + b_\Bp b_\Bp^\dagger
}
=
\frac{1}{2} \int
\frac{d^3 p}{(2\pi)^3}
\lr{
a_\Bp^\dagger a_\Bp - b_\Bp b_\Bp^\dagger
},
\end{dmath}
or
\boxedEquation{eqn:ProblemSet1Problem2:620}{
Q :=:
\frac{1}{2} \int
\frac{d^3 p}{(2\pi)^3}
\lr{
a_\Bp^\dagger a_\Bp - b_\Bp^\dagger b_\Bp
}.
}
Let's get a feel for the charge by applying it to the states corresponding to each creation operator
\begin{dmath}\label{eqn:ProblemSet1Problem2:640}
\begin{aligned}
\ket{\Bk}_a &= a^\dagger_\Bk \ket{0} \\
\ket{\Bk}_b &= b^\dagger_\Bk \ket{0}.
\end{aligned}
\end{dmath}

\begin{dmath}\label{eqn:ProblemSet1Problem2:660}
Q \ket{\Bk}_a =
\frac{1}{2} \int
\frac{d^3 p}{(2\pi)^3}
\lr{
a_\Bp^\dagger a_\Bp - b_\Bp^\dagger b_\Bp
}
a^\dagger_\Bk \ket{0}
=
\frac{1}{2} \int
\frac{d^3 p}{(2\pi)^3}
a_\Bp^\dagger a_\Bp
a^\dagger_\Bk \ket{0}
=
\frac{1}{2} \int
\frac{d^3 p}{(2\pi)^3}
a_\Bp^\dagger
\lr{
a^\dagger_\Bk
a_\Bp
+ (2\pi)^3 \delta^3(\Bk - \Bp)
}
\ket{0}
=
\inv{2}
a_\Bk^\dagger \ket{0}
=
\inv{2}
\ket{\Bk}_a.
\end{dmath}
and
\begin{dmath}\label{eqn:ProblemSet1Problem2:680}
Q \ket{\Bk}_b =
\frac{1}{2} \int
\frac{d^3 p}{(2\pi)^3}
\lr{
a_\Bp^\dagger a_\Bp - b_\Bp^\dagger b_\Bp
}
b^\dagger_\Bk \ket{0}
=
-\frac{1}{2} \int
\frac{d^3 p}{(2\pi)^3}
b_\Bp^\dagger b_\Bp
b^\dagger_\Bk \ket{0}
=
-\frac{1}{2} \int
\frac{d^3 p}{(2\pi)^3}
b_\Bp^\dagger
\lr{
b^\dagger_\Bk
b_\Bp
+ (2\pi)^3 \delta^3(\Bk - \Bp)
}
\ket{0}
=
-\inv{2}
b_\Bk^\dagger \ket{0}
=
-\inv{2}
\ket{\Bk}_b.
\end{dmath}
So, we could say that the particles associated with creation operator \( a_\Bp^\dagger \) have a (1/2) charge and
particles associated with creation operator \( b_\Bp^\dagger \) have a (-1/2) charge.

}
%}
