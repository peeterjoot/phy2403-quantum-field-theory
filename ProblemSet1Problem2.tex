%
% Copyright � 2018 Peeter Joot.  All Rights Reserved.
% Licenced as described in the file LICENSE under the root directory of this GIT repository.
%
\makeproblem{Part of Problem 2.2 from Peskin and Schroeder (reproduced below).}{qft:problemSet1:2}{
Consider a complex scalar field with action \( S = \int d^4x\lr{\partial_\mu \phi^\conj \partial^\mu \phi - m^2 \phi^\conj \phi}\).  When doing the variational principle consider \( \phi \) and \(\phi^\conj \) as independent, rather than their real and imaginary parts (this is equivalent, but more convenient).

\makesubproblem{}{qft:problemSet1:2a}
Show that \( H = \int d^3x \lr{ \pi^\conj \pi + \spacegrad \phi^\conj \cdot \spacegrad \phi + m^2 \phi^\conj \phi } \) and that the Klein-Gordon equation is obeyed by \( \phi \) and \( \phi^\conj\).
\makesubproblem{}{qft:problemSet1:2b}
Introduce complex amplitudes, diagonalize the Hamiltonian, and quantize the theory. Show that the theory has now two sets of particles.
\makesubproblem{}{qft:problemSet1:2c}
Write the charge conserved due to the global \( U(1) \) symmetry, \( Q = \int d^3 x \frac{i}{2} \lr{ \phi^\conj \pi^\conj - \pi \phi } \),
in
terms of creation and annihilation operators and find the charge of the particles of each type.
} % makeproblem

\makeanswer{qft:problemSet1:2}{
\makeSubAnswer{}{qft:problemSet1:2a}
Classically, evaluating the Euler-Lagrange equations gives us
\begin{dmath}\label{eqn:ProblemSet1Problem2:20}
\begin{aligned}
\PD{\phi}{\LL} &= -m^2 \phi^\conj \\
\PD{(\partial_\mu \phi)}{\LL} &= \partial^\mu \phi^\conj \\
\PD{\phi^\conj}{\LL} &= -m^2 \phi       \\
\PD{(\partial_\mu \phi^\conj)}{\LL} &= \partial^\mu \phi,
\end{aligned}
\end{dmath}
so the equations of the field are respectively
\begin{dmath}\label{eqn:ProblemSet1Problem2:40}
\begin{aligned}
\partial_\mu \partial^\mu \phi^\conj &=  -m^2 \phi^\conj \\
\partial_\mu \partial^\mu \phi &=  -m^2 \phi,
\end{aligned}
\end{dmath}
so we have a Klein-Gordon equation for each field variable \( \phi, \phi^\conj \)
as expected.
To find the Hamiltonian, note that the Lagrangian density written out explicitly is
\begin{dmath}\label{eqn:ProblemSet1Problem2:60}
\LL = \partial_0 \phi^\conj \partial_0 \phi - (\spacegrad \phi^\conj) \cdot (\spacegrad \phi) - m^2 \phi^\conj \phi,
\end{dmath}
so the conjugate momentum densities are
\begin{dmath}\label{eqn:ProblemSet1Problem2:80}
\begin{aligned}
\pi(\Bx, t) &= \PD{(\partial_0 \phi)}{\LL} = \partial_0 \phi^\conj \\
\pi^\conj(\Bx, t) &= \PD{(\partial_0 \phi^\conj)}{\LL} = \partial_0 \phi \\
\end{aligned}
\end{dmath}

The Hamiltonian (including a \( p \dot{q} \) term for each of \( \phi, \phi^\conj \)) is
\begin{dmath}\label{eqn:ProblemSet1Problem2:100}
H
= \int d^3 x \lr{ \pi \partial_0 \phi + \pi^\conj \partial_0 \phi^\conj - \LL }
=
\int d^3 x \lr{ \pi \pi^\conj + \pi^\conj \pi - \pi \pi^\conj +
(\spacegrad \phi^\conj) \cdot (\spacegrad \phi) + m^2 \phi^\conj \phi
 }
=
\int d^3 x \lr{ \pi^\conj \pi +
(\spacegrad \phi^\conj) \cdot (\spacegrad \phi) + m^2 \phi^\conj \phi
 }
\end{dmath}
\makeSubAnswer{}{qft:problemSet1:2b}
If we choose to canonically quantize the fields, we promote the fields to operators, and demand that
\begin{dmath}\label{eqn:ProblemSet1Problem2:120}
\begin{aligned}
\antisymmetric{\phi(\Bx)}{\pi(\By)} &= i \delta^3(\Bx - \By) \\
\antisymmetric{\phi^\conj(\Bx)}{\pi^\conj(\By)} &= i \delta^3(\Bx - \By),
\end{aligned}
\end{dmath}
also requiring \( \pi \pi^\conj, \phi^\conj \pi, \phi \pi^\conj \) to all commute.
It's easy enough to verify that the time evolution of such quantized \( \phi, \phi^\conj \) field operators matches the classical equations
\begin{dmath}\label{eqn:ProblemSet1Problem2:140}
-i \ddt{\phi(\Bx)}
= \antisymmetric{H(\By)}{\phi(\Bx)}
= \int d^3 y \antisymmetric{\pi^\conj(\By) \pi(\By)}{\phi(\Bx)}
= \int d^3 y \pi^\conj(\By) \antisymmetric{\pi(\By)}{\phi(\Bx)}
= -i \int d^3 y \pi^\conj(\By) \delta^3(\Bx - \By)
= -i \pi^\conj(\Bx),
\end{dmath}
\begin{dmath}\label{eqn:ProblemSet1Problem2:160}
-i \ddt{\phi^\conj(\Bx)}
= \antisymmetric{H(\By)}{\phi^\conj(\Bx)}
= \int d^3 y \antisymmetric{\pi^\conj(\By) \pi(\By)}{\phi^\conj(\Bx)}
= \int d^3 y \pi(\By) \antisymmetric{\pi^\conj(\By)}{\phi(\Bx)}
= -i \int d^3 y \pi(\By) \delta^3(\Bx - \By)
= -i \pi(\Bx),
\end{dmath}
which matches \cref{eqn:ProblemSet1Problem2:80} as expected.
We should also find the Klein-Gordon equations by computing the time evolution of the quantized \( \pi, \pi^\conj \) operators
\begin{dmath}\label{eqn:ProblemSet1Problem2:180}
-i \ddt{\pi(\Bx)}
= \antisymmetric{H(\By)}{\pi(\Bx)}
=
\int d^3 y \antisymmetric{
\spacegrad_\By \phi^\conj(\By) \cdot \spacegrad_\By \phi(\By)
}{ \pi(\Bx) }
+
m^2 \int d^3 y \antisymmetric{ \phi^\conj(\By) \phi(\By) }{ \pi(\Bx) }.
\end{dmath}
The second integral is easy
\begin{dmath}\label{eqn:ProblemSet1Problem2:200}
\int d^3 y \antisymmetric{ \phi^\conj(\By) \phi(\By) }{ \pi(\Bx) }
=
\int d^3 y
\phi^\conj(\By)
\antisymmetric{
\phi(\By) }{ \pi(\Bx) }
=
i
\int d^3 y
\phi^\conj(\By)
\delta^3(\Bx - \By)
=
i
\phi^\conj(\Bx).
\end{dmath}
The first integral in \cref{eqn:ProblemSet1Problem2:180}
takes a bit more work.  One of the integrals in that commutator is
\begin{dmath}\label{eqn:ProblemSet1Problem2:220}
\int d^3 y
\lr{ \spacegrad_\By \phi^\conj(\By) \cdot \spacegrad_\By \phi(\By) }
 \pi(\Bx)
=
\int d^3 y
\spacegrad_\By \phi^\conj(\By) \cdot \spacegrad_\By \lr{ \phi(\By) \pi(\Bx) }
=
\int d^3 y
\spacegrad_\By \phi^\conj(\By) \cdot \spacegrad_\By \lr{
\pi(\Bx)
\phi(\By)
+
i \delta^3(\Bx - \By)
}
=
\int d^3 y
\pi(\Bx) \lr{
\spacegrad_\By \phi^\conj(\By) \cdot \spacegrad_\By \phi(\By)
}
+
i \int d^3 y
\spacegrad_\By \phi^\conj(\By) \cdot \spacegrad_\By \delta^3(\Bx - \By).
\end{dmath}
This means that we have
\begin{dmath}\label{eqn:ProblemSet1Problem2:240}
\int d^3 y \antisymmetric{
\spacegrad_\By \phi^\conj(\By) \cdot \spacegrad_\By \phi(\By)
}{ \pi(\Bx) }
=
i \int d^3 y
\spacegrad_\By \phi^\conj(\By) \cdot \spacegrad_\By \delta^3(\Bx - \By)
=
i \int d^3 y
\spacegrad_\By \cdot \lr{
\delta^3(\Bx - \By)
\spacegrad_\By \phi^\conj(\By)
}
=
i \int d^3 y
\spacegrad_\By \cdot \lr{
   \delta^3(\Bx - \By)
   \spacegrad_\By \phi^\conj(\By)
}
-
i \int d^3 y
   \delta^3(\Bx - \By)
   \spacegrad_\By^2 \phi^\conj(\By)
=
-i \spacegrad^2 \phi^\conj(\Bx),
\end{dmath}
so
\begin{dmath}\label{eqn:ProblemSet1Problem2:260}
\ddt{\pi(\Bx)}
=
\spacegrad^2 \phi^\conj(\Bx) - m^2 \phi^\conj(\Bx),
\end{dmath}
or
\begin{dmath}\label{eqn:ProblemSet1Problem2:280}
\frac{d^2 \phi^\conj}{dt^2}
=
\spacegrad^2 \phi^\conj - m^2 \phi^\conj.
\end{dmath}
which is a KG equation for \( \phi^\conj \).
As the Hamiltonian is symmetric in \( \pi, \pi^\conj \) and \( \phi, \phi^\conj \) repeating this calculation for \( \dot{\pi}^\conj \) is
\begin{dmath}\label{eqn:ProblemSet1Problem2:300}
\ddt{\pi^\conj(\Bx)}
=
\spacegrad^2 \phi(\Bx) - m^2 \phi(\Bx),
\end{dmath}
which is also a KG equation for the quantized field \( \phi \)
\begin{dmath}\label{eqn:ProblemSet1Problem2:320}
\frac{d^2 \phi}{dt^2}
=
\spacegrad^2 \phi - m^2 \phi,
\end{dmath}
as expected.
\paragraph{Question:} what is meant here by ``Diagonalizing the Hamiltonian''?

\makeSubAnswer{}{qft:problemSet1:2c}

TODO.
}
