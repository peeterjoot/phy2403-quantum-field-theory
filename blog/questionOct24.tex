%
% Copyright � 2018 Peeter Joot.  All Rights Reserved.
% Licenced as described in the file LICENSE under the root directory of this GIT repository.
%
%{
\input{../latex/blogpost.tex}
\renewcommand{\basename}{questionOct24}
%\renewcommand{\dirname}{notes/phy1520/}
\renewcommand{\dirname}{notes/ece1228-electromagnetic-theory/}
%\newcommand{\dateintitle}{}
%\newcommand{\keywords}{}

\input{../latex/peeter_prologue_print2.tex}

\usepackage{peeters_layout_exercise}
\usepackage{peeters_braket}
\usepackage{peeters_figures}
\usepackage{siunitx}
\usepackage{verbatim}
%\usepackage{mhchem} % \ce{}
%\usepackage{macros_bm} % \bcM
%\usepackage{macros_qed} % \qedmarker
%\usepackage{txfonts} % \ointclockwise

\beginArtNoToc

\generatetitle{XXX}
%\chapter{XXX}
%\label{chap:questionOct24}
% \citep{sakurai2014modern} pr X.Y
% \citep{pozar2009microwave}
% \citep{qftLectureNotes}
% \citep{doran2003gap}
% \citep{jackson1975cew}
% \citep{griffiths1999introduction}

It was left as an exersize to show that given
\begin{equation}\label{eqn:questionOct24:120}
H = \int d^3 p \lr{ \inv{2} \pi^2 + \inv{2} \lr{ \spacegrad \phi}^2 + \frac{m^2}{2} \phi^2 },
\end{equation}
we obtain
\begin{equation}\label{eqn:questionOct24:140}
H_{\text{after}} =
\int d^3 x \omega_\Bp
\lr{ a_\Bp^\dagger - i \frac{\tilde{j}^\conj(p)}{\sqrt{2 \omega_\Bp}} }
\lr{ a_\Bp + i \frac{\tilde{j}(p)}{\sqrt{2 \omega_\Bp}} },
\end{equation}
If I do this problem, I get something close to this
\begin{equation}\label{eqn:questionOct24:160}
H_{\text{after}} =
\inv{2}
\int d^3 x \omega_\Bp
\lr{
\lr{ a_\Bp^\dagger - i \frac{\tilde{j}^\conj(p)}{\sqrt{2 \omega_\Bp}} }
\lr{ a_\Bp + i \frac{\tilde{j}(p)}{\sqrt{2 \omega_\Bp}} },
+
\lr{ a_\Bp + i \frac{\tilde{j}(p)}{\sqrt{2 \omega_\Bp}} },
\lr{ a_\Bp^\dagger - i \frac{\tilde{j}^\conj(p)}{\sqrt{2 \omega_\Bp}} }
}
\end{equation}
The operator products in the integrand has a structure like \( \hat{Z}^\dagger \hat{Z} + \hat{Z} \hat{Z}^\dagger \), which is almost like a \( 2 \times \text{Real} \) operation.  Assuming my calculation above is correct, what reason do we have to drop the second pair of operator products in the integrand?

%}
\EndArticle
%\EndNoBibArticle
