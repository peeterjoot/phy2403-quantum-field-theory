%
% Copyright � 2018 Peeter Joot.  All Rights Reserved.
% Licenced as described in the file LICENSE under the root directory of this GIT repository.
%
%{
\input{../latex/blogpost.tex}
\renewcommand{\basename}{squarerootpsigma}
%\renewcommand{\dirname}{notes/phy1520/}
\renewcommand{\dirname}{notes/ece1228-electromagnetic-theory/}
%\newcommand{\dateintitle}{}
%\newcommand{\keywords}{}

\input{../latex/peeter_prologue_print2.tex}

\usepackage{peeters_layout_exercise}
\usepackage{peeters_braket}
\usepackage{peeters_figures}
\usepackage{siunitx}
\usepackage{verbatim}
%\usepackage{mhchem} % \ce{}
%\usepackage{macros_bm} % \bcM
%\usepackage{macros_qed} % \qedmarker
%\usepackage{txfonts} % \ointclockwise

\newcommand{\osigma}[0]{\overbar{\sigma}}

\beginArtNoToc

%\generatetitle{Explicit form of the square roots of the p.sigma matrices}
\generatetitle{Explicit form of the square roots of the \( p \cdot \sigma \) matrices}
%\chapter{Explicit form of the square roots of the p.sigma matrices}
%\label{chap:squarerootpsigma}

With the help of Mathematica, a fairly compact form was found for the root of \( p \cdot \sigma \)
\begin{equation}\label{eqn:DiracUVmatricesExplicit:120}
\sqrt{ p \cdot \sigma }
=
\inv{
   \sqrt{ \omega_\Bp- \Norm{\Bp} } + \sqrt{ \omega_\Bp+ \Norm{\Bp} }
}
\begin{bmatrix}
\omega_\Bp- p^3 + \sqrt{ \omega_\Bp^2 - \Bp^2 } & - p^1 + i p^2 \\
- p^1 - i p^2 & \omega_\Bp+ p^3 + \sqrt{ \omega_\Bp^2 - \Bp^2 }
\end{bmatrix}.
\end{equation}
A bit of examination shows that we can do much better.  The leading scalar term can be simplified by squaring it
\begin{equation}\label{eqn:squarerootpsigma:140}
\begin{aligned}
\lr{ \sqrt{ \omega_\Bp- \Norm{\Bp} } + \sqrt{ \omega_\Bp+ \Norm{\Bp} } }^2
&=
\sqrt{ \omega_\Bp- \Norm{\Bp} } + \sqrt{ \omega_\Bp+ \Norm{\Bp} } + 2 \sqrt{ \omega_\Bp^2 - \Bp^2 }
\\&=
2 \omega_\Bp + 2 m,
\end{aligned}
\end{equation}
where the on-shell value of the energy \( \omega_\Bp^2 = m^2 + \Bp^2 \) has been inserted.  Using that again in the matrix, we have
\begin{equation}\label{eqn:squarerootpsigma:160}
\begin{aligned}
\sqrt{ p \cdot \sigma }
&=
\inv{\sqrt{ 2 \omega_\Bp + 2 m }}
\begin{bmatrix}
\omega_\Bp- p^3 + m & - p^1 + i p^2 \\
- p^1 - i p^2 & \omega_\Bp+ p^3 + m
\end{bmatrix}
\\&=
\inv{\sqrt{ 2 \omega_\Bp + 2 m }}
\lr{
   (\omega_\Bp + m) \sigma^0
   -p^1 \PauliX
   -p^2 \PauliY
   -p^3 \PauliZ
}
\\&=
\inv{\sqrt{ 2 \omega_\Bp + 2 m }}
\lr{
   (\omega_\Bp + m) \sigma^0
   -p^1 \sigma^1
   -p^2 \sigma^2
   -p^3 \sigma^3
}
\\&=
\inv{\sqrt{ 2 \omega_\Bp + 2 m }}
\lr{
   (\omega_\Bp + m) \sigma^0 - \Bsigma \cdot \Bp
}.
\end{aligned}
\end{equation}

We've now found a nice algebraic form for these matrix roots
%\begin{equation}\label{eqn:squarerootpsigma:180}
\boxedEquation{eqn:squarerootpsigma:180}{
\begin{aligned}
\sqrt{p \cdot \sigma} &= \inv{\sqrt{ 2 \omega_\Bp + 2 m }} \lr{ m + p \cdot \sigma } \\
\sqrt{p \cdot \osigma} &= \inv{\sqrt{ 2 \omega_\Bp + 2 m }} \lr{ m + p \cdot \osigma}.
\end{aligned}
\end{equation}
}

As a check, let's square one of these explicitly
\begin{equation}\label{eqn:squarerootpsigma:200}
\begin{aligned}
\lr{ \sqrt{p \cdot \sigma} }^2
&= \inv{2 \omega_\Bp + 2 m }
\lr{ m^2 + (p \cdot \sigma)^2 + 2 m (p \cdot \sigma) }
\\&= \inv{2 \omega_\Bp + 2 m }
\lr{ m^2 + (\omega_\Bp^2 - 2 \omega_\Bp \Bsigma \cdot \Bp + \Bp^2) + 2 m (p \cdot \sigma) }
\\&= \inv{2 \omega_\Bp + 2 m }
\lr{ 2 \omega_\Bp^2 - 2 \omega_\Bp \Bsigma \cdot \Bp + 2 m (\omega_\Bp - \Bsigma \cdot \Bp) }
\\&= \inv{2 \omega_\Bp + 2 m }
\lr{ 2 \omega_\Bp \lr{ \omega_\Bp + m }  - (2 \omega_\Bp + 2 m) \Bsigma \cdot \Bp }
\\&=
\omega_\Bp - \Bsigma \cdot \Bp
\\&=
p \cdot \sigma,
\end{aligned}
\end{equation}
which validates the result.

We can also put the spinor solutions \( u, v \) in a nice compact square-root-free format
\boxedEquation{eqn:squarerootpsigma:220}{
\begin{aligned}
u(p) &= \inv{\sqrt{ 2 m + 2 \omega_\Bp }}
\begin{bmatrix}
( m + p \cdot \sigma ) \zeta \\
( m + p \cdot \osigma ) \zeta \\
\end{bmatrix}
v(p) &=
\inv{\sqrt{ 2 m + 2 \omega_\Bp }}
\begin{bmatrix}
( m + p \cdot \sigma ) \eta \\
-( m + p \cdot \osigma ) \eta
\end{bmatrix}.
\end{aligned}
}
\Cref{eqn:squarerootpsigma:220} is probably a much nicer starting point for evaluating the various \( u, v, \ubar, \vbar \) product relationships.

We now also know that we can return to \cref{eqn:DiracUVmatricesExplicit:140} and put the explicit (root-free) representation of the \( u, v \) spinors in a slightly tidier form
\begin{equation}\label{eqn:squarerootpsigma:240}
\begin{aligned}
u^1(p) &=
\inv{\sqrt{ 2 m + 2 \omega_\Bp }}
\begin{bmatrix}
\omega_\Bp- p^3 + m \\
- p^1 - i p^2 \\
\omega_\Bp+ p^3 + m \\
 p^1 + i p^2 \\
\end{bmatrix} \\
u^2(p) &=
\inv{\sqrt{ 2 m + 2 \omega_\Bp }}
\begin{bmatrix}
 - p^1 + i p^2 \\
 \omega_\Bp+ p^3 + m \\
   p^1 - i p^2 \\
 \omega_\Bp- p^3 + m \\
\end{bmatrix} \\
v^1(p) &=
\inv{\sqrt{ 2 m + 2 \omega_\Bp }}
\begin{bmatrix}
\omega_\Bp- p^3 + m \\
- p^1 - i p^2 \\
-\omega_\Bp- p^3 + m \\
-p^1 - i p^2 \\
\end{bmatrix} \\
v^2(p) &=
\inv{\sqrt{ 2 m + 2 \omega_\Bp }}
\begin{bmatrix}
 - p^1 + i p^2 \\
 \omega_\Bp+ p^3 + m \\
  -p^1 + i p^2 \\
 -\omega_\Bp+ p^3 + m \\
\end{bmatrix}.
\end{aligned}
\end{equation}

%}
%\EndArticle
\EndNoBibArticle
