%
% Copyright � 2018 Peeter Joot.  All Rights Reserved.
% Licenced as described in the file LICENSE under the root directory of this GIT repository.
%
\makeproblem{ Scale invariance and conserved charge }{qft:problemSet1:4}{
Consider classical electrodynamics with the Lagrangian
\begin{dmath}\label{eqn:ProblemSet1Problem4:20}
S = \int d^4 x \lr{ -\inv{4} F_{\mu\nu} F^{\mu\nu} }.
\end{dmath}
Consider the following ``dilatation'' (or ``scale'') transformation:
\begin{dmath}\label{eqn:ProblemSet1Problem4:40}
\begin{aligned}
x_\mu &\rightarrow x'_\mu = e^d x_\mu \\
A_\mu(x) &\rightarrow A'_\mu(x') = e^{-d} A_\mu(x),
\end{aligned}
\end{dmath}
where \( d \) is a constant, called the dilatation parameter.

Dilatation invariance in QED (and QCD) is perhaps the simplest example of a symmetry, where the classical action is invariant, but the quantum theory is not (as you will learn later, in the spring class). Broken scale invariance arises because one has to introduce a short-distance cutoff (a UV ``regulator'') to define the quantum theory. (We already saw an indication of the need for a regulator when we considered the divergent zero point energy of the free quantum scalar field.)

\makesubproblem{}{qft:problemSet1:4a}
Show that the action is invariant under dilatations.
\makesubproblem{}{qft:problemSet1:4b}
Find the corresponding Noether current.
\makesubproblem{}{qft:problemSet1:4c}
Show that -- perhaps, after a redefinition of \( j_\mu \) ; notice that any conserved current \( j_\mu \) can be
redefined by adding to it \( \partial^\nu C_{\mu\nu} \), where \( C_{\mu\nu} \) is antisymmetric, without spoiling its conservation
(in this case \( C \) can depend on \( x^\mu, \partial^\mu \) and \( A^\mu \), of course) the dilatation current is simply related
to the energy-momentum tensor: \( j^{\text{con f}}_\mu = x_\nu {{T^\nu}_\mu}^{\text{con f}}\), where the symbol con f indicates that
these are the conformal energy-momentum tensor and dilatation current. Notice that this problem, secretly, requires you to also derive \( T^{\mu\nu} \) for the electromagnetic field.
\makesubproblem{}{qft:problemSet1:4d}
Show, then, that conservation of \(  j^{\text{con f}}_\mu  \) implies that the energy-momentum tensor of classical
electrodynamics is traceless (the trace of the tensor is defined as usual to be \( g_{\mu\nu} T^{\mu\nu}\)).
\makesubproblem{}{qft:problemSet1:4e}
Finally, open your classical electrodynamics books and recall the interpretation of the \( T^{00}, T^{xx},T^{yy} \), etc., components of the energy momentum tensor as energy density and pressure. Show that the tracelessness of \( T^{\mu\nu} \) is equivalent to the familiar relation
\begin{dmath}\label{eqn:ProblemSet1Problem4:680}
p = \rho/3
\end{dmath}
between the energy density and pressure of isotropic radiation -- the equation of state of blackbody radiation.
} % makeproblem

\makeanswer{qft:problemSet1:4}{
\makeSubAnswer{}{qft:problemSet1:4a}
With \( {x'}^\mu = e^{d} x^\mu \), the volume element transforms as
\begin{dmath}\label{eqn:ProblemSet1Problem4:60}
d^4 x' \rightarrow e^{4d} d^4 x.
\end{dmath}
The components of the four-gradient transform as
\begin{dmath}\label{eqn:ProblemSet1Problem4:80}
\PD{x'_\mu}{}
=
\PD{x'_\mu}{x_\mu}
\PD{x_\mu}{}
=
e^{-d}
\PD{x_\mu}{},
\end{dmath}
so
\begin{dmath}\label{eqn:ProblemSet1Problem4:100}
F'_{\mu\nu} =
\partial'_{\mu} A'_\nu
-
\partial'_{\nu} A'_\mu
=
e^{-2d} F_{\mu\nu}.
\end{dmath}
The action is therefore invariant
\begin{dmath}\label{eqn:ProblemSet1Problem4:120}
S'
= -\inv{4} \int d^4 x' F'_{\mu\nu} {F'}^{\mu\nu}
= -\inv{4} \int e^{4d} d^4 x e^{2d} F_{\mu\nu} e^{2d} F^{\mu\nu}
= -\inv{4} \int d^4 x F_{\mu\nu} F^{\mu\nu}
= S.
\end{dmath}
\makeSubAnswer{}{qft:problemSet1:4b}
We need the variation of the potential
\begin{dmath}\label{eqn:ProblemSet1Problem4:180}
\delta A_{\nu}
= A'_{\nu}(x) - A_{\nu}(x)
= A'_{\nu}(e^{-d} x') - A_{\nu}(x)
\approx A'_{\nu}((1 -d) x') - A_{\nu}(x)
= e^{-d} A_{\nu}((1 -d) x') - A_{\nu}(x)
\approx (1-d) \lr{ A_{\nu} - d x^\alpha \partial_\alpha A_{\nu}} - A_{\nu}
=
- d x^\alpha \partial_\alpha A_{\nu}
- d \lr{ A_{\nu} - d x^\alpha \partial_\alpha A_{\nu}}
\approx
-d(1+ x^\alpha \partial_\alpha )A_{\nu},
\end{dmath}
and the variation of the field
\begin{dmath}\label{eqn:ProblemSet1Problem4:140}
\delta F_{\mu\nu}
= F'_{\mu\nu}(x) - F_{\mu\nu}(x)
= F'_{\mu\nu}(e^{-d} x') - F_{\mu\nu}(x)
\approx F'_{\mu\nu}((1 -d) x') - F_{\mu\nu}(x)
= e^{-2d} F_{\mu\nu}((1 -d) x') - F_{\mu\nu}(x)
\approx (1-2d) \lr{ F_{\mu\nu} - d x^\alpha \partial_\alpha F_{\mu\nu}} - F_{\mu\nu}
=
- d x^\alpha \partial_\alpha F_{\mu\nu}
-2 d \lr{ F_{\mu\nu} - d x^\alpha \partial_\alpha F_{\mu\nu}}
\approx
-d(2+ x^\alpha \partial_\alpha )
F_{\mu\nu},
\end{dmath}
so the variation of the Lagrangian is
\begin{dmath}\label{eqn:ProblemSet1Problem4:160}
\delta \LL
=
-\inv{2} (\delta F_{\mu\nu}) F^{\mu\nu}
=
-\inv{2}
(-d)
F^{\mu\nu}
(2+ x^\alpha \partial_\alpha )F_{\mu\nu}
=
(d)
F^{\mu\nu}
F_{\mu\nu}
+
\frac{d}{2}
F^{\mu\nu} x^\alpha \partial_\alpha F_{\mu\nu}
=
(d)
F^{\mu\nu}
F_{\mu\nu}
+
\frac{d}{4}
x^\alpha \partial_\alpha \lr{ F_{\mu\nu} F^{\mu\nu} }
=
-4 (d) \LL - (d) x^\alpha \partial_\alpha \LL
=
-4 (d) \LL - (d) \lr{ \partial_\alpha (x^\alpha \LL) - \LL \partial_\alpha x^\alpha }
=
- (d) \partial_\alpha (x^\alpha \LL),
\end{dmath}
so the variational current (what is this called?) is
\begin{dmath}\label{eqn:ProblemSet1Problem4:200}
J^\mu_d
=
- (d) x^\mu \LL.
\end{dmath}
Finally, we need
\begin{dmath}\label{eqn:ProblemSet1Problem4:220}
\PD{(\partial_\mu A_\nu)}{\LL}
=
-\inv{2}
F^{\alpha\beta}
\PD{(\partial_\mu A_\nu)}{}\lr{ \partial_\alpha A_\beta - \partial_\beta A_\alpha}
=
-\inv{2}
\lr{
   F^{\mu\nu}
   -
   F^{\nu\mu}
}
=
-F^{\mu\nu}.
\end{dmath}
Combining \cref{eqn:ProblemSet1Problem4:220}, \cref{eqn:ProblemSet1Problem4:200}, and \cref{eqn:ProblemSet1Problem4:180} we can calculate the conserved current, which is (for \( d = 1 \)) is
\begin{dmath}\label{eqn:ProblemSet1Problem4:240}
j^\mu_{\text{dil}}
=
\PD{(\partial_\mu A_\nu)}{\LL} \delta A_\nu - J^\mu_d
=
F^{\mu\nu}
\lr{
   A_\nu + x^\alpha \partial_\alpha A_{\nu}
}
+
x^\mu \LL.
\end{dmath}
This can be put into a slightly nicer form
\begin{dmath}\label{eqn:ProblemSet1Problem4:260}
j^\mu_{\text{dil}}
=
F^{\mu\nu}
   A_\nu
+
F^{\mu\nu} x^\alpha
F_{\alpha\nu}
+
F^{\mu\nu} x^\alpha
\partial_\nu A_{\alpha}
+
x^\mu \LL
=
\cancel{
F^{\mu\nu}
   A_\nu
}
+
F^{\mu\nu} x^\alpha
F_{\alpha\nu}
+
\partial_\nu \lr{
   F^{\mu\nu} x^\alpha
   A_{\alpha}
}
-
A_{\alpha}
x^\alpha \cancel{\partial_\nu F^{\mu\nu} }
-
\cancel{
A_{\alpha}
F^{\mu\nu}
\partial_\nu x^\alpha
}
+
x^\mu \LL
=
F^{\mu\nu} x^\alpha
F_{\alpha\nu}
+
\partial_\nu \lr{
   F^{\mu\nu} x^\alpha
   A_{\alpha}
}
+
x^\mu \LL,
\end{dmath}
or
\boxedEquation{eqn:ProblemSet1Problem4:280}{
   j^\mu_{\text{dil}}
   =
   x^\alpha
   \lr{
      F^{\mu\nu}
      F_{\alpha\nu}
      +
      {\delta^\mu}_\alpha \LL
   }
   +
   \partial_\nu \lr{
      F^{\mu\nu} x^\alpha
      A_{\alpha}
   }
}
It was hinted that the complete derivative of an antisymmetric tensor may be dropped from the current, that's because
\begin{dmath}\label{eqn:ProblemSet1Problem4:560}
\partial_\mu \lr{ j^\mu + \partial_\nu C^{\mu\nu} }
=
\partial_\mu j^\mu + \partial_\mu \partial_\nu C^{\mu\nu}
=
\partial_\mu j^\mu,
\end{dmath}
since
the deriative operator \( \partial_\mu\partial_\nu \) is symmetric, and the sum of the contraction of symmetric and antisymmetric tensors is zero.  Since the
complete derivative term  \(
   F^{\mu\nu} x^\alpha
   A_{\alpha} \) is antisymmetric in \( \mu\nu \) so we may drop it from the current, leaving only dependence on the electromagnetic field \( F \).

\makeSubAnswer{}{qft:problemSet1:4c}
Having been given the secret that we have to calculate the energy momentum tensor, let's start with calculation of the conserved current associated with a spacetime translation
\begin{subequations}
\label{eqn:ProblemSet1Problem4:320}
\begin{equation}\label{eqn:ProblemSet1Problem4:300}
x_\mu \rightarrow x_\mu' = x_\mu + a_\mu.
\end{equation}
\begin{equation}\label{eqn:ProblemSet1Problem4:340}
%\phi(x) \rightarrow \phi'(x') = \phi'(x + a) \approx \phi(x) + a^\alpha \partial_\alpha \phi.
A_\nu(x) \rightarrow A_\nu'(x') = A_\nu(x) + a^\alpha \partial_\alpha A_\nu.
\end{equation}
\end{subequations}
The gradient \( \partial_\mu \) and volume element \( d^4 x\) are unchanged by a tranlation transformation.  The potential transforms as
\begin{dmath}\label{eqn:ProblemSet1Problem4:360}
\delta A_\nu
= A_\nu'(x) - A_\nu(x)
= A_\nu'(x' - a) - A_\nu(x)
\approx A_\nu(x) - a^\alpha \partial_\alpha A_\nu - A_\nu(x)
=
- a^\alpha \partial_\alpha A_\nu.
\end{dmath}
The field transforms as
\begin{dmath}\label{eqn:ProblemSet1Problem4:380}
\delta F_{\mu\nu}
= F_{\mu\nu}'(x) - F_{\mu\nu}(x)
= F_{\mu\nu}'(x' - a) - F_{\mu\nu}(x)
\approx F_{\mu\nu}(x) - a^\alpha \partial_\alpha F_{\mu\nu} - F_{\mu\nu}(x)
=
- a^\alpha \partial_\alpha F_{\mu\nu}.
\end{dmath}
Finally the Lagrangian density transforms as
\begin{dmath}\label{eqn:ProblemSet1Problem4:400}
\delta \LL
=
-\inv{2} (\delta F_{\mu\nu}) F^{\mu\nu}
=
\inv{2} a^\alpha \lr{ \partial_\alpha F_{\mu\nu} } F^{\mu\nu}
=
\inv{4} a^\alpha \partial_\alpha \lr{ F_{\mu\nu} F^{\mu\nu} }
=
- \partial_\alpha \lr{ a^\alpha \LL }.
\end{dmath}
That is
\begin{dmath}\label{eqn:ProblemSet1Problem4:420}
J^\mu_a = - a^\mu \LL.
\end{dmath}
The conserved current associated with spacetime translation is
\begin{dmath}\label{eqn:ProblemSet1Problem4:440}
j^\mu_a
=
\PD{(\partial_\mu A_\nu)}{\LL} \delta A_\nu - J^\mu_a
=
- F^{\mu\nu} (-a^\alpha \partial_\alpha A_\nu) + a^\mu \LL.
\end{dmath}
As was the case in \cref{eqn:ProblemSet1Problem4:280} we are able to put group all the explicit potential dependence in a discardable package
\begin{dmath}\label{eqn:ProblemSet1Problem4:460}
=
a^\alpha
F^{\mu\nu}
F_{\alpha\nu}
+
a^\alpha
F^{\mu\nu}
\partial_\nu A_\alpha
 + a^\mu \LL
=
a^\alpha
F^{\mu\nu}
F_{\alpha\nu}
+
a^\alpha
\partial_\nu
\lr{
F^{\mu\nu}
A_\alpha
}
-
a^\alpha
\cancel{ \partial_\nu F^{\mu\nu} }
A_\alpha
 + a^\mu \LL
\end{dmath}
or
\boxedEquation{eqn:ProblemSet1Problem4:480}{
   j^\mu_a
   =
   a^\alpha \lr{
      F^{\mu\nu}
      F_{\alpha\nu}
      +
      {\delta^\mu}_\alpha \LL
   }
   +
   \partial_\nu
   \lr{
      F^{\mu\nu}
      a^\alpha
      A_\alpha
   }
}
The factor \(
F^{\mu\nu}
a^\alpha
A_\alpha \) is completely antisymmetric in \( \mu\nu \) so we may drop it from the current.  From
\cref{eqn:ProblemSet1Problem4:280}, \cref{eqn:ProblemSet1Problem4:480} we can introduce
(conformal) dilatation \( \tilde{j_{\text{dil}}} \) and translation conservation
\( \tilde{j}_a \)
currents
\begin{dmath}\label{eqn:ProblemSet1Problem4:660}
\begin{aligned}
   \tilde{j}^\mu_{\text{dil}}
   &=
   -j^\mu_{\text{dil}}
   +
   \partial_\nu \lr{
      F^{\mu\nu} x^\alpha
      A_{\alpha}
   } \\
   \tilde{j}^\mu_a
   &=
   -j^\mu_a
   +
   \partial_\nu
   \lr{
      F^{\mu\nu}
      a^\alpha
      A_\alpha
   },
\end{aligned}
\end{dmath}
effectively dropping the complete derivative terms (also changing signs to match the literature \citep{jackson1975cew}).  That is
\boxedEquation{eqn:ProblemSet1Problem4:500}{
\begin{aligned}
\tilde{j}^\mu_{\text{dil}} &= x^\nu {\Theta^\mu}_\nu \\
\tilde{j}^\mu_{a} &= a^\nu {\Theta^\mu}_\nu \\
{\Theta^\mu}_\nu &=
   F^{\mu\sigma}
   F_{\sigma\nu}
   -
   {\delta^\mu}_\nu \LL.
\end{aligned}
}
Here we've factored out the common (conformal) energy momentum tensor \( {\Theta^\mu}_\nu \), which may also be written with upper indexes
\begin{dmath}\label{eqn:ProblemSet1Problem4:520}
\Theta^{\mu\nu} =
   F^{\mu\sigma}
   F_{\sigma\alpha}
   g^{\alpha\nu}
   -
   g^{\mu\nu} \LL,
\end{dmath}
which is symmetric with respect to index interchange
\begin{dmath}\label{eqn:ProblemSet1Problem4:540}
\Theta^{\nu\mu}
=
   F^{\nu\sigma}
   F_{\sigma\alpha}
   g^{\alpha\mu}
   -
   g^{\nu\mu} \LL
=
   g^{\beta\nu}
   F_{\beta\sigma}
   F^{\sigma\mu}
   -
   g^{\mu\nu} \LL
=
   F^{\mu\sigma}
   F_{\sigma\beta}
   g^{\beta\nu}
   -
   g^{\mu\nu} \LL
= \Theta^{\mu\nu}.
\end{dmath}
\makeSubAnswer{}{qft:problemSet1:4d}
We require the divergence of a Noether current to be zero, so for the dilatation current
\begin{dmath}\label{eqn:ProblemSet1Problem4:580}
0 = \partial_\mu \tilde{j}^\mu_{\text{dil}}
=
\lr{ \partial_\mu x^\nu }
{\Theta^\mu}_\nu
+
x^\nu
\partial_\mu {\Theta^\mu}_\nu
=
{\Theta^\mu}_\mu
+
x^\nu
\partial_\mu {\Theta^\mu}_\nu.
\end{dmath}
In particular for \( x = 0 \) we must have \(
{\Theta^\mu}_\mu  = 0 \).  Incidentally, given \( {\Theta^\mu}_\mu  = 0 \), then
for non-zero \( x \) we must also have \( \partial_\mu {\Theta^\mu}_\nu = 0 \).  That can be demonstrated directly utilizing the zero divergence of the Noether current for a spacetime translation
\begin{dmath}\label{eqn:ProblemSet1Problem4:600}
0
= \partial_\mu \tilde{j}^\mu_{a}
= a^\nu \partial_\mu {\Theta^\mu}_\nu.
\end{dmath}
As this is zero for all \( a \) we must have \( \partial_\mu {\Theta^\mu}_\nu = 0 \).

\makeSubAnswer{}{qft:problemSet1:4e}
The trace written out explicitly is
\begin{equation}\label{eqn:ProblemSet1Problem4:620}
0 =
{\Theta^\mu}_\mu
=
{\Theta^0}_0
+
{\Theta^1}_1
+
{\Theta^2}_2
+
{\Theta^3}_3
=
\Theta^{00}
-
\Theta^{11}
-
\Theta^{22}
-
\Theta^{33},
\end{equation}
Since \( \Theta^{00} = \inv{2}( \BE^2 + \BB^2 ) = \rho \), and \( -\Theta^{kj} = T^{(M)}_{kj} = E_k E_j + B_k B_j - \inv{2} \delta_{kj}(\BE^2 + \BB^2) \), where \( T^{(M)}_{kj} \) is the electromagnetic stress tensor (borrowing notation from \citep{jackson1975cew} again), we have
\begin{dmath}\label{eqn:ProblemSet1Problem4:700}
\rho = -\sum_{k = 1}^3 T^{(M)}_{kk}.
\end{dmath}
In \citep{griffiths1999introduction} \( T^{(M)}_{ij} \) is described as ``the force (per unit area) in the ith direction action on an element of surface oriented in the jth direction -- diagonal elements represent pressures, and off-diagonal elements are shears''.
Integratation of the stress tensor over a cube, as sketched in \cref{fig:outwards_normal_cube:outwards_normal_cubeFig1}, serves to illustrate this nicely, as only the diagonal elements contribute to such an integral.  If the total cubic face area is \( A = 6 \Delta A \), the total force of on the surface is
\begin{dmath}\label{eqn:ProblemSet1Problem4:720}
\BF
=
\int \lrT \cdot \Ba
=
\Be_1 \int \delta_{1k} \lr{ \evalbar{T^{(M)}_{k1}}{+} - \evalbar{T^{(M)}_{k1}}{-}}
+
\Be_2 \int \delta_{2k} \lr{ \evalbar{T^{(M)}_{k2}}{+} - \evalbar{T^{(M)}_{k2}}{-}}
+
\Be_3 \int \delta_{3k} \lr{ \evalbar{T^{(M)}_{k3}}{+} - \evalbar{T^{(M)}_{k3}}{-}}
=
\Delta A \Be_1 \lr{ \evalbar{T^{(M)}_{k1}}{+} - \evalbar{T^{(M)}_{k1}}{-}}
+
\Delta A \Be_2 \lr{ \evalbar{T^{(M)}_{k2}}{+} - \evalbar{T^{(M)}_{k2}}{-}}
+
\Delta A \Be_3 \lr{ \evalbar{T^{(M)}_{k3}}{+} - \evalbar{T^{(M)}_{k3}}{-}}
\end{dmath}
\imageFigure{../figures/phy2403-quantum-field-theory/outwards_normal_cubeFig1}{Cubic surface and outwards normals}{fig:outwards_normal_cube:outwards_normal_cubeFig1}{0.3}
Assuming isotropic fields, the total pressure of the fields on the surface is
\begin{dmath}\label{eqn:ProblemSet1Problem4:740}
p = \Abs{ \frac{2 \Delta A \sum_{k = 1}^3 T^{(M)}_{kk}}{6 \Delta A} }
= \inv{3} \rho,
\end{dmath}
which recovers \cref{eqn:ProblemSet1Problem4:680}.
}
