%
% Copyright � 2018 Peeter Joot.  All Rights Reserved.
% Licenced as described in the file LICENSE under the root directory of this GIT repository.
%
\makeproblem{ Scale invariance and conserved charge }{qft:problemSet1:4}{
Consider classical electrodynamics with the Lagrangian
\begin{dmath}\label{eqn:ProblemSet1Problem4:20}
S = \int d^4 x \lr{ -\inv{4} F_{\mu\nu} F^{\mu\nu} }.
\end{dmath}
Consider the following ``dilatation'' (or ``scale'') transformation:
\begin{dmath}\label{eqn:ProblemSet1Problem4:40}
\begin{aligned}
x_\mu &\rightarrow x'_\mu = e^d x_\mu \\
A_\mu(x) &\rightarrow A'_\mu(x') = e^{-d} A_\mu(x),
\end{aligned}
\end{dmath}
where \( d \) is a constant, called the dilatation parameter.
\makesubproblem{}{qft:problemSet1:4a}
Show that the action is invariant under dilatations.
\makesubproblem{}{qft:problemSet1:4b}
Find the corresponding Noether current.
\makesubproblem{}{qft:problemSet1:4c}
Show that -- perhaps, after a redefinition of \( j_\mu \) ; notice that any conserved current \( j_\mu \) can be
redefined by adding to it \( \partial^\nu C_{\mu\nu} \), where \( C_{\mu\nu} \) is antisymmetric, without spoiling its conservation
(in this case \( C \) can depend on \( x^\mu, \partial^\mu \) and \( A^\mu \), of course) the dilatation current is simply related
to the energy-momentum tensor: \( j^{\text{con f}}_\mu = x_\nu {{T^\nu}_\mu}^{\text{con f}}\), where the symbol con f indicates that
these are the conformal energy-momentum tensor and dilatation current. Notice that this problem, secretly, requires you to also derive \( T^{\mu\nu} \) for the electromagnetic field.
\makesubproblem{}{qft:problemSet1:4d}
Show, then, that conservation of \(  j^{\text{con f}}_\mu  \) implies that the energy-momentum tensor of classical
electrodynamics is traceless (the trace of the tensor is defined as usual to be \( g_{\mu\nu} T^{\mu\nu}\)).
\makesubproblem{}{qft:problemSet1:4e}
Finally, open your classical electrodynamics books and recall the interpretation of the \( T^{00}, T^{xx},T^{yy} \), etc., components of the energy momentum tensor as energy density and pressure. Show that the tracelessness of \( T^{\mu\nu} \) is equivalent to the familiar relation
\( p = \rho/3 \) between the energy density and pressure of isotropic radiation -- the equation of state of blackbody radiation.
Dilatation invariance in QED (and QCD) is perhaps the simplest example of a symmetry, where the classical action is invariant, but the quantum theory is not (as you will learn later, in the spring class). Broken scale invariance arises because one has to introduce a short-distance cutoff (a UV ``regulator'') to define the quantum theory. (We already saw an indication of the need for a regulator when we considered the divergent zero point energy of the free quantum scalar field.)
} % makeproblem

\makeanswer{qft:problemSet1:4}{
\makeSubAnswer{}{qft:problemSet1:4a}
TODO.
\makeSubAnswer{}{qft:problemSet1:4a}
TODO.
\makeSubAnswer{}{qft:problemSet1:4b}
TODO.
\makeSubAnswer{}{qft:problemSet1:4c}
TODO.
\makeSubAnswer{}{qft:problemSet1:4d}
TODO.
\makeSubAnswer{}{qft:problemSet1:4e}
TODO.
}
