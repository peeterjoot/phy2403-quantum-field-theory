%
% Copyright � 2018 Peeter Joot.  All Rights Reserved.
% Licenced as described in the file LICENSE under the root directory of this GIT repository.
%
\makeproblem{The action of an external source perturbation and particle creation}{qft:problemSet3:2}{

{\flushleft{In}} class, the problem of creation of  particles by an external source in quantum mechanics was discussed. Let us now study this using QFT and Feynman diagrams. Consider a massive scalar free field interacting with a classical source $j(x)$ via:
\begin{equation}
\label{h1}
H = H_0 + \int d^3 x (- j(x) \phi(x))~.
\end{equation}
The classical source $j(x)$ is nonzero only for a finite amount of time, i.e. it is turned on and off, is assumed localized in space, and thus has a well-defined four-dimensional Fourier transform (thus the source is not itself a generalized function).

\makesubproblem{}{qft:problemSet3:2a}
Argue---e.g. using our expressions for overlap of $\vert 0 \rangle$ and $\vert \Omega \rangle$ from class, as well as their meaning---that the probability that the source $j(x)$ creates no particles is 
\begin{equation}
\label{p1}
P(0) = \bigg\vert \langle 0 \vert T \left\{e^{ i \int d^4 x j(x) \phi_I(x)} \right\} \vert 0 \rangle \bigg\vert^2~.
\end{equation}
\makesubproblem{}{qft:problemSet3:2b}
Find the order-$j^2$ term in $P(0)$ and show that $P(0) = 1 - \lambda + {\cal{O}} (j^4)$, where
\begin{equation}
\label{p2}
\lambda = \int {d^3 p \over (2 \pi)^3} {1 \over 2 \omega_{\Bp}} \vert \tilde{j}(p) \vert^2~,~{\rm where} ~~ \tilde{j}(p) \equiv \int d^4 y e^{i p \cdot y} j(y)~.
\end{equation}
\makesubproblem{}{qft:problemSet3:2c}
Represent the term computed above as a Feynman diagram. Now represent the entire series for $P(0)$ in terms of Feynman diagrams. Show that the series exponentiates and, therefore, $P(0) = e^{- \lambda}$.
\makesubproblem{}{qft:problemSet3:2d}
Find the probability that the source creates one particle of momentum $\Bk$. First, compute this to order $j$ and then to all orders, using the trick above to sum the series. 
\makesubproblem{}{qft:problemSet3:2e}
Show that the probability of producing $n$ particles is $P(n) = {1 \over n!} \lambda^n e^{- \lambda}$, the Poisson distribution. 
\makesubproblem{}{qft:problemSet3:2f}
Show that $\sum\limits_{n=0}^\infty P(n) = 1$ and that $\langle N \rangle = \sum\limits_{n=0}^\infty n P(n) = \lambda$, where $\lambda$ is given in (\ref{p2}). Notice that the expression for the mean particle number $\langle N \rangle$ created exactly reproduces (when dimensionally reduced to $d=1$) the one from quantum mechanics given in class. Finally, compute the mean square fluctuation $\langle (N - \langle N \rangle)^2 \rangle$.
} % makeproblem

\makeanswer{qft:problemSet3:2}{
\makeSubAnswer{}{qft:problemSet3:2a}
TODO.
\makeSubAnswer{}{qft:problemSet3:2b}
TODO.
\makeSubAnswer{}{qft:problemSet3:2c}
TODO.
\makeSubAnswer{}{qft:problemSet3:2d}
TODO.
\makeSubAnswer{}{qft:problemSet3:2e}
TODO.
\makeSubAnswer{}{qft:problemSet3:2f}
TODO.
}
