%
% Copyright � 2018 Peeter Joot.  All Rights Reserved.
% Licenced as described in the file LICENSE under the root directory of this GIT repository.
%
\makeproblem{The action of an external source perturbation and particle creation}{qft:problemSet3:2}{

{\flushleft{In}} class, the problem of creation of  particles by an external source in quantum mechanics was discussed. Let us now study this using QFT and Feynman diagrams. Consider a massive scalar free field interacting with a classical source $j(x)$ via:
\begin{equation}
\label{h1}
H = H_0 + \int d^3 x (- j(x) \phi(x))~.
\end{equation}
The classical source $j(x)$ is nonzero only for a finite amount of time, i.e. it is turned on and off, is assumed localized in space, and thus has a well-defined four-dimensional Fourier transform (thus the source is not itself a generalized function).

\makesubproblem{}{qft:problemSet3:2a}
Argue---e.g. using our expressions for overlap of $\vert 0 \rangle$ and $\vert \Omega \rangle$ from class, as well as their meaning---that the probability that the source $j(x)$ creates no particles is
\begin{equation}
\label{p1}
P(0) = \bigg\vert \langle 0 \vert T \left\{e^{ i \int d^4 x j(x) \phi_I(x)} \right\} \vert 0 \rangle \bigg\vert^2~.
\end{equation}
\makesubproblem{}{qft:problemSet3:2b}
Find the order-$j^2$ term in $P(0)$ and show that $P(0) = 1 - \lambda + {\cal{O}} (j^4)$, where
\begin{equation}
\label{p2}
\lambda = \int {d^3 p \over (2 \pi)^3} {1 \over 2 \omega_{\Bp}} \vert \tilde{j}(p) \vert^2~,~{\rm where} ~~ \tilde{j}(p) \equiv \int d^4 y e^{i p \cdot y} j(y)~.
\end{equation}
\makesubproblem{}{qft:problemSet3:2c}
Represent the term computed above as a Feynman diagram. Now represent the entire series for $P(0)$ in terms of Feynman diagrams. Show that the series exponentiates and, therefore, $P(0) = e^{- \lambda}$.
\makesubproblem{}{qft:problemSet3:2d}
Find the probability that the source creates one particle of momentum $\Bk$. First, compute this to order $j$ and then to all orders, using the trick above to sum the series.
\makesubproblem{}{qft:problemSet3:2e}
Show that the probability of producing $n$ particles is $P(n) = {1 \over n!} \lambda^n e^{- \lambda}$, the Poisson distribution.
\makesubproblem{}{qft:problemSet3:2f}
Show that $\sum\limits_{n=0}^\infty P(n) = 1$ and that $\langle N \rangle = \sum\limits_{n=0}^\infty n P(n) = \lambda$, where $\lambda$ is given in (\ref{p2}). Notice that the expression for the mean particle number $\langle N \rangle$ created exactly reproduces (when dimensionally reduced to $d=1$) the one from quantum mechanics given in class. Finally, compute the mean square fluctuation $\langle (N - \langle N \rangle)^2 \rangle$.
} % makeproblem

\makeanswer{qft:problemSet3:2}{
\makeSubAnswer{}{qft:problemSet3:2a}
TODO.
\makeSubAnswer{}{qft:problemSet3:2b}
Expanding matrix element in powers of \( j \) we have
\begin{dmath}\label{eqn:ProblemSet3Problem2:100}
\bra{0} T\lr{ \exp\lr{ i \int d^4 x j(x) \phi_I(x) }} \ket{0}
=
\bra{0} 1 \ket{0}
+
i \bra{0} T\lr{ \int d^4 a j(a) \phi_I(a) } \ket{0}
+
\frac{i^2}{2!} \bra{0} T\lr{ \int d^4 a d^4 b j(a) \phi_I(a) j(b) \phi_I(b) } \ket{0}
+
\frac{i^3}{3!} \bra{0} T\lr{ \int d^4 a d^4 b d^4 c j(a) \phi_I(a) j(b) \phi_I(b) j(c) \phi_I(c) } \ket{0}
+
\cdots
\end{dmath}
Using Wick's theorem to evaluate the integrals, all the odd powers of \( j \) are zero.  We may evaluate the first non-zero integral by contracting the two fields
\begin{dmath}\label{eqn:ProblemSet3Problem2:120}
\bra{0} T\lr{ \int d^4 a d^4 b j(a) \phi_I(a) j(b) \phi_I(b) } \ket{0}
=
\contraction{\int d^4 a d^4 b j(a) }{\phi}{{}_I(a) j(b) }{\phi}
\int d^4 a d^4 b j(a) \phi_I(a) j(b) \phi_I(b)
=
\int d^4 a d^4 b j(a) D_F(a - b) j(b)
=
i \int d^4 a d^4 b \frac{d^4 p}{(2 \pi)^4} j(a) j(b) \frac{ e^{-i p \cdot (a - b) } }{p^2 - m^2 + i \epsilon}
=
i \int \frac{ d^4 p }{(2 \pi)^4} \inv{ p^2 - m^2 + i \epsilon}
\int d^4 a j(a) e^{-i p \cdot a}
\int d^4 b j(b) e^{i p \cdot b}
=
i \int \frac{ d^4 p }{(2 \pi)^4} \inv{ p^2 - m^2 + i \epsilon }
\tilde{j}(p)
\tilde{j}(-p).
\end{dmath}
Assuming that \( j(x) \) is real, this is
\begin{dmath}\label{eqn:ProblemSet3Problem2:140}
\bra{0} T\lr{ \int d^4 a d^4 b j(a) \phi_I(a) j(b) \phi_I(b) } \ket{0}
=
i \int \frac{ d^4 p }{(2 \pi)^4} \frac{
\Abs{\tilde{j}(p)}^2
}{p^2 - m^2 + i \epsilon}
=
\frac{i}{2 \pi} \int dp_0 \int \frac{d^3 p}{(2 \pi)^3} \frac{
\Abs{\tilde{j}(p)}^2
 }{p_0^2 - \Bp^2 - m^2 + i \epsilon}
=
\frac{i}{2 \pi} \int dp_0 \int \frac{d^3 p}{(2 \pi)^3} \frac{
\Abs{\tilde{j}(p)}^2
 }{p_0^2 - \omega_\Bp^2 + i \epsilon}.
\end{dmath}
Integrating \( p_0 \) over the lower half plane contour of \cref{fig:ps3p2:ps3p2Fig1}, which encloses the pole at \( p_0 \approx \omega_\Bp - i \epsilon \) we have
\imageFigure{../figures/phy2403-quantum-field-theory/ps3p2LowerHalfFeynmandContourFig2}{Feynman propagator contour in lower half plane.}{fig:ps3p2:ps3p2Fig1}{0.2}
\begin{dmath}\label{eqn:ProblemSet3Problem2:160}
\bra{0} T\lr{ \int d^4 a d^4 b j(a) \phi_I(a) j(b) \phi_I(b) } \ket{0}
=
\frac{-2 \pi i^2}{2 \pi} \int \frac{d^3 p}{(2 \pi)^3} \evalbar{ \frac{ \Abs{\tilde{j}(p)}^2 }{2 \omega_\Bp}
}{p_0 = \omega_\Bp}
=
\lambda,
\end{dmath}
where it has been assumed that \( \tilde{j}(p_0, \Bp) \rightarrow 0 \) along the infinite circular arc of the integration contour.
To second order in \( j \) our matrix element is
\begin{dmath}\label{eqn:ProblemSet3Problem2:180}
\bra{0} T\lr{ \exp\lr{ i \int d^4 x j(x) \phi_I(x) }} \ket{0}
=
1 - \frac{\lambda}{2}.
\end{dmath}
We can now use this as an initial approximation for the probability
\begin{dmath}\label{eqn:ProblemSet3Problem2:200}
P(0) =
\lr{ 1 - \frac{\lambda}{2} + O(\lambda^2) }^2
=
1 - \lambda + O(\lambda^2),
\end{dmath}
as desired.

\makeSubAnswer{}{qft:problemSet3:2c}
The diagram for the integral just computed is a single line segment as sketched in \cref{fig:ps3OneEdgeDiagram:ps3OneEdgeDiagramFig3}.
\imageFigure{../figures/phy2403-quantum-field-theory/ps3OneEdgeDiagramFig3}{\( j^2 \) diagram.}{fig:ps3OneEdgeDiagram:ps3OneEdgeDiagramFig3}{0.15}
The diagram for the \( j^4 \) integrals are sketched in
\cref{fig:ps3jfourthDiagram:ps3jfourthDiagramFig4}.
\imageFigure{../figures/phy2403-quantum-field-theory/ps3jfourthDiagramFig4}{\( j^4 \) diagrams.}{fig:ps3jfourthDiagram:ps3jfourthDiagramFig4}{0.2}
After that the diagrams get messier to enumerate.  We can see the pattern by considering a non-trivial example such as the \( j^6 \) integral.  For that each diagram has three edges, where all possible combinations \( ab, ac, ad, ae, af, bc, bd, be, bf, cd, ce, cf, de, df, ef \) are found for the ``first'' edge in each diagram.  This is a total of \( \binom{6}{2} = 15 \) edges.  For each such diagram, there are \( \binom{4}{2} \) choices for the next edge in the diagram (example: \( ab, cd, ef \) would be one diagram). The coefficent of this integral is therefore
\begin{dmath}\label{eqn:ProblemSet3Problem2:220}
\frac{i^6}{3!} \inv{6!} \binom{6}{2} \binom{4}{2}
=
\frac{(-1)^3}{3!} \inv{6!} \frac{6!}{4!2!} \frac{4!}{2!2!}
=
\frac{(-1)^3}{3! 2^3}.
\end{dmath}
Here the \( 3! \) downstairs is to compensate for the fact that there are \( 3 \times 2 \) possible orderings of each distinct pair of endpoints.  Example: \( \setlr{ ab, cd, ef }, \setlr{ ab, ef, cd}, \setlr{ cd, ab, ef }, \setlr{ cd, ef, ab }, \setlr{ ef, ab, cd }, \setlr{ ef, cd, ab } \).
The pattern of perfect cancellation is clear.  The \( j^{2n} \) order integral is
\begin{dmath}\label{eqn:ProblemSet3Problem2:240}
\frac{(-1)^n}{n!} \inv{(2n)!} \frac{2n!}{2n -2!2!} \cdots \frac{4!}{2!2!} \lambda^n
=
\frac{(-\lambda/2)^n}{n!},
\end{dmath}
so we find that
\begin{dmath}\label{eqn:ProblemSet3Problem2:260}
P(0) =
\lr{
\sum_{n = 0}^\infty
\frac{(-\lambda/2)^n}{n!}
}^2
=
(e^{-\lambda/2})^2
=
e^{-\lambda},
\end{dmath}
as desired.

\makeSubAnswer{}{qft:problemSet3:2d}
TODO.
\makeSubAnswer{}{qft:problemSet3:2e}
TODO.
\makeSubAnswer{}{qft:problemSet3:2f}
The sum of the probabilities is easy to compute
\begin{dmath}\label{eqn:ProblemSet3Problem2:20}
\sum_{n = 0}^\infty P(n)
=
e^{-\lambda} \sum_{n = 0}^\infty \inv{n!} \lambda^n
=
e^{-\lambda} e^{\lambda}
= 1.
\end{dmath}
The mean is
\begin{dmath}\label{eqn:ProblemSet3Problem2:40}
\expectation{N}
=
\sum_{n = 0}^\infty n P(n)
=
e^{-\lambda} \sum_{n = 1}^\infty \frac{n }{n!} \lambda^n
=
e^{-\lambda} \lambda \sum_{n = 1}^\infty \frac{1 }{(n-1)!} \lambda^{n-1}
=
e^{-\lambda} \lambda e^{\lambda}
=
\lambda.
\end{dmath}
For the mean square we first compute
\begin{dmath}\label{eqn:ProblemSet3Problem2:60}
\expectation{N^2}
\sum_{n = 0}^\infty n^2 P(n)
=
e^{-\lambda} \sum_{n = 1}^\infty \frac{n^2 }{n!} \lambda^n
=
e^{-\lambda} \lambda \sum_{n = 1}^\infty \frac{n }{(n-1)!} \lambda^{n-1}
=
e^{-\lambda} \lambda \sum_{n = 0}^\infty \frac{n+1 }{n!} \lambda^{n}
=
e^{-\lambda} \lambda \lr{ e^\lambda + \sum_{n = 0}^\infty \frac{n}{n!} \lambda^{n} }
=
\lambda +
e^{-\lambda} \lambda^2 e^{\lambda}
=
\lambda + \lambda^2,
\end{dmath}
so
\begin{dmath}\label{eqn:ProblemSet3Problem2:80}
\expectation{(N - \expectation{N})^2}
=
\expectation{N^2 - 2 N \expectation{N} + \expectation{N}^2}
=
\expectation{N^2} - 2 \expectation{N}^2 + \expectation{N}^2
=
\expectation{N^2} - \expectation{N}^2
=
\lambda + \lambda^2 - \lambda^2
= \lambda.
\end{dmath}
}
