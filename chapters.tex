%
% Copyright © 2018 Peeter Joot.  All Rights Reserved.
% Licenced as described in the file LICENSE under the root directory of this GIT repository.
%
%----------------------------------------------------------------------------------------
\part{Lecture notes}
   \chapter{Introduction.}
      %
% Copyright � 2018 Peeter Joot.  All Rights Reserved.
% Licenced as described in the file LICENSE under the root directory of this GIT repository.
%
%{
%%\input{../latex/blogpost.tex}
%%\renewcommand{\basename}{qftLecture1}
%%%\renewcommand{\dirname}{notes/phy1520/}
%%\renewcommand{\dirname}{notes/ece1228-electromagnetic-theory/}
%%%\newcommand{\dateintitle}{}
%%%\newcommand{\keywords}{}
%%
%%\input{../latex/peeter_prologue_print2.tex}
%%
%%\usepackage{peeters_layout_exercise}
%%\usepackage{peeters_braket}
%%\usepackage{peeters_figures}
%%\usepackage{siunitx}
%%\usepackage{verbatim}
%%%\usepackage{mhchem} % \ce{}
%%%\usepackage{macros_bm} % \bcM
%%%\usepackage{macros_qed} % \qedmarker
%%%\usepackage{txfonts} % \ointclockwise
%%
%%\beginArtNoToc
%%
%%\generatetitle{UofT QFT Fall 2018 phy2403 Lecture 1; What is a field?.  Taught by Prof. Erich Poppitz}
%%%\chapter{What is a field?}
%%%\label{chap:qftLecture1}
%%
%%\paragraph{DISCLAIMER: Very rough notes from class.  Some additional side notes, but otherwise barely edited.}
%%
% Monday Sept 10, 2018.
\section{What is a field?}
\index{field}

\begin{comment}
\end{comment}
A field is a map from space(time) to some set of numbers.  These set of numbers may be organized some how, possibly scalars, or vectors, ...

One example is the familiar spacetime vector, where \( \Bx \in \bbR^{d} \)

\begin{dmath}\label{eqn:qftLecture1:20}
(\Bx, t) \rightarrow \bbR^{\lr{d,1}}
\end{dmath}

Examples of fields:
\begin{enumerate}
\item \( 0 + 1 \) dimensional ``QFT'', where the spatial dimension is zero dimensional and we have one time dimension.  Fields in this case are just functions of time \( x(t) \).  That is, particle mechanics is a 0 + 1 dimensional classical field theory.  We know that classical mechanics is described by the action
\begin{dmath}\label{eqn:qftLecture1:40}
S = \frac{m}{2} \int dt \xdot^2.
\end{dmath}
This is non-relativistic.  We can make this relativistic by saying this is the first order term in the Taylor expansion
\begin{dmath}\label{eqn:qftLecture1:60}
S = - m c^2 \int dt \sqrt{ 1 - \xdot^2/c^2 }.
\end{dmath}
Classical field theory (of \( x(t) \)).  The ``QFT'' of \( x(t) \).  i.e. QM.
All of you know quantum mechanics.  If you don't just leave.  Not this way (pointing to the window), but this way (pointing to the door).
The solution of a quantum mechanical state is
\begin{dmath}\label{eqn:qftLecture1:80}
\bra{x} e^{-i H t/\Hbar } \ket{x'},
\end{dmath}
which can be found by evaluating the ``Feynman path integral''
\begin{dmath}\label{eqn:qftLecture1:100}
\sum_{\text{all paths x}} e^{i S[x]/\Hbar}
\end{dmath}
This will be particularly useful for QFT, despite the fact that such a sum is really hard to evaluate (try it for the Hydrogen atom for example).
\item \( 3 + 0 \) dimensional field theory, where we have 3 spatial dimensions and 0 time dimensions.  Classical equilibrium static systems.  The field may have a structure like
\begin{dmath}\label{eqn:qftLecture1:120}
\Bx \rightarrow \BM(\Bx),
\end{dmath}
for example, magnetization.
We can write the solution to such a system using the partition function
\begin{dmath}\label{eqn:qftLecture1:140}
Z \sim \sum_{\text{all} \BM(x)} e^{-E[\BM]/\kB T}.
\end{dmath}
For such a system the energy function may be like
\begin{dmath}\label{eqn:qftLecture1:160}
E[\BM] = \int d^3 \Bx \lr{ a \BM^2(\Bx) + b \BM^4(\Bx) + c \sum_{i = 1}^3 \lr{ \PD{x_i}{} \BM }
\cdot \lr{ \PD{x_i}{} \BM }
}.
\end{dmath}
There is an analogy between the partition function and the Feynman path integral, as both are summing over all possible energy states in both cases.
This will be probably be the last time that we mention the partition function and condensed matter physics in this term for this class.
\item \( 3 + 1 \) dimensional field theories, with 3 spatial dimensions and 1 time dimension.
Example, electromagnetism with \( \BE(\Bx, t), \BB(\Bx, t) \) or better use \( \BA(\Bx, t), \phi(\Bx, t) \).  The action is
\begin{dmath}\label{eqn:qftLecture1:180}
S = -\inv{16 \pi c} \int d^3 \Bx dt \lr{ \BE^2 - \BB^2 }.
\end{dmath}
This is our first example of a relativistic field theory in \( 3 + 1 \) dimensions.  It will take us a while to get there.
\end{enumerate}

These are examples of classical field theories, such as fluid dynamics and general relativity.  We want to consider electromagnetism because this is the place that we everything starts to fall apart (i.e. blackbody radiation, relating to the equilibrium states of radiating matter).  Part of the resolution of this was the quantization of the energy states, where we studied the normal modes of electromagnetic radiation in a box.  These modes can be considered an infinite number of radiating oscillators (the ultraviolet catastrophe).  This was resolved by Planck by requiring those energy states to be quantized (an excellent discussion of this can be found in \citep{bohm1989qt}.  In that sense you have already seen quantum field theory.

For electromagnetism the classical description is not always good.  Examples:
\begin{enumerate}
\item blackbody radiation.
\item electron energy \( e^2/r_\txte \) of a point charge diverges as \( r_\txte \rightarrow 0 \).
We can define the classical radius of the electron by
\begin{dmath}\label{eqn:qftLecture1:200}
\frac{e^2}{r^{\textrm{cl}}_{\txte}} \sim m_\txte c^2,
\end{dmath}
or
\begin{equation}\label{eqn:qftLecture1:220}
r^{\textrm{cl}}_{\txte} \sim \frac{m_\txte c^2}{e^2} \sim 10^{-15} \si{m}
\end{equation}
Don't treat this very seriously, but it becomes useful at frequencies \( \omega \sim c/r_\txte \), where \( r_\txte/c \) is approximately the time for light to cross a distance \( r_\txte \).
At frequencies like this, we should not believe the solutions that are obtained by classical electrodynamics.
In particular, self-accelerating solutions appear at these frequencies in classical EM.  This is approximately \( \omega_\conj \sim 10^{23} Hz \), or
\begin{dmath}\label{eqn:qftLecture1:240}
\begin{aligned}
\Hbar \omega_\conj
&\sim \lr{ 10^{-21} \,\si{MeV s}} \lr{ 10^{23} \,\si{1/s} }\\
&\sim 100 \si{MeV}.
\end{aligned}
\end{dmath}
% m_\txte \sim \inv{2} MeV
\end{enumerate}
At such frequencies particle creation becomes possible.

\section{Scales.}
\index{scale}

A (dimensionless) value that is very useful in determining scale is
\begin{equation}\label{eqn:qftLecture1:260}
\alpha = \frac{e^2}{4 \pi \Hbar c} \sim \inv{137},
\end{equation}
called the fine scale constant, which relates three important scales relevant to quantum mechanics, as sketched in
\cref{fig:Lecture1scales:Lecture1scalesFig1}.
\imageFigure{../figures/phy2403-quantum-field-theory/Lecture1scalesFig1}{Interesting scales in quantum mechanics.}{fig:Lecture1scales:Lecture1scalesFig1}{0.3}
\begin{itemize}
\item The Bohr radius (large end of the scale).
\item The Compton wavelength of the electron.
\item The classical radius of the electron.
\end{itemize}

\subsection{Bohr radius}
\index{Bohr radius}

A quick motivation for the Bohr radius was mentioned in passing in class while discussing scale, following the high school method of deriving the Balmer series (\citep{french1998iqp}).

That method assumes a circular electron trajectory (\(i = \Be_1 \Be_2\))
\begin{dmath}\label{eqn:qftLecture1:280}
\begin{aligned}
\Br &= r \Be_1 e^{i \omega t} \\
\Bv &= \omega r \Be_2 e^{i \omega t} \\
\Ba &= -\omega^2 r \Be_1 e^{i \omega t} \\
\end{aligned}
\end{dmath}
The Coulomb force (in cgs units) on the electron is
\begin{dmath}\label{eqn:qftLecture1:300}
\BF = m\Ba = -m \omega^2 r \Be_1 e^{i \omega t} = \frac{-e (e)}{r^2} \Be_1 e^{i \omega t},
\end{dmath}
or
\begin{dmath}\label{eqn:qftLecture1:320}
m \lr{ \frac{v}{r}}^2 r = \frac{e^2}{r^2},
\end{dmath}
giving
\begin{dmath}\label{eqn:qftLecture1:340}
m v^2 = \frac{e^2}{r}.
\end{dmath}
The energy of the system, including both Kinetic and potential (from an infinite reference point) is
\begin{dmath}\label{eqn:qftLecture1:360}
E
= \inv{2} m v^2 - \frac{e^2}{r}
= - \inv{2} m v^2 \sim \Hbar \omega = \Hbar \frac{v}{r},
\end{dmath}
or
\begin{dmath}\label{eqn:qftLecture1:380}
m v r \sim \Hbar.
\end{dmath}
Eliminating \( v \) using \cref{eqn:qftLecture1:340}, assuming a ground state radius \( r = a_0 \) gives
% \( m a_0 \sim \Hbar^2/e^2 \), orS
\begin{dmath}\label{eqn:qftLecture1:400}
a_0 \sim \frac{\Hbar^2}{m e^2}.
\end{dmath}
The Bohr radius is of the order \( 10^{-10} \si{m} \).

\subsection{Compton wavelength.}
\index{Compton wavelength}

When particle momentum starts approaching the speed of light, by the uncertainty relation (\(\Delta x \Delta p \sim \Hbar\)) the variation in position must be of the order
\begin{dmath}\label{eqn:qftLecture1:420}
\lambda_\txtc \sim \frac{\Hbar}{m_\txte c},
\end{dmath}
called the Compton wavelength.
Similarly, when the length scales are reduced to the Compton wavelength, the momentum increases to relativistic levels.
Because of the relativistic velocities at the Compton wavelength, particle creation and annihilation occurs and any theory has to account for multiple particle states.

\subsection{Relations.}

\index{fine structure constant}
Scaling the Bohr radius once by the fine structure constant, we obtain the Compton wavelength (after dropping factors of \( 4\pi \))
\begin{dmath}\label{eqn:qftLecture1:440}
a_0 \alpha
= \frac{\Hbar^2}{m e^2}
\frac{e^2}{4 \pi \Hbar c}
= \frac{\Hbar}{4 \pi m c}
\sim
\frac{\Hbar}{m c}
= \lambda_\txtc.
\end{dmath}
Scaling once more, we obtain (after dropping another \( 4\pi\)) the classical electron radius
\begin{dmath}\label{eqn:qftLecture1:460}
\lambda_\txtc \alpha
=
\frac{e^2}{4 \pi m c^2}
\sim
\frac{e^2}{m c^2}.
\end{dmath}

%}
%%\EndArticle

   \chapter{Units, and scales.}
      %
% Copyright � 2018 Peeter Joot.  All Rights Reserved.
% Licenced as described in the file LICENSE under the root directory of this GIT repository.
%
%{
\input{../latex/blogpost.tex}
\renewcommand{\basename}{qftLecture2}
%\renewcommand{\dirname}{notes/phy1520/}
\renewcommand{\dirname}{notes/ece1228-electromagnetic-theory/}
%\newcommand{\dateintitle}{}
%\newcommand{\keywords}{}

\input{../latex/peeter_prologue_print2.tex}

\usepackage{peeters_layout_exercise}
\usepackage{peeters_braket}
\usepackage{peeters_figures}
\usepackage{siunitx}
%\usepackage{mhchem} % \ce{}
%\usepackage{macros_bm} % \bcM
%\usepackage{macros_qed} % \qedmarker
%\usepackage{txfonts} % \ointclockwise

\beginArtNoToc

% Wednesday Sept 12, 2018.
\generatetitle{UofT QFT Fall 2018 Lecture 2, taught by Prof. Erich Poppitz}
%\chapter{UofT QFT Fall 2018 Lecture 2, taught by Prof. Erich Poppitz}

%}
\EndArticle
%\EndNoBibArticle

      \section{Problems.}
         %
% Copyright � 2015 Peeter Joot.  All Rights Reserved.
% Licenced as described in the file LICENSE under the root directory of this GIT repository.
%
\makeoproblem{Dimensional analysis.}
{qft:LukeProblemSet1:4}
{2015 ps1.4}
{

Even though we have set \( \Hbar = c = 1 \), we can still do dimensional analysis because we still have one unit left, mass (or 1/length). In \( d \) space-time dimensions (1 time and \( d-1 \) space), what is the dimension in mass units of a canonical free scalar field, \( \phi \)? (Work it out from the equal-time commutation relations.) Still in \( d \) dimensions, the Lagrange density for a scalar field with self-interactions might be of the form
\begin{equation}\label{eqn:LukeProblemSet1Problem4:20}
\LL = \inv{2} \lr{ \partial_\mu \phi}^2 - \sum_{n \ge 2} a_n \phi^n.
\end{equation}
\makesubproblem{}{qft:LukeProblemSet1:4a}
What is the dimension (again in mass units) of the Lagrange density?
\makesubproblem{}{qft:LukeProblemSet1:4b}
The action?
\makesubproblem{}{qft:LukeProblemSet1:4c}
The coefficients \( a_n \)? (as a check, whatever the value of \( d \), \(a_2\) had better have the dimensions of \(\textrm{mass}^2\) ).
} % makeproblem
\makeanswer{qft:LukeProblemSet1:4}{
\withproblemsetsParagraph{
\makeSubAnswer{}{qft:LukeProblemSet1:4a}
With \( \antisymmetric{\phi(\Bx)}{\pi(\By)} = i \deltathree(\Bx - \By) \), which is dimensionless, we have
\begin{equation}\label{eqn:LukeProblemSet1Problem4:40}
\begin{aligned}
1
&= [ \phi \pi ]
\\&= [ \phi^2 ] /L,
\end{aligned}
\end{equation}
%
so
\begin{equation}\label{eqn:LukeProblemSet1Problem4:60}
[\phi] = L^{1/2}.
\end{equation}

This means that the dimensions of the Lagrangian are
\begin{equation}\label{eqn:LukeProblemSet1Problem4:80}
\begin{aligned}
[\LL]
&= [(\partial_\mu \phi)^2]
\\&= \inv{L^2} L
\\&= \inv{L}.
\end{aligned}
\end{equation}
\makeSubAnswer{}{qft:LukeProblemSet1:4b}
The dimensions of the action are
\begin{equation}\label{eqn:LukeProblemSet1Problem4:100}
\begin{aligned}
[S]
&= [ \int d^d x \LL ]
\\&= L^d \inv{L}
\\&= L^{d-1}
\end{aligned}
\end{equation}
\makeSubAnswer{}{qft:LukeProblemSet1:4c}
The dimensions of the coefficients are found from
\begin{equation}\label{eqn:LukeProblemSet1Problem4:120}
\begin{aligned}
\inv{L}
&=
[a_n \phi^n]
\\&=
[a_n] L^{n/2},
\end{aligned}
\end{equation}
%
or
\begin{equation}\label{eqn:LukeProblemSet1Problem4:140}
[a_n] = L^{-1 - n/2}.
\end{equation}

For \( n = 2 \) that is \( [a_n] = L^{-1 - 2/2} = L^{-2} \).  Provided \( [L] = 1/[M] \) this is what is expected.  To see that is the case consider the dimensions of the ratio
\begin{equation}\label{eqn:qftProblemSet1Problem4:160}
\begin{aligned}
[\Hbar/c]
&= [ (M L^2/T)/(L/T) ]
\\&= [ M L ].
\end{aligned}
\end{equation}

If both \( \Hbar \) and \( c \) are dimensionless then the dimensions of length must be inverse mass.
}
}

   \chapter{Lorentz transformations.}
      %
% Copyright © 2018 Peeter Joot.  All Rights Reserved.
% Licenced as described in the file LICENSE under the root directory of this GIT repository.
%
%{
\section{Lorentz transformations.}
\index{Lorentz transformations}

The goal, perhaps not for today, is to study the simplest (relativistic) scalar field theory.  First studied classically, and then consider such a quantum field theory.
How is relativity implemented when we write the Lagrangian and action?

Our first step must be to consider Lorentz transformations and the Lorentz group.

Spacetime (Minkowski space) is \R{3,1} (or \R{d-1,1}).  Our coordinates are
\begin{equation}\label{eqn:qftLecture2:340}
(c t, x^1, x^2, x^3) = (c t, \Br).
\end{equation}
Here, we've scaled the time scale by \( c \) so that we measure time and space in the same dimensions.  We write this as
\begin{equation}\label{eqn:qftLecture2:360}
x^\mu = (x^0, x^1, x^2, x^3),
\end{equation}
%
where \( \mu = 0, 1, 2, 3 \), and call this a ``4-vector''.  These are called the space-time coordinates of an event, which tell us where and when an event occurs.

For two events whose spacetime coordinates differ by \( dx^0, dx^1, dx^2, dx^3 \) we introduce the notion of a space time \underline{interval}
\begin{equation}\label{eqn:qftLecture2:380}
\begin{aligned}
ds^2
&= c^2 dt^2
- (dx^1)^2
- (dx^2)^2
- (dx^3)^2 \\
&=
\sum_{\mu, \nu = 0}^3 g_{\mu\nu} dx^\mu dx^\nu
\end{aligned}
\end{equation}
Here \( g_{\mu\nu} \) is the Minkowski space metric, an object with two indexes that run from 0-3.  i.e. this is a diagonal matrix
\begin{equation}\label{eqn:qftLecture2:400}
g_{\mu\nu} \sim
\begin{bmatrix}
1 & 0 & 0 & 0 \\
0 & -1 & 0 & 0 \\
0 & 0 & -1 & 0 \\
0 & 0 & 0 & -1 \\
\end{bmatrix},
\end{equation}
that is
\begin{equation}\label{eqn:qftLecture2:420}
\begin{aligned}
g_{00} &= 1 \\
g_{11} &= -1 \\
g_{22} &= -1 \\
g_{33} &= -1.
\end{aligned}
\end{equation}
We will use the Einstein summation convention, where any repeated upper and lower indexes are considered summed over.  That is \cref{eqn:qftLecture2:380} is written with an implied sum
\begin{equation}\label{eqn:qftLecture2:440}
ds^2 = g_{\mu\nu} dx^\mu dx^\nu.
\end{equation}

Explicit expansion:
\begin{equation}\label{eqn:qftLecture2:460}
\begin{aligned}
ds^2 
&= g_{\mu\nu} dx^\mu dx^\nu \\
&=
g_{00} dx^0 dx^0
+g_{11} dx^1 dx^1
+g_{22} dx^2 dx^2
+g_{33} dx^3 dx^3 \\
&=
(1) dx^0 dx^0
+ (-1) dx^1 dx^1
+ (-1) dx^2 dx^2
+ (-1) dx^3 dx^3.
\end{aligned}
\end{equation}

Recall that rotations (with orthogonal matrix representations) are transformations that leave the dot product unchanged, that is
\begin{equation}\label{eqn:qftLecture2:480}
\begin{aligned}
(R \Bx) \cdot (R \By)
&= \Bx^\T R^\T R \By \\
&= \Bx^\T \By \\
&= \Bx \cdot \By,
\end{aligned}
\end{equation}
%
where \( R \) is a rotation orthogonal 3x3 matrix.  The set of such transformations that leave the dot product unchanged have orthonormal matrix representations \( R^\T R = 1 \).  We call the set of such transformations that have unit determinant the \(\SO{3}\) group.

We call a Lorentz transformation, if it is a linear transformation acting on 4 vectors that leaves the spacetime interval (i.e. the inner product of 4 vectors) invariant.  That is, a transformation that leaves
\begin{equation}\label{eqn:qftLecture2:500}
x^\mu y^\nu g_{\mu\nu} = x^0 y^0 - x^1 y^1 - x^2 y^2 - x^3 y^3
\end{equation}
unchanged.

Suppose that transformation has a 4x4 matrix form
\begin{equation}\label{eqn:qftLecture2:520}
{x'}^\mu = {\Lambda^\mu}_\nu x^\nu
\end{equation}

For an example of a possible \( \Lambda \), consider the transformation sketched in
\cref{fig:Lecture2:Lecture2Fig3}.
\imageFigure{../figures/phy2403-quantum-field-theory/Lecture2Fig3}{Boost transformation.}{fig:Lecture2:Lecture2Fig3}{0.2}
We know that boost has the form
\begin{equation}\label{eqn:qftLecture2:540}
\begin{aligned}
x &= \frac{x' + v t'}{\sqrt{1 - v^2/c^2}} \\
y &= y' \\
z &= z' \\
t &= \frac{t' + (v/c^2) x'}{\sqrt{1 - v^2/c^2}} \\
\end{aligned}
\end{equation}
(this is a boost along the x-axis, not y as I'd drawn),
or
\begin{equation}\label{eqn:qftLecture2:560}
\begin{bmatrix}
c t \\
x \\
y \\
z
\end{bmatrix}
=
\begin{bmatrix}
\inv{\sqrt{1 - v^2/c^2}} & \frac{v/c}{\sqrt{1 - v^2/c^2}} & 0 & 0 \\
\frac{v/c}{\sqrt{1 - v^2/c^2}} & \frac{1}{\sqrt{1 - v^2/c^2}} & 0 & 0 \\
0 & 0 & 1 & 0 \\
0 & 0 & 0 & 1 \\
\end{bmatrix}
\begin{bmatrix}
c t' \\
x' \\
y' \\
z'
\end{bmatrix}
\end{equation}

Other examples include rotations (\({\lambda^0}_0 = 1\) zeros in \( {\lambda^0}_k, {\lambda^k}_0 \), and a rotation matrix in the remainder.)
% submatrix:
%\begin{equation}\label{eqn:qftLecture2:580}
%\begin{bmatrix}
%1 & 0 & 0 & 0
%0 &
%0 &    R
%0 &
%\end{bmatrix}
%\end{equation}

Back to Lorentz transformations (\(\text{SO}(1,3)^+\)), let
\begin{equation}\label{eqn:qftLecture2:600}
\begin{aligned}
{x'}^\mu    &= {\Lambda^\mu}_\nu x^\nu \\
{y'}^\kappa &= {\Lambda^\kappa}_\rho y^\rho
\end{aligned}
\end{equation}

The dot product
\begin{equation}\label{eqn:qftLecture2:620}
\begin{aligned}
g_{\mu \kappa}
{x'}^\mu
{y'}^\kappa
&=
g_{\mu \kappa}
{\Lambda^\mu}_\nu
{\Lambda^\kappa}_\rho
x^\nu
y^\rho \\
&=
g_{\nu\rho}
x^\nu
y^\rho,
\end{aligned}
\end{equation}
where the last step introduces the invariance requirement of the transformation.  That is
%\begin{equation}\label{eqn:qftLecture2:640}
\boxedEquation{eqn:qftLecture2:640}{
g_{\nu\rho}
=
g_{\mu \kappa}
{\Lambda^\mu}_\nu
{\Lambda^\kappa}_\rho.
}
%\end{equation}
\paragraph{Upper and lower indexes}
\index{upper indexes}
\index{lower indexes}
We've defined
\begin{equation}\label{eqn:qftLecture2:660}
x^\mu = (t, x^1, x^2, x^3).
\end{equation}

We could also define a four vector with lower indexes
\begin{equation}\label{eqn:qftLecture2:680}
x_\nu = g_{\nu\mu} x^\mu = (t, -x^1, -x^2, -x^3).
\end{equation}
That is
\begin{equation}\label{eqn:qftLecture2:700}
\begin{aligned}
x_0 &= x^0 \\
x_1 &= -x^1 \\
x_2 &= -x^2 \\
x_3 &= -x^3,
\end{aligned}
\end{equation}
which allows us to write the dot product as simply \( x^\mu y_\mu \).

We can also define a metric tensor with upper indexes
\begin{equation}\label{eqn:qftLecture2:401}
g^{\mu\nu} \sim
\begin{bmatrix}
1 & 0 & 0 & 0 \\
0 & -1 & 0 & 0 \\
0 & 0 & -1 & 0 \\
0 & 0 & 0 & -1 \\
\end{bmatrix}
\end{equation}
This is the inverse matrix of \( g_{\mu\nu} \), and it satisfies
\begin{equation}\label{eqn:qftLecture2:720}
g^{\mu \nu} g_{\nu\rho} = {\delta^\mu}_\rho
\end{equation}

Exercise: Check:
\begin{equation}\label{eqn:qftLecture2:740}
g_{\mu\nu} x^\mu y^\nu = x_\nu y^\nu = x^\nu y_\nu
= g^{\mu\nu} x_\mu y_\nu = {\delta^\mu}_\nu x_\mu y^\nu
\end{equation}

Class ended around this point, but it appeared that we were heading this direction:

Returning to the Lorentz invariant and multiplying both sides of
\cref{eqn:qftLecture2:640} with an inverse Lorentz transformation \( \Lambda^{-1} \), we find
\begin{equation}\label{eqn:qftLecture2:760}
\begin{aligned}
g_{\nu\rho}
{\lr{\Lambda^{-1}}^\rho}_\alpha
&=
g_{\mu \kappa}
{\Lambda^\mu}_\nu
{\Lambda^\kappa}_\rho
{\lr{\Lambda^{-1}}^\rho}_\alpha \\
&=
g_{\mu \kappa}
{\Lambda^\mu}_\nu
{\delta^\kappa}_\alpha \\
&=
g_{\mu \alpha}
{\Lambda^\mu}_\nu,
\end{aligned}
\end{equation}
or
\begin{equation}\label{eqn:qftLecture2:780}
\lr{\Lambda^{-1}}_{\nu \alpha} = \Lambda_{\alpha \nu}.
\end{equation}
This is clearly analogous to \( R^\T = R^{-1} \), although the index notation obscures things considerably.  Prof. Poppitz said that next week this would all lead to showing that the determinant of any Lorentz transformation was \( \pm 1 \).

For what it's worth, it seems to me that this index notation makes life a lot harder than it needs to be, at least for a matrix related question (i.e. determinant of the transformation).  In matrix/column-(4)-vector notation, let \(x' = \Lambda x, y' = \Lambda y\) be two four vector transformations, then
\begin{equation}\label{eqn:qftLecture2:800}
x' \cdot y' = {x'}^T G y' = (\Lambda x)^T G \Lambda y = x^T ( \Lambda^T G \Lambda) y = x^T G y.
\end{equation}
so
\boxedEquation{eqn:qftLecture2:820}{
\Lambda^T G \Lambda = G.
}
Taking determinants of both sides gives
\begin{equation}\label{eqn:qftLecture2LorentzTransformation:n}
-(\det(\Lambda))^2 = -1,
\end{equation}
so \(\det(\Lambda) = \pm 1\).
%}

   %\chapter{Lorentz transformations and a scalar action.}
      %
% Copyright � 2017 Peeter Joot.  All Rights Reserved.
% Licenced as described in the file LICENSE under the root directory of this GIT repository.
%
\input{../latex/blogpost.tex}
\renewcommand{\basename}{qft3}
\renewcommand{\dirname}{notes/phy2403/}
\newcommand{\keywords}{PHY2403H}
\input{../latex/peeter_prologue_print2.tex}

%\usepackage{phy2403}
\usepackage{peeters_braket}
%\usepackage{peeters_layout_exercise}
\usepackage{peeters_figures}
\usepackage{mathtools}
\usepackage{siunitx}

\beginArtNoToc
\generatetitle{PHY2403H Quantum Field Theory.  Lecture 3: XXX.  Taught by Prof.\ Erich Poppitz}
%\chapter{XXX}
\label{chap:qft3}

\paragraph{Disclaimer}

Peeter's lecture notes from class.  These may be incoherent and rough.

These are notes for the UofT course PHY2403H, Quantum Field Theory, taught by Prof. Erich Poppitz, covering \textchapref{{1}} \citep{peskin1995introduction} content.

\paragraph{YYY}

\EndArticle
%\EndNoBibArticle

      \section{Problems.}
         \input{qftLukeProblemSet1Problem1.tex}
   \chapter{Classical field theory.}
      %
% Copyright © 2018 Peeter Joot.  All Rights Reserved.
% Licenced as described in the file LICENSE under the root directory of this GIT repository.
%
\section{Field theory.}

The electrostatic potential is an example of a scalar field \( \phi(\Bx) \) unchanged by \(\SO{3}\) rotations
\begin{equation}\label{eqn:qftLecture3:240}
\Bx \rightarrow \Bx' = O \Bx,
\end{equation}
that is
\begin{dmath}\label{eqn:qftLecture3:260}
\phi'(\Bx') = \phi(\Bx).
\end{dmath}
Here \( \phi'(\Bx') \) is the value of the (electrostatic) scalar potential in a primed frame.

However, the electrostatic field is not invariant under Lorentz transformation.
We postulate that there is some scalar field
\begin{dmath}\label{eqn:qftLecture3:280}
\phi'(x') = \phi(x),
\end{dmath}
where \( x' = \Lambda x \) is an \(\SO{1,3}\) transformation.
There are actually no stable particles (fields that persist at long distances) described by Lorentz scalar fields, although there are some unstable scalar fields such as the
Higgs, Pions, and Kaons.
However,
much of our homework and discussion will be focused on scalar fields, since
they are the
easiest to start with.

We need to first understand how derivatives \( \partial_\mu \phi(x) \) transform.  Using the chain rule
\begin{dmath}\label{eqn:qftLecture3:300}
\PD{x^\mu}{\phi(x)} =
\PD{x^\mu}{\phi'(x')}
=
\PD{{x'}^\nu}{\phi'(x')}
\PD{{x}^\mu}{{x'}^\nu}
=
\PD{{x'}^\nu}{\phi'(x')}
\partial_\mu \lr{
{\Lambda^\nu}_\rho x^\rho
}
=
\PD{{x'}^\nu}{\phi'(x')}
{\Lambda^\nu}_\mu
=
\PD{{x'}^\nu}{\phi(x)}
{\Lambda^\nu}_\mu.
\end{dmath}
Multiplying by the inverse \( {\lr{\Lambda^{-1}}^\mu}_\kappa \) we get
\begin{dmath}\label{eqn:qftLecture3:320}
\PD{{x'}^\kappa}{}
=
{\lr{\Lambda^{-1}}^\mu}_\kappa \PD{x^\mu}{}
\end{dmath}

This should be familiar to you, and is an analogue of the transformation of the
\begin{dmath}\label{eqn:qftLecture3:340}
d\Br \cdot \spacegrad_\Br
=
d\Br' \cdot \spacegrad_{\Br'}.
\end{dmath}

\section{Actions.}
\index{action}
We will start with a classical action, and quantize to determine a QFT.
In mechanics we have the particle position \( q(t) \), which is a classical field in 1+0 time and space dimensions.  Our action is
\begin{dmath}\label{eqn:qftLecture3:360}
S
= \int dt \LL(t)
= \int dt \lr{
\inv{2} \dot{q}^2 - V(q)
}.
\end{dmath}
This action depends on the position of the particle that is local in time.
You could imagine that we have a more complex action where the action depends on future or past times
\begin{dmath}\label{eqn:qftLecture3:380}
S
= \int dt' q(t') K( t' - t ),
\end{dmath}
but we don't seem to find such actions in classical mechanics.

\section{Principles determining the form of the action.}
%\paragraph{Principles determining the form of the action.}
\begin{itemize}
\item relativity (action is invariant under Lorentz transformation)
\item locality (action depends on fields and the derivatives at given \((t, \Bx)\).
\item Gauge principle (the action should be invariant under gauge transformation).  We won't discuss this in detail right now since we will start with studying scalar fields.
Recall that for Maxwell's equations a gauge transformation has the form
\begin{dmath}\label{eqn:qftLecture3:520}
\phi \rightarrow \phi + \dot{\chi}, \BA \rightarrow \BA - \spacegrad \chi.
\end{dmath}
\end{itemize}

Suppose we have a real scalar field \( \phi(x) \) where \( x \in \bbR^{1,d-1} \).  We will be integrating over space and time \( \int dt d^{d-1} x \) which we will write as \( \int d^d x \).  Our action is
\begin{dmath}\label{eqn:qftLecture3:400}
S = \int d^d x \lr{ \text{Some action density to be determined } }
\end{dmath}
The analogue of \( \dot{q}^2 \) is
\begin{dmath}\label{eqn:qftLecture3:420}
\lr{ \PD{x^\mu}{\phi} }
\lr{ \PD{x^\nu}{\phi} }
g^{\mu\nu}
=
(\partial_\mu \phi) (\partial_\nu \phi) g^{\mu\nu}
= \partial^\mu \phi \partial_\mu \phi.
\end{dmath}
This has both time and spatial components, that is
\begin{dmath}\label{eqn:qftLecture3:440}
\partial^\mu \phi \partial_\mu \phi =
\dotphi^2 - (\spacegrad \phi)^2,
\end{dmath}
so the desired simplest scalar action is
\begin{dmath}\label{eqn:qftLecture3:460}
S = \int d^d x \lr{ \dotphi^2 - (\spacegrad \phi)^2 }.
\end{dmath}
The measure transforms using a Jacobian, which we have seen is the Lorentz transform matrix, and has unit determinant
\begin{equation}\label{eqn:qftLecture3:480}
d^d x' = d^d x \Abs{ det( \Lambda^{-1} ) } = d^d x.
\end{equation}


      \section{Problems.}
         %
% Copyright © 2018 Peeter Joot.  All Rights Reserved.
% Licenced as described in the file LICENSE under the root directory of this GIT repository.
%
%\section{Problems.}

%\makeproblem{Matrix elements of Lorentz/metric product.}{problem:qftLecture3:520}{
%Justify \cref{eqn:qftLecture3:220} explicitly.
%} % problem
%
%\makeanswer{problem:qftLecture3:520}{
%Fixme.
%} % answer
%
\makeproblem{Four vector form of the Maxwell gauge transformation.}{problem:qftLecture3:540}{
Show that the transformation
\begin{equation}\label{eqn:qftLecture3:580}
A^\mu \rightarrow A^\mu + \partial^\mu \chi
\end{equation}
is the desired four-vector form of the gauge transformation \cref{eqn:qftLecture3:520}, that is
\begin{equation}\label{eqn:qftLecture3:540}
j^\nu = \partial_\mu {F'}^{\mu\nu}
= \partial_\mu F^{\mu\nu}.
\end{equation}
Also relate this four-vector gauge transformation to the spacetime split.
} % problem

\makeanswer{problem:qftLecture3:540}{
\begin{equation}\label{eqn:qftLecture3:560}
\partial_\mu {F'}^{\mu\nu}
=
\partial_\mu \lr{ \partial^\mu {A'}^\nu - \partial_\nu {A'}^\mu }
=
\partial_\mu \lr{
  \partial^\mu \lr{ A^\nu + \partial^\nu \chi }
- \partial_\nu \lr{ A^\mu + \partial^\mu \chi }
}
=
\partial_\mu {F}^{\mu\nu}
+
\partial_\mu \partial^\mu \partial^\nu \chi
-
\partial_\mu \partial^\nu \partial^\mu \chi
=
\partial_\mu {F}^{\mu\nu},
\end{equation}
by equality of mixed partials.
Expanding \cref{eqn:qftLecture3:580} explicitly we find
\begin{equation}\label{eqn:qftLecture3:600}
{A'}^\mu = A^\mu + \partial^\mu \chi,
\end{equation}
which is
\begin{equation}\label{eqn:qftLecture3:620}
\begin{aligned}
\phi' = {A'}^0 &= A^0 + \partial^0 \chi = \phi + \dot{\chi} \\
\BA' \cdot \Be_k = {A'}^k &= A^k + \partial^k \chi = \lr{ \BA - \spacegrad \chi } \cdot \Be_k.
\end{aligned}
\end{equation}
The last of which can be written in vector notation as \( \BA' = \BA - \spacegrad \chi \).
} % answer


         %
% Copyright � 2015 Peeter Joot.  All Rights Reserved.
% Licenced as described in the file LICENSE under the root directory of this GIT repository.
%
\makeoproblem{One dimensional string.}
{qft:LukeProblemSet1:3}
{2015 ps1.3}
{

A string of length \( a \), mass per unit length \( \sigma \) and under tension \(T\) is fixed at each end. The Lagrangian governing the time evolution of the transverse displacement \( y(x,t) \) is
\begin{equation}\label{eqn:LukeProblemSet1Problem3:20}
L = \int_0^a dx \lr{ \frac{\sigma}{2} \lr{ \PD{t}{y} }^2 - \frac{T}{2} \lr{ \PD{x}{y} }^2 },
\end{equation}
where \( x \) identifies position along the string from one end point.
\makesubproblem{}{qft:LukeProblemSet1:3a}
By expressing the displacement as a sine series Fourier expansion of the form
\begin{equation}\label{eqn:LukeProblemSet1Problem3:40}
y(x,t) = \sqrt{\frac{2}{a}} \sum_{n=1}^\infty \sin\lr{ \frac{ n \pi x }{a} } q_n(t).
\end{equation}

Show that the Lagrangian becomes
\begin{equation}\label{eqn:LukeProblemSet1Problem3:60}
L = \sum_{n=1}^\infty \lr{ \frac{\sigma}{2} \dot{q}_n^2 - \frac{T}{2} \lr{ \frac{ n \pi }{2} }^2 q_n^2 } .
\end{equation}
\makesubproblem{}{qft:LukeProblemSet1:3b}
Derive the equations of motion. Hence, show that the string is equivalent to an infinite set of decoupled
harmonic oscillators, and find their frequencies.
} % makeproblem
\makeanswer{qft:LukeProblemSet1:3}{
\withproblemsetsParagraph{
\makeSubAnswer{}{qft:LukeProblemSet1:3a}
First observe that the functions \( \braket{x}{n} = \sqrt{\frac{2}{a}} \sin\lr{ n \pi x/a } \) are orthonormal over the \( [0,a] \) domain.
\begin{equation}\label{eqn:qftProblemSet1Problem3:80}
\begin{aligned}
\braket{ n }{n}
&=
\frac{2}{a}
\int_0^a \sin^2\lr{ n \pi x/a } dx \\
&=
2
\int_0^1 \sin^2\lr{ n \pi u } du \\
&=
\int_0^1 \lr{ 1 - \cos\lr{ 2 n \pi u } } du \\
&=
1,
\end{aligned}
\end{equation}
and for \( n \ne m \)
\begin{equation}\label{eqn:qftProblemSet1Problem3:100}
\begin{aligned}
\braket{n}{m}
&=
\frac{2}{a}
\int_0^a \sin\lr{ n \pi x/a } \sin\lr{ m \pi x/a } dx \\
&=
2
\int_0^1 \sin\lr{ n \pi u } \sin\lr{ m \pi u } du \\
&=
-\inv{2}
\int_0^1
\lr{ e^{i n \pi u} - e^{-i n \pi u} }
\lr{ e^{i m \pi u} - e^{-i m \pi u} }
du \\
&=
-
\int_0^1
du
\lr{
\cos( ( n + m) \pi u ) - \cos( (m - n) \pi u )
} \\
&= 0,
\end{aligned}
\end{equation}
so
\begin{equation}\label{eqn:qftProblemSet1Problem3:120}
\begin{aligned}
L
&=
\int_0^a dx
\frac{2}{a}
\sum_{m,n = 1}^\infty
\sin\lr{ \frac{ n \pi x }{a} }
\sin\lr{ \frac{ m \pi x }{a} }
\lr{ \frac{\sigma}{2} \dot{q}_n \dot{q}_m
- \frac{T}{2}
\lr{ \frac{n \pi}{a} }
\lr{ \frac{m \pi}{a} } q_n q_m
}
\\&=
\sum_{m,n = 1}^\infty \delta_{nm}
\lr{ \frac{\sigma}{2} \dot{q}_n \dot{q}_m
- \frac{T}{2}
\lr{ \frac{n \pi}{a} }
\lr{ \frac{m \pi}{a} }
q_n q_m
}
\\&=
\sum_{n = 1}^\infty
\lr{ \frac{\sigma}{2} \lr{\dot{q}_n}^2
- \frac{T}{2}
\lr{ \frac{n \pi}{a} }^2 q_n^2
}.
\end{aligned}
\end{equation}
\makeSubAnswer{}{qft:LukeProblemSet1:3b}
We have an Euler-Lagrange equation for each \( q_n \).  The conjugate momenta are
\begin{equation}\label{eqn:qftProblemSet1Problem3:140}
\PD{\dot{q}_n}{L} = \sigma \dot{q}_n.
\end{equation}

We also have
\begin{equation}\label{eqn:qftProblemSet1Problem3:160}
\PD{q_n}{L} = - T \lr{ \frac{n \pi}{a} }^2 q_n,
\end{equation}
%
so we have
\begin{equation}\label{eqn:qftProblemSet1Problem3:180}
\ddot{q}_n = - \frac{T}{\sigma} \lr{ \frac{n \pi}{a} }^2 q_n.
\end{equation}

These have solutions
\begin{equation}\label{eqn:qftProblemSet1Problem3:200}
q_n(t) = A_{\pm} \exp\lr{ \pm i \sqrt{ \frac{T}{\sigma} } \frac{n \pi}{a} t }.
\end{equation}

The angular frequencies are
\begin{equation}\label{eqn:qftProblemSet1Problem3:220}
\omega_n = 2 \pi \nu_n = \sqrt{ \frac{T}{\sigma} } \frac{n \pi}{a},
\end{equation}
%
so the frequencies are
\begin{equation}\label{eqn:qftProblemSet1Problem3:240}
\nu_n = \sqrt{ \frac{T}{\sigma} } \frac{n }{2 a}.
\end{equation}
}
}

         \input{qftLukeProblemSet1Problem6.tex}
   \chapter{Scalar action, least action principle, Euler-Lagrange equations for a field, canonical quantization.}
      %
% Copyright � 2018 Peeter Joot.  All Rights Reserved.
% Licenced as described in the file LICENSE under the root directory of this GIT repository.
%
%{
%%%\input{../latex/blogpost.tex}
%%%\renewcommand{\basename}{qftLecture4}
%%%\renewcommand{\dirname}{notes/phy2403/}
%%%\newcommand{\keywords}{PHY2403H}
%%%\input{../latex/peeter_prologue_print2.tex}
%%%
%%%%\usepackage{phy2403}
%%%\usepackage{peeters_braket}
%%%%\usepackage{peeters_layout_exercise}
%%%\usepackage{peeters_figures}
%%%\usepackage{mathtools}
%%%\usepackage{siunitx}
%%%\usepackage{macros_cal}
%%%
%%%\beginArtNoToc
%%%\generatetitle{PHY2403H Quantum Field Theory.  Lecture 4: Scalar action, least action principle, Euler-Lagrange equations for a field, canonical quantization.  Taught by Prof.\ Erich Poppitz}
%\chapter{Scalar action, least action principle, Euler-Lagrange equations for a field, canonical quantization.}
\index{action}
\index{least action principle}
\index{Euler-Lagrange equations!field}
\index{canonical quantization}
\label{chap:qftLecture4}
%%%
%%%\paragraph{DISCLAIMER: Very rough notes from class.  Some additional side notes, but otherwise barely edited.}
%%%
%%%These are notes for the UofT course PHY2403H, Quantum Field Theory I, taught by Prof. Erich Poppitz fall 2018.
%%%%, covering \textchapref{{1}} \citep{peskin1995introduction} content.
%%%
\section{Principles (cont.)}

\begin{itemize}
\item Lorentz (Poincar\'e : Lorentz and spacetime translations)
\item locality
\item dimensional analysis
\item gauge invariance
\end{itemize}

These are the requirements for an action.  We postulated an action that had the form
\begin{dmath}\label{eqn:qftLecture4:20}
\int d^d x \partial_\mu \phi \partial^\mu \phi,
\end{dmath}
called the ``Kinetic term'', which mimics \( \int dt \dot{q}^2 \) that we'd see in quantum or classical mechanics.  In principle there exists an infinite number of local Poincar\'e invariant terms that we can write.  Examples:

\begin{itemize}
\item \( \partial_\mu \phi \partial^\mu \phi \)
\item \( \partial_\mu \phi \partial_\nu \partial^\nu \partial^\mu \phi \)
\item \( \lr{\partial_\mu \phi \partial^\mu \phi}^2 \)
\item \( f(\phi) \partial_\mu \phi \partial^\mu \phi \)
\item \( f(\phi, \partial_\mu \phi \partial^\mu \phi) \)
\item \( V(\phi) \)
\end{itemize}

It turns out that nature (i.e. three spatial dimensions and one time dimension) is described by a finite number of terms.  We will now utilize dimensional analysis to determine some of the allowed forms of the action for scalar field theories in \( d = 2, 3, 4, 5 \) dimensions.  Even though the real world is only \( d = 4 \), some of the \( d < 4 \) theories are relevant in condensed matter studies, and \( d = 5 \) is just for fun (but also applies to string theories.)

With \( [x] \sim \inv{M} \) in natural units, we must define \([\phi]\) such that the kinetic term is dimensionless in d spacetime dimensions

\begin{dmath}\label{eqn:qftLecture4:40}
\begin{aligned}
[d^d x] &\sim \inv{M^d} \\
[\partial_\mu] &\sim M
\end{aligned}
\end{dmath}

so it must be that
\begin{dmath}\label{eqn:qftLecture4:60}
[\phi] = M^{(d-2)/2}
\end{dmath}

It will be easier to characterize the dimensionality of any given term by the power of the mass units, that is

\begin{dmath}\label{eqn:qftLecture4:80}
\begin{aligned}
[\text{mass}] &= 1 \\
[d^d x] &= -d \\
[\partial_\mu] &= 1 \\
[\phi] &= (d-2)/2 \\
[S] &= 0.
\end{aligned}
\end{dmath}
Since the action is
\begin{dmath}\label{eqn:qftLecture4:100}
S = \int d^d x \lr{ \LL(\phi, \partial_\mu \phi) },
\end{dmath}
and because action had dimensions of \( \Hbar \), so in natural units, it must be dimensionless, the Lagrangian density dimensions must be \( [d] \).  We will abuse language in QFT and call the Lagrangian density the Lagrangian.

\subsection{\( d = 2 \).}

Because \( [\partial_\mu \phi \partial^\mu \phi ] = 2 \), the scalar field must be dimension zero, or in symbols
\begin{dmath}\label{eqn:qftLecture4:120}
[\phi] = 0.
\end{dmath}
This means that introducing any function \( f(\phi) = 1 + a \phi + b\phi^2 + c \phi^3 + \cdots \) is also dimensionless, and
\begin{dmath}\label{eqn:qftLecture4:140}
[f(\phi) \partial_\mu \phi \partial^\mu \phi ] = 2,
\end{dmath}
for any \( f(\phi) \).  Another implication of this is that the a potential term in the Lagrangian \( [V(\phi)] = 0 \) needs a coupling constant of dimension 2.  Letting \( \mu \) have mass dimensions, our Lagrangian must have the form
\begin{dmath}\label{eqn:qftLecture4:160}
f(\phi) \partial_\mu \phi \partial^\mu \phi + \mu^2 V(\phi).
\end{dmath}
An infinite number of coupling constants of positive mass dimensions for \( V(\phi) \) are also allowed.  If we have higher order derivative terms, then we need to compensate for the negative mass dimensions.   Example (still for \( d = 2 \)).
\begin{dmath}\label{eqn:qftLecture4:180}
\LL =
f(\phi) \partial_\mu \phi \partial^\mu \phi + \mu^2 V(\phi) + \inv{{\mu'}^2}\partial_\mu \phi \partial_\nu \partial^\nu \partial^\mu \phi + \lr{ \partial_\mu \phi \partial^\mu \phi }^2 \inv{\tilde{\mu}^2}.
\end{dmath}
The last two terms, called \underline{couplings} (i.e. any non-kinetic term), are examples of terms with negative mass dimension.  There is an infinite number of those in any theory in any dimension.

\paragraph{Definitions}

\begin{itemize}
\item Couplings that are dimensionless are called (classically) marginal.
\item Couplings that have positive mass dimension are called (classically) relevant.
\item Couplings that have negative mass dimension are called (classically) irrelevant.
\end{itemize}

In QFT we are generally interested in the couplings that are measurable at long distances for some given energy.  Classically irrelevant theories are generally not interesting in \( d > 2 \), so we are very lucky that we don't live in three dimensional space.  This means that we can get away with a finite number of classically marginal and relevant couplings in 3 or 4 dimensions.  This was mentioned in the Wilczek's article referenced in the class forum \citep{wilczek2007fundamental}\footnote{There's currently more in that article that I don't understand than I do, so it is hard to find it terribly illuminating.}

Long distance physics in any dimension is described by the marginal and relevant couplings.  The irrelevant couplings die off at low energy.  In two dimensions, a priori, an infinite number of marginal and relevant couplings are possible.  2D is a bad place to live!

\subsection{\( d = 3 \).}

Now we have
\begin{dmath}\label{eqn:qftLecture4:200}
[\phi] = \inv{2}
\end{dmath}
so that
\begin{dmath}\label{eqn:qftLecture4:220}
[\partial_\mu \phi \partial^\mu \phi] = 3.
\end{dmath}

A 3D Lagrangian could have local terms such as
\begin{dmath}\label{eqn:qftLecture4:240}
\LL = \partial_\mu \phi \partial^\mu \phi + m^2 \phi^2 + \mu^{3/2} \phi^3 + \mu' \phi^4
+ \lr{\mu''}{1/2} \phi^5
+ \lambda \phi^6.
\end{dmath}
where \( m, \mu, \mu'' \) all have mass dimensions, and \( \lambda \) is dimensionless.  i.e.
\( m, \mu, \mu'' \) are relevant, and \( \lambda \) marginal.  We stop at the sixth power, since any power after that will be irrelevant.

\subsection{\( d = 4 \).}

Now we have
\begin{dmath}\label{eqn:qftLecture4:260}
[\phi] = 1
\end{dmath}
so that
\begin{dmath}\label{eqn:qftLecture4:280}
[\partial_\mu \phi \partial^\mu \phi] = 4.
\end{dmath}

In this number of dimensions \( \phi^k \partial_\mu \phi \partial^\mu \) is an irrelevant coupling.

A 4D Lagrangian could have local terms such as
\begin{dmath}\label{eqn:qftLecture4:300}
\LL = \partial_\mu \phi \partial^\mu \phi + m^2 \phi^2 + \mu \phi^3 + \lambda \phi^4.
\end{dmath}
where \( m, \mu \) have mass dimensions, and \( \lambda \) is dimensionless.  i.e.
\( m, \mu \) are relevant, and \( \lambda \) is marginal.

\subsection{\( d = 5 \).}

Now we have
\begin{dmath}\label{eqn:qftLecture4:320}
[\phi] = \frac{3}{2},
\end{dmath}
so that
\begin{dmath}\label{eqn:qftLecture4:340}
[\partial_\mu \phi \partial^\mu \phi] = 5.
\end{dmath}

A 5D Lagrangian could have local terms such as
\begin{dmath}\label{eqn:qftLecture4:360}
\LL = \partial_\mu \phi \partial^\mu \phi + m^2 \phi^2 + \sqrt{\mu} \phi^3 + \inv{\mu'} \phi^4.
\end{dmath}
where \( m, \mu, \mu' \) all have mass dimensions.  In 5D there are no marginal couplings.  Dimension 4 is the last dimension where marginal couplings exist.  In condensed matter physics 4D is called the ``upper critical dimension''.

From the point of view of particle physics, all the terms in the Lagrangian must be the ones that are relevant at long distances.

\section{Least action principle.}
%\section{Least action principle (classical field theory).}
\index{least action principle}
\index{classical field theory}

Now we want to study 4D scalar theories.  We have some action
\begin{dmath}\label{eqn:qftLecture4:380}
S[\phi] = \int d^4 x \LL(\phi, \partial_\mu \phi).
\end{dmath}

Let's keep an example such as the following in mind
\begin{dmath}\label{eqn:qftLecture4:400}
\LL = \underbrace{\inv{2} \partial_\mu \phi \partial^\mu \phi}_{\text{Kinetic term}} - \underbrace{m^2 \phi - \lambda \phi^4}_{\text{all relevant and marginal couplings}}.
\end{dmath}
The even powers can be justified by assuming there is some symmetry that kills the odd powered terms.

We will be integrating over a space time region such as that depicted in \cref{fig:spacetimeCylinder:spacetimeCylinderFig1},
\imageFigure{../figures/phy2403-quantum-field-theory/spacetimeCylinderFig1}{Cylindrical spacetime boundary.}{fig:spacetimeCylinder:spacetimeCylinderFig1}{0.3}
where a cylindrical spatial cross section is depicted that we allow to tend towards infinity.  We demand that the field is fixed on the infinite spatial boundaries.  The easiest way to demand that the field dies off on the spatial boundaries, that is
\begin{dmath}\label{eqn:qftLecture4:420}
\lim_{\Abs{\Bx} \rightarrow \infty} \phi(\Bx) \rightarrow 0.
\end{dmath}
The functional \( \phi(\Bx, t) \) that obeys the boundary condition as stated extremizes \( S[\phi] \).

Extremizing the action means that we seek \( \phi(\Bx, t) \)
\begin{equation}\label{eqn:qftLecture4:440}
\delta S[\phi] = 0 = S[\phi + \delta \phi] - S[\phi].
\end{equation}

How do we compute the variation?
\begin{dmath}\label{eqn:qftLecture4:460}
\delta S = \int d^d x \lr{ \LL(\phi + \delta \phi, \partial_\mu \phi + \partial_\mu \delta \phi) - \LL(\phi, \partial_\mu \phi) }
= \int d^d x \lr{ \PD{\phi}{\LL} \delta \phi + \PD{(\partial_\mu \phi)}{\LL} (\partial_\mu \delta \phi) }
= \int d^d x \lr{ \PD{\phi}{\LL} \delta \phi
+ \partial_\mu \lr{ \PD{(\partial_\mu \phi)}{\LL} \delta \phi}
- \lr{ \partial_\mu \PD{(\partial_\mu \phi)}{\LL} } \delta \phi
}
=
\int d^d x
\delta \phi
\lr{ \PD{\phi}{\LL}
- \partial_\mu \PD{(\partial_\mu \phi)}{\LL} }
+ \int d^3 \sigma_\mu \lr{ \PD{(\partial_\mu \phi)}{\LL} \delta \phi }
\end{dmath}

If we are explicit about the boundary term, we write it as
\begin{dmath}\label{eqn:qftLecture4:480}
\begin{aligned}
\int &dt d^3 \Bx \lr{
\partial_t
\lr{ \PD{(\partial_t \phi)}{\LL} \delta \phi }
- \spacegrad \cdot \lr{ \PD{(\spacegrad \phi)}{\LL} \delta \phi } }
\\
&=
\int d^3 \Bx \evalrange{ \PD{(\partial_t \phi)}{\LL} \delta \phi }{t = -T}{t = T}
- \int dt d^2 \BS \cdot \lr{ \PD{(\spacegrad \phi)}{\LL} \delta \phi }.
\end{aligned}
\end{dmath}
but \( \delta \phi = 0 \) at \( t = \pm T \) and also at the spatial boundaries of the integration region.

This leaves
\begin{dmath}\label{eqn:qftLecture4:500}
\delta S[\phi] = \int d^d x \delta \phi
\lr{ \PD{\phi}{\LL} - \partial_\mu \PD{(\partial_\mu \phi)}{\LL} } = 0 \forall \delta \phi.
\end{dmath}
That is
%\begin{dmath}\label{eqn:qftLecture4:520}
\boxedEquation{eqn:qftLecture4:540}{
\PD{\phi}{\LL} - \partial_\mu \PD{(\partial_\mu \phi)}{\LL} = 0.
}
%\end{dmath}
This is the Euler-Lagrange equations for a single scalar field.

Returning to our sample scalar Lagrangian
\begin{dmath}\label{eqn:qftLecture4:560}
\LL = \inv{2} \partial_\mu \phi \partial^\mu \phi - \inv{2} m^2 \phi^2 - \frac{\lambda}{4} \phi^4.
\end{dmath}
This example is related to the Ising model which has a \( \phi \rightarrow -\phi \) symmetry.
\index{Ising model}
Applying the Euler-Lagrange equations, we have
\begin{dmath}\label{eqn:qftLecture4:580}
\PD{\phi}{\LL} = -m^2 \phi - \lambda \phi^3,
\end{dmath}
and
\begin{dmath}\label{eqn:qftLecture4:600}
\PD{(\partial_\mu \phi)}{\LL}
=
\PD{(\partial_\mu \phi)}{} \lr{
\inv{2} \partial_\nu \phi \partial^\nu \phi }
=
\inv{2} \partial^\nu \phi
\PD{(\partial_\mu \phi)}{}
\partial_\nu \phi
+
\inv{2} \partial_\nu \phi
\PD{(\partial_\mu \phi)}{}
\partial_\alpha \phi g^{\nu\alpha}
=
\inv{2} \partial^\mu \phi
+
\inv{2} \partial_\nu \phi g^{\nu\mu}
=
\partial^\mu \phi
\end{dmath}
so we have
\begin{dmath}\label{eqn:qftLecture4:620}
0
=
\PD{\phi}{\LL} -\partial_\mu
\PD{(\partial_\mu \phi)}{\LL}
=
-m^2 \phi - \lambda \phi^3 - \partial_\mu \partial^\mu \phi.
\end{dmath}

For \( \lambda = 0 \), the free field theory limit, this is just
\begin{dmath}\label{eqn:qftLecture4:640}
\partial_\mu \partial^\mu \phi + m^2 \phi = 0.
\end{dmath}
Written out from the observer frame, this is
\begin{dmath}\label{eqn:qftLecture4:660}
\partial_{tt} \phi - \spacegrad^2 \phi + m^2 \phi = 0.
\end{dmath}

With a non-zero mass term
\begin{dmath}\label{eqn:qftLecture4:680}
\lr{ \partial_{tt} - \spacegrad^2  + m^2 } \phi = 0,
\end{dmath}
is called the Klein-Gordan equation.

If we also had \( m = 0 \) we'd have
\begin{dmath}\label{eqn:qftLecture4:700}
\lr{ \partial_{tt} - \spacegrad^2 } \phi = 0,
\end{dmath}
which is the wave equation (for a massless free field).  This is also called the D'Alembert equation, which is familiar from electromagnetism where we have
\begin{dmath}\label{eqn:qftLecture4:720}
\begin{aligned}
\lr{ \partial_{tt} - \spacegrad^2 } \BE &= 0 \\
\lr{ \partial_{tt} - \spacegrad^2 } \BB &= 0,
\end{aligned}
\end{dmath}
in a source free region.


   \chapter{Klein-Gordon equation, SHOs, momentum space representation, raising and lowering operators.}
      %
% Copyright � 2017 Peeter Joot.  All Rights Reserved.
% Licenced as described in the file LICENSE under the root directory of this GIT repository.
%
\input{../latex/blogpost.tex}
\renewcommand{\basename}{qft5}
\renewcommand{\dirname}{notes/phy2403/}
\newcommand{\keywords}{PHY2403H}
\input{../latex/peeter_prologue_print2.tex}

%\usepackage{phy2403}
\usepackage{peeters_braket}
%\usepackage{peeters_layout_exercise}
\usepackage{peeters_figures}
\usepackage{mathtools}
\usepackage{siunitx}
\usepackage{macros_cal} % LL

\beginArtNoToc
\generatetitle{PHY2403H Quantum Field Theory.  Lecture 5: XXX.  Taught by Prof.\ Erich Poppitz}
%\chapter{XXX}
\label{chap:qft5}

\paragraph{Disclaimer}

\paragraph{DISCLAIMER: Very rough notes from class.  Some additional side notes, but otherwise barely edited.}

These are notes for the UofT course PHY2403H, Quantum Field Theory I, taught by Prof. Erich Poppitz fall 2018.
%, covering \textchapref{{1}} \citep{peskin1995introduction} content.

\section{Canonical quantization}

\begin{dmath}\label{eqn:qftLecture5:20}
L = \int d^3 x
\lr{
\inv{2} \lr{\partial_0 \phi}^2 - \inv{2} \lr{\spacegrad \phi}^2 - \frac{m^2}{2} \phi^2  - \frac{\lambda}{4} \phi^4
}
\end{dmath}

\begin{dmath}\label{eqn:qftLecture5:40}
S = \int dt L = \int dt d^3 x \LL
\end{dmath}

\begin{dmath}\label{eqn:qftLecture5:60}
\Pi(\Bx, t) = \frac{\delta \LL}{\delta \phidot(\Bx, t) } = \PD{\phidot(\Bx, t)}{\LL}
\end{dmath}

\begin{dmath}\label{eqn:qftLecture5:80}
H = \int d^3 x \lr{ \Pi(\Bx, t) \phidot(\Bx, t) - \LL }
= \int d^3 x
\lr{ \Pi^2 /2 + (\spacegrad phi)^2 + \inv{2} m^2 \phi^2 + \frac{\lambda}{4} \phi^4 }
\end{dmath}

\begin{dmath}\label{eqn:qftLecture5:100}
H = \int d^3 x \calH(\Bx, t)
\end{dmath}
\begin{dmath}\label{eqn:qftLecture5:120}
\calH(\Bx, t) =
\lr{ \Pi^2 /2 + (\spacegrad phi)^2 + \inv{2} m^2 \phi^2 + \frac{\lambda}{4} \phi^4 }
\end{dmath}

\paragraph{Canonical Commutation Relations (CCR)}:

\begin{dmath}\label{eqn:qftLecture5:140}
\antisymmetric{\hat{\Pi}(\Bx, t)}{\hat{\phi}(\By, t)} = -i \delta^3 (\Bx - \By)
\end{dmath}

This is in analogy to

\begin{dmath}\label{eqn:qftLecture5:160}
\antisymmetric{\hat{p}_i}{\hat{q}_j} = -i \delta_{ij},
\end{dmath}

To choose a representation, we may map the \( \Psi \) of QM \( \rightarrow \) to a wave functional \( \Psi[\phi] \)

\begin{dmath}\label{eqn:qftLecture5:180}
\hat{\phi}(\By, t) \Psi[\phi] = \phi(\By, t) \Psi[\phi]
\end{dmath}

This is similar to the QM wave functions

\begin{dmath}\label{eqn:qftLecture5:200}
\hat{q}_i \Psi(\setlr{q}) = q_i \Psi(q)
\hat{p}_i \Psi(\setlr{q}) =
-i \PD{q_i}{}
\Psi(p)
\end{dmath}

\begin{dmath}\label{eqn:qftLecture5:220}
\hat{\Pi}(\Bx, t) = -i \frac{\delta}{\delta \phi(\Bx, t)}
\end{dmath}

To quantize the Hamiltonian we just add hats

\begin{dmath}\label{eqn:qftLecture5:240}
\calH(\Bx, t)
=
\lr{ \hat{\Pi}^2 /2 + (\spacegrad \hat{\phi})^2 + \inv{2} m^2 \hat{\phi}^2 + \frac{\lambda}{4} \hat{\phi}^4 }
\end{dmath}

Recall the QM SHO
\begin{dmath}\label{eqn:qftLecture5:260}
\hat{H} = \inv{2} \hat{p}^2 + \inv{2} \omega^2 \hat{q}^2,
\end{dmath}
where
\begin{dmath}\label{eqn:qftLecture5:280}
\antisymmetric{\hat{p}}{\hat{q}} = -i
\end{dmath}

Recall the Heisenberg picture time evoluation operators

\begin{dmath}\label{eqn:qftLecture5:300}
\ddt{\hat{p}}
= i \antisymmetric{\hat{H}}{\hatp}
= i \frac{\omega^2}{2} \antisymmetric{\hatq^2}{\hatp} = i \omega^2 \hatq i = -i \omega^2 \hatq
\end{dmath}

\begin{dmath}\label{eqn:qftLecture5:320}
\ddt{\hat{q}}
= i \antisymmetric{\hat{H}}{\hatq} = i \inv{2} \antisymmetric{\hatp^2}{\hatq} = i(-i)\hatp = \hatp
\end{dmath}

so
\begin{dmath}\label{eqn:qftLecture5:340}
\ddot{\hat{q}} = \dot{\hat{p}} = - \omega^2 \hatq
\end{dmath}

We see that the Heisenberg operators obey the classical equations of motion.
Now we want to try this with the quantized QFT we've started with

\begin{dmath}\label{eqn:qftLecture5:360}
\dot{\hat{\Pi}}(\Bx, t)
= i \antisymmetric{\hatH}{\hat{\Pi}(\Bx, t)}
=
i \int d^3 \lr{ \inv{2} \antisymmetric{ \spacegrad \phi(\By) }^2 }{\Pihat{\Bx) }
+
i \int d^3 \lr{ \frac{m^2}{2} \antisymmetric{ \hatphi(\By)^2 }{\Pihat{\Bx) }
+
i \frac{\lambda}{4} \int d^3 \antisymmetric{ \hatphi(\By)^4 }{\Pihat{\Bx) }
=
i \int d^3 y \spacegrad \phi(\By) \antisymmetric{\spacegrad \phi(\By)}{\Pi(\Bx)}
+
i m^2 \int d^3 y \spacegrad \phi(\By) (+i) \delta^3(\Bx - \By)
+
i \frac{\lambda}{4} \int d^3 y 4 \phi(\By)^3 (+i) \delta^3(\Bx - \By)
=
i \int d^3 y \spacegrad \phi(\By) (+i) \spacegrad_\By \delta^3(\Bx - \By)
+
i m^2 \int d^3 y \spacegrad \phi(\By) (+i) \delta^3(\Bx - \By)
+
i \frac{\lambda}{4} \int d^3 y 4 \phi(\By)^3 (+i) \delta^3(\Bx - \By)
=
\lr{ -m^2 \phi - \lambda \phihat^3 + \spacegrad^2 \hatphi }(\Bx).
\end{dmath}
\begin{dmath}\label{eqn:qftLecture5:380}
\dot{\hat{\Pi}}(\Bx, t)
= i \antisymmetric{\hatH}{\hat{\phi}(\Bx, t)}
= i \inv{2} \int d^3 y \antisymmetric{\Pihat^2(\By)}{\hat{\phi}(\Bx)}
= \Pihat(\Bx)
\end{dmath}
\begin{dmath}\label{eqn:qftLecture5:400}
\ddot{\hat{\Pi}}(\Bx, t)
%= i \antisymmetric{\hatH}{\dot{\hat{\phi}}(\Bx, t)}
= \ddot{\Pihat}
=
\spacegrad^2 \hatphi
-m^2 \phi - \lambda \phihat^3.
\end{dmath}
That is
\begin{dmath}\label{eqn:qftLecture5:420}
\ddot{\phicap} - \spacegrad^2 \phihat + m^2 \phihat + \lambda \phihat^3 = 0,
\end{dmath}
which is the classical Euler-Lagrange equation, also obeyed by the
Heisenburg operator \( \phi(\Bx, t) \).  When \( \lambda = 0 \) this is the Klein-Gordon equation.

Dropping hats, let's consider the momentum space representation of our operators

\begin{dmath}\label{eqn:qftLecture5:440}
\phi(\Bx, t) = \int \frac{d^3 p}{(2\pi)^3} e^{i \Bp \cdot \Bx} \tilde{\phi}(\Bp, t)
\end{dmath}

\begin{dmath}\label{eqn:qftLecture5:460}
\phi^\conj\Bx, t) = \phi(\Bx, t) \leftrightarrow \tilde{phi}(\Bp, t) = \tilde{\phi}^\conj(-\Bp, t)
\end{dmath}

\begin{dmath}\label{eqn:qftLecture5:480}
\tilde{phi}(\Bp, t)
= \int d^3 x e^{-i \Bp \cdot \Bx} \phi(\Bx, t)
= \int d^3 x e^{-i \Bp \cdot \Bx} \int \frac{d^3 q}{(2 \pi)^3} e^{i \Bq \cdot \Bx} \tilde{\phi}(\Bq, t)
\end{dmath}

so

\boxedEquation{eqn:qftLecture5:500}{
\int d^3 x e^{i \BA \cdot \Bx} = (2 \pi)^3 \delta^3(\BA)
}

Wnt the EOM for \( \tilde{\phi}(\Bp, t) \) where the operator obeys the KG equation

\begin{dmath}\label{eqn:qftLecture5:520}
\lr{ \partial_t^2 - \spacegrad^2 + m^2 } \phi(\Bx, t) = 0
\end{dmath}

Inserting the transform relation \cref{eqn:qftLecture5:440} we get

Fixme: WORK OUT:
\begin{dmath}\label{eqn:qftLecture5:n}
\int \frac{d^3 p}{(2 \pi)^3} e^{i \Bp \cdot \Bx} \lr{
\ddot{\tphi}(\Bp, t) + \lr( \Bp^2 + m^2 } \tphi(\Bp, t) } = 0
\end{dmath}

With
\begin{dmath}\label{eqn:qftLecture5:n}
\omega_\Bq = \sqrt{ \Bq^2 + m^2 }
\end{dmath}
we find
\boxedEquation{eqn:qftLecture5:n}{
\ddot{\tphi}(\Bq, t) = - \omega_\Bq^2 \tphi(\Bq, t).
}
The Fourier components of the HP operators are SHOs!

As we have SHO's and know how to deal with these in QM, we use the same strategy, introducing raising and lowering operators
\begin{dmath}\label{eqn:qftLecture5:n}
\tphi(\Bq, t) = \inv{\sqrt{2 \omega_q}} \lr{ e^{-i \omega_\Bq t } a_\Bq + e^{i \omega_\Bq t} a^\conj_{-\Bq}
}
\end{dmath}

CHECK!! :
\begin{dmath}\label{eqn:qftLecture5:n}
\tphi^\conj(\Bq, t) = \tphi(-\Bq, t).
\end{dmath}

We will find (Wednesday) that
\begin{dmath}\label{eqn:qftLecture5:n}
\antisymmetric{\hata_\Bq}{\hata^+_\Bp} = \delta^3(\Bp - \Bq) (2 \pi)^3.
\end{dmath}

These are equivalent to 
\begin{dmath}\label{eqn:qftLecture5:n}
\antisymmetric{\hatPi(\By, t)}{\hatphi(\Bx, t)} = -i \delta^3(\Bx - \By)
\end{dmath}

\EndArticle
%\EndNoBibArticle

   \chapter{Canonical quantization, Simple Harmonic Oscillators, Symmetries.}
      %
% Copyright � 2017 Peeter Joot.  All Rights Reserved.
% Licenced as described in the file LICENSE under the root directory of this GIT repository.
%
%{
%%\input{../latex/blogpost.tex}
%%\renewcommand{\basename}{qftLecture6}
%%\renewcommand{\dirname}{notes/phy2403/}
%%\newcommand{\keywords}{PHY2403H}
%%\input{../latex/peeter_prologue_print2.tex}
%%
%%%\usepackage{phy2403}
%%\usepackage{peeters_braket}
%%%\usepackage{peeters_layout_exercise}
%%\usepackage{peeters_figures}
%%\usepackage{mathtools}
%%\usepackage{siunitx}
%%\usepackage{macros_cal} % LL
%%
%%\beginArtNoToc
%%\generatetitle{PHY2403H Quantum Field Theory.  Lecture 6: Canonical quantization, Simple Harmonic Oscillators, Symmetries.  Taught by Prof.\ Erich Poppitz}
%\chapter{Canonical quantization, Simple Harmonic Oscillators, Symmetries}
\index{canonical quantization}
\index{simple harmonic oscillator}
\index{symmetry}
%%\label{chap:qftLecture6}
%%
%%\paragraph{DISCLAIMER: Very rough notes from class, with some additional side notes (esp. the QM SHO review).}
%%
%%These are notes for the UofT course PHY2403H, Quantum Field Theory I, taught by Prof. Erich Poppitz fall 2018.
%%%, covering \textchapref{{1}} \citep{peskin1995introduction} content.
%%
\section{Quantization of Field Theory.}

We are engaging in the ``canonical'' or Hamiltonian method of quantization.  It is also possible to quantize using path integrals, but it is hard to prove that operators are unitary doing so.  In fact, the mechanism used to show unitarity from path integrals is often to find the Lagrangian and show that there is a Hilbert space (i.e. using canonical quantization).  Canonical quantization essentially demands that the fields obey a commutator relation of the following form
\begin{dmath}\label{eqn:qftLecture6:20}
\antisymmetric{\pi(\Bx, t)}{\phi(\By, t)} = -i \deltathree(\Bx - \By).
\end{dmath}
We assumed that the quantized fields obey the Hamiltonian relations
\begin{dmath}\label{eqn:qftLecture6:160}
\begin{aligned}
\ddt{\phi} &= i \antisymmetric{H}{\phi} \\
\ddt{\pi} &= i \antisymmetric{H}{\pi}.
\end{aligned}
\end{dmath}

We were working with the Hamiltonian density
\begin{dmath}\label{eqn:qftLecture6:40}
\calH =
\inv{2} (\pi(\Bx, t))^2
+
\inv{2} (\spacegrad \phi(\Bx, t))^2
+
\frac{m^2}{2} \phi^2
+
\frac{\lambda}{4} \phi^4,
\end{dmath}
which included a mass term \( m \) and a potential term (\(\lambda\)).  We will expand all quantities in Taylor series in \( \lambda \) assuming they have a structure such as
\begin{dmath}\label{eqn:qftLecture6:180}
\begin{aligned}
f(\lambda) =
c_0 \lambda^0
+ c_1 \lambda^1
+ c_2 \lambda^2
+ c_3 \lambda^3
+ \cdots
\end{aligned}
\end{dmath}
We will stop this \underline{perturbation theory} approach at \( O(\lambda^2) \), and will ignore functions such as \( e^{-1/\lambda} \).

Within perturbation theory, to leaving order, set \( \lambda = 0 \), so that \( \phi \) obeys the Klein-Gordon equation (if \( m = 0 \) we have just a d'Lambertian (wave equation)).

We can write our field as a Fourier transform
\begin{dmath}\label{eqn:qftLecture6:60}
\phi(\Bx, t) = \int \frac{d^3 p}{(2\pi)^3} e^{i \Bp \cdot \Bx} \tilde{\phi}(\Bp, t),
\end{dmath}
and due to a Hermitian assumption (i.e. real field) this implies
\begin{dmath}\label{eqn:qftLecture6:80}
\tilde{\phi}^\conj(\Bp, t) = \tilde{\phi}(-\Bp, t).
\end{dmath}

We found that the Klein-Gordon equation implied that the momentum space representation obey Harmonic oscillator equations
\begin{dmath}\label{eqn:qftLecture6:100}
\ddot{\tilde{\phi}}(\Bp, t) = - \omega_\Bp \tilde{\phi}(\Bp, t),
\end{dmath}
where \( \omega_\Bp = \sqrt{\Bp^2 + m^2} \).
The solution of \cref{eqn:qftLecture6:100} may be represented as
\begin{dmath}\label{eqn:qftLecture6:120}
\tilde{\phi}(\Bq, t) = \inv{\sqrt{2 \omega_\Bq}} \lr{
e^{-i \omega_\Bq t} a_\Bq
+
e^{i \omega_\Bq t} b_\Bq^\dagger
}.
\end{dmath}
This is a general solution, but imposing \( a_\Bq = b_{-\Bq} \) ensures \cref{eqn:qftLecture6:80} is satisfied.
This leaves us with
\begin{dmath}\label{eqn:qftLecture6:140}
\tilde{\phi}(\Bq, t) = \inv{\sqrt{2 \omega_\Bq}} \lr{
e^{-i \omega_\Bq t} a_\Bq
+
e^{i \omega_\Bq t} a_{-\Bq}^\dagger
}.
\end{dmath}
We want to show that iff
\begin{dmath}\label{eqn:qftLecture6:200}
\antisymmetric{a_\Bq}{a^\dagger_\Bp} = \lr{ 2 \pi }^3 \deltathree(\Bp - \Bq),
\end{dmath}
then
\begin{dmath}\label{eqn:qftLecture6:220}
\antisymmetric{\pi(\By, t)}{\phi(\Bx, t)} = -i \deltathree(\Bx - \By),
\end{dmath}
where everything else commutes (i.e. \(
\antisymmetric{a_\Bp}{a_\Bq} =
\antisymmetric{a_\Bp^\dagger}{a_\Bq^\dagger} = 0 \)).
We will only show one direction, but you can go the other way too.

\begin{dmath}\label{eqn:qftLecture6:240}
\phi(\Bx, t)
=
\int \frac{d^3 p}{(2\pi)^3 \sqrt{2 \omega_\Bp}} e^{i \Bp \cdot \Bx}
\lr{
   e^{-i \omega_\Bp t} a_\Bp
   +
   e^{i \omega_\Bp t} a_{-\Bp}^\dagger
}
\end{dmath}
\begin{dmath}\label{eqn:qftLecture6:260}
\pi(\Bx, t) = \dot{\phi}
=
i \int \frac{d^3 q}{(2\pi)^3 \sqrt{2 \omega_\Bq}} \omega_\Bq e^{i \Bq \cdot \Bx}
\lr{
   -e^{-i \omega_\Bq t} a_\Bq
   +
   e^{i \omega_\Bq t} a_{-\Bq}^\dagger
}.
\end{dmath}
The commutator is
\begin{dmath}\label{eqn:qftLecture6:280}
\antisymmetric{\pi(\By, t)}{\phi(\Bx, t)}
=
i \int \frac{d^3 p}{(2\pi)^3
\sqrt{2 \omega_\Bp}}
\frac{d^3 q}{(2\pi)^3 \sqrt{2 \omega_\Bq}}
\omega_\Bq
e^{i \Bp \cdot \By + i \Bq \cdot \Bx}
\antisymmetric
{
   -e^{-i \omega_\Bq t} a_\Bq
   +
   e^{i \omega_\Bq t} a_{-\Bq}^\dagger
}
{
   e^{-i \omega_\Bp t} a_\Bp
   +
   e^{i \omega_\Bp t} a_{-\Bp}^\dagger
}
=
i \int \frac{d^3 p}{(2\pi)^3
\sqrt{2 \omega_\Bp}}
\frac{d^3 q}{(2\pi)^3 \sqrt{2 \omega_\Bq}}
\omega_\Bq
e^{i \Bp \cdot \By + i \Bq \cdot \Bx}
\lr{
   - e^{i (\omega_\Bp -\omega_\Bq) t}
   \antisymmetric { a_\Bq } { a_{-\Bp}^\dagger }
   +
   e^{i (\omega_\Bq -\omega_\Bp) t}
   \antisymmetric{a_{-\Bq}^\dagger}{ a_\Bp }
}
=
i \int \frac{d^3 p}{(2\pi)^3
\sqrt{2 \omega_\Bp}}
\frac{d^3 q}{(2\pi)^3 \sqrt{2 \omega_\Bq}}
\omega_\Bq (2\pi)^3
e^{i \Bp \cdot \By + i \Bq \cdot \Bx}
\lr{
   - e^{i (\omega_\Bp -\omega_\Bq) t} \deltathree(\Bq + \Bp)
   - e^{i (\omega_\Bq -\omega_\Bp) t} \deltathree(-\Bq -\Bp)
}
=
- 2 i \int \frac{d^3 p}{(2\pi)^3
2 \omega_\Bp}
\omega_\Bp
e^{i \Bp \cdot (\By - \Bx)}
=
-i \deltathree(\By - \Bx),
\end{dmath}
which is what we wanted to prove.

\section{Free Hamiltonian.}
\index{Hamiltonian!free}
We call the \( \lambda = 0 \) case the ``free'' Hamiltonian.  Plugging in the creation and anhillation operator representation we have
\begin{dmath}\label{eqn:qftLecture6:300}
\begin{aligned}
H
&= \int d^3 x \lr{ \inv{2} \pi^2 + \inv{2} (\spacegrad \phi)^2 + \frac{m^2}{2} \phi^2 } \\
&=
\inv{2} \int d^3 x
\frac{d^3 p}{(2\pi)^3}
\frac{d^3 q}{(2\pi)^3}
\frac{
e^{i (\Bp + \Bq)\cdot \Bx}
}{\sqrt{2 \omega_\Bp}
\sqrt{2 \omega_\Bq}}
\Biglr{ \\
&\qquad   -
   (\omega_\Bp)
   (\omega_\Bq)
   \lr{
      -e^{-i \omega_\Bp t} a_\Bp
      +
      e^{i \omega_\Bp t} a_{-\Bp}^\dagger
   }
   \lr{
      -e^{-i \omega_\Bq t} a_\Bq
      +
      e^{i \omega_\Bq t} a_{-\Bq}^\dagger
   } \\
&\qquad-
   \Bp \cdot \Bq
   \lr{
      e^{-i \omega_\Bp t} a_\Bp
      +
      e^{i \omega_\Bp t} a_{-\Bp}^\dagger
   }
   \lr{
      e^{-i \omega_\Bq t} a_\Bq
      +
      e^{i \omega_\Bq t} a_{-\Bq}^\dagger
   } \\
&\qquad+
   m^2
   \lr{
      e^{-i \omega_\Bp t} a_\Bp
      +
      e^{i \omega_\Bp t} a_{-\Bp}^\dagger
   }
   \lr{
      e^{-i \omega_\Bq t} a_\Bq
      +
      e^{i \omega_\Bq t} a_{-\Bq}^\dagger
   }
}.
\end{aligned}
\end{dmath}
An immediate simplification is possible by identifying a delta function factor \( \int d^3 x e^{i(\Bp + \Bq) \cdot \Bx}/(2\pi)^3 = \deltathree(\Bp + \Bq) \), so
\begin{dmath}\label{eqn:qftLecture6:1060}
H
=
\inv{2}
\int
\frac{d^3 p}{(2\pi)^3}
\frac{1
}{2 \omega_\Bp}
\lr{
   -
   (\omega_\Bp)^2
   \lr{
      -e^{-i \omega_\Bp t} a_\Bp
      +
      e^{i \omega_\Bp t} a_{-\Bp}^\dagger
   }
   \lr{
      -e^{-i \omega_\Bp t} a_{-\Bp}
      +
      e^{i \omega_\Bp t} a_{\Bp}^\dagger
   }
}
+
   (\Bp^2 + m^2)
   \lr{
      e^{-i \omega_\Bp t} a_\Bp
      +
      e^{i \omega_\Bp t} a_{-\Bp}^\dagger
   }
   \lr{
      e^{-i \omega_\Bp t} a_{-\Bp}
      +
      e^{i \omega_\Bp t} a_{\Bp}^\dagger
   }
=
\inv{2} \int \frac{d^3 p}{(2 \pi)^3} \inv{2 \omega_\Bp}
\lr{
   a_\Bp a_{-\Bp}
   \lr{
      \cancel{-\omega_\Bp^2 e^{-2 i \omega_\Bp t}}
      +
      \cancel{\omega_\Bp^2 e^{-2 i \omega_\Bp t}}
   }
+
   a_{-\Bp}^\dagger a_{\Bp}^\dagger
   \lr{
      -\cancel{\omega_\Bp^2 e^{2 i \omega_\Bp t}}
      +
      \cancel{\omega_\Bp^2 e^{2 i \omega_\Bp t}}
   }
+
   a_\Bp a^\dagger_{\Bp} \omega^2_\Bp(1 + 1)
+
   a^\dagger_{-\Bp} a_{-\Bp} \omega^2_\Bp(1 + 1)
}
\end{dmath}
When all is said and done we are left with
\begin{dmath}\label{eqn:qftLecture6:320}
H =
\int \frac{d^3 p}{(2 \pi)^3} \frac{\omega_\Bp}{2} \lr{
   a^\dagger_{-\Bp}
   a_{-\Bp}
+
   a_{\Bp}
   a^\dagger_{\Bp}
}.
\end{dmath}
A final \( \Bp \rightarrow -\Bp \) transformation \footnote{\( \iiint_{-R}^R d^3 p \rightarrow (-1)^3 \iiint_R^{-R} d^3 p' = (-1)^6 \iiint_{-R}^R d^3 p' \).} in the first integral, puts the free Hamiltonian (\( \lambda = 0 \)) into a nice symmetric form
\boxedEquation{eqn:qftLecture6:340}{
H_0 =
\int \frac{d^3 p}{(2 \pi)^3} \frac{\omega_\Bp}{2} \lr{
   a^\dagger_{\Bp}
   a_{\Bp}
+
   a_{\Bp}
   a^\dagger_{\Bp}
}.
}

\paragraph{Vacuum energy density.}
\index{vacuum energy density}
From the commutator relationship \cref{eqn:qftLecture6:200} we can write
\begin{dmath}\label{eqn:qftLecture6:360}
a_\Bp a^\dagger_\Bq
=
a^\dagger_\Bq
a_\Bp
+ (2 \pi)^3 \deltathree(\Bp - \Bq),
\end{dmath}
so

\begin{dmath}\label{eqn:qftLecture6:380}
H_0 =
\int \frac{d^3 p}{(2 \pi)^3} \omega_\Bp
\lr{
   a^\dagger_{\Bp}
   a_{\Bp}
+
\inv{2} (2 \pi)^3 \deltathree(0)
}.
\end{dmath}
The delta function term can be interpreted using
\begin{dmath}\label{eqn:qftLecture6:400}
(2 \pi)^3 \deltathree(\Bq)
= \int d^3 x e^{i \Bq \cdot \Bx},
\end{dmath}
so when \( \Bq = 0 \)
\begin{equation}\label{eqn:qftLecture6:420}
(2 \pi)^3 \deltathree(0) = \int d^3 x = V.
\end{equation}

We can write the Hamiltonian now in terms of the volume
\boxedEquation{eqn:qftLecture6:440}{
H_0 =
\int \frac{d^3 p}{(2 \pi)^3} \omega_\Bp
   a^\dagger_{\Bp}
   a_{\Bp}
+ V_3
\int \frac{d^3 p}{(2 \pi)^3} \frac{\omega_\Bp }{2} \times 1.
}

\section{QM SHO review.}

In units with \( m = 1 \) the non-relativistic QM SHO has the Hamiltonian
\begin{dmath}\label{eqn:qftLecture6:460}
H
= \inv{2} p^2 + \frac{\omega^2}{2} q^2.
%=
%\inv{2} \lr{ (p + i \omega q)(p - i \omega q) - i \omega (q p - p q) }
%=
%\omega \lr{ \inv{\sqrt{2 \omega}} (p + i \omega q) \inv{\sqrt{2 \omega}}(p - i \omega q) + 1 }
\end{dmath}
%Observe that the we can almost factor the Hamiltonian into two conjugate operators since
%(p + i \omega q)(p - i \omega q)
%=
% = \omega \lr{ a^\dagger a + \inv{2}}.
If we define a position operator with a time-domain Fourier representation given by
\begin{dmath}\label{eqn:qftLecture6:480}
q =  \inv{\sqrt{2\omega}} \lr{ a e^{-i \omega t} + a^\dagger e^{i \omega t} },
\end{dmath}
where the Fourier coefficients \( a, a^\dagger \) are operator valued, then the momentum operator is
\begin{equation}\label{eqn:qftLecture6:500}
p = \dot{q} =
\frac{i\omega}{\sqrt{2\omega}} \lr{ -a e^{-i \omega t} + a^\dagger e^{i \omega t} },
\end{equation}
or inverting for \( a, a^\dagger \)
\begin{dmath}\label{eqn:qftLecture6:1080}
\begin{aligned}
a &= \sqrt{\frac{\omega}{2}} \lr{ q - \inv{i \omega} p } e^{-i \omega t} \\
a^\dagger &= \sqrt{\frac{\omega}{2}} \lr{ q + \inv{i \omega} p } e^{i \omega t}.
\end{aligned}
\end{dmath}
By inspection it is apparent that the product \( a^\dagger a \) will be related to the Hamiltonian (i.e. a difference of squares).  That product is
\begin{dmath}\label{eqn:qftLecture6:1120}
a^\dagger a
=
\frac{\omega}{2}
\lr{ q + \inv{i \omega} p }
\lr{ q - \inv{i \omega} p }
=
\frac{\omega}{2} \lr{ q^2 + \inv{\omega^2} p^2 - \inv{i \omega} \antisymmetric{q}{p} }
= \inv{2 \omega} \lr{ p^2 + \omega^2 q^2 - \omega },
\end{dmath}
or
\begin{dmath}\label{eqn:qftLecture6:1140}
H = \omega \lr{ a^\dagger a + \inv{2} }.
\end{dmath}
We can glean some of the properties of \( a, a^\dagger \) by computing the commutator of \( p, q \), since that has a well known value
\begin{dmath}\label{eqn:qftLecture6:520}
i = \antisymmetric{q}{p}
=
\frac{i\omega}{2 \omega} \antisymmetric
{ a e^{-i \omega t} + a^\dagger e^{i \omega t} }
{ -a e^{-i \omega t} + a^\dagger e^{i \omega t} }
=
\frac{i}{2} \lr{
\antisymmetric{a}{a^\dagger} -
\antisymmetric{a^\dagger}{a} }
=
i
\antisymmetric{a}{a^\dagger},
\end{dmath}
so
\begin{dmath}\label{eqn:qftLecture6:1100}
\antisymmetric{a}{a^\dagger} = 1.
\end{dmath}
The operator \( a^\dagger a \) is the workhorse of the Hamiltonian and worth studying independently.  In particular, assume that we have a set of states \( \ket{n} \) that are eigenstates of \( a^\dagger a \) with eigenvalues \( \lambda_n \), that is
\begin{dmath}\label{eqn:qftLecture6:1160}
a^\dagger a \ket{n} = \lambda_n \ket{n}.
\end{dmath}
The action of \( a^\dagger a \) on \( a^\dagger \ket{n} \) is easy to compute
\begin{dmath}\label{eqn:qftLecture6:1180}
a^\dagger a a^\dagger \ket{n}
=
a^\dagger \lr{ a^\dagger a + 1 } \ket{n}
=
\lr{ \lambda_n + 1 } a^\dagger \ket{n},
\end{dmath}
so \( \lambda_n + 1 \) is an eigenvalue of \( a^\dagger \ket{n} \).  The state \( a^\dagger \ket{n} \) has an energy eigenstate that is one unit of energy larger than \( \ket{n} \).  For this reason we called \( a^\dagger \) the raising (or creation) operator.
Similarly,
\begin{dmath}\label{eqn:qftLecture6:1200}
a^\dagger a a \ket{n}
=
\lr{ a a^\dagger - 1 } a \ket{n}
=
(\lambda_n - 1) a \ket{n},
\end{dmath}
so \( \lambda_n - 1 \) is the energy eigenvalue of \( a \ket{n} \), having one less unit of energy than \( \ket{n} \).
We call \( a \) the annihilation (or lowering) operator.
If we argue that there is a lowest energy state, perhaps designated as \( \ket{0} \) then we must have
\begin{dmath}\label{eqn:qftLecture6:560}
a\ket{0} = 0,
\end{dmath}
by the assumption that there are no energy eigenstates with less energy than \( \ket{0} \).
We can think of higher order states being constructed from the ground state from using the raising operator \( a^\dagger \)
\begin{dmath}\label{eqn:qftLecture6:580}
\ket{n} = \frac{(a^\dagger)^n}{\sqrt{n!}} \ket{0}.
\end{dmath}
%The Hamiltonian action
%\begin{dmath}\label{eqn:qftLecture6:600}
%\omega a^\dagger a \ket{n} = (n \omega) \ket{n} = E_n \ket{n}.
%\end{dmath}

\section{Discussion.}

We've diagonalized in the Fourier representation for the momentum space fields.  For every value of momentum \( \Bp \) we have a quantum SHO.

For our field space we call our space the Fock vacuum and
\begin{dmath}\label{eqn:qftLecture6:620}
a_\Bp\ket{0} = 0,
\end{dmath}
and call \( a_\Bp \) the ``annihilation operator'', and
call \( a^\dagger_\Bp \) the ``creation operator''.
We say that \( a^\dagger_\Bp \ket{0} \) is the creation of a state of a single particle of momentum \( \Bp \) by \( a^\dagger_\Bp \).

We are discarding the volume term, a procedure called ``normal ordering''.  We define
\begin{equation}\label{eqn:qftLecture6:640}
: \frac{a^\dagger a + a a^\dagger}{2} :
\equiv
a^\dagger a.
\end{equation}
We are essentially forgetting the vacuum energy as some sort of unobservable quantity, leaving us with the free Hamiltonian of
\begin{dmath}\label{eqn:qftLecture6:660}
H_0 =
\int \frac{d^3 p}{(2 \pi)^3} \omega_\Bp
   a^\dagger_{\Bp}
   a_{\Bp}.
\end{dmath}
Consider
\begin{dmath}\label{eqn:qftLecture6:680}
H_0
a^\dagger_\Bq \ket{0}
=
\int \frac{d^3 p}{(2 \pi)^3} \omega_\Bp
   a^\dagger_{\Bp}
   a_{\Bp}
a^\dagger_\Bq
\ket{0}
=
\int \frac{d^3 p}{(2 \pi)^3} \omega_\Bp
   a^\dagger_{\Bp}
\lr{
   a^\dagger_\Bq a_\Bp + (2 \pi)^3 \deltathree(\Bp - \Bq)
}
\ket{0}
=
\int \frac{d^3 p}{(2 \pi)^3} \omega_\Bp
   a^\dagger_{\Bp} \lr{
   a^\dagger_\Bq \cancel{a_\Bp \ket{0}}
+ (2 \pi)^3 \deltathree(\Bp - \Bq) \ket{0}
}
=
\omega_\Bq a^\dagger_\Bq \ket{0}.
\end{dmath}

\paragraph{Question:} Is it possible to modify the Lagrangian or Hamiltonian that we start with so that this vacuum ground state is eliminated?  Answer: Only by imposing super-symmetric constraints (that pairs this (Bosonic) Hamiltonian to a Fermonic system in a way that there is exact cancellation).

We will see that the momentum operator has the form
\begin{dmath}\label{eqn:qftLecture6:700}
\calP
=
\int \frac{d^3 p}{(2 \pi)^3} \Bp a^\dagger_\Bp a_\Bp.
\end{dmath}
We say that \( a^\dagger_\Bp a^\dagger_\Bq \ket{0} \) a two particle space with energy \( \omega_\Bp + \omega_q\), and \(
(a^\dagger_\Bp)^m (a^\dagger_\Bq)^n \ket{0} \equiv
(a^\dagger_\Bp)^m \ket{0} \otimes (a^\dagger_\Bq)^n \ket{0} \), a \( m + n \) particle space.

There is a connection to statistical mechanics that is of interest

\begin{dmath}\label{eqn:qftLecture6:720}
\expectation{E}
= \inv{Z} \sum_n E_n e^{-E_n/\kB T}
= \inv{Z} \sum_n \bra{n} e^{-\hat{H}/\kB T} \hat{H} \ket{n},
\end{dmath}
so for a SHO Hamiltonian system
\begin{dmath}\label{eqn:qftLecture6:740}
\expectation{E}
= \inv{Z} \sum_n e^{-E_n/\kB T} \bra{n} \hat{H} \ket{n}
= \inv{Z} \sum_n e^{-E_n/\kB T} \bra{n} \omega a^\dagger a \ket{n}
= \frac{\omega}{e^{\omega/\kB T} - 1 }
= \expectation{ \omega a^\dagger a }_{\kB T},
\end{dmath}
which is the \( \kB T \) ensemble average energy for a SHO system.  Note that this sum was evaluated by noting that \( \bra{n} a^\dagger a \ket{n} = n \) which leaves sums of the form
\begin{dmath}\label{eqn:qftLecture6:1220}
\frac{\sum_{n = 0}^\infty n a^n }{\sum_{n = 0}^\infty a^n}
=
a \frac{\sum_{n = 1}^\infty n a^{n-1} }{\sum_{n = 0}^\infty a^n}
=
a (1 - a) \frac{d}{da} \lr{ \inv{1 - a} }
=
\frac{a}{1 - a}.
\end{dmath}

If we consider a real scalar field of mass \( m \) we have \( \omega_\Bp = \sqrt{ \Bp^2 + m^2 } \), but for a Maxwell field \( \BE, \BB \) where \( m = 0 \), our dispersion relation is \( \omega_\Bp = \Abs{\Bp} \).

We will see that for a free Maxwell field (no charges or currents) the Hamiltonian is
\begin{dmath}\label{eqn:qftLecture6:760}
H_{\text{Maxwell}} =
\sum_{i = 1}^2
\int \frac{d^3 p}{(2 \pi)^3} \omega_\Bp {a^i}^\dagger_\Bp {a^i}_\Bp,
\end{dmath}
where \( i \) is a polarization index.

We expect that we can evaluate an average such as \cref{eqn:qftLecture6:740} for our field, and operate using the analogy

\begin{dmath}\label{eqn:qftLecture6:780}
\begin{aligned}
a a^\dagger &= a^\dagger a + 1 \\
a_\Bp a_\Bp^\dagger &= a_\Bp^\dagger a_\Bp + V_3.
\end{aligned}
\end{dmath}
so if we rescale by \( \sqrt{V_3} \)
\begin{dmath}\label{eqn:qftLecture6:800}
a_\Bp = \sqrt{V_3} \tilde{a}_\Bp,
\end{dmath}
then we have commutator relations like standard QM
\begin{dmath}\label{eqn:qftLecture6:820}
\tilde{a} \tilde{a}^\dagger = \tilde{a}^\dagger \tilde{a} + 1.
\end{dmath}

So we can immediately evaluate the energy expectation for our quantized fields
\begin{dmath}\label{eqn:qftLecture6:840}
\expectation{H_0}
=
\expectation{
\int \frac{d^3 p}{(2 \pi)^3} \omega_\Bp a_\Bp^\dagger a_\Bp
}
=
\int \frac{d^3 p}{(2 \pi)^3} \omega_\Bp V_3 \expectation{ \tilde{a}^\dagger_\Bp a_\Bp }
=
V_3
\int \frac{d^3 p}{(2 \pi)^3} \frac{\omega_\Bp}{e^{\omega_\Bp/\kB T} - 1}.
\end{dmath}
Using this with the Maxwell field, we have a factor of two from polarization
\begin{dmath}\label{eqn:qftLecture6:860}
U^{\text{Maxwell}} = 2 V_3
\int \frac{d^3 p}{(2 \pi)^3} \frac{\Abs{\Bp}}{e^{\omega_\Bp/\kB T} - 1},
\end{dmath}
which is Planck's law describing the blackbody energy spectrum.

\section{Switching gears: Symmetries.}
\index{symmetries}

The question is how to apply the CCR results to moving frames, which is done using Lorentz transformations.  Just like we know that the exponential of the Hamiltonian (times time) represents time translations, we will examine symmetries that relate results in different frames.

\paragraph{Examples.}

For scalar field(s) with action
\begin{dmath}\label{eqn:qftLecture6:880}
S = \int d^d x \LL(\phi^i, \partial_\mu \phi^i).
\end{dmath}
For example, we've been using our massive (Boson) real scalar field with Lagrangian density
\begin{dmath}\label{eqn:qftLecture6:900}
\LL = \inv{2} \partial_\mu \phi\partial^\mu \phi - \frac{m^2}{2} \phi^2 - V(\phi).
\end{dmath}

Internal symmetry example

\begin{dmath}\label{eqn:qftLecture6:920}
H = J \sum_{\expectation{n, n'}} \BS_n \cdot \BS_{n'},
\end{dmath}
where the sum means the sum over neighbouring indexes \( n, n' \) as sketched in
\cref{fig:adacentSpinGrid:adacentSpinGridFig1}.
\imageFigure{../figures/phy2403-quantum-field-theory/adacentSpinGridFig1}{Neighbouring spin cells.}{fig:adacentSpinGrid:adacentSpinGridFig1}{0.3}

Such a Hamiltonian is left invariant by the transformation \( \BS_n \rightarrow -\BS_n \) since the Hamiltonian is quadratic.

Suppose that \( \phi \rightarrow -\phi\) is a symmetry (it leaves the Lagrangian unchanged).  Example

\begin{dmath}\label{eqn:qftLecture6:940}
\phi =
\begin{bmatrix}
\phi^1 \\
\phi^2 \\
\vdots \\
\phi^n \\
\end{bmatrix}
\end{dmath}

the Lagrangian
\begin{dmath}\label{eqn:qftLecture6:960}
\LL = \inv{2} \partial_\mu \phi^\T \partial^\mu \phi - \frac{m^2}{2} \phi^\T \phi - V(\phi^\T \phi).
\end{dmath}
If \( O \) is any \( n \times n \) orthogonal matrix, then it is symmetry since
\begin{dmath}\label{eqn:qftLecture6:980}
\phi^\T \phi \rightarrow \phi^\T O^\T O \phi = \phi^\T \phi.
\end{dmath}

O(2) model, HW, problem 2.  Example for complex \( \phi \)
\begin{dmath}\label{eqn:qftLecture6:1000}
\phi \rightarrow e^{i \phi} \phi,
\end{dmath}
\begin{dmath}\label{eqn:qftLecture6:1020}
\phi = \frac{\psi_1 + i \psi_2}{\sqrt{2}}
\end{dmath}

\begin{dmath}\label{eqn:qftLecture6:1040}
\begin{bmatrix}
\psi_1 \\
\psi_2
\end{bmatrix}
\rightarrow
\begin{bmatrix}
\cos\alpha & \sin\alpha \\
-\sin\alpha & \cos\alpha
\end{bmatrix}
\begin{bmatrix}
\psi_1 \\
\psi_2
\end{bmatrix}
\end{dmath}

%}
%\EndNoBibArticle

   \chapter{Symmetries, translation currents, energy momentum tensor.}
      %
% Copyright � 2018 Peeter Joot.  All Rights Reserved.
% Licenced as described in the file LICENSE under the root directory of this GIT repository.
%
%{
%%\input{../latex/blogpost.tex}
%%\renewcommand{\basename}{qftLecture7}
%%\renewcommand{\dirname}{notes/phy2403/}
%%\newcommand{\keywords}{PHY2403H}
%%\input{../latex/peeter_prologue_print2.tex}
%%
%%%\usepackage{phy2403}
%%\usepackage{peeters_braket}
%%\usepackage{peeters_layout_exercise} % makedefinition
%%\usepackage{peeters_figures}
%%\usepackage{mathtools}
%%\usepackage{siunitx}
%%\usepackage{macros_cal} % LL
%%
%%\beginArtNoToc
%%\generatetitle{PHY2403H Quantum Field Theory.  Lecture 7: Symmetries, translation currents, energy momentum tensor.  Taught by Prof. Erich Poppitz}
%\chapter{Symmetries, translation currents, energy momentum tensor.}
\index{translation current}
\label{chap:qftLecture7}

%\paragraph{DISCLAIMER: Very rough notes from class, with some additional side notes.}
%
%These are notes for the UofT course PHY2403H, Quantum Field Theory I, taught by Prof. Erich Poppitz fall 2018.
%%, covering \textchapref{{1}} \citep{peskin1995introduction} content.
%
\section{Symmetries.}
\index{symmetries}

Given the complexities of the non-linear systems we want to investigate, examination of symmetries gives us simpler problems that we can solve.

\begin{itemize}
\item ``internal'' symmetries.  This means that the symmetries do not act on space time \( (\Bx, t) \).  An example is
\begin{dmath}\label{eqn:qftLecture7:20}
\phi^i =
\begin{bmatrix}
\psi_1 \\
\psi_2 \\
\vdots \\
\psi_N \\
\end{bmatrix}
\end{dmath}
If we map
\( \phi^i \rightarrow O^i_j \phi^j \) where \( O^\T O = 1 \), then we call this an internal symmetry.
The corresponding Lagrangian density might be something like
\begin{dmath}\label{eqn:qftLecture7:40}
\LL = \inv{2} \partial_\mu \Bphi \cdot \partial^\mu \Bphi - \frac{m^2}{2} \Bphi \cdot \Bphi - V(\Bphi \cdot \Bphi)
\end{dmath}
\item spacetime symmetries: Translations, rotations, boosts, dilatations.  We will consider continuous symmetries, which can be defined as a succession of infinitesimal transformations.
An example from \(O(2)\) is a rotation
\begin{dmath}\label{eqn:qftLecture7:60}
\begin{bmatrix}
\phi^1 \\
\phi^2 \\
\end{bmatrix}
\rightarrow
\begin{bmatrix}
\cos\alpha & \sin\alpha \\
-\sin\alpha & \cos\alpha \\
\end{bmatrix}
\begin{bmatrix}
\phi^1 \\
\phi^2
\end{bmatrix},
\end{dmath}
or if \( \alpha \sim 0 \)
\begin{dmath}\label{eqn:qftLecture7:80}
\begin{bmatrix}
\phi^1 \\
\phi^2 \\
\end{bmatrix}
\rightarrow
\begin{bmatrix}
1 & \alpha \\
-\alpha & 1\\
\end{bmatrix}
\begin{bmatrix}
\phi^1 \\
\phi^2
\end{bmatrix}
=
\begin{bmatrix}
\phi^1 \\
\phi^2
\end{bmatrix}
+
\alpha
\begin{bmatrix}
\phi^2 \\
-\phi^1
\end{bmatrix}
\end{dmath}
In index notation we write
\begin{dmath}\label{eqn:qftLecture7:100}
\phi^i \rightarrow \phi^i + \alpha e^{ij} \phi^j,
\end{dmath}
where \( \epsilon^{12} = +1, \epsilon^{21} = -1 \) is the completely antisymmetric tensor.  This can be written in more general form as
\begin{dmath}\label{eqn:qftLecture7:120}
\phi^i \rightarrow \phi^i + \delta \phi^i,
\end{dmath}
where \( \delta \phi^i \) is considered to be an infinitesimal transformation.
\end{itemize}

\index{symmetry}
\makedefinition{Symmetry}{dfn:qftLecture7:140}{
A symmetry means that there is some transformation
\begin{equation*}
\phi^i \rightarrow \phi^i + \delta \phi^i,
\end{equation*}
where
\( \delta \phi^i \) is an infinitesimal transformation, and the equations of motion are invariant under this transformation.
} % definition

\index{Noether's theorem}
\maketheorem{Noether's theorem (1st).}{thm:qftLecture7:160}{
If the equations of motion re invariant under \( \phi^\mu \rightarrow \phi^\mu + \delta \phi^\mu \), then there exists a conserved current \( j^\mu \) such that \( \partial_\mu j^\mu = 0 \).
} % theorem

Noether's first theorem applies to global symmetries, where the parameters are the same for all \( (\Bx, t)\).  Gauge symmetries are not examples of such global symmetries.

Given a Lagrangian density \( \LL(\phi(x), \phi_{,\mu}(x)) \), where \( \phi_{,\mu} \equiv \partial_\mu \phi \).  The action is
\begin{dmath}\label{eqn:qftLecture7:160}
S = \int d^d x \LL.
\end{dmath}
The equations of motion are invariant if under \( \phi(x) \rightarrow \phi'(x) = \phi(x) + \delta_\epsilon \phi(x)\), we have
\begin{dmath}\label{eqn:qftLecture7:180}
\LL(\phi) \rightarrow \LL'(\phi') = \LL(\phi) + \partial_\mu J_\epsilon^\mu(\phi) + O(\epsilon^2).
\end{dmath}
Then there exists a conserved current.  In QFT we say that the E.O.M's are ``on shell''.
Note that \cref{eqn:qftLecture7:180} is a
symmetry since we have added a total derivative to the Lagrangian which leaves the equations of motion of unchanged.

In general, the change of action under arbitrary variation of \( \delta \phi\) of the fields is
\begin{dmath}\label{eqn:qftLecture7:200}
\delta S
=
\int d^d x \delta \LL(\phi, \partial_\mu \phi)
=
\int d^d x \lr{
\PD{\phi}{\LL} \delta \phi
+
\PD{(\partial_\mu \phi)}{\LL} \delta \partial_\mu \phi
}
=
\int d^d x \lr{
\partial_\mu \lr{ \PD{(\partial_\mu \phi)}{\LL} } \delta \phi
+
\PD{(\partial_\mu \phi)}{\LL} \partial_\mu \delta \phi
}
=
\int d^d x
\partial_\mu \lr{ \frac{\delta \LL}{\delta(\partial_\mu \phi)} \delta \phi }.
\end{dmath}
However from \cref{eqn:qftLecture7:180}
\begin{dmath}\label{eqn:qftLecture7:220}
\delta_\epsilon \LL = \partial_\mu J_\epsilon^\mu(\phi, \partial_\mu \phi),
\end{dmath}
so after equating these variations we fine that
\begin{dmath}\label{eqn:qftLecture7:240}
\delta S = \int d^d x \delta_\epsilon \LL = \int d^d x \partial_\mu J_\epsilon^\mu,
\end{dmath}
or
\begin{dmath}\label{eqn:qftLecture7:260}
0 = \int d^d x
\partial_\mu \lr{ \frac{\delta \LL}{\delta(\partial_\mu \phi)} \delta \phi - J_\epsilon^\mu },
\end{dmath}
or \( \partial_\mu j^\mu = 0 \) provided
\boxedEquation{eqn:qftLecture7:280}{
j^\mu =
\frac{\delta \LL}{\delta(\partial_\mu \phi)} \delta_\epsilon \phi  - J_\epsilon^\mu.
}

Integrating the divergence of the current over a space time volume, perhaps that of \cref{fig:spacetimeCylinder:spacetimeCylinderFig1}, is also zero.  That is
%\imageFigure{../figures/phy2403-quantum-field-theory/spacetimeCylinderFig1}{Cylindrical spacetime boundary.}{fig:spacetimeCylinder:spacetimeCylinderFig1}{0.3}
\begin{dmath}\label{eqn:qftLecture7:300}
0 =
\int d^4 x \, \partial_\mu j^\mu
=
\int d^3 \Bx dt \, \partial_\mu j^\mu
=
\int d^3 \Bx dt \, \partial_t j^0 -
\cancel{\int d^3 \Bx dt \spacegrad \cdot \Bj},
\end{dmath}
where the spatial divergence is zero assuming there's no current leaving the volume on the infinite boundary (no \(\Bj\) at spatial infinity.)

We write
\begin{dmath}\label{eqn:qftLecture7:560}
Q = \int d^3x j^0,
\end{dmath}
and call this the on-shell charge associated with the symmetry.
\index{on-shell}

\section{Spacetime translation.}
\index{spacetime translation}

A spacetime translation has the form
\begin{equation}\label{eqn:qftLecture7:320}
x^\mu \rightarrow {x'}^\mu = x^\mu + a^\mu,
\end{equation}
where the fields transform as
\begin{equation}\label{eqn:qftLecture7:340}
\phi(x) \rightarrow \phi'(x') = \phi(x).
\end{equation}
Contrast this to a Lorentz transformation that had the form \( x^\mu \rightarrow {x'}^\mu = {\Lambda^\mu}_\nu x^\nu \).

If \(\phi'(x + a) = \phi(x) \), then
\begin{dmath}\label{eqn:qftLecture7:360}
\phi'(x) + a^\mu \partial_\mu \phi'(x) =
\phi'(x) + a^\mu \partial_\mu \phi(x) =
\phi(x),
\end{dmath}
so
\begin{dmath}\label{eqn:qftLecture7:380}
\phi'(x)
= \phi(x) - a^\mu \partial_\mu \phi'(x)
= \phi(x) + \delta_a \phi(x),
\end{dmath}
or
\begin{dmath}\label{eqn:qftLecture7:580}
\delta_a \phi(x) = - a^\mu \partial_\mu \phi(x).
\end{dmath}
Under \( \phi \rightarrow \phi - a^\mu \partial_\mu \phi \), we have
\begin{dmath}\label{eqn:qftLecture7:400}
\LL(\phi) \rightarrow \LL(\phi) - a^\mu \partial_\mu \LL.
\end{dmath}
Let's calculate this with our scalar theory Lagrangian
\begin{dmath}\label{eqn:qftLecture7:420}
\LL = \inv{2} \partial_\mu \phi \partial^\mu \phi - \frac{m^2}{2} \phi^2 - V(\phi).
\end{dmath}
The Lagrangian variation\footnote{Using: \( \partial_\alpha((1/2)\partial_\mu \phi \partial^\mu \phi) = 2(1/2) \partial_\mu \phi( \partial_\alpha \partial^\mu \phi)\).} is
\begin{dmath}\label{eqn:qftLecture7:440}
\evalbar{\delta \LL}{\phi \rightarrow \phi + \delta \phi, \delta\phi = - a^\mu \partial_\mu \phi}
=
(\partial_\mu \phi) \delta (\partial^\mu \phi) - m^2 \phi \delta \phi - \PD{\phi}{V} \delta \phi
=
(\partial_\mu \phi)(-a^\nu \partial_\nu \partial^\mu \phi) + m^2 \phi a^\nu \partial_\nu \phi + \PD{\phi}{V} a^\nu \partial_\nu \phi
=
- a^\nu \partial_\nu \lr{ \inv{2} \partial_\mu \phi \partial^\mu \phi - \frac{m^2}{2} \phi^2 - V(\phi) }
=
- a^\nu \partial_\nu \LL,
\end{dmath}
so the current is
\begin{dmath}\label{eqn:qftLecture7:600}
j^\mu
=
(\partial^\mu \phi) (-a^\nu \partial_\nu \phi) + a^\mu \LL
=
-a^\nu \lr{ \partial^\mu \phi \partial_\nu \phi - \delta^\mu_\nu \LL }.
\end{dmath}
We really have a current for each \( \nu \) direction and can make that explicit writing
\begin{dmath}\label{eqn:qftLecture7:460}
\delta_\nu \LL = -\partial_\nu \LL
= - \partial_\mu \lr{ {\delta^\mu}_\nu \LL }
= \partial_\mu {j^\mu}_\nu
\end{dmath}
we write
\begin{dmath}\label{eqn:qftLecture7:480}
{j^\mu}_\nu = \PD{x_\mu}{\phi} \lr{ - \PD{x^\nu}{\phi} } + {\delta^\mu}_\nu \LL,
\end{dmath}
where \( \nu \) are labels which coordinates are translated:
\begin{dmath}\label{eqn:qftLecture7:500}
\begin{aligned}
\partial_\nu \phi &= - \partial_\nu \phi \\
\partial_\nu \LL &= - \partial_\nu \LL.
\end{aligned}
\end{dmath}
We call the conserved quantities elements of the energy-momentum tensor, and write it as
\index{energy momentum tensor}
\boxedEquation{eqn:qftLecture7:520}{
{T^\mu}_\nu = -\PD{x_\mu}{\phi} \PD{x^\nu}{\phi} + {\delta^\mu}_\nu \LL.
}

Incidentally, we picked a non-standard sign convention for the tensor, as an explicit expansion of \( T^{00} \), the energy density component, shows
\begin{dmath}\label{eqn:qftLecture7:540}
{T^0}_0 =
-\PD{t}{\phi}
\PD{t}{\phi}
+\inv{2}
\PD{t}{\phi}
\PD{t}{\phi}
- \inv{2} (\spacegrad \phi) \cdot (\spacegrad \phi)
- \frac{m^2}{2} \phi^2 - V(\phi)
=
-\inv{2} \PD{t}{\phi} \PD{t}{\phi}
- \inv{2} (\spacegrad \phi) \cdot (\spacegrad \phi)
- \frac{m^2}{2} \phi^2 - V(\phi).
\end{dmath}
Had we translated by \( -a^\mu \) we'd have a positive definite tensor instead.

%%\section{Problems.}
%%
%%\makeproblem{Adding a total derivative to the Lagrangian}{problem:qftLecture7:560}{
%%Show that adding a total derivative to the Lagrangian density leaves the equations of motion unchanged.
%%} % problem
%%
%%\makeanswer{problem:qftLecture7:560}{
%%Given
%%\begin{dmath}\label{eqn:qftLecture7:620}
%%\LL' = \LL + \partial_\mu a^\mu
%%\end{dmath}
%%} % answer

%}
%\EndNoBibArticle

      \section{Problems.}
         %
% Copyright � 2016 Peeter Joot.  All Rights Reserved.
% Licenced as described in the file LICENSE under the root directory of this GIT repository.
%
%{
%\input{../latex/blogpost.tex}
%\renewcommand{\basename}{noetherCurrentScalarField}
%\renewcommand{\dirname}{notes/phy1520/}
%%\newcommand{\dateintitle}{}
%%\newcommand{\keywords}{}
%
%\input{../latex/peeter_prologue_print2.tex}
%
%\usepackage{peeters_layout_exercise}
%\usepackage{peeters_braket}
%\usepackage{peeters_figures}
%\usepackage{macros_cal}
%
%\beginArtNoToc
%
%\generatetitle{Energy-momentum tensor for a scalar field}
%\chapter{Energy-momentum tensor for a scalar field}
%\label{chap:noetherCurrentScalarField}
% \citep{sakurai2014modern} pr X.Y
\makeproblem{Energy-momentum tensor for a scalar field}{problem:noetherCurrentScalarField:1}{
It is claimed in \citep{LukeQFT} (3.2.1) that the momentum components of the energy-momentum tensor was found to be
\begin{equation}\label{eqn:noetherCurrentScalarField:20}
\Be_n \int d^3 x T^{0 n} = \int d^3 k \Bk a_{\Bk}^\dagger a_{\Bk}.
\end{equation}

\makesubproblem{}{problem:noetherCurrentScalarField:1:a}
Calculate this.

\makesubproblem{}{problem:noetherCurrentScalarField:1:b}
Calculate the other energy-momentum tensor components for the spacelike components.

\makesubproblem{}{problem:noetherCurrentScalarField:1:c}

Calculate the other energy-momentum tensor components for the Hamiltonian component.
} % problem

\makeanswer{problem:noetherCurrentScalarField:1}{
First, from the Noether current for the scalar field Lagrangian in question, what is the energy-momentum tensor explicitly?
\begin{equation}\label{eqn:noetherCurrentScalarField:40}
\begin{aligned}
T^{\mu \nu}
  &= \pi^\mu \partial^\nu \phi - g^{\mu \nu} \LL
\\&= \pi^\mu \partial^\nu \phi - g^{\mu \nu} \inv{2} \lr{ \partial_\alpha \phi \partial^\alpha \phi - \mu^2 \phi^2 }
\\&= \pi^\mu \pi^\nu - g^{\mu \nu} \inv{2} \lr{ \pi_\alpha \pi^\alpha - \mu^2 \phi^2 }
\\&= \pi^\mu \pi^\nu - \inv{2} g^{\mu \nu} g_{\alpha\beta} \pi^\beta \pi^\alpha + \inv{2} g^{\mu \nu} \mu^2 \phi^2.
\end{aligned}
\end{equation}

Consider some special cases for the indexes.  For \( \mu = \nu = 0 \), the result is the Hamiltonian density
\begin{equation}\label{eqn:noetherCurrentScalarField:200}
\begin{aligned}
T^{00}
  &= \pi^0 \pi^0 - \inv{2} g^{0 0} \pi_\alpha \pi^\alpha + \inv{2} g^{0 0} \mu^2 \phi^2
\\&= \pi^0 \pi^0 - \inv{2} \pi_\alpha \pi^\alpha + \inv{2} \mu^2 \phi^2
\\&= \inv{2} \pi^0 \pi^0 - \inv{2} \pi_n \pi^n + \inv{2} \mu^2 \phi^2
\\&= \inv{2} \pi^2 + \inv{2} (\spacegrad \phi)^2 + \inv{2} \mu^2 \phi^2,
\end{aligned}
\end{equation}
%
where \( \pi^2 = (\partial_0 \phi)^2 \ne \partial^2 \phi \).  For any \( \mu \ne \nu \) the off diagonal metric elements are zero, leaving just
\begin{equation}\label{eqn:noetherCurrentScalarField:220}
T^{\mu\nu} = \pi^\mu \pi^\nu.
\end{equation}

Finally, when \( n \ne 0 \), the remaining diagonal terms are
\begin{equation}\label{eqn:noetherCurrentScalarField:240}
\begin{aligned}
T^{nn}
  &= \pi^n \pi^n - \inv{2} g^{n n} \pi_\alpha \pi^\alpha + \inv{2} g^{n n} n^2 \phi^2
\\&= \pi^n \pi^n + \inv{2} \pi_\alpha \pi^\alpha - \inv{2} \mu^2 \phi^2
\\&= \inv{2} \pi^2 + \pi^n \pi^n - \inv{2} \pi^m \pi^m - \inv{2} \mu^2 \phi^2
\\&= \inv{2} \pi^2 + \inv{2} \pi^n \pi^n - \inv{2} \sum_{m\ne n,0} \pi^m \pi^m - \inv{2} \mu^2 \phi^2
\\&= \inv{2} \sum_{m = n,0} \pi^m \pi^m - \inv{2} \sum_{m\ne n,0} \pi^m \pi^m - \inv{2} \mu^2 \phi^2.
\end{aligned}
\end{equation}

The canonical momenta are
\begin{equation}\label{eqn:noetherCurrentScalarField:60}
\pi^\mu
=
\partial^\mu
\int \frac{d^3 k}{(2\pi)^{3/2} \sqrt{ 2 \omega_k }} \lr{ a_{\Bk} e^{-i k \cdot x} + a_{\Bk}^\dagger e^{i k \cdot x} },
\end{equation}
%
but
\begin{equation}\label{eqn:noetherCurrentScalarField:80}
\begin{aligned}
\partial^\mu e^{i k \cdot x}
&= \partial^\mu \exp\lr{ i k^\alpha x_\alpha } \\
&= i k^\mu \exp\lr{ i k \cdot x },
\end{aligned}
\end{equation}
%
so
\begin{equation}\label{eqn:noetherCurrentScalarField:100}
\begin{aligned}
\pi^\mu
&=
i
\int \frac{d^3 k k^\mu}{(2\pi)^{3/2} \sqrt{ 2 \omega_k }} \lr{ - a_{\Bk} e^{-i k \cdot x} + a_{\Bk}^\dagger e^{i k \cdot x} }
\\&=
i
\int \frac{d^3 k k^\mu}{(2\pi)^{3/2} \sqrt{ 2 \omega_k }} \lr{ - a_{\Bk} e^{-i \omega_k t + \Bk \cdot \Bx} + a_{\Bk}^\dagger e^{i \omega_k t - i \Bk \cdot \Bx} }
\\&=
i
\int \frac{d^3 k k^\mu}{(2\pi)^{3/2} \sqrt{ 2 \omega_k }}
\lr{
- a_{\Bk} e^{-i \omega_k t }
+ a_{-\Bk}^\dagger e^{i \omega_k t }
}
e^{ i \Bk \cdot \Bx}
.
\end{aligned}
\end{equation}

This gives
\begin{equation}\label{eqn:noetherCurrentScalarField:120}
\begin{aligned}
\int d^3 x \pi^\mu \pi^\nu
&=
-
\inv{2}
\int d^3 x \frac{d^3 k d^3 p}{(2\pi)^{3} } \frac{ k^\mu p^\nu}{\sqrt{ \omega_k \omega_p }}
\lr{
- a_{\Bk} e^{-i \omega_k t }
+ a_{-\Bk}^\dagger e^{i \omega_k t }
}
\lr{
- a_{\Bp} e^{-i \omega_p t }
+ a_{-\Bp}^\dagger e^{i \omega_p t }
}
e^{ i (\Bp + \Bk) \cdot \Bx}
\\&=
-
\inv{2}
\int d^3 k d^3 p \frac{ k^\mu p^\nu}{\sqrt{ \omega_k \omega_p }}
\lr{
- a_{\Bk} e^{-i \omega_k t }
+ a_{-\Bk}^\dagger e^{i \omega_k t }
}
\lr{
- a_{\Bp} e^{-i \omega_p t }
+ a_{-\Bp}^\dagger e^{i \omega_p t }
}
\deltathree( \Bp + \Bk )
\\&=
-
\inv{2}
\int d^3 k d^3 p \frac{ k^\mu p^\nu}{\omega_k}
\lr{
  a_{\Bk} a_{-\Bk} e^{-2 i \omega_k t }
- a_{\Bk} a_{\Bk}^\dagger
- a_{-\Bk}^\dagger a_{-\Bk}
+ a_{-\Bk}^\dagger a_{\Bk}^\dagger e^{2 i \omega_k t }
}
%\lr{
%- a_{\Bk} e^{-i \omega_k t }
%+ a_{-\Bk}^\dagger e^{i \omega_k t }
%}
%\lr{
%- a_{-\Bk} e^{-i \omega_k t }
%+ a_{\Bk}^\dagger e^{i \omega_k t }
%}
\deltathree(\Bp + \Bk)
%%%%
%=
%-\inv{2} \int d^3 x \inv{(2\pi)^3}
%\int d^3 k d^3 j \frac{k^\mu j^\nu}{\sqrt{\omega_k \omega_j}}
%\lr{ - a_{\Bk} e^{-i \omega_k t + \Bk \cdot \Bx} + a_{\Bk}^\dagger e^{i \omega_k t - i \Bk \cdot \Bx} }
%\lr{ - a_j e^{-i \omega_j t + \Bj \cdot \Bx} + a_j^\dagger e^{i \omega_j t - i \Bj \cdot \Bx} }
%=
%-\inv{2} \int d^3 x \inv{(2\pi)^3}
%\int d^3 k d^3 j \frac{k^\mu j^\nu}{\sqrt{\omega_k \omega_j}}
%\lr{
%  a_{\Bk} a_j e^{-i (\omega_j + \omega_k) t + (\Bj + \Bk) \cdot \Bx}
%- a_{\Bk} a_j^\dagger e^{i (\omega_j - \omega_k) t - i (\Bj -\Bk) \cdot \Bx}
%- a_{\Bk}^\dagger a_j e^{-i (\omega_j -\omega_k) t - (\Bk - \Bj) \cdot \Bx}
%+ a_{\Bk}^\dagger a_j^\dagger e^{i (\omega_j + \omega_k) t - i (\Bj + \Bk) \cdot \Bx}
%}
%=
%-\inv{2}
%\int d^3 k d^3 j \frac{k^\mu j^\nu}{\sqrt{\omega_k \omega_j}}
%\lr{
%  a_{\Bk} a_j e^{-i (\omega_j + \omega_k) t } \deltathree(\Bj + \Bk)
%- a_{\Bk} a_j^\dagger e^{i (\omega_j - \omega_k) t } \deltathree(\Bj -\Bk)
%- a_{\Bk}^\dagger a_j e^{-i (\omega_j -\omega_k) t } \deltathree (\Bk - \Bj)
%+ a_{\Bk}^\dagger a_j^\dagger e^{i (\omega_j + \omega_k) t } \deltathree (\Bj + \Bk)
%}
.
\end{aligned}
\end{equation}

Further reduction of the leading \( k^\mu p^\nu \) term has a sign that depends on the values of the indices.
\makeSubAnswer{}{problem:noetherCurrentScalarField:1:a}
First consider the momentum case where one of \( \mu \), or \( \nu \) is zero
\begin{equation}\label{eqn:noetherCurrentScalarField:140}
\int d^3 x \pi^\mu \pi^0 =
\int d^3 x \pi^0 \pi^\mu
=
-\inv{2}
\int d^3 k k^\mu
\lr{
  a_{\Bk} a_{-\Bk} e^{-2 i \omega_k t }
- a_{\Bk} a_{\Bk}^\dagger
- a_{\Bk}^\dagger a_{\Bk}
+ a_{\Bk}^\dagger a_{-\Bk}^\dagger e^{2 i \omega_k t }
}.
\end{equation}

For \( \mu \ne 0 \) this can be written as a vector operator
\begin{equation}\label{eqn:noetherCurrentScalarField:440}
\begin{aligned}
\Be_n \int d^3 x T^{0 n}
&=
-\inv{2}
\int d^3 k \Bk
\lr{
  a_{\Bk} a_{-\Bk} e^{-2 i \omega_k t }
+ a_{\Bk}^\dagger a_{-\Bk}^\dagger e^{2 i \omega_k t }
} \\
&\quad +
\inv{2}
\int d^3 k \Bk
\lr{
  a_{\Bk} a_{\Bk}^\dagger
+ a_{\Bk}^\dagger a_{\Bk}
}
\end{aligned}
\end{equation}

To get the desired result the time dependent terms have to be made to go away somehow.  Consider a spherical parameterization of the momentum space
\begin{equation}\label{eqn:noetherCurrentScalarField:460}
\Bk = k \lr{ \sin\theta \cos\phi, \sin\theta \sin\phi, \cos\theta },
\end{equation}
%
Note that the volume element is
\begin{equation}\label{eqn:noetherCurrentScalarField:480}
d^3 k = k^2 \sin\theta dk \wedge d\theta \wedge d\phi,
\end{equation}
where \( k \in [0, \infty]\), \(\theta \in [0, \pi]\), and \( \phi \in [0, 2\pi] \).  If we map \( \Bk \rightarrow -\Bk \), the volume element becomes
\begin{equation}\label{eqn:noetherCurrentScalarField:500}
d^3 k = (-k)^2 \sin\theta d(-k) \wedge d\theta \wedge d\phi,
\end{equation}
over the same angular intervals, but \( k \in [-\infty, 0]\).  Flipping the sign of the time dependent operator products gives
\begin{equation}\label{eqn:noetherCurrentScalarField:520}
\begin{aligned}
  a_{\Bk} a_{-\Bk} e^{-2 i \omega_k t }
+ a_{\Bk}^\dagger a_{-\Bk}^\dagger e^{2 i \omega_k t }
&\rightarrow
  a_{-\Bk} a_{\Bk} e^{-2 i \omega_k t }
+ a_{-\Bk}^\dagger a_{\Bk}^\dagger e^{2 i \omega_k t }
\\&=
  a_{\Bk} a_{-\Bk} e^{-2 i \omega_k t }
+ a_{\Bk}^\dagger a_{-\Bk}^\dagger e^{2 i \omega_k t },
\end{aligned}
\end{equation}
which shows that this is an even function in \( \Bk \).  The even characteristics of the volume element and time dependent terms and the odd character of the momentum vector \( \Bk \) can be used to show that these terms integrate out to zero.  Let's compute the integral by averaging the momentum operator using both parameterization sign options.  First write
\begin{equation}\label{eqn:noetherCurrentScalarField:540}
f(\Bk) =
  a_{\Bk} a_{-\Bk} e^{-2 i \omega_k t }
+ a_{\Bk}^\dagger a_{-\Bk}^\dagger e^{2 i \omega_k t },
\end{equation}
so
\begin{equation}\label{eqn:noetherCurrentScalarField:560}
\begin{aligned}
\int d^3 k \Bk f(\Bk)
&=
\inv{2} \int d^3 k \Bk f(\Bk)
+
\inv{2} \int d^3 k' \Bk' f(\Bk')
\\&=
\inv{2} \int_0^\infty k^2 dk \int_0^\pi \sin\theta d\theta \int_0^{2\pi}
k \kcap(\theta, \phi) %\lr{ \sin\theta \cos\phi, \sin\theta \sin\phi, \cos\theta }
f(\Bk)
+
\inv{2} \int_{-\infty}^0 k^2 d(-k) \int_0^\pi \sin\theta d\theta \int_0^{2\pi}
(-k) \kcap(\theta, \phi) %\lr{ \sin\theta \cos\phi, \sin\theta \sin\phi, \cos\theta }
f(-\Bk)
\\&=
\inv{2} \int_0^\pi \sin\theta d\theta \int_0^{2\pi} d\phi \kcap
\lr{
\int_0^\infty k^3 dk f(\Bk)
+
\int_{-\infty}^0 k^3 dk f(-\Bk)
}
\\&=
\inv{2} \int_0^\pi \sin\theta d\theta \int_0^{2\pi} d\phi \kcap
\lr{
\int_0^\infty k^3 dk f(\Bk)
-
\int_{0}^\infty k^3 dk f(\Bk)
}
\\&= 0,
\end{aligned}
\end{equation}
so the momentum is reduced to
\begin{equation}\label{eqn:noetherCurrentScalarField:580}
\begin{aligned}
\Be_n \int d^3 x T^{0 n}
&=
\inv{2}
\int d^3 k \Bk
\lr{
  a_{\Bk} a_{\Bk}^\dagger
+ a_{\Bk}^\dagger a_{\Bk}
}
\\&=
\inv{2}
\int d^3 k \Bk
\lr{
  2 a_{\Bk}^\dagger a_\Bk
+ \antisymmetric{a_\Bk}{a_{\Bk}^\dagger}
}
\\&=
\int d^3 k \Bk
\lr{
  a_{\Bk}^\dagger a_\Bk
+ \inv{2} \deltathree(0)
}.
\end{aligned}
\end{equation}

An argument like that of \citep{peskin1995introduction} can be used to dismiss the unphysical infinity associated with the ground state energy level, leaving just
\boxedEquation{eqn:noetherCurrentScalarField:600}{
\Be_n \int d^3 x T^{0 n}
=
\int d^3 k \Bk
a_{\Bk}^\dagger
  a_{\Bk}
.
}

\makeSubAnswer{}{problem:noetherCurrentScalarField:1:b}

For \( \mu = m \ne 0 \), and \( \nu = n \ne 0 \), we have
%
\begin{equation}\label{eqn:noetherCurrentScalarField:620}
\int d^3 x \pi^m \pi^n
=
\inv{2}
\int d^3 k \frac{ k^m k^n }{\omega_k}
\lr{
  a_{\Bk} a_{-\Bk} e^{-2 i \omega_k t }
- a_{\Bk} a_{\Bk}^\dagger
- a_{-\Bk}^\dagger a_{-\Bk}
+ a_{-\Bk}^\dagger a_{\Bk}^\dagger e^{2 i \omega_k t }
}.
\end{equation}

Can the time dependent terms be killed in this case?

\makeSubAnswer{}{problem:noetherCurrentScalarField:1:c}

TODO: some stuff is wrong here.

For \( \nu \ne 0 \)
%
\begin{equation}\label{eqn:noetherCurrentScalarField:160}
\begin{aligned}
\int d^3 x \pi^\mu \pi^\nu
&=
-\inv{2}
\int d^3 k \frac{k^\mu k^\nu}{\omega_k}
\lr{
- a_{\Bk} a_{-\Bk} e^{- 2 i \omega_k t }
- a_{\Bk} a_{\Bk}^\dagger
- a_{\Bk}^\dagger a_{\Bk}
- a_{\Bk}^\dagger a_{-\Bk}^\dagger e^{ 2 i \omega_k t }
}
\\&=
 \inv{2}
\int d^3 k \frac{k^\mu k^\nu}{\omega_k}
\lr{
  a_{\Bk} a_{-\Bk} e^{- 2 i \omega_k t }
+ a_{\Bk} a_{\Bk}^\dagger
+ a_{\Bk}^\dagger a_{\Bk}
+ a_{\Bk}^\dagger a_{-\Bk}^\dagger e^{ 2 i \omega_k t }
}.
\end{aligned}
\end{equation}

Here's a summary of these products

\begin{subequations}
\label{eqn:noetherCurrentScalarField:260}
\begin{equation}\label{eqn:noetherCurrentScalarField:300}
\int d^3 x \pi^0 \pi^0
=
-\inv{2}
\int d^3 k \omega_k
\lr{
  a_{\Bk} a_{-\Bk} e^{-2 i \omega_k t }
- a_{\Bk} a_{\Bk}^\dagger
- a_{\Bk}^\dagger a_{\Bk}
+ a_{\Bk}^\dagger a_{-\Bk}^\dagger e^{2 i \omega_k t }
},
\end{equation}
\begin{equation}\label{eqn:noetherCurrentScalarField:280}
\begin{aligned}
\int d^3 x \pi^n \pi^0
&= \int d^3 x \pi^0 \pi^n
\\&=
-\inv{2}
\int d^3 k k^n
\lr{
  a_{\Bk} a_{-\Bk} e^{-2 i \omega_k t }
- a_{\Bk} a_{\Bk}^\dagger
- a_{\Bk}^\dagger a_{\Bk}
+ a_{\Bk}^\dagger a_{-\Bk}^\dagger e^{2 i \omega_k t }
},
\end{aligned}
\end{equation}
%\begin{equation}\label{eqn:noetherCurrentScalarField:320}
%\int d^3 x \pi^n \pi^n
%=
% \inv{2}
%\int d^3 k \frac{k^n k^n}{\omega_k}
%\lr{
%  a_{\Bk} a_{-\Bk} e^{- 2 i \omega_k t }
%+ a_{\Bk} a_{\Bk}^\dagger
%+ a_{\Bk}^\dagger a_{\Bk}
%+ a_{\Bk}^\dagger a_{-\Bk}^\dagger e^{ 2 i \omega_k t }
%},
%\end{equation}
\begin{equation}\label{eqn:noetherCurrentScalarField:340}
\int d^3 x \pi^m \pi^n
=
\inv{2}
\int d^3 k \frac{k^m k^n}{\omega_k}
\lr{
  a_{\Bk} a_{-\Bk} e^{- 2 i \omega_k t }
+ a_{\Bk} a_{\Bk}^\dagger
+ a_{\Bk}^\dagger a_{\Bk}
+ a_{\Bk}^\dagger a_{-\Bk}^\dagger e^{ 2 i \omega_k t }
}.
\end{equation}
\end{subequations}

For the mass term it was previously found that
%
\begin{equation}\label{eqn:noetherCurrentScalarField:180}
\inv{2} \int d^3 x \mu^2 \phi^2
=
\frac{\mu^2}{4}
\int
d^3 k
\inv{ \omega_k }
\lr{
 a_{-\Bk} a_{\Bk} e^{- 2 i \omega_k t }
+a_{-\Bk}^\dagger a_{\Bk}^\dagger e^{2 i \omega_k t }
+a_{\Bk} a_{\Bk}^\dagger
+a_{\Bk}^\dagger a_{\Bk}
}.
\end{equation}

The Hamiltonian component has been previously calculated, and resolves to
%
\begin{equation}\label{eqn:noetherCurrentScalarField:360}
\int d^3 x T^{00}
=
\inv{2}
\int d^3 k
\omega_k
\lr{
  a_{\Bk} a_{\Bk}^\dagger
+ a_{\Bk}^\dagger a_{\Bk}
}.
\end{equation}

The other diagonal components, for \( r \ne s \ne t \) are
\begin{equation}\label{eqn:noetherCurrentScalarField:380}
\begin{aligned}
\int d^3 x T^{rr}
&=
\int d^3 x
\lr{
\inv{2} \sum_{m = r,0} \pi^m \pi^m - \inv{2} \sum_{m = s,t} \pi^m \pi^m - \inv{2} \mu^2 \phi^2
}
\\&=
\inv{4}
\int d^3 k \frac{(k^r)^2 - (k^s)^2 - (k^t)^2 - \mu^2}{\omega_k}
\lr{
  a_{\Bk} a_{-\Bk} e^{- 2 i \omega_k t }
+ a_{\Bk} a_{\Bk}^\dagger
+ a_{\Bk}^\dagger a_{\Bk}
+ a_{\Bk}^\dagger a_{-\Bk}^\dagger e^{ 2 i \omega_k t }
}
-\inv{4}
\int d^3 k \omega_k
\lr{
  a_{\Bk} a_{-\Bk} e^{-2 i \omega_k t }
- a_{\Bk} a_{\Bk}^\dagger
- a_{\Bk}^\dagger a_{\Bk}
+ a_{\Bk}^\dagger a_{-\Bk}^\dagger e^{2 i \omega_k t }
}
\\&=
\inv{4}
\int d^3 k \frac{(k^r)^2 - (k^s)^2 - (k^t)^2 - \mu^2 - \omega_k^2}{\omega_k}
\lr{
  a_{\Bk} a_{-\Bk} e^{- 2 i \omega_k t }
+ a_{\Bk}^\dagger a_{-\Bk}^\dagger e^{ 2 i \omega_k t }
}
+
\inv{4}
\int d^3 k \frac{(k^r)^2 - (k^s)^2 - (k^t)^2 - \mu^2 + \omega_k^2}{\omega_k}
\lr{
  a_{\Bk} a_{\Bk}^\dagger
+ a_{\Bk}^\dagger a_{\Bk}
}
\\&=
\inv{2}
\int d^3 k \frac{  (k^r)^2 - \omega_k^2}{\omega_k}
\lr{
  a_{\Bk} a_{-\Bk} e^{- 2 i \omega_k t }
+ a_{\Bk}^\dagger a_{-\Bk}^\dagger e^{ 2 i \omega_k t }
}
+
\inv{2}
\int d^3 k \frac{  (k^r)^2}{\omega_k}
\lr{
  a_{\Bk} a_{\Bk}^\dagger
+ a_{\Bk}^\dagger a_{\Bk}
}.
\end{aligned}
\end{equation}

This doesn't have the nice cancellation that killed the time dependent terms in the Hamiltonian.  Such cancellation also doesn't appear in the off diagonal energy-momentum tensor components, which are
%
\begin{equation}\label{eqn:noetherCurrentScalarField:400}
\begin{aligned}
\int d^3 x T^{n 0}
&=
\int d^3 x T^{n 0}
\\&=
-\inv{2}
\int d^3 k k^n
\lr{
  a_{\Bk} a_{-\Bk} e^{-2 i \omega_k t }
- a_{\Bk} a_{\Bk}^\dagger
- a_{\Bk}^\dagger a_{\Bk}
+ a_{\Bk}^\dagger a_{-\Bk}^\dagger e^{2 i \omega_k t }
},
\end{aligned}
\end{equation}
%
and for \( m \ne n \ne 0 \)
\begin{equation}\label{eqn:noetherCurrentScalarField:420}
\int d^3 x T^{m n}
=
\inv{2}
\int d^3 k \frac{k^m k^n}{\omega_k}
\lr{
  a_{\Bk} a_{-\Bk} e^{- 2 i \omega_k t }
+ a_{\Bk} a_{\Bk}^\dagger
+ a_{\Bk}^\dagger a_{\Bk}
+ a_{\Bk}^\dagger a_{-\Bk}^\dagger e^{ 2 i \omega_k t }
}.
\end{equation}

The \cref{eqn:noetherCurrentScalarField:400} result has time dependence that the stated result does not (but is linear in \( \Bk \) as desired)?  Did I miss something?
} % answer

%}
%\EndArticle

   \chapter{1st Noether theorem, spacetime translation current, energy momentum tensor, dilatation current.}
      %
% Copyright � 2017 Peeter Joot.  All Rights Reserved.
% Licenced as described in the file LICENSE under the root directory of this GIT repository.
%
\input{../latex/blogpost.tex}
\renewcommand{\basename}{qftLecture8}
\renewcommand{\dirname}{notes/phy2403/}
\newcommand{\keywords}{PHY2403H}
\input{../latex/peeter_prologue_print2.tex}

%\usepackage{phy2403}
\usepackage{peeters_braket}
\usepackage{peeters_layout_exercise}
\usepackage{peeters_figures}
\usepackage{mathtools}
\usepackage{siunitx}
\usepackage{macros_cal} % LL

\beginArtNoToc
\generatetitle{PHY2403H Quantum Field Theory.  Lecture 8: 1st Noether theorem, spacetime translation current, energy momentum tensor, dilatation current.  Taught by Prof.\ Erich Poppitz}
%\chapter{1st Noether theorem, dilatation current}
\label{chap:qftLecture8}

\paragraph{DISCLAIMER: Very rough notes from class, with some additional side notes.}

These are notes for the UofT course PHY2403H, Quantum Field Theory I, taught by Prof. Erich Poppitz fall 2018.
%, covering \textchapref{{1}} \citep{peskin1995introduction} content.

\section{1st Noether theorem.}

Recall that, given a transformation
\begin{dmath}\label{eqn:qftLecture8:20}
\phi(x) \rightarrow \phi(x) + \delta \phi(x),
\end{dmath}
such that the transformation of the Lagrangian is only changed by a total derivative
\begin{dmath}\label{eqn:qftLecture8:40}
\LL(\phi, \partial_\mu \phi) \rightarrow
\LL(\phi, \partial_\mu \phi)
+ \partial_\mu J_\epsilon^\mu,
\end{dmath}
then there is a conserved current
\begin{dmath}\label{eqn:qftLecture8:60}
j^\mu = \PD{(\partial_\mu \phi)}{\LL} \delta_\epsilon \phi - J_\epsilon^\mu.
\end{dmath}
Here \( \epsilon \) is an x-independent quantity (i.e. a \underline{global symmetry}).
This is in contrast to ``gauge symmetries'', which can be more accurately be categorized as a redundancy in the description.

As an example, give
\begin{dmath}\label{eqn:qftLecture8:80}
\phi(x) \rightarrow \phi(x) - a^\mu \partial_\mu \phi
\end{dmath}
\begin{dmath}\label{eqn:qftLecture8:100}
\LL(\phi, \partial_\mu \phi) \rightarrow
\LL(\phi, \partial_\mu \phi)
- a^\mu \partial_\mu \LL
\LL(\phi, \partial_\mu \phi)
+ \partial_\mu \lr{ -{\delta^\mu}_\nu a^\nu \partial_\mu \LL }
\end{dmath}
Here \( J^\mu_\epsilon = \evalbar{J^\mu_\epsilon}{\epsilon = a^\nu} \), and the current is
\begin{dmath}\label{eqn:qftLecture8:120}
J^\mu = (\partial^\mu \phi)(-a^\nu \partial_\nu \phi) + {\delta^{\mu}}_\nu a^\nu \LL.
\end{dmath}
In particular, we have one such current for each \( \nu \), and we write
\begin{dmath}\label{eqn:qftLecture8:140}
{T^\mu}_\nu = -(\partial^\mu \phi)(\partial_\nu \phi) + {\delta^{\mu}}_\nu \LL.
\end{dmath}
By Noether's theorem, we must have
\begin{dmath}\label{eqn:qftLecture8:160}
\partial_\mu
{T^\mu}_\nu = 0, \quad \forall \nu.
\end{dmath}
Fixme: check this!

\paragraph{Example: our potential Lagrangian}

\begin{dmath}\label{eqn:qftLecture8:180}
\LL = \inv{2} \partial^\mu \phi \partial_\nu \phi - \frac{m^2}{2} \phi^2 - \frac{\lambda}{4} \lambda^4
\end{dmath}
Written with upper indexes
\begin{dmath}\label{eqn:qftLecture8:200}
T^{\mu\nu}
= -(\partial^\mu \phi)(\partial^\nu \phi) + g^{\mu\nu} \LL
= -(\partial^\mu \phi)(\partial^\nu \phi) + g^{\mu\nu} \lr{
\inv{2} \partial^\alpha \phi \partial_\nu \phi - \frac{m^2}{2} \phi^2 - \frac{\lambda}{4} \lambda^4
}
\end{dmath}

There are 4 conserved currents \( J^{\mu(\nu)} = T^{\mu\nu} \).  Observe that this is symmetric (\( T^{\mu\nu} = T^{\nu\mu} \)).

We have four associated charges
\begin{dmath}\label{eqn:qftLecture8:220}
Q^\nu = \int d^3 x T^{0 \nu}.
\end{dmath}
We call
\begin{dmath}\label{eqn:qftLecture8:240}
Q^0 = \int d^3 x T^{0 0},
\end{dmath}
the energy density, and call
\begin{dmath}\label{eqn:qftLecture8:260}
P^i = \int d^3 x T^{0 i},
\end{dmath}
(i = 1,2,3) the momentum density.

writing this out explicitly the energy density is
\begin{dmath}\label{eqn:qftLecture8:280}
T^{00}
= - \dot{\phi}^2 + \inv{2} \lr{ \dot{\phi}^2 - (\spacegrad \phi)^2 - \frac{m^2}{2} - \frac{\lambda}{4} \phi^4}
= -\lr{
\inv{2} \dot{\phi}^2 + \inv{2} (\spacegrad \phi)^2 + \frac{m^2}{2} + \frac{\lambda}{4} \phi^4
},
\end{dmath}
and
\begin{dmath}\label{eqn:qftLecture8:300}
T^{0i} = \partial^0 \partial^i \phi,
\end{dmath}
\begin{dmath}\label{eqn:qftLecture8:320}
P^{i} = -\int d^3 x\partial^0 \phi \partial^i \phi
\end{dmath}
Since the energy density is negative definite (due to an arbitrary choice of translation sign), let's redefine \( T^{\mu\nu} \) to have a positive sign
\begin{dmath}\label{eqn:qftLecture8:340}
T^{00}
\equiv
\inv{2} \dot{\phi}^2 + \inv{2} (\spacegrad \phi)^2 + \frac{m^2}{2} + \frac{\lambda}{4} \phi^4,
\end{dmath}
and
\begin{dmath}\label{eqn:qftLecture8:360}
P^{i} = \int d^3 x\partial^0 \phi \partial^i \phi
\end{dmath}

As an operator we have
\begin{dmath}\label{eqn:qftLecture8:380}
\hatQ = \int d^3 x \hatT^{00} =
\int d^3 x
\lr{
\inv{2} \pihat^2 + \inv{2} (\spacegrad \phihat)^2 + \frac{m^2}{2} \phihat^2 + \frac{\lambda}{4} \phihat^4
}.
\end{dmath}
\begin{dmath}\label{eqn:qftLecture8:400}
\hatP^{i} = \int d^3 x \pihat \partial^i \phi
\end{dmath}

We showed that
\begin{dmath}\label{eqn:qftLecture8:420}
\ddt \hatO = i \antisymmetric{\hatH}{\hatO}
\end{dmath}
This implied that \( \phihat, \pihat \) obey the classical EOMs
\begin{equation}\label{eqn:qftLecture8:440}
\ddt \phihat = i \antisymmetric{\hatH}{\phihat} = \ddt{\pihat}
\end{equation}
\begin{equation}\label{eqn:qftLecture8:460}
\ddt \pihat = i \antisymmetric{\hatH}{\pihat} = ...
\end{equation}

In terms of creation and annihilation operators (for the \( \lambda = 0 \) free field), up to a constant
\begin{dmath}\label{eqn:qftLecture8:480}
\hatH
= \int d^3 x \hatT^{00}
= \int \frac{d^3 p}{(2 \pi)^3} \omega_\Bp \hata_\Bp^\dagger \hata_\Bp
\end{dmath}
Fixme: show that:
\begin{dmath}\label{eqn:qftLecture8:500}
\hatP^i
= \int d^3 x \pihat \partial^i \phihat
= \cdots
= \int \frac{d^3 p}{(2 \pi)^3} p^i \hata_\Bp^\dagger \hata_\Bp
\end{dmath}
Now we see the energy and momentum as conserved quantities associated with spacetime translation.

\section{Unitary operators}

In QM we say that \( \hat{\Bp} \) ``generates translations''.

With \( \hat{\Bp} \equiv -i \Hbar \spacegrad \) that translation is
\begin{equation}\label{eqn:qftLecture8:520}
\hatU = e^{i \Ba \cdot \hat{\Bp}} = e^{\Ba \cdot \spacegrad}
\end{equation}

In particular
\begin{dmath}\label{eqn:qftLecture8:540}
\bra{\Bx} \hatU \ket{\psi} = e^{\Ba \cdot \hat{\Bp} } \psi(\Bx) = \psi(\Bx + \Ba).
\end{dmath}

In one dimension
\begin{dmath}\label{eqn:qftLecture8:560}
\hatU \hat{x} \hatU^\dagger
=
e^{\Ba \cdot \hat{p} } \psi(\Bx)
e^{-\Ba \cdot \hat{p} }
= \hat{\Bx} + a \hat{\BOne}.
\end{dmath}
This uses the Baker-Campbell-Hausdorff formula.
\maketheorem{ Baker-Campbell-Hausdorff }{thm:qftLecture8:580}{
\begin{dmath}\label{eqn:qftLecture8:600}
e^{B} A e^{-B} = \sum_{n = 0}^\infty \inv{n!} \antisymmetric{B \cdots}{\antisymmetric{B}{A}},
\end{dmath}
where the n-th commutator is denoted above

\begin{itemize}
\item \( n = 1 \) : \( \antisymmetric{B}{A} \)
\item \( n = 2 \) : \( \antisymmetric{B}{\antisymmetric{B}{A}} \)
\item \( n = 3 \) : \( \antisymmetric{B}{\antisymmetric{B}{\antisymmetric{B}{A}}} \)
\end{itemize}
} % theorem

Proof:

\begin{dmath}\label{eqn:qftLecture8:620}
f(t) = e^{tB} A e^{-tB}
= f(0) + t f'(0) + \frac{t^2}{2} f''(0) + \cdots \frac{t^n}{n!} f^{(n)}(0)
\end{dmath}

\begin{dmath}\label{eqn:qftLecture8:640}
f(0) = A
\end{dmath}
\begin{dmath}\label{eqn:qftLecture8:660}
f'(t)
=
e^{tB} B A e^{-tB}
+
e^{tB} A (-B) e^{-tB}
=
e^{tB} \antisymmetric{B}{A} e^{-tB}
\end{dmath}
\begin{dmath}\label{eqn:qftLecture8:680}
f''(t) =
e^{tB} B \antisymmetric{B}{A} e^{-tB}
+
e^{tB} \antisymmetric{B}{A} (-B) e^{-tB}
=
e^{tB} \antisymmetric{B}{\antisymmetric{B}{A}} e^{-tB}.
\end{dmath}
From
\begin{dmath}\label{eqn:qftLecture8:700}
f(1)
= f(0) + f'(0) + \inv{2} f''(0) + \cdots \inv{n!} f^{(n)}(0)
\end{dmath}
we have
\begin{dmath}\label{eqn:qftLecture8:720}
e^{B} A e^{-B} = A +
\antisymmetric{B}{A} + \inv{2} \antisymmetric{B}{\antisymmetric{B}{A}} + \cdots
\end{dmath}

Example:
\begin{dmath}\label{eqn:qftLecture8:740}
e^{a \partial_x} x e^{-a \partial_x } = x + a \antisymmetric{\partial_x}{x} + \cdots
= x + a.
\end{dmath}

\paragraph{Application:}

\begin{dmath}\label{eqn:qftLecture8:760}
e^{i \text{Hermitian} } = \text{unitary}
\end{dmath}
\begin{dmath}\label{eqn:qftLecture8:860}
e^{i \text{Hermitian} } \times
e^{-i \text{Hermitian} }
= 1
\end{dmath}
So
\begin{dmath}\label{eqn:qftLecture8:780}
\hatU(\Ba) =
e^{i a^j \hat{p}^j }
\end{dmath}
is a unitary operator representing finite translations in a Hilbert space.

\begin{dmath}\label{eqn:qftLecture8:800}
\hatU(\Ba) \phihat(\Bx) \hatU^\dagger(\Ba)
=
e^{i a^j \hat{p}^j }
\phihat(\Bx)
e^{-i a^k \hat{p}^k }
=
\phihat(\Bx)
+ i a^j \antisymmetric{\hatP^j}{\phihat(\Bx)} + \frac{-a^{j_1} a^{j_2}}{2} \antisymmetric{\hatP^{j_1}}{\antisymmetric{\hatP^{j_2}}{\phihat(\Bx)}}
\end{dmath}

\begin{dmath}\label{eqn:qftLecture8:820}
\antisymmetric{\hatP^j}{\phihat(\Bx)}
=
\int d^3 y \antisymmetric{\pihat(\By) \partial^j \phihat(\By)}{\phihat(\Bx)}
=
\int d^3 y \antisymmetric{\pihat(\By)}{\phihat(\Bx} \partial^j \phihat(\By)
=
\int d^3 y (-i ) \delta^3(\By - \Bx) \partial^j \phihat(\By)
=
-i \partial^j \phihat(\Bx).
\end{dmath}

\begin{dmath}\label{eqn:qftLecture8:840}
\hatU(\Ba) \phihat(\Bx) \hatU^\dagger(\Ba)
= \phihat(\Bx) + i a^j (-i) \partial^j \phihat(\Bx) + \cdots
= \phihat(\Bx) + a^j \partial^j \phihat(\Bx) + \cdots
= \phihat(\Bx + \Ba)
\end{dmath}

\section{Continuous symmetries}

For all infinitesimal transformations, continuous symmetries lead to conserved charges \( Q \).  In QFT we map these charges to Hermitian operators \( Q \rightarrow \hatQ \).  We say that these charges are ``generators of the corresponding symmetry'' through unitary operators
\begin{dmath}\label{eqn:qftLecture8:880}
\hatU = e^{i \text{parameter} \hatQ}.
\end{dmath}
These represent the action of the symmetry in the Hilbert space.

\paragraph{Example: spatial translation}
\begin{dmath}\label{eqn:qftLecture8:900}
\hatU(\Ba) = e^{i \Ba \cdot \hat{\BP}}
\end{dmath}
\paragraph{Example: time translation}
\begin{dmath}\label{eqn:qftLecture8:920}
\hatU(t) = e^{i t \hat{H}}.
\end{dmath}

\section{Classical scalar theory}

For \( d > 2 \) let's look at
\begin{dmath}\label{eqn:qftLecture8:940}
S =
\int d^d x \lr{
\inv{2} \partial
\inv{2} \partial^\mu \phi \partial_\nu \phi - \frac{m^2}{2} \phi^2 - \lambda \phi^{d-2}
}
\end{dmath}
Take \( m^2, \lambda \rightarrow 0 \), the free massless scalar field.
We have a shift symmetry in this case since \( \phi(x) \rightarrow \phi(x) + \text{constant} \).
The current is just
\begin{dmath}\label{eqn:qftLecture8:960}
j^\mu
= \PD{(\partial_\mu \phi)}{\phi} \delta \phi - \cancel{J^\mu}
= \text{constant} \times \partial^\mu \phi
= \partial^\mu \phi,
\end{dmath}
where the constant factor has been set to one.
This current is clearly conserved since \( \partial_\mu J^\mu = \partial_\mu \partial^\mu \phi = 0\) (the equation of motion).

These are called ``Goldstein bosons''.

With \( m \rightarrow 0, \lambda \ne 0 \) we have
\begin{dmath}\label{eqn:qftLecture8:980}
S =
\int d^d x \lr{
\inv{2} \partial
\inv{2} \partial^\mu \phi \partial_\nu \phi - \tilde{\lambda} \phi^{d-2}
}
\end{dmath}
Here we have a scale or dilatation invariance
\begin{dmath}\label{eqn:qftLecture8:1000}
x \rightarrow e^{\lambda} x
\end{dmath}
\begin{dmath}\label{eqn:qftLecture8:1020}
\phi(x) \rightarrow \phi'(x') = e^{-(d-2) \lambda/2} \phi
\end{dmath}

\begin{dmath}\label{eqn:qftLecture8:1040}
d^3 x = e^{d\lambda} d^3 x
\end{dmath}

\begin{dmath}\label{eqn:qftLecture8:1060}
(\partial_\mu \phi)^2 \rightarrow e^{-(d-2)\lambda} e^{-2 \lambda} \lr{ \partial_\mu \phi(x) }^2
\end{dmath}

\begin{dmath}\label{eqn:qftLecture8:1080}
\phi^{2 d/(d -2)} \rightarrow \lr{\phi'(x')}^{2d/(d-2)}
=
e^{-(d-2)/2 \times 2d/(d-2) \lambda} \lr{ \phi }^{2d/(d-2)}
\end{dmath}

(switching to \(d = 4\))
to find Noether currrent ...

\begin{dmath}\label{eqn:qftLecture8:1100}
\delta_\lambda \phi = \phi'(x) - \phi(x)
= - \lambda(1 + x^\mu \partial_\mu ) \phi
\end{dmath}

In free field with \( \tilde{\lambda} = 0 \)

\begin{dmath}\label{eqn:qftLecture8:1120}
\delta_\lambda \lr{ \inv{2} \partial_\mu \phi \partial^\mu \phi}
= \partial^\mu \delta_\lambda (\partial_\mu \phi)
= \partial^\mu \phi \partial_\mu ( - \lambda( 1 + x^\nu \partial_\nu )\phi )
= -\lambda \lr{ \partial^\mu \phi \partial_\mu \phi + \partial^\mu \phi \partial_\mu \lr{ x^\nu \partial_\nu \phi } }
= -\lambda \lr{ 2 \partial^\mu \phi \partial_\mu \phi + \partial^\mu \phi x^\nu \partial_\nu \phi }
=
\end{dmath}

Wrong.  Starting over.

We need the transformation of the derivatives of \( \phi \).

\begin{dmath}\label{eqn:qftLecture8:1140}
\inv{2} \partial^\mu \phi \partial_\mu \phi
\rightarrow
\inv{2} {\partial'}^\mu \phi' {\partial'}_\mu \phi'
=
\inv{2} e^{-4\lambda}
\partial^\mu \phi \partial_\mu \phi
\end{dmath}
Let
\begin{dmath}\label{eqn:qftLecture8:1160}
\begin{aligned}
\Phi'(x')
&=
{\partial'}^\mu \phi' {\partial'}_\mu \phi' \\
\Phi(x) &=
\partial^\mu \phi \partial_\mu \phi,
\end{aligned}
\end{dmath}

so
\begin{dmath}\label{eqn:qftLecture8:1180}
\begin{aligned}
\Phi'(x') &= e^{-4 \lambda} \Phi(x) \\
\Phi'(e^{\lambda} x) &= e^{-4 \lambda} \Phi(x)
\end{aligned}
\end{dmath}
...
so

\begin{dmath}\label{eqn:qftLecture8:1200}
\delta \LL = - \lambda \lr{ 4 + x^\mu \partial_\mu } \lr{
\partial^\mu \phi \partial_\mu \phi
}
=
-\inv{2} \lambda \partial_\mu \lr{ x^\mu \partial_\nu \phi \partial^\nu \phi}
=
-\lambda \lr{ \partial_\mu x^\mu
\inv{2} (\partial_\nu \phi)^2 + x^\mu \partial_\mu \lr{ \inv{2} \partial_\nu \phi \partial^\nu \phi }
}
\end{dmath}

so
\begin{dmath}\label{eqn:qftLecture8:1220}
\begin{aligned}
J^\mu_\lambda &= \frac{-x^\mu}{2} \partial_\nu \phi \partial^\nu \phi \\
\delta \phi &= - \lr{ 1 + x^\mu \partial_\mu } \phi
\end{aligned}
\end{dmath}

\begin{dmath}\label{eqn:qftLecture8:1240}
j^\mu = \PD{(\partial_\mu \phi)}{\LL} \delta \phi - J^\mu
\end{dmath}

In the end, find
\begin{dmath}\label{eqn:qftLecture8:1260}
j^\mu = -\partial^\mu \phi \lr{ 1 + x^\nu \partial_\nu } \phi + \inv{2} x^\mu \partial_nu \phi \partial^\nu \phi
\end{dmath}

Flipping signs
\begin{dmath}\label{eqn:qftLecture8:1280}
j^\mu_{\text{dil}} = \partial^\mu \phi \lr{ 1 + x^\nu \partial_\nu } \phi - \inv{2} x^\mu \partial_nu \phi \partial^\nu \phi
\end{dmath}
or
\begin{dmath}\label{eqn:qftLecture8:1300}
j^\mu_{\text{dil}} = -x_\nu T^{\nu \mu} - \inv{2} \partial^\mu \phi^2
\end{dmath}

The current and \( T^{\mu\nu} \) can both be redefined \( j^{\mu'} = j^\mu + \partial_\nu C^{\nu\mu} \) adding an antisymetric \( C^{\mu\nu} = -C^{\nu\mu} \)

\begin{dmath}\label{eqn:qftLecture8:1320}
j^\mu_{\text{dil conformal}} = - x_\nu T^{\nu\mu}_{\text{conformal}}
\end{dmath}

\begin{dmath}\label{eqn:qftLecture8:1340}
\partial_\mu
j^\mu_{\text{dil conformal}} = - {{T_{\text{conformal}}}^\mu}_\mu
\end{dmath}

%}
%\EndArticle
\EndNoBibArticle

      \section{Problems.}
         \input{qftLukeProblemSet1Problem5.tex}
   \chapter{Unbroken and spontaneously broken symmetries, Higgs Lagrangian, scale invariance, Lorentz invariance, angular momentum quantization.}
      %
% Copyright � 2017 Peeter Joot.  All Rights Reserved.
% Licenced as described in the file LICENSE under the root directory of this GIT repository.
%
%%%\input{../latex/blogpost.tex}
%%%\renewcommand{\basename}{qftLecture9}
%%%\renewcommand{\dirname}{notes/phy2403/}
%%%\newcommand{\keywords}{PHY2403H}
%%%\input{../latex/peeter_prologue_print2.tex}
%%%
%%%%\usepackage{phy2403}
%%%\usepackage{peeters_braket}
%%%%\usepackage{peeters_layout_exercise}
%%%\usepackage{peeters_figures}
%%%\usepackage{mathtools}
%%%\usepackage{siunitx}
%%%\usepackage{enumerate}
%%%\usepackage{macros_cal} % LL
%%%\usepackage{mmacells}
%%%
%%%%\newcommand{\munu}[0]{\mu\nu}
%%%\newcommand{\ultensor}[3]{{{#1}^{#2}}_{#3}}
%%%%s/\\ultensor{\([^}]\+\)}{\([^}]\+\)}{\([^}]\+\)}/{{\1}^\2}_\3/g
%%%%s/ulLambda/ultensor{\\Lambda}/g
%%%\newcommand{\ulLambda}[2]{\ultensor{\Lambda}{#1}{#2}}
%%%\newcommand{\ulDelta}[2]{\ultensor{\delta}{#1}{#2}}
%%%
%%%\beginArtNoToc
%%%\generatetitle{PHY2403H Quantum Field Theory.  Lecture 9: Unbroken and spontaneously broken symmetries, Higgs Lagrangian, scale invariance, Lorentz invariance, angular momentum quantization.  Taught by Prof.\ Erich Poppitz}
\chapter{Unbroken and spontaneously broken symmetries, Higgs Lagrangian, scale invariance, Lorentz invariance, angular momentum quantization}
\label{chap:qftLecture9}

%\paragraph{DISCLAIMER: Very rough notes from class, with some additional side notes.}
%
%These are notes for the UofT course PHY2403H, Quantum Field Theory I, taught by Prof. Erich Poppitz fall 2018.
%%, covering \textchapref{{1}} \citep{peskin1995introduction} content.
%
\section{Last time}

We followed a sequence of operations
\begin{enumerate}
\item
Noether's theorem
\item \( \rightarrow \)
conserved currents
\item \( \rightarrow \)
charges (classical)
\item \( \rightarrow \)
``correspondence principle''
\item \( \rightarrow \hatQ \)
\end{enumerate}

\begin{itemize}
\item Hermitian operators
\item ``generators of symmetry"
\begin{dmath}\label{eqn:qftLecture9:20}
\hatU(\alpha) = e^{i \alpha \hatQ}
\end{dmath}
We found
\begin{dmath}\label{eqn:qftLecture9:40}
\hatU(\alpha) \phihat \hatU^\dagger(\alpha) = \phihat + i \alpha \antisymmetric{\hatQ}{\phihat} + \cdots
\end{dmath}
\end{itemize}

\paragraph{Example: internal symmetries:}
(non-spacetime), such as \( O(N)\) or \( U(1) \).

In QFT internal symmetries can have different ``\underline{modes of realization}''.

\begin{enumerate}[I]
\item ``Wigner mode''.  These are also called ``unbroken symmetries''.
\begin{dmath}\label{eqn:qftLecture9:60}
\hatQ \ket{0} = 0
\end{dmath}
i.e. \( \hatU(\alpha) \ket{0} = 0 \).
Ground state invariant.  Formally \( :\hatQ: \) annihilates \( \ket{0} \).
\( \antisymmetric{\hatQ}{\hatH} = 0 \) implies that all eigenstates are eigenstates of \( \hatQ \) in \( U(1) \).  Example from HW 1
\begin{dmath}\label{eqn:qftLecture9:80}
\hatQ = \text{``charge'' under \( U(1) \)}.
\end{dmath}
All states have definite charge, just live in QU.
\item ``Nambu-Goldstone mode'' (Landau-ginsburg).  This is also called a ``spontaneously broken symmetry''\footnote{
First encounter example (HWII, \( SU(2) \times SU(2) \rightarrow SU(2) \)).  Here a \( U(1) \) spontaneous broken symmetry.}.
\( H \) or \( L \) is invariant under symmetry, but ground state is not.
\end{enumerate}

Example:
\begin{dmath}\label{eqn:qftLecture9:100}
\LL = \partial_\mu \phi^\conj \partial^\mu \phi - V(\Abs{\phi}),
\end{dmath}
where
\begin{dmath}\label{eqn:qftLecture9:120}
V(\Abs{\phi}) = m^2 \phi^\conj \phi + \frac{\lambda}{4} \lr{ \phi^\conj \phi }^2.
\end{dmath}
When \( m^2 > 0 \) we have a Wigner mode, but when \( m^2 < 0 \) we have an issue: \( \phi = 0 \) is not a minimum of potential.
When \( m^2 < 0 \) we write
\begin{dmath}\label{eqn:qftLecture9:140}
V(\phi)
= - m^2 \phi^\conj \phi + \frac{\lambda}{4} \lr{ \phi^\conj \phi}^2
= \frac{\lambda}{4} \lr{
\lr{ \phi^\conj \phi}^2 - \frac{4}{\lambda} m^2 }
= \frac{\lambda}{4} \lr{
\phi^\conj \phi - \frac{2}{\lambda} m^2 }^2 - \frac{4 m^4}{\lambda^2},
\end{dmath}
or simply
\begin{dmath}\label{eqn:qftLecture9:780}
V(\phi)
=
\frac{\lambda}{4} \lr{ \phi^\conj \phi - v^2 }^2 + \text{const}.
\end{dmath}
The potential (called the Mexican hat potential) is illustrated in \cref{fig:mexicanHatPotential:mexicanHatPotentialFig1} for non-zero \( v \), and in
\cref{fig:mexicanHatPotential:mexicanHatPotentialFig2} for \( v = 0 \).
The following is a Mathematica code listing that can be used to play with this shape
\begin{mmaCell}[moredefined={potential, v},morepattern={x_, y_, \
v_},morefunctionlocal={x, y}]{Input}
  ClearAll[potential]
  potential[x_, y_, v_] := (\mmaPat{x}^2 + \mmaPat{y}^2 - v^2)^2

  Manipulate[
  Plot3D[ potential[x, y, v], \{x, -5, 5\}, \{y, -5, 5\}, PlotRange\
\(\pmb{\to}\)Full],
  \{\{v,4\}, 0, 10\}
  ]
\end{mmaCell}
\imageFigure{../figures/phy2403-quantum-field-theory/mexicanHatPotentialFig1}{Mexican hat potential.}{fig:mexicanHatPotential:mexicanHatPotentialFig1}{0.3}
\imageFigure{../figures/phy2403-quantum-field-theory/mexicanHatPotentialFig2}{Degenerate Mexican hat potential \( v = 0 \).}{fig:mexicanHatPotential:mexicanHatPotentialFig2}{0.3}
We choose to expand around some point on the minimum ring (it doesn't matter which one). % (P in the figure).
When there is no potential, we call the field massless (i.e. if we are in the minimum ring).
We expand as
\begin{dmath}\label{eqn:qftLecture9:160}
\phi(x) = v \lr{ 1 + \frac{\rho(x)}{v} } e^{i \alpha(x)/v },
\end{dmath}
so
\begin{dmath}\label{eqn:qftLecture9:180}
\frac{\lambda}{4}
\lr{\phi^\conj \phi - v^2}^2 =
\lr{
v^2 \lr{ 1 + \frac{\rho(x)}{v} }^2
- v^2
}^2
=
\frac{\lambda}{4}
v^4 \lr{ \lr{ 1 + \frac{\rho(x)}{v} }^2 - 1 }
=
\frac{\lambda}{4}
v^4
\lr{
   \frac{2 \rho}{v} + \frac{\rho^2}{v^2}
}^2.
\end{dmath}

\begin{dmath}\label{eqn:qftLecture9:200}
\partial_\mu \phi =
\lr{
v \lr{ 1 + \frac{\rho(x)}{v} } \frac{i}{v} \partial_\mu \alpha
+ \partial_\mu \rho
} e^{i \alpha}
\end{dmath}

so
\begin{dmath}\label{eqn:qftLecture9:220}
\LL
= \Abs{\partial \phi^\conj}^2 - \frac{\lambda}{4} \lr{ \Abs{\phi^\conj}^2 - v^2 }^2
=
\partial_\mu \rho \partial^\mu \rho + \partial_\mu \alpha \partial^\mu \alpha \lr{ 1 + \frac{\rho}{v} }
-
\frac{\lambda v^4}{4} \frac{ 4\rho^2}{v^2} + O(\rho^3)
=
\partial_\mu \rho \partial^\mu \rho
- \lambda v^2\rho^2
+
\partial_\mu \alpha \partial^\mu \alpha \lr{ 1 + \frac{\rho}{v} }.
\end{dmath}
We have two fields, \( \rho \) : a massive scalar field, the ``Higgs'', and a massless field \( \alpha \) (the Goldstone Boson).

\( U(1) \) symmetry acts on \( \phi(x) \rightarrow e^{i \omega } \phi(x) \) i.t.o \( \alpha(x) \rightarrow \alpha(x) + v \omega \).
\( U(1) \) global symmetry (broken) acts on the Goldstone field \( \alpha(x) \) by a constant shift.  (\(U(1)\) is still a symmetry of the Lagrangian.)

The current of the \( U(1) \) symmetry is:
\begin{dmath}\label{eqn:qftLecture9:240}
j_\mu = \partial_\mu \alpha \lr{ 1 + \text{higher dimensional \( \rho \) terms} }.
\end{dmath}

When we quantize
\begin{dmath}\label{eqn:qftLecture9:260}
\alpha(x) =
\int \frac{d^3p}{(2\pi)^3 \sqrt{ 2 \omega_p }} e^{i \omega_p t - i \Bp \cdot \Bx} \hata_\Bp^\dagger +
\int \frac{d^3p}{(2\pi)^3 \sqrt{ 2 \omega_p }} e^{-i \omega_p t + i \Bp \cdot \Bx} \hata_\Bp
\end{dmath}
\begin{dmath}\label{eqn:qftLecture9:280}
j^\mu(x) = \partial^\mu \alpha(x) =
\int \frac{d^3p}{(2\pi)^3 \sqrt{ 2 \omega_p }} \lr{ i \omega_\Bp - i \Bp } e^{i \omega_p t - i \Bp \cdot \Bx} \hata_\Bp^\dagger +
\int \frac{d^3p}{(2\pi)^3 \sqrt{ 2 \omega_p }} \lr{ -i \omega_\Bp + i \Bp } e^{-i \omega_p t + i \Bp \cdot \Bx} \hata_\Bp.
\end{dmath}

\begin{dmath}\label{eqn:qftLecture9:300}
j^\mu(x) \ket{0} \ne 0,
\end{dmath}
instead it creates a single particle state.

\section{Examples of symmetries}
In particle physics, examples of Wigner vs Nambu-Goldstone, ignoring gravity the only exact internal symmetry in the standard module is
\( (B\# - L\#) \), believed to be a \( U(1) \) symmetry in Wigner mode.

Here \(B\#\) is the Baryon number, and \( L\# \) is the Lepton number.  Examples:

\begin{itemize}
\item \( B(p) = 1 \), proton.
\item \( B(q) = 1/3 \), quark
\item \( B(e) = 1 \), electron
\item \( B(n) = 1 \), neutron.
\item \( L(p) = 1 \), proton.
\item \( L(q) = 0 \), quark.
\item \( L(e) = 0 \), electron.
\end{itemize}

The major use of global internal symmetries in the standard model is as ``approximate'' ones.  They become symmetries when one neglects some effect( ``terms in \( \LL \)'').
There are other approximate symmetries (use of group theory to find the Balmer series).
\paragraph{Example from HW2:}
QCD in limit
\begin{equation}\label{eqn:qftLecture9:320}
m_u = m_d = 0.
\end{equation}
\( m_u m_d \ll m_p \) (the products of the up-quark mass and the down-quark mass are much less than a composite one (name?)).
\( SU(2)_L \times SU(2)_R \rightarrow SU(2)_V \)
\paragraph{EWSB (Electro-Weak-Symmetry-Breaking) sector}
When the couplings \( g_2, g_1 = 0 \).  (\( g_2 \in SU(2), g_1 \in U(1) \)).

\section{Scale invariance}

\begin{dmath}\label{eqn:qftLecture9:340}
\begin{aligned}
x &\rightarrow e^{\lambda} x \\
\phi &\rightarrow e^{-\lambda} \phi \\
A_\mu &\rightarrow e^{-\lambda} A_\mu
\end{aligned}
\end{dmath}
Any unitary theory which is scale invariant is also \underline{conformal} invariant.  Conformal invariance means that angles are preserved.
The point here is that there is more than scale invariance.

We have classical internal global continuous symmetries.
These can be either
\begin{enumerate}
\item
``unbroken'' (Wigner mode)
\begin{dmath}\label{eqn:qftLecture9:360}
\hatQ\ket{0} = 0.
\end{dmath}
\item
``spontaneously broken''
\begin{dmath}\label{eqn:qftLecture9:380}
j^\mu(x) \ket{0} \ne 0
\end{dmath}
(creates Goldstone modes).
\item ``anomalous''.  Classical symmetries are not a symmetry of QFT.
Examples:
\begin{itemize}
\item Scale symmetry (to be studied in QFT II), although this is not truly internal.
\item In QCD again when \( \omega_\Bq = 0 \), a \( U(1\) symmetry (chiral symmetry) becomes exact, and cannot be preserved in QFT.
\item In the standard model (E.W sector), the Baryon number and Lepton numbers are not symmetries, but their difference \( B\# - L\# \) is a symmetry.
\end{itemize}
\end{enumerate}

\section{Lorentz invariance.}
We'd like to study the action of Lorentz symmetries on quantum states.  We are going to ``go by the book'', finding symmetries, currents, quantize, find generators, and so forth.

Under a Lorentz transformation
\begin{equation}\label{eqn:qftLecture9:400}
x^\mu \rightarrow {x'}^\mu = {\Lambda^\mu}_\nu x^\nu,
\end{equation}
We are going to consider infinitesimal Lorentz transformations
\begin{dmath}\label{eqn:qftLecture9:420}
{\Lambda^\mu}_\nu \approx
{\delta^\mu}_\nu + {\omega^\mu}_\nu
,
\end{dmath}
where \( {\omega^\mu}_\nu \) is small.
A Lorentz transformation \( \Lambda \) must satisfy \( \Lambda^\T G \Lambda = G \), or
\begin{dmath}\label{eqn:qftLecture9:800}
g_{\mu\nu} = \ulLambda{\alpha}{\mu} g_{\alpha \beta} \ulLambda{\beta}{\nu},
\end{dmath}
into which we insert the infinitesimal transformation representation
\begin{dmath}\label{eqn:qftLecture9:820}
0 =
- g_{\mu\nu} +
\lr{ {\delta^\alpha}_\mu + {\omega^\alpha}_\mu }
g_{\alpha \beta}
\lr{ {\delta^\beta}_\nu + {\omega^\beta}_\nu }
=
- g_{\mu\nu} +
\lr{
   g_{\mu \beta}
   +
   \omega_{\beta\mu}
}
\lr{ {\delta^\beta}_\nu + {\omega^\beta}_\nu }
=
- g_{\mu\nu} +
   g_{\mu \nu}
   +
   \omega_{\nu\mu}
+
\omega_{\mu\nu}
+
   \omega_{\beta\mu}
{\omega^\beta}_\nu.
\end{dmath}
The quadratic term can be ignored, leaving just
\begin{dmath}\label{eqn:qftLecture9:840}
0 =
   \omega_{\nu\mu}
+
\omega_{\mu\nu},
\end{dmath}
or
\begin{dmath}\label{eqn:qftLecture9:860}
   \omega_{\nu\mu} = - \omega_{\mu\nu}.
\end{dmath}
Note that \( \omega \) is a completely antisymmetric tensor, and like \( F_{\mu\nu} \) this has only 6 elements.
This means that the
infinitesimal transformation of the coordinates is
\begin{equation}\label{eqn:qftLecture9:440}
x^\mu \rightarrow {x'}^\mu \approx x^\mu + \omega^{\mu\nu} x_\nu,
\end{equation}
the field transforms as
\begin{equation}\label{eqn:qftLecture9:460}
\phi(x) \rightarrow \phi'(x') = \phi(x)
\end{equation}
or
\begin{dmath}\label{eqn:qftLecture9:760}
\phi'(x^\mu + \omega^{\mu\nu} x_\nu) =
\phi'(x) + \omega^{\mu\nu} x_\nu \partial_\mu\phi(x) = \phi(x),
\end{dmath}
so
\begin{dmath}\label{eqn:qftLecture9:480}
\delta \phi = \phi'(x) - \phi(x) =
-\omega^{\mu\nu} x_\nu \partial_\mu \phi.
\end{dmath}

Since \( \LL \) is a scalar
\begin{dmath}\label{eqn:qftLecture9:500}
\delta \LL
=
-\omega^{\mu\nu} x_\nu \partial_\mu \LL
=
-
\partial_\mu \lr{
   \omega^{\mu\nu} x_\nu \LL
}
+
(\partial_\mu x_\nu) \omega^{\mu\nu} \LL
=
\partial_\mu \lr{
-
   \omega^{\mu\nu} x_\nu \LL
},
\end{dmath}
since \( \partial_\nu x_\mu = g_{\nu\mu} \) is symmetric, and \( \omega \) is antisymmetric.
Our current is
\begin{dmath}\label{eqn:qftLecture9:520}
J^\mu_\omega
=
-
   \omega^{\mu\nu} x_\mu \LL
.
\end{dmath}
Our Noether current is
\begin{dmath}\label{eqn:qftLecture9:540}
j^\nu_{\omega^{\mu\rho}}
= \PD{\phi_{,\nu}}{\LL} \delta \phi - J^\mu_\omega
=
\partial^\nu \phi\lr{ - \omega^{\mu\rho} x_\rho \partial_\mu \phi } + \omega^{\nu \rho} x_\rho \LL
=
\omega^{\mu\rho}
\lr{
   \partial^\nu \phi\lr{ - x_\rho \partial_\mu \phi } + {\delta^{\nu}}_\mu x_\rho \LL
}
=
\omega^{\mu\rho} x_\rho
\lr{
   -\partial^\nu \phi \partial_\mu \phi + {\delta^{\nu}}_\mu \LL
}
\end{dmath}
We identify
\begin{dmath}\label{eqn:qftLecture9:560}
-
{T^\nu}_\mu =
   -\partial^\nu \phi \partial_\mu \phi + {\delta^{\nu}}_\mu \LL,
\end{dmath}
so the current is
\begin{equation}\label{eqn:qftLecture9:580}
j^\nu_{\omega_{\mu\rho}}
=
-\omega^{\mu\rho} x_\rho
{T^\nu}_\mu
=
-\omega_{\mu\rho} x^\rho
T^{\nu\mu}
.
\end{equation}
Define
\begin{dmath}\label{eqn:qftLecture9:600}
j^{\nu\mu\rho} = \inv{2} \lr{ x^\rho T^{\nu\mu} - x^{\mu} T^{\nu\rho} },
\end{dmath}
which retains the antisymmetry in \( \mu \rho \) yet still drops the parameter \( \omega^{\mu\rho} \).
To check that this makes sense, we can contract
\( j^{\nu\mu\rho} \) with \( \omega_{\rho\mu} \)
\begin{dmath}\label{eqn:qftLecture9:880}
j^{\nu\mu\rho} \omega_{\rho\mu}
= -\inv{2} \lr{ x^\rho T^{\nu\mu} - x^{\mu} T^{\nu\rho} }
\omega_{\mu\rho}
=
-\inv{2} x^\rho T^{\nu\mu}
\omega_{\mu\rho}
- \inv{2} x^{\mu} T^{\nu\rho}
\omega_{\rho\mu}
=
-\inv{2} x^\rho T^{\nu\mu}
\omega_{\mu\rho}
- \inv{2} x^{\rho} T^{\nu\mu}
\omega_{\mu\rho}
=
- x^{\rho} T^{\nu\mu}
\omega_{\mu\rho},
\end{dmath}
which matches \cref{eqn:qftLecture9:580} as desired.

\paragraph{Example.  Rotations \( \mu\rho = ij \)}

\begin{dmath}\label{eqn:qftLecture9:620}
J^{0 i j} \epsilon_{ijk}
=
\inv{2} \lr{ x^i T^{0j} - x^{j} T^{0i} } \epsilon_{ijk}
=
x^i T^{0j} \epsilon_{ijk}.
\end{dmath}
Observe that this has the structure of \( (\Bx \cross \Bp)_k \), where \( \Bp \) is the momentum density of the field.
Let
\begin{equation}\label{eqn:qftLecture9:640}
L_k \equiv Q_k = \int d^3 x J^{0ij} \epsilon_{ijk}.
\end{equation}
We can now quantize and build a generator
\begin{dmath}\label{eqn:qftLecture9:660}
\hatU(\Balpha)
= e^{i \Balpha \cdot \hat{\BL}}
= \exp\lr{i \alpha_k
\int d^3 x x^i \hat{T}^{0j} \epsilon_{ijk}
}
\end{dmath}
From \cref{eqn:qftLecture9:560} we can quantize with \( T^{0j} = \partial^0 \phi \partial^j \phi \rightarrow \pihat \lr{\spacegrad \phihat}_j\), or
\begin{dmath}\label{eqn:qftLecture9:900}
\hatU(\Balpha)
=
\exp\lr{i \alpha_k
\int d^3 x x^i \pihat (\spacegrad \phihat)_j \epsilon_{ijk}
}
=
\exp\lr{i \Balpha \cdot
\int d^3 x \pihat \spacegrad \phihat \cross \Bx
}
\end{dmath}
(up to a sign in the exponent which doesn't matter)
\begin{dmath}\label{eqn:qftLecture9:680}
\phihat(\By) \rightarrow \hatU(\alpha) \phihat(\By) \hatU^\dagger(\alpha)
\approx
\phihat(\By) +
i \Balpha \cdot
\antisymmetric{
   \int d^3 x \pihat(\Bx) \spacegrad \phihat(\Bx) \cross \Bx
}
{
   \phihat(\By)
}
=
\phihat(\By) +
i \Balpha \cdot
   \int d^3 x
(-i) \delta^3(\Bx - \By)
\spacegrad \phihat(\Bx) \cross \Bx
=
\phihat(\By) +
\Balpha \cdot \lr{ \spacegrad \phihat(\By ) \cross \By}
\end{dmath}
Explicitly, in coordinates, this is
\begin{dmath}\label{eqn:qftLecture9:700}
\phihat(\By) \rightarrow
\phihat(\By) +
\alpha^i
\lr{
   \partial^j \phihat(\By) y^k \epsilon_{jki}
}
=
\phihat(\By) -
\epsilon_{ikj} \alpha^i y^k \partial^j \phihat
=
\phihat( y^j - \epsilon^{ikj} \alpha^i y^k ).
\end{dmath}
This is a rotation.  To illustrate, pick \( \Balpha = (0, 0, \alpha) \), so \( y^j \rightarrow y^j - \epsilon^{ikj} \alpha y^k \delta_{i3} = y^j - \epsilon^{3kj} \alpha y^k \), or
\begin{dmath}\label{eqn:qftLecture9:920}
\begin{aligned}
y^1 &\rightarrow y^1 - \epsilon^{3k1} \alpha y^k = y^1 + \alpha y^2 \\
y^2 &\rightarrow y^2 - \epsilon^{3k2} \alpha y^k = y^2 - \alpha y^1 \\
y^3 &\rightarrow y^3 - \epsilon^{3k3} \alpha y^k = y^3,
\end{aligned}
\end{dmath}
or in matrix form
\begin{dmath}\label{eqn:qftLecture9:720}
\begin{bmatrix}
y^1 \\
y^2 \\
y^3 \\
\end{bmatrix}
\rightarrow
\begin{bmatrix}
1 & \alpha & 0 \\
-\alpha & 1 & 0 \\
0 & 0 & 1
\end{bmatrix}
\begin{bmatrix}
y^1 \\
y^2 \\
y^3 \\
\end{bmatrix}.
\end{dmath}

%\EndNoBibArticle

   \chapter{Lorentz boosts, generator of spacetime translation, Lorentz invariant field representation.}
      %
% Copyright � 2017 Peeter Joot.  All Rights Reserved.
% Licenced as described in the file LICENSE under the root directory of this GIT repository.
%
%{
\input{../latex/blogpost.tex}
\renewcommand{\basename}{qftLecture10}
\renewcommand{\dirname}{notes/phy2403/}
\newcommand{\keywords}{PHY2403H}
\input{../latex/peeter_prologue_print2.tex}

%\usepackage{phy2403}
\usepackage{peeters_braket}
%\usepackage{peeters_layout_exercise}
\usepackage{peeters_figures}
\usepackage{mathtools}
\usepackage{siunitx}

\newcommand{\ultensor}[3]{{{#1}^{#2}}_{#3}}
%\newcommand{\ulLambda}[2]{\ultensor{\Lambda}{#1}{#2}}
%\newcommand{\ulDelta}[2]{\ultensor{\delta}{#1}{#2}}

\beginArtNoToc
\generatetitle{PHY2403H Quantum Field Theory.  Lecture 10: XXX.  Taught by Prof.\ Erich Poppitz}
%\chapter{XXX}
\label{chap:qftLecture10}

\paragraph{DISCLAIMER: Very rough notes from class, with some additional side notes.}

These are notes for the UofT course PHY2403H, Quantum Field Theory I, taught by Prof. Erich Poppitz fall 2018.
%, covering \textchapref{{1}} \citep{peskin1995introduction} content.

\section{Lorentz transform symmetries.}

\begin{dmath}\label{eqn:qftLecture10:20}
x^\mu \rightarrow x^\mu + \omega^{\mu\nu} x_\nu
\end{dmath}

\begin{dmath}\label{eqn:qftLecture10:40}
\omega^{\mu\nu} = -\omega^{\nu\mu}
\end{dmath}

\begin{dmath}\label{eqn:qftLecture10:60}
\begin{aligned}
\omega^{ij} &= \text{rotations} \\
\omega^{0i} &= \text{boosts}
\end{aligned}
\end{dmath}

\begin{dmath}\label{eqn:qftLecture10:80}
J^{\nu (\mu\rho)} = \inv{2} \lr{ x^\rho T^{\nu\mu} - x^\mu T^{\nu\rho} }
\end{dmath}

\begin{dmath}\label{eqn:qftLecture10:100}
\partial_nu J^{\nu (\mu\rho)}
= \inv{2} \lr{
\partial_\nu x^\rho T^{\nu\mu}
+
x^\rho \cancel{\partial_\nu T^{\nu\mu} }
- \partial_\nu x^\mu T^{\nu\rho}
- x^\mu \cancel{\partial_\nu T^{\nu\rho} }
}
=
...
= 0.
\end{dmath}

\begin{dmath}\label{eqn:qftLecture10:120}
Q^{0i} = \int d^3 x J^{0 (oi)} = \inv{2} \int d^3 x \lr{ x^i T^{00} - x^0 T^{0i} }
\dot{Q}^{0i}
= \int d^3 x J^{0 (oi)}
= \inv{2} \int d^3 x \lr{ x^i \dot{T}^{00} - x^0 \dot{T}^{0i} }
= \inv{2} \int d^3 x \lr{ x^i (-\partial_j T^{j0}) - T^{0i} - x^0 (-\partial_j T^{ji}) }
= \inv{2} \int d^3 x \lr{ \partial_j (-x^i T^{j0}) + (\partial_j x^i) T^{j0}
- T^{0i} - x^0 (-\partial_j T^{ji}) }
= \text{surface terms}
\end{dmath}

so \( \dot{Q}^{0i} = 0 \).

Quantizing

\begin{dmath}\label{eqn:qftLecture10:140}
\antisymmetric{\hatT^{00}(\Bx)}{\phihat(\By)}
=
\inv{2}
\antisymmetric{\pihat^2(\Bx)}{\phihat(\By)}
=
\inv{2}
\antisymmetric{\pihat^2(\Bx)}{\phihat(\By)}
=
\pihat(\Bx)
\antisymmetric{\pihat(\Bx)}{\phihat(\By)}
= -i \delta^3(\Bx - \By) \pihat(\Bx)
\end{dmath}

Also:
\begin{dmath}\label{eqn:qftLecture10:160}
\antisymmetric{\hatT^{0i}(\Bx)}{\phihat(\By)}
=
\antisymmetric{\pihat(\Bx)\partial^i \phihat(\Bx)}{\phihat(\By)}
= -i \delta^3(\Bx - \By) \partial^i \phihat(\Bx)
\end{dmath}

\begin{dmath}\label{eqn:qftLecture10:180}
i \epsilon \antisymmetric{\hatQ^{0i}}{\phihat(\By)}
=
i
\frac{\epsilon}{2} \int d^3 x
\lr{
x^i
\antisymmetric{\hatT^{00}}{\phihat(\By)}
-
x^0
\antisymmetric{\hatT^{0i}}{\phihat(\By)}
}
=
\frac{\epsilon}{2} \lr{ y^i \pihat(\By) - y^0 \partial^i \phihat(\By) }
=
\frac{\epsilon}{2} \lr{ y^i \dot{\phihat}(\By) - y^0 \partial^i \phihat(\By) }
\end{dmath}

\begin{dmath}\label{eqn:qftLecture10:200}
e^{i \epsilon \hatQ^{0k} } \phihat(\By)
e^{-i \epsilon \hatQ^{0k} }
=
\phihat(\By)
+ \frac{\epsilon}{2} y^k \dot{\phihat}(\By)
- \frac{\epsilon}{2} y^0 \partial^k \phihat(\By)
+ \cdots
=
\phihat(\By)
+ \frac{\epsilon}{2} \lr{
   y^1 \dot{\phihat}(\By)
   -
   y^0 \PD{y^1}{\phihat}(\By)
}
=
\phihat(y^0 + \frac{\epsilon}{2} y^1,
y^1 + \frac{\epsilon}{2} y^2, y^3)
\end{dmath}

This is a boost.  Compare to

\begin{dmath}\label{eqn:qftLecture10:220}
\begin{aligned}
x^0 \rightarrow x^0 + \omega^{01} x_1 &= x^0 - \omega^{01} x^1 \\
x^1 \rightarrow x^1 + \omega^{10} x_0 &= x^1 - \omega^{01} x_0 = x^1 - \omega^{01} x^0
\end{aligned}
\end{dmath}

identification:

\begin{dmath}\label{eqn:qftLecture10:240}
\frac{\epsilon}{2} = - \omega^{01}.
\end{dmath}

\begin{dmath}\label{eqn:qftLecture10:260}
\hatU(\Lambda) = \exp\lr{-i \omega^{01} \int d^3 x \lr{ \hatT^{00} x^i - \hatT^{0i} x^0 }}
\end{dmath}

where

\begin{dmath}\label{eqn:qftLecture10:280}
\ultensor{\Lambda}{\mu}{\nu}
\approx
\ultensor{\delta}{\mu}{\nu}
+
\ultensor{\omega}{\mu}{\nu}
\end{dmath}

\begin{dmath}\label{eqn:qftLecture10:300}
\hatU(\Lambda) \phihat(t, \Bx)
\hatU^\dagger(\Lambda)
=
\phihat\lr{ \gamma (t - vx), \gamma(x - vt), y, z }.
\end{dmath}

Or

\begin{dmath}\label{eqn:qftLecture10:320}
\hatU(\Lambda) \phihat(x) \hatU^\dagger(\Lambda) =
\phihat(\Lambda x)
\end{dmath}
where \( (\Lambda x)^\mu = \ultensor{\Lambda}{\mu}{\nu} x^\nu \) and \( x \) is a four vector.

\section{Transformation of momentum states}
In the momentum space representation

\begin{dmath}\label{eqn:qftLecture10:340}
\phihat(x)
=
\int \frac{d^3 p}{(2 \pi)^3 \sqrt{2 \omega_\Bp}}  \lr{
   e^{i (\omega_\Bp t - \Bp \cdot \Bx)} \hata_\Bp
+
   e^{-i (\omega_\Bp t - \Bp \cdot \Bx)} \hata^\dagger_\Bp
}
=
\int \frac{d^3 p}{(2 \pi)^3 \sqrt{2 \omega_\Bp}}  \evalbar{
\lr{
   e^{i p^\mu x^\mu } \hata_\Bp
+
   e^{-i p^\mu x^\mu } \hata^\dagger_\Bp
}
}{p_0 = \omega_\Bp}
\hatU(\Lambda) \phihat(x) \hatU^\dagger(\Lambda)
=
\phihat(\Lambda x)
=
\int \frac{d^3 p}{(2 \pi)^3 \sqrt{2 \omega_\Bp}}  \evalbar{
\lr{
   e^{i p^\mu \ultensor{\Lambda}{\mu}{\nu} x^\nu } \hata_\Bp
+
   e^{-i p^\mu \ultensor{\Lambda}{\mu}{\nu} x^\nu } \hata^\dagger_\Bp
}
}{p_0 = \omega_\Bp}
=
\int \frac{dp^0 d^3 p}{(2\pi)^3} \delta(p_0^2 - \Bp^2 - m^2) \Theta(p^0) \sqrt{2 \omega_\Bp} e^{...} \hata_\Bp + hc
\end{dmath}

using \( \delta(f(x)) = \sum_{f(x_\conj) = 0} \frac{\delta(x - x_\conj)}{f'(x_\conj)} \)

this is
\begin{dmath}\label{eqn:qftLecture10:360}
=
\int \frac{dp^0 d^3 p}{(2\pi)^3}
\lr{
\frac{\delta(p_0 - \omega_\Bp)}{2 \omega_\Bp}
+
\frac{\delta(p_0 + \omega_\Bp)}{2 \omega_\Bp}
}
\Theta(p^0) \sqrt{2 \omega_\Bp}  \hata_\Bp + hc
\end{dmath}

but the \( \Theta(p^0) \) kills the second delta function.

We now have a more explicit Lorentz invariant structure

\begin{dmath}\label{eqn:qftLecture10:380}
\phihat(\Lambda x)
=
\int \frac{dp^0 d^3 p}{(2\pi)^3} \delta(p_0^2 - \Bp^2 - m^2) \Theta(p^0) \sqrt{2 \omega_\Bp}  \hata_\Bp + hc
\end{dmath}

In \cref{fig:constantMomentumSurface:constantMomentumSurfaceFig1}, the paraboloid depict the surfaces of constant energy-momentum \( p^0 = \sqrt{ \Bp^2 + m^2 } \).  Lorentz transformations shift points in the energy-momentum space along the paraboloid, but cannot change the sign of the energy coordinate, so \( \Theta(p^0) \) is Lorentz invariant, since the sign of the energy \( p^0 \) does not change under such a transformation.
\imageFigure{../figures/phy2403-quantum-field-theory/constantMomentumSurfaceFig1}{Surface of constant squared four-momentum.}{fig:constantMomentumSurface:constantMomentumSurfaceFig1}{0.3}

Let's change variables

\begin{dmath}\label{eqn:qftLecture10:400}
p^\lambda = \ultensor{\Lambda}{\lambda}{\rho} {p'}^{\rho}
\end{dmath}

so that

\begin{dmath}\label{eqn:qftLecture10:420}
p_\mu
\ultensor{\Lambda}{\mu}{\nu} x^\nu
=
\ultensor{\Lambda}{\lambda}{\rho} {p'}^\rho g_{\lambda\nu} \ultensor{\Lambda}{\nu}{\sigma} x^{\sigma}
=
{p'}^\rho g_{\rho\sigma} x^\sigma
\end{dmath}

which gives

\begin{dmath}\label{eqn:qftLecture10:440}
\phihat(\Lambda x)
=
\int \frac{d{p'}^0 d^3 p'}{(2\pi)^3} \delta({p'}_0^2 - {\Bp'}^2 - m^2) \Theta(p^0) \sqrt{2 \omega_{\Lambda \Bp'}} e^{i \Bp' \cdot \Bx}  \hata_{\Lambda \Bp'} + hc
=
\int \frac{dp^0 d^3 p}{(2\pi)^3} \delta({p}_0^2 - {\Bp}^2 - m^2) \Theta(p^0) \sqrt{2 \omega_{\Lambda \Bp}} e^{i \Bp \cdot \Bx}  \hata_{\Lambda \Bp} + hc
\end{dmath}

but

\begin{dmath}\label{eqn:qftLecture10:460}
\phihat(\Lambda x)
=
\int \frac{dp^0 d^3 p}{(2\pi)^3} \delta({p}_0^2 - {\Bp}^2 - m^2) \Theta(p^0) \sqrt{2 \omega_{\Bp}} e^{i \Bp \cdot \Bx}  \hata_{\Bp} + hc
\end{dmath}

We can now conclude that
\begin{dmath}\label{eqn:qftLecture10:480}
\sqrt{2 \omega_{\Lambda \Bp}} \hata_{\Lambda \Bp}
=
\hatU(\Lambda)
\sqrt{2 \omega_{ \Bp}} \hata_{ \Bp}
\hatU^\dagger(\Lambda)
\end{dmath}

In particular
\begin{dmath}\label{eqn:qftLecture10:500}
\sqrt{2 \omega_{ \Bp}} \hata^\dagger_{ \Bp} \ket{0} = \ket{\Bp}
\end{dmath}
and noting that \( \hatU(\Lambda) \ket{0}  = \ket{0} \) (i.e. the ground state is Lorentz invariant), we have
\begin{dmath}\label{eqn:qftLecture10:n}
\sqrt{2 \omega_{\Lambda \Bp}} \hata^\dagger_{\Lambda \Bp} \ket{0}
=
\hatU(\Lambda) \sqrt{ 2\omega_\Bp} \hata^\dagger_\Bp \hatU^\dagger(\Lambda) \hatU(\Lambda) \ket{0}
=
\hatU(\Lambda) \sqrt{ 2\omega_\Bp} \hata^\dagger_\Bp \ket{0}
=
\hatU(\Lambda) \ket{\Bp}.
\end{dmath}

%}
%\EndArticle
\EndNoBibArticle

   \chapter{Microcausality, Lorentz invariant measure, retarded time SHO Green's function.}
      %
% Copyright � 2017 Peeter Joot.  All Rights Reserved.
% Licenced as described in the file LICENSE under the root directory of this GIT repository.
%
%{
\input{../latex/blogpost.tex}
\renewcommand{\basename}{qftLecture11}
\renewcommand{\dirname}{notes/phy2403/}
\newcommand{\keywords}{PHY2403H}
\input{../latex/peeter_prologue_print2.tex}

%\usepackage{phy2403}
\usepackage{peeters_braket}
\usepackage{peeters_layout_exercise}
\usepackage{peeters_figures}
\usepackage{mathtools}
\usepackage{siunitx}
\usepackage{macros_cal} % LL

\newcommand{\ultensor}[3]{{{#1}^{#2}}_{#3}}

\beginArtNoToc
\generatetitle{PHY2403H Quantum Field Theory.  Lecture 11: Momentum matrix elements, spacelike surfaces, microcausality, Lorentz invariant measure, wave function Green's function, retarded time contour, advanced time contour.  Taught by Prof.\ Erich Poppitz}
%\chapter{Momentum matrix elements, spacelike surfaces, microcausality, Lorentz invariant measure, wave function Green's function, retarded time contour, advanced time contour.}
\label{chap:qftLecture11}

\paragraph{DISCLAIMER: Very rough notes from class, with some additional side notes.}

These are notes for the UofT course PHY2403H, Quantum Field Theory I, taught by Prof. Erich Poppitz fall 2018.
%, covering \textchapref{{1}} \citep{peskin1995introduction} content.

\section{Relativistic normalization.}

We will continue looking at the generator of spacetime translation \( \hatU(\Lambda) \), which has the property
\begin{dmath}\label{eqn:qftLecture11:40}
\hatU(\Lambda) \ket{0} = \ket{0},
\end{dmath}
%\underline{Lorentz invariance unbroken}
That is
\begin{dmath}\label{eqn:qftLecture11:760}
\hatU(\Lambda) = \BOne + \text{operators that anhillate the vacuum state}.
\end{dmath}

The action on a field was
\begin{dmath}\label{eqn:qftLecture11:60}
\hatU(\Lambda)
\phihat(x) \hatU^\dagger(\Lambda)
= \phihat(\Lambda x),
\end{dmath}
and the action on the anhillation operator was
\begin{dmath}\label{eqn:qftLecture11:300}
\hatU(\Lambda)
\sqrt{ 2 \omega_\Bp } \hata_\Bp
\hatU^\dagger(\Lambda)
=
\sqrt{ 2 \omega_{\Lambda \Bp} } \hata_{\Lambda \Bp}.
\end{dmath}

If \( \ket{\Bp_1} \) is the one particle state with momentum \( \Bp_1 \), then that momentum state can be generated from the ground state with the following normalized creation operation
\begin{dmath}\label{eqn:qftLecture11:780}
\ket{\Bp_1} = \sqrt{ 2 \omega_{\Bp_1} } \hata_{\Bp_1}^\dagger \ket{0}.
\end{dmath}

We can compute the matrix element between two matrix states using the creation operator representation
\begin{dmath}\label{eqn:qftLecture11:80}
\begin{aligned}
\braket{\Bp_2}{\Bp_1}
&=
\sqrt{ 2 \omega_{\Bp_1} }
\sqrt{ 2 \omega_{\Bp_2} }
\bra{0}
\hata_{\Bp_2}
\hata_{\Bp_1}^\dagger
\ket{0} \\
&=
\sqrt{ 2 \omega_{\Bp_1} }
\sqrt{ 2 \omega_{\Bp_2} }
\bra{0}
\lr{
   \hata_{\Bp_1}^\dagger
   \hata_{\Bp_2}
   +
   i (2 \pi)^3 \delta^3(\Bp - \Bq)
} \\
&=
\sqrt{ 2 \omega_{\Bp_1} }
\sqrt{ 2 \omega_{\Bp_2} }
(2 \pi)^3 \delta^3(\Bp_1 - \Bp_2) \\
&=
2 \omega_{\Bp_1}
(2 \pi)^3 \delta^3(\Bp_1 - \Bp_2).
\end{aligned}
\end{dmath}

\section{Spacelike surfaces.}

If \( x^\mu, p^\mu \) are four vectors, then \( p^\mu x_\mu = \text{invariant} = {p'}^\mu x'_\mu \).  The light cone is the surface \( p_0^2 = \Bp^2  \), whereas timelike four-momentum form a parabaloid surface \( p_0^2 - \Bp^2 = m^2 \) (i.e. \( E = \sqrt{ m^2 c^4 + \Bp^2 c^2 } \)).
The surface for constant spacelike points (i.e. all related by a Lorentz transformation) is illustrated in \cref{fig:spaceLikeAndLightCone:spaceLikeAndLightConeFig1}.  A boost moves a point up or down that surface along the energy axis.  It is therefore possible to use a sequence of boost and rotation to transform a point \( (E, \Bp) \rightarrow (-E, \Bp) \rightarrow (-E, -\Bp) \).  That is, any spacelike four-vector \( x \) may be transformed to \( -x \) using a Lorentz transformation.
\imageFigure{../figures/phy2403-quantum-field-theory/spaceLikeAndLightConeFig1}{Constant spacelike surface.}{fig:spaceLikeAndLightCone:spaceLikeAndLightConeFig1}{0.3}

\section{Condition on microcausality.}

We defined operators \( \phihat(\Bx) \), which was a Hermitian operator for the real scalar field.  For the complex scalar field we used \( \phihat(\Bx) = (\phihat_1 + \phihat_2)/\sqrt{2} \), where each of \( \phihat_1, \phihat_2 \) were Hermitian operators.  i.e. we can think of these operators as ``observables'', that is \( \phihat(\Bx) = \phihat^\dagger(\Bx) \).

We now want to show that these operators commute at spacelike separations, and see how this relates to the question of causality.  In particular, we want to see that an observation of one operator, will not effect the measurement of the other.

The condition of microcausality is
\begin{equation*}
\antisymmetric{\phihat(x)}{\phihat(y)} = 0
\end{equation*}
if \( x \sim y \), that is \( (x - y)^2 < 0 \).  That is, \( x, y \) are spacelike separated.

We wrote

\begin{dmath}\label{eqn:qftLecture11:160}
\phihat(x)
=
\int \frac{d^3 p}{(2 \pi)^3 \sqrt{2 \omega_\Bp}}
\evalbar{
e^{-i p \cdot x} }{p^0 = \omega_\Bp} \hata_\Bp
+
\int \frac{d^3 p}{(2 \pi)^3 \sqrt{2 \omega_\Bp}}
\evalbar{
e^{i p \cdot x} }{p^0 = \omega_\Bp} \hata^\dagger_\Bp
,
\end{dmath}
or \( \phihat(x) = \phihat_{-}(x) + \phihat_{+}(x) \), where
\begin{dmath}\label{eqn:qftLecture11:180}
\begin{aligned}
\phihat_{-}(x) &=
\int \frac{d^3 p}{(2 \pi)^3 \sqrt{2 \omega_\Bp}}
\evalbar{
e^{-i p \cdot x} }{p^0 = \omega_\Bp} \hata_\Bp \\
\phihat_{+}(x) &=
\int \frac{d^3 p}{(2 \pi)^3 \sqrt{2 \omega_\Bp}}
\evalbar{
e^{i p \cdot x} }{p^0 = \omega_\Bp} \hata^\dagger_\Bp
\end{aligned}
\end{dmath}

Compute the commutator
\begin{dmath}\label{eqn:qftLecture11:200}
D(x)
= \antisymmetric{\phihat_{-}(x)}{\phihat_{+}(0)}
=
\int \frac{d^3 p}{(2 \pi)^3 \sqrt{2 \omega_\Bp}}
\evalbar{ e^{-i p \cdot x} }{p^0 = \omega_\Bp}
\int \frac{d^3 k}{(2 \pi)^3 \sqrt{2 \omega_\Bk}}
\evalbar{ e^{i k \cdot 0} }{k^0 = \omega_\Bk}
\antisymmetric{\hata_\Bp }{\hata_\Bk^\dagger }
=
\int \frac{d^3 p}{(2 \pi)^3 \sqrt{2 \omega_\Bp}}
\evalbar{ e^{-i p \cdot x} }{p^0 = \omega_\Bp}
\int \frac{d^3 k}{(2 \pi)^3 \sqrt{2 \omega_\Bk}}
(2 \pi)^3 \delta^3(\Bp - \Bk),
\end{dmath}
\boxedEquation{eqn:qftLecture11:800}{
D(x)
=
\int \frac{d^3 p}{(2 \pi)^3 2 \omega_\Bp}
\evalbar{ e^{-i p \cdot x} }{p^0 = \omega_\Bp}.
}

Now about the commutator at two spacetime points
\begin{dmath}\label{eqn:qftLecture11:220}
\antisymmetric{\phihat(x)}{\phihat(y)}
=
\antisymmetric{\phihat_{-}(x) + \phihat_{+}(x)}{\phihat_{-}(y) + \phihat_{+}(y)}
=
\antisymmetric{\phihat_{-}(x)}{\phihat_{+}(y)}
+
\antisymmetric{\phihat_{+}(x)}{\phihat_{-}(y)}
=
-D(y - x) + D(x - y)
\end{dmath}

Find
\begin{dmath}\label{eqn:qftLecture11:240}
\begin{aligned}
\antisymmetric{\phihat(x)}{\phihat(y)} &= D(x - y) - D(y - x) \\
\antisymmetric{\phihat(x)}{\phihat(0)} &= D(x) - D(- x)
\end{aligned}
\end{dmath}

Let's look at \( D(x) \), \cref{eqn:qftLecture11:800}, a bit more closely.

\paragraph{Claim:}
\( D(x) \) is Lorentz invariant (has the same value for all \( x^\mu, {x'}^\mu \)

We can see this by writing this out as
\begin{dmath}\label{eqn:qftLecture11:280}
D(x)
=
\int \frac{d^3 p}{(2 \pi)^3 } dp^0
\delta( p_0^2 - \Bp^2 - m^2) \Theta(p^0)
e^{-i p \cdot x}
\end{dmath}

The exponential is Lorentz invariant, and the delta function has been put into a Lorentz invariant form.

\paragraph{Claim 1:}
\( D(x) = D(x') \) where \( x^2 = {x'}^2 \).

\paragraph{Claim 2:}
\( x^\mu, -x^\mu \) are related by Lorentz transformations if \( x^2 < 0 \).

From the figure, we see that \( D(x) = D(-x) \) for a spacelike point, which implies that
\(
\antisymmetric{\phihat(x)}{\phihat(0)} = 0 \) for a spacelike point \( x \).

We've shown this for free fields, but later we will see that this is the case for interacting fields too.

\section{Harmonic oscillator.}

\begin{dmath}\label{eqn:qftLecture11:320}
L = \inv{2} \qdot^2 - \frac{\omega^2}{t} q^2 - j(t) q
\end{dmath}

The term \( j(t) \) shifts the origin in a time dependent fashion (graphical illustration in class wiggling a hockey stick, as a sample of a harmonic oscillator).

%F5,6
\begin{dmath}\label{eqn:qftLecture11:340}
H = \frac{p^2}{2} + \frac{\omega^2}{t} q^2 + j(t) q
\end{dmath}

\begin{dmath}\label{eqn:qftLecture11:360}
\begin{aligned}
i \qdot_H(t) &= \antisymmetric{q_H}{H} = i p_H \\
i \pdot_H(t) &= \antisymmetric{p_H}{H} = -i \omega^2 q_H - i j(t)
\end{aligned}
\end{dmath}

\begin{dmath}\label{eqn:qftLecture11:380}
\ddot{q}_H(t) = - \omega^2 q_H(t) - j(t)
\end{dmath}
or
\begin{dmath}\label{eqn:qftLecture11:400}
(\partial_{tt} + \omega^2 ) q_H(t) = - j(t)
\end{dmath}

%%q_H(t > t_{\text{after}} = q_H( t, t < t_before) +
%%\int_{t_before}^{t_after} G_R(t - t') j(t') dt'
\begin{dmath}\label{eqn:qftLecture11:420}
q_H(t) = q_H^0( t ) +
\int G_R(t - t') j(t') dt'
\end{dmath}

This solves the equation provided \( G_R(t - t') \) has the property that
\boxedEquation{eqn:qftLecture11:440}{
(\partial_{tt} + \omega^2)
G_R(t - t')
= - \delta(t - t')
}

That is
\begin{dmath}\label{eqn:qftLecture11:460}
(\partial_{tt} + \omega^2)
q_H(t) =
(\partial_{tt} + \omega^2)
q_H^0( t )
+
(\partial_{tt} + \omega^2)
\int G_R(t - t') j(t') dt'
\end{dmath}

This function \( G_R \) is called the retarded Green's function.  We want to find this function, and as usual, we do this by taking the Fourier transform of \cref{eqn:qftLecture11:440}

\begin{dmath}\label{eqn:qftLecture11:480}
\int dt e^{i p t}
(\partial_{tt} + \omega^2) G_R(t - t')
=
-\int_{-\infty}^\infty dt e^{i p t}
\delta(t - t')
= - e^{i p t'}
\end{dmath}

Let
\begin{dmath}\label{eqn:qftLecture11:500}
G(t - t')  = \int \frac{dp }{2 \pi} e^{- i p'(t - t')} \tilde{G}(p'),
\end{dmath}
so

\begin{dmath}\label{eqn:qftLecture11:520}
- e^{i pt'}
=
\int dt e^{i p t}
(\partial_{tt} + \omega^2)
\int \frac{dp'}{2 \pi} e^{- i p'(t - t')} \tilde{G}(p')
=
\int dt e^{i p t} \int
\frac{dp'}{2 \pi} \lr{ -{p'}^2 + \omega^2 } e^{- i p'(t - t')} \tilde{G}(p')
=
\int dp' \lr{ -{p'}^2 + \omega^2 } e^{i p' t'} \delta(p - p') \tilde{G}(p')
=
\lr{ -{p}^2 + \omega^2 } \tilde{G}(p) e^{i p t'}
\end{dmath}

so
\begin{dmath}\label{eqn:qftLecture11:540}
\tilde{G}(p)
= \inv{p^2 - \omega^2}
\end{dmath}

Now

\begin{dmath}\label{eqn:qftLecture11:560}
G(t)
= \int \frac{dp}{2 \pi} e^{-i p t}
\tilde{G}(p)
\end{dmath}

Let's write the momentum space Green's function as
\begin{dmath}\label{eqn:qftLecture11:580}
\tilde{G}(p)
= \inv{(p - \omega)(p + \omega)}
\end{dmath}

The solution contained
\begin{dmath}\label{eqn:qftLecture11:600}
\int G(t - t') j(t') dt'.
\end{dmath}
Suppose \( j(t) = 0 \) for all \( t < t_0 \).  We want the effect of \( j(t) \) to be felt in the future, for example, \(j(t) \) is an impulse starting at some time.  We want \( G(t) \) to vanish at negative times.

We want the integral
\begin{dmath}\label{eqn:qftLecture11:620}
G(t)
= \int \frac{dp}{2 \pi} e^{-i p t}
\inv{(p - \omega)(p + \omega)}
\end{dmath}
to vanish when \( t < 0 \).

% t - t' > 0 ; t > t'
Start with \( t > 0 \) (that is \( t' < t \)), so that \( e^{-i p t} = e^{-i p \Abs{t}} \) which means that we have to integrate over a lower plane contour like \cref{fig:lowerContour:lowerContourFig1}, because the imaginary part of \( p \) is negative, but for \( t < 0 \) (that is \( t' > t \)), we want an upper plane contour like \cref{fig:upperContour:upperContourFig1}.
\imageFigure{../figures/phy2403-quantum-field-theory/lowerContourFig1.png}{Lower plane contour.}{fig:lowerContour:lowerContourFig1}{0.3}
\imageFigure{../figures/phy2403-quantum-field-theory/upperContourFig2.png}{Upper plane contour.}{fig:upperContour:upperContourFig1}{0.3}


Question: since we are integrating over the real line, how can we get away with deforming the contour?
Answer: it works.  If we do this we get a Green's function that makes sense (better answer later?)

We add an infinite circle, so that we can integrate over a closed contour, and pick the contour so that it is zero for \( t < 0 \) and non-zero (enclosed poles) for \( t > 0 \).

\begin{dmath}\label{eqn:qftLecture11:640}
G_R(t > 0)
= \int_C \frac{dp}{2 \pi} e^{-i p t}
\inv{(p - \omega)(p + \omega)}
=
\inv{2 \pi} (-2 \pi i) \lr{
   \frac{e^{-i \omega t}}{2 \omega}
   -
   \frac{e^{i \omega t}}{2 \omega}
}
=
-\frac{\sin(\omega t)}{\omega}.
\end{dmath}

Now we write the Green's function for all time as
\boxedEquation{eqn:qftLecture11:660}{
G_R(t) =
-\frac{\sin(\omega t)}{\omega} \Theta(t).
}

The question of what contour to pick can now be justified by the result, since this satisfies \cref{eqn:qftLecture11:440}.  If we wanted a Green's function that selected just future contributions we'd have used a ``bumps down'' contour.  There will be circumstances where we will use some of the other contour possibilities (\cref{fig:allTheContours:allTheContoursFig3}).  In particular, the bumps up and down contour will be used to derive the ``Feynman propagator'' that we'll use later.
\imageFigure{../figures/phy2403-quantum-field-theory/allTheContoursFig3}{All possible deformations around the poles.}{fig:allTheContours:allTheContoursFig3}{0.3}

\section{Field theory (where we are going).}

We will consider a massive real scalar field theory with an external source with action

\begin{dmath}\label{eqn:qftLecture11:680}
S = \int d^4 x \lr{
\inv{2} \partial_\mu \phi \partial^\mu \phi - \frac{m^2}{2} \phi^2 + j(x) \phi(x)
}
\end{dmath}

We don't have examples of currents that create scalar fields, but to study such as system, recall that
in electromagnetism we added sources to the field by adding a term like
\begin{dmath}\label{eqn:qftLecture11:700}
\int d^4 x A^\mu(x) j_\mu(x),
\end{dmath}
to our action.

The equation of motion can be found to be
\begin{dmath}\label{eqn:qftLecture11:720}
\lr{ \partial_\mu \partial^\mu + m^2 } \phi(x) = j(x).
\end{dmath}

We want to study the Green's function of this Klien-Gordon equation, defined to obey
\begin{dmath}\label{eqn:qftLecture11:740}
\lr{ \partial_\mu \partial^\mu + m^2 }_x G(x - y) = -i \delta^4(x - y),
\end{dmath}
where the \( -i \) factor is for convienience.
This is analogous to the Green's function that we just studied for the QM harmonic oscillator.

\makeproblem{Compute \( D(x-y) \) from the commutator.}{problem:qftLecture11:820}{
Generalize the derivation \cref{eqn:qftLecture11:800} by computing the commutator at two different space time points \( x, y \).
} % problem

\makeanswer{problem:qftLecture11:820}{
Let
\begin{dmath}\label{eqn:qftLecture11:860}
D(x - y)
= \antisymmetric{\phihat_{-}(x)}{\phihat_{+}(y)}
=
\int \frac{d^3 p}{(2 \pi)^3 \sqrt{2 \omega_\Bp}}
\evalbar{ e^{-i p \cdot x} }{p^0 = \omega_\Bp}
\int \frac{d^3 k}{(2 \pi)^3 \sqrt{2 \omega_\Bk}}
\evalbar{ e^{i k \cdot y} }{k^0 = \omega_\Bk}
\antisymmetric{\hata_\Bp }{\hata_\Bk^\dagger }
=
\int \frac{d^3 p}{(2 \pi)^3 \sqrt{2 \omega_\Bp}}
\evalbar{ e^{-i p \cdot x} }{p^0 = \omega_\Bp}
\int \frac{d^3 k}{(2 \pi)^3 \sqrt{2 \omega_\Bk}}
\evalbar{ e^{i k \cdot y} }{k^0 = \omega_\Bk}
(2 \pi)^3 \delta^3(\Bp - \Bk)
=
\int \frac{d^3 p}{(2 \pi)^3 2 \omega_\Bp}
\evalbar{ e^{-i p \cdot (x - y)} }{p^0 = \omega_\Bp}.
\end{dmath}
} % answer

\makeproblem{Verification of harmonic oscillator Green's function.}{problem:qftLecture11:2}{
Take the derivatives of a convolution of the Green's function \cref{eqn:qftLecture11:660} to show that it satisifies
\cref{eqn:qftLecture11:440}.
} % problem

\makeanswer{problem:qftLecture11:2}{
Let
\begin{dmath}\label{eqn:qftLecture11:880}
q(t)
= \int_{-\infty}^\infty G(t - t') j(t') dt'
= -\inv{\omega} \int_{-\infty}^\infty \sin(\omega(t - t')) \Theta(t - t') j(t') dt'.
\end{dmath}
We are free to add any \( q_0(t) \) that satisfies the homogeneous wave equation \( q_0''(t) + \omega^2 q_0(t) = 0 \) to our assumed convolution solution \cref{eqn:qftLecture11:880}, but that isn't interesting for this exersize.
Since \( \Theta(t - t') = 0 \) for \( t - t' < 0 \), or \( t' > t \), the convolution can be written as
\begin{dmath}\label{eqn:qftLecture11:900}
q(t)
= -\inv{\omega} \int_{-\infty}^t \sin(\omega(t - t')) j(t') dt',
\end{dmath}
which is now in a convient form to take derivatives.  We have contributions from the boundary's time dependence and from the integrand.  In particular
\begin{dmath}\label{eqn:qftLecture11:920}
\ddt{} \int_{a(t)}^{b(t)} g(x, t) dx
=
g(b(t)) b'(t) - g(a(t)) a'(t) + \int_a^b \frac{\partial}{\partial t} g(x, t) dx.
\end{dmath}
Assuming that \( j(-\infty) = 0 \), this gives
\begin{dmath}\label{eqn:qftLecture11:940}
\ddt{q(t)}
=
-\inv{\omega} \evalbar{\sin(\omega(t - t')) j(t') }{t' = t}
-\int_{-\infty}^t \cos(\omega(t - t')) j(t') dt'
=
-\int_{-\infty}^t \cos(\omega(t - t')) j(t') dt'.
\end{dmath}
For the second derivative we have
\begin{dmath}\label{eqn:qftLecture11:960}
q''(t) =
- \evalbar{ \cos(\omega(t - t')) j(t') }{t' = t}
+\omega \int_{-\infty}^t \sin(\omega(t - t')) j(t') dt'
=
-j(t) -\omega^2
\int_{-\infty}^t \frac{-\sin(\omega(t - t'))}{\omega} j(t') dt',
\end{dmath}
or
\begin{dmath}\label{eqn:qftLecture11:980}
q''(t) = -j(t) - \omega^2 q(t),
\end{dmath}
which is our forced Harmonic oscillator equation.
} % answer

%}
\EndNoBibArticle
%\EndArticle

   \chapter{Klein-Gordon Green's function, Feynman propagator path deformation, Wightman function, Retarded Green's function.}
      %
% Copyright � 2017 Peeter Joot.  All Rights Reserved.
% Licenced as described in the file LICENSE under the root directory of this GIT repository.
%
%{
%%%\input{../latex/blogpost.tex}
%%%\renewcommand{\basename}{qftLecture12}
%%%\renewcommand{\dirname}{notes/phy2403/}
%%%\newcommand{\keywords}{PHY2403H}
%%%\input{../latex/peeter_prologue_print2.tex}
%%%
%%%%\usepackage{phy2403}
%%%\usepackage{peeters_braket}
%%%\usepackage{peeters_layout_exercise}
%%%\usepackage{peeters_figures}
%%%\usepackage{mathtools}
%%%\usepackage{siunitx}
%%%\usepackage{macros_cal} % LL
%%%% :%s/\\tG/\\tilde{G}/g
%%%
%%%\newcommand{\ultensor}[3]{{{#1}^{#2}}_{#3}}
%%%
%%%\beginArtNoToc
%%%\generatetitle{PHY2403H Quantum Field Theory.  Lecture 12: Klein-Gordon Green's function, Feynman propagator path deformation, Wightman function, Retarded Green's function.  Taught by Prof.\ Erich Poppitz}
\chapter{Klein-Gordon Green's function, Feynman propagator path deformation, Wightman function, Retarded Green's function.}
\label{chap:qftLecture12}
\index{Green's function!Klein-Gordon}
\index{Feynman propagator}
\index{path deformation}
\index{Green's function!retarded time}

%%%\paragraph{DISCLAIMER: Very rough notes from class, with some additional side notes.}
%%%
%%%These are notes for the UofT course PHY2403H, Quantum Field Theory, taught by Prof. Erich Poppitz, fall 2018.
%%%%, covering \textchapref{{1}} \citep{peskin1995introduction} content.
%%%
\section{Green's functions for the forced Klein-Gordon equation.}

The problem were were preparing to do was to study the problem of ``particle creation by external classical source''.

We continue with a real scalar field, free, massive, but with an interaction with a source
\begin{dmath}\label{eqn:qftLecture12:20}
S_{\text{int}} = \int d^4 x j(x) \phi(x).
\end{dmath}

\paragraph{Modern application:} think of \( \phi \) has some SM field and think of \( j \) as due to inflaton (i.e. cosmological inflation interaction) oscillation.  In the inflationary model, the process of ``reheating'' creates all the matter in the universe.  We won't be talking about inflation, but will be considering a toy model that has some similar characteristics to the inflationary theory.
\index{inflation}

The equation of motion that we end up with is
\begin{dmath}\label{eqn:qftLecture12:40}
\lr{ \partial_\mu \partial^\mu + m^2} \phi = j,
\end{dmath}
and we wish to solve this using Green's function techniques.

\makedefinition{Klein-Gordon Green's function.}{dfn:qftLecture12:60}{
The QFT conventions for the Klein-Gordon Green's function is
\begin{equation*}
\lr{ \partial_\mu \partial^\mu + m^2} G(x - y) = -i \delta^4(x - y).
\end{equation*}
} % definition

As usual, we assume that it is possible to find a solution \( \phi \) by convolution
\begin{dmath}\label{eqn:qftLecture12:80}
\phi(x) = i \int d^4 y G(x - y) j(y).
\end{dmath}
\paragraph{Check:}
\begin{dmath}\label{eqn:qftLecture12:100}
\lr{ \partial_\mu \partial^\mu + m^2} \phi(x)
=
i
\lr{ \partial_\mu \partial^\mu + m^2}
\int d^4 y G(x - y) j(y)
=
i \int d^4 y (-i) \delta^4(x - y) j(y)
= j(x).
\end{dmath}

Also, as usual, we take out our Fourier transforms, the power tool of physics, and determine the structure of the Green's function by inverting the transform equation
\begin{dmath}\label{eqn:qftLecture12:120}
G(x - y) = \int \frac{d^4 p}{(2 \pi)^4} e^{-i p \cdot (x-y) } \tilde{G}(p).
\end{dmath}
Operating with Klein-Gordon gives
\begin{dmath}\label{eqn:qftLecture12:520}
\lr{ \partial_\mu \partial^\mu + m^2}
G(x)
=
\int \frac{d^4 p}{(2 \pi)^4}
\lr{ (-i p_\mu)(-i p^\mu) + m^2 }
e^{-i p \cdot (x-y) } \tilde{G}(p).
\end{dmath}
This must equal
\begin{dmath}\label{eqn:qftLecture12:140}
-i \delta^4(x - y) =
-i \int \frac{d^4 p}{(2 \pi)^4} e^{-i p \cdot (x -y)},
\end{dmath}
or
\begin{dmath}\label{eqn:qftLecture12:160}
\lr{ m^2 - p_\mu p^\mu } \tilde{G}(p) = -i.
\end{dmath}
The Green's function in the momentum domain is
\begin{dmath}\label{eqn:qftLecture12:180}
\tilde{G}(p) = \frac{i}{p^2 - m^2}.
\end{dmath}

The inverse transform provides the spatial domain representation of the Green's function
\begin{dmath}\label{eqn:qftLecture12:200}
G(x)
=
\int \frac{d^4 p}{(2 \pi)^4} e^{-i p \cdot x }
\frac{i}{(p^0)^2 - \Bp^2 - m^2}
=
\int \frac{d^3 p}{(2\pi)^3} e^{i \Bp \cdot \Bx}
\int \frac{d p_0}{2 \pi} e^{-i p_0 x^0 }
\frac{i}{(p_0 - \omega_\Bp)(p_0 + \omega_\Bp)}.
\end{dmath}

In the \( p_0 \) plane, we have two poles at \( p_0 = \pm \omega_\Bp \).
There are 4 ways to go around the poles, the retarded time deformation that we used to derive the Green's function for the harmonic oscillator, as sketched in \cref{fig:retardedTimeContours:retardedTimeContoursFig1}, the advanced time deformation sketched in \cref{fig:deformationForAdvancedTime:deformationForAdvancedTimeFig2}, and mixed deformations.
\imageFigure{../figures/phy2403-quantum-field-theory/retardedTimeContoursFig1}{Retarded time deformations and contours.}{fig:retardedTimeContours:retardedTimeContoursFig1}{0.3}
\imageFigure{../figures/phy2403-quantum-field-theory/deformationForAdvancedTimeFig2}{Advanced time deformation.}{fig:deformationForAdvancedTime:deformationForAdvancedTimeFig2}{0.1}

We will evaluate the integral using the ``Feynman propagator'' contour
sketched in \cref{fig:feynmanDeformation:feynmanDeformationFig3}.
Why we use the Feynman contour, and not the retarded contour can be justified by how well this works for the perturbation methods that will be developed later.
\imageFigure{../figures/phy2403-quantum-field-theory/feynmanDeformationFig3}{Feynman propagator deformation path.}{fig:feynmanDeformation:feynmanDeformationFig3}{0.1}

Consider each contour in turn.
\paragraph{Case I.  \( x^0 > 0 \)}

For this case, we use the lower half plane contour sketched in \cref{fig:feynmanContour:feynmanContourFig4}, which vanishes for \( \Im(p_0) < 0, x_0 > 0 \), where \( -i (i \Im(p_0) x_0) < 0 \).
\imageFigure{../figures/phy2403-quantum-field-theory/feynmanContourFig4}{Feynman propagator contour for \( t > 0 \).}{fig:feynmanContour:feynmanContourFig4}{0.3}

Here we pick up just the pole at \( p_0 = \omega_\Bp \), and take a negatively oriented path
\begin{dmath}\label{eqn:qftLecture12:220}
G_\txtF
=
\int \frac{d^3 p}{(2\pi)^3} e^{i \Bp \cdot \Bx}
\int \frac{d p_0}{2 \pi} e^{-i p_0 x^0 }
\frac{i}{(p_0 - \omega_\Bp)(p_0 + \omega_\Bp)}
=
\int \frac{d^3 p}{(2\pi)^3} e^{i \Bp \cdot \Bx}
(-2 \pi i)
\evalbar{\lr{ \frac{e^{-i p_0 x^0 }}{2 \pi}
\frac{i}{p_0 + \omega_\Bp} }}{p_0 = \omega_p}
=
\int \frac{d^3 p}{(2\pi)^3} e^{i \Bp \cdot \Bx}
\frac{-2 \pi i}{2 \pi} \frac{i e^{-i p_0 x^0 } }{2 \omega_\Bp}
=
\int \frac{d^3 p}{(2\pi)^3} e^{i \Bp \cdot \Bx}
\frac{ e^{-i \omega_\Bp x^0 } }{2 \omega_\Bp}.
\end{dmath}

\paragraph{Case II.  \( x^0 < 0 \)}

For \( x^0 < 0 \) we use an upper half plane contour with the same deformation around the poles.  This time
\begin{dmath}\label{eqn:qftLecture12:240}
G_\txtF
=
\int \frac{d^3 p}{(2\pi)^3} e^{i \Bp \cdot \Bx}
\int \frac{d p_0}{2 \pi} e^{-i p_0 x^0 }
\frac{i}{(p_0 - \omega_\Bp)(p_0 + \omega_\Bp)}
=
\int \frac{d^3 p}{(2\pi)^3} e^{i \Bp \cdot \Bx}
(+ 2 \pi i)
\evalbar{\lr{\frac{e^{-i p_0 x^0 }}{2 \pi}
\frac{i}{p_0 - \omega_\Bp}}}{p_0 = -\omega_\Bp}
=
\int \frac{d^3 p}{(2\pi)^3} e^{i \Bp \cdot \Bx}
\frac{+2 \pi i}{2 \pi} \frac{i e^{-i p_0 x^0 } }{-2 \omega_\Bp}
=
\int \frac{d^3 p}{(2\pi)^3} e^{i \Bp \cdot \Bx}
\frac{ e^{i \omega_\Bp x^0 } }{2 \omega_\Bp}.
\end{dmath}
We've obtained
a piecewise representation of the Green's function, where the only difference is the sign of the \( i \omega_\Bp x^0 \) exponential.

We can combine
\cref{eqn:qftLecture12:220}
\cref{eqn:qftLecture12:240} by using \( \Theta \) functions
\begin{dmath}\label{eqn:qftLecture12:260}
\int \frac{d^3 p}{(2\pi)^3 2 \omega_\Bp} e^{i \Bp \cdot \Bx}
\lr{
e^{-i \omega_\Bp x^0 } \Theta(x_0)
+
e^{i \omega_\Bp x^0 } \Theta(-x_0)
}.
\end{dmath}
The first integral (without the \(\Theta\) factor) is the Wightman function
\index{Wightman function}
%\begin{dmath}\label{eqn:qftLecture12:280}
\boxedEquation{eqn:qftLecture12:280}{
D(x)
=
\int \frac{d^3 p}{(2\pi)^3 2 \omega_\Bp} \evalbar{e^{-i p \cdot x}}{p^0 = \omega_\Bp}.
}
%\end{dmath}

For the second integral, we make a change of variables \( \Bp \rightarrow -\Bp \) leaving
\begin{dmath}\label{eqn:qftLecture12:300}
\int \frac{d^3 p}{(2\pi)^3 2 \omega_\Bp} e^{i \Bp \cdot \Bx + i \omega_\Bp x^0}
\rightarrow
\int \frac{d^3 p}{(2\pi)^3 2 \omega_\Bp} e^{-i \Bp \cdot \Bx + i \omega_\Bp x^0}
=
\int \frac{d^3 p}{(2\pi)^3 2 \omega_\Bp} e^{-i p \cdot x}
= D(-x),
\end{dmath}
so
%\begin{dmath}\label{eqn:qftLecture12:320}
\index{Feynman propagator}
\boxedEquation{eqn:qftLecture12:340}{
G_\txtF (x) = \Theta(x^0) D(x) + \Theta(-x^0) D(-x).
}
%\end{dmath}

%
% Copyright © 2017 Peeter Joot.  All Rights Reserved.
% Licenced as described in the file LICENSE under the root directory of this GIT repository.
%
\section{Pole shifting.}
\index{pole shifting}

Recall that the four dimensional form of the Green's function was
\begin{dmath}\label{eqn:qftLecture13:400}
D_F = i \int \frac{d^4 p}{(2 \pi)^4} e^{-i p \cdot x} \inv{ p^2 - m^2 }.
\end{dmath}
For the Feynman case, the contour that we were taking around the poles can also be accomplished by shifting the poles strategically, as sketched in \cref{fig:feynmanDeformationTwoWays:feynmanDeformationTwoWaysFig1}.

\imageFigure{../figures/phy2403-quantum-field-theory/feynmanDeformationTwoWaysFig1}{Feynman deformation or equivalent shift of the poles.}{fig:feynmanDeformationTwoWays:feynmanDeformationTwoWaysFig1}{0.3}

This shift can be expressed explicit algebraically by introducing an offset
\begin{dmath}\label{eqn:qftLecture13:420}
D_F = i \int \frac{d^4 p}{(2 \pi)^4} e^{-i p \cdot x} \inv{ p^2 - m^2 + i \epsilon },
\end{dmath}
which puts the poles at
\begin{dmath}\label{eqn:qftLecture13:440}
p^0
= \pm \sqrt{ \omega_\Bp^2 - i \epsilon }
= \pm \omega_\Bp \lr{ 1 - \frac{i \epsilon}{\omega_\Bp^2} }^{1/2}
= \pm \omega_\Bp \lr{ 1 - \inv{2} \frac{i \epsilon}{\omega_\Bp^2} }
=
\left\{
\begin{array}{l}
+\omega_\Bp - \inv{2} i \frac{\epsilon}{\omega_\Bp} \\
-\omega_\Bp + \inv{2} i \frac{\epsilon}{\omega_\Bp} \\
\end{array}
\right.
\end{dmath}



\section{Matrix element representation of the Wightman function.}
\index{Wightman function}

Recall that the Wightman function \cref{eqn:qftLecture12:280} also had a matrix element representation
\begin{dmath}\label{eqn:qftLecture12:360}
D(x) = \bra{0} \phi(x) \phi(0) \ket{0}.
\end{dmath}
This can be shown by expansion.
\begin{dmath}\label{eqn:qftLecture12:380}
\bra{0} \phi(x) \phi(0) \ket{0}
=
\bra{0}
\int \frac{d^3 p}{(2 \pi)^3} \inv{\sqrt{2 \omega_\Bp}} \evalbar{\lr{ a_\Bp e^{-i p \cdot x} + a_\Bp^\dagger e^{i p \cdot x} }}{p_0 = \omega_\Bp}
\int \frac{d^3 q}{(2 \pi)^3} \inv{\sqrt{2 \omega_\Bq}} \lr{ a_\Bq^\dagger + a_\Bq }
\ket{0}.
\end{dmath}
Since \( a_\Bq \ket{0} = 0 = \bra{0} a_\Bp^\dagger \), \cref{eqn:qftLecture12:380}
reduces to
\begin{dmath}\label{eqn:qftLecture12:540}
\bra{0} \phi(x) \phi(0) \ket{0}
=
\bra{0}
\int
\frac{d^3 p}{(2 \pi)^3}
\frac{d^3 q}{(2 \pi)^3}
\inv{\sqrt{2 \omega_\Bp}}
\inv{\sqrt{2 \omega_\Bq}}
\evalbar{\lr{ a_\Bp a_\Bq^\dagger e^{-i p \cdot x} }}{p_0 = \omega_\Bp}
\ket{0}
=
\bra{0}
\int
\frac{d^3 p}{(2 \pi)^3}
\frac{d^3 q}{(2 \pi)^3}
\inv{\sqrt{2 \omega_\Bp}}
\inv{\sqrt{2 \omega_\Bq}}
\evalbar{\lr{
\lr{
   a_\Bp
   a_\Bq^\dagger
   +
   \antisymmetric{
   a_\Bp
   }{
   a_\Bq^\dagger
   }
}
e^{-i p \cdot x} }}{p_0 = \omega_\Bp}
\ket{0}
=
\bra{0}
\int
\frac{d^3 p}{(2 \pi)^3}
\frac{d^3 q}{(2 \pi)^3}
\inv{\sqrt{2 \omega_\Bp}}
\inv{\sqrt{2 \omega_\Bq}}
\evalbar{\lr{
\lr{
   (2 \pi)^3 \delta^3(\Bp - \Bq)
}
e^{-i p \cdot x} }}{p_0 = \omega_\Bp}
=
\int \frac{d^3 p}{(2 \pi)^3} \evalbar{ \frac{e^{-i p \cdot x}}{2 \omega_\Bp} }
{p_0 = \omega_\Bp}.
\end{dmath}

\section{Retarded Green's function.}

Claim: Retarded Green's function (bumps up contour) can be written
\begin{dmath}\label{eqn:qftLecture12:400}
D_R(x) = \theta(x_0) (D(x) - D(-x)),
\end{dmath}
where \( D(x) \) is given by \cref{eqn:qftLecture12:280}.
Proof: The upper half plane contour (\(x_0 < 0\)) is zero since it encloses no poles.  For the
lower half plane contour we have
\begin{dmath}\label{eqn:qftLecture12:420}
\evalbar{D_R(x)}{x_0 > 0}
=
i \int \frac{d^3 p}{(2 \pi)^3} e^{i \Bp \cdot \Bx} \int \frac{dp_0}{2 \pi} e^{-i p_0 x^0 }
\frac{i}{(p_0 - \omega_\Bp)(p_0 + \omega_\Bp)}
=
i \int \frac{d^3 p}{(2 \pi)^3} e^{i \Bp \cdot \Bx} \frac{(-2 \pi i)}{2 \pi}
\lr{
e^{-i \omega_\Bp x^0 }
\frac{i}{2 \omega_\Bp}
+
e^{i \omega_\Bp x^0 }
\frac{i}{-2 \omega_\Bp}
}
=
\int \frac{d^3 p}{(2 \pi)^3} e^{i \Bp \cdot \Bx}
\frac{1}{2 \omega_\Bp}
\lr{
e^{-i \omega_\Bp x^0 }
-
e^{i \omega_\Bp x^0 }
}
=
D(x) - D(-x).
\end{dmath}

What does the field look like in terms of the propagator?
Assuming that \( \phi_0 \) satisfies the homogeneous equation, we have
\begin{dmath}\label{eqn:qftLecture12:440}
\phi(x)
= \phi_0(x) + i \int d^4 y D_R(x - y) j(y)
= \phi_0(x) + i \int d^3 y d y_0 \Theta(x_0 - y_0) \lr{ D(x - y) - D(y - x) } j(y).
\end{dmath}

Imagine that we have a windowed source function \( j(y^0, \By) \), as sketched in \cref{fig:windowedImpulse:windowedImpulseFig5}.

\imageFigure{../figures/phy2403-quantum-field-theory/windowedImpulseFig5}{Finite window impulse response.}{fig:windowedImpulse:windowedImpulseFig5}{0.2}

\begin{dmath}\label{eqn:qftLecture12:460}
\begin{aligned}
\evalbar{\phi(x)}{x^0 > t_{\text{after}}}
&= \phi_0(x) \\
&+ i \int d^4 y
\Biglr{
   \int \frac{d^3 p}{(2\pi)^3 2 \omega_\Bp } e^{-i p \cdot (x - y)} j(y)
   -
   \int \frac{d^3 p}{(2\pi)^3 2 \omega_\Bp } e^{i p \cdot (x - y)} j(y)
}.
\end{aligned}
\end{dmath}
Define
\begin{dmath}\label{eqn:qftLecture12:480}
\tilde{j}(p) = \int d^4 y e^{i p \cdot y} j(y),
\end{dmath}
which gives
\begin{dmath}\label{eqn:qftLecture12:500}
\evalbar{\phi(x)}{x^0 > t_{\text{after}}}
= \phi_0(x)
+ i \int \frac{d^3 p }{(2 \pi)^3}
\inv{
2 \omega_\Bp }
\evalbar{
\lr{
   e^{-i p \cdot x} \tilde{j}(p)
   - e^{i p \cdot x} \tilde{j}(-p)
}
}{p_0 = \omega_\Bp}.
\end{dmath}
We will interpret this in the next lecture, and start in on Feynman diagrams.

%}
%\EndNoBibArticle

   \chapter{Forced Klein-Gordon equation.}
      %
% Copyright � 2017 Peeter Joot.  All Rights Reserved.
% Licenced as described in the file LICENSE under the root directory of this GIT repository.
%
%{
\input{../latex/blogpost.tex}
\renewcommand{\basename}{qftLecture13}
\renewcommand{\dirname}{notes/phy2403/}
\newcommand{\keywords}{PHY2403H}
\input{../latex/peeter_prologue_print2.tex}

%\usepackage{phy2403}
\usepackage{peeters_braket}
\usepackage{peeters_layout_exercise}
\usepackage{peeters_figures}
\usepackage{mathtools}
\usepackage{siunitx}
\usepackage{macros_cal} % LL

\newcommand{\ultensor}[3]{{{#1}^{#2}}_{#3}}

\beginArtNoToc
\generatetitle{PHY2403H Quantum Field Theory.  Lecture 13: XXX.  Taught by Prof.\ Erich Poppitz}
%\chapter{XXX}
\label{chap:qftLecture13}

%%Peeter's lecture notes from class.  These may be incoherent and rough.
%%
%%These are notes for the UofT course PHY2403H, Quantum Field Theory, taught by Prof. Erich Poppitz, covering \textchapref{{1}} \citep{peskin1995introduction} content.

\paragraph{DISCLAIMER: Very rough notes from class, with some additional side notes.}

These are notes for the UofT course PHY2403H, Quantum Field Theory, taught by Prof. Erich Poppitz, fall 2018.
%, covering \textchapref{{1}} \citep{peskin1995introduction} content.

\section{Review: ``particle creation problem''.}

F1: windowed function from last lecture.

\begin{dmath}\label{eqn:qftLecture13:20}
\lr{ \partial_\mu \partial^\mu + m^2 } \phi = j
\end{dmath}

Our solution was
\begin{dmath}\label{eqn:qftLecture13:40}
\phi(x) = \phi(x_0) + i \int d^4 y D_R( x - y) j(y),
\end{dmath}
where \( \phi(x_0) \) obeys the homogeneous equation, and
\begin{dmath}\label{eqn:qftLecture13:60}
D_r(x - y) = \Theta(x^0 - y^0) \lr{ D(x - y) - D(y - x) },
\end{dmath}
and \( D(x) = \int \frac{d^3 p}{(2\pi)^3 2 \omega_\Bp } \evalbar{ e^{-i p \cdot x} }{p^0 = \omega_\Bp} \) is the Weightmann function.

For \( x^0 > t_{\text{after}} \)
\begin{dmath}\label{eqn:qftLecture13:80}
\phi(x)
=
\int \frac{d^3 p}{(2\pi)^3 \sqrt{ 2 \omega_\Bp }}
\evalbar{
   \lr{ e^{-i p \cdot x} a_\Bp + e^{i p \cdot x } a_\Bp^\dagger }
}{
   p^0 = \omega_\Bp
}
+ i
\int \frac{d^3 p}{(2\pi)^3 2 \omega_\Bp }
\evalbar{
   \lr{ e^{-i p \cdot x} \tilde{j}(p) + e^{i p \cdot x} \tilde{j}(p_0, -\Bp) }
}{
   p^0 = \omega_\Bp
}
\end{dmath}
where we have used \( \tilde{j}^\conj(p_0, \Bp) = \tilde{j}(p_0, -\Bp) \).  This gives

\begin{dmath}\label{eqn:qftLecture13:100}
\phi(x) =
\int \frac{d^3 p}{(2\pi)^3 \sqrt{ 2 \omega_\Bp } }
\evalbar{
e^{-i p \cdot x}
\lr{ a_\Bp + i \frac{\tilde{j}(p)}{\sqrt{2 \omega_\Bp}} }
+ e^{i p \cdot x }
\lr{ a_\Bp^\dagger - i \frac{\tilde{j}^\conj(p)}{\sqrt{2 \omega_\Bp}} }
}{
p^0 = \omega_\Bp
}
\end{dmath}

DIY: Given
\begin{dmath}\label{eqn:qftLecture13:120}
H = \int d^3 x \lr{ \inv{2} \pi^2 + \inv{2} \lr{ \spacegrad \phi}^2 + \frac{m^2}{2} \phi^2 },
\end{dmath}
we should get
\begin{dmath}\label{eqn:qftLecture13:140}
H_{\text{after}} =
\int d^3 x \omega_\Bp
\lr{ a_\Bp^\dagger - i \frac{\tilde{j}^\conj(p)}{\sqrt{2 \omega_\Bp}} }
\lr{ a_\Bp + i \frac{\tilde{j}(p)}{\sqrt{2 \omega_\Bp}} }
\end{dmath}

System in ground state
\begin{equation}\label{eqn:qftLecture13:160}
\bra{0} \hatH_{\text{before}} \ket{0} = \expectation{E}_{\text{before}} = 0.
\end{equation}
\begin{dmath}\label{eqn:qftLecture13:180}
\bra{0} \hatH_{\text{after}} \ket{0} = \expectation{E}_{\text{after}}
=
\int d^3 x \omega_\Bp
\frac{ \tilde{j}^\conj(p) \tilde{j}(p)}{2 \omega_\Bp}
=
\inv{2} \int d^3 x
\Abs{j(p)}^2.
\end{dmath}
We can identify
\begin{dmath}\label{eqn:qftLecture13:200}
N(\Bp) =
\frac{\Abs{j(p)}^2}{2 \omega_\Bp},
\end{dmath}
as the number density of particles with momentum \( \Bp \).

\makedefinition{Coherent state.}{dfn:qftLecture13:220}{
A coherent state is a state
\begin{equation*}
a \ket{\alpha} = \alpha \ket{\alpha}.
\end{equation*}
} % definition
For the SHO, if we solve for such a coherent state, we find
\begin{dmath}\label{eqn:qftLecture13:240}
\alpha = \text{const} \sum_{n = 0}^\infty \frac{\alpha^n}{n!} \lr{ a^\dagger }^n \ket{0}.
\end{dmath}
For this case, we have a coherent state
\begin{dmath}\label{eqn:qftLecture13:260}
a_p \ket{
\frac{j(p)}{2 \omega_\Bp}
}
=
\frac{j(p)}{2 \omega_\Bp}
\ket{
\frac{j(p)}{2 \omega_\Bp}
}.
\end{dmath}

\section{Feynman's Green's function}

\begin{dmath}\label{eqn:qftLecture13:280}
D_F(x)
=
\Theta(x^0) D(x) +
\Theta(-x^0) D(-x)
=
\Theta(x^0) \bra{0} \phi(x) \phi(0) \ket{0}
+\Theta(x^0) \bra{0} \phi(-x) \phi(0) \ket{0}
\end{dmath}
Utilizing a translation operation \( U(a) = e^{i a_\mu P^\mu } \), where \( U(a) \phi(y) U^\dagger(a) = \phi(y + a) \), this second operation can be written as
\begin{dmath}\label{eqn:qftLecture13:300}
\bra{0} \phi(-x) \phi(0) \ket{0}
=
\bra{0} U^\dagger(a) U(a) \phi(-x) U^\dagger(a) U(a) \phi(0) U^\dagger(a) U(a) \ket{0}
=
\bra{0} U(a) \phi(-x) U^\dagger(a) U(a) \phi(0) U^\dagger(a) \ket{0}
=
\bra{0} \phi(-x + a) \phi(a) \ket{0},
\end{dmath}
In particular, with \( a = x \)
\begin{dmath}\label{eqn:qftLecture13:320}
\bra{0} \phi(-x) \phi(0) \ket{0}
=
\bra{0} \phi(0) \phi(x) \ket{0},
\end{dmath}
so the Feynman's Green function can be written
\begin{dmath}\label{eqn:qftLecture13:340}
D_F(x) =
\Theta(x^0) \bra{0} \phi(x) \phi(0) \ket{0}
+\Theta(x^0) \bra{0} \phi(x) \phi(x) \ket{0}
=
\bra{0}
\lr{
\Theta(x^0)
\phi(x) \phi(0)
+
\Theta(-x^0)
\phi(0) \phi(x)
}
\ket{0}.
\end{dmath}
We define
\makedefinition{Time ordered product.}{dfn:qftLecture13:360}{
\begin{equation*}
T(\phi(x) \phi(y)) =
\left\{
\begin{array}{l l}
\phi(x)\phi(y) & \quad \mbox{\( x^0 > y^0 \)} \\
\phi(y)\phi(x) & \quad \mbox{\( x^0 < y^0 \)} \\
\end{array}
\right.
\end{equation*}
} % definition
so the Feynman's Green function can now be written in a very simple fashion
\boxedEquation{eqn:qftLecture13:380}{
D_F(x) = \bra{0} T(\phi(x) \phi(0)) \ket{0}.
}

\paragraph{Remark}

Recall that the four dimensional form of the Green's function was
\begin{dmath}\label{eqn:qftLecture13:400}
D_F = i \int \frac{d^4 p}{(2 \pi)^4} e^{-i p \cdot x} \inv{ p^2 - m^2 }
\end{dmath}

where we were going around the poles like
F2

We can make this explicit by introducing an offset
\begin{dmath}\label{eqn:qftLecture13:420}
D_F = i \int \frac{d^4 p}{(2 \pi)^4} e^{-i p \cdot x} \inv{ p^2 - m^2 + i \epsilon }
\end{dmath}
which puts the poles at

\begin{dmath}\label{eqn:qftLecture13:440}
(p^0)^2 = \pm \sqrt{ \omega_\Bp - i \epsilon }
= \pm \omega_\Bp \lr{ 1 - \frac{i \epsilon}{\omega_\Bp^2} }^{1/2}
= \pm \omega_\Bp \lr{ 1 - \inv{2} \frac{i \epsilon}{\omega_\Bp^2} }
=
\left\{
\begin{array}{l}
+\omega_\Bp - \inv{2} i \frac{\epsilon}{\omega_\Bp} \\
-\omega_\Bp + \inv{2} i \frac{\epsilon}{\omega_\Bp} \\
\end{array}
\right.
\end{dmath}

\section{Interacting field theory: pertubation theory in QFT.}

We perturb the Hamiltonian
\begin{dmath}\label{eqn:qftLecture13:500}
H = H_0 + H_{\text{int}}
\end{dmath}
where \( H_0 \) is the free Hamiltonian and \( H_{\text{int}} \) is the interaction term (the perturbation).

\paragraph{Example:}

\begin{equation}\label{eqn:qftLecture13:460}
\begin{aligned}
H_0 &= SHO = \frac{p^2}{2} + \frac{\omega^2 q^2}{2} \\
H_{\text{int}} &= \lambda q^4,
\end{aligned}
\end{equation}
i.e.  the anharmonic oscillator.

In QFT
\begin{dmath}\label{eqn:qftLecture13:480}
\begin{aligned}
H_0 &=
\int d^3 x \lr{ \inv{2} \pi^2 + \inv{2} \lr{ \spacegrad \phi}^2 + \frac{m^2}{2} \phi^2 } \\
H_{\text{int}} &=
\lambda \int d^3 x \phi^4.
\end{aligned}
\end{dmath}

We will expand the interaction in small \( \lambda \).  Pertubation theory is the expansion in a small dimensionless coupling constant, such as
\begin{itemize}
\item \( \lambda \) in \( \lambda \phi^4 \) theory,
\item \( \alpha = e^2/4 \pi \sim \inv{137} \) in QED, and
\item \( \alpha_s \) in QCD.
\end{itemize}

\section{Pertubation theory, interaction representation and Dyson formula}

\begin{dmath}\label{eqn:qftLecture13:520}
H = H_0 + H_{\text{int}}
\end{dmath}
Example interaction
\begin{dmath}\label{eqn:qftLecture13:540}
H_{\text{int}} = \lambda \int d^3 x \phi^4
\end{dmath}

We know all there is to know about \( H_0 \) (decoupled SHOs, ...)
\begin{dmath}\label{eqn:qftLecture13:560}
H_0 \ket{0} = \ket{0} E^0_vac
\end{dmath}
where \( E^0_vac = 0 \).  Assume
\begin{dmath}\label{eqn:qftLecture13:580}
\lr{ H_0 + H_{\text{int}} } \ket{\Omega} = \ket{\Omega} E_vac,
\end{dmath}
where the ground state energy of the perturbed system is zero when \( \lambda = 0 \).  That is \( E_vac(\lambda = 0 ) = 0 \).

So for
\begin{dmath}\label{eqn:qftLecture13:600}
\evalbar{\phi(x) }{x^0 = t_0, \text{some fixed value}}
=
\int \frac{d^3}{(2 \pi)^3 \sqrt{ 2 \omega_\Bp } }
\evalbar{
   \lr{
   e^{-i p \cdot x} a_\Bp
   + e^{i p \cdot x} a_\Bp^\dagger }
   }
{
p^0 = \omega_\Bp
}.
\end{dmath}
Let's call \( \phi(\Bx, t_0) \) the free Schr\"{o}dinger operator, where
\( \phi(\Bx, t_0) \) is evaluated at a fixed value of \( t_0 \).  At such a point, the Schr\"{o}dinger and Heisenberg pictures concide.
\begin{dmath}\label{eqn:qftLecture13:620}
\antisymmetric{\phi(\Bx, t_0)}{\pi(\By, t_0)} = i \delta^3(\Bx - \By).
\end{dmath}

Normally (QM) one defines the Heisenberg oeprator as
\begin{dmath}\label{eqn:qftLecture13:640}
O_H = e^{i H(t - t_0)} O_S e^{-i H(t - t_0)},
\end{dmath}
where \( O_H \) depends on time, and \( O_S \) is defined at a fixed time \( t_0 \), usually 0.
From \cref{eqn:qftLecture13:640} we find
\begin{dmath}\label{eqn:qftLecture13:660}
\ddt{O_H} = i \antisymmetric{H}{O_H}.
\end{dmath}
The equivalent of \cref{eqn:qftLecture13:640} in QFT is very complicated.  We'd like to develop an intermediate picture.

We will define an intermediate picture, called the ``interaction representation'', which is equivalent to the Heisenberg picture with respect to \( H_0 \).
\makedefinition{Intermediate picture operator.}{dfn:qftLecture13:680}{
\begin{equation*}
\phi_I(t, \Bx) =
e^{i H_0(t - t_0) }
\phi(t_0, \Bx)
e^{-i H_0(t - t_0) }.
\end{equation*}
} % definition

This is familiar, and is the Heisenberg picture operator that we had in free QFT
\begin{dmath}\label{eqn:qftLecture13:700}
\phi_I(t, \Bx) =
\int \frac{d^3}{(2 \pi)^3 \sqrt{ 2 \omega_\Bp } }
\evalbar{
   \lr{
   e^{-i p \cdot x} a_\Bp
   + e^{i p \cdot x} a_\Bp^\dagger }
   }
{
p^0 = \omega_\Bp
},
\end{dmath}
where \( x_0 = t \).

The Hiesenberg picture operator is
\begin{dmath}\label{eqn:qftLecture13:720}
\phi_H(t, \Bx)
=
\phi(t, \Bx)
=
e^{i H(t - t_0) }
e^{-i H_0(t - t_0) }
\lr{
   e^{i H_0(t - t_0) }
   \phi_S(t_0, \Bx)
   e^{-i H_0(t - t_0) }
}
e^{i H_0(t - t_0) }
e^{-i H(t - t_0) }
=
e^{i H(t - t_0) }
e^{-i H_0(t - t_0) }
\phi_I(t, \Bx)
e^{-i H_0(t - t_0) }
e^{i H(t - t_0) }
\end{dmath}
or
\begin{dmath}\label{eqn:qftLecture13:760}
\phi_H(t, \Bx)
=
U^\dagger(t, t_0)
\phi_I(t_0, \Bx)
U(t, t_0),
\end{dmath}
where
\begin{dmath}\label{eqn:qftLecture13:740}
U(t, t_0) =
e^{i H_0(t - t_0) }
e^{-i H(t - t_0) }.
\end{dmath}

We want to apply perturbutation techniques to find \( U(t, t_0) \) which is complicated.

\begin{dmath}\label{eqn:qftLecture13:780}
i \PD{t}{} U(t, t_0)
=
i e^{i H_0(t - t_0) } i H_0
e^{-i H(t - t_0) }
+
i e^{i H_0(t - t_0) }
e^{-i H(t - t_0) } (-i H)
=
e^{i H_0(t - t_0) }
\lr{ -H_0 + H }
e^{-i H(t - t_0) }
=
e^{i H_0(t - t_0) }
H_{\text{int}}
e^{-i H_0(t - t_0) }
e^{i H_0(t - t_0) }
e^{-i H(t - t_0) }
\end{dmath}
so we have
\boxedEquation{eqn:qftLecture13:800}{
i \PD{t}{} U(t, t_0)
=
H_{\text{int}, I}(t) U(t, t_0).
}
For the (Schr\"{o}dinger) interaction \( H_{\text{int}} = \
\lambda \int d^3 x \phi^4(\Bx, t_0)  \), what we really mean by
\( H_{\text{int}, I}(t) \) is
\begin{dmath}\label{eqn:qftLecture13:820}
H_{\text{int}, I}(t) = \lambda \int d^3 x \phi_I^4(\Bx, t)
\end{dmath}

Solving \cref{eqn:qftLecture13:800} to \( O(\lambda^0) \) gives
\begin{dmath}\label{eqn:qftLecture13:840}
i \PD{t}{} U(t, t_0) = 0,
\end{dmath}
or
\begin{dmath}\label{eqn:qftLecture13:860}
U(t, t_0) = 1 + O(\lambda).
\end{dmath}
Assume
Let's rejig things by taking \( \lambda \) out of \( H_{\text{int}, I}(t) \) as
\begin{dmath}\label{eqn:qftLecture13:880}
\begin{aligned}
H_{\text{int}, I}(t) &= \int d^3 x \phi_I^4(\Bx, t) \\
i \PD{t}{} U(t, t_0)
&=
\lambda H_{\text{int}, I}(t) U(t, t_0).
\end{aligned}
\end{dmath}
and assume that
\begin{dmath}\label{eqn:qftLecture13:900}
U(t, t_0)
=
1
+ \lambda U_1(t, t_0)
+ \lambda^2 U_2(t, t_0)
+ \cdots
+ \lambda^n U_n(t, t_0)
\end{dmath}

Plugging into \cref{eqn:qftLecture13:880} and equate equal powers of \( \lambda \) on two sides
\begin{dmath}\label{eqn:qftLecture13:1160}
i \lambda \PD{t}{}U_1(t, t_0)
+ i \lambda^2 \PD{t}{}U_2(t, t_0)
+ \cdots
+ i \lambda^n \PD{t}{}U_n(t, t_0)
=
\lambda H_{\text{int}, I}(t)
\lr{
1
+ \lambda U_1(t, t_0)
+ \lambda^2 U_2(t, t_0)
+ \cdots
+ \lambda^n U_n(t, t_0)
},
\end{dmath}
so
\begin{dmath}\label{eqn:qftLecture13:1180}
\PD{t}{}U_k(t, t_0) = -i H_{\text{int}, I}(t) U_{k-1}(t, t_0).
\end{dmath}

Let's consider each power in turn
\paragraph{\(O(\lambda^1)\):}
\begin{dmath}\label{eqn:qftLecture13:920}
\lambda i \PD{t}{U_1(t, t_0)} = \lambda
H_{\text{int}, I}(t) U_0(t, t_0)
\end{dmath}
or
\begin{dmath}\label{eqn:qftLecture13:940}
i \PD{t}{U_1(t, t_0)} = H_{\text{int}, I}(t),
\end{dmath}
which has solution
\begin{dmath}\label{eqn:qftLecture13:960}
U_1(t, t_0) = -i \int_{t_0}^t H_{\text{int}, I}(t') dt'.
\end{dmath}

\paragraph{\(O(\lambda^2)\):}
\begin{dmath}\label{eqn:qftLecture13:980}
\lambda^2 i \PD{t}{U_2(t, t_0)} = \lambda^2
H_{\text{int}, I}(t) U_1(t, t_0),
\end{dmath}
or
\begin{dmath}\label{eqn:qftLecture13:1000}
i \PD{t}{U_2(t, t_0)} = H_{\text{int}, I}(t) U_1(t, t_0),
\end{dmath}
FIXME: expand this out explicitly (as jotted and erased on the board quickly)
which has solution
\begin{dmath}\label{eqn:qftLecture13:1020}
U_2(t, t_0)
= (-i )^2
\int_{t_0}^t H_{\text{int}, I}(t'') dt''.
\int_{t_0}^{t''} H_{\text{int}, I}(t') dt'
= (-i )^2
\int_{t_0}^t dt''
\int_{t_0}^{t''}
dt'
H_{\text{int}, I}(t'')
H_{\text{int}, I}(t').
\end{dmath}
The integration range is sketched in
F3 (the upper half of the triangle).

Claim: We can integrate over the entire square, and divide by two, provided we keep the time ordering
\begin{dmath}\label{eqn:qftLecture13:1040}
U_2(t, t_0)
= \frac{(-i )^2}{2}
\int_{t_0}^t dt''
\int_{t_0}^{t''}
dt'
T(H_{\text{int}, I}(t'') H_{\text{int}, I}(t') )
\end{dmath}

\paragraph{\(O(\lambda^3)\):}
\begin{dmath}\label{eqn:qftLecture13:1060}
\lambda^3 i \PD{t}{U_3(t, t_0)} = \lambda^3 H_{\text{int}, I}(t) U_2(t, t_0),
\end{dmath}
so
\begin{dmath}\label{eqn:qftLecture13:1080}
U_3(t, t_0)
=
-i
\int_{t_0}^t dt'''
 H_{\text{int}, I}(t''') dt'' U_2(t''', t_0)
=
(-i)^3
\int_{t_0}^t dt'''
\int_{t_0}^{t'''} dt''
\int_{t_0}^{t''} dt'
H_{\text{int}, I}(t''')
H_{\text{int}, I}(t'')
H_{\text{int}, I}(t')
\end{dmath}

Let's return to the claim.

\begin{dmath}\label{eqn:qftLecture13:1100}
\frac{(-i)^2}{2}
\int_{t_0}^t dt''
\int_{t_0}^t dt'
T( H_I(t'') H_I(t') )
=
\frac{(-i)^2}{2}
\int_{t_0}^t dt''
\int_{t_0}^t dt'
\Theta(t''- t')
H_I(t'') H_I(t')
+
\frac{(-i)^2}{2}
\int_{t_0}^t dt''
\int_{t_0}^t dt'
\Theta(t'- t'')
H_I(t') H_I(t'')
=
\frac{(-i)^2}{2}
\int_{t_0}^t dt''
\int_{t_0}^{t''} dt'
%\Theta(t''- t')
H_I(t'') H_I(t')
+
\frac{(-i)^2}{2}
\int_{t_0}^{t'} dt''
\int_{t_0}^t dt'
%\Theta(t'- t'')
H_I(t') H_I(t'')
=
\frac{(-i)^2}{2}
\int_{t_0}^t dt''
\int_{t_0}^{t''} dt'
%\Theta(t''- t')
H_I(t'') H_I(t')
+
\frac{(-i)^2}{2}
\int_{t_0}^{t''} dt'
\int_{t_0}^{t} dt''
H_I(t') H_I(t'')
=
U_2(t, t_0),
\end{dmath}
where we changed integration variables in second integral.

Summarizing
\begin{dmath}\label{eqn:qftLecture13:1120}
\begin{aligned}
U_0 &= 1 \\
U_1 &= -i \int_{t_0}^t dt_1 H_I(t_1) \\
U_2 &= \frac{(-i)^2}{2}
\int_{t_0}^t dt_1
\int_{t_0}^t dt_2
T( H_I(t_1)
H_I(t_2) ) \\
U_3 &= \frac{(-i)^3}{3!}
\int_{t_0}^t dt_1
\int_{t_0}^t dt_2
\int_{t_0}^t dt_3
T( H_I(t_1)
H_I(t_2)
H_I(t_3)
) \\
U_n &= \frac{(-i)^n}{n!}
\int_{t_0}^t dt_1
\int_{t_0}^t dt_2
\int_{t_0}^t dt_3
\cdots
\int_{t_0}^t dt_n
T( H_I(t_1)
H_I(t_2)
\cdots
H_I(t_n)
) \\
\end{aligned}
\end{dmath}

Summing we find
\begin{dmath}\label{eqn:qftLecture13:1140}
U(t, t_0) = T \exp\lr{-i
\int_{t_0}^t dt_1 H_I(t')
}
=
\sum_{n = 0}^\infty
\frac{(-i)^n}{n!} \int_{t_0}^t dt_1 \cdots dt_n T( H_I(t_1) \cdots H_I(t_n) ).
\end{dmath}

Goal: \( \bra{\Omega} T(\phi(x_1) \cdots \phi(x_n)) \ket{\Omega} \).

%}
%\EndArticle
\EndNoBibArticle

   \chapter{Coherent states, Number density.}
      %
% Copyright © 2018 Peeter Joot.  All Rights Reserved.
% Licenced as described in the file LICENSE under the root directory of this GIT repository.
%
\section{Digression: coherent states.}
\index{coherent state}
\makedefinition{Coherent state.}{dfn:qftLecture13:220}{
A coherent state is an eigenstate of the destruction operator
\begin{equation*}
a \ket{\alpha} = \alpha \ket{\alpha}.
\end{equation*}
} % definition
For the SHO, if we solve for such a coherent state, we find
\begin{dmath}\label{eqn:qftLecture13:240}
\ket{\alpha} = \text{constant} \times \sum_{n = 0}^\infty \frac{\alpha^n}{n!} \lr{ a^\dagger }^n \ket{0}.
\end{dmath}
If we assume the existence of a coherent state
\begin{dmath}\label{eqn:qftLecture13:260}
a_\Bp \ket{
\frac{\tilde{j}(p)}{\sqrt{2 \omega_\Bp}}
}
=
\frac{\tilde{j}(p)}{\sqrt{2 \omega_\Bp}}
\ket{
\frac{\tilde{j}(p)}{\sqrt{2 \omega_\Bp}}
},
\end{dmath}
then the expectation value of the number operator with respect to this state is the number density identified in \cref{eqn:qftLecture13:200}
\begin{equation}\label{eqn:qftLecture13:1200}
\bra{
\frac{\tilde{j}(p)}{\sqrt{2 \omega_\Bp}}
}
a_\Bp^\dagger a_\Bp
\ket{
\frac{\tilde{j}(p)}{\sqrt{2 \omega_\Bp}}
} = \frac{\Abs{\tilde{j}(p)}^2}{2 \omega_\Bp} = N(\Bp).
\end{equation}

      \section{Problems.}
         \input{qftLukeProblemSet1Problem2.tex}
   \chapter{Time ordered product, perturbation theory, Heisenberg picture, interaction picture, Dyson's formula.}
      %
% Copyright © 2017 Peeter Joot.  All Rights Reserved.
% Licenced as described in the file LICENSE under the root directory of this GIT repository.
%
%{
\section{Feynman's Green's function}
\index{Feynman propagator}

\begin{dmath}\label{eqn:qftLecture13:280}
D_F(x)
=
\Theta(x^0) D(x) +
\Theta(-x^0) D(-x)
=
\Theta(x^0) \bra{0} \phi(x) \phi(0) \ket{0}
+\Theta(x^0) \bra{0} \phi(-x) \phi(0) \ket{0}.
\end{dmath}
Utilizing a translation operation \( U(a) = e^{i a_\mu P^\mu } \), where \( U(a) \phi(y) U^\dagger(a) = \phi(y + a) \), this second operation can be written as
\begin{dmath}\label{eqn:qftLecture13:300}
\bra{0} \phi(-x) \phi(0) \ket{0}
=
\bra{0} U^\dagger(a) U(a) \phi(-x) U^\dagger(a) U(a) \phi(0) U^\dagger(a) U(a) \ket{0}
=
\bra{0} U(a) \phi(-x) U^\dagger(a) U(a) \phi(0) U^\dagger(a) \ket{0}
=
\bra{0} \phi(-x + a) \phi(a) \ket{0},
\end{dmath}
In particular, with \( a = x \)
\begin{dmath}\label{eqn:qftLecture13:320}
\bra{0} \phi(-x) \phi(0) \ket{0}
=
\bra{0} \phi(0) \phi(x) \ket{0},
\end{dmath}
so the Feynman's Green function can be written
\begin{dmath}\label{eqn:qftLecture13:340}
D_F(x) =
\Theta(x^0) \bra{0} \phi(x) \phi(0) \ket{0}
+\Theta(x^0) \bra{0} \phi(x) \phi(x) \ket{0}
=
\bra{0}
\lr{
\Theta(x^0)
\phi(x) \phi(0)
+
\Theta(-x^0)
\phi(0) \phi(x)
}
\ket{0}.
\end{dmath}
We define
\index{time ordered product}
\makedefinition{Time ordered product.}{dfn:qftLecture13:360}{
The time ordered product of two operators is defined as
\begin{equation*}
T(\phi(x) \phi(y)) =
\left\{
\begin{array}{l l}
\phi(x)\phi(y) & \quad \mbox{\( x^0 > y^0 \)} \\
\phi(y)\phi(x) & \quad \mbox{\( x^0 < y^0 \)} \\
\end{array}
\right.,
\end{equation*}
or
\begin{equation*}
T(\phi(x) \phi(y)) =
\phi(x)\phi(y) \Theta(x^0 - y^0)
+
\phi(y)\phi(x) \Theta(y^0 - x^0).
\end{equation*}
} % definition

Using this helpful construct, the Feynman's Green function can now be written in a very simple fashion
\boxedEquation{eqn:qftLecture13:380}{
D_F(x) = \bra{0} T(\phi(x) \phi(0)) \ket{0}.
}

\section{Interacting field theory: perturbation theory in QFT.}
\index{perturbation theory}

We perturb the Hamiltonian
\begin{dmath}\label{eqn:qftLecture13:500}
H = H_0 + H_{\text{int}},
\end{dmath}
where \( H_0 \) is the free Hamiltonian and \( H_{\text{int}} \) is the interaction term (the perturbation).

\paragraph{Example:}

\begin{equation}\label{eqn:qftLecture13:460}
\begin{aligned}
H_0 &= SHO = \frac{p^2}{2} + \frac{\omega^2 q^2}{2} \\
H_{\text{int}} &= \lambda q^4.
\end{aligned}
\end{equation}
i.e.  the anharmonic oscillator.

In QFT
\begin{dmath}\label{eqn:qftLecture13:480}
\begin{aligned}
H_0 &=
\int d^3 x \lr{ \inv{2} \pi^2 + \inv{2} \lr{ \spacegrad \phi}^2 + \frac{m^2}{2} \phi^2 } \\
H_{\text{int}} &=
\lambda \int d^3 x \phi^4.
\end{aligned}
\end{dmath}

We will expand the interaction in small \( \lambda \).  Perturbation theory is the expansion in a small dimensionless coupling constant, such as
\begin{itemize}
\item \( \lambda \) in \( \lambda \phi^4 \) theory,
\item \( \alpha = e^2/4 \pi \sim \inv{137} \) in QED, and
\item \( \alpha_s \) in QCD.
\end{itemize}

\section{Perturbation theory, interaction representation and Dyson formula}
\index{interaction representation}
\index{Dyson formula}

\begin{dmath}\label{eqn:qftLecture13:520}
H = H_0 + H_{\text{int}}
\end{dmath}
Example interaction
\begin{dmath}\label{eqn:qftLecture13:540}
H_{\text{int}} = \lambda \int d^3 x \phi^4.
\end{dmath}

We know all there is to know about \( H_0 \) (decoupled SHOs, ...)
\begin{dmath}\label{eqn:qftLecture13:560}
H_0 \ket{0} = \ket{0} E^0_{\text{vac}}
\end{dmath}
where \( E^0_{\text{vac}} = 0 \).  Assume
\begin{dmath}\label{eqn:qftLecture13:580}
\lr{ H_0 + H_{\text{int}} } \ket{\Omega} = \ket{\Omega} E_{\text{vac}},
\end{dmath}
where the ground state energy of the perturbed system is zero when \( \lambda = 0 \).  That is \( E_{\text{vac}}(\lambda = 0 ) = 0 \).

So for
\begin{dmath}\label{eqn:qftLecture13:600}
\evalbar{\phi(x) }{x^0 = t_0, \text{some fixed value}}
=
\int \frac{d^3}{(2 \pi)^3 \sqrt{ 2 \omega_\Bp } }
\evalbar{
   \lr{
   e^{-i p \cdot x} a_\Bp
   + e^{i p \cdot x} a_\Bp^\dagger }
   }
{
p^0 = \omega_\Bp
}.
\end{dmath}
Let's call \( \phi(\Bx, t_0) \) the free Schr\"{o}dinger operator, where
\( \phi(\Bx, t_0) \) is evaluated at a fixed value of \( t_0 \).  At such a point, the Schr\"{o}dinger and Heisenberg pictures coincide.
\begin{dmath}\label{eqn:qftLecture13:620}
\antisymmetric{\phi(\Bx, t_0)}{\pi(\By, t_0)} = i \deltathree(\Bx - \By).
\end{dmath}

Normally (QM) one defines the Heisenberg operator as
\begin{dmath}\label{eqn:qftLecture13:640}
O_H = e^{i H(t - t_0)} O_S e^{-i H(t - t_0)},
\end{dmath}
where \( O_H \) depends on time, and \( O_S \) is defined at a fixed time \( t_0 \), usually 0.
From \cref{eqn:qftLecture13:640} we find
\begin{dmath}\label{eqn:qftLecture13:660}
\ddt{O_H} = i \antisymmetric{H}{O_H}.
\end{dmath}
The equivalent of \cref{eqn:qftLecture13:640} in QFT is very complicated.  We'd like to develop an intermediate picture.

We will define an intermediate picture, called the ``interaction representation'', which is equivalent to the Heisenberg picture with respect to \( H_0 \).
\index{interaction picture}
\makedefinition{Interaction picture operator.}{dfn:qftLecture13:680}{
\begin{equation*}
\phi_I(t, \Bx) =
e^{i H_0(t - t_0) }
\phi(t_0, \Bx)
e^{-i H_0(t - t_0) }.
\end{equation*}
} % definition

This is familiar, and is the Heisenberg picture operator that we had in free QFT
\begin{dmath}\label{eqn:qftLecture13:700}
\phi_I(t, \Bx) =
\int \frac{d^3}{(2 \pi)^3 \sqrt{ 2 \omega_\Bp } }
\evalbar{
   \lr{
   e^{-i p \cdot x} a_\Bp
   + e^{i p \cdot x} a_\Bp^\dagger }
   }
{
p^0 = \omega_\Bp
},
\end{dmath}
where \( x_0 = t \).

The Heisenberg picture operator is
\begin{dmath}\label{eqn:qftLecture13:720}
\phi_H(t, \Bx)
=
\phi(t, \Bx)
=
e^{i H(t - t_0) }
e^{-i H_0(t - t_0) }
\lr{
   e^{i H_0(t - t_0) }
   \phi_S(t_0, \Bx)
   e^{-i H_0(t - t_0) }
}
e^{i H_0(t - t_0) }
e^{-i H(t - t_0) }
=
e^{i H(t - t_0) }
e^{-i H_0(t - t_0) }
\phi_I(t, \Bx)
e^{-i H_0(t - t_0) }
e^{i H(t - t_0) }
\end{dmath}
or
\begin{dmath}\label{eqn:qftLecture13:760}
\phi_H(t, \Bx)
=
U^\dagger(t, t_0)
\phi_I(t_0, \Bx)
U(t, t_0),
\end{dmath}
where
\begin{dmath}\label{eqn:qftLecture13:740}
U(t, t_0) =
e^{i H_0(t - t_0) }
e^{-i H(t - t_0) }.
\end{dmath}

We want to apply perturbation techniques to find \( U(t, t_0) \) which is complicated.

\begin{dmath}\label{eqn:qftLecture13:780}
i \PD{t}{} U(t, t_0)
=
i e^{i H_0(t - t_0) } i H_0
e^{-i H(t - t_0) }
+
i e^{i H_0(t - t_0) }
e^{-i H(t - t_0) } (-i H)
=
e^{i H_0(t - t_0) }
\lr{ -H_0 + H }
e^{-i H(t - t_0) }
=
e^{i H_0(t - t_0) }
H_{\text{int}}
e^{-i H_0(t - t_0) }
e^{i H_0(t - t_0) }
e^{-i H(t - t_0) },
\end{dmath}
so we have
\boxedEquation{eqn:qftLecture13:800}{
i \PD{t}{} U(t, t_0)
=
H_{\text{int}, I}(t) U(t, t_0).
}
For the (Schr\"{o}dinger) interaction \( H_{\text{int}} = \
\lambda \int d^3 x \phi^4(\Bx, t_0)  \), what we really mean by
\( H_{\text{int}, I}(t) \) is
\begin{dmath}\label{eqn:qftLecture13:820}
H_{\text{int}, I}(t) = \lambda \int d^3 x \phi_I^4(\Bx, t).
\end{dmath}

It will be more convenient to remove the explicit \( \lambda \) factor from the interaction Hamiltonian, and write instead
\begin{dmath}\label{eqn:qftLecture13:880}
H_{\text{int}, I}(t) = \int d^3 x \phi_I^4(\Bx, t),
\end{dmath}
so the equation to solve is
\begin{dmath}\label{eqn:qftLecture13:1220}
i \PD{t}{} U(t, t_0)
=
\lambda H_{\text{int}, I}(t) U(t, t_0).
\end{dmath}

We assume that
\begin{dmath}\label{eqn:qftLecture13:900}
U(t, t_0)
=
U_0(t, t_0)
+ \lambda U_1(t, t_0)
+ \lambda^2 U_2(t, t_0)
+ \cdots
+ \lambda^n U_n(t, t_0).
\end{dmath}

Plugging into \cref{eqn:qftLecture13:880} we have
\begin{dmath}\label{eqn:qftLecture13:1160}
\begin{aligned}
  i &\lambda^0 \PD{t}{}U_0(t, t_0)
+ i \lambda^1 \PD{t}{}U_1(t, t_0)
+ i \lambda^2 \PD{t}{}U_2(t, t_0)
+ \cdots
+ i \lambda^n \PD{t}{}U_n(t, t_0) \\
&=
\lambda H_{\text{int}, I}(t)
\lr{
1
+ \lambda U_1(t, t_0)
+ \lambda^2 U_2(t, t_0)
+ \cdots
+ \lambda^n U_n(t, t_0)
},
\end{aligned},
\end{dmath}
so
equating equal powers of \( \lambda \) on each side gives a recurrence relation for each \( U_k, k > 0 \)
\begin{dmath}\label{eqn:qftLecture13:1180}
\PD{t}{}U_k(t, t_0) = -i H_{\text{int}, I}(t) U_{k-1}(t, t_0).
\end{dmath}

Let's consider each power in turn.
\paragraph{\(O(\lambda^0)\):}

Solving \cref{eqn:qftLecture13:800} to \( O(\lambda^0) \) gives
\begin{dmath}\label{eqn:qftLecture13:840}
i \PD{t}{} U_0(t, t_0) = 0,
\end{dmath}
or
\begin{dmath}\label{eqn:qftLecture13:860}
U(t, t_0) = 1 + O(\lambda).
\end{dmath}

\paragraph{\(O(\lambda^1)\):}
%%\begin{dmath}\label{eqn:qftLecture13:920}
%%\lambda i \PD{t}{U_1(t, t_0)} = \lambda
%%H_{\text{int}, I}(t) U_0(t, t_0)
%%\end{dmath}
%%or
\begin{dmath}\label{eqn:qftLecture13:940}
\PD{t}{U_1(t, t_0)} = -i H_{\text{int}, I}(t),
\end{dmath}
which has solution
\begin{dmath}\label{eqn:qftLecture13:960}
U_1(t, t_0) = -i \int_{t_0}^t H_{\text{int}, I}(t') dt'.
\end{dmath}

\paragraph{\(O(\lambda^2)\):}
%%\begin{dmath}\label{eqn:qftLecture13:980}
%%\lambda^2 i \PD{t}{U_2(t, t_0)} = \lambda^2
%%H_{\text{int}, I}(t) U_1(t, t_0),
%%\end{dmath}
%%or
\begin{dmath}\label{eqn:qftLecture13:1000}
\PD{t}{U_2(t, t_0)}
= -i H_{\text{int}, I}(t) U_1(t, t_0)
= (-i)^2 H_{\text{int}, I}(t)
\int_{t_0}^t H_{\text{int}, I}(t') dt',
\end{dmath}
which has solution
\begin{dmath}\label{eqn:qftLecture13:1020}
U_2(t, t_0)
= (-i )^2
\int_{t_0}^t H_{\text{int}, I}(t'') dt''
\int_{t_0}^{t''} H_{\text{int}, I}(t') dt'
= (-i )^2
\int_{t_0}^t dt''
\int_{t_0}^{t''}
dt'
H_{\text{int}, I}(t'')
H_{\text{int}, I}(t').
\end{dmath}
\paragraph{\(O(\lambda^3)\):}
\begin{dmath}\label{eqn:qftLecture13:1060}
\PD{t}{U_3(t, t_0)}
=
-i
H_{\text{int}, I}(t) U_2(t, t_0),
\end{dmath}
so
\begin{dmath}\label{eqn:qftLecture13:1240}
U_3(t, t_0)
=
-i
\int_{t_0}^t dt'''
H_{\text{int}, I}(t''') U_2(t''', t_0)
=
(-i )^3
\int_{t_0}^t dt'''
H_{\text{int}, I}(t''')
\int_{t_0}^{t'''} dt''
\int_{t_0}^{t''}
dt'
H_{\text{int}, I}(t'')
H_{\text{int}, I}(t')
=
(-i)^3
\int_{t_0}^t dt'''
\int_{t_0}^{t'''} dt''
\int_{t_0}^{t''} dt'
H_{\text{int}, I}(t''')
H_{\text{int}, I}(t'')
H_{\text{int}, I}(t').
\end{dmath}

\paragraph{Simplifying the integration region.}

For the two fold integral, the integration range is the upper triangular region sketched in \cref{fig:upperTriangleIntegrationRegion:upperTriangleIntegrationRegionFig2}.
\imageFigure{../figures/phy2403-quantum-field-theory/upperTriangleIntegrationRegionFig2}{Upper triangular integration region.}{fig:upperTriangleIntegrationRegion:upperTriangleIntegrationRegionFig2}{0.3}

\paragraph{Claim:} We can integrate over the entire square, and divide by two, provided we keep the time ordering
\begin{dmath}\label{eqn:qftLecture13:1040}
U_2(t, t_0)
= \frac{(-i )^2}{2}
\int_{t_0}^t dt''
\int_{t_0}^{t''}
dt'
T(H_{\text{int}, I}(t'') H_{\text{int}, I}(t') )
\end{dmath}

Demonstration:
\begin{dmath}\label{eqn:qftLecture13:1100}
\begin{aligned}
\frac{(-i)^2}{2}
&\int_{t_0}^t dt''
\int_{t_0}^t dt'
T( H_I(t'') H_I(t') ) \\
&=
\frac{(-i)^2}{2}
\int_{t_0}^t dt''
\int_{t_0}^t dt'
\Theta(t''- t')
H_I(t'') H_I(t')
+
\frac{(-i)^2}{2}
\int_{t_0}^t dt''
\int_{t_0}^t dt'
\Theta(t'- t'')
H_I(t') H_I(t''),
\end{aligned}
\end{dmath}
but the \( \Theta(t'' - t') \) function is non-zero only for \( t'' - t' > 0 \), or \( t' < t'' \), and the
\( \Theta(t' - t'') \) function is non-zero only for \( t' - t'' > 0 \), or \( t'' < t' \), so we can adjust the integration ranges for
\begin{dmath}\label{eqn:qftLecture13:1260}
\begin{aligned}
\frac{(-i)^2}{2}
&\int_{t_0}^t dt''
\int_{t_0}^t dt'
T( H_I(t'') H_I(t') ) \\
&=
\frac{(-i)^2}{2}
\int_{t_0}^t dt''
\int_{t_0}^{t''} dt'
H_I(t'') H_I(t')
+
\frac{(-i)^2}{2}
\int_{t_0}^{t'} dt''
\int_{t_0}^t dt'
H_I(t') H_I(t'') \\
&=
\frac{(-i)^2}{2}
\int_{t_0}^t dt''
\int_{t_0}^{t''} dt'
H_I(t'') H_I(t')
+
\frac{(-i)^2}{2}
\int_{t_0}^t dt''
\int_{t_0}^{t''} dt'
H_I(t'') H_I(t') \\
&=
U_2(t, t_0),
\end{aligned}
\end{dmath}
where we swapped integration variables in second integral.  We can clearly do the same thing for the higher order repeated integrals, but instead of a \(1/2 = 1/2!\) adjustment for the number of orderings, we will require a \( 1/n! \) adjustment for an \( n \)-fold integral.

\paragraph{Summary:}
\begin{dmath}\label{eqn:qftLecture13:1120}
\begin{aligned}
U_0 &= 1 \\
U_1 &= -i \int_{t_0}^t dt_1 H_I(t_1) \\
U_2 &= \frac{(-i)^2}{2}
\int_{t_0}^t dt_1
\int_{t_0}^t dt_2
T( H_I(t_1)
H_I(t_2) ) \\
U_3 &= \frac{(-i)^3}{3!}
\int_{t_0}^t dt_1
\int_{t_0}^t dt_2
\int_{t_0}^t dt_3
T( H_I(t_1)
H_I(t_2)
H_I(t_3)
) \\
U_n &= \frac{(-i)^n}{n!}
\int_{t_0}^t dt_1
\int_{t_0}^t dt_2
\int_{t_0}^t dt_3
\cdots
\int_{t_0}^t dt_n
T( H_I(t_1)
H_I(t_2)
\cdots
H_I(t_n)
) \\
\end{aligned}
\end{dmath}

Summing we find
\begin{dmath}\label{eqn:qftLecture13:1140}
U(t, t_0) = T \exp\lr{-i
\int_{t_0}^t dt_1 H_I(t')
}
=
\sum_{n = 0}^\infty
\frac{(-i)^n}{n!} \int_{t_0}^t dt_1 \cdots dt_n T( H_I(t_1) \cdots H_I(t_n) ).
\end{dmath}

This is called Dyson's formula.

\section{Next time.}

Our goal is to compute: \( \bra{\Omega} T(\phi(x_1) \cdots \phi(x_n)) \ket{\Omega} \).

%}

      %
% Copyright � 2018 Peeter Joot.  All Rights Reserved.
% Licenced as described in the file LICENSE under the root directory of this GIT repository.
%
%{
%%%\input{../latex/blogpost.tex}
%%%\renewcommand{\basename}{nonhomoKGhamiltonian}
%%%%\renewcommand{\dirname}{notes/phy1520/}
%%%\renewcommand{\dirname}{notes/ece1228-electromagnetic-theory/}
%%%%\newcommand{\dateintitle}{}
%%%%\newcommand{\keywords}{}
%%%
%%%\input{../latex/peeter_prologue_print2.tex}
%%%
%%%\usepackage{peeters_layout_exercise}
%%%\usepackage{peeters_braket}
%%%\usepackage{peeters_figures}
%%%\usepackage{siunitx}
%%%\usepackage{verbatim}
%%%%\usepackage{mhchem} % \ce{}
%%%%\usepackage{macros_bm} % \bcM
%%%%\usepackage{macros_qed} % \qedmarker
%%%%\usepackage{txfonts} % \ointclockwise
%%%
%%%\beginArtNoToc
%%%
%%%\generatetitle{Hamiltonian for the non-homogeneous Klein-Gordon equation}
%\chapter{Hamiltonian for the non-homogeneous Klein-Gordon equation}
%\label{chap:nonhomoKGhamiltonian}
\section{Problems.}

\makeproblem{Hamiltonian with forcing term.}{problem:qftLecture13:1280}{
Prove \cref{eqn:qftLecture13:140}.
} % problem

\makeanswer{problem:qftLecture13:1280}{
In class we derived the field for the non-homogeneous Klein-Gordon equation
\begin{dmath}\label{eqn:nonhomoKGhamiltonian:20}
\phi(x)
= \int \frac{d^3 p}{(2\pi)^3} \inv{\sqrt{2 \omega_\Bp}}
\evalbar{
\lr{
   e^{-i p \cdot x} \lr{ a_\Bp + \frac{ i \tilde{j}(p) }{\sqrt{2 \omega_\Bp}} }
   +
   e^{i p \cdot x} \lr{ a_\Bp^\dagger - \frac{ i \tilde{j}^\conj(p) }{\sqrt{2 \omega_\Bp}} }
}
}
{
p^0 = \omega_\Bp
}
= \int \frac{d^3 p}{(2\pi)^3} \inv{\sqrt{2 \omega_\Bp}}
\lr{
   e^{-i \omega_\Bp t + i \Bp \cdot \Bx} \lr{ a_\Bp + \frac{ i \tilde{j}(p) }{\sqrt{2 \omega_\Bp}} }
   +
   e^{i \omega_\Bp t - i \Bp \cdot \Bx} \lr{ a_\Bp^\dagger - \frac{ i \tilde{j}^\conj(p) }{\sqrt{2 \omega_\Bp}} }
}.
\end{dmath}
This means that we have
\begin{equation}\label{eqn:nonhomoKGhamiltonian:40}
\begin{aligned}
\pi = \dot{\phi}
&= \int \frac{d^3 p}{(2\pi)^3} \frac{i \omega_\Bp}{\sqrt{2 \omega_\Bp}}
\lr{
   - e^{-i \omega_\Bp t + i \Bp \cdot \Bx} \lr{ a_\Bp + \frac{ i \tilde{j}(p) }{\sqrt{2 \omega_\Bp}} }
   +
   e^{i \omega_\Bp t - i \Bp \cdot \Bx} \lr{ a_\Bp^\dagger - \frac{ i \tilde{j}^\conj(p) }{\sqrt{2 \omega_\Bp}} }
} \\
(\spacegrad \phi)_k =
&= \int \frac{d^3 p}{(2\pi)^3} \frac{i p_k}{\sqrt{2 \omega_\Bp}}
\lr{
     e^{-i \omega_\Bp t + i \Bp \cdot \Bx} \lr{ a_\Bp + \frac{ i \tilde{j}(p) }{\sqrt{2 \omega_\Bp}} }
   -
   e^{i \omega_\Bp t - i \Bp \cdot \Bx} \lr{ a_\Bp^\dagger - \frac{ i \tilde{j}^\conj(p) }{\sqrt{2 \omega_\Bp}} }
},
\end{aligned}
\end{equation}
and could plug these into the Hamiltonian
\begin{dmath}\label{eqn:nonhomoKGhamiltonian:60}
H = \int d^3 p \lr{ \inv{2} \pi^2 + \inv{2} \lr{ \spacegrad \phi}^2 + \frac{m^2}{2} \phi^2 },
\end{dmath}
to find \( H \) in terms of \( \tilde{j} \) and \( a_\Bp^\dagger, a_\Bp \).  The result was mentioned in class, and it was left as an exercise to verify.

There's an easy way and a dumb way to do this exercise.  I did it the dumb way, and then after suffering through two long pages, where the equations were so long that I had to write on the paper sideways, I realized the way I should have done it.

The easy way is to observe that we've already done exactly this for the case \( \tilde{j} = 0 \), which had the answer
\begin{dmath}\label{eqn:nonhomoKGhamiltonian:80}
H = \inv{2} \int \frac{d^3 p}{(2 \pi)^3} \omega_\Bp \lr{ a_\Bp^\dagger a_\Bp + a_\Bp a_\Bp^\dagger }.
\end{dmath}
To handle this more general case, all we have to do is apply a transformation
\begin{dmath}\label{eqn:nonhomoKGhamiltonian:100}
a_\Bp \rightarrow
a_\Bp + \frac{i \tilde{j}(p)}{\sqrt{2 \omega_\Bp}},
\end{dmath}
to \cref{eqn:nonhomoKGhamiltonian:80}, which gives
\begin{dmath}\label{eqn:nonhomoKGhamiltonian:120}
H
=
\inv{2} \int \frac{d^3 p}{(2 \pi)^3} \omega_\Bp \lr{\lr{ a_\Bp + \frac{i \tilde{j}(p)}{\sqrt{2 \omega_\Bp}} }^\dagger\lr{ a_\Bp + \frac{i \tilde{j}(p)}{\sqrt{2 \omega_\Bp}} } +\lr{ a_\Bp + \frac{i \tilde{j}(p)}{\sqrt{2 \omega_\Bp}} }\lr{ a_\Bp + \frac{i \tilde{j}(p)}{\sqrt{2 \omega_\Bp}} }^\dagger }
=
\inv{2} \int \frac{d^3 p}{(2 \pi)^3} \omega_\Bp \lr{\lr{ a_\Bp^\dagger - \frac{i \tilde{j}^\conj(p)}{\sqrt{2 \omega_\Bp}} } \lr{ a_\Bp + \frac{i \tilde{j}(p)}{\sqrt{2 \omega_\Bp}} } +\lr{ a_\Bp + \frac{i \tilde{j}(p)}{\sqrt{2 \omega_\Bp}} }\lr{ a_\Bp^\dagger - \frac{i \tilde{j}^\conj(p)}{\sqrt{2 \omega_\Bp}} }
}.
\end{dmath}

Like the \( \tilde{j} = 0 \) case, we can use normal ordering.  This is easily seen by direct expansion:
\begin{dmath}\label{eqn:nonhomoKGhamiltonian:140}
\begin{aligned}
\lr{ a_\Bp^\dagger - \frac{i \tilde{j}^\conj(p)}{\sqrt{2 \omega_\Bp}} } \lr{ a_\Bp + \frac{i \tilde{j}(p)}{\sqrt{2 \omega_\Bp}} }
&=
a_\Bp^\dagger a_\Bp
- \frac{i \tilde{j}^\conj(p) a_\Bp}{\sqrt{2 \omega_\Bp}}
+ \frac{ a_\Bp^\dagger i \tilde{j}^\conj(p)}{\sqrt{2 \omega_\Bp}}
+ \frac{\Abs{j}^2}{2 \omega_\Bp} \\
\lr{ a_\Bp + \frac{i \tilde{j}(p)}{\sqrt{2 \omega_\Bp}} }\lr{ a_\Bp^\dagger - \frac{i \tilde{j}^\conj(p)}{\sqrt{2 \omega_\Bp}} }
&=
a_\Bp^\dagger a_\Bp
+ \frac{i \tilde{j}^\conj(p) a_\Bp^\dagger}{\sqrt{2 \omega_\Bp}}
- \frac{ a_\Bp i \tilde{j}^\conj(p)}{\sqrt{2 \omega_\Bp}}
+ \frac{\Abs{j}^2}{2 \omega_\Bp}.
\end{aligned}
\end{dmath}
Because \( \tilde{j} \) is just a complex valued function, it commutes with \( a_\Bp, a_\Bp^\dagger \), and these are equal up to the normal ordering, allowing us to write
\begin{equation}\label{eqn:nonhomoKGhamiltonian:160}
:H: =
\int \frac{d^3 p}{(2 \pi)^3} \omega_\Bp \lr{ a_\Bp^\dagger - \frac{i \tilde{j}^\conj(p)}{\sqrt{2 \omega_\Bp}}} \lr{ a_\Bp + \frac{i \tilde{j}(p)}{\sqrt{2 \omega_\Bp}} },
\end{equation}
which is the result mentioned in class (albeit without the explicit normal ordering syntax.)
} % answer

%}
%\EndNoBibArticle

   \chapter{Time evolution, Hamiltonian pertubation, ground state.}
      %
% Copyright � 2017 Peeter Joot.  All Rights Reserved.
% Licenced as described in the file LICENSE under the root directory of this GIT repository.
%
%{
%%\input{../latex/blogpost.tex}
%%\renewcommand{\basename}{qftLecture14}
%%\renewcommand{\dirname}{notes/phy2403/}
%%\newcommand{\keywords}{PHY2403H}
%%\input{../latex/peeter_prologue_print2.tex}
%%
%%%\usepackage{phy2403}
%%\usepackage{peeters_braket}
%%%\usepackage{peeters_layout_exercise}
%%\usepackage{peeters_figures}
%%\usepackage{mathtools}
%%\usepackage{siunitx}
%%\usepackage{macros_cal} % LL
%%
%%\newcommand{\ultensor}[3]{{{#1}^{#2}}_{#3}}
%%% https://tex.stackexchange.com/a/68357/15
%%\DeclareMathOperator*{\SumInt}{%
%%\mathchoice%
%%  {\ooalign{$\displaystyle\sum$\cr\hidewidth$\displaystyle\int$\hidewidth\cr}}
%%  {\ooalign{\raisebox{.14\height}{\scalebox{.7}{$\textstyle\sum$}}\cr\hidewidth$\textstyle\int$\hidewidth\cr}}
%%  {\ooalign{\raisebox{.2\height}{\scalebox{.6}{$\scriptstyle\sum$}}\cr$\scriptstyle\int$\cr}}
%%  {\ooalign{\raisebox{.2\height}{\scalebox{.6}{$\scriptstyle\sum$}}\cr$\scriptstyle\int$\cr}}
%%}
%%
%%\beginArtNoToc
%%\generatetitle{PHY2403H Quantum Field Theory.  Lecture 14: Time evolution, Hamiltonian perturbation, ground state.  Taught by Prof.\ Erich Poppitz}
\chapter{Time evolution, Hamiltonian pertubation, ground state}
\index{time evolution}
\index{pertubation!Hamiltonian}
\index{ground state}
\label{chap:qftLecture14}

%,$ s/_Iint/_{\\text{I,int}}
%,$ s/_int/_{\\text{int}}

%%Peeter's lecture notes from class.  These may be incoherent and rough.
%%
%%These are notes for the UofT course PHY2403H, Quantum Field Theory, taught by Prof. Erich Poppitz, covering \textchapref{{1}} \citep{peskin1995introduction} content.

%\paragraph{DISCLAIMER: Very rough notes from class, with some additional side notes.}
%
%These are notes for the UofT course PHY2403H, Quantum Field Theory, taught by Prof. Erich Poppitz, fall 2018.
%%, covering \textchapref{{1}} \citep{peskin1995introduction} content.
%
\section{Review}

Given a field \( \phi(t_0, \Bx) \), satisfying the commutation relations
\begin{dmath}\label{eqn:qftLecture14:20}
\antisymmetric{\pi(t_0, \Bx)}{\phi(t_0, \By)} = -i \delta(\Bx - \By)
\end{dmath}
we introduced an interaction picture field given by
\begin{dmath}\label{eqn:qftLecture14:40}
\phi_I(t, x) = e^{i H_0(t- t_0)} \phi(t_0, \Bx) e^{-iH_0(t - t_0)}
\end{dmath}
related to the Heisenberg picture representation by
\begin{dmath}\label{eqn:qftLecture14:60}
\phi_H(t, x)
= e^{i H(t- t_0)} \phi(t_0, \Bx) e^{-iH(t - t_0)}
= U^\dagger(t, t_0) \phi_I(t, \Bx) U(t, t_0),
\end{dmath}
where \( U(t, t_0) \) is the time evolution operator.
\begin{dmath}\label{eqn:qftLecture14:80}
U(t, t_0) =
e^{i H_0(t - t_0)}
e^{-i H(t - t_0)}
\end{dmath}
We argued that
\begin{dmath}\label{eqn:qftLecture14:100}
i \PD{t}{} U(t, t_0) = H_{\text{I,int}}(t) U(t, t_0)
\end{dmath}
We found the glorious expression
\boxedEquation{eqn:qftLecture14:120}{
\begin{aligned}
U(t, t_0)
&= T \exp{\lr{ -i \int_{t_0}^t H_{\text{I,int}}(t') dt'}} \\
&=
\sum_{n = 0}^\infty \frac{(-i)^n}{n!} \int_{t_0}^t dt_1 dt_2 \cdots dt_n T\lr{ H_{\text{I,int}}(t_1) H_{\text{I,int}}(t_2) \cdots H_{\text{I,int}}(t_n) }
\end{aligned}
}

However, what we are really after is
\begin{dmath}\label{eqn:qftLecture14:140}
\bra{\Omega} T(\phi(x_1) \cdots \phi(x_n)) \ket{\Omega}
\end{dmath}
Such a product has many labels and names, and we'll describe it as ``vacuum expectation values of time-ordered products of arbitrary \#s of local Heisenberg operators''.

\section{Perturbation}
\index{perturbation}
Following \S 4.2, \citep{peskin1995introduction}.

\begin{dmath}\label{eqn:qftLecture14:160}
\begin{aligned}
H &= \text{exact Hamiltonian} = H_0 + H_{\text{int}}
\\
H_0 &= \text{free Hamiltonian.
}
\end{aligned}
\end{dmath}
We know all about \( H_0 \) and assume that it has a lowest (ground state) \( \ket{0} \), the ``vacuum'' state of \( H_0 \).

\( H \) has eigenstates, in particular \( H \) is assumed to have a unique ground state \( \ket{\Omega} \) satisfying
\begin{dmath}\label{eqn:qftLecture14:180}
H \ket{\Omega}  = \ket{\Omega} E_0,
\end{dmath}
and has states \( \ket{n} \), representing excited (non-vacuum states with energies > \( E_0 \)).
These states are assumed to be a complete basis
\begin{dmath}\label{eqn:qftLecture14:200}
\BOne = \ket{\Omega}\bra{\Omega} + \sum_n \ket{n}\bra{n} + \int dn \ket{n}\bra{n}.
\end{dmath}
The latter terms may be written with a superimposed sum-integral notation as
\begin{dmath}\label{eqn:qftLecture14:440}
\sum_n + \int dn
=
\SumInt_n,
\end{dmath}
so the identity operator takes the more compact form
\begin{dmath}\label{eqn:qftLecture14:460}
\BOne = \ket{\Omega}\bra{\Omega} + \SumInt_n \ket{n}\bra{n}.
\end{dmath}

For some time \( T \) we have
\begin{dmath}\label{eqn:qftLecture14:220}
e^{-i H T} \ket{0} = e^{-i H T}
\lr{
   \ket{\Omega}\braket{\Omega}{0} + \SumInt_n \ket{n}\braket{n}{0}
}.
\end{dmath}

We now wish to argue that the \( \SumInt_n \) term can be ignored.
\paragraph{Argument 1:}

This is something of a fast one, but one can consider a formal transformation \( T \rightarrow T(1 - i \epsilon) \), where \( \epsilon \rightarrow 0^+ \), and consider very large \( T \).  This gives
\begin{dmath}\label{eqn:qftLecture14:240}
\lim_{T \rightarrow \infty, \epsilon \rightarrow 0^+}
e^{-i H T(1 - i \epsilon)} \ket{0}
=
\lim_{T \rightarrow \infty, \epsilon \rightarrow 0^+}
e^{-i H T(1 - i \epsilon)}
\lr{
   \ket{\Omega}\braket{\Omega}{0} + \SumInt_n \ket{n}\braket{n}{0}
}
=
\lim_{T \rightarrow \infty, \epsilon \rightarrow 0^+}
e^{-i E_0 T - E_0 \epsilon T}
   \ket{\Omega}\braket{\Omega}{0} + \SumInt_n e^{-i E_n T - \epsilon E_n T} \ket{n}\braket{n}{0}
=
\lim_{T \rightarrow \infty, \epsilon \rightarrow 0^+}
e^{-i E_0 T - E_0 \epsilon T}
\lr{
   \ket{\Omega}\braket{\Omega}{0} + \SumInt_n e^{-i (E_n -E_0) T - \epsilon T (E_n - E_0)} \ket{n}\braket{n}{0}
}
\end{dmath}
The limits are evaluated by first taking \( T \) to infinity, then only after that take \( \epsilon \rightarrow 0^+ \).  Doing this, the sum is dominated by the ground state contribution, since each excited state also has a \( e^{-\epsilon T(E_n - E_0)} \) suppression factor (in addition to the leading suppression factor).

\paragraph{Argument 2:}
With the hand waving required for the argument above, it's worth pointing other (less formal) ways to arrive at the same result.  We can write
\begin{dmath}\label{eqn:qftLecture14:260}
\SumInt \ket{n}\bra{n} \rightarrow
\sum_k \int \frac{d^3 p}{(2 \pi)^3} \ket{\Bp, k}\bra{\Bp, k}
\end{dmath}
where \( k \) is some unknown quantity that we are summing over.
If we have
\begin{dmath}\label{eqn:qftLecture14:280}
H \ket{\Bp, k} = E_{\Bp, k} \ket{\Bp, k},
\end{dmath}
then
\begin{dmath}\label{eqn:qftLecture14:300}
e^{-i H T} \SumInt \ket{n}\bra{n}
=
\sum_k \int \frac{d^3 p}{(2 \pi)^3} \ket{\Bp, k} e^{-i E_{\Bp, k}} \bra{\Bp, k}.
\end{dmath}
If we take matrix elements
\begin{dmath}\label{eqn:qftLecture14:320}
\bra{A}
e^{-i H T} \SumInt \ket{n}\bra{n} \ket{B}
=
\sum_k \int \frac{d^3 p}{(2 \pi)^3} \braket{A}{\Bp, k} e^{-i E_{\Bp, k}} \braket{\Bp, k}{B}
=
\sum_k \int \frac{d^3 p}{(2 \pi)^3} e^{-i E_{\Bp, k}} f(\Bp).
\end{dmath}
If we assume that \( f(\Bp) \) is a well behaved smooth function, we have ``infinite'' frequency oscillation within the envelope provided by the amplitude of that function, as depicted in \cref{fig:RiemannLebesque:RiemannLebesqueFig1}.
The Riemann-Lebesgue lemma \citep{wiki:RiemannLebesgue} describes such integrals, the result of which is that such an integral goes to zero.  This is a different sort of hand waving argument, but either way, we can argue that only the ground state contributes to the sum
\cref{eqn:qftLecture14:220}
 above.
\imageFigure{../figures/phy2403-quantum-field-theory/RiemannLebesqueFig1}{High frequency oscillations within envelope of well behaved function.}{fig:RiemannLebesque:RiemannLebesqueFig1}{0.3}

\paragraph{Ground state of the perturbed Hamiltonian.}
\index{ground state}

With the excited states ignored, we are left with
\begin{dmath}\label{eqn:qftLecture14:340}
e^{-i H T} \ket{0} = e^{-i E_0 T} \ket{\Omega}\braket{\Omega}{0}
\end{dmath}
in the \( T \rightarrow \infty(1 - i \epsilon) \) limit.  We can now write the ground state as

\begin{dmath}\label{eqn:qftLecture14:360}
\ket{\Omega}
=
\evalbar{
\frac{ e^{i E_0 T - i H T } \ket{0} }{
\braket{\Omega}{0}
}
}{ T \rightarrow \infty(1 - i \epsilon) }
=
\evalbar{
   \frac{ e^{- i H T } \ket{0} }{
   e^{-i E_0 T} \braket{\Omega}{0}
   }
}{ T \rightarrow \infty(1 - i \epsilon) }.
\end{dmath}
Shifting the very large \( T \rightarrow T + t_0 \) shouldn't change things, so
\begin{dmath}\label{eqn:qftLecture14:480}
\ket{\Omega}
=
\evalbar{
   \frac{ e^{- i H (T + t_0) } \ket{0} }{
   e^{-i E_0 (T + t_0) } \braket{\Omega}{0}
   }
}{ T \rightarrow \infty(1 - i \epsilon) }.
\end{dmath}
%so we may insert an \( e^{-i H_0(-T -t_0)} \) factor without effect
%\begin{dmath}\label{eqn:qftLecture14:500}
%\ket{\Omega}
%%=
%%\evalbar{
%%\frac{ e^{i H (t_0 - (-T)) } e^{ -i H_0 (-T - t_0) } \ket{0} }{
%%e^{-i E_0(t_0 - (-T))} \braket{\Omega}{0}
%%}
%%}{ T \rightarrow \infty(1 - i \epsilon) }.
%%\end{dmath}

A bit of manipulation shows that the operator in the numerator has the structure of a time evolution operator.
%%With
%%\begin{dmath}\label{eqn:qftLecture14:380}
%%U(t, t_0)
%%= e^{i H_0(t - t_0)} e^{-i H(t - t_0)}
%%= T \exp{\lr{ -i \int_{t_0}^t H_{\text{I,int}}(t') dt'}}
%%\end{dmath}
%%
\paragraph{Claim: (DIY):}
\Cref{eqn:qftLecture14:80}, \cref{eqn:qftLecture14:120} may be generalized to
\begin{equation}\label{eqn:qftLecture14:400}
U(t, t') = e^{i H_0(t - t_0)} e^{-i H(t - t')} e^{-i H_0(t' - t_0)} =
T \exp{\lr{ -i \int_{t'}^t H_{\text{I,int}}(t'') dt''}}.
\end{equation}
Observe that we recover \cref{eqn:qftLecture14:120} when \( t' = t_0 \).
Using \cref{eqn:qftLecture14:400} we find
\begin{dmath}\label{eqn:qftLecture14:520}
U(t_0, -T) \ket{0}
= e^{i H_0(t_0 - t_0)} e^{-i H(t_0 + T)} e^{-i H_0(-T - t_0)} \ket{0}
= e^{-i H(t_0 + T)} e^{-i H_0(-T - t_0)} \ket{0}
= e^{-i H(t_0 + T)} \ket{0},
\end{dmath}
where we use the fact that \(
%\begin{equation}\label{eqn:qftLecture14:540}
e^{i H_0 \tau} \ket{0} = \lr{ 1 + i H_0 \tau + \cdots } \ket{0} = 1 \ket{0},
%\end{equation}
\)
since \( H_0 \ket{0} = 0 \).

We are left with
\index{ground state}
\boxedEquation{eqn:qftLecture14:420}{
\ket{\Omega}
= \frac{U(t_0, -T) \ket{0} }{e^{-i E_0(t_0 - (-T))} \braket{\Omega}{0}}.
}

We are close to where we want to be.  Wednesday we finish off, and then start scattering and Feynman diagrams.

%}
%\EndArticle

   \chapter{Perturbation ground state, time evolution operator, time ordered product, interaction.}
      %
% Copyright � 2017 Peeter Joot.  All Rights Reserved.
% Licenced as described in the file LICENSE under the root directory of this GIT repository.
%
%{
\input{../latex/blogpost.tex}
\renewcommand{\basename}{qftLecture15}
\renewcommand{\dirname}{notes/phy2403/}
\newcommand{\keywords}{PHY2403H}
\input{../latex/peeter_prologue_print2.tex}

%\usepackage{phy2403}
\usepackage{peeters_braket}
%\usepackage{peeters_layout_exercise}
\usepackage{peeters_figures}
\usepackage{mathtools}
\usepackage{siunitx}
\usepackage{macros_cal} % LL

\newcommand{\ultensor}[3]{{{#1}^{#2}}_{#3}}

\beginArtNoToc
\generatetitle{PHY2403H Quantum Field Theory.  Lecture 15: XXX.  Taught by Prof.\ Erich Poppitz}
%\chapter{XXX}
\label{chap:qftLecture15}

\paragraph{Disclaimer}

%%Peeter's lecture notes from class.  These may be incoherent and rough.
%%
%%These are notes for the UofT course PHY2403H, Quantum Field Theory, taught by Prof. Erich Poppitz, covering \textchapref{{1}} \citep{peskin1995introduction} content.

\paragraph{DISCLAIMER: Very rough notes from class, with some additional side notes.}

These are notes for the UofT course PHY2403H, Quantum Field Theory, taught by Prof. Erich Poppitz, fall 2018.
%, covering \textchapref{{1}} \citep{peskin1995introduction} content.

\section{Review}

We developed the interaction picture repreresentation, which is really the Heisenberg picture with respect to \( H_0 \).

Recall that we found
\begin{equation}\label{eqn:qftLecture15:20}
U(t, t') = e^{i H_0(t - t_0)} e^{-i H(t - t')} e^{-i H_0(t' - t_0)},
\end{equation}
with solution
\begin{dmath}\label{eqn:qftLecture15:200}
U(t, t')
=
T \exp{\lr{ -i \int_{t'}^t H_{\text{I,int}}(t'') dt''}},
\end{dmath}
\begin{dmath}\label{eqn:qftLecture15:220}
U(t, t')^\dagger
=
T \exp{\lr{ i \int_{t'}^{t} H_{\text{I,int}}(t'') dt''}}
=
T \exp{\lr{ -i \int_{t}^{t'} H_{\text{I,int}}(t'') dt''}}
= U(t', t),
\end{dmath}
and can use this to calculate the time evolution of a field
\begin{dmath}\label{eqn:qftLecture15:40}
\phi(\Bx, t)
=
U^\dagger(t, t_0)
\phi_I(\Bx, t)
U(t, t_0)
\end{dmath}
and found the ground state ket for \( H \) was
\begin{dmath}\label{eqn:qftLecture15:60}
\ket{\Omega}
=
\evalbar{
\frac{ U(t_0, -T) \ket{0} }
{
e^{-i E_0(T - t_0)} \braket{\Omega}{0}
}
}{T \rightarrow \infty(1 - i \epsilon)}.
\end{dmath}
\paragraph{Question:} What's the point of this, since it is self referential?
\paragraph{Answer:} We will see, and also see that it goes away.  Alternatively, you can write it as
\begin{equation*}
\ket{\Omega} \braket{\Omega}{0}
=
\evalbar{
\frac{ U(t_0, -T) \ket{0} }
{
e^{-i E_0(T - t_0)}
}
}{T \rightarrow \infty(1 - i \epsilon)}.
\end{equation*}

We can also show that
\begin{dmath}\label{eqn:qftLecture15:80}
\bra{\Omega}
=
\evalbar{
\frac{ \bra{0} U(T, t_0) }
{
e^{-i E_0(T - t_0)} \braket{0}{\Omega}
}
}{T \rightarrow \infty(1 - i \epsilon)}.
\end{dmath}

Our goal is still toe calculate
\begin{dmath}\label{eqn:qftLecture15:100}
\bra{\Omega} T \phi(x) \phi(y) \ket{\Omega}.
\end{dmath}
Claim: the ``LSZ'' theorem (a neat way of writing this) relates this to S matrix elements.

Assuming \( x^0 > y^0 \)

\begin{dmath}\label{eqn:qftLecture15:120}
\bra{\Omega} \phi(x) \phi(y) \ket{\Omega}
=
\frac{
\bra{0}
U(T, t_0)
U^\dagger(x^0, t^0)
\phi_I(x)
U(x^0, t^0)
%
U^\dagger(y^0, t^0)
\phi_I(y)
U(y^0, t^0)
U(t_0, -T)
\ket{0}
}
{
e^{-i 2 E_0 T} \Abs{\braket{0}{\Omega}}^2
}
\end{dmath}

Normalize \( \braket{\Omega}{\Omega} = 1 \), gives

\begin{dmath}\label{eqn:qftLecture15:140}
1
=
\frac{\bra{0} U(T, t_0) U(t_0, -T) \ket{0}}
{
e^{-i 2 E_0 T} \Abs{\braket{0}{\Omega}}^2
}
=
\frac{\bra{0} U(T, -T) \ket{0}}
{
e^{-i 2 E_0 T} \Abs{\braket{0}{\Omega}}^2
},
\end{dmath}
so that
\begin{dmath}\label{eqn:qftLecture15:240}
\bra{\Omega} \phi(x) \phi(y) \ket{\Omega}
=
\frac{
\bra{0}
U(T, t_0)
U^\dagger(x^0, t^0)
\phi_I(x)
U(x^0, t^0)
%
U^\dagger(y^0, t^0)
\phi_I(y)
U(y^0, t^0)
U(t_0, -T)
\ket{0}
}
{
   \bra{0} U(T, -T) \ket{0}
}
\end{dmath}

For \( t_1 > t_2 > t_3 \)
\begin{dmath}\label{eqn:qftLecture15:280}
U(t_1, t_2) U(t_2, t_3)
=
T e^{-i \int_{t_2}^{t_1} H_I}
T e^{-i \int_{t_3}^{t_2} H_I}
=
T \lr{
e^{-i \int_{t_2}^{t_1} H_I}
e^{-i \int_{t_3}^{t_2} H_I}
}
=
T(
e^{-i \int_{t_3}^{t_1} H_I}
),
\end{dmath}
with an end result of
\begin{dmath}\label{eqn:qftLecture15:320}
U(t_1, t_2) U(t_2, t_3) = U(t_1, t_3).
\end{dmath}
(DIY: work through the details -- this is a problem in \citep{peskin1995introduction})

This gives
\begin{dmath}\label{eqn:qftLecture15:300}
\bra{\Omega} \phi(x) \phi(y) \ket{\Omega}
=
\frac{
\bra{0}
U(T, x^0)
\phi_I(x)
U(x^0, y^0)
\phi_I(y)
U(y^0, -T)
\ket{0}
}
{
   \bra{0} U(T, -T) \ket{0}
}.
\end{dmath}

If \( y^0 > x^0 \) we have the same result, but the \( y \)'s will come first.

\paragraph{Claim:}
\begin{dmath}\label{eqn:qftLecture15:340}
\bra{\Omega} \phi(x) \phi(y) \ket{\Omega}
=
\frac{
   \bra{0}
   T\lr{
      \phi_I(x)
      \phi_I(y)
      e^{-i \int_{-T}^T H_{\text{I,int}}(t') dt'}
   }
   \ket{0}
}
{
   \bra{0}
      T ( e^{-i \int_{-T}^T H_{\text{I,int}}(t') dt'} )
   \ket{0}
}.
\end{dmath}

More generally
\boxedEquation{eqn:qftLecture15:360}{
\bra{\Omega}
\phi_I(x_1) \cdots
\phi_I(x_n)
\ket{\Omega}
=
\frac{
   \bra{0}
   T\lr{
\phi_I(x_1) \cdots
\phi_I(x_n)
      e^{-i \int_{-T}^T H_{\text{I,int}}(t') dt'}
   }
   \ket{0}
}
{
   \bra{0}
      T ( e^{-i \int_{-T}^T H_{\text{I,int}}(t') dt'} )
   \ket{0}
}.
}
This is the holy grail of pertubation theory.

In QFT II you will see this written in a path integral representation
\begin{dmath}\label{eqn:qftLecture15:380}
\bra{\Omega}
\phi_I(x_1) \cdots
\phi_I(x_n)
\ket{\Omega}
=
\frac
{
   \int [\calD \phi] \phi(x_1) \phi(x_2) \cdots \phi(x_n) e^{-i S[\phi]}
}
{
   \int [\calD \phi] e^{-i S[\phi]}
}.
\end{dmath}

\section{Unpacking it.}

\begin{dmath}\label{eqn:qftLecture15:400}
\int_{-T}^T H_{\text{I,int}}(t)
=
\int_{-T}^T
\int d^3 \Bx \frac{\lambda}{4} \lr{ \phi_I(\Bx, t) }^4
=
\int d^4 x
\frac{\lambda}{4} \lr{ \phi_I }^4
\end{dmath}

so we have
\begin{dmath}\label{eqn:qftLecture15:420}
\frac{
   \bra{0}
   T\lr{
\phi_I(x_1) \cdots
\phi_I(x_n)
      e^{-i \frac{\lambda}{4} \int d^4 x \phi_I^4(x) }
   }
   \ket{0}
}
{
   \bra{0}
      T
      e^{-i \frac{\lambda}{4} \int d^4 x \phi_I^4(x) }
   \ket{0}
}.
\end{dmath}

The numerator expands as
\begin{dmath}\label{eqn:qftLecture15:440}
   \bra{0} T\lr{ \phi_I(x_1) \cdots \phi_I(x_n) } \ket{0}
-i \frac{\lambda}{4} \int d^4 x
   \bra{0} T\lr{ \phi_I(x_1) \cdots \phi_I(x_n) \phi_I^4(x) }
+
\inv{2}
\lr{-i \frac{\lambda}{4}}^2 \int d^4 x d^4 y
   \bra{0} T\lr{ \phi_I(x_1) \cdots \phi_I(x_n)
      \phi_I^4(x)
      \phi_I^4(y)
} \ket{0}
+ \cdots
\end{dmath}
so we see that the problem ends up being the calculation of time ordered products.

\section{Calculating pertubation}
Let's simplify notation, dropping interaction picture suffixes, writing \( \phi(x_i) = \phi_i \).

Let's calculate \(
   \bra{0} T\lr{ \phi_1 \cdots \phi_n } \ket{0}
\).  For \( n = 2 \) we have

\begin{dmath}\label{eqn:qftLecture15:n}
\bra{0} T\lr{ \phi_1 \cdots \phi_n } \ket{0}
= D_F(x_1 - x_2) \equiv D_F(1-2)
\end{dmath}

\paragraph{TO BE CONTINUED.}
The rest of the lecture was very visual, and hard to type up.  I'll do so later.

%
% Copyright © 2018 Peeter Joot.  All Rights Reserved.
% Licenced as described in the file LICENSE under the root directory of this GIT repository.
%
%{
\makeproblem{\( U(T, t_0) U(t_0, -T) \) }{problem:qftLecture15Problems:1}{
Show that
\begin{equation*}
U(T, t_0) U(t_0, -T) = U(T, -T).
\end{equation*}
} % problem

\makeanswer{problem:qftLecture15Problems:1}{
We can see that from
\begin{equation}\label{eqn:qftLecture15Problems:160}
\begin{aligned}
U(T, t_0) &= e^{i H_0(T - t_0)} e^{-i H(T - t_0)} \cancel{e^{-i H_0(t_0 - t_0)}}  \\
U(t_0, -T) &= \cancel{e^{i H_0(t_0 - t_0)}} e^{-i H(t_0 - -T)} e^{-i H_0(-T - t_0)},
\end{aligned}
\end{equation}
so
\begin{equation}\label{eqn:qftLecture15Problems:180}
\begin{aligned}
U(T, t_0)
U(t_0, -T)
&=
e^{i H_0(T - t_0)} e^{-i H(T - t_0)} e^{-i H(t_0 + T)} e^{-i H_0(-T - t_0)}
\\&=
e^{i H_0(T - t_0)} e^{-i H 2 T } e^{-i H_0(-T - t_0)},
\end{aligned}
\end{equation}
whereas
\begin{equation}\label{eqn:qftLecture15Problems:260}
\begin{aligned}
U(T, -T)
  &= e^{i H_0(T - t_0)} e^{-i H(T - -T)} e^{-i H_0(-T - t_0)}
\\&= e^{i H_0(T - t_0)} e^{-i H 2 T} e^{-i H_0(-T - t_0)}.
\end{aligned}
\end{equation}
} % answer
%}


%}
\EndArticle
%\EndNoBibArticle

      %
% Copyright � 2017 Peeter Joot.  All Rights Reserved.
% Licenced as described in the file LICENSE under the root directory of this GIT repository.
%
%{
%%%\input{../latex/blogpost.tex}
%%%\renewcommand{\basename}{qftLecture15b}
%%%\renewcommand{\dirname}{notes/phy2403/}
%%%\newcommand{\keywords}{PHY2403H}
%%%\input{../latex/peeter_prologue_print2.tex}
%%%
%%%%\usepackage{phy2403}
%%%\usepackage{peeters_braket}
%%%\usepackage{peeters_layout_exercise}
%%%\usepackage{peeters_figures}
%%%\usepackage{mathtools}
%%%\usepackage{siunitx}
%%%\usepackage{macros_cal} % LL
%%%\usepackage{simplewick}
%%%
%%%\newcommand{\ultensor}[3]{{{#1}^{#2}}_{#3}}
%%%\newcommand{\normalorder}[1]{\text{:\({#1}\):}}
%%%
%%%\beginArtNoToc
%%%\generatetitle{PHY2403H Quantum Field Theory.  Lecture 15b: Wick's theorem, vacuum expectation, Feynman diagrams, \(\phi^4\) interaction, tree level diagrams, scattering, cross section, differential cross section.  Taught by Prof.\ Erich Poppitz}
\index{Wick's theorem}
\index{vacuum expectation}
\index{Feynman diagram}
\index{tree level diagram}
\index{scattering}
\index{cross section}
\index{scattering cross section}
\index{differential cross section}
\label{chap:qftLecture15b}

%%Peeter's lecture notes from class.  These may be incoherent and rough.
%%
%%These are notes for the UofT course PHY2403H, Quantum Field Theory, taught by Prof. Erich Poppitz, covering \textchapref{{1}} \citep{peskin1995introduction} content.

%\paragraph{DISCLAIMER: Very rough notes from class, with some additional side notes.}
%
%These are notes for the UofT course PHY2403H, Quantum Field Theory, taught by Prof. Erich Poppitz, fall 2018.
%%, covering \textchapref{{1}} \citep{peskin1995introduction} content.
%
\section{Wick contractions}

Here's a double dose of short hand, first an abbreviation for the Feynman propagator
\begin{dmath}\label{eqn:qftLecture15b:20}
D_F(1-2) \equiv D_F(x_1, x_2),
\end{dmath}
and second
\begin{dmath}\label{eqn:qftLecture15b:40}
\contraction{}{\phi}{{}_i}{\phi} \phi_i \phi_j = D_F(i - j),
\end{dmath}
which is called a contraction.

Contractions allow time ordered products to be written in a compact form.  In HW4 we are set with the task of demonstrating how this is done (i.e. proving \underline{Wick's theorem}.)

\maketheorem{Wick's theorem.}{thm:qftLecture15b:80}{
Sounds like stating the theorem is difficult, but the rough idea (from the example below) is that the time ordering of the fields has all the combinations of the pairwise contractions and normal ordered fields.
} % theorem

Illustrating by example for the time ordering of \( n = 4 \) fields, we have
\begin{dmath}\label{eqn:qftLecture15b:60}
T( \phi_1 \phi_2 \phi_3 \phi_4)
=
\normalorder{ \phi_1 \phi_2 \phi_3 \phi_4 }
+
\contraction{}{\phi}{{}_1}{\phi} \phi_1 \phi_2 \normalorder{ \phi_3 \phi_4 }
+
\contraction{}{\phi}{{}_1}{\phi} \phi_1 \phi_3 \normalorder{ \phi_2 \phi_4 }
+
\contraction{}{\phi}{{}_1}{\phi} \phi_1 \phi_4 \normalorder{ \phi_2 \phi_3 }
+
\contraction{}{\phi}{{}_2}{\phi} \phi_2 \phi_3 \normalorder{ \phi_1 \phi_4 }
+
\contraction{}{\phi}{{}_2}{\phi} \phi_2 \phi_4 \normalorder{ \phi_1 \phi_3 }
+
\contraction{}{\phi}{{}_3}{\phi} \phi_3 \phi_4 \normalorder{ \phi_1 \phi_2 }
+
\contraction{}{\phi}{{}_1}{\phi} \phi_1 \phi_2
\contraction{}{\phi}{{}_3}{\phi} \phi_3 \phi_4
+
\contraction{}{\phi}{{}_1}{\phi} \phi_1 \phi_3
\contraction{}{\phi}{{}_2}{\phi} \phi_2 \phi_4
+
\contraction{}{\phi}{{}_1}{\phi} \phi_1 \phi_4
\contraction{}{\phi}{{}_2}{\phi} \phi_2 \phi_3.
\end{dmath}

\maketheorem{Corollary: Vacuum expectation of Wick's theorem expansion}{thm:qftLecture15b:100}{
For \( n \) even
\begin{equation*}
\bra{0} T(\phi_1 \phi_2 \cdots \phi_n) \ket{0}
=
\contraction{}{\phi}{{}_1}{\phi} \phi_1 \phi_2
\contraction{}{\phi}{{}_3}{\phi} \phi_3 \phi_4
\contraction{}{\phi}{{}_5}{\phi} \phi_5 \phi_6
\cdots
\contraction{}{\phi}{{}_{n-1}}{\phi} \phi_{n-1} \phi_n
+ \text{all other terms}.
\end{equation*}
For \( n \) odd, this vanishes.
} % theorem

\section{Simplest Feynman diagrams}
For \( n = 4 \) we have
\begin{dmath}\label{eqn:qftLecture15b:120}
\bra{0} T(\phi_1 \phi_2 \phi_3 \phi_4) \ket{0}
=
\contraction{}{\phi}{{}_1}{\phi} \phi_1 \phi_2
\contraction{}{\phi}{{}_3}{\phi} \phi_3 \phi_4
+
\contraction{}{\phi}{{}_1}{\phi} \phi_1 \phi_3
\contraction{}{\phi}{{}_2}{\phi} \phi_2 \phi_4
+
\contraction{}{\phi}{{}_1}{\phi} \phi_1 \phi_4
\contraction{}{\phi}{{}_2}{\phi} \phi_2 \phi_3,
\end{dmath}
the set of Wick contractions can be written pictorially \cref{fig:qftLecture15b:qftLecture15bFig1}, and are called Feynman diagrams
\imageFigure{../figures/phy2403-quantum-field-theory/qftLecture15bFig1}{Simplest Feynman diagrams.}{fig:qftLecture15b:qftLecture15bFig1}{0.15}

These are the very simplest Feynman diagrams.

\section{\( \phi^4 \) interaction}
Introducing another shorthand, we will use an expectation like notation to designate the matrix element for the vacuum state
\begin{dmath}\label{eqn:qftLecture15b:140}
\expectation{\text{blah}} = \bra{0}\text{blah} \ket{0}.
\end{dmath}
For the \( \phi^4 \) theory, this allows us to write the numerator of the perturbed ground state interaction as
\begin{dmath}\label{eqn:qftLecture15:340}
\bra{\Omega} \phi(x) \phi(y) \ket{\Omega}
\sim
   \bra{0}
   T\lr{
      \phi_I(x)
      \phi_I(y)
      e^{-i \int_{-T}^T H_{\text{I,int}}(t') dt'}
   }
\ket{0}
=
\expectation{
      \phi_I(x)
      \phi_I(y)
      e^{-i \int d^4 z \phi^4(z)}
}.
\end{dmath}
To first order, this is
\begin{dmath}\label{eqn:qftLecture15b:360}
\expectation{ T \phi_x \phi_y } - i \frac{\lambda}{4} \int d^4 z \expectation{
T \phi_x \phi_y \phi_z \phi_z \phi_z \phi_z
},
\end{dmath}
The first braket has the pictorial representation sketched in \cref{fig:qftLecture15b:qftLecture15bFigF2a}.
\imageFigure{../figures/phy2403-quantum-field-theory/qftLecture15bFigF2a}{First integral diagram.}{fig:qftLecture15b:qftLecture15bFigF2a}{0.1}
whereas the second has the diagrams sketched in
\cref{fig:qftLecture15b:qftLecture15bFigF2b}.
%\imageTwoFigures{path1}{path2}{fancy plots}{fig:blah}{scale=0.3}
\imageTwoFigures
{../figures/phy2403-quantum-field-theory/qftLecture15bFigF2b}
{../figures/phy2403-quantum-field-theory/qftLecture15bFigF2c}
{Second integral diagrams.}{fig:qftLecture15b:qftLecture15bFigF2b}{scale=0.3}

We can depict the entire second integral in diagrams as sketched in \cref{fig:qftLecture15b:qftLecture15bFig3}.
\imageFigure{../figures/phy2403-quantum-field-theory/qftLecture15bFig3}{Integrals as diagrams.}{fig:qftLecture15b:qftLecture15bFig3}{0.3}

Solving for the perturbed ground state can now be thought of as reduced to drawing pictures.  Each line from \( x \rightarrow x' \) represents a propagator \( D_F( x - x' ) \), and each vertex \( -i \lambda \int d^4 z \times \text{symmetry coefficients} \).\footnote{Symmetry coefficients weren't discussed until the next lecture.  This means making combinatorial arguments to count the number of equivalent diagrams.}

We may also translate back from the diagrams to an algebraic representation.  For the first order \( \phi^4 \) interaction, that is
\begin{dmath}\label{eqn:qftLecture15b:380}
\expectation{ T \phi_x \phi_y }
- \frac{i \lambda }{4} \int d^4 z D_F(x - y) D_F^2( z - z )
+ D_F(x - z) D_F( y - z).
\end{dmath}

Other diagrams can be similarly translated.  For example
F5
represents
\begin{dmath}\label{eqn:qftLecture15b:400}
\int d^4 z D^2(z - z)
=
V_3 T \lr{ \int \frac{d^4 p}{(2\pi)^4} \inv{\Bp^2 - m^2 + i \epsilon } }^2.
\end{dmath}
Clearly, additional interpretation will be required, since this diverges.  The resolution of this unfortunately has to be deferred to QFT II, where renormalization is covered.

\section{Tree level diagrams.}
We would like to only discuss tree level diagrams, which exclude diagrams like
\cref{fig:qftLecture15b:qftLecture15bFig6}
\footnote{I think this is what is referred to as connected, amputated graphs in the next lecture.  Such diagrams are the ones of interest for scattering and decay problems.}.
\imageFigure{../figures/phy2403-quantum-field-theory/qftLecture15bFig6}{Not a tree level diagram.}{fig:qftLecture15b:qftLecture15bFig6}{0.2}

For the braket \footnote{I'd written: \( \expectation{ \int \phi_1 \phi_2 \phi_3 \phi_4 \lambda \int d^4 z \phi_z \phi_z \phi_z \phi_z } \).  Is this two fold integral what was intended, or my correction in \cref{eqn:qftLecture15b:420}?}
\begin{dmath}\label{eqn:qftLecture15b:420}
\expectation{ \int d^4 z \phi_1 \phi_2 \phi_3 \phi_4 \phi_z \phi_z \phi_z \phi_z
}
\end{dmath}
we draw diagrams like those of \cref{fig:qftLecture15b:qftLecture15bFig7},
the first of which is a tree level diagram.
\imageFigure{../figures/phy2403-quantum-field-theory/qftLecture15bFig7}{First order interaction diagrams.}{fig:qftLecture15b:qftLecture15bFig7}{0.2}

\section{Scattering.}
In QM we did lots of scattering problems as sketched in \cref{fig:qftLecture15b:qftLecture15bFig8},
and were able to compute the reflected and transmitted wave functions and quantities such as the reflection and transmission coefficients
\imageFigure{../figures/phy2403-quantum-field-theory/qftLecture15bFig8}{Reflection and transmission of wave packets.}{fig:qftLecture15b:qftLecture15bFig8}{0.3}
\begin{dmath}\label{eqn:qftLecture15b:440}
\begin{aligned}
R &= \frac
{\Abs{ \Psi_{\text{ref}}}^2}
{\Abs{ \Psi_{\text{in}}}^2} \\
T &= \frac
{\Abs{ \Psi_{\text{trans}}}^2}
{\Abs{ \Psi_{\text{in}}}^2}.
\end{aligned}
\end{dmath}
We'd like to consider scattering in some region of space with a non-zero potential, such as the scattering of a plane wave with known electron flux rate as sketched in
\cref{fig:qftLecture15b:qftLecture15bFig9}.
We can imagine that we have a detector capable of measuring the number of electrons with momentum \( \Bp_{\text{out}} \) per unit time.
\imageFigure{../figures/phy2403-quantum-field-theory/qftLecture15bFig9}{Plane wave scattering off a potential.}{fig:qftLecture15b:qftLecture15bFig9}{0.4}

\makedefinition{Total cross section (X-section).}{dfn:qftLecture15b:460}{
\begin{equation*}
\sigma_{\text{total}}
=
\frac{
\text{number of scattering events with \( \Bp_{\text{out}} \ne \Bk_{\text{in}} \) per unit time}
}
{
\text{Flux of incoming particles}
},
\end{equation*}
where the flux is the number of particles crossing a unit area in unit time.
} % definition

Units of the x-section are (with \( \hbar = c = 1 \))
\begin{equation}\label{eqn:qftLecture15b:480}
[\sigma] = \text{area} = \inv{M^2}.
\end{equation}

The concept of scattering cross section may not be new, as it can even be encountered in classical mechanics.  One such scenario is sketched in \cref{fig:qftLecture15b:qftLecture15bFig10} where the cross section is just the area
\begin{dmath}\label{eqn:qftLecture15b:500}
\sigma = \pi R^2.
\end{dmath}
\imageFigure{../figures/phy2403-quantum-field-theory/qftLecture15bFig10}{Classical scattering.}{fig:qftLecture15b:qftLecture15bFig10}{0.3}
Other classical fields where cross section is encountered includes antenna theory (radar scattering profiles, ...).

\makedefinition{Differential cross section.}{dfn:qftLecture15b:520}{
\begin{equation*}
\frac{d^3 \sigma}{dp_x dp_y dp_z} = \frac{
\text{number of scattering events with \( \Bp_{\text{out}} \) between \( (\Bp, \Bp + \Delta \Bp )\)}
}
{
\text{flux}
}.
\end{equation*}
} % definition

%}
%\EndNoBibArticle

      \section{Problems.}
         %
% Copyright © 2018 Peeter Joot.  All Rights Reserved.
% Licenced as described in the file LICENSE under the root directory of this GIT repository.
%
%{
\makeproblem{\( U(T, t_0) U(t_0, -T) \) }{problem:qftLecture15Problems:1}{
Show that
\begin{equation*}
U(T, t_0) U(t_0, -T) = U(T, -T).
\end{equation*}
} % problem

\makeanswer{problem:qftLecture15Problems:1}{
We can see that from
\begin{equation}\label{eqn:qftLecture15Problems:160}
\begin{aligned}
U(T, t_0) &= e^{i H_0(T - t_0)} e^{-i H(T - t_0)} \cancel{e^{-i H_0(t_0 - t_0)}}  \\
U(t_0, -T) &= \cancel{e^{i H_0(t_0 - t_0)}} e^{-i H(t_0 - -T)} e^{-i H_0(-T - t_0)},
\end{aligned}
\end{equation}
so
\begin{equation}\label{eqn:qftLecture15Problems:180}
\begin{aligned}
U(T, t_0)
U(t_0, -T)
&=
e^{i H_0(T - t_0)} e^{-i H(T - t_0)} e^{-i H(t_0 + T)} e^{-i H_0(-T - t_0)}
\\&=
e^{i H_0(T - t_0)} e^{-i H 2 T } e^{-i H_0(-T - t_0)},
\end{aligned}
\end{equation}
whereas
\begin{equation}\label{eqn:qftLecture15Problems:260}
\begin{aligned}
U(T, -T)
  &= e^{i H_0(T - t_0)} e^{-i H(T - -T)} e^{-i H_0(-T - t_0)}
\\&= e^{i H_0(T - t_0)} e^{-i H 2 T} e^{-i H_0(-T - t_0)}.
\end{aligned}
\end{equation}
} % answer
%}

         %
% Copyright � 2018 Peeter Joot.  All Rights Reserved.
% Licenced as described in the file LICENSE under the root directory of this GIT repository.
%
%{
%\input{../latex/blogpost.tex}
%\renewcommand{\basename}{braOmega}
%%\renewcommand{\dirname}{notes/phy1520/}
%\renewcommand{\dirname}{notes/ece1228-electromagnetic-theory/}
%%\newcommand{\dateintitle}{}
%%\newcommand{\keywords}{}
%
%\input{../latex/peeter_prologue_print2.tex}
%
%\usepackage{peeters_layout_exercise}
%\usepackage{peeters_braket}
%\usepackage{peeters_figures}
%\usepackage{siunitx}
%\usepackage{verbatim}
%%\usepackage{mhchem} % \ce{}
%%\usepackage{macros_bm} % \bcM
%%\usepackage{macros_qed} % \qedmarker
%%\usepackage{txfonts} % \ointclockwise
%
%% qftLecture14:
%% https://tex.stackexchange.com/a/68357/15
%\DeclareMathOperator*{\SumInt}{%
%\mathchoice%
%  {\ooalign{$\displaystyle\sum$\cr\hidewidth$\displaystyle\int$\hidewidth\cr}}
%  {\ooalign{\raisebox{.14\height}{\scalebox{.7}{$\textstyle\sum$}}\cr\hidewidth$\textstyle\int$\hidewidth\cr}}
%  {\ooalign{\raisebox{.2\height}{\scalebox{.6}{$\scriptstyle\sum$}}\cr$\scriptstyle\int$\cr}}
%  {\ooalign{\raisebox{.2\height}{\scalebox{.6}{$\scriptstyle\sum$}}\cr$\scriptstyle\int$\cr}}
%}
%
%\beginArtNoToc
%
%\generatetitle{PHY2403 (QFT I).  Pondering the ground state bra formula.}
%%\chapter{XXX}
%%\label{chap:braOmega}
%

\makeproblem{Pondering the ground state bra formula.}{problem:braOmega:1}{
Prove \cref{eqn:qftLecture15:80}.  What is wrong with conjugating
\cref{eqn:qftLecture15:60} to find
\begin{equation*}
\bra{\Omega}
=
\evalbar{
   \frac{ \bra{0} U(-T, t_0) }
   {
   e^{+i E_0(T - t_0)} \braket{0}{\Omega}
   }
}
{
T \rightarrow \infty( 1 - i \epsilon )
}.
\end{equation*}
} % problem

\makeanswer{problem:braOmega:1}{
While there is nothing wrong with stating
\begin{dmath}\label{eqn:braOmega:100}
\lr{
   \frac{ U(t_0, -T) \ket{0} }
   {
   e^{-i E_0(T - t_0)} \braket{\Omega}{0}
   }
}^\dagger
=
   \frac{ \bra{0} U(-T, t_0) }
   {
   e^{+i E_0(T - t_0)} \braket{0}{\Omega}
   },
\end{dmath}
the limit point \( \infty(1 - i \epsilon) \) also needs to be changed with this conjugation.  So \cref{eqn:braOmega:100} is correct, but it is only part of the story, and should really be stated as
\begin{dmath}\label{eqn:braOmega:120}
\bra{\Omega}
=
\evalbar{
   \frac{ \bra{0} U(-T, t_0) }
   {
   e^{+i E_0(T - t_0)} \braket{0}{\Omega}
   }
}{T \rightarrow \infty(1 + i \epsilon)}.
\end{dmath}
This is awkward because now our expressions for \( \bra{\Omega} \) and \( \ket{\Omega} \) approach \( T \) from different directions, and we want to evaluate both with a single limiting argument.

To resolve this, we really have to start back with the identity expansion we used in lecture 14, and write
\begin{dmath}\label{eqn:braOmega:140}
\bra{0} e^{-i H T}
=
\lr{
\braket{0}{\Omega}\bra{\Omega}
 + \SumInt_n \braket{0}{n} \bra{n}
}
e^{-i H T}
=
\braket{0}{\Omega}\bra{\Omega}
e^{-i E_0 T}
 + \SumInt_n \braket{0}{n} \bra{n} e^{-i E_n T}.
\end{dmath}
We argued (as does the text) that approaching to as \( T( 1 - i \epsilon) \) kills off the energetic states since
\begin{dmath}\label{eqn:braOmega:160}
\bra{n} e^{-i E_n T}
\rightarrow
\bra{n} e^{-i E_n T} e^{-E_n T \epsilon}
\end{dmath}
and the exponential damping factor is smaller for each \( E_n > E_0 \), so it can be neglected in the large \( T \) limit, leaving
\begin{dmath}\label{eqn:braOmega:180}
\bra{0} e^{-i H T}
=
\lim_{T \rightarrow \infty(1 - i \epsilon)}
\braket{0}{\Omega}\bra{\Omega}.
\end{dmath}
As we did for \( \ket{\Omega} \) we can shift the large time \( T \) by a small constant (this time \( -t_0 \) instead of \( t_0 \)), to give
\begin{dmath}\label{eqn:braOmega:200}
\bra{\Omega}
=
\lim_{T \rightarrow \infty(1 - i \epsilon)}
\frac{ \bra{0} e^{-i H T} }
{
\braket{0}{\Omega} e^{-i E_0 T}
}
\approx
\lim_{T \rightarrow \infty(1 - i \epsilon)}
\frac{ \bra{0} e^{-i H (T - t_0)} }
{
\braket{0}{\Omega} e^{-i E_0 (T - t_0)}
}
=
\lim_{T \rightarrow \infty(1 - i \epsilon)}
\frac{ \bra{0} e^{i H_0( T - t_0)} e^{-i H (T - t_0)} }
{
\braket{0}{\Omega} e^{-i E_0 (T - t_0)}
}
=
\lim_{T \rightarrow \infty(1 - i \epsilon)}
\frac{ \bra{0} U(T, t_0) }
{
\braket{0}{\Omega} e^{-i E_0 (T - t_0)}
},
\end{dmath}
where the projective property \( \bra{0} e^{i H_0 \alpha} = \bra{0} \) has been used to insert a no-op (i.e. \( \bra{0} H_0 = 0 \)).  This recovers the result stated in class (also:
\citep{peskin1995introduction}
eq. (4.29).)
} % answer

%}
%\EndArticle
%\EndNoBibArticle

   \chapter{Differential cross section, scattering, pair production, transition amplitude, decay rate, S-matrix, connected and amputated diagrams, vacuum fluctuation, symmetry coefficient.}
      %
% Copyright � 2017 Peeter Joot.  All Rights Reserved.
% Licenced as described in the file LICENSE under the root directory of this GIT repository.
%
%{
\input{../latex/blogpost.tex}
\renewcommand{\basename}{qftLecture16}
\renewcommand{\dirname}{notes/phy2403/}
\newcommand{\keywords}{PHY2403H}
\input{../latex/peeter_prologue_print2.tex}

%\usepackage{phy2403}
\usepackage{peeters_braket}
\usepackage{peeters_layout_exercise}
\usepackage{peeters_figures}
\usepackage{mathtools}
\usepackage{siunitx}
\usepackage{macros_cal} % LL
\usepackage{simplewick}
\usepackage{verbatim}

\newcommand{\ultensor}[3]{{{#1}^{#2}}_{#3}}

\beginArtNoToc
\generatetitle{PHY2403H Quantum Field Theory.  Lecture 16: XXX.  Taught by Prof.\ Erich Poppitz}
%\chapter{XXX}
\label{chap:qftLecture16}

%%Peeter's lecture notes from class.  These may be incoherent and rough.
%%
%%These are notes for the UofT course PHY2403H, Quantum Field Theory, taught by Prof. Erich Poppitz, covering \textchapref{{1}} \citep{peskin1995introduction} content.

\paragraph{DISCLAIMER: Very rough notes from class, with some additional side notes.}

These are notes for the UofT course PHY2403H, Quantum Field Theory, taught by Prof. Erich Poppitz, fall 2018.
%, covering \textchapref{{1}} \citep{peskin1995introduction} content.

\section{Review}

We finished by defining the differential cross section

\makedefinition{Differential cross section.}{dfn:qftLecture16:20}{
\begin{equation*}
\frac{d^3 \sigma}{dp_x dp_y dp_z} = \frac{
\text{number of scattering events with \( \Bp_{\txtf} \) between \( (\Bp_\txtf, \Bp_\txtf + d \Bp_\txtf )\)}
}
{
\text{flux of incoming particles}
}.
\end{equation*}
} % definition

\section{Scattering}

In QFT we typically study \( 2 \rightarrow n \) inelastic scattering.  Most commonly the nature of the final state particles are different from the nature of the incoming state.

For example, we can collide two electrons, and can get muon and anti-muon particles
F1
or pions
F2, or even both
F3

In the \( \lambda \phi^4 \) theory we can have scattering events such as
F4a
F4b

How to calculate in QFT.  Initial state of 2 particles \( A, B \) with initital state
\begin{dmath}\label{eqn:qftLecture16:40}
\ket{\Bk_A, \Bk_B }_{\text{in}, T \rightarrow -\infty}
\end{dmath}
and final n-particle state
\begin{dmath}\label{eqn:qftLecture16:60}
\ket{\Bp_1, \Bp_2, \cdots, \Bp_n }_{\text{out}, T \rightarrow +\infty}
\end{dmath}
The QM transition amplitude from the initial to the final state is
\begin{dmath}\label{eqn:qftLecture16:80}
\prescript{}{\text{out}}{\bra{\Bp_1, \Bp_2, \cdots, \Bp_n }}
\ket{\Bk_A, \Bk_B }_{\text{in}}
=
\bra{\Bp_1, \Bp_2, \cdots, \Bp_n } e^{-2 i H T}
\ket{\Bk_A, \Bk_B }.
\end{dmath}
This is the amplitude for \( A B \rightarrow 1 \cdots n \).
Ultimately, we want the scattering x-section.

We will also be interested in decay rates, as there are unstable particles in QFT that can decay.  This doesn't happen in \( \lambda \phi^4 \) theory.
In a theory with 2 scalar fields \( \Phi, \varphi \) with \( m_\Phi > 2 m_\varphi \).  A possible interaction for such a theory is
\begin{dmath}\label{eqn:qftLecture16:100}
H_{\text{int}} = \mu \Phi \varphi^2,
\end{dmath}
which would permit \( \Phi \rightarrow \varphi \varphi \) decays.
HW4 has a coupling like \( (h/V) \partial_\mu \phi^a \partial^\mu \phi^a \) for which a \( h \rightarrow \phi^a \phi^a \) decay is possible.

\makedefinition{Decay rate.}{dfn:qftLecture16:120}{
The decay rate is defined as
\begin{equation*}
\Gamma =
\frac{
\text{
Number of decays \( \Phi \rightarrow \varphi \varphi \) in unit time
}
}
{
\text{
Number of \( \Phi \) particles present
}
}
\end{equation*}
} % definition

What is the amplitude for such a decay transition?
\begin{dmath}\label{eqn:qftLecture16:140}
\bra{\Bk_\phi}_{\text{in}, T \rightarrow -\infty} \rightarrow
\bra{\Bk_1, \Bk_2}_{\text{out}, T \rightarrow +\infty}.
\end{dmath}
The amplitude for \( \Bk_\phi \rightarrow \Bk_1, \Bk_2 \).
\begin{dmath}\label{eqn:qftLecture16:160}
\bra{\Bk_1, \Bk_2} e^{-i 2 H T } \ket{\Bk_\phi}
=
\prescript{}{\text{out}}{\braket{\Bk_1, \Bk_2}{\Bk_\phi}}
\end{dmath}

\paragraph{mysterious seeming statement something like}: ``The decays are essentially due to interactions with vacuum fluctuations.''

\section{Calculating interactions}

We write
\begin{dmath}\label{eqn:qftLecture16:180}
\begin{aligned}
\prescript{}{\text{out}}{\braket{ \Bp_1, \cdots \Bp_n }{ \Bk_A, \Bk_B }}_{\text{in}}
&=
\lim_{T \rightarrow \infty}
\bra{ \Bp_1, \cdots \Bp_n } e^{-i 2 H T } \ket{ \Bk_A, \Bk_B } \\
&=
\bra{ \Bp_1, \cdots \Bp_n } \hatS \ket{ \Bk_A, \Bk_B } \\
&=
\bra{ \Bp_1, \cdots \Bp_n } \BOne + i \hatT \ket{ \Bk_A, \Bk_B },
\end{aligned}
\end{dmath}
where \( \hatS \) is called the S-matrix or scattering matrix, which is decomposed into a unit portion \( \BOne \) which is a convient way to exclude events with no scattering.  \( \BOne \) contributes for \( n = 2 \) only, but is an \( n \) scattering amplitude.
We are really interested in the \( i \hatT \) portion of this amplitude
\begin{equation}\label{eqn:qftLecture16:200}
\bra{ \Bp_1, \cdots \Bp_n } i \hatT \ket{ \Bk_A, \Bk_B }
=
(2 \pi)^4 \delta^4( \Bk_A + \Bk_B - \sum_{i = 1}^n \Bp_i )
\times
i M( \Bk_A + \Bk_B \rightarrow \Bp_1 \cdots \Bp_n ).
\end{equation}
This amounts to a definition of \( M \).
Recall that we found
\begin{dmath}\label{eqn:qftLecture16:220}
U(T, -T)
= T \lr{ e^{-i \int_{-T}^T H_I(t') dt'} }
=
e^{i H_0(T - t_0)}
e^{-i 2 H T}
e^{-i H_0(-T - t_0)}.
\end{dmath}
We want to replace the \( e^{-i 2 H T} \) in the matrix element above with \( U \).

In perturbation theory, we assume (conjecture) that
\begin{dmath}\label{eqn:qftLecture16:240}
\ket{ \Bk_A, \Bk_B }
\sim
\ket{ \Bk_A, \Bk_B }_\txto
\sim
\text{const}\, a^\dagger_{\Bk_A} a^\dagger_{\Bk_B} \ket{0}
\end{dmath}

Because we'll be squaring the amplitudes, we can assume that the \( e^{i H_0(T-t_0)} \) will result in just phase factors that won't survive, so in \cref{eqn:qftLecture16:180} we can insert \( U \)
\begin{dmath}\label{eqn:qftLecture16:280}
\prescript{}{\text{out}}{\braket{ \Bp_1, \cdots \Bp_n }{ \Bk_A, \Bk_B }}_{\text{in}}
=
\lim_{T \rightarrow \infty}
\bra{ \Bp_1, \cdots \Bp_n } U(T, -T) \ket{ \Bk_A, \Bk_B }
\end{dmath}

\begin{dmath}\label{eqn:qftLecture16:300}
\bra{ \Bp_1, \cdots \Bp_n } i\hatT { \Bk_A, \Bk_B }
=
\lim_{T \rightarrow \infty(1 - i \epsilon) }
\prescript{}{0}{
\bra{ \Bp_1, \cdots \Bp_n }
T( e^{-i \int_{-T}^T H_i(t') dt' } )
 \ket{ \Bk_A, \Bk_B }
}_0
\end{dmath}

These are connected and amputated graphs.

\paragraph{What is ``connected and amputated''?}

Explaining by example.  \( n = 2, \lambda \phi^4/4! \).

\begin{dmath}\label{eqn:qftLecture16:320}
\bra{0}
a_{\Bp_1}
a_{\Bp_2}
\lr{
\cancel{1}
- \frac{i \lambda}{4!} \int d^4 x \phi_I^4(x)
+ \inv{2} \lr{ \frac{i \lambda}{4!}}^2 \int d^4 x d^4 y \phi_I^4(x) \phi_I^4(y)
+ \cdots
}
a_{\Bk_A}^\dagger
a_{\Bk_B}^\dagger
\ket{0}
\end{dmath}
Here time ordering operations are implied, but not written explicitly.
Also, the ``amputated'' indicates that we are going to be dropping the \( 1 \) portion of the exponential expansion (as we've also dropped that in \cref{eqn:qftLecture16:300}).
We will also be using a relativistic normalization so that the \(
a_{\Bk_A}^\dagger
a_{\Bk_B}^\dagger \) terms include
\( \sqrt{
2 \omega_{\Bk_A}
2 \omega_{\Bk_B} } \)
contributions and the \(
a_{\Bp_1}
a_{\Bp_2}
\) include
\( \sqrt{
2 \omega_{\Bp_1}
2 \omega_{\Bp_2} } \) contributions.

\begin{dmath}\label{eqn:qftLecture16:340}
T
\contraction{}{\phi}{{}_I(x_1)}{\phi}
\phi_I(x_1)\phi_I(x_2)
= D_F(x_1 - x_2)
\end{dmath}

When we look at
\begin{dmath}\label{eqn:qftLecture16:360}
\contraction{}{\phi}{{}_I(x_1)}{a}
\phi_I(x_1) a^\dagger_{\Bk}
\sqrt{ 2 \omega_\Bk}
=
\int \frac{d^3 p}{(2 \pi)^3} \frac{e^{-i p \cdot x}}{\sqrt{2 \omega_\Bp}}
\contraction{}{a}{{}_\Bp}{a}
a_\Bp a^\dagger_\Bk
\sqrt{ 2 \omega_\Bk}
=
\int \frac{d^3 p}{(2 \pi)^3} \frac{e^{-i p \cdot x}}{\sqrt{2 \omega_\Bp}}
\delta^3(\Bp - \Bk)
\sqrt{ 2 \omega_\Bk}
= e^{-i k \cdot x}.
\end{dmath}
Similarily
\begin{dmath}\label{eqn:qftLecture16:n}
\contraction{}{a}{{}_\Bp}{\phi}
a_\Bp \phi_I(x_1)
\sqrt{ 2 \omega_\Bp}
=
\int \frac{d^3 k}{(2 \pi)^3} \frac{e^{i k \cdot x}}{\sqrt{2 \omega_\Bk}}
\contraction{}{a}{{}_\Bp}{a}
a_\Bp a^\dagger_\Bk
\sqrt{ 2 \omega_\Bk}
=
\int \frac{d^3 k}{(2 \pi)^3} \frac{e^{i k \cdot x}}{\sqrt{2 \omega_\Bk}}
\delta^3(\Bp - \Bk)
\sqrt{ 2 \omega_\Bk}
= e^{+i p \cdot x}.
\end{dmath}

Summarizing
\begin{dmath}\label{eqn:qftLecture16:380}
\begin{aligned}
\contraction{}{\phi}{{}_I(x_1)}{a}
\phi_I(x_1) a^\dagger_{\Bp}
&= e^{-i p \cdot x} \\
\contraction{}{a}{{}_\Bp}{\phi}
a_\Bp \phi_I(x_1)
&= e^{i p \cdot x}.
\end{aligned}
\end{dmath}

\begin{comment}
\section{junk}
\begin{dmath}\label{eqn:qftLecture15b:260}
%\sideset{_a^b}{'}{x}
\prescript{14}{2}{\mathbf{C}}
\end{dmath}
\begin{dmath}\label{eqn:qftLecture16:420}
\prescript{}{2}{\mathbf{C}}
\end{dmath}
\end{comment}

%}
%\EndArticle
\EndNoBibArticle

   \chapter{Scattering, decay, cross sections in a scalar theory.}
      %
% Copyright � 2017 Peeter Joot.  All Rights Reserved.
% Licenced as described in the file LICENSE under the root directory of this GIT repository.
%
%{
%%\input{../latex/blogpost.tex}
%%\renewcommand{\basename}{qftLecture17}
%%\renewcommand{\dirname}{notes/phy2403/}
%%\newcommand{\keywords}{PHY2403H}
%%\input{../latex/peeter_prologue_print2.tex}
%%
%%%\usepackage{phy2403}
%%\usepackage{peeters_braket}
%%%\usepackage{peeters_layout_exercise}
%%\usepackage{peeters_figures}
%%\usepackage{mathtools}
%%\usepackage{siunitx}
%%\usepackage{macros_cal} % LL
%%
%%\newcommand{\ultensor}[3]{{{#1}^{#2}}_{#3}}
%%\newcommand{\deltathree}[0]{\delta^{(3)}}
%%\newcommand{\deltafour}[0]{\delta^{(4)}}
%%
%%\beginArtNoToc
%%\generatetitle{PHY2403H Quantum Field Theory.  Lecture 17: Scattering, decay, cross sections in a scalar theory.  Taught by Prof.\ Erich Poppitz}
%\chapter{Scattering, decay, cross sections in a scalar theory.}
\label{chap:qftLecture17}
%%
%%%%Peeter's lecture notes from class.  These may be incoherent and rough.
%%%%
%%%%These are notes for the UofT course PHY2403H, Quantum Field Theory, taught by Prof. Erich Poppitz, covering \textchapref{{1}} \citep{peskin1995introduction} content.
%%
%%\paragraph{DISCLAIMER: Very rough notes from class, with some additional side notes.}
%%
%%These are notes for the UofT course PHY2403H, Quantum Field Theory, taught by Prof. Erich Poppitz, fall 2018.
%%%, covering \textchapref{{1}} \citep{peskin1995introduction} content.

\section{Review: S-matrix}

We defined an \( S-\)matrix
\begin{equation}\label{eqn:qftLecture17:20}
\bra{f} S \ket{i} = S_{fi} = \lr{ 2 \pi }^4 \deltafour \lr{ \sum \lr{p_i - \sum_{p_f} } } i M_{fi},
\end{equation}
where
\begin{equation}\label{eqn:qftLecture17:40}
i M_{fi} = \sum \text{ all connected amputated Feynman diagrams }.
\end{equation}
The matrix element \( \bra{f} S \ket{i} \) is the amplitude of the transition from the initial to the final state.
In general this can get very complicated, as the number of terms grows factorially with the order.

We also talked about decays.
\section{Scattering in a scalar theory}
Suppose that we have a scalar theory with a light field \( \Phi, M \) and a heavy field \( \varphi, m \), where \( m > 2 M \).  Perhaps we have an interaction with a \( z^2 \) symmetry so that the interaction potential is quadratic in \( \Phi \)
\begin{dmath}\label{eqn:qftLecture17:60}
V_{\text{int}} = \mu \varphi \Phi \Phi.
\end{dmath}
We may have \( \Phi \Phi \rightarrow \Phi \Phi \) scattering.

We will denote diagrams using a double line for \( \phi \) and a single line for \( \Phi \), as sketched in
\cref{fig:qftLecture17:qftLecture17Fig1}.
\imageFigure{../figures/phy2403-quantum-field-theory/qftLecture17Fig1}{Particle line convention.}{fig:qftLecture17:qftLecture17Fig1}{0.2}

There are three possible diagrams:
\imageThreeFiguresOneLine{../figures/phy2403-quantum-field-theory/qftLecture17Fig2a}{../figures/phy2403-quantum-field-theory/qftLecture17Fig2b}{../figures/phy2403-quantum-field-theory/qftLecture17Fig2c}{Possible diagrams.}{fig:qftLecture17:qftLecture17Fig2}{scale=0.3}
%\cref{fig:qftLecture17:qftLecture17Fig2a}.
%\imageFigure{../figures/phy2403-quantum-field-theory/qftLecture17Fig2a}{CAPTION: qftLecture17Fig2a}{fig:qftLecture17:qftLecture17Fig2a}{0.3}
%\cref{fig:qftLecture17:qftLecture17Fig2b}.
%\imageFigure{../figures/phy2403-quantum-field-theory/qftLecture17Fig2b}{CAPTION: qftLecture17Fig2b}{fig:qftLecture17:qftLecture17Fig2b}{0.3}
%\cref{fig:qftLecture17:qftLecture17Fig2c}.
%\imageFigure{../figures/phy2403-quantum-field-theory/qftLecture17Fig2c}{CAPTION: qftLecture17Fig2c}{fig:qftLecture17:qftLecture17Fig2c}{0.3}

The first we will call the s-channel, which has amplitude

\begin{dmath}\label{eqn:qftLecture17:80}
A(\text{s-channel}) \sim \frac{i}{p^2 - m^2 + i \epsilon} =
\frac{i}{s - m^2 + i \epsilon}
\end{dmath}

\begin{dmath}\label{eqn:qftLecture17:100}
(p_1 + p_2)^2 = s
\end{dmath}
In the centre of mass frame
\begin{dmath}\label{eqn:qftLecture17:120}
\Bp_1 = - \Bp_2,
\end{dmath}
so
\begin{dmath}\label{eqn:qftLecture17:140}
s = \lr{ p_1^0 + p_2^0 }^2 = E_{\text{cm}}^2.
\end{dmath}

To the next order we have a diagram like
\cref{fig:qftLecture17:qftLecture17Fig3}.
\imageFigure{../figures/phy2403-quantum-field-theory/qftLecture17Fig3}{Higher order.}{fig:qftLecture17:qftLecture17Fig3}{0.2}
and can have additional virtual particles created, with diagrams like \cref{fig:qftLecture17:qftLecture17Fig4}.
\imageFigure{../figures/phy2403-quantum-field-theory/qftLecture17Fig4}{More virtual particles.}{fig:qftLecture17:qftLecture17Fig4}{0.1}

We will see (QFT II) that this leads to an addition imaginary \( i \Gamma \) term in the propagator
\begin{dmath}\label{eqn:qftLecture17:160}
\frac{i}{s - m^2 + i \epsilon}
\rightarrow
\frac{i}{s - m^2 - i m \Gamma + i \epsilon}.
\end{dmath}
If we choose to zoom into the such a figure, as sketched in
\cref{fig:qftLecture17:qftLecture17Fig5},
we find that it contains the interaction of interest for our diagram, so we can
(looking forward to currently unknown material) know that our diagram also has such an imaginary \( i \Gamma \) term in its
propagator.
\imageFigure{../figures/phy2403-quantum-field-theory/qftLecture17Fig5}{Zooming into the diagram for a higher order virtual particle creation event.}{fig:qftLecture17:qftLecture17Fig5}{0.3}

Assuming such a term, the squared amplitude becomes
\begin{dmath}\label{eqn:qftLecture17:180}
\evalbar{\sigma}{s \text{near} m^2}
\sim
\Abs{A_s}^2 \sim \inv{(s - m)^2 + m^2 \Gamma^2}
\end{dmath}

This is called a resonance (name?), and is sketched in
\cref{fig:qftLecture17:qftLecture17Fig6}.
\imageFigure{../figures/phy2403-quantum-field-theory/qftLecture17Fig6}{Resonance.}{fig:qftLecture17:qftLecture17Fig6}{0.3}

Where are the poles of the modified propagator?

\begin{dmath}\label{eqn:qftLecture17:220}
\frac{i}{s - m^2 - i m \Gamma + i \epsilon}
=
\frac{i}{p_0^2 - \Bp^2 - m^2 - i m \Gamma + i \epsilon}
\end{dmath}

The pole is found, neglecting \( i \epsilon \), is found at
\begin{dmath}\label{eqn:qftLecture17:200}
p_0 = \sqrt{ \omega_\Bp^2 + i m \Gamma }
= \omega_\Bp \sqrt{ 1 + \frac{i m \Gamma }{\omega_\Bp^2} }
\approx \omega_\Bp + \frac{i m \Gamma }{2 \omega_\Bp}
\end{dmath}

\section{Decay rates.}

We have an initial state
\begin{dmath}\label{eqn:qftLecture17:240}
\ket{i} = \ket{k},
\end{dmath}
and final state
\begin{dmath}\label{eqn:qftLecture17:260}
\ket{f} = \ket{p_1^f, p_2^f \cdots p_n^f}.
\end{dmath}
We defined decay rate as the ratio of the number of initial particles to the number of final particles.

The probability is
\begin{dmath}\label{eqn:qftLecture17:280}
\rho \sim \Abs{\bra{f} S \ket{i}}^2
=
(2 \pi)^4 \deltafour( p_{\text{in}} - \sum p_f )
(2 \pi)^4 \deltafour( p_{\text{in}} - \sum p_f )
\times \Abs{ M_{fi} }^2
\end{dmath}

Saying that \( \delta(x) f(x) = \delta(x) f(0) \) we can set the argument of one of the delta functions to zero, which gives us a vacuum volume element factor
\begin{dmath}\label{eqn:qftLecture17:300}
(2 \pi)^4
\deltafour( p_{\text{in}} - \sum p_f )  =
(2 \pi)^4
\deltafour( 0 )
= V_3 T,
\end{dmath}
so
\begin{dmath}\label{eqn:qftLecture17:320}
\frac{\text{probability for \( i \rightarrow f\)}}{\text{unit time}}
\sim
(2 \pi)^4 \deltafour( p_{\text{in}} - \sum p_f )
V_3
\times \Abs{ M_{fi} }^2
\end{dmath}

\begin{dmath}\label{eqn:qftLecture17:340}
\braket{\Bk}{\Bk} = 2 \omega_\Bk V_3
\end{dmath}

coming from

\begin{dmath}\label{eqn:qftLecture17:360}
\braket{k}{p} = (2 \pi)^3 2 \omega_\Bp \deltathree(\Bp - \Bk)
\end{dmath}
so
\begin{dmath}\label{eqn:qftLecture17:380}
\braket{k}{k} = 2 \omega_\Bp V_3
\end{dmath}

\begin{dmath}\label{eqn:qftLecture17:400}
\frac{\text{probability for \(i \rightarrow f\)}}{\text{unit time}}
\sim
\frac{
(2 \pi)^4 \deltafour( p_{\text{in}} - \sum p_f )
\Abs{ M_{fi} }^2 V_3
}
{
2 \omega_\Bk V_3
2 \omega_{\Bp_1}
\cdots
2 \omega_{\Bp_n} V_3^n
}
\end{dmath}

If we multiply the number of final states with \( p_i^f \in (p_i^f, p_i^f + dp_i^f) \) for a particle in a box
\begin{dmath}\label{eqn:qftLecture17:420}
p_x = \frac{ 2 \pi n_x}{L}
\end{dmath}

\begin{dmath}\label{eqn:qftLecture17:440}
\Delta p_x = \frac{ 2 \pi }{L} \Delta n_x
\end{dmath}

\begin{dmath}\label{eqn:qftLecture17:460}
\Delta n_x
=
\frac{L}{2 \pi} \Delta p_x
\end{dmath}

and

\begin{dmath}\label{eqn:qftLecture17:480}
\Delta n_x
\Delta n_y
\Delta n_z
= \frac{V_3}{(2 \pi)^3 }
\Delta p_x
\Delta p_y
\Delta p_z
\end{dmath}

\begin{dmath}\label{eqn:qftLecture17:500}
\Gamma
=
\frac{\text{number of events \( i \rightarrow f \)}}{\text{unit time}}
=
\prod_{f} \frac{ d^3 p}{(2 \pi)^3 2 \omega_{\Bp^f} }
 \frac{ (2 \pi)^4 \deltafour( k - \sum_f p^f ) \Abs{M_{fi}}^2 }
{
2 \omega_{\Bk}
}
\end{dmath}

Note that everything here is Lorentz invariant except for the denominator of the second term ( \(2 \omega_{\Bk}\)).  This is a well known result (the decay rate changes in different frames).

\section{Cross section.}

For \( 2 \rightarrow \text{many} \) transitions

\begin{dmath}\label{eqn:qftLecture17:520}
\frac{\text{probability \( i \rightarrow f \)}}{\text{unit time}}
\times \lr{
\text{ number of final states with \( p_f \in (p_f, p_f + dp_f) \)
}
}
=
 \frac{ (2 \pi)^4 \deltafour( \sum p_i - \sum_f p^f ) \Abs{M_{fi}}^2 \cancel{V_3} }
{
2 \omega_{\Bk_1} V_3
2 \omega_{\Bk_2} \cancel{V_3 }
}
\prod_{f} \frac{ d^3 p}{(2 \pi)^3 2 \omega_{\Bp^f} }
\end{dmath}

We need to divide by the flux

In the CM frame, as sketched in
\cref{fig:qftLecture17:qftLecture17Fig7},
the current is
\begin{dmath}\label{eqn:qftLecture17:540}
\Bj = n \Bv_1 - n \Bv_2,
\end{dmath}
so if the density is
\begin{dmath}\label{eqn:qftLecture17:560}
n = \inv{V_3},
\end{dmath}
(one particle in \(V_3\)), then
\begin{dmath}\label{eqn:qftLecture17:580}
\Bj = \frac{\Bv_1 - \Bv_2}{V_3}.
\end{dmath}
\imageFigure{../figures/phy2403-quantum-field-theory/qftLecture17Fig7}{Centre of mass frame.}{fig:qftLecture17:qftLecture17Fig7}{0.1}

This is where \citep{peskin1995introduction} stop,
\begin{dmath}\label{eqn:qftLecture17:640}
\sigma
=
 \frac{ (2 \pi)^4 \deltafour( \sum p_i - \sum_f p^f ) \Abs{M_{fi}}^2 \cancel{V_3} }
{
2 \omega_{\Bk_1}
2 \omega_{\Bk_2}
\Abs{\Bv_1 - \Bv_2}
}
\prod_{f} \frac{ d^3 p}{(2 \pi)^3 2 \omega_{\Bp^f} }
\end{dmath}

There is, however, a nice Lorentz invariant generalization
\begin{dmath}\label{eqn:qftLecture17:600}
j = \inv{ V_3 \omega_{k_A} \omega_{k_B}} \sqrt{ (k_A - k_B)^2 - m_A^2 m_B^2 }
\end{dmath}

(Claim: DIY)
\begin{dmath}\label{eqn:qftLecture17:620}
\evalbar{j}{CM} =
\inv{V_3}
\lr{
   \frac{\Abs{\Bk}}{\omega_{k_A}}
   +
   \frac{\Abs{\Bk}}{\omega_{k_B}}
}
=
\inv{V_3} \lr{ \Abs{\Bv_A} + \Abs{\Bv_B} }
=
\inv{V_3} \Abs{\Bv_1 - \Bv_2 }
\end{dmath}

\begin{dmath}\label{eqn:qftLecture17:660}
\sigma
=
 \frac{ (2 \pi)^4 \deltafour( \sum p_i - \sum_f p^f ) \Abs{M_{fi}}^2 \cancel{V_3} }
{
4 \sqrt{ (k_A - k_B)^2 - m_A^2 m_B^2 }
}
\prod_{f} \frac{ d^3 p}{(2 \pi)^3 2 \omega_{\Bp^f} }.
\end{dmath}

%}
%\EndArticle


   \chapter{Problem Set 1.}

      %
% Copyright � 2018 Peeter Joot.  All Rights Reserved.
% Licenced as described in the file LICENSE under the root directory of this GIT repository.
%
\makeoproblem{
%Back to classics: relativistic electrodynamics and variational principle.
Electrodynamics, variational principle.
}{qft:problemSet1:1}{2018 Hw1.I}{
Given the action
In terms of the four-vector potential \(A\), the Lagrangian density of the electromagnetic field,
interacting with a charged particle of mass m can be written as follows:
\index{Lagrangian density!electromagnetic}
\begin{equation}\label{eqn:ProblemSet1Problem1:20}
S =
\int_{\text{all spacetime}}
d^4 x
\lr{
-\inv{4}
F_{\mu\nu}
F^{\mu\nu}
- A_\mu j^\mu
}
- m
\int_{\text{worldline}}
ds.
\end{equation}
\index{field strength tensor}
Here, \( F_{\mu\nu} \equiv \partial_\mu A_\nu - \partial_\nu A_\mu \) field strength tensor. The current \( j^\mu \) is the current corresponding to the particle which can be written as:
\begin{equation}\label{eqn:ProblemSet1Problem1:40}
j^\mu(x) = e
\int_{\text{worldline}}
dX^\mu (\tau) \deltafour ( x - X(\tau) ),
\end{equation}
where \( \deltafour(x) \) is a four-dimensional delta function.  All indices are raised and lowered by means of the
metric tensor \( g_{\mu\nu} \) and its inverse \( g^{\mu\nu} \).

The last term in \cref{eqn:ProblemSet1Problem1:20} is the relativistic kinetic energy of the particle and the integral is over the particle's worldline, $X^\mu(\tau)$. Note that $\tau$ is a parameter used to describe the particle's location along the worldline. One can  take this parameter be equal to $x^0$, so that  $X^\mu(\tau)$ means ($X^0 = x^0$, $X^i = X^i(x^0)$), where $\BX(x^0)$ is simply the trajectory of the particle (such a choice of parametrization can be useful, but is not required).
Notice also that the term involving the current in \cref{eqn:ProblemSet1Problem1:20}, after substitution of \cref{eqn:ProblemSet1Problem1:40} simply becomes
\begin{equation}\label{eqn:ProblemSet1Problem1:860}
- e \int\limits_{worldline} d X^\mu(\tau) A_\mu(X(\tau))~  ,
\end{equation}
which is the usual   coupling of a charged particle to the electromagnetic field (choose the $\tau = x^0$ parameterization of the worldline to see this). Whether you use this form of the one of \cref{eqn:ProblemSet1Problem1:20} depends on the problem you're solving (this is a hint).

The dynamical degrees of freedom in the action \cref{eqn:ProblemSet1Problem1:20} are the four-vector potential $A_\mu$ and the particle position $X^\mu(\tau)$.
\makesubproblem{}{qft:problemSet1:1a}
Use the identification \( A^0 = \phi \),
the scalar potential, and \((A^1,A^2,A^3) = \BA\), the vector potential,
to convince yourself that
\( F_{01} = E_x, F_{02} = E_y, F_{03} = E_z\), and that \(F_{12} = - B_z, F_{31} = -B_y, F_{23} = - B_x \).
\makesubproblem{}{qft:problemSet1:1b}
Prove the identity
\begin{equation}\label{eqn:ProblemSet1Problem1:120}
\epsilon^{\mu\nu\alpha\beta} \partial_\nu F_{\alpha \beta} = 0,
\end{equation}
and use this to show that
the source free Maxwell's equations can be recovered directly from the
definition of \( F_{ij} \).
\makesubproblem{}{qft:problemSet1:1c}
Write the Euler-Lagrange equations obtained when varying
\cref{eqn:ProblemSet1Problem1:20}
with respect to \( A_\mu \).  Show
that they can be cast in terms of the field strength tensor \( F \) and \( j \). Note that when varying
with respect to \( A_\mu \), the current is kept fixed. Using the \( \BE \) and \( \BB \) fields as the appropriate
components of \( F \), show that the Euler-Lagrange equations for \( A_\mu \)
 from
\cref{eqn:ProblemSet1Problem1:20}
reduce to the
Maxwell equations familiar to you from electrodynamics.
\makesubproblem{}{qft:problemSet1:1d}
Finally, write the Euler-Lagrange equation varying with respect to the worldline of the particle.
Show that they give \( m dU^\mu/ds = e F^{\mu\nu} U_\nu \), where \( U^\mu = dX^\mu/ds \) is the four velocity of the
particle and \( F \) is, of course, taken at the particle's position. Convince yourself that this is
the relativistic Lorentz force equation.
\index{Lorentz force equation}
\index{Maxwell's equations}

\paragraph{The} point of this problem is to make sure you remember/learn how the action principle works in  electrodynamics. The two coupled equations, obtained by varying w.r.t. $A_\mu$ and $X^\mu$ complete the equations of classical electrodynamics. Feel free to use \citep{landau1980classical}, or \citep{poppitzphy450} while solving this problem.
} % makeproblem

\makeanswer{qft:problemSet1:1}{
\withproblemsetsParagraph{
\makeSubAnswer{}{qft:problemSet1:1a}
With \( k = \setlr{1, 2,3} \),
\begin{equation}\label{eqn:ProblemSet1Problem1:60}
\begin{aligned}
\sum_k F_{0k} \Be_k
&= \sum_k \lr{ \partial_0 A_k - \partial_k A_0 } \Be_k
\\&= - \sum_k \lr{ \PD{t}{A^k} - \PD{x^k}{\phi} } \Be_k
\\&= - \PD{t}{\BA} - \spacegrad \phi
\\&= \BE.
\end{aligned}
\end{equation}
which is the conventional scalar, plus vector potential definition of the electric field in natural units.
For the magnetic field, it's easier to work backwards
\begin{equation}\label{eqn:ProblemSet1Problem1:80}
\begin{aligned}
\BB
&= \spacegrad \cross \BA
\\&= \epsilon_{ijk} \partial_i A^j \Be_k,
\end{aligned}
\end{equation}
or, for each cyclic permutation of \( i j k = \setlr{1,2,3}\)
\begin{equation}\label{eqn:ProblemSet1Problem1:100}
\begin{aligned}
B^i
&= \partial_j A^k - \partial_k A^j
\\&= -\partial_j A_k + \partial_k A_j
\\&= F_{kj}
\\&= -F_{jk},
\end{aligned}
\end{equation}
%
\makeSubAnswer{}{qft:problemSet1:1b}
To prove \cref{eqn:ProblemSet1Problem1:120}, we use explicit expansion and an index exchange
\begin{equation}\label{eqn:ProblemSet1Problem1:140}
\begin{aligned}
&=
\epsilon^{\mu\nu\alpha\beta} \partial_\nu \lr{ \partial_\alpha A_\beta - \partial_\beta A_\alpha}
\\&=
\epsilon^{\mu\nu\alpha\beta} \partial_\nu \partial_\alpha A_\beta
-\epsilon^{\mu\nu\beta\alpha} \partial_\nu \partial_\beta A_\alpha
\\&=
2 \epsilon^{\mu\nu\alpha\beta} \partial_\nu \partial_\alpha A_\beta,
\end{aligned}
\end{equation}
but because the partials are symmetric in \( \nu \alpha \) (assuming sufficient continuity of the fields components), and because the sum is antisymmetric in the same indexes, the result is zero as claimed.

Expanding \cref{eqn:ProblemSet1Problem1:120} explicitly for \( \nu = 0 \), we find Gauss's law for the magnetic field
\begin{equation}\label{eqn:ProblemSet1Problem1:160}
\begin{aligned}
0
&=
\epsilon^{ijk} \partial_i F_{jk}
\\&=
-\partial_i B^i
\\&= -\spacegrad \cdot \BB,
\end{aligned}
\end{equation}
For \( \nu = 1 \) % 2 3 0
\begin{equation}\label{eqn:ProblemSet1Problem1:180}
\begin{aligned}
0
&= \partial_2 F_{30} + \partial_3 F_{02} + \partial_0 F_{23}
\\&= -\partial_2 E^3 + \partial_3 E^2 - \PD{t}{B^1}
\\&= - (\spacegrad \cross \BE)_x - \PD{t}{B_x},
\end{aligned}
\end{equation}
and for \( \nu = 2 \) % 3 0 1
\begin{equation}\label{eqn:ProblemSet1Problem1:300}
\begin{aligned}
0
&= \partial_3 F_{01} + \partial_0 F_{13} + \partial_1 F_{30}
\\&= \partial_3 E^1
+ \PD{t}{B^2}
- \partial_1 E^3
\\&= (\spacegrad \cross \BE)_y + \PD{t}{B_y},
\end{aligned}
\end{equation}
and for \( \nu = 3 \) % 0 1 2
\begin{equation}\label{eqn:ProblemSet1Problem1:200}
\begin{aligned}
0
&= \partial_0 F_{12} + \partial_1 F_{20} + \partial_2 F_{01}
\\&=
- \PD{t}{B^3}
- \partial_1 E^2
+ \partial_2 E^1
\\&= - (\spacegrad \cross \BE)_z - \PD{t}{B_z},
\end{aligned}
\end{equation}
so
\begin{equation}\label{eqn:ProblemSet1Problem1:220}
0 = \spacegrad \cross \BE + \PD{t}{\BB},
\end{equation}
which is Faraday's law.

\makeSubAnswer{}{qft:problemSet1:1c}
For the source dependent Maxwell's equations we vary the action.
Recall that for a single field Lagrangian density \( \LL = \LL(\phi, \partial_\mu \phi) \) the variation of the action \( S = \int \LL \) can be found by Taylor expansion
\begin{equation}\label{eqn:ProblemSet1Problem1:240}
\begin{aligned}
\delta S
&= \int d^4 x \delta \LL
\\&= \int d^4 x \PD{\phi}{\LL} \delta \phi + \int d^4 x \PD{(\partial_\nu \phi)}{\LL} \delta (\partial_\nu \phi)
\\&= \int d^4 x \PD{\phi}{\LL} \delta \phi + \int d^4 x \PD{(\partial_\nu \phi)}{\LL} \partial_\nu \delta \phi
\\&=
  \int d^4 x \PD{\phi}{\LL} \delta \phi
+ \int d^4 x \partial_\nu \lr{ \PD{(\partial_\nu \phi)}{\LL} \delta \phi }
- \int d^4 x \partial_\nu \lr{ \PD{(\partial_\nu \phi)}{\LL} } \delta \phi
\\&=
  \int d^4 x \delta \phi \lr{
\PD{\phi}{\LL}
- \partial_\nu \lr{ \PD{(\partial_\nu \phi)}{\LL} }
}
\end{aligned}
\end{equation}
Assuming that \( \delta \phi \) is stationary at the boundaries killed the second integral in the second last step.  Setting \( \delta S = 0 \) gives the Euler-Lagrange equations for a Lagrangian density that is dependent on a single field and its first derivatives
\begin{equation}\label{eqn:ProblemSet1Problem1:260}
0 =
\PD{\phi}{\LL}
- \partial_\nu \lr{ \PD{(\partial_\nu \phi)}{\LL} }.
\end{equation}
For a multiple particle field we must Taylor expand around each field variable, so we have one equation for each field
\begin{equation}\label{eqn:ProblemSet1Problem1:280}
0 =
\PD{A_\mu}{\LL}
- \partial_\nu \lr{ \PD{(\partial_\nu A_\mu)}{\LL} }.
\end{equation}
We wish to apply \cref{eqn:ProblemSet1Problem1:280} to the field Lagrangian density
\begin{equation}\label{eqn:ProblemSet1Problem1:320}
\LL =
-\inv{4}
F_{\mu\nu}
F^{\mu\nu}
- A_\mu j^\mu,
\end{equation}
and vary with respect to the fields \( A_{\mu} \) (or \( A^{\mu} \)).

The first order partials are trivial
\begin{equation}\label{eqn:ProblemSet1Problem1:340}
\PD{A_\mu}{\LL} = - j^\mu,
\end{equation}
but we have to do a bit more work for the rest
\begin{equation}\label{eqn:ProblemSet1Problem1:360}
\begin{aligned}
\PD{(\partial_\nu A_\mu)}{\LL}
&=
-\inv{2}
F^{\alpha\beta} \PD{(\partial_\nu A_\mu)}{} F_{\alpha\beta}
\\&=
-\inv{2}
F^{\alpha\beta} \PD{(\partial_\nu A_\mu)}{} \lr{
\partial_\alpha A_\beta -
\partial_\beta A_\alpha}
\\&=
-\inv{2}
F^{\nu\mu}
+
\inv{2}
F^{\mu\nu}
\\&=
F^{\mu\nu}.
\end{aligned}
\end{equation}
Putting the pieces together, we have
\begin{equation}\label{eqn:ProblemSet1Problem1:380}
0 = -j^\mu - \partial_\nu F^{\mu\nu},
\end{equation}
or
%\begin{boxed}\label{eqn:ProblemSet1Problem1:400}
\boxedEquation{eqn:ProblemSet1Problem1:420}{
\partial_\mu F^{\mu\nu} = j^\nu.
}
%\end{boxed}

For \( \nu = 0 \) this is
\begin{equation}\label{eqn:ProblemSet1Problem1:440}
\partial_\mu F^{\mu 0} = j^0,
\end{equation}
or
\begin{equation}\label{eqn:ProblemSet1Problem1:460}
\begin{aligned}
\rho
&=
\partial_k F^{k 0}
\\&=
-\partial_k F_{k 0}
\\&=
\partial_k F_{0 k}
\\&=
\spacegrad \cdot \BE,
\end{aligned}
\end{equation}
which is Gauss's law.

% for the other indexes \( \nu \) we have
\begin{equation}\label{eqn:ProblemSet1Problem1:480}
\begin{aligned}
j^1
&=
\partial_\mu F^{\mu 1}
\\&=
\partial_0 F^{0 1}
+
\partial_2 F^{2 1}
+
\partial_3 F^{3 1}
\\&=
- \PD{t}{E_x}
+ \partial_2 B_z
- \partial_3 B_y
\\&= \lr{ -\BE + \spacegrad \cross \BB } \cdot \Be_1
\end{aligned}
\end{equation}
\begin{equation}\label{eqn:ProblemSet1Problem1:500}
\begin{aligned}
j^2
&=
\partial_\mu F^{\mu 2}
\\&=
\partial_3 F^{3 2}
+
\partial_0 F^{0 2}
+
\partial_1 F^{1 2}
\\&=
  \partial_3 B_x
- \PD{t}{E_y}
- \partial_1 B_z
\\&= \lr{ -\BE + \spacegrad \cross \BB } \cdot \Be_2
\end{aligned}
\end{equation}
\begin{equation}\label{eqn:ProblemSet1Problem1:520}
\begin{aligned}
j^3
&=
\partial_\mu F^{\mu 3}
\\&=
\partial_0 F^{0 3}
+
\partial_1 F^{1 3}
+
\partial_2 F^{2 3}
\\&=
- \PD{t}{E_z}
+ \partial_1 B_y
- \partial_2 B_x
\\&= \lr{ -\BE + \spacegrad \cross \BB } \cdot \Be_3,
\end{aligned}
\end{equation}
so
\begin{equation}\label{eqn:ProblemSet1Problem1:540}
\BJ = -\PD{t}{\BE} + \spacegrad \cross \BB,
\end{equation}
which recovers the Ampere-Maxwell equation.

\makeSubAnswer{}{qft:problemSet1:1d}
The portion of the action that is dependent on the worldline is
%in terms of a parameterization \( X(s) \) is
\begin{equation}\label{eqn:ProblemSet1Problem1:560}
S =
%\int_{\text{worldline}} ds \lr{ - m - e A_\mu \frac{dX^\mu}{ds} }
\int_{\text{worldline}} \lr{ - m ds - e A_\mu dX^\mu }.
\end{equation}

Let's consider the variation of each of these terms separately, starting with \( \delta ds \)
\begin{equation}\label{eqn:ProblemSet1Problem1:580}
\begin{aligned}
\delta \int ds
&=
\delta \int \sqrt{ dX^\mu dX_\mu }
\\&=
\int \inv{2 ds} 2 dX^\mu \delta dX_\mu
\\&=
\int \frac{dX^\mu}{ds} d \delta X_\mu
\\&=
\int d \lr{ \frac{dX^\mu}{ds} \delta X_\mu } - d \lr{ \frac{dX^\mu}{ds} } \delta X_\mu
\\&=
\evalbar{ \frac{dX^\mu}{ds} \delta X_\mu }{\Delta s} - \int d \lr{ \frac{dX^\mu}{ds} } \delta X_\mu.
\end{aligned}
\end{equation}
The endpoints of the worldline are presumed to be stationary, which kills the boundary term, leaving just
\begin{equation}\label{eqn:ProblemSet1Problem1:600}
\delta \int ds = - \int d U^\mu \delta X_\mu.
\end{equation}
Now let's compute the variation of the potential term
\begin{equation}\label{eqn:ProblemSet1Problem1:620}
\begin{aligned}
\delta \int A_\mu dX^\mu
&=
\int (\delta A_\mu) dX^\mu
+
\int A_\mu \delta dX^\mu
\\&=
\int \partial_\nu A_\mu \delta X^\nu dX^\mu
-
\int d A_\mu \delta X^\mu
\\&=
\int \partial_\nu A_\mu \delta X^\nu U^\mu ds
-
\int \partial_\nu A_\mu dX^\nu \delta X^\mu
\\&=
\int \lr{ \partial_\nu A_\mu U^\mu \delta X^\nu
-
\partial_\nu A_\mu U^\nu \delta X^\mu
}
ds
\\&=
\int \lr{ \partial_\nu A_\mu - \partial_\mu A_\nu } U^\mu \delta X^\nu ds
\\&=
\int F_{\nu\mu} U^\mu \delta X^\nu ds
\\&=
\int F^{\nu\mu} U_\mu \delta X_\nu ds.
\end{aligned}
\end{equation}
Here the boundary term has been dropped again after integration by parts, and an index switcheroo was done to factor out a common
\( U^\mu \delta X^\nu ds \) term from the integrand, and we finish off with a set of raising and lowering operations on all the matched indexes.  Putting the pieces back together we have
\begin{equation}\label{eqn:ProblemSet1Problem1:640}
\begin{aligned}
\delta S
&=
  \int
\lr{
-m \dot{U}^\nu
-
e
F^{\nu\mu} U_\mu
}
\delta X_\nu ds
\\&=
  \int
\lr{
m \dot{U}^\mu
-
e
F^{\mu\nu} U_\nu
}
\delta X_\mu ds
.
\end{aligned}
\end{equation}
Requiring \( \delta S = 0 \) for all worldline path variations \( \delta X_\mu \) means that the equations of motion are
%\begin{equation}\label{eqn:ProblemSet1Problem1:660}
\boxedEquation{eqn:ProblemSet1Problem1:660}{
m \frac{dU^\mu}{ds} = e F^{\mu\nu} U_\nu,
}
%\end{equation}
as expected.

To unpack this and obtain the conventional Lorentz force equation we need to relate the proper time derivatives to the time of a stationary observer
\begin{equation}\label{eqn:ProblemSet1Problem1:680}
\frac{d}{ds} =
\frac{dt}{ds}
\frac{d}{dt},
% = (1, \Bv) \frac{dt}{ds},
\end{equation}
The stationary observer's world line is \( X^\mu = (t, \Bx) \), and the spacetime interval on that worldline is
\begin{equation}\label{eqn:ProblemSet1Problem1:700}
ds^2 = dt^2 - d\Bx^2,
\end{equation}
or
\begin{equation}\label{eqn:ProblemSet1Problem1:720}
\lr{\frac{ds}{dt}}^2 = 1 - {\frac{dx}{dt}}^2 = 1 - \Bv^2.
\end{equation}
\Cref{eqn:ProblemSet1Problem1:680} can now be written as
\begin{equation}\label{eqn:ProblemSet1Problem1:740}
\frac{d}{ds} =
\inv{\sqrt{ 1 - \Bv^2 }}
\frac{d}{dt}
\equiv \gamma
\frac{d}{dt}.
\end{equation}
In particular, the proper velocity is
\begin{equation}\label{eqn:ProblemSet1Problem1:760}
U^\mu = \gamma \lr{ 1, \Bv }.
\end{equation}

First inserting \( \mu = 0 \) into \cref{eqn:ProblemSet1Problem1:660} now gives
\begin{equation}\label{eqn:ProblemSet1Problem1:780}
\begin{aligned}
\frac{d}{ds} \frac{m}{\sqrt{1 - \Bv^2}}
&= e F^{0 k} U_k
\\&= (-1)^2 e F_{0 k} U^k
\\&= e \BE \cdot \Bv \gamma,
\end{aligned}
\end{equation}
or
\begin{equation}\label{eqn:ProblemSet1Problem1:800}
\frac{d}{dt} \frac{m}{\sqrt{1 - \Bv^2}} = e \BE \cdot \Bv.
\end{equation}
This is the timelike portion of the Lorentz force equation in non-covariant form and natural units (cf. \citep{landau1980classical} eq. (17.7).)

For the \( \mu \ne 0 \) case, we find
\begin{equation}\label{eqn:ProblemSet1Problem1:820}
\begin{aligned}
\gamma \frac{d}{dt} \frac{m \Bv}{\sqrt{1 - \Bv^2}}
&= e F^{j \nu} U_\nu \Be_j
\\&= e F^{j 0} \Be_j
- e \sum_{1 \le (j \ne k) \le 3} F^{j k} v^k \Be_j \gamma
\\&= e \BE + e \epsilon_{jki} B^i v^k \Be_j \gamma
\\&= e \BE + e \Bv \cross \BB \gamma,
\end{aligned}
\end{equation}
or
\begin{equation}\label{eqn:ProblemSet1Problem1:840}
\frac{d\Bp}{dt} = e \BE + e \Bv \cross \BB,
\end{equation}
which is the Lorentz force equation in natural units in terms of \( \Bp = d(\gamma m \Bv)/dt \), the relativistically correct momentum from the viewpoint of a stationary observer.
}}

      %
% Copyright � 2018 Peeter Joot.  All Rights Reserved.
% Licenced as described in the file LICENSE under the root directory of this GIT repository.
%
%{
\makeproblem{Part of Problem 2.2 from Peskin and Schroeder (reproduced below).}{qft:problemSet1:2}{
Consider a complex scalar field with action \( S = \int d^4x\lr{\partial_\mu \phi^\dagger \partial^\mu \phi - m^2 \phi^\dagger \phi}\).  When doing the variational principle consider \( \phi \) and \(\phi^\dagger \) as independent, rather than their real and imaginary parts (this is equivalent, but more convenient).

\makesubproblem{}{qft:problemSet1:2a}
Show that \( H = \int d^3x \lr{ \pi^\dagger \pi + \spacegrad \phi^\dagger \cdot \spacegrad \phi + m^2 \phi^\dagger \phi } \) and that the Klein-Gordon equation is obeyed by \( \phi \) and \( \phi^\dagger\).
\makesubproblem{}{qft:problemSet1:2b}
Introduce complex amplitudes, diagonalize the Hamiltonian, and quantize the theory. Show that the theory has now two sets of particles.
\makesubproblem{}{qft:problemSet1:2c}
Write the charge conserved due to the global \( U(1) \) symmetry, \( Q = \int d^3 x \frac{i}{2} \lr{ \phi^\dagger \pi^\dagger - \pi \phi } \),
in
terms of creation and annihilation operators and find the charge of the particles of each type.
} % makeproblem

\makeanswer{qft:problemSet1:2}{
\makeSubAnswer{}{qft:problemSet1:2a}
Classically, evaluating the Euler-Lagrange equations gives us
\begin{dmath}\label{eqn:ProblemSet1Problem2:20}
\begin{aligned}
\PD{\phi}{\LL} &= -m^2 \phi^\dagger \\
\PD{(\partial_\mu \phi)}{\LL} &= \partial^\mu \phi^\dagger \\
\PD{\phi^\dagger}{\LL} &= -m^2 \phi       \\
\PD{(\partial_\mu \phi^\dagger)}{\LL} &= \partial^\mu \phi,
\end{aligned}
\end{dmath}
so the equations of the field are respectively
\begin{dmath}\label{eqn:ProblemSet1Problem2:40}
%\boxedEquation{eqn:ProblemSet1Problem2:40}{
\begin{aligned}
\partial_\mu \partial^\mu \phi^\dagger &=  -m^2 \phi^\dagger \\
\partial_\mu \partial^\mu \phi &=  -m^2 \phi.
\end{aligned}
%}
\end{dmath}
These are Klein-Gordon equations for each field variable \( \phi, \phi^\dagger \) as expected, although this can be made more explicit written out explicitly in the stationary observer frame
\boxedEquation{eqn:ProblemSet1Problem2:340}{
\begin{aligned}
\lr{ \partial_{tt} - \spacegrad^2 + m^2 } \phi^\dagger &= 0 \\
\lr{ \partial_{tt} - \spacegrad^2 + m^2 } \phi &= 0 \\
\end{aligned}
}
To find the Hamiltonian, note that the Lagrangian density written out explicitly is
\begin{dmath}\label{eqn:ProblemSet1Problem2:60}
\LL = \partial_0 \phi^\dagger \partial_0 \phi - (\spacegrad \phi^\dagger) \cdot (\spacegrad \phi) - m^2 \phi^\dagger \phi,
\end{dmath}
so the conjugate momentum densities are
\begin{dmath}\label{eqn:ProblemSet1Problem2:80}
\begin{aligned}
\pi(\Bx, t) &= \PD{(\partial_0 \phi)}{\LL} = \partial_0 \phi^\dagger \\
\pi^\dagger(\Bx, t) &= \PD{(\partial_0 \phi^\dagger)}{\LL} = \partial_0 \phi \\
\end{aligned}
\end{dmath}

The Hamiltonian (including a \( p \dot{q} \) term for each of \( \phi, \phi^\dagger \)) is
\begin{dmath}\label{eqn:ProblemSet1Problem2:100}
H
= \int d^3 x \lr{ \pi \partial_0 \phi + \pi^\dagger \partial_0 \phi^\dagger - \LL }
=
\int d^3 x \lr{ \pi \pi^\dagger + \pi^\dagger \pi - \pi \pi^\dagger +
(\spacegrad \phi^\dagger) \cdot (\spacegrad \phi) + m^2 \phi^\dagger \phi
 }
=
\int d^3 x \lr{ \pi^\dagger \pi +
(\spacegrad \phi^\dagger) \cdot (\spacegrad \phi) + m^2 \phi^\dagger \phi
 }
\end{dmath}
\makeSubAnswer{}{qft:problemSet1:2b}
To canonically quantize the fields, we promote the fields to operators, demand that we have commutators for conjugate pairs of operators
\begin{equation}\label{eqn:ProblemSet1Problem2:120}
\antisymmetric{\phi(\Bx)}{\pi^\dagger(\By)} =
\antisymmetric{\phi^\dagger(\Bx)}{\pi(\By)} = i \delta^3(\Bx - \By),
\end{equation}
and requiring that all the other operator pairs \( \phi \phi^\dagger, \pi \pi^\dagger, \phi\pi, \phi^\dagger \pi^\dagger \) commute\footnote{As I discovered the hard way doing this assignment is it also possible to find the KG equation by demanding \(
\antisymmetric{\phi(\Bx)}{\pi(\By)} =
\antisymmetric{\phi^\dagger(\Bx)}{\pi^\dagger(\By)} = i \delta^3(\Bx - \By) \), however, doing so leads to trouble when attempting to find the pairs of properly behaving creation and annilation operators.  For a nice discussion that motivates the ``proper choice'', requiring commutators for conjugate pairs of position-momentum operators, see \citep{DavidMayrhofer} where the author starts with separate real and imaginary fields and builds the complex representation systematically.}.
If we compute the time evolution of such quantized \( \phi, \phi^\dagger \) field operators using Hamiltonian time evolution operators, the result differs from the classical case by a conjugatation operation
\begin{dmath}\label{eqn:ProblemSet1Problem2:140}
\ddt{\phi(\Bx)}
= i \antisymmetric{H(\By)}{\phi(\Bx)}
= i \int d^3 y \antisymmetric{\pi^\dagger(\By) \pi(\By)}{\phi(\Bx)}
= i \int d^3 y \,\pi(\By) \antisymmetric{\pi^\dagger(\By)}{\phi(\Bx)}
= \int d^3 y \,\pi(\By) \delta^3(\Bx - \By)
= \pi(\Bx),
\end{dmath}
\begin{dmath}\label{eqn:ProblemSet1Problem2:160}
\ddt{\phi^\dagger(\Bx)}
= i \antisymmetric{H(\By)}{\phi^\dagger(\Bx)}
= i \int d^3 y \antisymmetric{\pi^\dagger(\By) \pi(\By)}{\phi^\dagger(\Bx)}
= i \int d^3 y \pi^\dagger(\By) \antisymmetric{\pi(\By)}{\phi(\Bx)}
= \int d^3 y \pi^\dagger(\By) \delta^3(\Bx - \By)
= \pi^\dagger(\Bx),
\end{dmath}
which differs from \cref{eqn:ProblemSet1Problem2:80}.  However, should we compute the time evolution of the momentum operators,
this conjugation difference is ``cancelled'' and we end up with the Klein-Gordon equations in the end.
\begin{dmath}\label{eqn:ProblemSet1Problem2:180}
\ddt{\pi(\Bx)}
= i \antisymmetric{H(\By)}{\pi(\Bx)}
=
i \int d^3 y \antisymmetric{
\spacegrad_\By \phi^\dagger(\By) \cdot \spacegrad_\By \phi(\By)
}{ \pi(\Bx) }
+
i m^2 \int d^3 y \antisymmetric{ \phi^\dagger(\By) \phi(\By) }{ \pi(\Bx) }.
\end{dmath}
The second integral is easy
\begin{dmath}\label{eqn:ProblemSet1Problem2:200}
\int d^3 y \antisymmetric{ \phi^\dagger(\By) \phi(\By) }{ \pi(\Bx) }
=
\int d^3 y\,
\phi(\By)
\antisymmetric{
\phi(\By) }{ \pi(\Bx) }
=
i
\int d^3 y\,
\phi(\By)
\delta^3(\Bx - \By)
=
i
\phi(\Bx).
\end{dmath}
To evaluate the first integral in \cref{eqn:ProblemSet1Problem2:180} we can make use of the linearity to find
\begin{dmath}\label{eqn:ProblemSet1Problem2:220}
i \int d^3 y \antisymmetric{ \spacegrad_\By \phi^\dagger(\By) \cdot \spacegrad_\By \phi(\By) }{ \pi(\Bx) }
=
i \int d^3 y \, \spacegrad_\By \antisymmetric{\phi^\dagger(\By)}{\pi(\Bx)} \cdot \spacegrad_\By \phi(\By)
=
- \int d^3 y \, \spacegrad_\By \delta(\Bx - \By) \cdot \spacegrad_\By \phi(\By)
=
- \int d^3 y \lr{
\spacegrad_\By \cdot \lr{ \delta(\Bx - \By) \spacegrad_\By \phi(\By) }
-
\delta(\Bx - \By) \spacegrad_\By^2 \phi(\By)
}
=
- \int dA_y 
\delta(\Bx - \By) \PD{n}{\phi(\By)}
+
\spacegrad^2 \phi(\Bx).
\end{dmath}
Provided we have some justification for declaring the first integral zero (I'm not sure what that was in class, as the delta function isn't really well behaved unless integrated over a volume), we are left with
\begin{dmath}\label{eqn:ProblemSet1Problem2:260}
\ddt{\pi(\Bx)}
=
\spacegrad^2 \phi(\Bx) - m^2 \phi(\Bx),
\end{dmath}
or
\begin{dmath}\label{eqn:ProblemSet1Problem2:280}
\frac{d^2 \phi}{dt^2}
=
\spacegrad^2 \phi - m^2 \phi.
\end{dmath}
which is a KG equation for \( \phi\).
As the Hamiltonian is symmetric in \( \pi, \pi^\dagger \) and \( \phi, \phi^\dagger \) repeating this calculation for \( \dot{\pi}^\dagger \) gives
\begin{dmath}\label{eqn:ProblemSet1Problem2:300}
\ddt{\pi^\dagger(\Bx)}
=
\spacegrad^2 \phi^\dagger(\Bx) - m^2 \phi^\dagger(\Bx),
\end{dmath}
which is also a KG equation for the quantized field \( \phi \)
\begin{dmath}\label{eqn:ProblemSet1Problem2:320}
\frac{d^2 \phi^\dagger}{dt^2}
=
\spacegrad^2 \phi^\dagger - m^2 \phi^\dagger,
\end{dmath}
as expected.

\makeSubAnswer{}{qft:problemSet1:2c}

TODO.
}
%}

      %
% Copyright � 2018 Peeter Joot.  All Rights Reserved.
% Licenced as described in the file LICENSE under the root directory of this GIT repository.
%
\makeproblem{
Zero point energy, an exercise in unit conversion, and scales related to the ``cosmological constant problem''
}{qft:problemSet1:3}{
In class, we showed that the zero-point energy of the quantized massless scalar field (we are taking this case, because in the physically relevant case of electrodynamics, the number of degrees of freedom and the associated vacuum energy is the same as that of two massless scalar fields) can be written as:
\begin{dmath}\label{eqn:ProblemSet1Problem3:20}
E_{\text{vac}} = V_3 \int \frac{d^3 k}{(2\pi)^3} \frac{\omega_k}{2}.
\end{dmath}
where \( V_3 \) is the (large, i.e., almost infinite) volume of space. This expression diverges, because we assume that electromagnetic fields and photons of arbitrarily large momenta exist. There's no justification to this, as particle physicists have only probed the Standard Model up to energies of order a few \( \si{TeV} \). Assume, then, that the integral above is cut off at some maximum value of the momentum \( \Lambda \) (called the ``UV cutoff''), say of order \( 10 \,\si{TeV} \).
\makesubproblem{}{qft:problemSet1:3a}
What is the value of the vacuum energy density \( \rho_{\text{vac}} \), in units of \( \si{g/cm3} \).
\makesubproblem{}{qft:problemSet1:3b}
What value should \( \Lambda \) have in order that \( \rho_{\text{vac}} \) matches the observed value of the ``dark energy'', of order
\( \rho_{\text{dark}} \sim 10^{-29} \, \si{g/cm^3} \).
Express \( \Lambda \) both as a high-energy scale cutoff and as a short-distance cutoff.
\makesubproblem{}{qft:problemSet1:3c}
What is the ratio of \( \rho_{\text{vac}} \) for \( \Lambda \sim M_{\text{Planck}} \) to \( \rho_{\text{dark}} \)?
\makesubproblem{}{qft:problemSet1:3d}
Note that the zero-point energies of phonons -- the zero point energies of the quantized collective sound oscillations of nuclei in a crystal -- are given, up to simple numerical factors counting the numbers of polarizations (which we won't worry about here) by an expression similar to the above.
This is because phonons are massless scalar fields propagating with the speed of sound instead of speed of light.
Notice that this difference is irrelevant as \( c \) appears in \( E_{\text{vac}} \) simply: \( k \) is a wavevector and \( \omega_k = c k \) -- a frequency (secretly multiplied by \( \Hbar \), of course).
In the case of phonons, however, we are well aware that a cutoff scale exists and we understand well its nature: it is given by the interatomic separation, as the notion of phonons does not make sense for shorter wavelengths.
Now take \( k_{\text{max}} = \Lambda \sim 1/a_0 \), with \( a_0 \) of order the Bohr radius and estimate the energy density of the zero point fluctuations in a crystal.

Compare your result to the typical rest energy (i.e. mass) density of crystals.
The results from the first three items above lead to a puzzle commonly referred to as the ``cosmological constant problem''.
There are various proposals for its solution, ranging from cancellations between the contributions of high and low momentum oscillators, anthropic principle (multiverse) considerations, modifications of gravity at long distances, to name a few.
The issue awaits your input!
} % makeproblem

\makeanswer{qft:problemSet1:3}{
\makeSubAnswer{}{qft:problemSet1:3a}
TODO.
\makeSubAnswer{}{qft:problemSet1:3b}
TODO.
\makeSubAnswer{}{qft:problemSet1:3c}
TODO.
\makeSubAnswer{}{qft:problemSet1:3d}
TODO.
}

      %
% Copyright � 2018 Peeter Joot.  All Rights Reserved.
% Licenced as described in the file LICENSE under the root directory of this GIT repository.
%
\makeproblem{ Scale invariance and conserved charge }{qft:problemSet1:4}{
Consider classical electrodynamics with the Lagrangian
\begin{dmath}\label{eqn:ProblemSet1Problem4:20}
S = \int d^4 x \lr{ -\inv{4} F_{\mu\nu} F^{\mu\nu} }.
\end{dmath}
Consider the following ``dilatation'' (or ``scale'') transformation:
\begin{dmath}\label{eqn:ProblemSet1Problem4:40}
\begin{aligned}
x_\mu &\rightarrow x'_\mu = e^d x_\mu \\
A_\mu(x) &\rightarrow A'_\mu(x') = e^{-d} A_\mu(x),
\end{aligned}
\end{dmath}
where \( d \) is a constant, called the dilatation parameter.
\makesubproblem{}{qft:problemSet1:4a}
Show that the action is invariant under dilatations.
\makesubproblem{}{qft:problemSet1:4b}
Find the corresponding Noether current.
\makesubproblem{}{qft:problemSet1:4c}
Show that -- perhaps, after a redefinition of \( j_\mu \) ; notice that any conserved current \( j_\mu \) can be
redefined by adding to it \( \partial^\nu C_{\mu\nu} \), where \( C_{\mu\nu} \) is antisymmetric, without spoiling its conservation
(in this case \( C \) can depend on \( x^\mu, \partial^\mu \) and \( A^\mu \), of course) the dilatation current is simply related
to the energy-momentum tensor: \( j^{\text{con f}}_\mu = x_\nu {{T^\nu}_\mu}^{\text{con f}}\), where the symbol con f indicates that
these are the conformal energy-momentum tensor and dilatation current. Notice that this problem, secretly, requires you to also derive \( T^{\mu\nu} \) for the electromagnetic field.
\makesubproblem{}{qft:problemSet1:4d}
Show, then, that conservation of \(  j^{\text{con f}}_\mu  \) implies that the energy-momentum tensor of classical
electrodynamics is traceless (the trace of the tensor is defined as usual to be \( g_{\mu\nu} T^{\mu\nu}\)).
\makesubproblem{}{qft:problemSet1:4e}
Finally, open your classical electrodynamics books and recall the interpretation of the \( T^{00}, T^{xx},T^{yy} \), etc., components of the energy momentum tensor as energy density and pressure. Show that the tracelessness of \( T^{\mu\nu} \) is equivalent to the familiar relation
\( p = \rho/3 \) between the energy density and pressure of isotropic radiation -- the equation of state of blackbody radiation.
Dilatation invariance in QED (and QCD) is perhaps the simplest example of a symmetry, where the classical action is invariant, but the quantum theory is not (as you will learn later, in the spring class). Broken scale invariance arises because one has to introduce a short-distance cutoff (a UV ``regulator'') to define the quantum theory. (We already saw an indication of the need for a regulator when we considered the divergent zero point energy of the free quantum scalar field.)
} % makeproblem

\makeanswer{qft:problemSet1:4}{
\makeSubAnswer{}{qft:problemSet1:4a}
TODO.
\makeSubAnswer{}{qft:problemSet1:4a}
TODO.
\makeSubAnswer{}{qft:problemSet1:4b}
TODO.
\makeSubAnswer{}{qft:problemSet1:4c}
TODO.
\makeSubAnswer{}{qft:problemSet1:4d}
TODO.
\makeSubAnswer{}{qft:problemSet1:4e}
TODO.
}

      %
% Copyright � 2018 Peeter Joot.  All Rights Reserved.
% Licenced as described in the file LICENSE under the root directory of this GIT repository.
%
\makeproblem{Observability of the zero point energy: the Casimir force.}{qft:problemSet1:5}{
In class, when discussing the quantization of the real scalar field, we found the sum of zero
point energies of the harmonic oscillators (one per each \( \Bk \) ) into which we decomposed the field:
\begin{dmath}\label{eqn:ProblemSet1Problem5:20}
E_{\text{zero point}} =
V_3 \int \frac{d^3 k}{(2\pi)^3} \frac{\Hbar \omega_\Bk}{2}.
\end{dmath}
\( \vdots \)
\makesubproblem{}{qft:problemSet1:5a}
Show that the boundary conditions on the plates impose a quantization condition on the allowed values of field momentum perpendicular to the plates, i.e. \( k_x = n\pi/a, n = 0, \pm 1,  \pm 2, \cdots \) [e.g., recall your waveguide physics].
\makesubproblem{}{qft:problemSet1:5b}
Consider now the contribution to the energy of the vacuum fluctuations of the field in the space between the plates and find the zero point energy per unit area of the plates...
%To do this, replace the integral over \( k_x \) in
%\cref{eqn:ProblemSet1Problem5:20}
%by a sum over \( n \), \( \int dk_x = (\pi/a)\sum_n\) [Hint: to save work, use the fact that the correct expression should have the property that as the plates are removed, \( a \rightarrow \infty \), the energy (per unit volume) should give back
%\cref{eqn:ProblemSet1Problem5:20}
%]. Does the resulting expression for the zero point energy still diverge?
\makesubproblem{}{qft:problemSet1:5c}
Show now, starting from
\cref{eqn:ProblemSet1Problem5:20}
, with integral replaced by sum, that the difference between the zero point energies per unit area, in the space between the plates in the presence of the plates and without the plates is:
\begin{dmath}\label{eqn:ProblemSet1Problem5:40}
\Delta E_{\text{vac}}(a) = \Hbar c \int_0^\infty \frac{dk}{2\pi} k \lr{ \frac{k}{4}
   + \inv{2} \sum_{n = 1}^\infty \sqrt{ k^2 + \frac{n^2 \pi^2}{a^2} }
   - \inv{2} \int_0^\infty dn \sqrt{ k^2 + \frac{n^2 \pi^2}{a^2} }
}.
\end{dmath}
where, obviously, \( k \) is radial wave vector in \(y, z\)-directions.
\makesubproblem{}{qft:problemSet1:5d}
The expression
\cref{eqn:ProblemSet1Problem5:40}
is still ill-defined, as every single term is infinite
...
Show that \cref{eqn:ProblemSet1Problem5:40} (with the cutoff \( f(k) \) as described in the original problem spec) can be written as:
\begin{dmath}\label{eqn:ProblemSet1Problem5:260}
\Delta E_{\text{vac}}(a)
=
\frac{\Hbar c \pi^2}{8 a^3} \lr{
   \inv{2} F(0) + \sum_{n = 1}^\infty F(n) - \int_0^\infty dn F(n)
},
\end{dmath}
where
\begin{dmath}\label{eqn:ProblemSet1Problem5:280}
F(n) = \int_0^\infty du \sqrt{ u + n^2 } f((\pi/a) \sqrt{u + n^2}).
\end{dmath}
\makesubproblem{}{qft:problemSet1:5e}
To calculate ..., use the Euler-Maclaurin formula:...
\makesubproblem{}{qft:problemSet1:5f}
Show, now, that the final result for the Casimir energy per unit area of the plates is:...
\makesubproblem{}{qft:problemSet1:5g}
To get some idea of what experimentalists have to go through, estimate the force acting on plates of area \( 1 \si{cm^2}\) a micron apart...
\makesubproblem{}{qft:problemSet1:5h}
A final bonus question: what if the scalar field had a mass, \( m \)?  Would you expect an effect if \( m \gg 1/a \)? What if \( m \ll 1/a\)?
} % makeproblem

\makeanswer{qft:problemSet1:5}{
\makeSubAnswer{}{qft:problemSet1:5a}
Our scalar massless field satisfies the KG equation \( \lr{ \partial_{00} - \spacegrad^2 } \phi = 0 \), which has a plane wave superposition solution
\begin{dmath}\label{eqn:ProblemSet1Problem5:60}
\phi(\Bx, t) =
\alpha e^{i\omega t - i \Bk \cdot \Bx} +
\beta e^{-i\omega t + i \Bk \cdot \Bx},
\end{dmath}
where \( \omega^2 = \Bk^2 c^2 \).  At the boundaries
\begin{dmath}\label{eqn:ProblemSet1Problem5:80}
\begin{aligned}
\phi(0, 0)
&=
\alpha + \beta
= 0 \\
\phi(a, 0)
&=
\alpha e^{-i k_x a}
+
\beta e^{i k_x a}
= 0,
\end{aligned}
\end{dmath}
so
\begin{equation}\label{eqn:ProblemSet1Problem5:100}
e^{-i k_x a} = e^{i k_x a}.
\end{equation}
We must have \( e^{ 2 i k_x a } = 1 \), or
\begin{dmath}\label{eqn:ProblemSet1Problem5:120}
2 k_x a = 2 \pi n,
\end{dmath}
which provides the
\begin{dmath}\label{eqn:ProblemSet1Problem5:140}
k_x = \frac{\pi n}{a}
\end{dmath}
quantization constraint.

\makeSubAnswer{}{qft:problemSet1:5b}
Making the discrete substitution for \( k_x \), the vacuum energy per unit area is
\begin{dmath}\label{eqn:ProblemSet1Problem5:160}
\frac{E}{A}
= \inv{A}
\frac{A a}{(2\pi)^3} \int d^3 k \frac{\Hbar c \Abs{\Bk}}{2}
=
\frac{a}{(2\pi)^3} \frac{\Hbar c}{2} \int dk_y dk_z \lr{ \int dk_x} \sqrt{ k_x^2 + k_y^2 + k_z^2 }
= \frac{a \Hbar c}{16 \pi^3} \int dk_y dk_z \lr{ \frac{\pi}{a} \sum_{n = -\infty}^\infty } \sqrt{ k_x^2 + k_y^2 + k_z^2 }
= \frac{\Hbar c}{8 \pi} \int_{k = 0}^\infty k dk \sum_{n = -\infty}^\infty \sqrt{ \lr{ \frac{n \pi}{a}}^2 + k^2 }
= \frac{\Hbar c}{8 \pi} \int_{k = 0}^\infty k dk
\lr{ k + 2 \sum_{n = 1}^\infty \sqrt{ \lr{ \frac{n \pi}{a}}^2 + k^2 } },
\end{dmath}
so the energy per unit area (\(A\)) between the plates is
\begin{dmath}\label{eqn:ProblemSet1Problem5:180}
\frac{E}{A}
= \frac{\Hbar c}{8 \pi} \int_{k = 0}^\infty k dk
\lr{ k + 2 \sum_{n = 1}^\infty \sqrt{ \lr{ \frac{n \pi}{a}}^2 + k^2 } }.
\end{dmath}
As \( \int k^2 dk = k^3/3 \) is unbounded for large \( k \), this expression still diverges.
\makeSubAnswer{}{qft:problemSet1:5c}
The presence of the plates was accounted for by summing over \( k_x = \pi n/a \) for discrete \( n \).  The abscence of the boundaries may be accounted for by performing the integral over all values of \( n \), as in
\begin{dmath}\label{eqn:ProblemSet1Problem5:200}
\frac{E}{A}
= \frac{a}{(2\pi)^3} (2 \pi) \frac{\Hbar c}{2} \int_{k= 0}^\infty k dk \int dk_x \sqrt{ k^2 + k_x^2 }
= \frac{a \Hbar c}{8 \pi^2} \int_{k= 0}^\infty k dk \int_{k_x = -\infty}^\infty dk_x \sqrt{ k^2 + k_x^2 }
= \frac{a \Hbar c}{8 \pi^2} \frac{\pi}{a} \int_{k= 0}^\infty k dk \int_{n = -\infty}^\infty dn \sqrt{ k^2 + \lr{\frac{ n \pi }{a}}^2 }
= \frac{\Hbar c}{4 \pi} \int_{k= 0}^\infty k dk \int_{n = 0}^\infty dn \sqrt{ k^2 + \lr{\frac{ n \pi }{a}}^2 }.
\end{dmath}
The difference of \cref{eqn:ProblemSet1Problem5:180} and \cref{eqn:ProblemSet1Problem5:200} yeilds
\cref{eqn:ProblemSet1Problem5:40} as desired.
\makeSubAnswer{}{qft:problemSet1:5d}
Introducing the cutoff function \( f(k) \) into the integrand of \cref{eqn:ProblemSet1Problem5:40}, and making a change of variables \( k = \pi x /a \), we have
\begin{dmath}\label{eqn:ProblemSet1Problem5:220}
\Delta E_{\text{vac}}(a)
= \int_0^\infty \frac{dk}{2\pi} k \lr{
   \frac{k}{4} f(k)
   + \inv{2} \sum_{n = 1}^\infty \sqrt{ k^2 + \frac{n^2 \pi^2}{a^2} }
   f(\sqrt{k^2 + \frac{n^2 \pi^2}{a^2} } )
   - \inv{2} \int_0^\infty dn \sqrt{ k^2 + \frac{n^2 \pi^2}{a^2} }
   f(\sqrt{k^2 + \frac{n^2 \pi^2}{a^2} } )
}
=
\frac{\Hbar c \pi^2}{4 a^3}
\int_0^\infty dx x \lr{
   \frac{x}{2} f((\pi/a) x)
   + \sum_{n = 1}^\infty \sqrt{ x^2 + n^2 }
   f((\pi/a) \sqrt{x^2 + n^2})
   - \int_0^\infty dn \sqrt{ x^2 + n^2 }
   f((\pi/a) \sqrt{x^2 + n^2})
}.
\end{dmath}
Now let \( u = x^2 \)
\begin{dmath}\label{eqn:ProblemSet1Problem5:240}
\Delta E_{\text{vac}}(a)
=
\frac{\Hbar c \pi^2}{8 a^3}
\int_0^\infty du \lr{
   \frac{\sqrt{u}}{2} f((\pi/a) \sqrt{u})
   + \sum_{n = 1}^\infty \sqrt{ u + n^2 }
   f((\pi/a) \sqrt{u + n^2})
   - \int_0^\infty dn \sqrt{ u + n^2 }
   f((\pi/a) \sqrt{u + n^2})
}
=
\frac{\Hbar c \pi^2}{8 a^3} \lr{
   \inv{2} F(0) + \sum_{n = 1}^\infty F(n) - \int_0^\infty dn F(n)
},
\end{dmath}
which recovers \cref{eqn:ProblemSet1Problem5:260} as desired.

\makeSubAnswer{}{qft:problemSet1:5e}
To calculate the derivatives of \cref{eqn:ProblemSet1Problem5:280} we make a \( v = u + n^2 \) change of variables
\begin{dmath}\label{eqn:ProblemSet1Problem5:300}
F(n) = \int_{n^2}^\infty dv \sqrt{ v } f((\pi/a) \sqrt{v}),
\end{dmath}
and utilize
\begin{dmath}\label{eqn:ProblemSet1Problem5:320}
\frac{d}{du} \int_u^v f(t) dt = f(v) \frac{dv}{dt} - f(u) \frac{du}{dt},
\end{dmath}
so the first derivative is
\begin{dmath}\label{eqn:ProblemSet1Problem5:340}
F'(n)
= -n f((\pi/a) n) \frac{dn^2}{dn}
= -2 n^2 f((\pi/a) n),
\end{dmath}
the second is
\begin{dmath}\label{eqn:ProblemSet1Problem5:360}
F''(n)
= -4 n f((\pi/a) n) - 2 n^2 (\pi/a) f'((\pi/a) n),
\end{dmath}
the third is
\begin{dmath}\label{eqn:ProblemSet1Problem5:380}
F'''(n)
=
-4 f((\pi/a) n)
-4 n (\pi/a) f'((\pi/a) n)
- 4 n (\pi/a) f'((\pi/a) n),
- 2 n^2 (\pi/a)^2 f''((\pi/a) n).
\end{dmath}
Any higher order derivatives are dependent on \( f^{(k)}(\pi n/a), k \ge 1 \), so are zero at \(n = 0\) by construction.  Summarizing the values at \( n = 0 \) we have
\begin{dmath}\label{eqn:ProblemSet1Problem5:400}
\begin{aligned}
F'(0) &= 0 \\
F''(0) &= 0 \\
F'''(0) &= -4 \\
F^{(k)}(0) &= 0, \qquad k > 3.
\end{aligned}
\end{dmath}

\makeSubAnswer{}{qft:problemSet1:5f}
The original problem statement included the following statement of the Euler-Maclaurin formula:
\begin{dmath}\label{eqn:ProblemSet1Problem5:420}
   \inv{2} F(0) + \sum_{n = 1}^\infty F(n) - \int_0^\infty dn F(n) = -\inv{2!} B_2 F'(0) - \inv{4!} B_4 F'''(0) + \cdots,
\end{dmath}
so
\begin{dmath}\label{eqn:ProblemSet1Problem5:440}
   \inv{2} F(0) + \sum_{n = 1}^\infty F(n) - \int_0^\infty dn F(n) = \inv{4!} \inv{30} (-4) = -\inv{180}.
\end{dmath}
Inserting \cref{eqn:ProblemSet1Problem5:440} into
\cref{eqn:ProblemSet1Problem5:240} gives
\Delta E_{\text{vac}}(a)
=
-\frac{\Hbar c \pi^2}{8 a^3} \inv{180}
=
-\frac{\Hbar c \pi^2}{1440 a^3},
\end{dmath}
which is the desired result.
\makeSubAnswer{}{qft:problemSet1:5g}
TODO.
\makeSubAnswer{}{qft:problemSet1:5h}
TODO.
}


   \chapter{Problem Set 2.}

      %
% Copyright � 2018 Peeter Joot.  All Rights Reserved.
% Licenced as described in the file LICENSE under the root directory of this GIT repository.
%
\makeproblem{Spacetime behaviour of various Green's functions}{qft:problemSet2:1}{
Here, you'll study some properties of
\begin{equation}\label{eqn:ProblemSet2Problem1:20}
D(x) \equiv \antisymmetric{\phihat_{-}(x)}{\phihat_{+}(x)} = \int \frac{d^3 p}{(2\pi)^3 2 \omega_p} e^{-i \omega_p t + i \Bp \cdot \Bx}.
\end{equation}
\makesubproblem{}{qft:problemSet2:1a}
For m = 0 (``photon''), show that:
\begin{dmath}\label{eqn:ProblemSet2Problem1:40}
D(x) = -\inv{2 \pi^2} \calP \inv{t^2 - r^2} - \frac{i}{8 \pi} \lr{
\frac{\delta(t - r)}{r}
-\frac{\delta(t + r)}{r}
},
\end{dmath}
where \( r = \Abs{\Bx} \). Notice that \( D(x) \) is singular on the light cone \( t = r\). Does it vanish for spacelike separations?

Hint: Please recall that (and why!)
\begin{dmath}\label{eqn:ProblemSet2Problem1:60}
\inv{a \pm i \epsilon} = \calP \inv{a} \mp i \pi \delta(a)
\end{dmath}
(here \( \calP \) denotes ``principal value integration'',
as this relation is to be understood in terms of generalized functions, i.e. in the back of your mind it always needs to be integrated over a with suitable smooth and integrable ``test functions''). Note
also that what looks like a ``half-delta-function integral'' \( \int_0^\infty dy e^{i x y} \)
should really be understood as
\(
\lim_{\epsilon\rightarrow 0} \int_0^\infty dy e^{-\epsilon y + i x y}
\)
\makesubproblem{}{qft:problemSet2:1b}
For \( m^2 > 0 \), study the behavior of \( D(x) \) for spacelike \( x \) and find the asymptotic behavior for
\( -x^2 \gg 1/m^2 \) (i.e., at spacelike separations larger than the particle's Compton wavelength).
} % makeproblem

\makeanswer{qft:problemSet2:1}{
\makeSubAnswer{}{qft:problemSet2:1a}
TODO.
\makeSubAnswer{}{qft:problemSet2:1b}
TODO.
}

      %
% Copyright � 2018 Peeter Joot.  All Rights Reserved.
% Licenced as described in the file LICENSE under the root directory of this GIT repository.
%
\makeproblem{A model with \(SU(2)_L \times SU(2)_R\) internal global symmetry: chiral symmetry and the Higgs}{qft:problemSet2:2}{
This problem introduces a model to describe the symmetry realization of the nonabelian chiral symmetry in QCD (quantum chromodynamics).
The word ``chiral'' should become clear later in this class, but the ``nonabelian'' part will be clear below.
\(SU(2)_L \times SU(2)_R\) is an exact symmetry of QCD in the limit when the ``current masses'' of the \( u \) and \( d \) quark, \( m_u \) and \( m_d \), are taken to vanish.
In the real world, it is an approximate symmetry, in the sense that \( m_u \) and \( m_d \) are small compared to the intrinsic scale of QCD, given, say, by the proton mass (\( m_{u,d} \sim \,\si{MeV} \ll 1 \,\si{GeV} \)).
This is, thus, an example of an ``approximate symmetry''.

Closer to the theory you will study below, the scalar model with \(SU(2)_L \times SU(2)_R\) symmetry, is really the same as the Higgs sector in the Standard Model, in the limit when the electromagnetic and weak interactions are turned off.
\(SU(2)_L \times SU(2)_R\) becomes a symmetry in this limit.
It is only an approximate symmetry, as the electromagnetic and weak couplings (which explicitly break it) are dimensionless numbers smaller then unity.

Finally, to end the preaching preamble, the notion of approximate symmetries is not new and you have, for sure, been exposed to its usefulness when studying the hydrogen atom spectrum in quantum mechanics.

\makesubproblem{}{qft:problemSet2:2a}
The Lagrangian you will study is that of two complex scalar fields, assembled into a column \( \Phi = (\phi_1,\phi_2)^\T \) (the \( \T \) is here so I do not have to go through the trouble to write a column instead of a row).
It is given by:
\begin{dmath}\label{eqn:ProblemSet2Problem2:10} %(1)
\LL = \partial_\mu \Phi^\dagger \partial_\mu \Phi - m^2 \Phi^\dagger \Phi - \lambda \lr{\Phi^\dagger\Phi}^2.
\end{dmath}
Show that
\cref{eqn:ProblemSet2Problem2:10}
%(1)
is invariant under an \(SU(2)_L\) global symmetry transformation \( \Phi \rightarrow  U_L \Phi \), where \( U_L^\dagger U_L = 1 \) is a \( 2 \times 2 \) unitary matrix of unit determinant.
In addition, the Lagrangian has a \( U(1) \) symmetry, not part of \(SU(2)_L\), acting as \( \Phi \rightarrow  e^{i\alpha}\Phi\).
Find the currents and conserved charges under these symmetries.

Hint: recall that an infinitesimal \(SU(2)_L\) transformation can be written as \( U_L \approx \sigma^0 + i\omega_a \frac{\sigma^a}{2} \), where \( \sigma^0 \) is the unit \( 2 \times 2 \) matrix, \( \sigma^a, a = 1, 2, 3 \) are the Pauli matrices, and \( \omega_a \) are the three parameters of infinitesimal \(SU(2)_L\) transformations.

\makesubproblem{}{qft:problemSet2:2b}
Show that the charge operators, \( \hatQ^L_a, a = 1,2,3\), conserved due to \(SU(2)_L\) invariance, obey the angular momentum algebra, i.e., \( \antisymmetric{\hatQ^L_1}{ \hatQ^L_2 } = i \hatQ^L_3 \) (plus cyclic permutations).

\makesubproblem{}{qft:problemSet2:2c}
The Lagrangian \cref{eqn:ProblemSet2Problem2:10} has, however, a larger symmetry than simply the above \(SU(2)_L\).
To begin seeing this, instead of using \( \Phi = (\phi^1,\phi^2)^\T \) introduce the real and imaginary parts of \( \phi^{1,2} \).
Use \( \phi^1 = \psi^1 +i \psi^2, \phi^2 = \psi^3 +i \psi^4 \), and introducing \( \Psi = (\psi^1,\psi^2,\psi^3,\psi^4)^\T\), show that \cref{eqn:ProblemSet2Problem2:10} can be written as:
\begin{dmath}\label{eqn:ProblemSet2Problem2:20} % , (2)
\LL = a \partial_\mu \Psi^\T \partial^\mu \Psi -b m^2 \Psi^\T \Psi - c \lambda(\Psi^\T\Psi)^2
\end{dmath}
on the way determining the (pure numbers) \(a, b, c\).
The Lagrangian \cref{eqn:ProblemSet2Problem2:20} has, clearly, an \( O(4) \) symmetry, i.e., is invariant under \( \Psi \rightarrow  O \Psi\), where \( O \) is a \( 4 \times 4 \) orthogonal matrix, \( O^\T O = 1\).
Is there a continuous U(1) allowed in this case?

Comment: I will spare you finding the currents for \(SO(4)\) (\(SO(4)\) matrices are the restriction of \(O(4)\) matrices to the ones with unit determinant).
What you will do next, instead, is to use the equivalence of Lie algebras \(SO(4) \approx SU(2)_L \times SU(2)_R\), which will come about by another change of variables (see below).
Notice also that, as it comes, \(SO(4)\) happens to be the Euclidean version of \(SO(1,3)\).

\makesubproblem{}{qft:problemSet2:2d}
To expose the \(SU(2)_L \times SU(2)_R\) symmetry of \cref{eqn:ProblemSet2Problem2:10}, now use the following change of variables.
Consider, instead of \(\Phi\) in \cref{eqn:ProblemSet2Problem2:10} the \(2 \times 2\) matrix \(H\) made up by components of \(\Phi\) as follows:
\begin{equation}\label{eqn:ProblemSet2Problem2:30}
H \equiv \inv{\sqrt{2}} (i\sigma^2\Phi^\conj,\Phi) =
\inv{\sqrt{2}}
\begin{bmatrix}
\phi_2^\conj & \phi_1 \\
-\phi_1^\conj & \phi_2
\end{bmatrix}
\end{equation}
Show that under \(SU(2) \) transformations,

\begin{dmath}\label{eqn:ProblemSet2Problem2:520}
H \rightarrow  \inv{\sqrt{2}} (i\sigma^2 (U_L\Phi)^\conj, U_L \Phi) = \inv{\sqrt{2}} (U_L i \sigma^2 \Phi^\conj, U_L\Phi) = U_L H.
\end{dmath}

Hint: the tricky part is to show that \( i\sigma^2(U_L\Phi)^\conj = i\sigma^2 U_L^\conj \Phi^\conj = U_L i \sigma^2 \Phi^\conj\).
What you need to show, then, is that \( \sigma^2 U_L \sigma^2 = U_L^\conj \) (this fact will be very useful in our future studies of spinors, so make sure you understand it).

\makesubproblem{}{qft:problemSet2:2e}
Using the change of variables \cref{eqn:ProblemSet2Problem2:30}, show that
\begin{dmath}\label{eqn:ProblemSet2Problem2:40}
H^\dagger H = \inv{2}
\begin{bmatrix}
\Abs{\phi_1}^2 + \Abs{\phi_2}^2 & 0 \\
0 & \Abs{\phi_1}^2 + \Abs{\phi_2}^2
\end{bmatrix},
\end{dmath}
% (4)
and, hence, that
\cref{eqn:ProblemSet2Problem2:10} %(1)
can be written as
\begin{dmath}\label{eqn:ProblemSet2Problem2:50}
\LL = \trace{
   \lr{
      \partial \mu H^\dagger \partial^\mu H
   }
}
- m^2 \trace{
   \lr{
      H^\dagger H
   }
}
- \lambda \lr{
   \trace{
      H^\dagger H
   }
}^2
\end{dmath}
%(5)
where \( \trace{}\) denotes the matrix trace.
Show that now \cref{eqn:ProblemSet2Problem2:50} has \(SU(2)_L \times SU(2)_R\) symmetry,
acting on \( H \) as
\begin{dmath}\label{eqn:ProblemSet2Problem2:60}
H \rightarrow U_L H U_R^\dagger,
\end{dmath}
% (6)
where the action of \( U_R^\dagger \) on the right is pure convention (we could have taken \( U_R \) instead).
\( U_L \) and \( U_R \) are two sets of independent \(SU(2) \) transformations.
The \( L \) and \( R \) (left and right) names are self-evident in the way \cref{eqn:ProblemSet2Problem2:60} is written.
Show that under \(SU(2)_L \times SU(2)_R\), we have \( \delta H = i \omega_a^L \frac{\sigma^a}{2} H - i\omega_b^R H \frac{\sigma^b}{2}\).

Hint: clearly, the only thing you need to show is \(SU(2)_R\) invariance, as \(SU(2)_L\) was already shown.

\makesubproblem{}{qft:problemSet2:2f}
Show that the left and right \(SU(2) \) conserved currents can be written as
\begin{dmath}\label{eqn:ProblemSet2Problem2:80}
\begin{aligned}
j^{\mu,a}_L &= \frac{i}{2} \trace{
\lr{
   \partial_\mu H^\dagger \sigma^a H - H^\dagger \sigma^a \partial_\mu H
}
}  \\
j^{\mu,b}_R &= \frac{i}{2} \trace{
\lr{
   \partial_\mu H \sigma^b H^\dagger - H \sigma^b \partial_\mu H^\dagger
}
}
\end{aligned}
\end{dmath}
and that the corresponding generators \( \hatQ^{L,R}_a \) obey the commutation relations of two commuting angular momentum algebras.

Hint: notice that both currents are Hermitean and that the left is obtained from the right by interchanging \( H \) with \( H^\dagger\).
} % makeproblem

\makeanswer{qft:problemSet2:2}{
\makeSubAnswer{}{qft:problemSet2:2a}
Let's consider the \( SU(2)_L \) case first.  Noting that \( (\sigma^a)^\dagger = \sigma^a \), the transformed fields are
\begin{dmath}\label{eqn:ProblemSet2Problem2:100}
\begin{aligned}
\Phi' &= e^{i \Bsigma \cdot \Bomega/2} \Phi \\
{\Phi'}^\dagger &= \Phi^\dagger e^{-i \Bsigma \cdot \Bomega/2},
\end{aligned}
\end{dmath}
so \( {\Phi'}^\dagger \Phi' = \Phi^\dagger \Phi \), and
so \( \partial_\mu {\Phi'}^\dagger \partial^\mu \Phi' = \partial_\mu \Phi^\dagger \partial^\mu \Phi \).
This shows that the Lagrangian density is invariant under this transformation.

The variation of the field is
\begin{dmath}\label{eqn:ProblemSet2Problem2:120}
\delta \Phi
= \Phi' - \Phi
\approx \lr{ 1 + i \Bsigma \cdot \Bomega/2} \Phi - \Phi
=
\frac{i}{2} \Bsigma \cdot \Bomega \Phi,
\end{dmath}
so
\begin{dmath}\label{eqn:ProblemSet2Problem2:140}
\delta (\Phi^\dagger \Phi)
=
(\delta \Phi^\dagger) \Phi + \Phi^\dagger \delta \Phi
=
\frac{i}{2} \lr{
-\Phi^\dagger \Bsigma \cdot \Bomega \Phi
+ \Phi^\dagger \Bsigma \cdot \Bomega \Phi
}
=
0,
\end{dmath}
and
\begin{dmath}\label{eqn:ProblemSet2Problem2:160}
\delta (\partial_\mu \Phi^\dagger \partial^\mu \Phi)
=
\partial_\mu (\delta \Phi^\dagger) \partial^\mu \Phi
+
\partial_\mu \Phi^\dagger \partial^\mu (\delta \Phi)
=
\frac{i}{2}
\lr{
   - \partial_\mu \Phi^\dagger \Bsigma \cdot \Bomega \partial^\mu \Phi
   + \partial_\mu \Phi^\dagger \Bsigma \cdot \Bomega \partial^\mu \Phi
}
=
0,
\end{dmath}
so \( \delta \LL = 0 \).  To calculate the conserved current, we have to be slightly careful with the order of operations so that the matrix products are compatible
\begin{dmath}\label{eqn:ProblemSet2Problem2:180}
j^\mu_\Bomega
=
\PD{(\partial_\mu \Phi)}{\LL} \delta \Phi
+
\delta \Phi^\dagger
\PD{(\partial_\mu \Phi^\dagger)}{\LL}
=
\frac{i}{2}
\lr{
   \partial^\mu \Phi^\dagger (\Bsigma \cdot \Bomega) \Phi
   -
   \Phi^\dagger (\Bsigma \cdot \Bomega) \partial^\mu \Phi
},
\end{dmath}
or
\begin{dmath}\label{eqn:ProblemSet2Problem2:200}
j^{\mu a} =
\frac{i}{2}
\lr{
   \partial^\mu \Phi^\dagger \sigma^a \Phi
   -
   \Phi^\dagger \sigma^a \partial^\mu \Phi
},
\end{dmath}
where \( j^\mu_\Bomega = \omega_a j^{\mu a} \).

For the \( U(1) \) case we clearly have \( \LL' = \LL \).  The variation is
\begin{dmath}\label{eqn:ProblemSet2Problem2:220}
\delta \Phi
= \Phi' - \Phi
\approx (1 + i\alpha) \Phi - \Phi
=
i\alpha \Phi,
\end{dmath}
so
\begin{dmath}\label{eqn:ProblemSet2Problem2:240}
\delta (\Phi^\dagger \Phi)
=
(\delta \Phi^\dagger) \Phi
+
\Phi^\dagger (\delta \Phi)
=
i \alpha
\lr{
   -\Phi^\dagger \Phi
   +
   \Phi^\dagger \Phi
}
=
0,
\end{dmath}
and
\begin{dmath}\label{eqn:ProblemSet2Problem2:260}
\delta (\partial_\mu \Phi^\dagger \partial^\mu \Phi)
=
\partial_\mu (\delta \Phi^\dagger) \partial^\mu \Phi
+
\partial_\mu \Phi^\dagger \partial^\mu (\delta \Phi)
=
i \alpha
\lr{
   -\partial_\mu \Phi^\dagger \partial^\mu \Phi
   +
   \partial_\mu \Phi^\dagger \partial^\mu \Phi
}
=
0,
\end{dmath}
so \( \delta \LL = 0 \).
The conserved current, again being careful of the order, is
\begin{dmath}\label{eqn:ProblemSet2Problem2:280}
j^\mu_\alpha
=
\PD{(\partial_\mu \Phi)}{\LL} \delta \Phi
+
\delta \Phi^\dagger
\PD{(\partial_\mu \Phi^\dagger)}{\LL}
=
i \alpha
\lr{
   (\partial^\mu \Phi^\dagger) \Phi
   -
   \Phi^\dagger (\partial^\mu \Phi)
}.
\end{dmath}

\makeSubAnswer{}{qft:problemSet2:2b}
Let's work with the individual fields so the commutators can be computed more easily.  Expanding out the matrices, we have
\begin{dmath}\label{eqn:ProblemSet2Problem2:300}
Q^a
= \frac{i}{2} \int d^3 x
\lr{
   \partial^0 \Phi^\dagger \sigma^a \Phi
   -
   \Phi^\dagger \sigma^a \partial^0 \Phi
}
=
\frac{i}{2} \int d^3 x
\lr{
   \Pi^\dagger \sigma^a \Phi
   -
   \Phi^\dagger \sigma^a \Pi
}
=
\frac{i}{2} \int d^3 x
\lr{
   \pi^\dagger_r \sigma^a_{rs} \phi_s
   -
   \phi^\dagger_r \sigma^a_{rs} \pi_s
}.
\end{dmath}
To simplify the commutator expansion, assume that
\( r,s \) indexed functions are functions of \( \Bx \) and
\( m,n \) indexed functions are functions of \( \By \), for
\begin{dmath}\label{eqn:ProblemSet2Problem2:320}
\antisymmetric{Q^a}{Q^b}
=
-\frac{1}{4} \int d^3 x d^3 y
\sigma^a_{rs}
\sigma^b_{mn}
\antisymmetric
{
   \pi^\dagger_r \phi_s
   -
   \phi^\dagger_r \pi_s
}
{
   \pi^\dagger_m \phi_n
   -
   \phi^\dagger_m \pi_n
}
=
\frac{1}{4} \int d^3 x d^3 y
\sigma^a_{rs}
\sigma^b_{mn}
\lr{
   \antisymmetric
   {
      \pi^\dagger_r \phi_s
   }
   {
      \phi^\dagger_m \pi_n
   }
   +
   \antisymmetric
   {
      \phi^\dagger_r \pi_s
   }
   {
      \pi^\dagger_m \phi_n
   }
}
=
\frac{1}{4} \int d^3 x d^3 y
\sigma^a_{rs}
\sigma^b_{mn}
\lr{
      \pi^\dagger_r
      \phi^\dagger_m
      \phi_s
      \pi_n
   -
      \phi^\dagger_m
      \pi^\dagger_r
      \pi_n
      \phi_s
   +
      \phi^\dagger_r
      \pi^\dagger_m
      \pi_s
      \phi_n
   -
      \pi^\dagger_m
      \phi^\dagger_r
      \phi_n
      \pi_s
}
=
\frac{1}{4} \int d^3 x d^3 y
\sigma^a_{rs}
\sigma^b_{mn}
\lr{
   \lr{
      \phi^\dagger_m
      \pi^\dagger_r
+
      \antisymmetric{ \pi^\dagger_r }{ \phi^\dagger_m}
   }
      \phi_s
      \pi_n
   -
      \phi^\dagger_m
      \pi^\dagger_r
      \pi_n
      \phi_s
   +
   \lr{
         \pi^\dagger_m
         \phi^\dagger_r
      +  \antisymmetric{ \phi^\dagger_r}{ \pi^\dagger_m}
   }
      \pi_s
      \phi_n
   -
      \pi^\dagger_m
      \phi^\dagger_r
      \phi_n
      \pi_s
}
=
\frac{1}{4} \int d^3 x d^3 y
\sigma^a_{rs}
\sigma^b_{mn}
\lr{
     \phi^\dagger_m \pi^\dagger_r \antisymmetric{ \phi_s}{ \pi_n}
   + \antisymmetric{ \pi^\dagger_r }{ \phi^\dagger_m} \phi_s \pi_n
   + \pi^\dagger_m \phi^\dagger_r \antisymmetric{ \pi_s}{ \phi_n}
   + \antisymmetric{ \phi^\dagger_r}{ \pi^\dagger_m} \pi_s \phi_n
}.
\end{dmath}
Each of these commutators has a \( \delta(\Bx - \By) \) term, leaving
\begin{dmath}\label{eqn:ProblemSet2Problem2:340}
\antisymmetric{Q^a}{Q^b}
=
\frac{i}{4} \int d^3 x
\sigma^a_{rs}
\sigma^b_{mn}
\lr{
     \phi^\dagger_m \pi^\dagger_r \delta_{sn}
   - \delta_{rm} \phi_s \pi_n
   - \pi^\dagger_m \phi^\dagger_r \delta_{sn}
   + \delta_{rm} \pi_s \phi_n
}
=
\frac{i}{4} \int d^3 x
   \sigma^a_{rs}
\lr{
   \sigma^b_{ms}
   \lr{
        \phi^\dagger_m \pi^\dagger_r
      - \pi^\dagger_m \phi^\dagger_r
   }
+
   \sigma^b_{rn}
   \lr{
        \pi_s \phi_n
      - \phi_s \pi_n
   }
}
=
\frac{i}{4} \int d^3 x
\lr{
   (\phi^\dagger_m \sigma^b_{ms})
   (\pi^\dagger_r \sigma^a_{rs})
-
   (\pi^\dagger_m \sigma^b_{ms})
   (\phi^\dagger_r \sigma^a_{rs})
+
   (\sigma^a_{rs} \pi_s )
   (\sigma^b_{rn} \phi_n)
-
   (\sigma^a_{rs} \phi_s )
   (\sigma^b_{rn} \pi_n)
}
=
\frac{i}{4} \int d^3 x
\lr{
   \Phi^\dagger \sigma^b \sigma^a \Pi
-
   \Pi^\dagger \sigma^b \sigma^a \Phi
+
   \Pi^\dagger \sigma^a \sigma^b \Phi
-
   \Phi^\dagger \sigma^a \sigma^b \Pi
}
=
\frac{i}{4} \int d^3 x
\lr{
   \Pi^\dagger \antisymmetric{\sigma^a}{\sigma^b} \Phi
   -\Phi^\dagger \antisymmetric{\sigma^a}{\sigma^b} \Pi
}
=
\frac{i}{4} \int d^3 x
\lr{
   \Pi^\dagger \antisymmetric{\sigma^a}{\sigma^b} \Phi
   -\Phi^\dagger \antisymmetric{\sigma^a}{\sigma^b} \Pi
}
=
-\frac{1}{2} \int d^3 x
\epsilon^{a b c}
\lr{
   \Pi^\dagger \sigma^c \Phi
   -\Phi^\dagger \sigma^c \Pi
}
=
i \epsilon^{a b c} Q^c,
\end{dmath}
as desired.
\makeSubAnswer{}{qft:problemSet2:2c}
Let's consider the mass term first, which becomes
\begin{dmath}\label{eqn:ProblemSet2Problem2:360}
\Phi^\dagger \Phi
=
\phi_1^\dagger \phi_1
+
\phi_2^\dagger \phi_2
=
(\psi^1 - i \psi^2)
(\psi^1 + i \psi^2)
+
(\psi^3 - i \psi^4)
(\psi^3 + i \psi^4)
=
(\psi^1)^2
+
(\psi^2)^2
+
(\psi^3)^2
+
(\psi^4)^2
+
i (\psi^1 \psi^2 - \psi^2 \psi^1)
+
i (\psi^3 \psi^4 - \psi^4 \psi^3).
\end{dmath}
Since \( \Phi^\dagger \Phi \) is a real scalar in the original representation, the imaginary parts of this representation must also be zero (i.e. \( \psi^1, \psi^2 \) and \( \psi^3, \psi^4 \) each respectively commute).  This leaves
\begin{dmath}\label{eqn:ProblemSet2Problem2:380}
\Phi^\dagger \Phi
= \Psi^\T \Psi,
\end{dmath}
so \( b, c = 1 \).  For the derivative term, we have
\begin{dmath}\label{eqn:ProblemSet2Problem2:400}
\partial_\mu \Phi^\dagger \partial^\mu \Phi
=
\partial_\mu \phi_1^\dagger \partial^\mu \phi_1
+
\partial_\mu \phi_2^\dagger \partial^\mu \phi_2
=
\partial_\mu (\psi^1 - i \psi^2)
\partial^\mu (\psi^1 + i \psi^2)
+
\partial_\mu (\psi^3 - i \psi^4)
\partial^\mu (\psi^3 + i \psi^4)
=
\partial_\mu \psi^1
\partial^\mu \psi^1
+
\partial_\mu \psi^2
\partial^\mu \psi^2
+
\partial_\mu \psi^3
\partial^\mu \psi^3
+
\partial_\mu \psi^4
\partial^\mu \psi^4
+
i (\partial_\mu \psi^1 \partial^\mu \psi^2 - \partial_\mu \psi^2 \partial^\mu \psi^1)
+
i (\partial_\mu \psi^3 \partial^\mu \psi^4 - \partial_\mu \psi^4 \partial^\mu \psi^3).
=
\partial_\mu \Psi^\T \partial^\mu \Psi
+
i (\partial_\mu \psi^1 \partial^\mu \psi^2 - \partial^\mu \psi^2 \partial_\mu \psi^1)
+
i (\partial_\mu \psi^3 \partial^\mu \psi^4 - \partial^\mu \psi^4 \partial_\mu \psi^3),
\end{dmath}
where a matched raising and lowering operation has been performed on half the terms.  Because of the \( \psi^{1,2} \) and \( \psi^{3,4} \) commutation properties observed previously, the imaginary terms are killed, leaving
\begin{dmath}\label{eqn:ProblemSet2Problem2:420}
\partial_\mu \Phi^\dagger \partial^\mu \Phi
=
\partial_\mu \Psi^\T \partial^\mu \Psi,
\end{dmath}
so \( a = 1 \).

For the question of the \( U(1) \) symmetry, suppose that \( \Psi \rightarrow e^{i\alpha} \Psi \).  We then have
\begin{dmath}\label{eqn:ProblemSet2Problem2:440}
\delta \LL = 2 i \alpha \LL - 2 i \alpha \lambda \lr{ \Psi^\T \Psi }^2,
\end{dmath}
which does not have the required four-divergence form required for a conserved current, so there is no \( U(1) \) symmetry.
\makeSubAnswer{}{qft:problemSet2:2d}
We want to examine the transformation of \( \sigma^2 (U_L\Phi)^\conj \), which, to first order in \( \Bomega \) is
\begin{dmath}\label{eqn:ProblemSet2Problem2:460}
\sigma^2 (U_L\Phi)^\conj
\rightarrow
\sigma^2 U_L^\conj \Phi^\conj
\approx
\Phi^\conj - \frac{i}{2} \sigma^2 \omega_a (\sigma^a)^\conj \Phi^\conj
\end{dmath}
Because \( \sigma^1 = \PauliX, \sigma^3 = \PauliZ \) are real, and \( \sigma^2 \) is purely imaginary, we have \( (\sigma^1)^\conj = \sigma^1, (\sigma^3)^\conj = \sigma^3 \), and
\begin{equation}\label{eqn:ProblemSet2Problem2:480}
(\sigma^2)^\conj =
\lr{\PauliY}^\conj =
\begin{bmatrix}
0 & i \\
-i & 0 \\
\end{bmatrix} = -\sigma^2.
\end{equation}
Utilizing these conjugation relations, and the commutatation identities \( \sigma^i \sigma^j = -\sigma^j \sigma^i \) for \( i \ne j \), we have
\begin{dmath}\label{eqn:ProblemSet2Problem2:500}
\sigma^2 (U_L\Phi)^\conj
\rightarrow
\Phi^\conj - \frac{i}{2} \lr{
\omega_1 \sigma^2 (\sigma^1 )^\conj
+ \omega_2 \sigma^2 (\sigma^2 )^\conj
+ \omega_3 \sigma^2 (\sigma^3 )^\conj
}
\Phi^\conj
=
\Phi^\conj - \frac{i}{2} \lr{
\omega_1 \sigma^2 \sigma^1
- \omega_2 \sigma^2 \sigma^2
+ \omega_3 \sigma^2 \sigma^3
}
\Phi^\conj
=
\Phi^\conj - \frac{i}{2} \lr{
- \omega_1 \sigma^1 \sigma^2
- \omega_2 \sigma^2 \sigma^2
- \omega_3 \sigma^3 \sigma^2
}
\Phi^\conj
=
\Phi^\conj + \frac{i}{2} \lr{
  \omega_1 \sigma^1
+ \omega_2 \sigma^2
+ \omega_3 \sigma^3
}
\sigma^2
\Phi^\conj
=
U_L \sigma^2 \Phi^\conj.
\end{dmath}
Plugging into \( H = \inv{\sqrt{2}} (i\sigma^2\Phi^\conj,\Phi) \), we have
\begin{dmath}\label{eqn:ProblemSet2Problem2:560}
H
\rightarrow \inv{\sqrt{2}} (i\sigma^2(U_L \Phi)^\conj,U_L \Phi)
= \inv{\sqrt{2}} (U_L i\sigma^2\Phi^\conj,U_L \Phi)
= U_L H,
\end{dmath}
proving \cref{eqn:ProblemSet2Problem2:520} as desired.

Incidentally, \cref{eqn:ProblemSet2Problem2:500} shows that
\begin{dmath}\label{eqn:ProblemSet2Problem2:540}
\sigma^2 U_L^\conj
=
U_L \sigma^2,
\end{dmath}
the identity that was claimed to be important for future spinor theory work.

\makeSubAnswer{}{qft:problemSet2:2e}

\begin{dmath}\label{eqn:ProblemSet2Problem2:580}
H^\dagger H
=
\inv{2}
\begin{bmatrix}
\phi_2^\conj & \phi_1 \\
-\phi_1^\conj & \phi_2
\end{bmatrix}
\begin{bmatrix}
\phi_2 & -\phi_1 \\
\phi_1^\conj & \phi_2^\conj
\end{bmatrix}
=
%\begin{bmatrix}
%\phi_2^\conj \phi_2 + \phi_1 \phi_1^\conj
%\end{bmatrix}
\end{dmath}

TODO.
\makeSubAnswer{}{qft:problemSet2:2f}
TODO.
}

      %
% Copyright � 2018 Peeter Joot.  All Rights Reserved.
% Licenced as described in the file LICENSE under the root directory of this GIT repository.
%
\makeproblem{\( SU(2)_L \times SU(2)_R\), realized in the Wigner and Nambu-Goldstone modes.
}{qft:problemSet2:3}{
Consider now our Lagrangian
\cref{eqn:ProblemSet2Problem2:50}
and imagine that \( m^2 < 0\), for whatever reason (nobody knows, really), while \( \lambda\) is still positive. This now becomes the Higgs Lagrangian of the Standard Model.

\makesubproblem{}{qft:problemSet2:3a}
Show that the classical potential in
\cref{eqn:ProblemSet2Problem2:50}
now becomes:
\begin{dmath}\label{eqn:ProblemSet2Problem3:20}
V = -\Abs{m^2}
\trace{H^\dagger H}
+ \lambda
\lr{ \trace{H^\dagger H} }^2
= \lambda \lr{
   \Abs{\phi_1}^2
   +
   \Abs{\phi_2}^2
   -
   \frac{\Abs{m^2}}{2 \lambda}
}^2
+ \text{const}.
\end{dmath}
\makesubproblem{}{qft:problemSet2:3b}
Clearly, there are extrema of the potential when
\(
   \Abs{\phi_1}^2
   +
   \Abs{\phi_2}^2
= 0 \)
and when
\(
   \Abs{\phi_1}^2
   +
   \Abs{\phi_2}^2
=
   \frac{\Abs{m^2}}{2 \lambda}
 \)
The second one has, clearly, smaller energy density. To quantize the theory, we now have to choose which classical minimum to expand around. Show that, if we expand around
\(
   \Abs{\phi_1}^2
   +
   \Abs{\phi_2}^2
= 0 \)
, we will find that the \( \phi _{1,2} \) excitations are tachyons, even classically. This signals an instability, rather than a faster-than-light propagation and shows that we have chosen the wrong value of \( \Phi \) to build our quantum theory.
\makesubproblem{}{qft:problemSet2:3c}
Thus, consider the
\(
   \Abs{\phi_1}^2
   +
   \Abs{\phi_2}^2
=
   \frac{\Abs{m^2}}{2 \lambda}
 \)
minimum of \( V \). This is really a set of minima. In fact
the set parameterized by
\(
   \Abs{\phi_1}^2
   +
   \Abs{\phi_2}^2
=
   \text{const}
 \)
is also known as a three sphere (\(S^3\), embedded in a four-dimensional space parameterized by
\(\psi^{1\cdots4}\) - not the spacetime!). To build the quantum theory, we will choose a point on this three sphere (a.k.a. the ``vacuum manifold'' - the set of field values that minimize the potential). We will now study the small fluctuations around the chosen point and the spectrum of the theory in this vacuum. There is an infinite number of parameterizations that can be used to do this, but I will suggest one that makes the symmetries the clearest.
 Thus, use the \(H\)-representation and take
\begin{dmath}\label{eqn:ProblemSet2Problem3:40}
H(x) = \frac{\Abs{m}}{2\sqrt{\lambda}} ( 1 + h(x) ) e^{i \phi^a(x) \sigma^a }
\end{dmath}
The logic here is as follows. When \( h(x)\) and \( \phi^a(x) \) vanish (i.e. there are no excitations), the
parameterization
\cref{eqn:ProblemSet2Problem3:40}
is equivalent, by
\cref{eqn:ProblemSet2Problem2:40}
, to taking a specific point on the vacuum manifold,
i.e. the one where \( \phi_1 = 0 \) and \( \phi_2 = \Abs{m}/\sqrt{2\lambda} \).
The fields \( h(x) \) and \( \phi^a(x) \) parameterize the
fluctuations around this ground state (for sure, they can be mapped - the map is nonlinear - to the fluctuations of the fields
\( \phi_{1,2} \) around the chosen vacuum value for \( \phi_2\).
What you will do now is take the form
\cref{eqn:ProblemSet2Problem3:40}
, plug it into the Lagrangian
\cref{eqn:ProblemSet2Problem2:50}
with
\( m^2 = -\Abs{m^2}\),
and expand what you find to second order in the fields \( h(x) \) and \( \phi^a(x)\).
Show that the field \( h(x) \) has a mass and find an expression for it.
Show that the fields \( \phi^a(x) \) remain massless and that their Lagrangian (not just to quadratic order) only contains derivatives.

The latter point can be seen pretty simply by noting that \( H(x) \) from
\cref{eqn:ProblemSet2Problem3:40}
can be written as
\begin{dmath}\label{eqn:ProblemSet2Problem3:60}
H(x) = \frac{\Abs{m}}{\sqrt{ 2 \lambda } }\Omega(x) ( 1 + h(x) ),
\end{dmath}
with \( \Omega^\dagger \Omega = 1 \) and \( \det(\Omega(x)) = 1 \).
In this parameterization \( \Omega(x) \) fluctuations
  correspond to going around the vacuum manifold \( S^3 \), while the \( h(x) \) fluctuations are along the ``radial'' directions away from the minimum.
The latter cost energy, hence \( h \) is massive (the Higgs field!), while the \( \Omega(x) \) only cost energy if the x-dependence is nontrivial.
The \( \phi^a(x)\) (or \( \Omega(x) \)) are equivalent parameterizations of the Goldstone fields. What you found here is an example of a general story: if a theory has a continuous symmetry, which is not a symmetry of the ground state, there is a number of massless Goldstone (or Nambu-Goldstone) modes. For internal symmetries like the ones we are considering here, their number is equal to the number of broken generators.

In the Standard Model, \(h(x)\) is indeed the Higgs field. The fields \(\phi^a(x)\) actually become the longitudinal components of the W and Z-bosons (one usually says that they are ``eaten'', a manifestation of the Landau-Anderson-Higgs-Brout-Englert-Guralnik-Hagen-... mechanism).

\makesubproblem{}{qft:problemSet2:3d}
One question that was not discussed and remained a bit obscure is that of the unbroken part of the symmetry. The original Lagrangian has \( SU(2)_L \times SU(2)_R \) symmetry. The value of
\( H(x) \) in the vacuum, denoted by \( \expectation{H}\), is given by
\cref{eqn:ProblemSet2Problem3:40}
with \( h = \phi^a = 0 \) and is
 \( \expectation{H} \sim\)
unit matrix.
Show that, while  \( \expectation{H} \) is not invariant under \( SU(2)_L \times SU(2)_R \) for arbitrary \( SU(2)_L \) and \( SU(2)_R \) transformations, it is invariant under
\cref{eqn:ProblemSet2Problem2:60}
with
\( U_L = U_R\). Such \( SU(2)_L \times SU(2)_R \) transformations with \( U_L = U_R \) are called ``diagonal'' or ``vector'' \( SU(2)_V \) transformations.
These remain unbroken in the vacuum. In the electroweak theory, the third component of \( SU(2)_V \) is identified with electromagnetic \( U(1)\).
Show that the current associated with \( SU(2)_V \) transformations has the form:
\begin{dmath}\label{eqn:ProblemSet2Problem3:80}
j_\mu^{V,a} = \frac{i}{2} \trace{\lr{
\partial_\mu H^\dagger \antisymmetric{\sigma^a}{H}
+
\partial_\mu H \antisymmetric{\sigma^a}{H^\dagger}
}}
\end{dmath}
Show also that the other ``linear'' combination of \( SU(2)_L \) and \( SU(2)_R \),
\cref{eqn:ProblemSet2Problem2:60}
with \( U_R = U_L^\dagger \) corresponds to the current (not conserved!) usually called the ``axial current''
\begin{dmath}\label{eqn:ProblemSet2Problem3:120}
j_\mu^{A,a} = \frac{i}{2} \trace{
\partial_\mu H^\dagger \symmetric{\sigma^a}{H} - \partial_\mu H \symmetric{\sigma^a}{H^\dagger},
}
\end{dmath}
where \( \symmetric{A}{B} = AB + BA \) denotes the anticommutator.

\makesubproblem{}{qft:problemSet2:3e}
Show that to linear order in the fields \( h(x),\phi^a(x) \), the a-th axial current is simply
\begin{dmath}\label{eqn:ProblemSet2Problem3:100}
j^{A,a} \sim \expectation{H} \partial_\mu \phi^a,
\end{dmath}
and find the constant in front. Thus, when the quantum operator corresponding to
\cref{eqn:ProblemSet2Problem3:100} % (12)
acts on the vacuum, it creates a quantum of the Goldstone boson (times the momentum and the ``Goldstone boson decay constant'' which is really equal to  \( \expectation{H} \)).

Show also that, to leading nontrivial order in the fields, the conserved vector current \( j^{V,a} \) is
quadratic in the fields \( \phi^a\).

In QCD, the relation
\cref{eqn:ProblemSet2Problem3:100} % (12)
and the algebra of the currents \( j^{V,A} \) constitute the basis of an approach to
soft-pion physics (soft means low energy) known as ``current algebra''.

Here, we studied the Nambu-Goldstone mode. In the Wigner mode, when \( m^2 > 0\), there are no massless particles, as is easy to convince yourselves.
} % makeproblem

\makeanswer{qft:problemSet2:3}{
\makeSubAnswer{}{qft:problemSet2:3a}
To expand the potential note that
\begin{dmath}\label{eqn:ProblemSet2Problem3:140}
\trace{\lr{
H^\dagger H
}}
=
\inv{2}
\trace{\lr{
\begin{bmatrix}
-i \Phi^\T \sigma^2 \\
\Phi^\dagger
\end{bmatrix}
\begin{bmatrix}
i \sigma^2 \Phi^\conj & \Phi
\end{bmatrix}
}}
=
\inv{2}\lr{ \Phi^\T \Phi^\dagger + \Phi^\dagger \Phi }
=
\inv{2}\lr{
   \phi_1 \phi_1^\conj + \phi_2 \phi_2^\conj + \phi_1^\conj \phi_1 + \phi_2^\conj \phi^2
}
=
\Abs{\phi_1}^2 + \Abs{\phi_2}^2,
\end{dmath}
so we have
\begin{dmath}\label{eqn:ProblemSet2Problem3:160}
V = -\Abs{m}^2
\trace{\lr{
H^\dagger H
}}
+ \lambda\lr{
\trace{\lr{
H^\dagger H
}}}^2
=
-
\Abs{m}^2
\lr{
   \Abs{\phi_1}^2 + \Abs{\phi_2}^2
}
+ \lambda
\lr{
   \Abs{\phi_1}^2 + \Abs{\phi_2}^2
}^2
=
\lambda\lr{
   \lr{
      \Abs{\phi_1}^2 + \Abs{\phi_2}^2
   }^2
   -
   \frac{\Abs{m}^2 }{\lambda}
   \lr{
      \Abs{\phi_1}^2 + \Abs{\phi_2}^2
   }
}.
\end{dmath}
Completing the square gives
\begin{dmath}\label{eqn:ProblemSet2Problem3:180}
V
=
\lambda\lr{
      \Abs{\phi_1}^2 + \Abs{\phi_2}^2
   -
   \frac{\Abs{m}^2 }{2\lambda}
}^2
   -
\lambda
\lr{
   \frac{\Abs{m}^2 }{2\lambda}
}^2,
\end{dmath}
which proves the result and shows that the constant is \( - \frac{\Abs{m}^4 }{4\lambda} \).
\makeSubAnswer{}{qft:problemSet2:3b}
From \cref{eqn:ProblemSet2Problem3:160} the first order expansion, ignoring constant terms, around \( \Abs{\phi_1}^2 + \Abs{\phi_2}^2 = 0 \) is
\begin{equation}\label{eqn:ProblemSet2Problem3:360}
V = -\Abs{m^2} \lr{ \Abs{\phi_1}^2 + \Abs{\phi_2}^2 } = -\Abs{m^2} \Phi^\dagger \Phi.
\end{equation}
The Langrangian density, to first order, may be written in the compact form
\begin{dmath}\label{eqn:ProblemSet2Problem3:200}
\LL = \partial_\mu \Phi^\dagger \partial^\mu \Phi + \Abs{m}^2 \Phi^\dagger \Phi.
\end{dmath}
The equations of motion are
\begin{dmath}\label{eqn:ProblemSet2Problem3:220}
\begin{aligned}
\partial_\mu \partial^\mu \Phi &= \Abs{m}^2 \Phi \\
\partial_\mu \partial^\mu \Phi^\dagger &= \Abs{m}^2 \Phi^\dagger
\end{aligned},
\end{dmath}
or, \( \partial_\mu \partial^\mu \psi = \Abs{m}^2 \psi \) for any \( \psi \in \phi_1, \phi_2, \phi_1^\conj, \phi_2^\conj \).

Suppose that one of these wave functions has a Fourier transform representation
\begin{dmath}\label{eqn:ProblemSet2Problem3:240}
\psi(x) = \int \frac{d^4 p}{(2\pi)^4} e^{i p \cdot x} \tilde{\psi}.
\end{dmath}
Such a solution must satisfy the equations of motion
\begin{dmath}\label{eqn:ProblemSet2Problem3:260}
0
=
\lr{
   \partial_{tt} - \spacegrad^2 - \Abs{m^2}
}
\psi
=
\lr{
   \partial_{tt} - \spacegrad^2 - \Abs{m^2}
}
\int \frac{d^4 p}{(2\pi)^4} e^{i \omega t - i \Bp \cdot \Bx} \tilde{\psi}.
=
\int \frac{d^4 p}{(2\pi)^4}
\lr{
   (i\omega)^2 - (-i \Bp)^2 -\Abs{m}^2
}
e^{i \omega t - i \Bp \cdot \Bx} \tilde{\psi},
\end{dmath}
so
\begin{dmath}\label{eqn:ProblemSet2Problem3:280}
0 = -\omega^2 + \Bp^2 - \Abs{m}^2,
\end{dmath}
or
\begin{dmath}\label{eqn:ProblemSet2Problem3:300}
\omega = \sqrt{\Bp^2 - \Abs{m}^2}.
\end{dmath}
Any \( \Abs{\Bp} < \Abs{m} \) results in an imaginary angular frequency.  For example, at \( \Bp = 0 \), we have
\begin{dmath}\label{eqn:ProblemSet2Problem3:320}
\omega = \pm i \Abs{m}.
\end{dmath}
In particular
\begin{dmath}\label{eqn:ProblemSet2Problem3:340}
p_0 x^0
=
\omega t
= \pm i \Abs{m} t
= \pm \Abs{m} ( i t ).
\end{dmath}
We see that the angular momentum constraint on the system \cref{eqn:ProblemSet2Problem3:280} results in the imaginary time that is characteristic of tachonic solutions.

\makeSubAnswer{}{qft:problemSet2:3c}
It seems reasonable that we can assume that \( h(x) \) and \( \phi^a(x) \) in
\cref{eqn:ProblemSet2Problem3:40} are all real valued scalar (non-matrix) functions.  That is \( h(x) \) has the role of radial extension or compression of the field magnitude, and the exponential is of the form \( e^{i \Bsigma \cdot \Bphi(x) } \), a matrix valued rotation operator, where \( \Bphi = (\phi^1, \phi^2, \phi^3)\).  Given that assumption, \( H^\dagger H \) can be computed with relative ease, and has only radial dependence
\begin{dmath}\label{eqn:ProblemSet2Problem3:380}
\trace{\lr{H^\dagger H}}
=
\frac{\Abs{m}^2}{4 \lambda} (1 + h(x))^2 \trace{\lr{ e^{-i \Bsigma \cdot \Bphi} e^{i \Bsigma \cdot \Bphi} }}
=
\frac{\Abs{m}^2}{4 \lambda} (1 + h(x))^2 \trace{\BOne}
=
\frac{\Abs{m}^2}{2 \lambda} (1 + h)^2.
\end{dmath}
ProblemSet2Problem3.tex
For the derivative quadratic form, it is expediant to use the form \cref{eqn:ProblemSet2Problem3:60}, which gives
\begin{dmath}\label{eqn:ProblemSet2Problem3:400}
\partial_\mu H^\dagger \partial^\mu H
=
\frac{\Abs{m}^2}{4 \lambda}
\lr{
   \partial_\mu h \Omega^\dagger
   + (1 + h) \partial_\mu \Omega^\dagger
}
\lr{
   \partial^\mu h \Omega
   + (1 + h)
\partial^\mu \Omega
}
=
\frac{\Abs{m}^2}{4 \lambda}
\lr{
   \partial_\mu h \Omega^\dagger \partial^\mu h \Omega
   + (1 + h)
      \lr{
         \partial_\mu h
         \Omega^\dagger (\partial^\mu \Omega)
       +
         \partial^\mu h
         (\partial_\mu \Omega^\dagger) \Omega
      }
   + (1 + h)^2 \partial_\mu \Omega^\dagger \partial^\mu \Omega
}
\end{dmath}
where we have made the usual assumptions that the independent fields \((h, \Omega)\) commute.
Because \( \Omega^\dagger \Omega = 1 \), we have
\begin{dmath}\label{eqn:ProblemSet2Problem3:480}
\partial_\mu h
\Omega^\dagger (\partial^\mu \Omega)
 +
\partial^\mu h
(\partial_\mu \Omega^\dagger) \Omega
=
\partial_\mu h
\lr{
   \Omega^\dagger (\partial^\mu \Omega)
    +
   (\partial^\mu \Omega^\dagger) \Omega
}
=
\partial_\mu h
\lr{
   \partial^\mu (\Omega^\dagger \Omega) - (\partial^\mu \Omega^\dagger) \Omega
    +
   (\partial^\mu \Omega^\dagger) \Omega
}
=
   \partial^\mu (1)
= 0.
\end{dmath}
All the cross terms with both \( h \) and \( \Omega \) derivatives are zero (to all orders, not just quadratric).

Taking traces (and using cyclic permutation of the matrices in the trace operations),
the Lagrangian density is now determined to quadratic order
\begin{dmath}\label{eqn:ProblemSet2Problem3:500}
\LL =
\frac{\Abs{m}^2}{2 \lambda}
   \partial_\mu h \partial^\mu h
+
\frac{\Abs{m}^2}{4 \lambda}
   \trace{\lr{
      \partial_\mu \Omega^\dagger \partial^\mu \Omega
   }}
+ \Abs{m}^2
\frac{\Abs{m}^2}{2 \lambda} \lr{ 1 + h }^2
- \lambda
\lr{\frac{\Abs{m}^2}{2 \lambda}}^2
\lr{ 1 + h }^4.
\end{dmath}
Observe that the Lagrangian density can be split into two independent parts, one for the radial field \( h \), and another for the rotation field \( \Omega \).  Rescaling to drop the common constant factor \( \Abs{m}^2/2\lambda \), the radial Lagrangian is
\begin{dmath}\label{eqn:ProblemSet2Problem3:560}
\LL_h
=
\partial_\mu h \partial^\mu h
+ \Abs{m}^2
\lr{ 1 + h }^2
-
\frac{\Abs{m}^2}{2}
\lr{ 1 + h }^4
=
\partial_\mu h \partial^\mu h
-\frac{\Abs{m}^2}{2}
\lr{
   \lr{ 1 + h }^4
   -2 \lr{ 1 + h }^2
}
=
\partial_\mu h \partial^\mu h
-\frac{\Abs{m}^2}{2}
\lr{
   \lr{ 1 + h }^2 - 1
}^2
+ \cancel{\text{const.}}
=
\partial_\mu h \partial^\mu h
-\frac{\Abs{m}^2}{2}
\lr{ 2 h + h^2
}^2
=
\partial_\mu h \partial^\mu h
- \frac{\Abs{m}^2}{2} h^2
\lr{ 2 + h }^2
=
\partial_\mu h \partial^\mu h
- 2 \Abs{m}^2 h^2
+ O(h^3).
\end{dmath}
This shows that the mass of the \( h \) field is \( \sqrt{2} \Abs{m} \).

The only remaining task is to express the Lagrangian density for \( \phi^a \) in terms of those field instead of \( \Omega \).  To evaluate those derivatives, we can utilize a first order Taylor expansion

\begin{dmath}\label{eqn:ProblemSet2Problem3:580}
\partial_\mu \Omega
=
\partial_\mu \lr{ \BOne + i \Bsigma \cdot \Bphi }
=
i \Bsigma \cdot \partial_\mu \Bphi,
\end{dmath}
so the rotation Lagrangian density is
\begin{dmath}\label{eqn:ProblemSet2Problem3:600}
\LL_\Bphi =
\inv{2} \trace{\lr{
(-i \Bsigma \cdot \partial_\mu \Bphi)
(i \Bsigma \cdot \partial^\mu \Bphi)
}}
=
(\partial_\mu \Bphi) \cdot (\partial^\mu \Bphi)
=
(\partial_\mu \phi^a) (\partial^\mu \phi^a),
\end{dmath}
where we use the fact that \( \trace{\lr{(\Bsigma \cdot \Bx)(\Bsigma \cdot \By)}} = 2 \Bx \cdot \By \).

The full Lagrangian density, to quadratic order, is
\boxedEquation{eqn:ProblemSet2Problem3:620}{
\LL
= \LL_h + \LL_\Bphi
=
\partial_\mu h \partial^\mu h
- 2 \Abs{m}^2 h^2
+
\partial_\mu \phi^a \partial^\mu \phi^a.
}

\makeSubAnswer{}{qft:problemSet2:3d}
TODO.
\makeSubAnswer{}{qft:problemSet2:3e}
TODO.
}

      %
% Copyright � 2018 Peeter Joot.  All Rights Reserved.
% Licenced as described in the file LICENSE under the root directory of this GIT repository.
%
\makeproblem{Playing with the non-relativistic limit}{qft:problemSet2:4}{
Consider a real scalar relativistic field theory of mass m with \( \lambda \phi^4 \) interaction. Let there be \( N \) particles of momenta labeled by \( p_1,\cdots, p_N\), whose energies are such that they are insufficient to create any new particles. Nevertheless, the particles can scatter and exchange momenta. In what follows you will study this N-particle nonrelativistic limit in some detail.
\makesubproblem{}{qft:problemSet2:4a}
Write down the Hamiltonian of the field theory, including the interaction term, restricted to the N-particle sector of Hilbert space. (Use the creation and annihilation operator representation, i.e. write the result as sums of products of creation and annihilation operators of particles of various momenta.)
\makesubproblem{}{qft:problemSet2:4b}
Does the resulting Hamiltonian preserve particle number? Is there an associated symmetry? What is the operator that generates it?
\makesubproblem{}{qft:problemSet2:4c}
Consider now the interaction term in your reduced (to the N-particle sector of Hilbert space) Hamiltonian. How does a typical interaction term (for given configurations of momenta) act on an N-particle state? What kinds of scattering processes does it describe?
\makesubproblem{}{qft:problemSet2:4d}
What do you think is the potential, in x-space, that allows the various particles to scatter and exchange momentum? How would you describe the resulting nonrelativistic quantum system to friends who never took QFT but are well-versed in quantum mechanics?
} % makeproblem

\makeanswer{qft:problemSet2:4}{
\makeSubAnswer{}{qft:problemSet2:4a}
The Lagrangian density of a massive scalar field with a \( \lambda \phi^4 \) interaction has the form
\begin{dmath}\label{eqn:ProblemSet2Problem4:20}
\LL = \inv{2} \partial_\mu \phi \partial^\mu \phi - \inv{2} m^2 \phi^2 - \lambda \phi^4.
\end{dmath}
The corresponding Hamiltonian is
\begin{dmath}\label{eqn:ProblemSet2Problem4:40}
H = \inv{2} \int d^3x \lr{ \pi^2 + \frac{m^2}{2} (\spacegrad \phi)^2 + m^2 \phi^2 } + \lambda \int d^3 x \phi^4.
\end{dmath}
In terms of creation and annihilation operators, we know the form of the non-interaction portion of the Hamiltonian, which in normal order is
\begin{dmath}\label{eqn:ProblemSet2Problem4:60}
H_0 = \int \frac{d^3 p}{(2 \pi)^3} \omega_\Bp a_\Bp^\dagger a_\Bp,
\end{dmath}
but the interaction contribution is much messier
\begin{dmath}\label{eqn:ProblemSet2Problem4:80}
H_{\text{int}}
=
\lambda \int d^3 x \frac{ d^3 p d^3 k d^3 q d^3 s}{4 (2 \pi)^{12} \sqrt{
\omega_\Bp  \omega_\Bk  \omega_\Bq  \omega_\Bs
} }
\lr{ a_\Bp e^{-i p \cdot x} + a_\Bp e^{i p \cdot x} }
\lr{ a_\Bk e^{-i k \cdot x} + a_\Bk e^{i k \cdot x} }
\lr{ a_\Bq e^{-i q \cdot x} + a_\Bq e^{i q \cdot x} }
\lr{ a_\Bs e^{-i s \cdot x} + a_\Bs e^{i s \cdot x} }
=
\lambda \int d^3 x \frac{ d^3 p d^3 k d^3 q d^3 s}{4 (2 \pi)^{12} \sqrt{
\omega_\Bp  \omega_\Bk  \omega_\Bq  \omega_\Bs
} }
\lr{ a_\Bp e^{-i \omega_\Bp t + i \Bp \cdot \Bx} + a_\Bp e^{i \omega_\Bp t - i \Bp \cdot \Bx} }
\lr{ a_\Bk e^{-i \omega_\Bk t + i \Bk \cdot \Bx} + a_\Bk e^{i \omega_\Bk t - i \Bk \cdot \Bx} }
\lr{ a_\Bq e^{-i \omega_\Bq t + i \Bq \cdot \Bx} + a_\Bq e^{i \omega_\Bq t - i \Bq \cdot \Bx} }
\lr{ a_\Bs e^{-i \omega_\Bs t + i \Bs \cdot \Bx} + a_\Bs e^{i \omega_\Bs t - i \Bs \cdot \Bx} }.
=
\lambda \int \frac{ d^3 p d^3 k d^3 q d^3 s}{4 (2 \pi)^{9} \sqrt{
\omega_\Bp  \omega_\Bk  \omega_\Bq  \omega_\Bs
} }
\lr{
   a_\Bp a_\Bk a_\Bq a_\Bs e^{-i (\omega_\Bp + \omega_\Bk + \omega_\Bq + \omega_\Bs)t} \delta^3( \Bp + \Bk + \Bq + \Bs )
   +
   a_\Bp a_\Bk a_\Bq a_\Bs^\dagger e^{-i (\omega_\Bp + \omega_\Bk + \omega_\Bq - \omega_\Bs)t} \delta^3( \Bp + \Bk + \Bq - \Bs )
   +
   \cdots
   +
   a_\Bp^\dagger a_\Bk^\dagger a_\Bq^\dagger a_\Bs^\dagger e^{-i (-\omega_\Bp - \omega_\Bk - \omega_\Bq - \omega_\Bs)t} \delta^3( -\Bp - \Bk - \Bq - \Bs )
}
=
\lambda \int \frac{ d^3 p d^3 k d^3 q }{4 (2 \pi)^{9}
}
\lr{
   \inv{\sqrt{
      \omega_\Bp  \omega_\Bk  \omega_\Bq  \omega_{-\Bp - \Bk - \Bq}
   }}
   a_\Bp a_\Bk a_\Bq a_{-\Bp -\Bk - \Bq} e^{-i (\omega_\Bp + \omega_\Bk + \omega_\Bq + \omega_{-\Bp -\Bk -\Bq})t}
+
   \inv{\sqrt{
      \omega_\Bp  \omega_\Bk  \omega_\Bq  \omega_{\Bp + \Bk + \Bq}
   }}
   a_\Bp a_\Bk a_\Bq a_{\Bp + \Bk + \Bq}^\dagger e^{-i (\omega_\Bp + \omega_\Bk + \omega_\Bq - \omega_{\Bp + \Bk + \Bq})t}
+
   \cdots
+
   \inv{\sqrt{
      \omega_\Bp  \omega_\Bk  \omega_\Bq  \omega_{-\Bp - \Bk - \Bq}
   }}
   a_\Bp^\dagger a_\Bk^\dagger a_\Bq^\dagger a_{-\Bp -\Bk -\Bq}^\dagger e^{-i (-\omega_\Bp - \omega_\Bk - \omega_\Bq - \omega_{-\Bp -\Bk -\Bq})t}
}.
\end{dmath}
Assuming we can normal order these terms as in \( H_0 \), we can rewrite the interaction as
\begin{dmath}\label{eqn:ProblemSet2Problem4:100}
H_{\text{int}}
=
\lambda \int \frac{ d^3 p d^3 k d^3 q }{4 (2 \pi)^{9} }
\lr{
   \binom{4}{0}
      \inv{\sqrt{
         \omega_\Bp  \omega_\Bk  \omega_\Bq  \omega_{-\Bp - \Bk - \Bq}
      }}
      a_\Bp a_\Bk a_\Bq a_{-\Bp -\Bk - \Bq} e^{-i (\omega_\Bp + \omega_\Bk + \omega_\Bq + \omega_{-\Bp -\Bk -\Bq})t}
   +
   \binom{4}{1}
      \inv{\sqrt{
         \omega_\Bp  \omega_\Bk  \omega_\Bq  \omega_{\Bp - \Bk - \Bq}
      }}
      a_\Bp^\dagger a_\Bk a_\Bq a_{\Bp - \Bk - \Bq} e^{-i (-\omega_\Bp + \omega_\Bk + \omega_\Bq + \omega_{\Bp - \Bk - \Bq})t}
   +
   \binom{4}{2}
      \inv{\sqrt{
         \omega_\Bp  \omega_\Bk  \omega_\Bq  \omega_{\Bp + \Bk - \Bq}
      }}
      a_\Bp^\dagger a_\Bk^\dagger a_\Bq a_{\Bp + \Bk - \Bq} e^{-i (-\omega_\Bp - \omega_\Bk + \omega_\Bq + \omega_{\Bp + \Bk - \Bq})t}
   +
   \binom{4}{3}
      \inv{\sqrt{
         \omega_\Bp  \omega_\Bk  \omega_\Bq  \omega_{\Bp + \Bk _ \Bq}
      }}
      a_\Bp^\dagger a_\Bk^\dagger a_\Bq^\dagger a_{\Bp + \Bk + \Bq} e^{-i (-\omega_\Bp - \omega_\Bk - \omega_\Bq + \omega_{\Bp + \Bk + \Bq})t}
   +
   \binom{4}{4}
      \inv{\sqrt{
         \omega_\Bp  \omega_\Bk  \omega_\Bq  \omega_{-\Bp - \Bk - \Bq}
      }}
      a_\Bp^\dagger a_\Bk^\dagger a_\Bq^\dagger a_{-\Bp -\Bk -\Bq}^\dagger e^{-i (-\omega_\Bp - \omega_\Bk - \omega_\Bq - \omega_{-\Bp -\Bk -\Bq})t}
}
\end{dmath}
If we restrict the allowed momenta to the discrete set \( \Bp \in \setlr{ \Bp_1, \Bp_2, \cdots \Bp_N} \), the total Hamiltonian including the interaction term
takes the form
\begin{dmath}\label{eqn:ProblemSet2Problem4:120}
\text{\(:H:\)} =
\sum_{i = 1}^N \omega_{\Bp_i} a_{\Bp_i}^\dagger a_{\Bp_i}
+
\frac{
\lambda
}{4 }
\sum_{j,m,n = 1}^N
\lr{
   \binom{4}{0}
      \inv{\sqrt{
         \omega_{\Bp_j}  \omega_{\Bp_m}  \omega_{\Bp_n}  \omega_{-{\Bp_j} - {\Bp_m} - {\Bp_n}}
      }}
      a_{\Bp_j} a_{\Bp_m} a_{\Bp_n} a_{-\Bp -\Bk - \Bq} e^{-i (\omega_{\Bp_j} + \omega_{\Bp_m} + \omega_{\Bp_n} + \omega_{-{\Bp_j} -{\Bp_m} -{\Bp_n}})t}
   +
   \binom{4}{1}
      \inv{\sqrt{
         \omega_{\Bp_j}  \omega_{\Bp_m}  \omega_{\Bp_n}  \omega_{{\Bp_j} - {\Bp_m} - {\Bp_n}}
      }}
      a_{\Bp_j}^\dagger a_{\Bp_m} a_{\Bp_n} a_{{\Bp_j} - {\Bp_m} - {\Bp_n}} e^{-i (-\omega_{\Bp_j} + \omega_{\Bp_m} + \omega_{\Bp_n} + \omega_{{\Bp_j} - {\Bp_m} - {\Bp_n}})t}
   +
   \binom{4}{2}
      \inv{\sqrt{
         \omega_{\Bp_j}  \omega_{\Bp_m}  \omega_{\Bp_n}  \omega_{{\Bp_j} + {\Bp_m} - {\Bp_n}}
      }}
      a_{\Bp_j}^\dagger a_{\Bp_m}^\dagger a_{\Bp_n} a_{{\Bp_j} + {\Bp_m} - {\Bp_n}} e^{-i (-\omega_{\Bp_j} - \omega_{\Bp_m} + \omega_{\Bp_n} + \omega_{{\Bp_j} + {\Bp_m} - {\Bp_n}})t}
   +
   \binom{4}{3}
      \inv{\sqrt{
         \omega_{\Bp_j}  \omega_{\Bp_m}  \omega_{\Bp_n}  \omega_{{\Bp_j} + {\Bp_m} - {\Bp_n}}
      }}
      a_{\Bp_j}^\dagger a_{\Bp_m}^\dagger a_{\Bp_n}^\dagger a_{{\Bp_j} + {\Bp_m} + {\Bp_n}} e^{-i (-\omega_{\Bp_j} - \omega_{\Bp_m} - \omega_{\Bp_n} + \omega_{{\Bp_j} + {\Bp_m} + {\Bp_n}})t}
   +
   \binom{4}{4}
      \inv{\sqrt{
         \omega_{\Bp_j}  \omega_{\Bp_m}  \omega_{\Bp_n}  \omega_{-{\Bp_j} - {\Bp_m} - {\Bp_n}}
      }}
      a_{\Bp_j}^\dagger a_{\Bp_m}^\dagger a_{\Bp_n}^\dagger a_{-{\Bp_j} -{\Bp_m} -{\Bp_n}}^\dagger e^{-i (-\omega_{\Bp_j} - \omega_{\Bp_m} - \omega_{\Bp_n} - \omega_{-{\Bp_j} -{\Bp_m} -{\Bp_n}})t}
}.
\end{dmath}
When we did the same sort of calculation for \( (\spacegrad \phi)^2 + m^2 \phi^2 \) all the time dependent terms cancelled nicely, but that isn't obviously the case here.
However, we haven't used the non-relativistic (low energy) constraint.  That constraint can be expressed as \( \Bp^2 \ll m^2 \), in which case \( \omega_\Bp = \sqrt{ \Bp^2 + m^2 } \sim m \), the mass of each of the particles.  Incorporating that into our N-particle Hamiltonian, we have
\begin{dmath}\label{eqn:ProblemSet2Problem4:140}
\text{\(:H:\)} =
\sum_{i = 1}^N \omega_{\Bp_i} a_{\Bp_i}^\dagger a_{\Bp_i}
+
\frac{
\lambda
}{4 m^2 }
\sum_{j,m,n = 1}^N
\lr{
   \binom{4}{0}
      a_{\Bp_j} a_{\Bp_m} a_{\Bp_n} a_{-\Bp -\Bk - \Bq} e^{- 4 i m t}
   +
   \binom{4}{1}
      a_{\Bp_j}^\dagger a_{\Bp_m} a_{\Bp_n} a_{{\Bp_j} - {\Bp_m} - {\Bp_n}} e^{-3 i m t}
   +
   \binom{4}{2}
      a_{\Bp_j}^\dagger a_{\Bp_m}^\dagger a_{\Bp_n} a_{{\Bp_j} + {\Bp_m} - {\Bp_n}}
   +
   \binom{4}{3}
      a_{\Bp_j}^\dagger a_{\Bp_m}^\dagger a_{\Bp_n}^\dagger a_{{\Bp_j} + {\Bp_m} + {\Bp_n}} e^{ 3 i m t }
   +
   \binom{4}{4}
      a_{\Bp_j}^\dagger a_{\Bp_m}^\dagger a_{\Bp_n}^\dagger a_{-{\Bp_j} -{\Bp_m} -{\Bp_n}}^\dagger e^{4 i m t}
}.
\end{dmath}
Presuming there's a good argument to kill off the time dependent terms, the N-sector Hamiltonian is reduced to just
\begin{dmath}\label{eqn:ProblemSet2Problem4:160}
\text{\(:H:\)} =
\sum_{i = 1}^N \omega_{\Bp_i} a_{\Bp_i}^\dagger a_{\Bp_i}
+
\frac{
3 \lambda
}{2 m^2 }
\sum_{j,m,n = 1}^N
      a_{\Bp_j}^\dagger a_{\Bp_m}^\dagger a_{\Bp_n} a_{{\Bp_j} + {\Bp_m} - {\Bp_n}}.
\end{dmath}

The only annoying aspect to this Hamiltonian is the \( a_{{\Bp_j} + {\Bp_m} - {\Bp_n}} \) operator in the interaction term, which is not clear to me how to interpret.  That seems to imply that it is possible to create particles with linear combinations of momentum that may not be in the original set of \( N \) particle momenta.  I think that this can be further fudged by invoking the non-relativisitic constraint again, and decreeing that each of the uniquely indexed creation and anhillation operators are distinguishable only by index, so we can write the N-particle non-relativisitic sector Hamiltonian as
\begin{dmath}\label{eqn:ProblemSet2Problem4:170}
\text{\(:H:\)} =
\sum_{i = 1}^N \omega_{\Bp_i}
a_{i}^\dagger a_{i}
+
\frac{
3 \lambda
}{2 m^2 }
\sum_{r,s,t,u = 1}^N
      a_{r}^\dagger a_{s}^\dagger a_{t} a_{u}.
\end{dmath}

\makeSubAnswer{}{qft:problemSet2:4b}
Yes, with the number of creation and anhillation operators matched, this Hamiltonian preserves particle number (one particle is created for each particle destroyed).
The symmetry appears to be one associated with a permutation operation in the interaction.

%The hint seems to suggest that particle number is conserved, even though the interaction term does not have the structure of a number operator.  I have to conclude (too late for problem set submission) that I don't really understand what is meant by preservation of particle number in this case, and will need to see the problem set solution or discuss this in office hours to understand what is being asked for.
\makeSubAnswer{}{qft:problemSet2:4c}
Continued freehand, time allowing.
%If we designate an N-particle momentum state by
%\begin{dmath}\label{eqn:ProblemSet2Problem4:180}
%\ket{\Bp_1, \Bp_2, \cdots \Bp_N} =
%a_{\Bp_1}^\dagger
%a_{\Bp_2}^\dagger
%\cdots
%a_{\Bp_N}^\dagger \ket{0, 0, \cdots, 0},
%\end{dmath}
%then the interaction terms action on such a state is
%\begin{dmath}\label{eqn:ProblemSet2Problem4:200}
%      a_{\Bp_j}^\dagger a_{\Bp_m}^\dagger a_{\Bp_n} a_{{\Bp_j} + {\Bp_m} - {\Bp_n}}
%\ket{\Bp_1, \Bp_2, \cdots \Bp_N}.
%\end{dmath}
%I'm not sure if this is meaningful, or how to interpret it, and think that I'm going to have to get explanation about what this abstraction means.  I'm also not sure what is meant by the question ``What kinds of scattering processes does it describe.''
\makeSubAnswer{}{qft:problemSet2:4d}
Also continued freehand, time allowing.
%Not attempted.
}


\ifthenelse{\boolean{redacted}}%
{%
\relax%
}%
{%
   \chapter{Problem Set 3.}

      %
% Copyright � 2018 Peeter Joot.  All Rights Reserved.
% Licenced as described in the file LICENSE under the root directory of this GIT repository.
%
\makeproblem{Interaction energy between static external charges}{qft:problemSet3:1}{
\makesubproblem{}{qft:problemSet3:1a}
 Calculate the vacuum expectation value of the time ordered exponential 
\begin{equation}
\label{one}
\langle 0 \vert T e^{\;i \int d^4 x\;  g \; j(x)\;  \phi(x)} \vert 0 \rangle
\end{equation}
for the case of a massive free real scalar field. Here, $g$ is a coupling constant, which we shall call the ``Yukawa coupling". Show, e.g. using Wick's theorem, that the answer is
 \begin{equation}
\label{two}
  e^{\;-{g^2\over 2} \int d^4 x d^4 y\;  j(x)  D_F(x-y) j(y) } ~,
   \end{equation}
   which is really the exponential of the second order term and $D_F$ is the Feynman propagator.
\makesubproblem{}{qft:problemSet3:1b}
Consider the case where $j(t,\Bx) = \theta(T - t) \theta(T+t) \left( \delta^{(3)}(\Bx) - \delta^{(3)}(\Bx - \BR)\right)$. This source term represents two external opposite ``charges"\footnote{In other words, classical particles linearly coupled to $\phi$ (if $\phi$ was the electrostatic potential $A^0$, this would really be the electromagnetic charge.) For a discussion of whether an interaction like you will study can arise from a realistic QFT, see comment in 2. below.} a distance $R = |\BR|$ apart, created at $t=-T$ and existing for time $2 T$.  Show that, in the limit $T \gg R \gg 1/m$, Eq.~(\ref{two}) {\it is proportional to}:
   \begin{equation}
   \label{three}
   e^{ - i 2T V(R)},
   \end{equation}
   where $V(R)$ is the Yukawa potential (\citep{wiki:yukawaPotential}). 
   
    {\flushleft{Hint:}} {\small Recall that $\lim\limits_{T \rightarrow \infty} \int\limits_{-T}^T d x e^{i p x} = 2 \pi \delta(p)$ as well as the usual relation $(2 \pi \delta(p))^2 = 2 \pi \delta(p) 2 T$.}
    
      {\small  {\flushleft{T}he} result (\ref{three}) means that ``{\it two static sources of scalar field a distance $R$ apart interact via the Yukawa potential}." This is because (\ref{three}) is  the evolution operator (it is  $ \sim e^{- i H t}$, for  $t=2T$)  of the field theory 
   in the presence of the static external sources (or, more appropriately, (\ref{three}) is the contribution to the evolution operator that has to do with the interaction between the sources induced by the field). Thus, it is natural to call the quantity multiplying $- i 2 T$  and depending on $R$, the interaction potential $V(R)$ between the sources.}

 Do opposite-sign ``charges" attract or repel? How about same-sign? 
 
 {\small  Notice that when the ``charges" are also considered as part of a QFT and, therefore, $j(x)$  in (\ref{one}) is replaced by an appropriate QFT expression, one finds more interesting results. Namely,  the Yukawa interaction between two fermions is always attractive---whether it is between two particles, two anti-particles, or between a particle and an anti-particle. The way to establish this, as well an alternative derivation of the expression for $V(R)$ you found in (\ref{three}), is to start  from the scattering of (anti)fermions via scalar exchange and then take the nonrelativistic limit. A comparison with quantum-mechanical Born scattering yields then an expression for $V(R)$. 
 
 This result quoted above is of great interest in nuclear physics, where single-pion exchange   operates via $V(R)$, and turns out to be attractive between nucleons and between nucleons and anti-nucleons.  }
   
\makesubproblem{}{qft:problemSet3:1c}
What do you think is the significance of the various limits $T \gg R \gg 1/m$? Also, what is the  meaning of the factors you omitted upon going from (\ref{two}) to (\ref{three})?
} % makeproblem

\makeanswer{qft:problemSet3:1}{
\makeSubAnswer{}{qft:problemSet3:1a}
\begin{dmath}\label{eqn:ProblemSet3Problem1:20}
\bra{0} T e^{i \int d^4 x  g  j(x)  \phi(x)} \ket{0}
=
\bra{0} T (1) \ket{0}
+
i g
\bra{0} T \int d^4 x j(x) \phi(x) \ket{0}
+
- \frac{g^2}{2}
\int d^4 x d^4 y 
\bra{0} T 
\contraction{ j(x) }{\phi}{(x) j(y) }{\phi}
j(x) \phi(x) j(y) \phi(y) 
\ket{0}
+ \cdots
\end{dmath}
Using Wick's theorem, the first order term is zero (odd number of creation and anhillation operators), so to first order, we have
\begin{dmath}\label{eqn:ProblemSet3Problem1:40}
\bra{0} T e^{i \int d^4 x  g  j(x)  \phi(x)} \ket{0}
=
1 - \frac{g^2}{2} \int d^4 x d^4 y j(x) D_F(x - y) j(y) + \cdots
\approx
\exp\lr{ - \frac{g^2}{2} \int d^4 x d^4 y j(x) D_F(x - y) j(y) }
\end{dmath}

\makeSubAnswer{}{qft:problemSet3:1b}
TODO.
\makeSubAnswer{}{qft:problemSet3:1c}
TODO.
}

      %
% Copyright � 2018 Peeter Joot.  All Rights Reserved.
% Licenced as described in the file LICENSE under the root directory of this GIT repository.
%
\makeproblem{description}{qft:problemSet3:2}{
\makesubproblem{}{qft:problemSet3:2a}
} % makeproblem

\makeanswer{qft:problemSet3:2}{
\makeSubAnswer{}{qft:problemSet3:2a}

TODO.
}

      %
% Copyright � 2018 Peeter Joot.  All Rights Reserved.
% Licenced as described in the file LICENSE under the root directory of this GIT repository.
%
\makeproblem{``Radiation" by accelerated source and ``IR catastrophe''}{qft:problemSet3:3}{

{\flushleft{This}} is a baby problem having to do with radiation of scalar particles. (As we will not have too much time to study the radiation of electromagnetic fields this term, it is a good opportunity.) Consider the coupling of a classical particle to a scalar field (remember Homework 1, Problem 1, where a similar coupling to the electromagnetic field was considered):
\begin{equation}
\label{scalar1}
S_{int} = e \int\limits_{worldline} ds \phi(x(s))~,
\end{equation}
where $x(s)$ is the worldline of the particle and $e$ is its scalar charge (what is its dimension?). The coupling (\ref{scalar1}) corresponds to a ``current" $j(x)$ coupling to $\phi$ as in Problem {\bf II.} above:
\begin{equation}
\label{scalar2}
S_{int} = e \int\limits_{worldline} ds \phi(x(s) =   \int d^4 x j(x) \phi(x)~, ~~ {\rm where} ~~ j(x) = e \int\limits_{worldline} ds \delta^{(4)}(x - x(s))~,
\end{equation} 
is the  current. 

\makesubproblem{}{qft:problemSet3:3a}
Consider a particle of mass $M$, whose worldline is given by:
\begin{equation}
x^\mu(s) = {p^\mu_{i} \over M} s, ~{\rm for} ~~ s<0 ~ {\rm and} ~~ x^\mu(s) = {p^\mu_{f} \over M} s, ~{\rm for} ~~ s>0~,
\label{scalar3}
\end{equation}
where $p^\mu_i$ and $p^\mu_f$ are the initial and final four-momenta of the particle (both obeying $p^\mu p_\mu = M^2$,  with $p^0 > 0$, of course). The physical meaning of this trajectory is that the particle undergoes acceleration at $x^0=0$, suddenly changing its four-momentum from $p_i$ to $p_f$. Show that the Fourier transform of the current, as defined in (\ref{p2}) above, is given by:
\begin{equation}
\label{current1}
\tilde{j}(p) ={ i e M \over p \cdot p_f} - { i e M \over p \cdot p_i} 
\end{equation}
To make the TA's life (and yours) easier, in getting (\ref{current1}), consider without loss of generality, trajectories with $p_i = (M,0,0,0)$ and $p_f = (\sqrt{M^2 + q^2},q,0,0)$. \footnote{Recall  the ``half-delta function" integrals from Homework 2, Problem 1 and ignore the $i \epsilon$ factors which should be present in the denominators in (\ref{current1}) as they will not be important for what follows.}

\makesubproblem{}{qft:problemSet3:3b}
Now study the expression for the average number of particles produced, $\lambda$, or $\langle N \rangle$,  of Eq.~(\ref{p2}), as well as the average energy $\langle E \rangle$, which you can easily come up with, from (\ref{p2}). From now on, consider the case where the mass of the produced particles ($\phi$-quanta) is zero. This has two advantages:  simplifications in the various formulae as well as giving us the feeling that we are actually looking at something close to radiation of photons. 

Show that the integrals over the momenta of the emitted ``photons" in $\langle N\rangle$ and $\langle E\rangle$ diverge at large $p$. 

{\small  {\flushleft{T}}his is because our trajectory has a sudden change of momentum at $s=0$. We expect that the formulae for the radiated ``photons" is still valid for sufficiently small momenta where the nature of the kink is not relevant (presumably for momenta less than the inverse time during which a smooth change of momenta occurs, i.e. momenta smaller than the reciprocal of the scattering time). Thus, we now 
 suppose there is a high-momentum cutoff. }
 
 Let us then study  the convergence of the small-$p$ integrals over the momenta of the emitted particles in $\langle N\rangle$ and $\langle E\rangle$. This counts the number or energy  of ``soft" photons emitted. 
 Show what while $\langle E\rangle$ is finite, the expression for $\langle N \rangle$ diverges for small $\Bp$. 
 
{\small \flushleft{T}his divergence in the number of soft photons radiated by a classical source is called the ``infrared catastrophe", in the case of QED. A similar answer is obtained using a tree-level QFT calculation of the radiation of soft photons. Note one interesting fact: the divergence of the integral determining $\langle N \rangle$ is logarithmic: $ \langle N \rangle \sim e^2 \log {k_{max} \over k_{min}}$, where the IR cutoff $k_{min}$ is introduced to make the integral finite. You see now that $e^2$ (really, the fine structure constant $\alpha \sim 1/137$, in QED) is multiplied by a large $\log$, which can be bigger than $137$. This is a first indication that perturbation theory breaks down and some resummation  of diagrams may be  needed. Indeed, in QED, the infrared divergence is cancelled after adding ``loop" effects, see Section 6.5 of Peskin and Schroeder.  }
 
 {\small {\flushleft{T}}he point of this problem was to illustrate two things. First, it shows (within  this classical calculation of the overlap between free and interacting vacua) how the two vacua can be orthogonal (in the case of massless $\phi$, due to infrared (small momenta) problems). Second, it points toward something---the infrared divergencies in QED, and the resulting ``Sudakov double logs"---that you will study later, either in QFT2 or by yourselves.}
} % makeproblem

\makeanswer{qft:problemSet3:3}{
\makeSubAnswer{}{qft:problemSet3:3a}
TODO.
\makeSubAnswer{}{qft:problemSet3:3b}
TODO.
}

      %
% Copyright � 2018 Peeter Joot.  All Rights Reserved.
% Licenced as described in the file LICENSE under the root directory of this GIT repository.
%
\makeproblem{Where is the particle?}{qft:problemSet3:4}{
{\flushleft{In}} class, we did mention that, by analogy with non relativistic quantum mechanics, the state $\hat\phi(\Bx,t=0) \vert 0\rangle$ allows us to say something along the lines that {\it ``the operator $\hat\phi(\Bx)_+$ creates a particle at $\Bx$"}. 
These words are based on noticing  that in QM, we have 
$$\vert\Bx\rangle \sim \sum_{\Bp} e^{ i \Bp \cdot \Bx} \vert \Bp \rangle,$$
 where $\vert\Bx\rangle$ is an eigenstate of the position operator with eigenvalue $\Bx$ and $\Bp$ is, likewise, an eigenstate of momentum. On the other hand, in free massive scalar theory, the state $\hat\phi(\Bx,t=0) \vert 0\rangle$ can be similarly expressed as $$\hat\phi(\Bx,t=0) \vert 0\rangle = \int {d^3 p \over (2 \pi)^3 \sqrt{2 \omega_{\Bp}}} e^{ - i \Bp \cdot \Bx} \hat{a}^\dagger_{\Bp} \vert 0 \rangle =  \int {d^3 p \over (2 \pi)^3 2 \omega_{\Bp}} e^{ - i \Bp \cdot \Bx}   \vert p \rangle,$$ where $\vert p \rangle$
  is the relativistically normalized momentum eigenstate. Comparing the above two equations, reading from left to right, we are compelled to utter the words quoted in the beginning.
  
  
 Accepting this interpretation literally, we are next faced with explaining the following. Consider the state $\vert \Bzero,0 \rangle = \hat\phi(\Bzero,t=0) \vert 0\rangle$, interpreted (as per the above dissussion) as a particle created at $\Bx=0$ at $t=0$. Similarly, the state $$\vert \By,t \rangle = \hat\phi(\By,t) \vert 0 \rangle$$ is that of  a particle at $\By$ at $t$. Notice that these are free fields so their time evolution is trivial. Then, by the usual Born rule  of quantum mechanics (which we accept in QFT), the inner product 
    $$
    \langle \By,t \vert \Bzero,0 \rangle $$
    would be ``{\it the amplitude that the particle created at $\Bzero$ at $t=0$ 
    is found at $\By$ at $t$}". Notice that this is exactly the kind of answer that the quantum-mechanical propagator,  often denoted precisely by  $\langle \By,t \vert \Bzero,0 \rangle$, gives.
    A problem with this arises when one realizes that 
     $$
    \langle \By,t \vert \Bzero,0 \rangle  =\langle 0\vert \hat{\phi}(\By,t) \hat{\phi}(\Bzero,0)\vert0 \rangle = D(\By,t) \ne 0 ~ {\rm for} ~ (\By,t) \sim (\Bzero,0)~.$$
In other words, this amplitude is nonzero for spacelike separations (as you explicitly showed in Homework 2, Problem 1, Part 2).
The point of the simple exercise below is to argue that the above interpretation of this amplitude should be taken with a grain of salt, i.e. not too literally, as far as relativity is concerned, of course. 

The question we will ask is: to what extent is this particle at $\Bx=0$ localized? In quantum mechanics, we answer this question by pointing out that for an eigenstate of $\hat{x}$, whose wave function is $\delta(x -x')$,  the probability to find the particle anywhere but at $x=x'$ is zero. Trying to pursue this in QFT, a conundrum that arises is that we do not have wave functions for particles. Recall that we have wave functionals, which determine the probability that {\it the field} has this or that value. The coordinate, on the other hand, is an argument,   not an operator (hence ``observable") in the theory---just like time in QM, which is also not an operator; after all we 
said ``QM=QFT in $d=1$". 
The best we can do is to consider the state $\vert \By,0 \rangle$ and ask where its properties identifiable in QFT---energy or momentum---are localized.

 Thus,  consider the expectation value of $\hat{T}_{00}(\Bx,t)$ (assumed normal-ordered) in this state: 
$$
\rho(\By, \Bx, t) \equiv \langle \By,0 \vert T_{00}(\Bx,t)\vert \By,0 \rangle~.
$$
From the Born rule, the natural interpretation of the above quantity is the value of the energy density of the state $\vert \By,0 \rangle$ observed at $(\Bx,t)$---spacelike or not w.r.t. $(\By,0)$.


\makesubproblem{}{qft:problemSet3:4a}
Show, using the translation operator, that $\rho(\By, \Bx, t) =  \tilde\rho(0, \Bx - \By, t) \equiv \tilde\rho(\Bx-\By,t)$, where the last equality defines the new energy density $\tilde\rho(\Bx,t)$.
\makesubproblem{}{qft:problemSet3:4b}
Using Wick's theorem---really, a baby-version thereof---express $\tilde\rho(\Bx,t)$ in terms of $D(\Bx,t)$ and its derivatives.
\makesubproblem{}{qft:problemSet3:4c}
Using the knowledge acquired from Homework 2, study how well is the particle's energy localized, already at $t=0$.


Are you surprised by the result?  Are you comforted?
 
 {\small {\flushleft W}e didn't have time, apart from Problem 4 of Homework 2, to dwell much on the nonrelativisic limit. 
This limit can be achieved by forgetting the antiparticles and then defining non-relativistic fields. This is very well described in either Tong's or Luke's notes. For those of you studying cold atoms, it is 
definitely a must-read! }

\bigskip

My final comment is that the most concise formulation of causality that goes beyond simply stating that the commutators vanish for spacelike separations is the one first due to Stueckelberg (1940's) and then finessed by Bogoljubov (1950's). 

They consider the expectation value of an operator $\hat{O}(x)$   in a state prepared by the action of an operator $U[g]\vert 0\rangle$.
$U[g]$ is an evolution operator (see below) which is a functional of some classical fields $g(y)$ used to prepare the state of the field (e.g. external e.m. fields using to focus, accelerate, etc., the particles; $g(y)$ could also be used to turn on and off the interactions in different space time regions). Thus the object of study is:
$$ \langle \hat{O}(x) \rangle =  \langle 0 \vert U^\dagger[g] \hat{O}(x) U[g] \vert 0 \rangle~.$$ 
The causality condition, then, is that
$$ {\delta \langle \hat{O}(x) \rangle \over \delta g[y]} = 0 ~ {\rm for } ~ x\sim y~.$$
Now, recalling the form of the evolution operator,  $U[g] = T e^{  i \int dt d^3 x L_I(t, \Bx, g(\Bx,t))}$, and the Baker-Campbell-Hausdorf formula, it should be clear how the vanishing of the commutators outside the light cone becomes  relevant for the above condition  to hold. 
For Bogoljubov, the vanishing commutators are a {\it consequence} of the causality condition given in terms of variational derivatives, as expressed above; he derives the $S$-matrix expansion from that requirement along with a few others (locality and Lorentz invariance, basically). 

{\flushleft{T}he reason to include this comment was to close the loop on something that I mentioned in class, now that we've seen what $U[g]$ may look like. 
 }

%\begin{comment}
%   {\flushleft {\bf II.}} {\it Is gravity scalar?  }
%  
% 
%  \smallskip
%{\flushleft{Here,}} you are going to study, using the results from {\bf I.}, the question whether gravity can be described by a relativistic massless scalar field. After all, the Yukawa potential is $\sim 1/r$, in the limit when the mass goes to zero, which is just like the Newtonian potential.
%
%We'll take gravity to be described by a massless free scalar $\phi$, the ``scalar graviton", 
%just like the previous problem but with zero mass. The matter fields will have mass and will be described by another scalar field, $\psi$, this time massive. Now we have to decide how to couple these two. We shall attempt to do it this way: we shall couple the field $\phi$ to the field $\psi$ as follows: $H_{int} = g \int d^3 x T^\mu_{(\psi) \; \mu}(x) \phi(x)$. Here, $T^\mu_{(\psi) \; \mu}(x)$ is the trace of the energy-momentum tensor of the $\psi$ field and $g$ is a coupling constant. 
%\begin{enumerate}
%\item
%We shall not dwell much on the dynamics of the $\psi$ field and will just consider its $T^\mu_{(\psi) \; \mu}$ an external source of $\phi$. Still, we need to find a form for  $T^\mu_{(\psi) \; \mu}$ appropriate for a non relativistic static particle. 
% Consider...
% \item Find the constant $g$ such that the Newton law between two static masses is correctly reproduced by (\ref{three}).
% \end{enumerate}
% {\small \flushleft{Th}e reason scalar gravity does not work is that the coupling $T^\mu_{(\psi) \; \mu}(x) \phi(x)$ means that the ``scalar graviton" $\phi$ does not couple to the electromagnetic field (remember that $T^\mu_\mu =0$ there). On the other hand, the bending of light in the gravitational field is an experimental fact, so there should be a coupling between the two, not contained in our model. }
% 
%\bigskip
%\end{comment}
%
%
} % makeproblem

\makeanswer{qft:problemSet3:4}{
\makeSubAnswer{}{qft:problemSet3:4a}
TODO.
\makeSubAnswer{}{qft:problemSet3:4b}
TODO.
\makeSubAnswer{}{qft:problemSet3:4c}
TODO.
}


   \chapter{Problem Set 4.}

      %
% Copyright � 2018 Peeter Joot.  All Rights Reserved.
% Licenced as described in the file LICENSE under the root directory of this GIT repository.
%
\makeoproblem{The Wick theorem(s)}{qft:problemSet4:1}{2018 HW4.I}{

{\bf The mother of all Wick theorem(s): }
Let $A_1, A_2,... $ and $B$ denote a set of either creation or annihilation operators. In other words, $A_i = a_{k_i}$ or $a_{k_i}^\dagger$ (as well as $B$; $B$ is just like one of the $A$'s, but we'll use the letter $B$ to denote an operator which is singled out, as it is needed in the proof). Next, define a contraction $A_i A_k$ as follows:
\begin{equation}
\label{c1}
\contraction{O_1}{A}{{}_i O_2}{A}
O_1 A_i O_2 A_k =   O_1 O_2 \contraction{}{A}{{}_i} {A}  A_i A_k ~,
 \end{equation}
 where $O_1$, $O_2$ are arbitrary strings of operators. The above equation signifies the fact that the ``contraction" is a $c$-number, i.e. commutes with all operators. It is defined as follows:
  \begin{equation}
\label{c2}
  \contraction{}{A}{{}_i} {A}  A_i A_j = \left\{ \begin{array}l 0, \; {\rm if } \; \; A_i = a_{k_i}, A_j = a_{k_j} \;\; {\rm or} \;\; A_i = a_{k_i}^\dagger, A_j = a_{k_j}^\dagger\cr 0, \; {\rm if } \; \; A_i = a_{k_i}^\dagger \; \; {\rm and} \; \; A_j = a_{k_j} \cr
(2 \pi)^3  \delta^{(3)}(k_i - k_j),  \; {\rm if } \; \; A_i = a_{k_i} \; \; {\rm and} \; \; A_j = a_{k_j}^\dagger
 \end{array}~ \right.
 \end{equation}
 Put in words, the contraction vanishes if both $A$'s are creation (or both are annihilation operators), as indicated in the first line in \cref{c2}. The contraction is also zero if the operator to the right is an annihilation one, as per the second line in \cref{c2}. Finally, the contraction is equal to the commutator of $a_{k_i}$ with $a_{k_j}^\dagger$ in the case when the creation operator is to the left of the annihilation operator.

 Finally, we use $:A \ldots B:$ to denote the expression where all annihilation operators appear to the right of all creation operators, i.e. the usual normal ordered expression. Then,
 Wick's theorem---as used in many body physics---is formulated as follows:
\begin{dmath}\label{w1}
\begin{aligned}
 A_1 \ldots A_n &=
\normalorder{A_1 \ldots A_n}  \\
&+
\normalorder{\contraction{}{A}{{}_1}{A} A_1 A_2 A_3\ldots A_n}
+ \ldots +
\normalorder{\contraction{}{A}{{}_1\ldots}{A} A_1 \ldots A_{n-1} A_n}
+
\normalorder{\contraction{}{A}{{}_1 \ldots}{A} A_1 \ldots A_{n}}  \\
&+
\normalorder{\contraction{}{A}{{}_1}{A} A_1 A_2 \contraction{}{A}{{}_3}{A} A_3 A_4 \ldots A_n}
 + \ldots.
\end{aligned}
\end{dmath}
% \begin{eqnarray}
% \label{w1}
% A_1 \ldots A_n &=& \;\;\; :A_1 \ldots A_n:\nonumber \\
% & & + :\contraction{}{A}{{}_1}{A} A_1 A_2 A_3\ldots A_n: + \ldots + :\contraction{}{A}{{}_1\ldots}{A} A_1 \ldots A_{n-1} A_n: + :\contraction{}{A}{{}_1 \ldots}{A} A_1 \ldots A_{n}: \\
% && + :\contraction{}{A}{{}_1}{A} A_1 A_2 \contraction{}{A}{{}_3}{A} A_3 A_4 \ldots A_n: + \ldots \nonumber~.
% \end{eqnarray}
 The first line contains the normal-ordered product of all operators without contractions, the second line---all possible terms with one contraction (not involving only $A_1$ of course, but all single-contraction terms, which would be painful to indicate), the third line has all possible two-contraction terms, etc.

Now, you will prove \cref{w1} in steps.
\makesubproblem{}{qft:problemSet4:1a}
Prove the following Lemma:
 \begin{equation}
 \label{lemma}
  :A_1 A_2 \ldots A_n: B =  :A_1 A_2 \ldots A_n  B: + \sum\limits_{1 \le k \le n}:A_1 \ldots \contraction{}{A}{{}_k \ldots A_n}{B} A_k \ldots A_n B:
 \end{equation} Argue that if $B$ is an annihilation operator, the Lemma is trivial. Thus, consider $B$ to be a creation operator.
 Notice  that if any of the $A_{1,...,n}$ are creation operators, they can be taken to the left of the normal products in \cref{lemma} (because all their contractions with $B$ are zero). Thus, if the \cref{lemma} is proven for arbitrary $n$ for the case when all $A_i$'s are annihilation operators, the general case is obtained by multiplying on the left with the desired number of creation operators. Thus, it suffices to prove the Lemma for the case when all $A_i$'s are annihilation operators. Also
notice
  Thus, after proving the Lemma for $n=1$, use induction to show that it holds for any $n$. Assuming it holds for some number $n$, go to the case $n+1$ by multiplying \cref{lemma}  by some annihilation operator $A_0$ on the left and show that the Lemma holds for $n+1$ operators.

  By the chain of logic described above, you have proven \cref{lemma}.

Notice also that the lemma \cref{lemma} holds also if the product $$:A_1 A_2 \ldots A_n:$$ is replaced by
\begin{dmath}\label{eqn:ProblemSet4Problem1:160}
\normalorder{
A_1 \contraction{}{A}{{}_2\ldots}{A} A_2 \ldots  A_p \ldots A_n},
\end{dmath}
  i.e. with the product of operators with an arbitrary number of contractions (one, as written above), with a trivial modification of the last term (since, obviously, you can not contract $B$ with contractions).
\makesubproblem{}{qft:problemSet4:1b}
Now prove the actual Wick theorem \cref{w1}. Assuming that it holds for $n=2$. Imagine that \cref{w1} holds for $n$ operators and prove that it holds for $n+1$, using \cref{lemma}.
\makesubproblem{}{qft:problemSet4:1c}
{\bf An intermediate step:} Let now $A_i$ and $B$ be operators expressed as some linear combinations of creation and annihilation operators. In particular the subscripts $i$ may now indicate spatial dependence, rather than momentum eigenvalues. Now, define the contraction as follows:
   \begin{equation}
    \label{c5}
  \contraction{}{A}{{}_i} {A}  A_i A_j = \langle 0 \vert A_i A_j \vert 0 \rangle~,
 \end{equation}
 where $\vert 0 \rangle$ is the Fock vacuum.
Notice that \cref{c5} is equivalent to \cref{c2} when $A_i$'s are either creation or annihilation operators. Argue that \cref{w1} holds verbatim.

\makesubproblem{}{qft:problemSet4:1d}
{\bf The time-ordered Wick theorem:} Use the above Wick theorem to prove the time-ordered version. Notice that, despite appearances, there is not much left to do. Now, we have space-time rather than momentum space arguments and the theorem is now formulated as follows:
%\begin{eqnarray}
% \label{w2}
%T( A_1 \ldots A_n ) &=& \;\;\; :A_1 \ldots A_n:\nonumber \\
% & & + :\contraction{}{A}{{}_1}{A} A_1 A_2 A_3\ldots A_n: + \ldots + :\contraction{}{A}{{}_1\ldots}{A} A_1 \ldots A_{n-1} A_n: + :\contraction{}{A}{{}_1\ldots}{A} A_1 \ldots A_{n}: \\
% && + :\contraction{}{A}{{}_1}{A} A_1 A_2 \contraction{}{A}{{}_3}{A} A_3 A_4 \ldots A_n: + \ldots \nonumber~,
%\end{eqnarray}
\begin{equation}\label{w2}
\begin{aligned}
T( A_1 \ldots A_n ) &= :A_1 \ldots A_n: \\
&\qquad   + :\contraction{}{A}{{}_1}{A} A_1 A_2 A_3\ldots A_n: + \ldots + :\contraction{}{A}{{}_1\ldots}{A} A_1 \ldots A_{n-1} A_n: + :\contraction{}{A}{{}_1\ldots}{A} A_1 \ldots A_{n}: \\
&\qquad  + :\contraction{}{A}{{}_1}{A} A_1 A_2 \contraction{}{A}{{}_3}{A} A_3 A_4 \ldots A_n: + \ldots ~,
\end{aligned}
\end{equation}

 with the difference that $A_i$ are fields (we are considering real scalar fields),  $1 \ldots n$ denote space-time points, and the contraction is now the Feynman propagator, e.g. $D_F(x_1-x_2)$, etc.

 Notice that  to prove (\ref{w2}) one can consider a particular time ordering. Then the $T$ product becomes the normal product of operators (as they are assumed ordered). The space-time dependence can be taken out by Fourier transform which multiplies every term.  Every operator is a sum of creation and annihilation operators. Their commutators are exactly the ones giving rise to the contraction in \cref{c2}, on one hand, and to the function $D(x_i -x_j)$ after Fourier transform, on the other (recall that this function appears in the Feynman propagator).
Convince yourselves, using \cref{c5},  that this proves the theorem.

\makesubproblem{}{qft:problemSet4:1e}
For extra bonus, generalize all theorems above to anti commuting fields.
} % makeproblem

\makeanswer{qft:problemSet4:1}{
\withproblemsetsParagraph{
\makeSubAnswer{}{qft:problemSet4:1a}
The normal ordered sequence \( \normalorder{ A_1 A_2 \ldots A_n} B \) has the form
\begin{dmath}\label{eqn:ProblemSet4Problem1:20}
a_{k_1}^\dagger
\ldots
a_{k_r}^\dagger
a_{k_{r+1}}
\ldots
a_{k_n}
B
\end{dmath}
so if \( B = a_{k_{n+1}} \) is an anhillation operator the result is already normal ordered.  Since the Wick contraction with such a \( B \) is zero for all \( A_i \), that is
\begin{dmath}\label{eqn:ProblemSet4Problem1:40}
\begin{aligned}
\contraction{}{a}{{}_{k_1}^\dagger \ldots a_{k_r}^\dagger a_{k_{r+1}} \ldots a_{k_n} }{a}
a_{k_1}^\dagger \ldots a_{k_r}^\dagger a_{k_{r+1}} \ldots a_{k_n} a_{k_{n+1}}
&= 0 \\
\contraction{a_{k_1}^\dagger }{a}{{}_{k_2}^\dagger \ldots a_{k_r}^\dagger a_{k_{r+1}} \ldots a_{k_n}}{a}
a_{k_1}^\dagger a_{k_2}^\dagger \ldots a_{k_r}^\dagger a_{k_{r+1}} \ldots a_{k_n} a_{k_{n+1}}
&= 0 \\
\vdots & \\
\contraction{a_{k_1}^\dagger \ldots a_{k_r}^\dagger a_{k_{r+1}} \ldots }{a}{{}_{k_n}}{a}
a_{k_1}^\dagger \ldots a_{k_r}^\dagger a_{k_{r+1}} \ldots a_{k_n} a_{k_{n+1}}
&= 0,
\end{aligned}.
\end{dmath}
Summarizing, we see that for anhillation operators \( B \) we have
\begin{dmath}\label{eqn:ProblemSet4Problem1:100}
\normalorder{A_1 A_2 \ldots A_n} B =  \normalorder{A_1 A_2 \ldots A_n  B},
\end{dmath}
and
\begin{dmath}\label{eqn:ProblemSet4Problem1:200}
\sum\limits_{1 \le k \le n} \normalorder{A_1 \ldots \contraction{}{A}{{}_k \ldots A_n}{B} A_k \ldots A_n B} = 0,
\end{dmath}
so \cref{lemma} is valid for any anhillation operator \( B \).

For creation operators \( B \), we can cast the commutatation relations into Wick form
\begin{dmath}\label{eqn:ProblemSet4Problem1:60}
a_m a_k^\dagger
=
a_k^\dagger a_m + (2\pi)^3 \deltathree(k - m)
=
a_k^\dagger a_m +
\contraction{}{a}{{}_m}{a}
a_m a_k^\dagger,
\end{dmath}
and use this iteratively to percolate \( B = a_{n+1}^\dagger \) through \( \normalorder{A_1 A_2 \ldots A_n} \).  That is
\begin{dmath}\label{eqn:ProblemSet4Problem1:80}
\normalorder{A_1 A_2 \ldots A_n} B
=
a_{k_1}^\dagger \ldots a_{k_r}^\dagger a_{k_{r+1}} \ldots a_{k_n} a_{k_{n+1}}^\dagger
=
a_{k_1}^\dagger \ldots a_{k_r}^\dagger
a_{k_{r+1}}
\ldots
a_{k_{n-1}}
\lr{
   a_{k_{n+1}}^\dagger
   a_{k_n}
   +
   \contraction{}{a}{{}_{k_n} }{a}
   a_{k_n} a_{k_{n+1}}^\dagger
}
=
a_{k_1}^\dagger \ldots a_{k_r}^\dagger
a_{k_{r+1}}
\ldots
a_{k_{n-1}}
a_{k_{n+1}}^\dagger
a_{k_n}
+
a_{k_1}^\dagger \ldots a_{k_r}^\dagger
a_{k_{r+1}}
\contraction{}{a}{{}_{k_n} }{a}
a_{k_n} a_{k_{n+1}}^\dagger
=
a_{k_1}^\dagger \ldots a_{k_r}^\dagger
a_{k_{r+1}}
\ldots
a_{k_{n-2}}
\lr{
a_{k_{n+1}}^\dagger
a_{k_{n-1}}
+
\contraction{}{a}{{}_{k_{n+1}} }{a}
a_{k_{n-1}} a_{k_{n+1}}^\dagger
}
a_{k_n}
+
a_{k_1}^\dagger \ldots a_{k_r}^\dagger
a_{k_{r+1}}
\contraction{}{a}{{}_{k_n} }{a}
a_{k_n} a_{k_{n+1}}^\dagger
=
a_{k_1}^\dagger \ldots a_{k_r}^\dagger
a_{k_{r+1}}
\ldots
a_{k_{n-2}}
a_{k_{n+1}}^\dagger
a_{k_{n-1}}
a_{k_n}
+
a_{k_1}^\dagger \ldots a_{k_r}^\dagger
a_{k_{r+1}}
\ldots
a_{k_{n-2}}
\contraction{}{a}{{}_{k_{n+1}} }{a}
a_{k_{n-1}} a_{k_{n+1}}^\dagger
a_{k_n}
+
a_{k_1}^\dagger \ldots a_{k_r}^\dagger
a_{k_{r+1}}
\contraction{}{a}{{}_{k_n} }{a}
a_{k_n} a_{k_{n+1}}^\dagger.
\end{dmath}
We can continue like this until \( a^\dagger_{k_{n+1}} \) has percolated all the way through the anhillation operators
\begin{dmath}\label{eqn:ProblemSet4Problem1:120}
\normalorder{A_1 A_2 \ldots A_n} B
=
a_{k_1}^\dagger \ldots a_{k_r}^\dagger
\contraction{}{a}{{}_{k_{r+1}}}{a}
a_{k_{r+1}}
a_{k_{n+1}}^\dagger
\ldots
a_{k_{n-1}}
a_{k_n}
+
\cdots
+
a_{k_1}^\dagger \ldots a_{k_r}^\dagger
a_{k_{r+1}}
\ldots
a_{k_{n-2}}
\contraction{}{a}{{}_{k_{n+1}} }{a}
a_{k_{n-1}} a_{k_{n+1}}^\dagger
a_{k_n}
+
a_{k_1}^\dagger \ldots a_{k_r}^\dagger
a_{k_{r+1}}
\ldots
\contraction{}{a}{{}_{k_n} }{a}
a_{k_n} a_{k_{n+1}}^\dagger.
\end{dmath}
By the Wick operator definition, this may be rewritten with the \( a^\dagger_{k_{n+1}} \) at the end, that is
\begin{dmath}\label{eqn:ProblemSet4Problem1:140}
\normalorder{A_1 A_2 \ldots A_n} B
=
a_{k_1}^\dagger \ldots a_{k_r}^\dagger
\contraction{}{a}{{}_{k_{r+1}}
\ldots
a_{k_{n-1}}
a_{k_n}}
{a}
a_{k_{r+1}}
\ldots
a_{k_{n-1}}
a_{k_n}
a_{k_{n+1}}^\dagger
+
\cdots
+
a_{k_1}^\dagger \ldots a_{k_r}^\dagger
a_{k_{r+1}}
\ldots
a_{k_{n-2}}
\contraction{}{a}{{}_{k_{n-1}}
a_{k_n}
}{a}
a_{k_{n-1}}
a_{k_n}
a_{k_{n+1}}^\dagger
+
a_{k_1}^\dagger \ldots a_{k_r}^\dagger
a_{k_{r+1}}
\ldots
\contraction{}{a}{{}_{k_n} }{a}
a_{k_n} a_{k_{n+1}}^\dagger.
\end{dmath}
Since \( \contraction{}{a}{{}^\dagger_{k_{i}}}{a} a^\dagger_{k_{i}} a_{k_{n+1}}^\dagger = 0 \) we may add in contractions with all the creation operators.  Doing that completes the proof of \cref{lemma}.

For a contraction such as that of \cref{eqn:ProblemSet4Problem1:160} we need only modify the trailing sum of contractions.  That is
\begin{equation}\label{eqn:ProblemSet4Problem1:180}
\normalorder{A_1 \contraction{}{A}{{}_2\ldots}{A} A_2 \ldots  A_p \ldots A_n} B
=
\normalorder{A_1 \contraction{}{A}{{}_2\ldots}{A} A_2 \ldots  A_p \ldots A_n B}
+
\sum\limits_{1 \ge k, k \notin \setlr{2,p}, k \le n}
\bcontraction{: A_1 A_2 \ldots }{A}{{}_k \ldots  A_p \ldots A_n :}{B}
: A_1 \contraction{}{A}{{}_2\ldots A_k \ldots}{A} A_2 \ldots A_k \ldots  A_p \ldots A_n : B.
\end{equation}
This will clearly also be the case if the operator sequence \( \normalorder{A_1 \ldots A_n} \) includes any number of other contractions.

\makeSubAnswer{}{qft:problemSet4:1b}
Let's start in a more pedestrian fashion than diving straight into the induction, considering the first few values of \( n \) explicitly.
\begin{itemize}
\item \( n = 2 \).
We'd like to expand \( A B \), say, in terms of contractions.  Because \( A = \normalorder{A} \), that is
\begin{dmath}\label{eqn:ProblemSet4Problem1:220}
A B
= \normalorder{A} B
=
\normalorder{ A B } +
\contraction{}{A}{}{B}
A B
,
\end{dmath}
by \cref{lemma}.  This proves \cref{w1} for the \( n = 2 \) case.
\item \( n = 3 \).  Let's now expand \( A B C \)
\begin{dmath}\label{eqn:ProblemSet4Problem1:240}
A B C
=
(A B) C
=
\lr{
   \normalorder{ A B } +
   \contraction{}{A}{}{B}
   A B
} C.
\end{dmath}
We are now able to apply \cref{lemma} to \( \normalorder{ AB } C \), to find
\begin{dmath}\label{eqn:ProblemSet4Problem1:260}
A B C
=
\lr{
   \normalorder{ A B } +
   \contraction{}{A}{}{B}
   A B
} C
=
\normalorder{ A B C }
+
\normalorder{ A \contraction{}{ B }{}{C } B C }
+
\normalorder{ \contraction{}{ A }{B}{C } A B C }
+
(\contraction{}{A}{}{B}
   A B ) C
=
\normalorder{ A B C }
+
\normalorder{ A \contraction{}{ B }{}{C } B C }
+
\normalorder{ \contraction{}{ A }{B}{C } A B C }
+
\normalorder{ \contraction{}{A}{}{B} A B C },
\end{dmath}
where we also made use of \( \normalorder{ \contraction{}{A}{}{B} A B } C = \normalorder{ \contraction{}{A}{}{B} A B C } \).
This proves \cref{w1} for the \( n = 3 \) case.
\item \( n = 4 \).  This time we have
\begin{dmath}\label{eqn:ProblemSet4Problem1:300}
\begin{aligned}
A B C D
&=
(A B C) D \\
&=
\lr{
\normalorder{ A B C }
+
\normalorder{ A \contraction{}{ B }{}{C } B C }
+
\normalorder{ \contraction{}{ A }{B}{C } A B C }
+
\normalorder{ \contraction{}{A}{}{B} A B C }
} D \\
&=
\normalorder{ A B C D } +
\normalorder{
   \contraction{}{A}{B C}{D}
   A B C D
}
+
\normalorder{
   \contraction{A}{B}{C}{D}
   A B C D
}
+
\normalorder{
   \contraction{A B}{C}{}{D}
   A B C D
} \\
&\qquad+
\normalorder{
   \bcontraction{}{A}{\contraction{}{ B }{}{C } B C}{ D}
   A \contraction{}{ B }{}{C } B C D
}
+
\normalorder{
   \bcontraction{A}{B}{C}{D}
   \contraction{}{ A }{B}{C }
   A B C D
}
+
\normalorder{
   \contraction{}{A}{}{B} A B
   \bcontraction{}{C}{}{D} C D
}.
\end{aligned}
\end{dmath}
This time we have the normal ordering of all the operators, of all the operators with one set of contractions and of all the operators with two sets of contractions (although in that last case, the normal ordering is redundant.)

\item \( n + 1 \).  The way to proceed is now clear, but is just hard to write.
\begin{dmath}\label{eqn:ProblemSet4Problem1:280}
\begin{aligned}
A_1 \ldots A_n A_{n+1}
&=
(A_1 \ldots A_n) A_{n+1} \\
&=
\Biglr{
\normalorder{A_1 \ldots A_n} \\
&\qquad+
\normalorder{\contraction{}{A}{{}_1}{A} A_1 A_2 A_3\ldots A_n}
+ \ldots +
\normalorder{\contraction{}{A}{{}_1\ldots}{A} A_1 \ldots A_{n-1} A_n}
+
\normalorder{\contraction{}{A}{{}_1 \ldots}{A} A_1 \ldots A_{n}} \\
&\qquad+
\normalorder{\contraction{}{A}{{}_1}{A} A_1 A_2 \contraction{}{A}{{}_3}{A} A_3 A_4 \ldots A_n}
 + \ldots
} A_{n+1} \\
&=
\normalorder{A_1 \ldots A_n A_{n+1}} + \sum_{k \ne n + 1}
\normalorder{
\contraction{A_1 \ldots }{A}{{}_k \ldots A_n }{A}
A_1 \ldots A_k \ldots A_n A_{n+1}
}  \\
&\qquad+
\normalorder{\contraction{}{A}{{}_1}{A} A_1 A_2 A_3\ldots A_n A_{n+1}}
+
\sum_{k \notin \setlr{1,2,n+1}}
\normalorder{\contraction{}{A}{{}_1}{A} A_1 A_2 \ldots
\contraction{}{A}{{}_k\ldots A_n }{A}
A_k\ldots A_n A_{n+1}
}
+ \cdots \\
&\qquad+
\normalorder{\contraction{}{A}{{}_1}{A} A_1 A_2 \contraction{}{A}{{}_3}{A} A_3 A_4 \ldots A_n A_{n+1}}
+
\sum_{k \notin \setlr{1,2,3,4,n+1}}
\normalorder{
   \contraction{}{A}{{}_1}{A} A_1 A_2 \contraction{}{A}{{}_3}{A} A_3 A_4 \ldots
   \contraction{}{A}{{}_k \ldots A_n}{A}
   A_k \ldots A_n A_{n+1}
}
+ \cdots
\end{aligned}
\end{dmath}
Reading between the dots, we see that this is the sum of all possible normal-ordered contractions, completing the proof of \cref{w1}\footnote{While this barrage of visually discordant contractions can be thought of as a proof of the result, I find the concrete example of the \( n = 4 \) case much more satisfying as a ``proof'', despite not being general.  There the idea is clear, even without the formalism of an inductive proof.}.

\end{itemize}

\makeSubAnswer{}{qft:problemSet4:1c}
Let's first show that our new contraction definition \cref{c5} is equivalent to \cref{c2} when the operators are creation and anhillation operators.  It's easy to see that all the zero cases from \cref{c2} are recovered from this new definition
\begin{dmath}\label{eqn:ProblemSet4Problem1:320}
\begin{aligned}
\contraction{}{a}{{}_{\Bp}^\dagger }{a}
a_{\Bp}^\dagger a_{\Bq}^\dagger &= \bra{0} a_{\Bp}^\dagger a_{\Bq}^\dagger \ket{0} = 0 \\
\contraction{}{a}{{}_{\Bp}^\dagger }{a}
a_{\Bp}^\dagger a_{\Bq} &= \bra{0} a_{\Bp}^\dagger a_{\Bq} \ket{0} = 0 \\
\contraction{}{a}{{}_\Bq}{a}
a_{\Bp} a_{\Bq} &= \bra{0} a_{\Bp} a_{\Bq} \ket{0} = 0,
\end{aligned}
\end{dmath}
The only non-zero case is
\begin{dmath}\label{eqn:ProblemSet4Problem1:340}
\contraction{}{a}{{}_{\Bp}}{a}
a_{\Bp} a_{\Bq}^\dagger
=
\bra{0} a_{\Bp} a_{\Bq}^\dagger \ket{0}
\bra{0}
\lr{
a_{\Bq}^\dagger
a_{\Bp}
+
(2 \pi)^3 \deltathree(\Bp - \Bq)
}
\ket{0}
=
(2 \pi)^3 \deltathree(\Bp - \Bq),
\end{dmath}
which also matches \cref{c2} as desired, showing that \cref{c5} provides a nice compact representation of the contraction operator for any pair of creation and anhillation operators.

Now let's consider a pair of time dependent linear combinations of creation and anhillation operators.  Let
\begin{dmath}\label{eqn:ProblemSet4Problem1:360}
\begin{aligned}
A_i
&= \int \frac{d^3 p}{(2 \pi)^3 \sqrt{ 2 \omega_\Bp }}
\lr{
   e^{i p \cdot x} a_\Bp^\dagger
   +
   e^{-i p \cdot x} a_\Bp
} \\
A_j
&= \int \frac{d^3 q}{(2 \pi)^3 \sqrt{ 2 \omega_\Bq }}
\lr{
   e^{i q \cdot y} a_\Bq^\dagger
   +
   e^{-i q \cdot y} a_\Bq
}.
\end{aligned}
\end{dmath}
For such a combination let's show that \cref{w1} still applies.
\begin{dmath}\label{eqn:ProblemSet4Problem1:380}
A_i A_j
=
\int \frac{d^3 p d^3 q}{(2 \pi)^6 \sqrt{ 2 \omega_\Bp 2 \omega_\Bq}}
\lr{
   e^{i p \cdot x}
   a_\Bp^\dagger
   +
   e^{-i p \cdot x}
   a_\Bp
}
\lr{
   e^{i q \cdot y}
   a_\Bq^\dagger
   +
   e^{-i q \cdot y}
   a_\Bq
}
=
\int \frac{d^3 p d^3 q}{(2 \pi)^6 \sqrt{ 2 \omega_\Bp 2 \omega_\Bq}}
\lr{
   e^{i p \cdot x}
   e^{i q \cdot y}
   a_\Bp^\dagger
   a_\Bq^\dagger
+
   e^{-i p \cdot x}
   e^{i q \cdot y}
   a_\Bp
   a_\Bq^\dagger
+
   e^{i p \cdot x}
   e^{-i q \cdot y}
   a_\Bp^\dagger
   a_\Bq
+
   e^{-i p \cdot x}
   e^{-i q \cdot y}
   a_\Bp
   a_\Bq
}
=
\int \frac{d^3 p d^3 q}{(2 \pi)^6 \sqrt{ 2 \omega_\Bp 2 \omega_\Bq}}
\lr{
   e^{i p \cdot x}
   e^{i q \cdot y}
   a_\Bp^\dagger
   a_\Bq^\dagger
+
   e^{-i p \cdot x}
   e^{i q \cdot y}
\lr{
   a_\Bq^\dagger
   a_\Bp
+ (2 \pi)^3 \deltathree( \Bp - \Bq)
}
+
   e^{i p \cdot x}
   e^{-i q \cdot y}
   a_\Bp^\dagger
   a_\Bq
+
   e^{-i p \cdot x}
   e^{-i q \cdot y}
   a_\Bp
   a_\Bq
}
=
\normalorder{A_i A_j}
+
\int \frac{d^3 p d^3 q}{(2 \pi)^6 \sqrt{ 2 \omega_\Bp 2 \omega_\Bq}}
   e^{-i p \cdot x}
   e^{i q \cdot y}
   (2 \pi)^3 \deltathree(\Bp - \Bq).
\end{dmath}
However,
\begin{dmath}\label{eqn:ProblemSet4Problem1:400}
\contraction{}{A}{{}_i }{A}
A_i A_j
=
\bra{0}
\int \frac{d^3 p d^3 q}{(2 \pi)^6 \sqrt{ 2 \omega_\Bp 2 \omega_\Bq}}
\lr{
   e^{i p \cdot x}
   a_\Bp^\dagger
   +
   e^{-i p \cdot x}
   a_\Bp
}
\lr{
   e^{i q \cdot y}
   a_\Bq^\dagger
   +
   e^{-i q \cdot y}
   a_\Bq
}
\ket{0}
=
\bra{0}
\int \frac{d^3 p d^3 q}{(2 \pi)^6 \sqrt{ 2 \omega_\Bp 2 \omega_\Bq}}
   e^{-i p \cdot x}
   e^{i q \cdot y}
   a_\Bp
   a_\Bq^\dagger
\ket{0}
=
\bra{0}
\int \frac{d^3 p d^3 q}{(2 \pi)^6 \sqrt{ 2 \omega_\Bp 2 \omega_\Bq}}
   e^{-i p \cdot x}
   e^{i q \cdot y}
\lr{
   a_\Bq^\dagger
   a_\Bp
+ (2 \pi)^3 \deltathree(\Bp - \Bq)
}
\ket{0}
=
\int \frac{d^3 p d^3 q}{(2 \pi)^6 \sqrt{ 2 \omega_\Bp 2 \omega_\Bq}}
   e^{-i p \cdot x}
   e^{i q \cdot y}
 (2 \pi)^3 \deltathree(\Bp - \Bq).
\end{dmath}
It happens that we have a sybolic designation for this combination, namely \(
\contraction{}{A}{{}_i }{A}
A_i A_j = D(x - y) \), but the take away is really just the \( n = 2 \) statement of Wick's theorem
\begin{dmath}\label{eqn:ProblemSet4Problem1:420}
A_i A_j =
\normalorder{
A_i A_j }
+
\contraction{}{A}{{}_i }{A}
A_i A_j
,
\end{dmath}
which we see now applies to both pure creation and anhillation operators, as well as the combinations that we use to represent fields.  We could have just as easily have used a less specific linear combination than a presumed field -- had we done so, we'd have the same result, but wouldn't have been able to identify the contraction as \( D(x-y) \).

As \cref{eqn:ProblemSet4Problem1:420} was the starting point for the inductive procedure that we used to prove \cref{w1}, that theorem holds verbatim as desired.

\makeSubAnswer{}{qft:problemSet4:1d}
We are now asked to make one final redefinition of the contraction operator
\begin{dmath}\label{eqn:ProblemSet4Problem1:440}
  \contraction{}{A}{{}_i} {A}  A_i A_j = \langle 0 \vert T(A_i A_j)\vert 0 \rangle~,
\end{dmath}
This is clearly still identical to either of the previous definitions when \( A_i, A_j \) are creation and anhillation operators.

Let's consider a couple concrete cases again, starting with
\( n = 2 \) case again, writing \( A = A_i = \phi(x), B = A_j = \phi(y) \) defined by
\cref{eqn:ProblemSet4Problem1:360}.
For the \( x^0 > y^0 \) case we have
\begin{dmath}\label{eqn:ProblemSet4Problem1:520}
T( A B )
=
A B
=
\int \frac{d^3 p d^3 q}{(2 \pi)^6 \sqrt{ 2 \omega_\Bp 2 \omega_\Bq}}
\lr{
   e^{i p \cdot x}
   a_\Bp^\dagger
   +
   e^{-i p \cdot x}
   a_\Bp
}
\lr{
   e^{i q \cdot y}
   a_\Bq^\dagger
   +
   e^{-i q \cdot y}
   a_\Bq
}
=
\normalorder{AB}
+
\int \frac{d^3 p d^3 q}{(2 \pi)^6 \sqrt{ 2 \omega_\Bp 2 \omega_\Bq}}
   e^{i p \cdot x}
\contraction{}{
   a}{{}_\Bp
   e^{-i q \cdot y}}
   {a}
   a_\Bp
   e^{-i q \cdot y}
   a_\Bq^\dagger
=
\int \frac{d^3 p }{(2 \pi)^3 2 \omega_\Bp }
   e^{-i p \cdot (x-y)}
=
\normalorder{AB}
+
D(x - y),
\end{dmath}
on the other hand, if \( x^0 < y^0 \), using the same procedure, we must have
\begin{dmath}\label{eqn:ProblemSet4Problem1:540}
T( A B )
=
B A
=
\normalorder{BA}
+
\int \frac{d^3 p d^3 q}{(2 \pi)^6 \sqrt{ 2 \omega_\Bp 2 \omega_\Bq}}
   e^{i p \cdot x}
\contraction{}{
   a}{{}_\Bp
   e^{-i q \cdot y}}
   {a}
   a_\Bp
   e^{-i q \cdot y}
   a_\Bq^\dagger
=
\normalorder{AB}
+
D(y - x),
\end{dmath}
so
\begin{dmath}\label{eqn:ProblemSet4Problem1:480}
T(AB) = \normalorder{AB} + D_F(x - y).
\end{dmath}
Since \( D_F(x - y) = \bra{0} T( \phi(x) \phi(y)) \ket{0} = \contraction{}{A}{}{B} AB \), we have
\begin{dmath}\label{eqn:ProblemSet4Problem1:500}
T(AB)
= \normalorder{AB} + \contraction{}{A}{}{B} AB
= \normalorder{AB} + \normalorder{\contraction{}{A}{}{B} AB },
\end{dmath}
   which proves \cref{w2} for the \( n = 2 \) case.

   Now consider the \( n = 3 \) case, where \( A,B \) are defined as above, and \( C = \phi(z) \) is a third field.
   \begin{enumerate}[I.]
   \item For \( x^0 > y^0 > z^0 \), we have
   \begin{dmath}\label{eqn:ProblemSet4Problem1:560}
   T(A B C)
   =
   A B C
   =
   T(A B) C
   =
   \lr{
   \normalorder{AB} + \normalorder{\contraction{}{A}{}{B} AB }
   }
   C
   =
   \normalorder{\contraction{A}{B}{}{C} ABC }
   +
   \normalorder{\contraction{}{A}{B}{C} ABC }
   +
   \normalorder{\contraction{}{A}{}{B} AB } C,
   \end{dmath}
   \item For \( y^0 > x^0 > z^0 \), we have
   \begin{dmath}\label{eqn:ProblemSet4Problem1:580}
   T( A B C )
   =
   B A C
   =
   T( B A ) C
   =
   T( A B ) C
   \end{dmath}
   which equals \cref{eqn:ProblemSet4Problem1:560}.
   \item For \( z^0 > x^0 > y^0 \), we have
   \begin{dmath}\label{eqn:ProblemSet4Problem1:600}
   T( A B C )
   =
   C A B
   =
   C T( A B )
   =
   C
   \lr{
   \normalorder{AB} + \normalorder{\contraction{}{A}{}{B} AB }
   }
   =
   \normalorder{\contraction{}{C}{}{A} CAB }
   +
   \normalorder{\contraction{}{C}{A}{B} CAB }
   +
   C \normalorder{\contraction{}{A}{}{B} AB },
   \end{dmath}
   which also equals \cref{eqn:ProblemSet4Problem1:560}.
   \item For \( z^0 > y^0 > x^0 \), we have
   \begin{dmath}\label{eqn:ProblemSet4Problem1:620}
   T( A B C )
   = C B A
   = C T( B A )
   = C T( A B ),
   \end{dmath}
   which equals \cref{eqn:ProblemSet4Problem1:600}.
   \item For \( x^0 > z^0 > y^0 \), we have
   \begin{dmath}\label{eqn:ProblemSet4Problem1:640}
T( A B C )
=
A C B
=
T(A C) B
=
\lr{
\normalorder{AC} + \normalorder{\contraction{}{A}{}{C} AC }
}
B
=
\normalorder{\contraction{}{A}{C}{B} ACB }
+
\normalorder{\contraction{A}{C}{}{B} ACB }
+
\normalorder{\contraction{}{A}{}{C} AC } B,
\end{dmath}
which eauals \cref{eqn:ProblemSet4Problem1:560}.
\item For \( y^0 > z^0 > x^0 \), we have
\begin{dmath}\label{eqn:ProblemSet4Problem1:660}
T( A B C )
=
B C A
=
B T( C A )
=
B T( A C ),
\end{dmath}
which equals \cref{eqn:ProblemSet4Problem1:640}.
\end{enumerate}
All cases considered, we have now proven \cref{w2} for the \( n = 3 \) case.

Regardless of the time ordering of the fields, we end up with all possible combinations of contractions between all pairs of fields.  It is clear how this would generalize to higher numbers of fields.  This demonstration leaves me sufficiently convinced of the proof of the theorem, as desired.

%\makeSubAnswer{}{qft:problemSet4:1e}
%TODO.
}
}

      %
% Copyright � 2018 Peeter Joot.  All Rights Reserved.
% Licenced as described in the file LICENSE under the root directory of this GIT repository.
%
\makeproblem{The ``$h \rightarrow WW, ZZ$" Higgs-decay width.}{qft:problemSet4:2}{
From the $SU(2)_L \times SU(2)_R$ model of Homework 2---really, the Higgs Lagrangian of the Standard Model, find the coupling of the $h$-particle (the Higgs boson) to the $\phi^a$ particles (these are now Goldstone bosons, in the electroweak theory, they become the longitudinal components of the $W$ and $Z$ particles). Canonically normalizing $h$ and $\phi^a$, this coupling has the form
\begin{equation}
\label{gg1}
const. \; h \; \partial_\mu \phi^a \partial^\mu \phi^a~.
\end{equation}

\makesubproblem{}{qft:problemSet4:2a}
 Determine the value of $const.$ for canonically normalized $h$ and $\phi^a$.
\makesubproblem{}{qft:problemSet4:2b}
Use this coupling to compute the width  $\Gamma(h \rightarrow \phi^3 \phi^3)$ of the Higgs particle to decay to two longitudinal (say) $Z$-bosons (hence the index $3$).
\makesubproblem{}{qft:problemSet4:2c}
 Plug in some numbers. Use the fact that the vacuum expectation value $|m|/\sqrt{\lambda}  = 246$ GeV  and the fact that $m_h = 125$GeV to get a number for the lifetime. Compare to the total width of the Higgs from \url{http://pdg.lbl.gov/2012/reviews/rpp2012-rev-higgs-boson.pdf}, see figure 5 there, as well to the partial width to $WW$ given in Figure 4 there.
\makesubproblem{}{qft:problemSet4:2d}
 At the same time, determine the values of $|m|$ and $\lambda$ separately. Is $\lambda \ll 1$ (i.e. perturbative)?

{\flushleft {\small Notice that this calculation would have been physically relevant had the Higgs been heavy, $m_h \gg m_W \sim 100$ GeV. This is because the $h\rightarrow WW$ decay then is dominated (in this limit) by the decay into the longitudinal component, which is really the Goldstone boson field $\phi^a$ (in this limit, the result is independent of the gauge couplings $g_{1,2}$ of the Standard Model). Nonetheless, having some real numbers in this class is good.}}
} % makeproblem

\makeanswer{qft:problemSet4:2}{
\makeSubAnswer{}{qft:problemSet4:2a}
Here's a reminder and summary of the Higgs Lagrangian we will be working with in this problem
\begin{dmath}\label{eqn:ProblemSet4Problem2:640}
\LL = \trace{
   \lr{
      \partial_\mu H^\dagger \partial^\mu H
   }
}
- V,
\end{dmath}
where
\begin{dmath}\label{eqn:ProblemSet4Problem2:660}
V =
-\Abs{m}^2 \trace{
   \lr{
      H^\dagger H
   }
}
+ \lambda
\lr{
   \trace{
      H^\dagger H
   }
}^2.
\end{dmath}
It was postulated that the field had a radial component \( h \), the Higgs field, and an rotational component \( \Omega \), where the total field was given by
\begin{dmath}\label{eqn:ProblemSet4Problem2:680}
H(x) = \frac{\Abs{m}}{2 \sqrt{ \lambda } }\Omega(x) ( 1 + h(x) ),
\end{dmath}
where
\begin{equation}\label{eqn:ProblemSet4Problem2:700}
\Omega = e^{ i \Bsigma \cdot \Bphi } = e^{i \phi^a(x) \sigma^a }.
\end{equation}

Assuming that \( h(x) \) and \( \phi^a(x) \) commute, \( H^\dagger H \) can be computed with relative ease, and has only radial dependence
\begin{dmath}\label{eqn:ProblemSet4Problem2:380}
\trace{\lr{H^\dagger H}}
=
\frac{\Abs{m}^2}{4 \lambda} (1 + h(x))^2 \trace{\lr{ e^{-i \Bsigma \cdot \Bphi} e^{i \Bsigma \cdot \Bphi} }}
=
\frac{\Abs{m}^2}{4 \lambda} (1 + h(x))^2 \trace{\BOne}
=
\frac{\Abs{m}^2}{2 \lambda} (1 + h)^2.
\end{dmath}
For the derivative quadratic form, we find
\begin{dmath}\label{eqn:ProblemSet4Problem2:400}
\partial_\mu H^\dagger \partial^\mu H
=
\frac{\Abs{m}^2}{4 \lambda}
\lr{
   \partial_\mu h \Omega^\dagger
   + (1 + h) \partial_\mu \Omega^\dagger
}
\lr{
   \partial^\mu h \Omega
   + (1 + h)
\partial^\mu \Omega
}
=
\frac{\Abs{m}^2}{4 \lambda}
\lr{
   \partial_\mu h \Omega^\dagger \partial^\mu h \Omega
   + (1 + h)
      \lr{
         \partial_\mu h
         \Omega^\dagger (\partial^\mu \Omega)
       +
         \partial^\mu h
         (\partial_\mu \Omega^\dagger) \Omega
      }
   + (1 + h)^2 \partial_\mu \Omega^\dagger \partial^\mu \Omega
}.
\end{dmath}
Because \( \Omega^\dagger \Omega = 1 \), we have
\begin{dmath}\label{eqn:ProblemSet4Problem2:480}
\partial_\mu h
\Omega^\dagger (\partial^\mu \Omega)
 +
\partial^\mu h
(\partial_\mu \Omega^\dagger) \Omega
=
\partial_\mu h
\lr{
   \Omega^\dagger (\partial^\mu \Omega)
    +
   (\partial^\mu \Omega^\dagger) \Omega
}
=
\partial_\mu h
\lr{
   \partial^\mu (\Omega^\dagger \Omega) - (\partial^\mu \Omega^\dagger) \Omega
    +
   (\partial^\mu \Omega^\dagger) \Omega
}
=
   \partial^\mu (1)
= 0.
\end{dmath}
All the cross terms with both \( h \) and \( \Omega \) derivatives are zero (to all orders, not just quadratic).

Taking traces (and using cyclic permutation of the matrices in the trace operations),
the Lagrangian density is now determined
\begin{dmath}\label{eqn:ProblemSet4Problem2:500}
\LL =
\frac{\Abs{m}^2}{2 \lambda}
   \partial_\mu h \partial^\mu h
+
\frac{\Abs{m}^2}{4 \lambda} ( 1 + h )^2
   \trace{\lr{
      \partial_\mu \Omega^\dagger \partial^\mu \Omega
   }}
+ \Abs{m}^2
\frac{\Abs{m}^2}{2 \lambda} \lr{ 1 + h }^2
- \lambda
\lr{\frac{\Abs{m}^2}{2 \lambda}}^2
\lr{ 1 + h }^4
=
\frac{\Abs{m}^2}{\lambda} \LL',
\end{dmath}
where \( \LL' \) is the ``canonically normalized'' \footnote{Canonically normalized is assumed to mean that there's a one-half factor on the kinetic terms} Lagrangian
\begin{dmath}\label{eqn:ProblemSet4Problem2:720}
\LL' =
   \inv{2} \partial_\mu h \partial^\mu h
+
\inv{4}
( 1 + h )^2
   \trace{\lr{
      \partial_\mu \Omega^\dagger \partial^\mu \Omega
   }}
+
\inv{2}
\Abs{m}^2
\lr{ 1 + h }^2
-
\frac{\Abs{m}^2}{4}
\lr{ 1 + h }^4.
\end{dmath}

The coupling, let's call it \( c \), is to first order in \( h \) is
\begin{dmath}\label{eqn:ProblemSet4Problem2:740}
c = \inv{4} 2 h
   \trace{\lr{
      \partial_\mu \Omega^\dagger \partial^\mu \Omega
   }}.
\end{dmath}
Looking at these derivatives, to first order, we have
\begin{dmath}\label{eqn:ProblemSet4Problem2:580}
\partial_\mu \Omega
=
\partial_\mu \lr{ \BOne + i \Bsigma \cdot \Bphi }
=
i \Bsigma \cdot \partial_\mu \Bphi,
\end{dmath}
so
\begin{dmath}\label{eqn:ProblemSet4Problem2:760}
c
=
\inv{2} h
\trace{\lr{
   (-i \Bsigma \cdot \partial_\mu \Bphi^\dagger)
   (i \Bsigma \cdot \partial^\mu \Bphi)
}}
=
\inv{2}
h
\trace{\lr{
   (\Bsigma \cdot \partial_\mu \Bphi)
   (\Bsigma \cdot \partial^\mu \Bphi)
}},
\end{dmath}
where the real nature of each of the \( \phi^a \)'s has been used to eliminate the \( \dagger\).
The structure of this trace is that of
\begin{dmath}\label{eqn:ProblemSet4Problem2:780}
\trace{\lr{
   (\Bsigma \cdot \Bx)
   (\Bsigma \cdot \By)
}}
=
x^a y^b
\trace{\lr{
   \sigma^a \sigma^b
}}
=
x^a y^b
\left\{
\begin{array}{l l}
2 & \quad \mbox{\( a = b \)} \\
0 & \quad \mbox{\( a \ne b \)} \\
\end{array}
\right.
=
2 \Bx \cdot \By,
\end{dmath}
so the coupling is
\begin{dmath}\label{eqn:ProblemSet4Problem2:800}
c = h
\partial_\mu \phi^a \partial^\mu \phi^a.
\end{dmath}
This answers the question of the constant, which we find is just \( 1 \) after canonical normalization.

\makeSubAnswer{}{qft:problemSet4:2b}
TODO.
\makeSubAnswer{}{qft:problemSet4:2c}
TODO.
\makeSubAnswer{}{qft:problemSet4:2d}
TODO.
}

      %
% Copyright � 2018 Peeter Joot.  All Rights Reserved.
% Licenced as described in the file LICENSE under the root directory of this GIT repository.
%
\makeproblem{description}{qft:problemSet4:3}{
\makesubproblem{}{qft:problemSet4:3a}
} % makeproblem

\makeanswer{qft:problemSet4:3}{
\makeSubAnswer{}{qft:problemSet4:3a}

TODO.
}

      %
% Copyright � 2018 Peeter Joot.  All Rights Reserved.
% Licenced as described in the file LICENSE under the root directory of this GIT repository.
%
\makeproblem{Lorentz transforms of spinors---some useful identities}{qft:problemSet4:4}{

{\flushleft{C}}onsider the matrix $$\Lambda_{1\over 2} = e^{- {i\over 2} \omega_{\mu\nu} S^{\mu\nu}}~.$$
Here, $S^{\mu\nu} = {i\over 4} [\gamma^\mu, \gamma^\nu ]$ is as defined in class, in terms of the four $\gamma$-matrices (notice that, when using the representation of the $\gamma$ matrices in terms of Pauli matrices, the matrix $\Lambda_{1\over 2}$ looks like two sets of $M$ (and $M^*$) matrices discussed in class, now combined into one four-by-four object).

\makesubproblem{}{qft:problemSet4:4a}
Show that $\Lambda_{1\over 2}^{-1} \gamma^\mu \Lambda_{1\over 2} = \Lambda^\mu_{\; \nu} \gamma^\nu$, where $ \Lambda^\mu_{\; \nu}$ is the usual Lorentz transformation acting on vectors. (Feel free to show this for the infinitesimal form of the transformations, but then argue that the finite form holds as well.)
\makesubproblem{}{qft:problemSet4:4b}
Show that $\Lambda_{1\over 2}^{\dagger} \gamma^0 \Lambda_{1\over 2} =  \gamma^0$.
\makesubproblem{}{qft:problemSet4:4c}
Consider the fermion bilinear $\bar\psi \gamma^\mu \gamma^\nu \psi ={1\over 2} \bar\psi \{ \gamma^\mu ,\gamma^\nu \} \psi + {1\over 2} \bar\psi [\gamma^\mu, \gamma^\nu] \psi$, where $\{A,B\} = AB + BA$ is the anticommutator. Show that the two terms on the right transform as a scalar and a second-rank tensor, respectively, under Lorentz transformations.
} % makeproblem

\makeanswer{qft:problemSet4:4}{
\makeSubAnswer{}{qft:problemSet4:4a}
TODO.
\makeSubAnswer{}{qft:problemSet4:4b}
TODO.
\makeSubAnswer{}{qft:problemSet4:4c}
TODO.
}

%
}%

   \chapter{Independent study problems.}

      \input{scalarFieldCreationOpCommutator.tex}
      \input{scalarFieldHamiltonian.tex}
