%
% Copyright © 2018 Peeter Joot.  All Rights Reserved.
% Licenced as described in the file LICENSE under the root directory of this GIT repository.
%
%----------------------------------------------------------------------------------------
\part{Lecture notes}
   \chapter{Introduction.}
      %
% Copyright � 2018 Peeter Joot.  All Rights Reserved.
% Licenced as described in the file LICENSE under the root directory of this GIT repository.
%
%{
%%\input{../latex/blogpost.tex}
%%\renewcommand{\basename}{qftLecture1}
%%%\renewcommand{\dirname}{notes/phy1520/}
%%\renewcommand{\dirname}{notes/ece1228-electromagnetic-theory/}
%%%\newcommand{\dateintitle}{}
%%%\newcommand{\keywords}{}
%%
%%\input{../latex/peeter_prologue_print2.tex}
%%
%%\usepackage{peeters_layout_exercise}
%%\usepackage{peeters_braket}
%%\usepackage{peeters_figures}
%%\usepackage{siunitx}
%%\usepackage{verbatim}
%%%\usepackage{mhchem} % \ce{}
%%%\usepackage{macros_bm} % \bcM
%%%\usepackage{macros_qed} % \qedmarker
%%%\usepackage{txfonts} % \ointclockwise
%%
%%\beginArtNoToc
%%
%%\generatetitle{UofT QFT Fall 2018 phy2403 Lecture 1; What is a field?.  Taught by Prof. Erich Poppitz}
%%%\chapter{What is a field?}
%%%\label{chap:qftLecture1}
%%
%%\paragraph{DISCLAIMER: Very rough notes from class.  Some additional side notes, but otherwise barely edited.}
%%
% Monday Sept 10, 2018.
\section{What is a field?}

\begin{comment}
\end{comment}
A field is a map from space(time) to some set of numbers.  These set of numbers may be organized some how, possibly scalars, or vectors, ...

One example is the familiar spacetime vector, where \( \Bx \in \bbR^{d} \)

\begin{dmath}\label{eqn:qftLecture1:20}
(\Bx, t) \rightarrow \bbR^{\lr{d,1}}
\end{dmath}

Examples of fields:
\begin{enumerate}
\item \( 0 + 1 \) dimensional ``QFT'', where the spatial dimension is zero dimensional and we have one time dimension.  Fields in this case are just functions of time \( x(t) \).  That is, particle mechanics is a 0 + 1 dimensional classical field theory.  We know that classical mechanics is described by the action
\begin{dmath}\label{eqn:qftLecture1:40}
S = \frac{m}{2} \int dt \xdot^2.
\end{dmath}
This is non-relativistic.  We can make this relativistic by saying this is the first order term in the Taylor expansion
\begin{dmath}\label{eqn:qftLecture1:60}
S = - m c^2 \int dt \sqrt{ 1 - \xdot^2/c^2 }.
\end{dmath}
Classical field theory (of \( x(t) \)).  The ``QFT'' of \( x(t) \).  i.e. QM.
All of you know quantum mechanics.  If you don't just leave.  Not this way (pointing to the window), but this way (pointing to the door).
The solution of a quantum mechanical state is
\begin{dmath}\label{eqn:qftLecture1:80}
\bra{x} e^{-i H t/\Hbar } \ket{x'},
\end{dmath}
which can be found by evaluating the ``Feynman path integral''
\begin{dmath}\label{eqn:qftLecture1:100}
\sum_{\text{all paths x}} e^{i S[x]/\Hbar}
\end{dmath}
This will be particularly useful for QFT, despite the fact that such a sum is really hard to evaluate (try it for the Hydrogen atom for example).
\item \( 3 + 0 \) dimensional field theory, where we have 3 spatial dimensions and 0 time dimensions.  Classical equilibrium static systems.  The field may have a structure like
\begin{dmath}\label{eqn:qftLecture1:120}
\Bx \rightarrow \BM(\Bx),
\end{dmath}
for example, magnetization.
We can write the solution to such a system using the partition function
\begin{dmath}\label{eqn:qftLecture1:140}
Z \sim \sum_{\text{all} \BM(x)} e^{-E[\BM]/\kB T}.
\end{dmath}
For such a system the energy function may be like
\begin{dmath}\label{eqn:qftLecture1:160}
E[\BM] = \int d^3 \Bx \lr{ a \BM^2(\Bx) + b \BM^4(\Bx) + c \sum_{i = 1}^3 \lr{ \PD{x_i}{} \BM }
\cdot \lr{ \PD{x_i}{} \BM }
}.
\end{dmath}
There is an analogy between the partition function and the Feynman path integral, as both are summing over all possible energy states in both cases.
This will be probably be the last time that we mention the partition function and condensed matter physics in this term for this class.
\item \( 3 + 1 \) dimensional field theories, with 3 spatial dimensions and 1 time dimension.
Example, electromagnetism with \( \BE(\Bx, t), \BB(\Bx, t) \) or better use \( \BA(\Bx, t), \phi(\Bx, t) \).  The action is
\begin{dmath}\label{eqn:qftLecture1:180}
S = -\inv{16 \pi c} \int d^3 \Bx dt \lr{ \BE^2 - \BB^2 }.
\end{dmath}
This is our first example of a relativistic field theory in \( 3 + 1 \) dimensions.  It will take us a while to get there.
\end{enumerate}

These are examples of classical field theories, such as fluid dynamics and general relativity.  We want to consider electromagnetism because this is the place that we everything starts to fall apart (i.e. blackbody radiation, relating to the equilibrium states of radiating matter).  Part of the resolution of this was the quantization of the energy states, where we studied the normal modes of electromagnetic radiation in a box.  These modes can be considered an infinite number of radiating oscillators (the ultraviolet catastrophe).  This was resolved by Planck by requiring those energy states to be quantized (an excellent discussion of this can be found in \citep{bohm1989qt}.  In that sense you have already seen quantum field theory.

For electromagnetism the classical description is not always good.  Examples:
\begin{enumerate}
\item blackbody radiation.
\item electron energy \( e^2/r_\txte \) of a point charge diverges as \( r_\txte \rightarrow 0 \).
We can define the classical radius of the electron by
\begin{dmath}\label{eqn:qftLecture1:200}
\frac{e^2}{r^{\textrm{cl}}_{\txte}} \sim m_\txte c^2,
\end{dmath}
or
\begin{equation}\label{eqn:qftLecture1:220}
r^{\textrm{cl}}_{\txte} \sim \frac{m_\txte c^2}{e^2} \sim 10^{-15} \si{m}
\end{equation}
Don't treat this very seriously, but it becomes useful at frequencies \( \omega \sim c/r_\txte \), where \( r_\txte/c \) is approximately the time for light to cross a distance \( r_\txte \).
At frequencies like this, we should not believe the solutions that are obtained by classical electrodynamics.
In particular, self-accelerating solutions appear at these frequencies in classical EM.  This is approximately \( \omega_\conj \sim 10^{23} Hz \), or
\begin{dmath}\label{eqn:qftLecture1:240}
\begin{aligned}
\Hbar \omega_\conj
&\sim \lr{ 10^{-21} \,\si{MeV s}} \lr{ 10^{23} \,\si{1/s} }\\
&\sim 100 \si{MeV}.
\end{aligned}
\end{dmath}
% m_\txte \sim \inv{2} MeV
\end{enumerate}
At such frequencies particle creation becomes possible.

\section{Scales.}

A (dimensionless) value that is very useful in determining scale is
\begin{equation}\label{eqn:qftLecture1:260}
\alpha = \frac{e^2}{4 \pi \Hbar c} \sim \inv{137},
\end{equation}
called the fine scale constant, which relates three important scales relevant to quantum mechanics, as sketched in
\cref{fig:Lecture1scales:Lecture1scalesFig1}.
\imageFigure{../figures/phy2403-quantum-field-theory/Lecture1scalesFig1}{Interesting scales in quantum mechanics.}{fig:Lecture1scales:Lecture1scalesFig1}{0.3}
\begin{itemize}
\item The Bohr radius (large end of the scale).
\item The Compton wavelength of the electron.
\item The classical radius of the electron.
\end{itemize}

\subsection{Bohr radius}

A quick motivation for the Bohr radius was mentioned in passing in class while discussing scale, following the high school method of deriving the Balmer series (\citep{french1998iqp}).

That method assumes a circular electron trajectory (\(i = \Be_1 \Be_2\))
\begin{dmath}\label{eqn:qftLecture1:280}
\begin{aligned}
\Br &= r \Be_1 e^{i \omega t} \\
\Bv &= \omega r \Be_2 e^{i \omega t} \\
\Ba &= -\omega^2 r \Be_1 e^{i \omega t} \\
\end{aligned}
\end{dmath}
The Coulomb force (in cgs units) on the electron is
\begin{dmath}\label{eqn:qftLecture1:300}
\BF = m\Ba = -m \omega^2 r \Be_1 e^{i \omega t} = \frac{-e (e)}{r^2} \Be_1 e^{i \omega t},
\end{dmath}
or
\begin{dmath}\label{eqn:qftLecture1:320}
m \lr{ \frac{v}{r}}^2 r = \frac{e^2}{r^2},
\end{dmath}
giving
\begin{dmath}\label{eqn:qftLecture1:340}
m v^2 = \frac{e^2}{r}.
\end{dmath}
The energy of the system, including both Kinetic and potential (from an infinite reference point) is
\begin{dmath}\label{eqn:qftLecture1:360}
E
= \inv{2} m v^2 - \frac{e^2}{r}
= - \inv{2} m v^2 \sim \Hbar \omega = \Hbar \frac{v}{r},
\end{dmath}
or
\begin{dmath}\label{eqn:qftLecture1:380}
m v r \sim \Hbar.
\end{dmath}
Eliminating \( v \) using \cref{eqn:qftLecture1:340}, assuming a ground state radius \( r = a_0 \) gives
% \( m a_0 \sim \Hbar^2/e^2 \), orS
\begin{dmath}\label{eqn:qftLecture1:400}
a_0 \sim \frac{\Hbar^2}{m e^2}.
\end{dmath}
The Bohr radius is of the order \( 10^{-10} \si{m} \).

\subsection{Compton wavelength.}

When particle momentum starts approaching the speed of light, by the uncertainty relation (\(\Delta x \Delta p \sim \Hbar\)) the variation in position must be of the order
\begin{dmath}\label{eqn:qftLecture1:420}
\lambda_\txtc \sim \frac{\Hbar}{m_\txte c},
\end{dmath}
called the Compton wavelength.
Similarly, when the length scales are reduced to the Compton wavelength, the momentum increases to relativistic levels.
Because of the relativistic velocities at the Compton wavelength, particle creation and annihilation occurs and any theory has to account for multiple particle states.

\subsection{Relations.}

Scaling the Bohr radius once by the fine structure constant, we obtain the Compton wavelength (after dropping factors of \( 4\pi \))
\begin{dmath}\label{eqn:qftLecture1:440}
a_0 \alpha
= \frac{\Hbar^2}{m e^2}
\frac{e^2}{4 \pi \Hbar c}
= \frac{\Hbar}{4 \pi m c}
\sim
\frac{\Hbar}{m c}
= \lambda_\txtc.
\end{dmath}
Scaling once more, we obtain (after dropping another \( 4\pi\)) the classical electron radius
\begin{dmath}\label{eqn:qftLecture1:460}
\lambda_\txtc \alpha
=
\frac{e^2}{4 \pi m c^2}
\sim
\frac{e^2}{m c^2}.
\end{dmath}

%}
%%\EndArticle

   \chapter{Units, and scales.}
      %
% Copyright � 2018 Peeter Joot.  All Rights Reserved.
% Licenced as described in the file LICENSE under the root directory of this GIT repository.
%
%{
\input{../latex/blogpost.tex}
\renewcommand{\basename}{qftLecture2}
%\renewcommand{\dirname}{notes/phy1520/}
\renewcommand{\dirname}{notes/ece1228-electromagnetic-theory/}
%\newcommand{\dateintitle}{}
%\newcommand{\keywords}{}

\input{../latex/peeter_prologue_print2.tex}

\usepackage{peeters_layout_exercise}
\usepackage{peeters_braket}
\usepackage{peeters_figures}
\usepackage{siunitx}
%\usepackage{mhchem} % \ce{}
%\usepackage{macros_bm} % \bcM
%\usepackage{macros_qed} % \qedmarker
%\usepackage{txfonts} % \ointclockwise

%\newcommand{m_\txte}[0]{m_\txte}

\beginArtNoToc

% Wednesday Sept 12, 2018.
\generatetitle{UofT QFT Fall 2018 Lecture 2, taught by Prof. Erich Poppitz}
%\chapter{UofT QFT Fall 2018 Lecture 2, taught by Prof. Erich Poppitz}

At Compton wavelength, multiple particle pair production is possible

%%%\Delta p \Delta x \sim \Hbar
%%%
%%%Suppose
%%%
%%%\Delta x \sim \frac{\Hbar}{m_\txte c}
%%%
%%%[paper]
%%%
\section{Natural units.}

\begin{dmath}\label{eqn:qftLecture2:20}
\begin{aligned}
[\Hbar] &= [\text{action}] = M \frac{L^2}{T^2} T = \frac{M L^2}{T} \\
[c]    &= [\text{velocity}] = \frac{L}{T} \\
       &  [\text{energy}] = M \frac{L^2}{T^2}.
\end{aligned}
\end{dmath}

Setting \( c = 1 \) means

\begin{dmath}\label{eqn:qftLecture2:240}
\frac{L}{T} = 1
\end{dmath}

and setting \( \Hbar = 1 \) means

\begin{dmath}\label{eqn:qftLecture2:260}
[\Hbar] = [\text{action}] = M L \cancel{\frac{L}{T}} = M L
\end{dmath}

therefore

\begin{dmath}\label{eqn:qftLecture2:280}
[L] = \inv{\text{mass}}
\end{dmath}

and

\begin{dmath}\label{eqn:qftLecture2:300}
[\text{energy}] = M \cancel{\frac{L^2}{T^2}} = \text{mass}\, \si{eV}
\end{dmath}

Summary


\begin{itemize}
\item
\( \text{energy} \sim \si{eV} \)
\item
\( \text{distance} \sim \inv{M} \)
\item
\( \text{time} \sim \inv{M} \)
\end{itemize}

%e (dimensionless)
From:
\begin{dmath}\label{eqn:qftLecture2:320}
\alpha = \frac{e^2}{4 \pi \cancel{\Hbar c}}
\end{dmath}

which is dimensionless (\(1/137\)), so electric charge is dimensionless.

Some useful numbers in natural units

\begin{dmath}\label{eqn:qftLecture2:40}
\begin{aligned}
   m_\txte &\sim 10^{-27} \si{g} \sim 0.5 \si{MeV} \\
   m_\txtp &\sim 2000 m_\txte \sim 1 \si{GeV} \\
   m_\pi &\sim 140 \si{MeV} \\
   m_\mu &\sim 105 \si{MeV} \\
   \Hbar c &\sim 200 \si{MeV} \,\si{fm} = 1
\end{aligned}
\end{dmath}

%distance \inv{\si{MeV}} \sim 10^{-11} \si{cm} 10^{-15} m, or 10^{-13} cm.

\section{Gravity}

Interaction energy of two particles

\begin{dmath}\label{eqn:qftLecture2:60}
G_\txtN \frac{m_1 m_2}{r}
\end{dmath}

\begin{dmath}\label{eqn:qftLecture2:80}
[\text{energy}] \sim [G_\txtN] \frac{M^2}{L}
\end{dmath}

\begin{dmath}\label{eqn:qftLecture2:100}
[G_\txtN]
\sim
[\text{energy}] \frac{L}{M^2}
\end{dmath}

but energy x distance is dimensionless (action) in our units

\begin{dmath}\label{eqn:qftLecture2:120}
[G_\txtN]
\sim
{\text{dimensionless}}{M^2}
\end{dmath}

\begin{dmath}\label{eqn:qftLecture2:140}
\frac{G_\txtN}{\Hbar c} \sim \inv{M^2} \sim \frac{1}{10^{20} \si{GeV}}
\end{dmath}

Planck mass

\begin{dmath}\label{eqn:qftLecture2:160}
M_{\text{Planck}} \sim \sqrt{\frac{\Hbar c}{G_\txtN}}
\sim 10^{-4} g \sim \inv{\lr{10^{20} \si{GeV}}^2}
\end{dmath}

(Verbal discussion of cross section, roughly sounds like the number of events per unit time, related to the flux of some source through an area).

We'll compute the cross section of a number of different systems in this course.

F2.

QED: assume highly relativistic

\begin{dmath}\label{eqn:qftLecture2:180}
\alpha \sim e^2
\end{dmath}

cross section

\begin{dmath}\label{eqn:qftLecture2:200}
\sigma \sim \frac{\alpha^2}{E^2} \lr{ 1 + O(\alpha) + O(\alpha^2) + \cdots }
\end{dmath}

In gravity we could consider scattering of electrons, where \( G_\txtN \) takes the place of \( \alpha \).  However, \( G_\txtN \) has dimensions.

For electron-electron scattering due to gravitons

\begin{dmath}\label{eqn:qftLecture2:220}
\sigma \sim \frac{G_\txtN^2 E^2}{1 + G_\txtN E^2 + \cdots }
\end{dmath}

Now the cross section grows with energy.  This will cause some problems (violating unitarity: probabilities greater than 1!) when \( O(G_\txtN E^2) = 1 \).

In any quantum field theories when the coupling constant is not-dimensionless we have the same sort of problems at some scale.

The point is that we can get far considering just dimensional analysis.

If the coupling constant has a dimension \((1/\text{mass})^N\,, N > 0\), then unitarity will be violated at high energy.  One such theory is the Fermi theory of beta decay (electro-weak theory), which had a coupling constant with dimensions inverse-mass-squared.  The relevant scale for beta decay was 4 Fermi, or \( G_\txtF \sim (1/{100 \si{GeV}})^2 \).  This was the motivation for introducing the Higgs theory, which was motivated by restoring unitarity.

\section{Lorentz transformations.}

The goal, perhaps not for today, is to study the simplest (relativistic) scalar field theory.  First studied classically, and then consider such a quantum field theory.
How is relativity implemented when we write the Lagrandian and action?

Our first step must be to consider Lorentz transformations and the Lorentz group.

Spacetime (Minkowski space) is \R{3,1} (or \R{d-1,1}).  Our coordinates are

\begin{dmath}\label{eqn:qftLecture2:340}
(c t, x^1, x^2, x^3) = (c t, \Br).
\end{dmath}

Here, we've scaled the time scale by \( c \) so that we measure time and space in the same dimensions.  We write this as

\begin{dmath}\label{eqn:qftLecture2:360}
x^\mu = (x^0, x^1, x^2, x^3),
\end{dmath}

where \( \mu = 0, 1, 2, 3 \), and call this a ``4-vector''.  These are called the space-time coordinates of an event, which tell us where and when an event occurs.

For two events whose spacetime coordinates differ by \( dx^0, dx^1, dx^2, dx^3 \) we introduce the notion of a space time \underline{interval}

\begin{dmath}\label{eqn:qftLecture2:380}
ds^2 = c^2 dt^2
- (dx^1)^2
- (dx^2)^2
- (dx^3)^2
=
\sum_{\mu, \nu = 0}^3 g_{\mu\nu} dx^\mu dx^\nu
\end{dmath}

Here \( g_{\mu\nu} \) is the Minkowski space metric, an object with two indexes that run from 0-3.  i.e. this is a diagonal matrix

\begin{dmath}\label{eqn:qftLecture2:400}
g_{\mu\nu} \sim
\begin{bmatrix}
1 & 0 & 0 & 0 \\
0 & -1 & 0 & 0 \\
0 & 0 & -1 & 0 \\
0 & 0 & 0 & -1 \\
\end{bmatrix}
\end{dmath}

i.e.
\begin{dmath}\label{eqn:qftLecture2:420}
\begin{aligned}
g_{00} &= 1 \\
g_{11} &= -1 \\
g_{22} &= -1 \\
g_{33} &= -1 \\
\end{aligned}
\end{dmath}

We will use the Einstein summation convention, where any repeated upper and lower indexes are considered summed over.  That is \cref{eqn:qftLecture2:380} is written with an implied sum
\begin{dmath}\label{eqn:qftLecture2:440}
ds^2 = g_{\mu\nu} dx^\mu dx^\nu.
\end{dmath}

Explicit expansion:
\begin{dmath}\label{eqn:qftLecture2:460}
ds^2 = g_{\mu\nu} dx^\mu dx^\nu.
=
g_{00} dx^0 dx^0
+g_{11} dx^1 dx^1
+g_{22} dx^2 dx^2
+g_{33} dx^3 dx^3
=
(1) dx^0 dx^0
+ (-1) dx^1 dx^1
+ (-1) dx^2 dx^2
+ (-1) dx^3 dx^3.
\end{dmath}

Recall that rotations (with orthogonal matrix representations) are transformations that leave the dot product unchanged, that is

\begin{dmath}\label{eqn:qftLecture2:480}
(R \Bx) \cdot (R \By)
= \Bx^\T R^\T R \By
= \Bx^\T \By
= \Bx \cdot \By,
\end{dmath}

where \( R \) is a rotation orthogonal 3x3 matrix.  The set of such transformations that leave the dot product unchanged have orthonormal matrix representations \( R^\T R = 1 \).  We call the set of such transformations that have unit determinant the SO(3) group.

We call a Lorentz transformation, if it is a linear transformation acting on 4 vectors that leaves the spacetime interval (i.e. the inner product of 4 vectors) invariant.  That is, a transformation that leaves
\begin{dmath}\label{eqn:qftLecture2:500}
x^\mu y^\nu g_{\mu\nu} = x^0 y^0 - x^1 y^1 - x^2 y^2 - x^3 y^3
\end{dmath}
unchanged.

Suppose that transformation has a 4x4 matrix form

\begin{dmath}\label{eqn:qftLecture2:520}
{x'}^\mu = {\Lambda^\mu}_\nu x^\nu
\end{dmath}

Example of a possible \( \Lambda \).

F3

\begin{dmath}\label{eqn:qftLecture2:540}
\begin{aligned}
x &= \frac{x' + v t'}{\sqrt{1 - v^2/c^2}} \\
y &= y' \\
z &= z' \\
t &= \frac{t' + (v/c^2) x'}{\sqrt{1 - v^2/c^2}} \\
\end{aligned}
\end{dmath}

or

\begin{dmath}\label{eqn:qftLecture2:560}
\begin{bmatrix}
c t \\
x \\
y \\
z
\end{bmatrix}
=
\begin{bmatrix}
\inv{\sqrt{1 - v^2/c^2}} & \frac{v/c}{\sqrt{1 - v^2/c^2}} & 0 & 0 \\
\frac{v/c}{\sqrt{1 - v^2/c^2}} & \frac{1}{\sqrt{1 - v^2/c^2}} & 0 & 0 \\
0 & 0 & 1 & 0 \\
0 & 0 & 0 & 1 \\
\end{bmatrix}
\begin{bmatrix}
c t' \\
x' \\
y' \\
z'
\end{bmatrix}
\end{dmath}

Other examples include rotations (\({\lambda^0}_0 = 1\) zeros in \( {\lambda^0}_k, {\lambda^k}_0 \), and a rotation matrix in the remainder.)

% submatrix:
%\begin{dmath}\label{eqn:qftLecture2:580}
%\begin{bmatrix}
%1 & 0 & 0 & 0
%0 &
%0 &    R
%0 &
%\end{bmatrix}
%\end{dmath}

Back to Lorentz transformations (\(\text{SO}(1,3)^+\)), let

\begin{dmath}\label{eqn:qftLecture2:600}
\begin{aligned}
{x'}^\mu    &= {\Lambda^\mu}_\nu x^\nu \\
{y'}^\kappa &= {\Lambda^\kappa}_\rho y^\rho
\end{aligned}
\end{dmath}

The dot product
\begin{dmath}\label{eqn:qftLecture2:620}
g_{\mu \kappa}
{x'}^\mu
{y'}^\kappa
=
g_{\mu \kappa}
{\Lambda^\mu}_\nu
{\Lambda^\kappa}_\rho
x^\nu
y^\rho
=
g_{\nu\rho}
x^\nu
y^\rho,
\end{dmath}
where the last step introduces the invariance requirement of the transformation.  That is

%\begin{dmath}\label{eqn:qftLecture2:640}
\boxedEquation{eqn:qftLecture2:640}{
g_{\nu\rho}
=
g_{\mu \kappa}
{\Lambda^\mu}_\nu
{\Lambda^\kappa}_\rho.
}
%\end{dmath}

\paragraph{Upper and lower indexes}

We've defined

\begin{dmath}\label{eqn:qftLecture2:660}
x^\mu = (t, x^1, x^2, x^3)
\end{dmath}

We could also define a four vector with lower indexes
\begin{dmath}\label{eqn:qftLecture2:680}
x_\nu = g_{\nu\mu} x^\mu = (t, -x^1, -x^2, -x^3).
\end{dmath}
That is
\begin{dmath}\label{eqn:qftLecture2:700}
\begin{aligned}
x_0 &= x^0 \\
x_1 &= -x^1 \\
x_2 &= -x^2 \\
x_3 &= -x^3.
\end{aligned}
\end{dmath}

which allows us to write the dot product as simply \( x^\mu y_\mu \).

We can also define a metric tensor with upper indexes

\begin{dmath}\label{eqn:qftLecture2:401}
g^{\mu\nu} \sim
\begin{bmatrix}
1 & 0 & 0 & 0 \\
0 & -1 & 0 & 0 \\
0 & 0 & -1 & 0 \\
0 & 0 & 0 & -1 \\
\end{bmatrix}
\end{dmath}
This is the inverse matrix of \( g_{\mu\nu} \), and it satisfies
\begin{dmath}\label{eqn:qftLecture2:720}
g^{\mu \nu} g_{\nu\rho} = {\delta^\mu}_\rho
\end{dmath}

Exersize: Check:

\begin{dmath}\label{eqn:qftLecture2:740}
g_{\mu\nu} x^\mu y^\nu = x_\nu y^\nu = x^\nu y_\nu
= g^{\mu\nu} x_\mu y_\nu = {\delta^\mu}_\nu x_\mu y^\nu
\end{dmath}

Class ended around this point, but it appeared that we were heading this direction:

Returning to the Lorentz invariant and multiplying both sides of
\cref{eqn:qftLecture2:640} with an inverse Lorentz transformation \( \Lambda^{-1} \), we find
\begin{dmath}\label{eqn:qftLecture2:760}
g_{\nu\rho}
{\lr{\Lambda^{-1}}^\rho}_\alpha
=
g_{\mu \kappa}
{\Lambda^\mu}_\nu
{\Lambda^\kappa}_\rho
{\lr{\Lambda^{-1}}^\rho}_\alpha
=
g_{\mu \kappa}
{\Lambda^\mu}_\nu
{\delta^\kappa}_\alpha
=
g_{\mu \alpha}
{\Lambda^\mu}_\nu,
\end{dmath}
or
\begin{dmath}\label{eqn:qftLecture2:780}
\lr{\Lambda^{-1}}_{\nu \alpha} = \Lambda_{\alpha \nu}.
\end{dmath}
This is clearly analogous to \( R^\T = R^{-1} \), although the index notation obscures things considerably.  Prof. Poppitz said that next week this would all lead to showing that the determinant of any Lorentz transformation was \( \pm 1 \).

For what it's worth, it seems to me that this index notation makes life a lot harder than it needs to be, at least for a matrix related question (i.e. determinant of the transformation).  In matrix/column-(4)-vector notation, let \(x' = \Lambda x, y' = \Lambda y\) be two four vector transformations, then
\begin{dmath}\label{eqn:qftLecture2:800}
x' \cdot y' = {x'}^T G y' = (\Lambda x)^T G \Lambda y = x^T ( \Lambda^T G \Lambda) y = x^T G y.
\end{dmath}
so
\boxedEquation{eqn:qftLecture2:820}{
\Lambda^T G \Lambda = G.
}
Taking determinants of both sides gives \(-(det(\Lambda))^2 = -1\), and thus \(det(\Lambda) = \pm 1\).

%}
%\EndArticle
\EndNoBibArticle

      \section{Problems.}
         %
% Copyright � 2015 Peeter Joot.  All Rights Reserved.
% Licenced as described in the file LICENSE under the root directory of this GIT repository.
%
\makeproblem{Dimensional analysis.}{qft:problemSet1:4}{

Even though we have set \( \Hbar = c = 1 \), we can still do dimensional analysis because we still have one unit left, mass (or 1/length). In \( d \) space-time dimensions (1 time and \( d-1 \) space), what is the dimension in mass units of a canonical free scalar field, \( \phi \)? (Work it out from the equal-time commutation relations.) Still in \( d \) dimensions, the Lagrange density for a scalar field with self-interactions might be of the form

\begin{dmath}\label{eqn:ProblemSet1Problem4:20}
\LL = \inv{2} \lr{ \partial_\mu \phi}^2 - \sum_{n \ge 2} a_n \phi^n.
\end{dmath}

\makesubproblem{}{qft:problemSet1:4a}
What is the dimension (again in mass units) of the Lagrange density?
\makesubproblem{}{qft:problemSet1:4b}
The action?
\makesubproblem{}{qft:problemSet1:4c}
The coefficients \( a_n \)? (as a check, whatever the value of \( d \), \(a_2\) had better have the dimensions of \(\textrm{mass}^2\) ).
} % makeproblem

\makeanswer{qft:problemSet1:4}{
\makeSubAnswer{}{qft:problemSet1:4a}

With \( \antisymmetric{\phi(\Bx)}{\Pi(\By)} = i \delta^3(\Bx - \By) \), which is dimensionless, we have

\begin{dmath}\label{eqn:ProblemSet1Problem4:40}
1
= [ \phi \Pi ]
= [ \phi^2 ] /L,
\end{dmath}

so

\begin{dmath}\label{eqn:ProblemSet1Problem4:60}
[\phi] = L^{1/2}.
\end{dmath}

This means that the dimensions of the Lagrangian are
\begin{dmath}\label{eqn:ProblemSet1Problem4:80}
[\LL]
= [(\partial_\mu \phi)^2]
= \inv{L^2} L
= \inv{L}.
\end{dmath}

\makeSubAnswer{}{qft:problemSet1:4b}

The dimensions of the action are

\begin{dmath}\label{eqn:ProblemSet1Problem4:100}
[S]
= [ \int d^d x \LL ]
= L^d \inv{L}
= L^{d-1}
\end{dmath}

\makeSubAnswer{}{qft:problemSet1:4c}

The dimensions of the coefficients are found from

\begin{dmath}\label{eqn:ProblemSet1Problem4:120}
\inv{L}
 =
[a_n \phi^n]
=
[a_n] L^{n/2},
\end{dmath}

or

\begin{dmath}\label{eqn:ProblemSet1Problem4:140}
[a_n] = L^{-1 - n/2}.
\end{dmath}

For \( n = 2 \) that is \( [a_n] = L^{-1 - 2/2} = L^{-2} \).  Provided \( [L] = 1/[M] \) this is what is expected.  To see that is the case consider the dimensions of the ratio

\begin{dmath}\label{eqn:qftProblemSet1Problem4:160}
[\Hbar/c]
= [ (M L^2/T)/(L/T) ]
= [ M L ].
\end{dmath}

If both \( \Hbar \) and \( c \) are dimensionless then the dimensions of length must be inverse mass.
}

   \chapter{Lorentz transformations.}
      %
% Copyright © 2018 Peeter Joot.  All Rights Reserved.
% Licenced as described in the file LICENSE under the root directory of this GIT repository.
%
%{
\section{Lorentz transformations.}
\index{Lorentz transformations}

The goal, perhaps not for today, is to study the simplest (relativistic) scalar field theory.  First studied classically, and then consider such a quantum field theory.
How is relativity implemented when we write the Lagrangian and action?

Our first step must be to consider Lorentz transformations and the Lorentz group.

Spacetime (Minkowski space) is \R{3,1} (or \R{d-1,1}).  Our coordinates are

\begin{dmath}\label{eqn:qftLecture2:340}
(c t, x^1, x^2, x^3) = (c t, \Br).
\end{dmath}

Here, we've scaled the time scale by \( c \) so that we measure time and space in the same dimensions.  We write this as

\begin{dmath}\label{eqn:qftLecture2:360}
x^\mu = (x^0, x^1, x^2, x^3),
\end{dmath}
%
where \( \mu = 0, 1, 2, 3 \), and call this a ``4-vector''.  These are called the space-time coordinates of an event, which tell us where and when an event occurs.

For two events whose spacetime coordinates differ by \( dx^0, dx^1, dx^2, dx^3 \) we introduce the notion of a space time \underline{interval}

\begin{dmath}\label{eqn:qftLecture2:380}
ds^2 = c^2 dt^2
- (dx^1)^2
- (dx^2)^2
- (dx^3)^2
=
\sum_{\mu, \nu = 0}^3 g_{\mu\nu} dx^\mu dx^\nu
\end{dmath}

Here \( g_{\mu\nu} \) is the Minkowski space metric, an object with two indexes that run from 0-3.  i.e. this is a diagonal matrix

\begin{dmath}\label{eqn:qftLecture2:400}
g_{\mu\nu} \sim
\begin{bmatrix}
1 & 0 & 0 & 0 \\
0 & -1 & 0 & 0 \\
0 & 0 & -1 & 0 \\
0 & 0 & 0 & -1 \\
\end{bmatrix}
\end{dmath}

i.e.
\begin{dmath}\label{eqn:qftLecture2:420}
\begin{aligned}
g_{00} &= 1 \\
g_{11} &= -1 \\
g_{22} &= -1 \\
g_{33} &= -1 \\
\end{aligned}
\end{dmath}

We will use the Einstein summation convention, where any repeated upper and lower indexes are considered summed over.  That is \cref{eqn:qftLecture2:380} is written with an implied sum
\begin{dmath}\label{eqn:qftLecture2:440}
ds^2 = g_{\mu\nu} dx^\mu dx^\nu.
\end{dmath}

Explicit expansion:
\begin{dmath}\label{eqn:qftLecture2:460}
ds^2 = g_{\mu\nu} dx^\mu dx^\nu
=
g_{00} dx^0 dx^0
+g_{11} dx^1 dx^1
+g_{22} dx^2 dx^2
+g_{33} dx^3 dx^3
=
(1) dx^0 dx^0
+ (-1) dx^1 dx^1
+ (-1) dx^2 dx^2
+ (-1) dx^3 dx^3.
\end{dmath}

Recall that rotations (with orthogonal matrix representations) are transformations that leave the dot product unchanged, that is

\begin{dmath}\label{eqn:qftLecture2:480}
(R \Bx) \cdot (R \By)
= \Bx^\T R^\T R \By
= \Bx^\T \By
= \Bx \cdot \By,
\end{dmath}
%
where \( R \) is a rotation orthogonal 3x3 matrix.  The set of such transformations that leave the dot product unchanged have orthonormal matrix representations \( R^\T R = 1 \).  We call the set of such transformations that have unit determinant the \(\SO{3}\) group.

We call a Lorentz transformation, if it is a linear transformation acting on 4 vectors that leaves the spacetime interval (i.e. the inner product of 4 vectors) invariant.  That is, a transformation that leaves
\begin{dmath}\label{eqn:qftLecture2:500}
x^\mu y^\nu g_{\mu\nu} = x^0 y^0 - x^1 y^1 - x^2 y^2 - x^3 y^3
\end{dmath}
unchanged.

Suppose that transformation has a 4x4 matrix form

\begin{dmath}\label{eqn:qftLecture2:520}
{x'}^\mu = {\Lambda^\mu}_\nu x^\nu
\end{dmath}

For an example of a possible \( \Lambda \), consider the transformation sketched in
\cref{fig:Lecture2:Lecture2Fig3}.
\imageFigure{../figures/phy2403-quantum-field-theory/Lecture2Fig3}{Boost transformation.}{fig:Lecture2:Lecture2Fig3}{0.2}
We know that boost has the form
\begin{dmath}\label{eqn:qftLecture2:540}
\begin{aligned}
x &= \frac{x' + v t'}{\sqrt{1 - v^2/c^2}} \\
y &= y' \\
z &= z' \\
t &= \frac{t' + (v/c^2) x'}{\sqrt{1 - v^2/c^2}} \\
\end{aligned}
\end{dmath}
(this is a boost along the x-axis, not y as I'd drawn),
or
\begin{dmath}\label{eqn:qftLecture2:560}
\begin{bmatrix}
c t \\
x \\
y \\
z
\end{bmatrix}
=
\begin{bmatrix}
\inv{\sqrt{1 - v^2/c^2}} & \frac{v/c}{\sqrt{1 - v^2/c^2}} & 0 & 0 \\
\frac{v/c}{\sqrt{1 - v^2/c^2}} & \frac{1}{\sqrt{1 - v^2/c^2}} & 0 & 0 \\
0 & 0 & 1 & 0 \\
0 & 0 & 0 & 1 \\
\end{bmatrix}
\begin{bmatrix}
c t' \\
x' \\
y' \\
z'
\end{bmatrix}
\end{dmath}

Other examples include rotations (\({\lambda^0}_0 = 1\) zeros in \( {\lambda^0}_k, {\lambda^k}_0 \), and a rotation matrix in the remainder.)
% submatrix:
%\begin{dmath}\label{eqn:qftLecture2:580}
%\begin{bmatrix}
%1 & 0 & 0 & 0
%0 &
%0 &    R
%0 &
%\end{bmatrix}
%\end{dmath}

Back to Lorentz transformations (\(\text{SO}(1,3)^+\)), let
\begin{dmath}\label{eqn:qftLecture2:600}
\begin{aligned}
{x'}^\mu    &= {\Lambda^\mu}_\nu x^\nu \\
{y'}^\kappa &= {\Lambda^\kappa}_\rho y^\rho
\end{aligned}
\end{dmath}

The dot product
\begin{dmath}\label{eqn:qftLecture2:620}
g_{\mu \kappa}
{x'}^\mu
{y'}^\kappa
=
g_{\mu \kappa}
{\Lambda^\mu}_\nu
{\Lambda^\kappa}_\rho
x^\nu
y^\rho
=
g_{\nu\rho}
x^\nu
y^\rho,
\end{dmath}
where the last step introduces the invariance requirement of the transformation.  That is

%\begin{dmath}\label{eqn:qftLecture2:640}
\boxedEquation{eqn:qftLecture2:640}{
g_{\nu\rho}
=
g_{\mu \kappa}
{\Lambda^\mu}_\nu
{\Lambda^\kappa}_\rho.
}
%\end{dmath}

\paragraph{Upper and lower indexes}
\index{upper indexes}
\index{lower indexes}

We've defined

\begin{dmath}\label{eqn:qftLecture2:660}
x^\mu = (t, x^1, x^2, x^3)
\end{dmath}

We could also define a four vector with lower indexes
\begin{dmath}\label{eqn:qftLecture2:680}
x_\nu = g_{\nu\mu} x^\mu = (t, -x^1, -x^2, -x^3).
\end{dmath}
That is
\begin{dmath}\label{eqn:qftLecture2:700}
\begin{aligned}
x_0 &= x^0 \\
x_1 &= -x^1 \\
x_2 &= -x^2 \\
x_3 &= -x^3.
\end{aligned}
\end{dmath}

which allows us to write the dot product as simply \( x^\mu y_\mu \).

We can also define a metric tensor with upper indexes

\begin{dmath}\label{eqn:qftLecture2:401}
g^{\mu\nu} \sim
\begin{bmatrix}
1 & 0 & 0 & 0 \\
0 & -1 & 0 & 0 \\
0 & 0 & -1 & 0 \\
0 & 0 & 0 & -1 \\
\end{bmatrix}
\end{dmath}
This is the inverse matrix of \( g_{\mu\nu} \), and it satisfies
\begin{dmath}\label{eqn:qftLecture2:720}
g^{\mu \nu} g_{\nu\rho} = {\delta^\mu}_\rho
\end{dmath}

Exercise: Check:
\begin{dmath}\label{eqn:qftLecture2:740}
g_{\mu\nu} x^\mu y^\nu = x_\nu y^\nu = x^\nu y_\nu
= g^{\mu\nu} x_\mu y_\nu = {\delta^\mu}_\nu x_\mu y^\nu
\end{dmath}

Class ended around this point, but it appeared that we were heading this direction:

Returning to the Lorentz invariant and multiplying both sides of
\cref{eqn:qftLecture2:640} with an inverse Lorentz transformation \( \Lambda^{-1} \), we find
\begin{dmath}\label{eqn:qftLecture2:760}
g_{\nu\rho}
{\lr{\Lambda^{-1}}^\rho}_\alpha
=
g_{\mu \kappa}
{\Lambda^\mu}_\nu
{\Lambda^\kappa}_\rho
{\lr{\Lambda^{-1}}^\rho}_\alpha
=
g_{\mu \kappa}
{\Lambda^\mu}_\nu
{\delta^\kappa}_\alpha
=
g_{\mu \alpha}
{\Lambda^\mu}_\nu,
\end{dmath}
or
\begin{dmath}\label{eqn:qftLecture2:780}
\lr{\Lambda^{-1}}_{\nu \alpha} = \Lambda_{\alpha \nu}.
\end{dmath}
This is clearly analogous to \( R^\T = R^{-1} \), although the index notation obscures things considerably.  Prof. Poppitz said that next week this would all lead to showing that the determinant of any Lorentz transformation was \( \pm 1 \).

For what it's worth, it seems to me that this index notation makes life a lot harder than it needs to be, at least for a matrix related question (i.e. determinant of the transformation).  In matrix/column-(4)-vector notation, let \(x' = \Lambda x, y' = \Lambda y\) be two four vector transformations, then
\begin{dmath}\label{eqn:qftLecture2:800}
x' \cdot y' = {x'}^T G y' = (\Lambda x)^T G \Lambda y = x^T ( \Lambda^T G \Lambda) y = x^T G y.
\end{dmath}
so
\boxedEquation{eqn:qftLecture2:820}{
\Lambda^T G \Lambda = G.
}
Taking determinants of both sides gives \(-(det(\Lambda))^2 = -1\), and thus \(det(\Lambda) = \pm 1\).
%}

   %\chapter{Lorentz transformations and a scalar action.}
      %
% Copyright � 2017 Peeter Joot.  All Rights Reserved.
% Licenced as described in the file LICENSE under the root directory of this GIT repository.
%
\input{../latex/blogpost.tex}
\renewcommand{\basename}{qft3}
\renewcommand{\dirname}{notes/phy2403/}
\newcommand{\keywords}{PHY2403H}
\input{../latex/peeter_prologue_print2.tex}

%\usepackage{phy2403}
\usepackage{peeters_braket}
%\usepackage{peeters_layout_exercise}
\usepackage{peeters_figures}
\usepackage{mathtools}
\usepackage{siunitx}
\usepackage{macros_cal}

\beginArtNoToc
\generatetitle{PHY2403H Quantum Field Theory.  Lecture 3: Lorentz transformations and a scalar action.  Taught by Prof.\ Erich Poppitz}
%\chapter{Lorentz transformations and a scalar action.}
\label{chap:qft3}

\paragraph{Disclaimer}

Peeter's lecture notes from class.  These may be incoherent and rough.

These are notes for the UofT course PHY2403H, Quantum Field Theory, taught by Prof. Erich Poppitz, covering \textchapref{{1}} \citep{peskin1995introduction} content.

\section{Determinant of Lorentz transformations}
%SO(1,d-1),d=3

We require that Lorentz transformations leave the dot product invariant, that is \( x \cdot y = x' \cdot y' \), or
\begin{dmath}\label{eqn:qftLecture3:20}
x^\mu g_{\mu\nu} y^\nu = {x'}^\mu g_{\mu\nu} {y'}^\nu.
\end{dmath}
Explicitly, with coordinate transformations
\begin{dmath}\label{eqn:qftLecture3:40}
\begin{aligned}
{x'}^\mu &= {\Lambda^\mu}_\rho x^\rho \\
{y'}^\mu &= {\Lambda^\mu}_\rho y^\rho
\end{aligned}
\end{dmath}
such a requirement is equivalent to demanding that
\begin{dmath}\label{eqn:qftLecture3:500}
x^\mu g_{\mu\nu} y^\nu
=
{\Lambda^\mu}_\rho x^\rho
g_{\mu\nu}
{\Lambda^\nu}_\kappa y^\kappa
=
x^\mu
{\Lambda^\alpha}_\mu
g_{\alpha\beta}
{\Lambda^\beta}_\nu
y^\nu,
\end{dmath}
or
\begin{dmath}\label{eqn:qftLecture3:60}
g_{\mu\nu}
%= {\Lambda^k}_\mu g_{\kappa \rho} {\Lambda^\rho}_\nu
=
{\Lambda^\alpha}_\mu
g_{\alpha\beta}
{\Lambda^\beta}_\nu
\end{dmath}

multiplying by the inverse we find
\begin{dmath}\label{eqn:qftLecture3:200}
g_{\mu\nu}
{\lr{\Lambda^{-1}}^\nu}_\lambda
=
%{\Lambda^\kappa}_\mu g_{\kappa \rho} {\Lambda^\rho}_\nu
{\Lambda^\alpha}_\mu
g_{\alpha\beta}
{\Lambda^\beta}_\nu
{\lr{\Lambda^{-1}}^\nu}_\lambda
=
%{\Lambda^\kappa}_\mu g_{\kappa \lambda}
{\Lambda^\alpha}_\mu
g_{\alpha\lambda}
\end{dmath}
or as elements of matrices
\begin{dmath}\label{eqn:qftLecture3:220}
(G \Lambda^{-1})_{\mu\lambda}
=
(\Lambda G)_{\lambda\mu}
=
((G \Lambda)^\T)_{\lambda\mu}
\end{dmath}
or
\begin{dmath}\label{eqn:qftLecture3:80}
G \Lambda^{-1}
=
(G \Lambda)^\T.
\end{dmath}

Taking determinants, using the fact that the determinant of a product is the product of determinants, we find
\begin{dmath}\label{eqn:qftLecture3:100}
\cancel{det(G) }
det(\Lambda^{-1})
=
\cancel{det(G)} det(\Lambda),
\end{dmath}
or
\begin{dmath}\label{eqn:qftLecture3:120}
det(\Lambda)^2 = 1,
\end{dmath}
or
\( det(\Lambda)^2 = \pm 1 \).  We will generally ignore the case of reflections in spacetime that have a negative determinant.

Smart-alec Peeter pointed out that we can do the same thing easier in matrix notation
\begin{dmath}\label{eqn:qftLecture3:140}
\begin{aligned}
x' &= \Lambda x \\
y' &= \Lambda y
\end{aligned}
\end{dmath}
where
\begin{dmath}\label{eqn:qftLecture3:160}
x' \cdot y'
=
(x')^\T G y'
=
x^\T \Lambda^\T G \Lambda y,
\end{dmath}
which we require to be \( x \cdot y = x^\T G y \) for all four vectors \( x, y \), that is
\begin{dmath}\label{eqn:qftLecture3:180}
\Lambda^\T G \Lambda = G.
\end{dmath}
We can find the result \cref{eqn:qftLecture3:120} immediately without having to first translate from index notation to matrices.

\section{Field theory}

The electrostatic potential is an example of a scalar field \( \phi(\Bx) \) unchanged by SO(3) rotations
\begin{equation}\label{eqn:qftLecture3:240}
\Bx \rightarrow \Bx' = O \Bx,
\end{equation}
that is
\begin{dmath}\label{eqn:qftLecture3:260}
\phi'(\Bx') = \phi(\Bx).
\end{dmath}
Here \( \phi'(\Bx') \) is the value of the (elecrostatic) scalar potential in a primed frame.

However, the electrostatic field is not invariant under Lorentz transformation.
We postulate that there is some scalar field
\begin{dmath}\label{eqn:qftLecture3:280}
\phi'(x') = \phi(x),
\end{dmath}
where \( x' = \Lambda x \) is an SO(1,3) transformation.
There are actually no stable particles (fields that persist at long distances) described by Lorentz scalar fields, although there are some unstable scalar fields such as the
Higgs, Pions, and Kaons.
However,
much of our homework and discussion will be focused on scalar fields, since
they are the
easiest to start with.

We need to first understand how derivatives \( \partial_\mu \phi(x) \) transform.  Using the chain rule
\begin{dmath}\label{eqn:qftLecture3:300}
\PD{x^\mu}{\phi(x)} =
\PD{x^\mu}{\phi'(x')}
=
\PD{{x'}^\nu}{\phi'(x')}
\PD{{x}^\mu}{{x'}^\nu}
=
\PD{{x'}^\nu}{\phi'(x')}
\partial_\mu \lr{
{\Lambda^\nu}_\rho x^\rho
}
=
\PD{{x'}^\nu}{\phi'(x')}
{\Lambda^\nu}_\mu
=
\PD{{x'}^\nu}{\phi(x)}
{\Lambda^\nu}_\mu.
\end{dmath}
Multiplying by the inverse \( {\lr{\Lambda^{-1}}^\mu}_\kappa \) we get
\begin{dmath}\label{eqn:qftLecture3:320}
\PD{{x'}^\kappa}{}
=
{\lr{\Lambda^{-1}}^\mu}_\kappa \PD{x^\mu}{}
\end{dmath}

This should be familiar to you, and is an analogue of the transformation of the
\begin{dmath}\label{eqn:qftLecture3:340}
d\Br \cdot \spacegrad_\Br
=
d\Br' \cdot \spacegrad_{\Br'}.
\end{dmath}

\section{Actions}
We will start with a classical action, and quantize to determine a QFT.
In mechanics we have the particle position \( q(t) \), which is a classical field in 1+0 time and space dimensions.  Our action is
\begin{dmath}\label{eqn:qftLecture3:360}
S
= \int dt \LL(t)
= \int dt \lr{
\inv{2} \dot{q}^2 - V(q)
}.
\end{dmath}
This action depends on the position of the particle that is local in time.
You could imagine that we have a more complex action where the action depends on future or past times
\begin{dmath}\label{eqn:qftLecture3:380}
S
= \int dt' q(t') K( t' - t ),
\end{dmath}
but we don't seem to find such actions in classical mechanics.

\paragraph{Principles determining the form of the action.}
\begin{itemize}
\item relativity (action is invariant under Lorentz transformation)
\item locality (action depends on fields and the derivatives at given \((t, \Bx)\).
\item Gauge principle (the action should be invariant under gauge transformation).  We won't discuss this in detail right now since we will start with studying scalar fields.
For Maxwell's equations that is (Fixme)
\( \phi \rightarrow \phi + \dot{\chi}, \BA \rightarrow \BA - \spacegrad \chi \), or in four vector notation
(Fixme).
%\( A^\mu \rightarrow A^\mu + \partial ... \), or
\end{itemize}

Suppose we have a real scalar field \( \phi(x) \) where \( x \in \bbR^{1,d-1} \).  We will be integrating over space and time \( \int dt d^{d-1} x \) which we will write as \( \int d^d x \).  Our action is
\begin{dmath}\label{eqn:qftLecture3:400}
S = \int d^d x \lr{ \text{Some action density to be determined } }
\end{dmath}
The analogue of \( \dot{q}^2 \) is
\begin{dmath}\label{eqn:qftLecture3:420}
\lr{ \PD{x^\mu}{\phi} }
\lr{ \PD{x^\nu}{\phi} }
g^{\mu\nu}
=
(\partial_\mu \phi) (\partial_\nu \phi) g^{\mu\nu}
= \partial^\mu \phi \partial_\mu \phi.
\end{dmath}
This has both time and spatial components, that is
\begin{dmath}\label{eqn:qftLecture3:440}
\partial^\mu \phi \partial_\mu \phi =
\dotphi^2 - (\spacegrad \phi)^2,
\end{dmath}
so the desired simplest scalar action is
\begin{dmath}\label{eqn:qftLecture3:460}
S = \int d^d x \lr{ \dotphi^2 - (\spacegrad \phi)^2 }.
\end{dmath}
The measure transforms using a Jacobian, which we have seen is the Lorentz transform matrix, and has unit determinant
\begin{dmath}\label{eqn:qftLecture3:480}
d^d x' = d^d x \Abs{ det( {\Lambda^\mu}_\nu) } = d^d x
\end{dmath}

\section{Problems.}

\makeproblem{Matrix elements of Lorentz/metric product.}{problem:qftLecture3:520}{
Justify \cref{eqn:qftLecture3:220} explicitly.
} % problem

\makeanswer{problem:qftLecture3:520}{
Fixme.
} % answer

\EndArticle
%\EndNoBibArticle

      \section{Problems.}
         %
% Copyright � 2015 Peeter Joot.  All Rights Reserved.
% Licenced as described in the file LICENSE under the root directory of this GIT repository.
%
\makeoproblem{Lorentz transformation.}
{qft:LukeProblemSet1:1}
{2015 ps1.1}
{

A Lorentz transformation \( x^\mu \rightarrow {x'}^\mu = {\wedge^\mu}_\nu x^\nu \) is such that it preserves the Minkowski metric \( \eta_{\mu\nu} \) meaning that \( \eta_{\mu\nu} x^\mu x^\nu = \eta_{\mu\nu} {x'}^\mu {x'}^\nu \) for all \( x \).

\makesubproblem{}{qft:LukeProblemSet1:1a}
Show that this implies that
\begin{dmath}\label{eqn:LukeProblemSet1Problem1:20}
\eta_{\mu\nu}
=
\eta_{\sigma\tau} {\wedge^\sigma}_\mu {\wedge^\tau}_\nu.
\end{dmath}
\makesubproblem{}{qft:LukeProblemSet1:1b}
Use this result to show that an infinitesimal transformation of the form
\begin{dmath}\label{eqn:LukeProblemSet1Problem1:40}
{\wedge^\mu}_\nu = {\delta^\mu}_\nu + {\omega^\mu}_\nu
\end{dmath}
is a Lorentz transformation when \( \omega^{\mu\nu} \) is antisymmetric i.e. \( \omega^{\mu\nu} = -\omega^{\nu\mu} \). (Note that there an antisymmetric \( 4 \times 4 \) matrix has six parameters, as does a Lorentz transformation - 3 rotations and 3 boosts - so the counting works out).
\makesubproblem{}{qft:LukeProblemSet1:1c}
Write down the matrix form for \( {\omega^\mu}_\nu \) that corresponds to a rotation through an infinitesimal angle \( \theta \) about the \( x^3\)-axis.
\makesubproblem{}{qft:LukeProblemSet1:1d}
Do the same for a boost along the \( x^1 \)-axis by an infinitesimal velocity \(v\).
} % makeproblem

\makeanswer{qft:LukeProblemSet1:1}{
\withproblemsetsParagraph{
\makeSubAnswer{}{qft:LukeProblemSet1:1a}

%{x'}^\mu &= \lr{ {\delta^\mu}_\alpha + {\omega^\mu}_\alpha } x^\alpha \\
%{x'}_\mu &= \lr{ {\delta_\mu}^\beta + {\omega_\mu}^\beta } x_\beta
%{x'}^\mu &= {\wedge^\mu}_\alpha x^\alpha \\
%{x'}_\mu &= {\wedge_\mu}^\beta x_\beta,
%
The dot product of the transformed coordinates is

\begin{dmath}\label{eqn:qftProblemSet1Problem1:60}
\eta_{\mu\nu} {x'}^\mu {x'}^\nu
=
\eta_{\mu\nu}
{\wedge^\mu}_\alpha x^\alpha
{\wedge^\nu}_\beta x^\beta
=
\eta_{\sigma\tau}
{\wedge^\sigma}_\mu
{\wedge^\tau}_\nu
x^\mu
x^\nu,
\end{dmath}

where the last step is just a change of indexes \( \mu \rightarrow \sigma, \nu \rightarrow \tau, \alpha \rightarrow \mu, \beta \rightarrow \nu \).  The identity \cref{eqn:LukeProblemSet1Problem1:20} can be read off directly.

\makeSubAnswer{}{qft:LukeProblemSet1:1b}

\begin{dmath}\label{eqn:qftProblemSet1Problem1:80}
\eta_{\sigma\tau} {\wedge^\sigma}_\mu {\wedge^\tau}_\nu
=
\eta_{\sigma\tau}
\lr{ {\delta^\sigma}_\mu + {\omega^\sigma}_\mu }
\lr{ {\delta^\tau}_\nu + {\omega^\tau}_\nu }
=
\lr{ \eta_{\mu\tau} + \omega_{\tau\mu} }
\lr{ {\delta^\tau}_\nu + {\omega^\tau}_\nu }
=
  \eta_{\mu\tau} {\delta^\tau}_\nu
+ \eta_{\mu\tau} {\omega^\tau}_\nu
+ \omega_{\tau\mu} {\delta^\tau}_\nu
+ \omega_{\tau\mu} {\omega^\tau}_\nu
=
  \eta_{\mu\nu}
+ \omega_{\mu\nu}
+ \omega_{\nu\mu}
+ \omega_{\tau\mu} {\omega^\tau}_\nu
=
  \eta_{\mu\nu}
+ \omega_{\mu\nu}
- \omega_{\mu\nu}
+ O(\omega^2)
=
\eta_{\mu\nu}.
\end{dmath}

\makeSubAnswer{}{qft:LukeProblemSet1:1c}

With a \( \gamma_0^2 = 1, \gamma_k^2 = -1 \) metric, a rotation in the x-y plane around the z-axis can be written as

\begin{dmath}\label{eqn:qftProblemSet1Problem1:100}
\gamma_1 x^1 + \gamma_2 x^2
\rightarrow
\lr{ \gamma_1 x^1 + \gamma_2 x^2 } e^{\gamma_2 \gamma_1 \theta }
=
\lr{ \gamma_1 x^1 + \gamma_2 x^2 } \lr{ \cos \theta + \gamma_2 \gamma_1 \sin\theta }
=
\gamma_1 x^1 \cos\theta + \gamma_2 x^2 \cos\theta
+
\gamma_2 x^1 \sin\theta - \gamma_1 x^2 \sin\theta,
\end{dmath}

or
\begin{equation}\label{eqn:qftProblemSet1Problem1:120}
{
\begin{bmatrix}
x^1 \\
x^2 \\
\end{bmatrix}
}'
=
\begin{bmatrix}
\cos\theta & -\sin\theta \\
\sin\theta & \cos\theta
\end{bmatrix}
\begin{bmatrix}
x^1 \\
x^2 \\
\end{bmatrix},
\end{equation}

so in the small angle approximation, with a \( \gamma_0, \gamma_, \gamma_2, \gamma_3 \) basis, we have

\begin{dmath}\label{eqn:qftProblemSet1Problem1:140}
{\omega^\nu}_\mu
=
\begin{bmatrix}
0 & 0 & 0 & 0 \\
0 & 0 & -\theta & 0 \\
0 & \theta & 0 & 0 \\
0 & 0 & 0 & 0 \\
\end{bmatrix}.
\end{dmath}

\makeSubAnswer{}{qft:LukeProblemSet1:1d}

For the boost the rotation is also an exponential

\begin{dmath}\label{eqn:qftProblemSet1Problem1:160}
\gamma_1 x^1 + \gamma_0 x^0
\rightarrow
\lr{ \gamma_1 x^1 + \gamma_0 x^0 } e^{\gamma_0 \gamma_1 \alpha }
=
\lr{ \gamma_1 x^1 + \gamma_0 x^0 } \lr{ \cosh \alpha + \gamma_0 \gamma_1 \sinh\alpha }
=
\gamma_1 x^1 \cosh \alpha + \gamma_0 x^0 \cosh \alpha
+ \gamma_0 x^1 \sinh\alpha + \gamma_1 x^0 \sinh\alpha,
\end{dmath}

or
\begin{dmath}\label{eqn:qftProblemSet1Problem1:180}
{\begin{bmatrix}
x^0 \\
x^1 \\
\end{bmatrix} }'
=
\begin{bmatrix}
\cosh\alpha & \sinh\alpha \\
\sinh\alpha & \cosh\alpha \\
\end{bmatrix}
\begin{bmatrix}
x^0 \\
x^1 \\
\end{bmatrix}
\end{dmath}

The rapidity angle \( \alpha \) can be related to velocity by considering a spacetime difference in position

\begin{dmath}\label{eqn:qftProblemSet1Problem1:200}
\Delta
{\begin{bmatrix}
x^0 \\
x^1 \\
\end{bmatrix} }'
=
\begin{bmatrix}
\cosh\alpha \Delta x^0 + \sinh\alpha \Delta x^1 \\
\sinh\alpha \Delta x^0 + \cosh\alpha \Delta x^1 \\
\end{bmatrix},
\end{dmath}

For a particle fixed at the origin in the unprimed frame (i.e. \( \Delta x^1 = 0\, \forall t \)), we have
\begin{dmath}\label{eqn:qftProblemSet1Problem1:220}
\Delta
{\begin{bmatrix}
x^0 \\
x^1 \\
\end{bmatrix} }'
=
\begin{bmatrix}
\cosh\alpha \Delta x^0 \\
\sinh\alpha \Delta x^0
\end{bmatrix}.
\end{dmath}

In particular
\begin{dmath}\label{eqn:qftProblemSet1Problem1:240}
\frac{\Delta {x'}^1}{
\Delta {x'}^0}
=
\tanh \alpha.
\end{dmath}

If the unprimed frame is moving at velocity \( v \) along the x-axis, then the primed frame is moving at \( -v \), or

\begin{dmath}\label{eqn:qftProblemSet1Problem1:260}
-v = \tanh \alpha.
\end{dmath}

Noting that \( \cosh^2 \alpha - \sinh^2 \alpha = 1 \),

\begin{dmath}\label{eqn:qftProblemSet1Problem1:280}
v^2 = \frac{ \sinh^2 \alpha }{ 1 + \sinh^2 \alpha },
\end{dmath}

so
\begin{dmath}\label{eqn:qftProblemSet1Problem1:300}
\sinh^2 \alpha \lr{ v^2 - 1 } = -v^2,
\end{dmath}

or
\begin{dmath}\label{eqn:qftProblemSet1Problem1:320}
\sinh\alpha = \pm \frac{v}{\sqrt{1 - v^2}}.
\end{dmath}

We also have

\begin{dmath}\label{eqn:qftProblemSet1Problem1:340}
\cosh^2\alpha = \frac{v^2}{1 - v^2} + 1 = \frac{1}{1 - v^2}.
\end{dmath}

Picking the negative sign in \cref{eqn:qftProblemSet1Problem1:320} to match \cref{eqn:qftProblemSet1Problem1:260}, we have

\begin{dmath}\label{eqn:qftProblemSet1Problem1:181}
{\begin{bmatrix}
x^0 \\
x^1 \\
\end{bmatrix} }'
=
\inv{\sqrt{1 - v^2}}
\begin{bmatrix}
1           & -v \\
- v         & 1
\end{bmatrix}
\begin{bmatrix}
x^0 \\
x^1 \\
\end{bmatrix}.
\end{dmath}

In the small velocity limit, this gives
\begin{dmath}\label{eqn:qftProblemSet1Problem1:360}
{\omega^\mu}_\nu =
\begin{bmatrix}
0 & -v & 0 & 0 \\
-v & 0 & 0 & 0 \\
0 & 0 & 0 & 0 \\
0 & 0 & 0 & 0 \\
\end{bmatrix}.
\end{dmath}

}
}

   \chapter{Classical field theory.}
      %
% Copyright © 2018 Peeter Joot.  All Rights Reserved.
% Licenced as described in the file LICENSE under the root directory of this GIT repository.
%
\section{Field theory.}

The electrostatic potential is an example of a scalar field \( \phi(\Bx) \) unchanged by \(\SO{3}\) rotations
\begin{equation}\label{eqn:qftLecture3:240}
\Bx \rightarrow \Bx' = O \Bx,
\end{equation}
that is
\begin{dmath}\label{eqn:qftLecture3:260}
\phi'(\Bx') = \phi(\Bx).
\end{dmath}
Here \( \phi'(\Bx') \) is the value of the (electrostatic) scalar potential in a primed frame.

However, the electrostatic field is not invariant under Lorentz transformation.
We postulate that there is some scalar field
\begin{dmath}\label{eqn:qftLecture3:280}
\phi'(x') = \phi(x),
\end{dmath}
where \( x' = \Lambda x \) is an \(\SO{1,3}\) transformation.
There are actually no stable particles (fields that persist at long distances) described by Lorentz scalar fields, although there are some unstable scalar fields such as the
Higgs, Pions, and Kaons.
However,
much of our homework and discussion will be focused on scalar fields, since
they are the
easiest to start with.

We need to first understand how derivatives \( \partial_\mu \phi(x) \) transform.  Using the chain rule
\begin{dmath}\label{eqn:qftLecture3:300}
\PD{x^\mu}{\phi(x)} =
\PD{x^\mu}{\phi'(x')}
=
\PD{{x'}^\nu}{\phi'(x')}
\PD{{x}^\mu}{{x'}^\nu}
=
\PD{{x'}^\nu}{\phi'(x')}
\partial_\mu \lr{
{\Lambda^\nu}_\rho x^\rho
}
=
\PD{{x'}^\nu}{\phi'(x')}
{\Lambda^\nu}_\mu
=
\PD{{x'}^\nu}{\phi(x)}
{\Lambda^\nu}_\mu.
\end{dmath}
Multiplying by the inverse \( {\lr{\Lambda^{-1}}^\mu}_\kappa \) we get
\begin{dmath}\label{eqn:qftLecture3:320}
\PD{{x'}^\kappa}{}
=
{\lr{\Lambda^{-1}}^\mu}_\kappa \PD{x^\mu}{}
\end{dmath}

This should be familiar to you, and is an analogue of the transformation of the
\begin{dmath}\label{eqn:qftLecture3:340}
d\Br \cdot \spacegrad_\Br
=
d\Br' \cdot \spacegrad_{\Br'}.
\end{dmath}

\section{Actions.}
\index{action}
We will start with a classical action, and quantize to determine a QFT.
In mechanics we have the particle position \( q(t) \), which is a classical field in 1+0 time and space dimensions.  Our action is
\begin{dmath}\label{eqn:qftLecture3:360}
S
= \int dt \LL(t)
= \int dt \lr{
\inv{2} \dot{q}^2 - V(q)
}.
\end{dmath}
This action depends on the position of the particle that is local in time.
You could imagine that we have a more complex action where the action depends on future or past times
\begin{dmath}\label{eqn:qftLecture3:380}
S
= \int dt' q(t') K( t' - t ),
\end{dmath}
but we don't seem to find such actions in classical mechanics.

\section{Principles determining the form of the action.}
%\paragraph{Principles determining the form of the action.}
\begin{itemize}
\item relativity (action is invariant under Lorentz transformation)
\item locality (action depends on fields and the derivatives at given \((t, \Bx)\).
\item Gauge principle (the action should be invariant under gauge transformation).  We won't discuss this in detail right now since we will start with studying scalar fields.
Recall that for Maxwell's equations a gauge transformation has the form
\begin{dmath}\label{eqn:qftLecture3:520}
\phi \rightarrow \phi + \dot{\chi}, \BA \rightarrow \BA - \spacegrad \chi.
\end{dmath}
\end{itemize}

Suppose we have a real scalar field \( \phi(x) \) where \( x \in \bbR^{1,d-1} \).  We will be integrating over space and time \( \int dt d^{d-1} x \) which we will write as \( \int d^d x \).  Our action is
\begin{dmath}\label{eqn:qftLecture3:400}
S = \int d^d x \lr{ \text{Some action density to be determined } }
\end{dmath}
The analogue of \( \dot{q}^2 \) is
\begin{dmath}\label{eqn:qftLecture3:420}
\lr{ \PD{x^\mu}{\phi} }
\lr{ \PD{x^\nu}{\phi} }
g^{\mu\nu}
=
(\partial_\mu \phi) (\partial_\nu \phi) g^{\mu\nu}
= \partial^\mu \phi \partial_\mu \phi.
\end{dmath}
This has both time and spatial components, that is
\begin{dmath}\label{eqn:qftLecture3:440}
\partial^\mu \phi \partial_\mu \phi =
\dotphi^2 - (\spacegrad \phi)^2,
\end{dmath}
so the desired simplest scalar action is
\begin{dmath}\label{eqn:qftLecture3:460}
S = \int d^d x \lr{ \dotphi^2 - (\spacegrad \phi)^2 }.
\end{dmath}
The measure transforms using a Jacobian, which we have seen is the Lorentz transform matrix, and has unit determinant
\begin{equation}\label{eqn:qftLecture3:480}
d^d x' = d^d x \Abs{ det( \Lambda^{-1} ) } = d^d x.
\end{equation}


      \section{Problems.}
         %
% Copyright © 2018 Peeter Joot.  All Rights Reserved.
% Licenced as described in the file LICENSE under the root directory of this GIT repository.
%
%\section{Problems.}

%\makeproblem{Matrix elements of Lorentz/metric product.}{problem:qftLecture3:520}{
%Justify \cref{eqn:qftLecture3:220} explicitly.
%} % problem
%
%\makeanswer{problem:qftLecture3:520}{
%Fixme.
%} % answer
%
\makeproblem{Four vector form of the Maxwell gauge transformation.}{problem:qftLecture3:540}{
Show that the transformation
\begin{dmath}\label{eqn:qftLecture3:580}
A^\mu \rightarrow A^\mu + \partial^\mu \chi
\end{dmath}
is the desired four-vector form of the gauge transformation \cref{eqn:qftLecture3:520}, that is
\begin{equation}\label{eqn:qftLecture3:540}
j^\nu = \partial_\mu {F'}^{\mu\nu}
= \partial_\mu F^{\mu\nu}.
\end{equation}
Also relate this four-vector gauge transformation to the spacetime split.
} % problem

\makeanswer{problem:qftLecture3:540}{
\begin{dmath}\label{eqn:qftLecture3:560}
\partial_\mu {F'}^{\mu\nu}
=
\partial_\mu \lr{ \partial^\mu {A'}^\nu - \partial_\nu {A'}^\mu }
=
\partial_\mu \lr{
  \partial^\mu \lr{ A^\nu + \partial^\nu \chi }
- \partial_\nu \lr{ A^\mu + \partial^\mu \chi }
}
=
\partial_\mu {F}^{\mu\nu}
+
\partial_\mu \partial^\mu \partial^\nu \chi
-
\partial_\mu \partial^\nu \partial^\mu \chi
=
\partial_\mu {F}^{\mu\nu},
\end{dmath}
by equality of mixed partials.
Expanding \cref{eqn:qftLecture3:580} explicitly we find
\begin{dmath}\label{eqn:qftLecture3:600}
{A'}^\mu = A^\mu + \partial^\mu \chi,
\end{dmath}
which is
\begin{dmath}\label{eqn:qftLecture3:620}
\begin{aligned}
\phi' = {A'}^0 &= A^0 + \partial^0 \chi = \phi + \dot{\chi} \\
\BA' \cdot \Be_k = {A'}^k &= A^k + \partial^k \chi = \lr{ \BA - \spacegrad \chi } \cdot \Be_k.
\end{aligned}
\end{dmath}
The last of which can be written in vector notation as \( \BA' = \BA - \spacegrad \chi \).
} % answer


         %
% Copyright � 2015 Peeter Joot.  All Rights Reserved.
% Licenced as described in the file LICENSE under the root directory of this GIT repository.
%
\makeoproblem{One dimensional string.}
{qft:LukeProblemSet1:3}
{2015 ps1.3}
{

A string of length \( a \), mass per unit length \( \sigma \) and under tension \(T\) is fixed at each end. The Lagrangian governing the time evolution of the transverse displacement \( y(x,t) \) is
\begin{equation}\label{eqn:LukeProblemSet1Problem3:20}
L = \int_0^a dx \lr{ \frac{\sigma}{2} \lr{ \PD{t}{y} }^2 - \frac{T}{2} \lr{ \PD{x}{y} }^2 },
\end{equation}
where \( x \) identifies position along the string from one end point.
\makesubproblem{}{qft:LukeProblemSet1:3a}
By expressing the displacement as a sine series Fourier expansion of the form
\begin{equation}\label{eqn:LukeProblemSet1Problem3:40}
y(x,t) = \sqrt{\frac{2}{a}} \sum_{n=1}^\infty \sin\lr{ \frac{ n \pi x }{a} } q_n(t).
\end{equation}

Show that the Lagrangian becomes
\begin{equation}\label{eqn:LukeProblemSet1Problem3:60}
L = \sum_{n=1}^\infty \lr{ \frac{\sigma}{2} \dot{q}_n^2 - \frac{T}{2} \lr{ \frac{ n \pi }{2} }^2 q_n^2 } .
\end{equation}
\makesubproblem{}{qft:LukeProblemSet1:3b}
Derive the equations of motion. Hence, show that the string is equivalent to an infinite set of decoupled
harmonic oscillators, and find their frequencies.
} % makeproblem
\makeanswer{qft:LukeProblemSet1:3}{
\withproblemsetsParagraph{
\makeSubAnswer{}{qft:LukeProblemSet1:3a}
First observe that the functions \( \braket{x}{n} = \sqrt{\frac{2}{a}} \sin\lr{ n \pi x/a } \) are orthonormal over the \( [0,a] \) domain.
\begin{equation}\label{eqn:qftProblemSet1Problem3:80}
\begin{aligned}
\braket{ n }{n}
&=
\frac{2}{a}
\int_0^a \sin^2\lr{ n \pi x/a } dx \\
&=
2
\int_0^1 \sin^2\lr{ n \pi u } du \\
&=
\int_0^1 \lr{ 1 - \cos\lr{ 2 n \pi u } } du \\
&=
1,
\end{aligned}
\end{equation}
and for \( n \ne m \)
\begin{equation}\label{eqn:qftProblemSet1Problem3:100}
\begin{aligned}
\braket{n}{m}
&=
\frac{2}{a}
\int_0^a \sin\lr{ n \pi x/a } \sin\lr{ m \pi x/a } dx \\
&=
2
\int_0^1 \sin\lr{ n \pi u } \sin\lr{ m \pi u } du \\
&=
-\inv{2}
\int_0^1
\lr{ e^{i n \pi u} - e^{-i n \pi u} }
\lr{ e^{i m \pi u} - e^{-i m \pi u} }
du \\
&=
-
\int_0^1
du
\lr{
\cos( ( n + m) \pi u ) - \cos( (m - n) \pi u )
} \\
&= 0,
\end{aligned}
\end{equation}
so
\begin{equation}\label{eqn:qftProblemSet1Problem3:120}
L
=
\int_0^a dx
\frac{2}{a}
\sum_{m,n = 1}^\infty
\sin\lr{ \frac{ n \pi x }{a} }
\sin\lr{ \frac{ m \pi x }{a} }
\lr{ \frac{\sigma}{2} \dot{q}_n \dot{q}_m
- \frac{T}{2}
\lr{ \frac{n \pi}{a} }
\lr{ \frac{m \pi}{a} } q_n q_m
}
=
\sum_{m,n = 1}^\infty \delta_{nm}
\lr{ \frac{\sigma}{2} \dot{q}_n \dot{q}_m
- \frac{T}{2}
\lr{ \frac{n \pi}{a} }
\lr{ \frac{m \pi}{a} }
q_n q_m
}
=
\sum_{n = 1}^\infty
\lr{ \frac{\sigma}{2} \lr{\dot{q}_n}^2
- \frac{T}{2}
\lr{ \frac{n \pi}{a} }^2 q_n^2
}.
\end{equation}
\makeSubAnswer{}{qft:LukeProblemSet1:3b}
We have an Euler-Lagrange equation for each \( q_n \).  The conjugate momenta are
\begin{equation}\label{eqn:qftProblemSet1Problem3:140}
\PD{\dot{q}_n}{L} = \sigma \dot{q}_n.
\end{equation}

We also have
\begin{equation}\label{eqn:qftProblemSet1Problem3:160}
\PD{q_n}{L} = - T \lr{ \frac{n \pi}{a} }^2 q_n,
\end{equation}
%
so we have
\begin{equation}\label{eqn:qftProblemSet1Problem3:180}
\ddot{q}_n = - \frac{T}{\sigma} \lr{ \frac{n \pi}{a} }^2 q_n.
\end{equation}

These have solutions
\begin{equation}\label{eqn:qftProblemSet1Problem3:200}
q_n(t) = A_{\pm} \exp\lr{ \pm i \sqrt{ \frac{T}{\sigma} } \frac{n \pi}{a} t }.
\end{equation}

The angular frequencies are
\begin{equation}\label{eqn:qftProblemSet1Problem3:220}
\omega_n = 2 \pi \nu_n = \sqrt{ \frac{T}{\sigma} } \frac{n \pi}{a},
\end{equation}
%
so the frequencies are
\begin{equation}\label{eqn:qftProblemSet1Problem3:240}
\nu_n = \sqrt{ \frac{T}{\sigma} } \frac{n }{2 a}.
\end{equation}
}
}

         %
% Copyright � 2015 Peeter Joot.  All Rights Reserved.
% Licenced as described in the file LICENSE under the root directory of this GIT repository.
%
\makeproblem{Maxwell Lagrangian with mass term.}{qft:problemSet1:6}{
(You can probably find this worked out in lots of places, but it's good practice with working with four-vectors, so I strongly encourage you to do it yourself!)
Consider the Lagrangian for a real vector field \( A^\mu \):

\begin{dmath}\label{eqn:qftproblemSet1Problem6:20}
\LL
=
- \inv{2}
\partial_\alpha A_\beta(x) \partial^\alpha A^\beta(x)
+
\inv{2}
\partial_\alpha A^\alpha(x)
\partial_\beta A^\beta(x)
+
\frac{\mu^2}{2} A_\alpha(x) A^\alpha(x).
\end{dmath}

\makesubproblem{}{qft:problemSet1:6a}
Show that this leads to the field equations

\begin{dmath}\label{eqn:qftproblemSet1Problem6:40}
\lr{ g_{\alpha\beta} \lr{ \delSquaredBox + \mu^2 } - \partial_\alpha \partial_\beta } A^\beta(x) = 0,
\end{dmath}

and that the field \( A^\alpha(x) \) satisfies the Lorentz condition

\begin{dmath}\label{eqn:qftproblemSet1Problem6:60}
\partial_\alpha A^\alpha(x) = 0.
\end{dmath}

(NB: If you are not careful with your indices and Einstein summation convention you will get yourself hopelessly messed up here.)

\makesubproblem{}{qft:problemSet1:6b}

Consider the limiting case of a massless field, \( \mu \rightarrow 0 \), and identify the field \( A^\mu \) with the scalar and vector potentials of electrodynamics: \( A^\mu = (\phi, \BA) \), where

\begin{equation}\label{eqn:qftproblemSet1Problem6:80}
\BE = - \spacegrad \phi - \PD{t}{\BA}
\end{equation}
\begin{equation}\label{eqn:qftproblemSet1Problem6:100}
\BB = \spacegrad \cross \BA.
\end{equation}

Show that the field equations reproduce two of Maxwell's equations, and that the other two hold as identities given the definitions of \( \BE \) and \( \BB \) in terms of \( \phi \) and \( \BA \).
} % makeproblem

\makeanswer{qft:problemSet1:6}{
\makeSubAnswer{}{qft:problemSet1:6a}

First rewrite the Lagrangian slightly

\begin{dmath}\label{eqn:qftproblemSet1Problem6:120}
\LL
=
- \inv{2}
\partial_\alpha A_\beta(x) \partial^\alpha A^\beta(x)
+
\inv{2} g_{\tau\beta}
\partial_\alpha A^\alpha(x)
\partial^\tau A^\beta(x)
+
\frac{\mu^2}{2} A_\alpha(x) A^\alpha(x),
\end{dmath}

to compute

\begin{dmath}\label{eqn:qftProblemSet1Problem6:140}
\partial^\mu \PD{\partial^\mu A^\nu}{\LL}
=
\partial^\mu
\lr{
-
\partial_\mu A_\nu
+
g_{\mu\nu}
\partial_\alpha A^\alpha(x)
}
=
\PD{A^\nu}{\LL}
=
+
\mu^2 A_\nu,
\end{dmath}

or
\begin{dmath}\label{eqn:qftProblemSet1Problem6:160}
0
=
-
\delSquaredBox A_\nu + \partial_\nu \partial_\alpha A^\alpha - \mu^2 A_\nu
=
\lr{ -
g_{\nu \alpha} \lr{ \delSquaredBox + \mu^2} + \partial_\nu \partial_\alpha } A^\alpha.
\end{dmath}

After a sign switch and change of indexes, we have the desired result.  Operating on this with \( \partial^\nu \) gives

\begin{dmath}\label{eqn:qftProblemSet1Problem6:180}
0 =
\lr{ -
\partial_\alpha \lr{ \delSquaredBox + \mu^2} + \delSquaredBox \partial_\alpha } A^\alpha
=
- \mu^2 \partial_\alpha A^\alpha.
\end{dmath}

Unless \( \mu = 0 \) we must have a zero four-divergence \( \partial_\alpha A^\alpha = 0 \).

\makeSubAnswer{}{qft:problemSet1:6b}

In the \( \mu \rightarrow 0 \) case with zero divergence, the field equation is just

\begin{dmath}\label{eqn:qftProblemSet1Problem6:200}
0
= \delSquaredBox A_\nu
= \partial^\alpha \partial_\alpha A_\nu
= \partial^\alpha \partial_\alpha A_\nu
- \partial_\nu \partial^\alpha A_\alpha
= \partial^\alpha
\lr{
\partial_\alpha A_\nu
- \partial_\nu A_\alpha
}
=
\partial^\alpha F_{\alpha \nu}.
\end{dmath}

Now consider the various index combinations of the electromagnetic field \( F_{\mu \nu} \).  When one index is zero we have the electric field components

\begin{dmath}\label{eqn:qftProblemSet1Problem6:220}
F_{0 k}
=
\partial_0 A_k - \partial_k A_0
=
-\PD{t}{A^k} - \PD{x^k}{\phi}
=
\BE \cdot \Be_k.
\end{dmath}

The remaining are the magnetic field components, for example
\begin{dmath}\label{eqn:qftProblemSet1Problem6:240}
F_{12}
=
\partial_1 A_2 - \partial_2 A_1
=
-\partial_1 A^2 + \partial_2 A^1
=
-\BB \cdot \Be_3.
\end{dmath}

By cyclic permutation we have
\begin{equation}\label{eqn:qftProblemSet1Problem6:260}
\begin{aligned}
B_3 &= -F_{12} \\
B_1 &= -F_{23} \\
B_2 &= -F_{31}.
\end{aligned}
\end{equation}

The field relation \cref{eqn:qftProblemSet1Problem6:200} for \( \nu = 0 \) expands to

\begin{dmath}\label{eqn:qftProblemSet1Problem6:280}
0
=
\partial^k F_{k 0}
=
- \spacegrad \cdot \BE,
\end{dmath}

which is one of the (source-less) Maxwell equations.

For the other indexes, the expansion is like
\begin{dmath}\label{eqn:qftProblemSet1Problem6:300}
0
=
\partial^\alpha F_{\alpha 1}
=
\partial^2 F_{2 1}
+
\partial^3 F_{3 1}
+
\partial^0 F_{0 1}
=
-\partial_2 (B_3)
-\partial_3 (-B_2)
+ \partial_t E_1
=
\lr{ \PD{t}{\BE} - \spacegrad \cross \BB} \cdot \Be_1.
\end{dmath}

Using cyclic permutation, we must have
\begin{dmath}\label{eqn:qftProblemSet1Problem6:320}
0 = \PD{t}{\BE} - \spacegrad \cross \BB,
\end{dmath}

another of the source free Maxwell equations.

}

   \chapter{Scalar action, least action principle, Euler-Lagrange equations for a field, canonical quantization.}
      %
% Copyright � 2018 Peeter Joot.  All Rights Reserved.
% Licenced as described in the file LICENSE under the root directory of this GIT repository.
%
%{
%%%\input{../latex/blogpost.tex}
%%%\renewcommand{\basename}{qftLecture4}
%%%\renewcommand{\dirname}{notes/phy2403/}
%%%\newcommand{\keywords}{PHY2403H}
%%%\input{../latex/peeter_prologue_print2.tex}
%%%
%%%%\usepackage{phy2403}
%%%\usepackage{peeters_braket}
%%%%\usepackage{peeters_layout_exercise}
%%%\usepackage{peeters_figures}
%%%\usepackage{mathtools}
%%%\usepackage{siunitx}
%%%\usepackage{macros_cal}
%%%
%%%\beginArtNoToc
%%%\generatetitle{PHY2403H Quantum Field Theory.  Lecture 4: Scalar action, least action principle, Euler-Lagrange equations for a field, canonical quantization.  Taught by Prof.\ Erich Poppitz}
\chapter{Scalar action, least action principle, Euler-Lagrange equations for a field, canonical quantization.}
\index{action}
\index{least action principle}
\index{Euler-Lagrange equations!field}
\index{canonical quantization}
\label{chap:qftLecture4}
%%%
%%%\paragraph{DISCLAIMER: Very rough notes from class.  Some additional side notes, but otherwise barely edited.}
%%%
%%%These are notes for the UofT course PHY2403H, Quantum Field Theory I, taught by Prof. Erich Poppitz fall 2018.
%%%%, covering \textchapref{{1}} \citep{peskin1995introduction} content.
%%%
\section{Principles (cont.)}

\begin{itemize}
\item Lorentz (Poincar\'e : Lorentz and spacetime translations)
\item locality
\item dimensional analysis
\item gauge invariance
\end{itemize}

These are the requirements for an action.  We postulated an action that had the form
\begin{dmath}\label{eqn:qftLecture4:20}
\int d^d x \partial_\mu \phi \partial^\mu \phi,
\end{dmath}
called the ``Kinetic term'', which mimics \( \int dt \dot{q}^2 \) that we'd see in quantum or classical mechanics.  In principle there exists an infinite number of local Poincar\'e invariant terms that we can write.  Examples:

\begin{itemize}
\item \( \partial_\mu \phi \partial^\mu \phi \)
\item \( \partial_\mu \phi \partial_\nu \partial^\nu \partial^\mu \phi \)
\item \( \lr{\partial_\mu \phi \partial^\mu \phi}^2 \)
\item \( f(\phi) \partial_\mu \phi \partial^\mu \phi \)
\item \( f(\phi, \partial_\mu \phi \partial^\mu \phi) \)
\item \( V(\phi) \)
\end{itemize}

It turns out that nature (i.e. three spatial dimensions and one time dimension) is described by a finite number of terms.  We will now utilize dimensional analysis to determine some of the allowed forms of the action for scalar field theories in \( d = 2, 3, 4, 5 \) dimensions.  Even though the real world is only \( d = 4 \), some of the \( d < 4 \) theories are relevant in condensed matter studies, and \( d = 5 \) is just for fun (but also applies to string theories.)

With \( [x] \sim \inv{M} \) in natural units, we must define \([\phi]\) such that the kinetic term is dimensionless in d spacetime dimensions

\begin{dmath}\label{eqn:qftLecture4:40}
\begin{aligned}
[d^d x] &\sim \inv{M^d} \\
[\partial_\mu] &\sim M
\end{aligned}
\end{dmath}

so it must be that
\begin{dmath}\label{eqn:qftLecture4:60}
[\phi] = M^{(d-2)/2}
\end{dmath}

It will be easier to characterize the dimensionality of any given term by the power of the mass units, that is

\begin{dmath}\label{eqn:qftLecture4:80}
\begin{aligned}
[\text{mass}] &= 1 \\
[d^d x] &= -d \\
[\partial_\mu] &= 1 \\
[\phi] &= (d-2)/2 \\
[S] &= 0.
\end{aligned}
\end{dmath}
Since the action is
\begin{dmath}\label{eqn:qftLecture4:100}
S = \int d^d x \lr{ \LL(\phi, \partial_\mu \phi) },
\end{dmath}
and because action had dimensions of \( \Hbar \), so in natural units, it must be dimensionless, the Lagrangian density dimensions must be \( [d] \).  We will abuse language in QFT and call the Lagrangian density the Lagrangian.

\section{\( d = 2 \).}

Because \( [\partial_\mu \phi \partial^\mu \phi ] = 2 \), the scalar field must be dimension zero, or in symbols
\begin{dmath}\label{eqn:qftLecture4:120}
[\phi] = 0.
\end{dmath}
This means that introducing any function \( f(\phi) = 1 + a \phi + b\phi^2 + c \phi^3 + \cdots \) is also dimensionless, and
\begin{dmath}\label{eqn:qftLecture4:140}
[f(\phi) \partial_\mu \phi \partial^\mu \phi ] = 2,
\end{dmath}
for any \( f(\phi) \).  Another implication of this is that the a potential term in the Lagrangian \( [V(\phi)] = 0 \) needs a coupling constant of dimension 2.  Letting \( \mu \) have mass dimensions, our Lagrangian must have the form
\begin{dmath}\label{eqn:qftLecture4:160}
f(\phi) \partial_\mu \phi \partial^\mu \phi + \mu^2 V(\phi).
\end{dmath}
An infinite number of coupling constants of positive mass dimensions for \( V(\phi) \) are also allowed.  If we have higher order derivative terms, then we need to compensate for the negative mass dimensions.   Example (still for \( d = 2 \)).
\begin{dmath}\label{eqn:qftLecture4:180}
\LL =
f(\phi) \partial_\mu \phi \partial^\mu \phi + \mu^2 V(\phi) + \inv{{\mu'}^2}\partial_\mu \phi \partial_\nu \partial^\nu \partial^\mu \phi + \lr{ \partial_\mu \phi \partial^\mu \phi }^2 \inv{\tilde{\mu}^2}.
\end{dmath}
The last two terms, called \underline{couplings} (i.e. any non-kinetic term), are examples of terms with negative mass dimension.  There is an infinite number of those in any theory in any dimension.

\paragraph{Definitions}

\begin{itemize}
\item Couplings that are dimensionless are called (classically) marginal.
\item Couplings that have positive mass dimension are called (classically) relevant.
\item Couplings that have negative mass dimension are called (classically) irrelevant.
\end{itemize}

In QFT we are generally interested in the couplings that are measurable at long distances for some given energy.  Classically irrelevant theories are generally not interesting in \( d > 2 \), so we are very lucky that we don't live in three dimensional space.  This means that we can get away with a finite number of classically marginal and relevant couplings in 3 or 4 dimensions.  This was mentioned in the Wilczek's article referenced in the class forum \citep{wilczek2007fundamental}\footnote{There's currently more in that article that I don't understand than I do, so it is hard to find it terribly illuminating.}

Long distance physics in any dimension is described by the marginal and relevant couplings.  The irrelevant couplings die off at low energy.  In two dimensions, a priori, an infinite number of marginal and relevant couplings are possible.  2D is a bad place to live!

\section{\( d = 3 \).}

Now we have
\begin{dmath}\label{eqn:qftLecture4:200}
[\phi] = \inv{2}
\end{dmath}
so that
\begin{dmath}\label{eqn:qftLecture4:220}
[\partial_\mu \phi \partial^\mu \phi] = 3.
\end{dmath}

A 3D Lagrangian could have local terms such as
\begin{dmath}\label{eqn:qftLecture4:240}
\LL = \partial_\mu \phi \partial^\mu \phi + m^2 \phi^2 + \mu^{3/2} \phi^3 + \mu' \phi^4
+ \lr{\mu''}{1/2} \phi^5
+ \lambda \phi^6.
\end{dmath}
where \( m, \mu, \mu'' \) all have mass dimensions, and \( \lambda \) is dimensionless.  i.e.
\( m, \mu, \mu'' \) are relevant, and \( \lambda \) marginal.  We stop at the sixth power, since any power after that will be irrelevant.

\section{\( d = 4 \).}

Now we have
\begin{dmath}\label{eqn:qftLecture4:260}
[\phi] = 1
\end{dmath}
so that
\begin{dmath}\label{eqn:qftLecture4:280}
[\partial_\mu \phi \partial^\mu \phi] = 4.
\end{dmath}

In this number of dimensions \( \phi^k \partial_\mu \phi \partial^\mu \) is an irrelevant coupling.

A 4D Lagrangian could have local terms such as
\begin{dmath}\label{eqn:qftLecture4:300}
\LL = \partial_\mu \phi \partial^\mu \phi + m^2 \phi^2 + \mu \phi^3 + \lambda \phi^4.
\end{dmath}
where \( m, \mu \) have mass dimensions, and \( \lambda \) is dimensionless.  i.e.
\( m, \mu \) are relevant, and \( \lambda \) is marginal.

\section{\( d = 5 \).}

Now we have
\begin{dmath}\label{eqn:qftLecture4:320}
[\phi] = \frac{3}{2},
\end{dmath}
so that
\begin{dmath}\label{eqn:qftLecture4:340}
[\partial_\mu \phi \partial^\mu \phi] = 5.
\end{dmath}

A 5D Lagrangian could have local terms such as
\begin{dmath}\label{eqn:qftLecture4:360}
\LL = \partial_\mu \phi \partial^\mu \phi + m^2 \phi^2 + \sqrt{\mu} \phi^3 + \inv{\mu'} \phi^4.
\end{dmath}
where \( m, \mu, \mu' \) all have mass dimensions.  In 5D there are no marginal couplings.  Dimension 4 is the last dimension where marginal couplings exist.  In condensed matter physics 4D is called the ``upper critical dimension''.

From the point of view of particle physics, all the terms in the Lagrangian must be the ones that are relevant at long distances.

\section{Least action principle (classical field theory).}
\index{least action principle}
\index{classical field theory}

Now we want to study 4D scalar theories.  We have some action
\begin{dmath}\label{eqn:qftLecture4:380}
S[\phi] = \int d^4 x \LL(\phi, \partial_\mu \phi).
\end{dmath}

Let's keep an example such as the following in mind
\begin{dmath}\label{eqn:qftLecture4:400}
\LL = \underbrace{\inv{2} \partial_\mu \phi \partial^\mu \phi}_{\text{Kinetic term}} - \underbrace{m^2 \phi - \lambda \phi^4}_{\text{all relevant and marginal couplings}}.
\end{dmath}
The even powers can be justified by assuming there is some symmetry that kills the odd powered terms.

We will be integrating over a space time region such as that depicted in \cref{fig:spacetimeCylinder:spacetimeCylinderFig1},
\imageFigure{../figures/phy2403-quantum-field-theory/spacetimeCylinderFig1}{Cylindrical spacetime boundary.}{fig:spacetimeCylinder:spacetimeCylinderFig1}{0.3}
where a cylindrical spatial cross section is depicted that we allow to tend towards infinity.  We demand that the field is fixed on the infinite spatial boundaries.  The easiest way to demand that the field dies off on the spatial boundaries, that is
\begin{dmath}\label{eqn:qftLecture4:420}
\lim_{\Abs{\Bx} \rightarrow \infty} \phi(\Bx) \rightarrow 0.
\end{dmath}
The functional \( \phi(\Bx, t) \) that obeys the boundary condition as stated extremizes \( S[\phi] \).

Extremizing the action means that we seek \( \phi(\Bx, t) \)
\begin{equation}\label{eqn:qftLecture4:440}
\delta S[\phi] = 0 = S[\phi + \delta \phi] - S[\phi].
\end{equation}

How do we compute the variation?
\begin{dmath}\label{eqn:qftLecture4:460}
\delta S = \int d^d x \lr{ \LL(\phi + \delta \phi, \partial_\mu \phi + \partial_\mu \delta \phi) - \LL(\phi, \partial_\mu \phi) }
= \int d^d x \lr{ \PD{\phi}{\LL} \delta \phi + \PD{(\partial_mu \phi)}{\LL} (\partial_\mu \delta \phi) }
= \int d^d x \lr{ \PD{\phi}{\LL} \delta \phi
+ \partial_\mu \lr{ \PD{(\partial_mu \phi)}{\LL} \delta \phi}
- \lr{ \partial_\mu \PD{(\partial_mu \phi)}{\LL} } \delta \phi
}
=
\int d^d x
\delta \phi
\lr{ \PD{\phi}{\LL}
- \partial_\mu \PD{(\partial_mu \phi)}{\LL} }
+ \int d^3 \sigma_\mu \lr{ \PD{(\partial_\mu \phi)}{\LL} \delta \phi }
\end{dmath}

If we are explicit about the boundary term, we write it as
\begin{dmath}\label{eqn:qftLecture4:480}
\int dt d^3 \Bx \partial_t \lr{ \PD{(\partial_t \phi)}{\LL} \delta \phi }
- \spacegrad \cdot \lr{ \PD{(\spacegrad \phi)}{\LL} \delta \phi }
=
\int d^3 \Bx \evalrange{ \PD{(\partial_t \phi)}{\LL} \delta \phi }{t = -T}{t = T}
- \int dt d^2 \BS \cdot \lr{ \PD{(\spacegrad \phi)}{\LL} \delta \phi }.
\end{dmath}
but \( \delta \phi = 0 \) at \( t = \pm T \) and also at the spatial boundaries of the integration region.

This leaves
\begin{dmath}\label{eqn:qftLecture4:500}
\delta S[\phi] = \int d^d x \delta \phi
\lr{ \PD{\phi}{\LL} - \partial_\mu \PD{(\partial_mu \phi)}{\LL} } = 0 \forall \delta \phi.
\end{dmath}
That is
%\begin{dmath}\label{eqn:qftLecture4:520}
\boxedEquation{eqn:qftLecture4:540}{
\PD{\phi}{\LL} - \partial_\mu \PD{(\partial_mu \phi)}{\LL} = 0.
}
%\end{dmath}
This is the Euler-Lagrange equations for a single scalar field.

Returning to our sample scalar Lagrangian
\begin{dmath}\label{eqn:qftLecture4:560}
\LL = \inv{2} \partial_\mu \phi \partial^\mu \phi - \inv{2} m^2 \phi^2 - \frac{\lambda}{4} \phi^4.
\end{dmath}
This example is related to the Ising model which has a \( \phi \rightarrow -\phi \) symmetry.
Applying the Euler-Lagrange equations, we have
\begin{dmath}\label{eqn:qftLecture4:580}
\PD{\phi}{\LL} = -m^2 \phi - \lambda \phi^3,
\end{dmath}
and
\begin{dmath}\label{eqn:qftLecture4:600}
\PD{(\partial_\mu \phi)}{\LL}
=
\PD{(\partial_\mu \phi)}{} \lr{
\inv{2} \partial_\nu \phi \partial^\nu \phi }
=
\inv{2} \partial^\nu \phi
\PD{(\partial_\mu \phi)}{}
\partial_\nu \phi
+
\inv{2} \partial_\nu \phi
\PD{(\partial_\mu \phi)}{}
\partial_\alpha \phi g^{\nu\alpha}
=
\inv{2} \partial^\mu \phi
+
\inv{2} \partial_\nu \phi g^{\nu\mu}
=
\partial^\mu \phi
\end{dmath}
so we have
\begin{dmath}\label{eqn:qftLecture4:620}
0
=
\PD{\phi}{\LL} -\partial_\mu
\PD{(\partial_\mu \phi)}{\LL}
=
-m^2 \phi - \lambda \phi^3 - \partial_\mu \partial^\mu \phi.
\end{dmath}

For \( \lambda = 0 \), the free field theory limit, this is just
\begin{dmath}\label{eqn:qftLecture4:640}
\partial_\mu \partial^\mu \phi + m^2 \phi = 0.
\end{dmath}
Written out from the observer frame, this is
\begin{dmath}\label{eqn:qftLecture4:660}
(\partial_t)^2 \phi - \spacegrad^2 \phi + m^2 \phi = 0.
\end{dmath}

With a non-zero mass term
\begin{dmath}\label{eqn:qftLecture4:680}
\lr{ \partial_t^2 - \spacegrad^2  + m^2 } \phi = 0,
\end{dmath}
is called the Klein-Gordan equation.

If we also had \( m = 0 \) we'd have
\begin{dmath}\label{eqn:qftLecture4:700}
\lr{ \partial_t^2 - \spacegrad^2 } \phi = 0,
\end{dmath}
which is the wave equation (for a massless free field).  This is also called the D'Alembert equation, which is familiar from electromagnetism where we have
\begin{dmath}\label{eqn:qftLecture4:720}
\begin{aligned}
\lr{ \partial_t^2 - \spacegrad^2 } \BE &= 0 \\
\lr{ \partial_t^2 - \spacegrad^2 } \BB &= 0,
\end{aligned}
\end{dmath}
in a source free region.

\section{Canonical quantization.}
\index{canonical quantization}

The harmonic oscillator described by
\begin{dmath}\label{eqn:qftLecture4:740}
\LL = \inv{2} \dot{q}^2 - \frac{\omega^2}{2} q^2<
\end{dmath}
has solution \(\ddot{q} = - \omega^2 q\).
With
\begin{equation}\label{eqn:qftLecture4:760}
p = \PD{\dot{q}}{\LL} = \dot{q},
\end{equation}
the Hamiltonian is given by
\begin{dmath}\label{eqn:qftLecture4:780}
H(p,q) = \evalbar{p \dot{q} - \LL}{\dot{q}(p, q)}
= p p - \inv{2} p^2 + \frac{\omega^2}{2} q^2 = \frac{p^2}{2} + \frac{\omega^2}{2} q^2.
\end{dmath}

In QM we quantize by mapping Poisson brackets to commutators.
\begin{dmath}\label{eqn:qftLecture4:800}
\antisymmetric{\hatp}{\hatq} = -i
\end{dmath}
One way to represent is to say that states are \( \Psi(\hatq) \), a wave function, \( \hatq \) acts by \( q \)
\begin{dmath}\label{eqn:qftLecture4:820}
\hatq \Psi = q \Psi(q)
\end{dmath}
With
\begin{dmath}\label{eqn:qftLecture4:840}
\hatp = -i \PD{q}{},
\end{dmath}
so
\begin{dmath}\label{eqn:qftLecture4:860}
\antisymmetric{ -i \PD{q}{} } { q} = -i
\end{dmath}

\paragraph{Returning to the field Lagrangian.}

Let's introduce an explicit space time split.  We'll write
\begin{dmath}\label{eqn:qftLecture4:880}
L = \int d^3 x \lr{
\inv{2} (\partial_0 \phi(\Bx, t))^2 - \inv{2} \lr{ \spacegrad \phi(\Bx, t) }^2 - \frac{m^2}{2} \phi
},
\end{dmath}
so that the action is
\begin{dmath}\label{eqn:qftLecture4:900}
S = \int dt L.
\end{dmath}
The dynamical variables are \( \phi(\Bx) \).  We define
\begin{dmath}\label{eqn:qftLecture4:920}
\pi(\Bx, t) = \frac{\delta L}{\delta (\partial_0 \phi(\Bx, t))}
=
\partial_0 \phi(\Bx, t)
=
\dot{\phi}(\Bx, t),
\end{dmath}
called the canonical momentum, or the momentum conjugate to \( \phi(\Bx, t) \).
Why \( \delta \)?  Has to do with an implicit Dirac function to eliminate the integral?

\begin{dmath}\label{eqn:qftLecture4:940}
H
= \int d^3 x \evalbar{\lr{ \pi(\bar{\Bx}, t) \dot{\phi}(\bar{\Bx}, t) - L }}{\dot{\phi}(\bar{\Bx}, t) = \pi(x, t) }
= \int d^3 x \lr{ (\pi(\Bx, t))^2 - \inv{2} (\pi(\Bx, t))^2 + \inv{2} (\spacegrad \phi)^2 + \frac{m}{2} \phi^2 },
\end{dmath}
or
\begin{dmath}\label{eqn:qftLecture4:960}
H
= \int d^3 x \lr{ \inv{2} (\pi(\Bx, t))^2 + \inv{2} (\spacegrad \phi(\Bx, t))^2 + \frac{m}{2} (\phi(\Bx, t))^2 }
\end{dmath}

In analogy to the momentum, position commutator in QM
\begin{dmath}\label{eqn:qftLecture4:1000}
\antisymmetric{\hat{p}_i}{\hat{q}_j} = -i \delta_{ij},
\end{dmath}
we ``quantize'' the scalar field theory by promoting \( \pi, \phi \) to operators and insisting that they also obey a commutator relationship
\begin{dmath}\label{eqn:qftLecture4:980}
\antisymmetric{\pi(\Bx, t)}{\phi(\By, t)} = -i \deltathree(\Bx - \By).
\end{dmath}
Note that in this commutator, the fields are evaluated at different spatial points, but at the same time.

%}
%%%\EndArticle

   \chapter{Klein-Gordon equation, SHOs, momentum space representation, raising and lowering operators.}
      %
% Copyright � 2017 Peeter Joot.  All Rights Reserved.
% Licenced as described in the file LICENSE under the root directory of this GIT repository.
%
%{
%%\input{../latex/blogpost.tex}
%%\renewcommand{\basename}{qftLecture5}
%%\renewcommand{\dirname}{notes/phy2403/}
%%\newcommand{\keywords}{PHY2403H}
%%\input{../latex/peeter_prologue_print2.tex}
%%
%%%\usepackage{phy2403}
%%\usepackage{peeters_braket}
%%%\usepackage{peeters_layout_exercise}
%%\usepackage{peeters_figures}
%%\usepackage{mathtools}
%%\usepackage{siunitx}
%%\usepackage{macros_cal} % LL
%%
%%\beginArtNoToc
%%\generatetitle{PHY2403H Quantum Field Theory.  Lecture 5: Klein-Gordon equation, SHOs, momentum space representation, raising and lowering operators..  Taught by Prof.\ Erich Poppitz}
\chapter{Klein-Gordon equation, SHOs, momentum space representation, raising and lowering operators.}
\index{Klein-Gordon}
\index{simple harmonic oscillator}
\index{momentum space representation}
\index{raising operator}
\index{lowering operator}
\label{chap:qftLecture5}

%\paragraph{DISCLAIMER: Very rough notes from class.  Some additional side notes, but otherwise barely edited.}
%
%These are notes for the UofT course PHY2403H, Quantum Field Theory I, taught by Prof. Erich Poppitz fall 2018.
%%, covering \textchapref{{1}} \citep{peskin1995introduction} content.

\section{Canonical quantization.}
\index{canonical quantization}

Last time we introduced a Lagrangian density associated with the Klein-Gordon equation (with a quadratic potential coupling)
\begin{dmath}\label{eqn:qftLecture5:20}
L = \int d^3 x
\lr{
\inv{2} \lr{\partial_0 \phi}^2 - \inv{2} \lr{\spacegrad \phi}^2 - \frac{m^2}{2} \phi^2  - \frac{\lambda}{4} \phi^4
}.
\end{dmath}
This Lagrangian density was related to the action by
\begin{equation}\label{eqn:qftLecture5:40}
S = \int dt L = \int dt d^3 x \LL,
\end{equation}
with momentum canonically conjugate to the field \( \phi \) defined as
\begin{equation}\label{eqn:qftLecture5:60}
\Pi(\Bx, t) = \frac{\delta \LL}{\delta \phidot(\Bx, t) } = \PD{\phidot(\Bx, t)}{\LL}
\end{equation}

The Hamiltonian defined as
\begin{dmath}\label{eqn:qftLecture5:80}
H = \int d^3 x \lr{ \Pi(\Bx, t) \phidot(\Bx, t) - \LL },
\end{dmath}
led to
\begin{dmath}\label{eqn:qftLecture5:680}
H
= \int d^3 x
\lr{ \inv{2} \Pi^2 + (\spacegrad \phi)^2 + \inv{2} m^2 \phi^2 + \frac{\lambda}{4} \phi^4 }.
\end{dmath}
Like the Lagrangian density, we may introduce a Hamiltonian density \( \calH \) as
\begin{dmath}\label{eqn:qftLecture5:100}
H = \int d^3 x \calH(\Bx, t).
\end{dmath}
For our Klein-Gordon system, this is
\begin{dmath}\label{eqn:qftLecture5:120}
\calH(\Bx, t) =
\inv{2} \Pi^2 + (\spacegrad \phi)^2 + \inv{2} m^2 \phi^2 + \frac{\lambda}{4} \phi^4.
\end{dmath}

\paragraph{Canonical Commutation Relations (CCR)}:
\index{canonical commutation relations}
\index{CCR}

We quantize the system by promoting our fields to Heisenberg-Picture (HP) operators, and imposing commutation relations
\begin{dmath}\label{eqn:qftLecture5:140}
\antisymmetric{\hat{\Pi}(\Bx, t)}{\hat{\phi}(\By, t)} = -i \delta^3 (\Bx - \By),
\end{dmath}
which is analogous to
\begin{dmath}\label{eqn:qftLecture5:160}
\antisymmetric{\hat{p}_i}{\hat{q}_j} = -i \delta_{ij}.
\end{dmath}

To choose a representation, we may map the \( \Psi \) of QM \( \rightarrow \) to a wave functional \( \Psi[\phi] \)
\begin{dmath}\label{eqn:qftLecture5:180}
\hat{\phi}(\By, t) \Psi[\phi] = \phi(\By, t) \Psi[\phi]
\end{dmath}

This is similar to the QM wave functions
\begin{dmath}\label{eqn:qftLecture5:200}
\begin{aligned}
\hat{q}_i \Psi(\setlr{q}) &= q_i \Psi(q) \\
\hat{p}_i \Psi(\setlr{q}) &= -i \PD{q_i}{} \Psi(p)
\end{aligned}
\end{dmath}

Our momentum operator is quantized by expressing it in terms of a variational derivative
\begin{dmath}\label{eqn:qftLecture5:220}
\hat{\Pi}(\Bx, t) = -i \frac{\delta}{\delta \phi(\Bx, t)}.
\end{dmath}
(Fixme: I'm not really sure exactly what is meant by using the variation derivative \(\delta\) notation here), and to
quantize the Hamiltonian we just add hats, assuming that our fields are all now HP operators
\begin{dmath}\label{eqn:qftLecture5:240}
\hat{\calH}(\Bx, t)
=
\inv{2} \hat{\Pi}^2 + (\spacegrad \hat{\phi})^2 + \inv{2} m^2 \hat{\phi}^2 + \frac{\lambda}{4} \hat{\phi}^4.
\end{dmath}

\paragraph{QM SHO review}
\index{simple harmonic oscillator}
Recall the QM SHO had a Hamiltonian
\begin{dmath}\label{eqn:qftLecture5:260}
\hat{H} = \inv{2} \hat{p}^2 + \inv{2} \omega^2 \hat{q}^2,
\end{dmath}
where
\begin{dmath}\label{eqn:qftLecture5:280}
\antisymmetric{\hat{p}}{\hat{q}} = -i,
\end{dmath}
and that
HP time evolution operators \( O \) satisfied
\begin{dmath}\label{eqn:qftLecture5:700}
\ddt{\hatO} = i \antisymmetric{\hatH}{\hatO}.
\end{dmath}
In particular
\begin{dmath}\label{eqn:qftLecture5:300}
\ddt{\hat{p}}
= i \antisymmetric{\hat{H}}{\hatp}
= i \frac{\omega^2}{2} \antisymmetric{\hatq^2}{\hatp}
= i \frac{\omega^2}{2} (2 i \hatq)
= -i \omega^2 \hatq,
\end{dmath}
and
\begin{dmath}\label{eqn:qftLecture5:320}
\ddt{\hat{q}}
= i \antisymmetric{\hat{H}}{\hatq}
= i \inv{2} \antisymmetric{\hatp^2}{\hatq}
= \frac{i}{2}(-2 i \hatp )
= \hatp.
\end{dmath}
Applying the time evolution operator twice, we find
\begin{equation}\label{eqn:qftLecture5:340}
\frac{d^2}{dt^2}{\hat{q}}
= \ddt{\hat{p}}
= - \omega^2 \hatq.
\end{equation}
We see that the Heisenberg operators obey the classical equations of motion.

Now we want to try this with the quantized QFT fields we've promoted to operators
\begin{dmath}\label{eqn:qftLecture5:360}
\ddt{\hat{\Pi}}(\Bx, t)
= i \antisymmetric{\hatH}{\hat{\Pi}(\Bx, t)}
=
i \int d^3 y \inv{2} \antisymmetric{ \lr{\spacegrad \phihat(\By) }^2 }{\Pihat(\Bx) }
+
i \int d^3 y \frac{m^2}{2} \antisymmetric{ \phihat(\By)^2 }{\Pihat(\Bx) }
+
i \frac{\lambda}{4} \int d^3 \antisymmetric{ \phihat(\By)^4 }{\Pihat(\Bx) }.
\end{dmath}

Starting with the non-gradient commutators, and
utilizing the HP field analogues of the relations \( \antisymmetric{\hatq^n}{\hatp} = n i \hatq^{n-1} \), we find
\begin{equation}\label{eqn:qftLecture5:780}
\int d^3 y \antisymmetric{ \lr{ \phihat(\By) }^2 }{\Pihat(\Bx) }
=
\int d^3 y 2 i \phihat(\By) \delta^3(\Bx - \By)
= 2 i \phihat(\Bx).
\end{equation}
\begin{equation}\label{eqn:qftLecture5:740}
\int d^3 y \antisymmetric{ \lr{ \phihat(\By) }^4 }{\Pihat(\Bx) }
=
\int d^3 y 4 i \phihat(\By)^3 \delta^3(\Bx - \By)
= 4 i \phihat(\Bx)^3.
\end{equation}
For the gradient commutators, we have more work.  Prof Poppitz blitzed through that, just calling it integration by parts.  I had trouble seeing what he was doing, so here's a more explicit dumb expansion required to calculate the commutator
\begin{dmath}\label{eqn:qftLecture5:720}
\int d^3 y (\spacegrad \phihat(\By))^2 \Pihat(\Bx)
=
\int d^3 y
\lr{ \spacegrad \phihat(\By) \cdot \spacegrad \phihat(\By) } \Pihat(\Bx)
=
\int d^3 y
\spacegrad \phihat(\By) \cdot
\lr{ \spacegrad (\phihat(\By) \Pihat(\Bx)) }
=
\int d^3 y
\spacegrad \phihat(\By) \cdot
\lr{ \spacegrad (\Pihat(\Bx) \phihat(\By) + i \delta^3(\Bx - \By)) }
=
\int d^3 y
\Biglr{
   \spacegrad \lr{ \phihat(\By) \Pihat(\Bx) } \cdot \spacegrad \phihat(\By)
   + i
   \spacegrad \phihat(\By) \cdot \spacegrad \delta^3(\Bx - \By)
}
=
\int d^3 y
\Biglr{
   \spacegrad \lr{ \Pihat(\Bx) \phihat(\By) + i \delta^3(\Bx - \By) } \cdot \spacegrad \phihat(\By)
   + i
   \spacegrad \phihat(\By) \cdot \spacegrad \delta^3(\Bx - \By)
}
=
\int d^3 y
   \Pihat(\Bx)
\lr{
   \spacegrad \phihat(\By) \cdot \spacegrad \phihat(\By)
}
+ 2 i
\int d^3 y
   \spacegrad \phihat(\By) \cdot \spacegrad \delta^3(\Bx - \By)
=
\int d^3 y
   \Pihat(\Bx) \spacegrad^2 \phihat(\By)
+
2 i
\int d^3 y
   \spacegrad \cdot \lr{ \delta^3(\Bx - \By) \spacegrad \phihat(\By) }
-
2 i
\int d^3 y
   \delta^3(\Bx - \By) \spacegrad^2 \phihat(\By)
=
\int d^3 y
   \Pihat(\Bx) \spacegrad^2 \phihat(\By)
+
2 i
\int_\partial d^2 y
   \delta^3(\Bx - \By)
   \ncap \cdot \spacegrad \phihat(\By)
-
2 i \spacegrad^2 \phihat(\Bx).
\end{dmath}
Here we take advantage of the fact that the derivative operators \( \spacegrad = \spacegrad_\By \) commute with \( \Pihat(\Bx) \), and use the identity
\( \spacegrad \cdot (a \spacegrad b) = (\spacegrad a) \cdot (\spacegrad b) + a \spacegrad^2 b \), so the commutator is
\begin{dmath}\label{eqn:qftLecture5:800}
\int d^3 y \antisymmetric{(\spacegrad \phihat(\By))^2}{\Pihat(\Bx)}
=
2 i
\int_\partial d^2 y
   \delta^3(\Bx - \By)
   \ncap \cdot \spacegrad \phihat(\By)
-
2 i \spacegrad^2 \phihat(\Bx)
=
-
2 i \spacegrad^2 \phihat(\Bx),
\end{dmath}
where the boundary integral is presumed to be zero (without enough justification.)
All the pieces can now be put back together
\begin{dmath}\label{eqn:qftLecture5:820}
\ddt{} \hat{\Pi}(\Bx, t)
=
\spacegrad^2 \phihat(\Bx, t)
-
m^2 \phihat(\Bx, t)
-
\lambda \phihat^3(\Bx, t).
\end{dmath}

Now, for the \( \phihat \) time evolution, which is much easier
\begin{dmath}\label{eqn:qftLecture5:380}
\ddt{\hat{\phi}}(\Bx, t)
= i \antisymmetric{\hatH}{\hat{\phi}(\Bx, t)}
= i \inv{2} \int d^3 y \antisymmetric{\Pihat^2(\By)}{\hat{\phi}(\Bx)}
= i \inv{2} \int d^3 y (-2 i) \Pihat(\By, t) \delta^3(\Bx - \By)
= \Pihat(\Bx, t)
\end{dmath}
\begin{dmath}\label{eqn:qftLecture5:400}
\frac{d^2}{dt^2}{\hat{\phi}}(\Bx, t)
=
\spacegrad^2 \phi
-m^2 \phi - \lambda \phihat^3.
\end{dmath}
That is
\begin{dmath}\label{eqn:qftLecture5:420}
\ddot{\phihat} - \spacegrad^2 \phihat + m^2 \phihat + \lambda \phihat^3 = 0,
\end{dmath}
which is the classical Euler-Lagrange equation, also obeyed by the
Heisenberg operator \( \phi(\Bx, t) \).  When \( \lambda = 0 \) this is the Klein-Gordon equation.

\section{Momentum space representation.}
\index{momentum space representation}
Dropping hats, we now consider the momentum space representation of our operators, as determined by Fourier transform pairs
\begin{dmath}\label{eqn:qftLecture5:440}
\begin{aligned}
\phi(\Bx, t) &= \int \frac{d^3 p}{(2\pi)^3} e^{i \Bp \cdot \Bx} \tilde{\phi}(\Bp, t) \\
\tilde{\phi}(\Bp, t) &= \int d^3 x e^{-i \Bp \cdot \Bx} \phi(\Bx, t).
\end{aligned}
\end{dmath}

We can discover a representation of the delta function by applying these both in turn
\begin{dmath}\label{eqn:qftLecture5:480}
\tilde{\phi}(\Bp, t)
= \int d^3 x e^{-i \Bp \cdot \Bx} \int \frac{d^3 q}{(2 \pi)^3} e^{i \Bq \cdot \Bx} \tilde{\phi}(\Bq, t)
\end{dmath}
so
\boxedEquation{eqn:qftLecture5:500}{
\int d^3 x e^{i \BA \cdot \Bx} = (2 \pi)^3 \delta^3(\BA)
}

Also observe that \( \phi^\conj(\Bx, t) = \phi(\Bx, t) \) iff \( \tilde{\phi}(\Bp, t) = \tilde{\phi}^\conj(-\Bp, t) \).

We want the equations of motion for \( \tilde{\phi}(\Bp, t) \) where the operator obeys the Klein-Gordon equation
\begin{dmath}\label{eqn:qftLecture5:520}
\lr{ \partial_t^2 - \spacegrad^2 + m^2 } \phi(\Bx, t) = 0
\end{dmath}

Inserting the transform relation \cref{eqn:qftLecture5:440} we get
\begin{dmath}\label{eqn:qftLecture5:540}
\int \frac{d^3 p}{(2 \pi)^3} e^{i \Bp \cdot \Bx}
\lr{
\ddot{\tilde{\phi}}(\Bp, t) + \lr{ \Bp^2 + m^2 }
\tilde{\phi}(\Bp, t)
}
= 0,
\end{dmath}
or
\boxedEquation{eqn:qftLecture5:580}{
\ddot{\tilde{\phi}}(\Bp, t) = - \omega_\Bp^2 \,\tilde{\phi}(\Bp, t),
}
where
\begin{dmath}\label{eqn:qftLecture5:560}
\omega_\Bp = \sqrt{ \Bp^2 + m^2 }.
\end{dmath}
The Fourier components of the HP operators are SHOs!

As we have SHO's and know how to deal with these in QM, we use the same strategy, introducing raising and lowering operators
\begin{dmath}\label{eqn:qftLecture5:600}
\tilde{\phi}(\Bp, t) = \inv{\sqrt{2 \omega_\Bp}} \lr{ e^{-i \omega_\Bp t } a_\Bp + e^{i \omega_\Bp t} a^\dagger_{-\Bp}.
}
\end{dmath}

Observe that
\begin{dmath}\label{eqn:qftLecture5:840}
\tilde{\phi}^\dagger(-\Bp, t)
= \inv{\sqrt{2 \omega_\Bp}} \lr{ e^{i \omega_\Bp t } a^\dagger_{-\Bp} + e^{-i \omega_\Bp t} a_{\Bp} }
=
\tilde{\phi}(\Bp, t),
\end{dmath}
or
\begin{dmath}\label{eqn:qftLecture5:620}
\tilde{\phi}^\dagger(\Bp, t) = \tilde{\phi}(-\Bp, t),
\end{dmath}
so \( \phi(\Bp, t) \) has a real representation in terms of \( a_\Bp \).

We will find (Wednesday) that
\begin{dmath}\label{eqn:qftLecture5:640}
\antisymmetric
{a_\Bq}
{a^\dagger_\Bp}
= 
(2 \pi)^3
\delta^3(\Bp - \Bq),
\end{dmath}
which are equivalent to
\begin{dmath}\label{eqn:qftLecture5:660}
\antisymmetric{\tilde{\Pi}(\Bp, t)}{\tilde{\phi}(\Bq, t)} = -i \delta^3(\Bp - \Bq).
\end{dmath}

%}
%\EndNoBibArticle

   \chapter{Canonical quantization, Simple Harmonic Oscillators, Symmetries.}
      %
% Copyright � 2018 Peeter Joot.  All Rights Reserved.
% Licenced as described in the file LICENSE under the root directory of this GIT repository.
%
%{
%%\input{../latex/blogpost.tex}
%%\renewcommand{\basename}{qftLecture6}
%%\renewcommand{\dirname}{notes/phy2403/}
%%\newcommand{\keywords}{PHY2403H}
%%\input{../latex/peeter_prologue_print2.tex}
%%
%%%\usepackage{phy2403}
%%\usepackage{peeters_braket}
%%%\usepackage{peeters_layout_exercise}
%%\usepackage{peeters_figures}
%%\usepackage{mathtools}
%%\usepackage{siunitx}
%%\usepackage{macros_cal} % LL
%%
%%\beginArtNoToc
%%\generatetitle{PHY2403H Quantum Field Theory.  Lecture 6: Canonical quantization, Simple Harmonic Oscillators, Symmetries.  Taught by Prof.\ Erich Poppitz}
%\chapter{Canonical quantization, Simple Harmonic Oscillators, Symmetries}
\index{canonical quantization}
\index{simple harmonic oscillator}
\index{symmetry}
%%\label{chap:qftLecture6}
%%
%%\paragraph{DISCLAIMER: Very rough notes from class, with some additional side notes (esp. the QM SHO review).}
%%
%%These are notes for the UofT course PHY2403H, Quantum Field Theory I, taught by Prof. Erich Poppitz fall 2018.
%%%, covering \textchapref{{1}} \citep{peskin1995introduction} content.
%%
\section{Quantization of Field Theory.}

We are engaging in the ``canonical'' or Hamiltonian method of quantization.  It is also possible to quantize using path integrals, but it is hard to prove that operators are unitary doing so.  In fact, the mechanism used to show unitarity from path integrals is often to find the Lagrangian and show that there is a Hilbert space (i.e. using canonical quantization).  Canonical quantization essentially demands that the fields obey a commutator relation of the following form
\begin{dmath}\label{eqn:qftLecture6:20}
\antisymmetric{\pi(\Bx, t)}{\phi(\By, t)} = -i \deltathree(\Bx - \By).
\end{dmath}
We assumed that the quantized fields obey the Hamiltonian relations
\begin{dmath}\label{eqn:qftLecture6:160}
\begin{aligned}
\ddt{\phi} &= i \antisymmetric{H}{\phi} \\
\ddt{\pi} &= i \antisymmetric{H}{\pi}.
\end{aligned}
\end{dmath}

We were working with the Hamiltonian density
\begin{dmath}\label{eqn:qftLecture6:40}
\calH =
\inv{2} (\pi(\Bx, t))^2
+
\inv{2} (\spacegrad \phi(\Bx, t))^2
+
\frac{m^2}{2} \phi^2
+
\frac{\lambda}{4} \phi^4,
\end{dmath}
which included a mass term \( m \) and a potential term (\(\lambda\)).  We will expand all quantities in Taylor series in \( \lambda \) assuming they have a structure such as
\begin{dmath}\label{eqn:qftLecture6:180}
\begin{aligned}
f(\lambda) =
c_0 \lambda^0
+ c_1 \lambda^1
+ c_2 \lambda^2
+ c_3 \lambda^3
+ \cdots
\end{aligned}
\end{dmath}
We will stop this \underline{perturbation theory} approach at \( O(\lambda^2) \), and will ignore functions such as \( e^{-1/\lambda} \).

Within perturbation theory, to leaving order, set \( \lambda = 0 \), so that \( \phi \) obeys the Klein-Gordon equation (if \( m = 0 \) we have just a d'Lambertian (wave equation)).

We can write our field as a Fourier transform
\begin{dmath}\label{eqn:qftLecture6:60}
\phi(\Bx, t) = \int \frac{d^3 p}{(2\pi)^3} e^{i \Bp \cdot \Bx} \tilde{\phi}(\Bp, t),
\end{dmath}
and due to a Hermitian assumption (i.e. real field) this implies
\begin{dmath}\label{eqn:qftLecture6:80}
\tilde{\phi}^\conj(\Bp, t) = \tilde{\phi}(-\Bp, t).
\end{dmath}

We found that the Klein-Gordon equation implied that the momentum space representation obey Harmonic oscillator equations
\begin{dmath}\label{eqn:qftLecture6:100}
\ddot{\tilde{\phi}}(\Bp, t) = - \omega_\Bp \tilde{\phi}(\Bp, t),
\end{dmath}
where \( \omega_\Bp = \sqrt{\Bp^2 + m^2} \).
The solution of \cref{eqn:qftLecture6:100} may be represented as
\begin{dmath}\label{eqn:qftLecture6:120}
\tilde{\phi}(\Bq, t) = \inv{\sqrt{2 \omega_\Bq}} \lr{
e^{-i \omega_\Bq t} a_\Bq
+
e^{i \omega_\Bq t} b_\Bq^\dagger
}.
\end{dmath}
This is a general solution, but imposing \( a_\Bq = b_{-\Bq} \) ensures \cref{eqn:qftLecture6:80} is satisfied.
This leaves us with
\begin{dmath}\label{eqn:qftLecture6:140}
\tilde{\phi}(\Bq, t) = \inv{\sqrt{2 \omega_\Bq}} \lr{
e^{-i \omega_\Bq t} a_\Bq
+
e^{i \omega_\Bq t} a_{-\Bq}^\dagger
}.
\end{dmath}
We want to show that iff
\begin{dmath}\label{eqn:qftLecture6:200}
\antisymmetric{a_\Bq}{a^\dagger_\Bp} = \lr{ 2 \pi }^3 \deltathree(\Bp - \Bq),
\end{dmath}
then
\begin{dmath}\label{eqn:qftLecture6:220}
\antisymmetric{\pi(\By, t)}{\phi(\Bx, t)} = -i \deltathree(\Bx - \By),
\end{dmath}
where everything else commutes (i.e. \(
\antisymmetric{a_\Bp}{a_\Bq} =
\antisymmetric{a_\Bp^\dagger}{a_\Bq^\dagger} = 0 \)).
We will only show one direction, but you can go the other way too.

\begin{dmath}\label{eqn:qftLecture6:240}
\phi(\Bx, t)
=
\int \frac{d^3 p}{(2\pi)^3 \sqrt{2 \omega_\Bp}} e^{i \Bp \cdot \Bx}
\lr{
   e^{-i \omega_\Bp t} a_\Bp
   +
   e^{i \omega_\Bp t} a_{-\Bp}^\dagger
}
\end{dmath}
\begin{dmath}\label{eqn:qftLecture6:260}
\pi(\Bx, t) = \dot{\phi}
=
i \int \frac{d^3 q}{(2\pi)^3 \sqrt{2 \omega_\Bq}} \omega_\Bq e^{i \Bq \cdot \Bx}
\lr{
   -e^{-i \omega_\Bq t} a_\Bq
   +
   e^{i \omega_\Bq t} a_{-\Bq}^\dagger
}.
\end{dmath}
The commutator is
\begin{dmath}\label{eqn:qftLecture6:280}
\antisymmetric{\pi(\By, t)}{\phi(\Bx, t)}
=
i \int \frac{d^3 p}{(2\pi)^3
\sqrt{2 \omega_\Bp}}
\frac{d^3 q}{(2\pi)^3 \sqrt{2 \omega_\Bq}}
\omega_\Bq
e^{i \Bp \cdot \By + i \Bq \cdot \Bx}
\antisymmetric
{
   -e^{-i \omega_\Bq t} a_\Bq
   +
   e^{i \omega_\Bq t} a_{-\Bq}^\dagger
}
{
   e^{-i \omega_\Bp t} a_\Bp
   +
   e^{i \omega_\Bp t} a_{-\Bp}^\dagger
}
=
i \int \frac{d^3 p}{(2\pi)^3
\sqrt{2 \omega_\Bp}}
\frac{d^3 q}{(2\pi)^3 \sqrt{2 \omega_\Bq}}
\omega_\Bq
e^{i \Bp \cdot \By + i \Bq \cdot \Bx}
\lr{
   - e^{i (\omega_\Bp -\omega_\Bq) t}
   \antisymmetric { a_\Bq } { a_{-\Bp}^\dagger }
   +
   e^{i (\omega_\Bq -\omega_\Bp) t}
   \antisymmetric{a_{-\Bq}^\dagger}{ a_\Bp }
}
=
i \int \frac{d^3 p}{(2\pi)^3
\sqrt{2 \omega_\Bp}}
\frac{d^3 q}{(2\pi)^3 \sqrt{2 \omega_\Bq}}
\omega_\Bq (2\pi)^3
e^{i \Bp \cdot \By + i \Bq \cdot \Bx}
\lr{
   - e^{i (\omega_\Bp -\omega_\Bq) t} \deltathree(\Bq + \Bp)
   - e^{i (\omega_\Bq -\omega_\Bp) t} \deltathree(-\Bq -\Bp)
}
=
- 2 i \int \frac{d^3 p}{(2\pi)^3
2 \omega_\Bp}
\omega_\Bp
e^{i \Bp \cdot (\By - \Bx)}
=
-i \deltathree(\By - \Bx),
\end{dmath}
which is what we wanted to prove.

\section{Free Hamiltonian.}
\index{Hamiltonian!free}
We call the \( \lambda = 0 \) case the ``free'' Hamiltonian.  Plugging in the creation and annihilation operator representation we have
\begin{dmath}\label{eqn:qftLecture6:300}
\begin{aligned}
H
&= \int d^3 x \lr{ \inv{2} \pi^2 + \inv{2} (\spacegrad \phi)^2 + \frac{m^2}{2} \phi^2 } \\
&=
\inv{2} \int d^3 x
\frac{d^3 p}{(2\pi)^3}
\frac{d^3 q}{(2\pi)^3}
\frac{
e^{i (\Bp + \Bq)\cdot \Bx}
}{\sqrt{2 \omega_\Bp}
\sqrt{2 \omega_\Bq}}
\Biglr{ \\
&\qquad   -
   (\omega_\Bp)
   (\omega_\Bq)
   \lr{
      -e^{-i \omega_\Bp t} a_\Bp
      +
      e^{i \omega_\Bp t} a_{-\Bp}^\dagger
   }
   \lr{
      -e^{-i \omega_\Bq t} a_\Bq
      +
      e^{i \omega_\Bq t} a_{-\Bq}^\dagger
   } \\
&\qquad-
   \Bp \cdot \Bq
   \lr{
      e^{-i \omega_\Bp t} a_\Bp
      +
      e^{i \omega_\Bp t} a_{-\Bp}^\dagger
   }
   \lr{
      e^{-i \omega_\Bq t} a_\Bq
      +
      e^{i \omega_\Bq t} a_{-\Bq}^\dagger
   } \\
&\qquad+
   m^2
   \lr{
      e^{-i \omega_\Bp t} a_\Bp
      +
      e^{i \omega_\Bp t} a_{-\Bp}^\dagger
   }
   \lr{
      e^{-i \omega_\Bq t} a_\Bq
      +
      e^{i \omega_\Bq t} a_{-\Bq}^\dagger
   }
}.
\end{aligned}
\end{dmath}
An immediate simplification is possible by identifying a delta function factor \( \int d^3 x e^{i(\Bp + \Bq) \cdot \Bx}/(2\pi)^3 = \deltathree(\Bp + \Bq) \), so
\begin{dmath}\label{eqn:qftLecture6:1060}
H
=
\inv{2}
\int
\frac{d^3 p}{(2\pi)^3}
\frac{1
}{2 \omega_\Bp}
\lr{
   -
   (\omega_\Bp)^2
   \lr{
      -e^{-i \omega_\Bp t} a_\Bp
      +
      e^{i \omega_\Bp t} a_{-\Bp}^\dagger
   }
   \lr{
      -e^{-i \omega_\Bp t} a_{-\Bp}
      +
      e^{i \omega_\Bp t} a_{\Bp}^\dagger
   }
}
+
   (\Bp^2 + m^2)
   \lr{
      e^{-i \omega_\Bp t} a_\Bp
      +
      e^{i \omega_\Bp t} a_{-\Bp}^\dagger
   }
   \lr{
      e^{-i \omega_\Bp t} a_{-\Bp}
      +
      e^{i \omega_\Bp t} a_{\Bp}^\dagger
   }
=
\inv{2} \int \frac{d^3 p}{(2 \pi)^3} \inv{2 \omega_\Bp}
\lr{
   a_\Bp a_{-\Bp}
   \lr{
      \cancel{-\omega_\Bp^2 e^{-2 i \omega_\Bp t}}
      +
      \cancel{\omega_\Bp^2 e^{-2 i \omega_\Bp t}}
   }
+
   a_{-\Bp}^\dagger a_{\Bp}^\dagger
   \lr{
      -\cancel{\omega_\Bp^2 e^{2 i \omega_\Bp t}}
      +
      \cancel{\omega_\Bp^2 e^{2 i \omega_\Bp t}}
   }
+
   a_\Bp a^\dagger_{\Bp} \omega^2_\Bp(1 + 1)
+
   a^\dagger_{-\Bp} a_{-\Bp} \omega^2_\Bp(1 + 1)
}
\end{dmath}
When all is said and done we are left with
\begin{dmath}\label{eqn:qftLecture6:320}
H =
\int \frac{d^3 p}{(2 \pi)^3} \frac{\omega_\Bp}{2} \lr{
   a^\dagger_{-\Bp}
   a_{-\Bp}
+
   a_{\Bp}
   a^\dagger_{\Bp}
}.
\end{dmath}
A final \( \Bp \rightarrow -\Bp \) transformation \footnote{\( \iiint_{-R}^R d^3 p \rightarrow (-1)^3 \iiint_R^{-R} d^3 p' = (-1)^6 \iiint_{-R}^R d^3 p' \).} in the first integral, puts the free Hamiltonian (\( \lambda = 0 \)) into a nice symmetric form
\boxedEquation{eqn:qftLecture6:340}{
H_0 =
\int \frac{d^3 p}{(2 \pi)^3} \frac{\omega_\Bp}{2} \lr{
   a^\dagger_{\Bp}
   a_{\Bp}
+
   a_{\Bp}
   a^\dagger_{\Bp}
}.
}

\paragraph{Vacuum energy density.}
\index{vacuum energy density}
From the commutator relationship \cref{eqn:qftLecture6:200} we can write
\begin{dmath}\label{eqn:qftLecture6:360}
a_\Bp a^\dagger_\Bq
=
a^\dagger_\Bq
a_\Bp
+ (2 \pi)^3 \deltathree(\Bp - \Bq),
\end{dmath}
so

\begin{dmath}\label{eqn:qftLecture6:380}
H_0 =
\int \frac{d^3 p}{(2 \pi)^3} \omega_\Bp
\lr{
   a^\dagger_{\Bp}
   a_{\Bp}
+
\inv{2} (2 \pi)^3 \deltathree(0)
}.
\end{dmath}
The delta function term can be interpreted using
\begin{dmath}\label{eqn:qftLecture6:400}
(2 \pi)^3 \deltathree(\Bq)
= \int d^3 x e^{i \Bq \cdot \Bx},
\end{dmath}
so when \( \Bq = 0 \)
\begin{equation}\label{eqn:qftLecture6:420}
(2 \pi)^3 \deltathree(0) = \int d^3 x = V.
\end{equation}

We can write the Hamiltonian now in terms of the volume
\boxedEquation{eqn:qftLecture6:440}{
H_0 =
\int \frac{d^3 p}{(2 \pi)^3} \omega_\Bp
   a^\dagger_{\Bp}
   a_{\Bp}
+ V_3
\int \frac{d^3 p}{(2 \pi)^3} \frac{\omega_\Bp }{2} \times 1.
}

\section{QM SHO review.}

In units with \( m = 1 \) the non-relativistic QM SHO has the Hamiltonian
\begin{dmath}\label{eqn:qftLecture6:460}
H
= \inv{2} p^2 + \frac{\omega^2}{2} q^2.
%=
%\inv{2} \lr{ (p + i \omega q)(p - i \omega q) - i \omega (q p - p q) }
%=
%\omega \lr{ \inv{\sqrt{2 \omega}} (p + i \omega q) \inv{\sqrt{2 \omega}}(p - i \omega q) + 1 }
\end{dmath}
%Observe that the we can almost factor the Hamiltonian into two conjugate operators since
%(p + i \omega q)(p - i \omega q)
%=
% = \omega \lr{ a^\dagger a + \inv{2}}.
If we define a position operator with a time-domain Fourier representation given by
\begin{dmath}\label{eqn:qftLecture6:480}
q =  \inv{\sqrt{2\omega}} \lr{ a e^{-i \omega t} + a^\dagger e^{i \omega t} },
\end{dmath}
where the Fourier coefficients \( a, a^\dagger \) are operator valued, then the momentum operator is
\begin{equation}\label{eqn:qftLecture6:500}
p = \dot{q} =
\frac{i\omega}{\sqrt{2\omega}} \lr{ -a e^{-i \omega t} + a^\dagger e^{i \omega t} },
\end{equation}
or inverting for \( a, a^\dagger \)
\begin{dmath}\label{eqn:qftLecture6:1080}
\begin{aligned}
a &= \sqrt{\frac{\omega}{2}} \lr{ q - \inv{i \omega} p } e^{-i \omega t} \\
a^\dagger &= \sqrt{\frac{\omega}{2}} \lr{ q + \inv{i \omega} p } e^{i \omega t}.
\end{aligned}
\end{dmath}
By inspection it is apparent that the product \( a^\dagger a \) will be related to the Hamiltonian (i.e. a difference of squares).  That product is
\begin{dmath}\label{eqn:qftLecture6:1120}
a^\dagger a
=
\frac{\omega}{2}
\lr{ q + \inv{i \omega} p }
\lr{ q - \inv{i \omega} p }
=
\frac{\omega}{2} \lr{ q^2 + \inv{\omega^2} p^2 - \inv{i \omega} \antisymmetric{q}{p} }
= \inv{2 \omega} \lr{ p^2 + \omega^2 q^2 - \omega },
\end{dmath}
or
\begin{dmath}\label{eqn:qftLecture6:1140}
H = \omega \lr{ a^\dagger a + \inv{2} }.
\end{dmath}
We can glean some of the properties of \( a, a^\dagger \) by computing the commutator of \( p, q \), since that has a well known value
\begin{dmath}\label{eqn:qftLecture6:520}
i = \antisymmetric{q}{p}
=
\frac{i\omega}{2 \omega} \antisymmetric
{ a e^{-i \omega t} + a^\dagger e^{i \omega t} }
{ -a e^{-i \omega t} + a^\dagger e^{i \omega t} }
=
\frac{i}{2} \lr{
\antisymmetric{a}{a^\dagger} -
\antisymmetric{a^\dagger}{a} }
=
i
\antisymmetric{a}{a^\dagger},
\end{dmath}
so
\begin{dmath}\label{eqn:qftLecture6:1100}
\antisymmetric{a}{a^\dagger} = 1.
\end{dmath}
The operator \( a^\dagger a \) is the workhorse of the Hamiltonian and worth studying independently.  In particular, assume that we have a set of states \( \ket{n} \) that are eigenstates of \( a^\dagger a \) with eigenvalues \( \lambda_n \), that is
\begin{dmath}\label{eqn:qftLecture6:1160}
a^\dagger a \ket{n} = \lambda_n \ket{n}.
\end{dmath}
The action of \( a^\dagger a \) on \( a^\dagger \ket{n} \) is easy to compute
\begin{dmath}\label{eqn:qftLecture6:1180}
a^\dagger a a^\dagger \ket{n}
=
a^\dagger \lr{ a^\dagger a + 1 } \ket{n}
=
\lr{ \lambda_n + 1 } a^\dagger \ket{n},
\end{dmath}
so \( \lambda_n + 1 \) is an eigenvalue of \( a^\dagger \ket{n} \).  The state \( a^\dagger \ket{n} \) has an energy eigenstate that is one unit of energy larger than \( \ket{n} \).  For this reason we called \( a^\dagger \) the raising (or creation) operator.
Similarly,
\begin{dmath}\label{eqn:qftLecture6:1200}
a^\dagger a a \ket{n}
=
\lr{ a a^\dagger - 1 } a \ket{n}
=
(\lambda_n - 1) a \ket{n},
\end{dmath}
so \( \lambda_n - 1 \) is the energy eigenvalue of \( a \ket{n} \), having one less unit of energy than \( \ket{n} \).
We call \( a \) the annihilation (or lowering) operator.
If we argue that there is a lowest energy state, perhaps designated as \( \ket{0} \) then we must have
\begin{dmath}\label{eqn:qftLecture6:560}
a\ket{0} = 0,
\end{dmath}
by the assumption that there are no energy eigenstates with less energy than \( \ket{0} \).
We can think of higher order states being constructed from the ground state from using the raising operator \( a^\dagger \)
\begin{dmath}\label{eqn:qftLecture6:580}
\ket{n} = \frac{(a^\dagger)^n}{\sqrt{n!}} \ket{0}.
\end{dmath}
%The Hamiltonian action
%\begin{dmath}\label{eqn:qftLecture6:600}
%\omega a^\dagger a \ket{n} = (n \omega) \ket{n} = E_n \ket{n}.
%\end{dmath}

\section{Discussion.}

We've diagonalized in the Fourier representation for the momentum space fields.  For every value of momentum \( \Bp \) we have a quantum SHO.

For our field space we call our space the Fock vacuum and
\begin{dmath}\label{eqn:qftLecture6:620}
a_\Bp\ket{0} = 0,
\end{dmath}
and call \( a_\Bp \) the ``annihilation operator'', and
call \( a^\dagger_\Bp \) the ``creation operator''.
We say that \( a^\dagger_\Bp \ket{0} \) is the creation of a state of a single particle of momentum \( \Bp \) by \( a^\dagger_\Bp \).

We are discarding the volume term, a procedure called ``normal ordering''.  We define
\begin{equation}\label{eqn:qftLecture6:640}
: \frac{a^\dagger a + a a^\dagger}{2} :
\equiv
a^\dagger a.
\end{equation}
We are essentially forgetting the vacuum energy as some sort of unobservable quantity, leaving us with the free Hamiltonian of
\begin{dmath}\label{eqn:qftLecture6:660}
H_0 =
\int \frac{d^3 p}{(2 \pi)^3} \omega_\Bp
   a^\dagger_{\Bp}
   a_{\Bp}.
\end{dmath}
Consider
\begin{dmath}\label{eqn:qftLecture6:680}
H_0
a^\dagger_\Bq \ket{0}
=
\int \frac{d^3 p}{(2 \pi)^3} \omega_\Bp
   a^\dagger_{\Bp}
   a_{\Bp}
a^\dagger_\Bq
\ket{0}
=
\int \frac{d^3 p}{(2 \pi)^3} \omega_\Bp
   a^\dagger_{\Bp}
\lr{
   a^\dagger_\Bq a_\Bp + (2 \pi)^3 \deltathree(\Bp - \Bq)
}
\ket{0}
=
\int \frac{d^3 p}{(2 \pi)^3} \omega_\Bp
   a^\dagger_{\Bp} \lr{
   a^\dagger_\Bq \cancel{a_\Bp \ket{0}}
+ (2 \pi)^3 \deltathree(\Bp - \Bq) \ket{0}
}
=
\omega_\Bq a^\dagger_\Bq \ket{0}.
\end{dmath}

\paragraph{Question:} Is it possible to modify the Lagrangian or Hamiltonian that we start with so that this vacuum ground state is eliminated?  Answer: Only by imposing super-symmetric constraints (that pairs this (bosonic) Hamiltonian to a fermionic system in a way that there is exact cancellation).

We will see that the momentum operator has the form
\begin{dmath}\label{eqn:qftLecture6:700}
\calP
=
\int \frac{d^3 p}{(2 \pi)^3} \Bp a^\dagger_\Bp a_\Bp.
\end{dmath}
We say that \( a^\dagger_\Bp a^\dagger_\Bq \ket{0} \) a two particle space with energy \( \omega_\Bp + \omega_q\), and \(
(a^\dagger_\Bp)^m (a^\dagger_\Bq)^n \ket{0} \equiv
(a^\dagger_\Bp)^m \ket{0} \otimes (a^\dagger_\Bq)^n \ket{0} \), a \( m + n \) particle space.

There is a connection to statistical mechanics that is of interest

\begin{dmath}\label{eqn:qftLecture6:720}
\expectation{E}
= \inv{Z} \sum_n E_n e^{-E_n/\kB T}
= \inv{Z} \sum_n \bra{n} e^{-\hat{H}/\kB T} \hat{H} \ket{n},
\end{dmath}
so for a SHO Hamiltonian system
\begin{dmath}\label{eqn:qftLecture6:740}
\expectation{E}
= \inv{Z} \sum_n e^{-E_n/\kB T} \bra{n} \hat{H} \ket{n}
= \inv{Z} \sum_n e^{-E_n/\kB T} \bra{n} \omega a^\dagger a \ket{n}
= \frac{\omega}{e^{\omega/\kB T} - 1 }
= \expectation{ \omega a^\dagger a }_{\kB T},
\end{dmath}
which is the \( \kB T \) ensemble average energy for a SHO system.  Note that this sum was evaluated by noting that \( \bra{n} a^\dagger a \ket{n} = n \) which leaves sums of the form
\begin{dmath}\label{eqn:qftLecture6:1220}
\frac{\sum_{n = 0}^\infty n a^n }{\sum_{n = 0}^\infty a^n}
=
a \frac{\sum_{n = 1}^\infty n a^{n-1} }{\sum_{n = 0}^\infty a^n}
=
a (1 - a) \frac{d}{da} \lr{ \inv{1 - a} }
=
\frac{a}{1 - a}.
\end{dmath}

If we consider a real scalar field of mass \( m \) we have \( \omega_\Bp = \sqrt{ \Bp^2 + m^2 } \), but for a Maxwell field \( \BE, \BB \) where \( m = 0 \), our dispersion relation is \( \omega_\Bp = \Norm{\Bp} \).

We will see that for a free Maxwell field (no charges or currents) the Hamiltonian is
\begin{dmath}\label{eqn:qftLecture6:760}
H_{\text{Maxwell}} =
\sum_{i = 1}^2
\int \frac{d^3 p}{(2 \pi)^3} \omega_\Bp {a^i}^\dagger_\Bp {a^i}_\Bp,
\end{dmath}
where \( i \) is a polarization index.

We expect that we can evaluate an average such as \cref{eqn:qftLecture6:740} for our field, and operate using the analogy

\begin{dmath}\label{eqn:qftLecture6:780}
\begin{aligned}
a a^\dagger &= a^\dagger a + 1 \\
a_\Bp a_\Bp^\dagger &= a_\Bp^\dagger a_\Bp + V_3.
\end{aligned}
\end{dmath}
so if we rescale by \( \sqrt{V_3} \)
\begin{dmath}\label{eqn:qftLecture6:800}
a_\Bp = \sqrt{V_3} \tilde{a}_\Bp,
\end{dmath}
then we have commutator relations like standard QM
\begin{dmath}\label{eqn:qftLecture6:820}
\tilde{a} \tilde{a}^\dagger = \tilde{a}^\dagger \tilde{a} + 1.
\end{dmath}

So we can immediately evaluate the energy expectation for our quantized fields
\begin{dmath}\label{eqn:qftLecture6:840}
\expectation{H_0}
=
\expectation{
\int \frac{d^3 p}{(2 \pi)^3} \omega_\Bp a_\Bp^\dagger a_\Bp
}
=
\int \frac{d^3 p}{(2 \pi)^3} \omega_\Bp V_3 \expectation{ \tilde{a}^\dagger_\Bp a_\Bp }
=
V_3
\int \frac{d^3 p}{(2 \pi)^3} \frac{\omega_\Bp}{e^{\omega_\Bp/\kB T} - 1}.
\end{dmath}
Using this with the Maxwell field, we have a factor of two from polarization
\begin{dmath}\label{eqn:qftLecture6:860}
U^{\text{Maxwell}} = 2 V_3
\int \frac{d^3 p}{(2 \pi)^3} \frac{\Norm{\Bp}}{e^{\omega_\Bp/\kB T} - 1},
\end{dmath}
which is Planck's law describing the blackbody energy spectrum.


   \chapter{Symmetries, translation currents, energy momentum tensor.}
      %
% Copyright � 2017 Peeter Joot.  All Rights Reserved.
% Licenced as described in the file LICENSE under the root directory of this GIT repository.
%
%{
%%\input{../latex/blogpost.tex}
%%\renewcommand{\basename}{qftLecture7}
%%\renewcommand{\dirname}{notes/phy2403/}
%%\newcommand{\keywords}{PHY2403H}
%%\input{../latex/peeter_prologue_print2.tex}
%%
%%%\usepackage{phy2403}
%%\usepackage{peeters_braket}
%%\usepackage{peeters_layout_exercise} % makedefinition
%%\usepackage{peeters_figures}
%%\usepackage{mathtools}
%%\usepackage{siunitx}
%%\usepackage{macros_cal} % LL
%%
%%\beginArtNoToc
%%\generatetitle{PHY2403H Quantum Field Theory.  Lecture 7: Symmetries, translation currents, energy momentum tensor.  Taught by Prof. Erich Poppitz}
\chapter{Symmetries, translation currents, energy momentum tensor.}
\index{translation current}
\label{chap:qftLecture7}

%\paragraph{DISCLAIMER: Very rough notes from class, with some additional side notes.}
%
%These are notes for the UofT course PHY2403H, Quantum Field Theory I, taught by Prof. Erich Poppitz fall 2018.
%%, covering \textchapref{{1}} \citep{peskin1995introduction} content.
%
\section{Symmetries.}
\index{symmetries}

Given the complexities of the non-linear systems we want to investigate, examination of symmetries gives us simpler problems that we can solve.

\begin{itemize}
\item ``internal'' symmetries.  This means that the symmetries do not act on space time \( (\Bx, t) \).  An example is
\begin{dmath}\label{eqn:qftLecture7:20}
\phi^i =
\begin{bmatrix}
\psi_1 \\
\psi_2 \\
\vdots \\
\psi_N \\
\end{bmatrix}
\end{dmath}
If we map
\( \phi^i \rightarrow O^i_j \phi^j \) where \( O^\T O = 1 \), then we call this an internal symmetry.
The corresponding Lagrangian density might be something like
\begin{dmath}\label{eqn:qftLecture7:40}
\LL = \inv{2} \partial_\mu \Bphi \cdot \partial^\mu \Bphi - \frac{m^2}{2} \Bphi \cdot \Bphi - V(\Bphi \cdot \Bphi)
\end{dmath}
\item spacetime symmetries: Translations, rotations, boosts, dilatations.  We will consider continuous symmetries, which can be defined as a succession of infinitesimal transformations.
An example from \(O(2)\) is a rotation
\begin{dmath}\label{eqn:qftLecture7:60}
\begin{bmatrix}
\phi^1 \\
\phi^2 \\
\end{bmatrix}
\rightarrow
\begin{bmatrix}
\cos\alpha & \sin\alpha \\
-\sin\alpha & \cos\alpha \\
\end{bmatrix}
\begin{bmatrix}
\phi^1 \\
\phi^2
\end{bmatrix},
\end{dmath}
or if \( \alpha \sim 0 \)
\begin{dmath}\label{eqn:qftLecture7:80}
\begin{bmatrix}
\phi^1 \\
\phi^2 \\
\end{bmatrix}
\rightarrow
\begin{bmatrix}
1 & \alpha \\
-\alpha & 1\\
\end{bmatrix}
\begin{bmatrix}
\phi^1 \\
\phi^2
\end{bmatrix}
=
\begin{bmatrix}
\phi^1 \\
\phi^2
\end{bmatrix}
+
\alpha
\begin{bmatrix}
\phi^2 \\
-\phi^1
\end{bmatrix}
\end{dmath}
In index notation we write
\begin{dmath}\label{eqn:qftLecture7:100}
\phi^i \rightarrow \phi^i + \alpha e^{ij} \phi^j,
\end{dmath}
where \( \epsilon^{12} = +1, \epsilon^{21} = -1 \) is the completely antisymmetric tensor.  This can be written in more general form as
\begin{dmath}\label{eqn:qftLecture7:120}
\phi^i \rightarrow \phi^i + \delta \phi^i,
\end{dmath}
where \( \delta \phi^i \) is considered to be an infinitesimal transformation.
\end{itemize}

\makedefinition{Symmetry}{dfn:qftLecture7:140}{
A symmetry means that there is some transformation
\begin{equation*}
\phi^i \rightarrow \phi^i + \delta \phi^i,
\end{equation*}
where
\( \delta \phi^i \) is an infinitesimal transformation, and the equations of motion are invariant under this transformation.
} % definition

\maketheorem{Noether's theorem (1st).}{thm:qftLecture7:160}{
If the equations of motion re invariant under \( \phi^\mu \rightarrow \phi^\mu + \delta \phi^\mu \), then there exists a conserved current \( j^\mu \) such that \( \partial_\mu j^\mu = 0 \).
} % theorem

Noether's first theorem applies to global symmetries, where the parameters are the same for all \( (\Bx, t)\).  Gauge symmetries are not examples of such global symmetries.

Given a Lagrangian density \( \LL(\phi(x), \phi_{,\mu}(x)) \), where \( \phi_{,\mu} \equiv \partial_\mu \phi \).  The action is
\begin{dmath}\label{eqn:qftLecture7:160}
S = \int d^d x \LL.
\end{dmath}
The equations of motion are invariant if under \( \phi(x) \rightarrow \phi'(x) = \phi(x) + \delta_\epsilon \phi(x)\), we have
\begin{dmath}\label{eqn:qftLecture7:180}
\LL(\phi) \rightarrow \LL'(\phi') = \LL(\phi) + \partial_\mu J_\epsilon^\mu(\phi) + O(\epsilon^2).
\end{dmath}
Then there exists a conserved current.  In QFT we say that the E.O.M's are ``on shell''.
Note that \cref{eqn:qftLecture7:180} is a
symmetry since we have added a total derivative to the Lagrangian which leaves the equations of motion of unchanged.

In general, the change of action under arbitrary variation of \( \delta \phi\) of the fields is
\begin{dmath}\label{eqn:qftLecture7:200}
\delta S
=
\int d^d x \delta \LL(\phi, \partial_\mu \phi)
=
\int d^d x \lr{
\PD{\phi}{\LL} \delta \phi
+
\PD{(\partial_\mu \phi)}{\LL} \delta \partial_\mu \phi
}
=
\int d^d x \lr{
\partial_\mu \lr{ \PD{(\partial_\mu \phi)}{\LL} } \delta \phi
+
\PD{(\partial_\mu \phi)}{\LL} \partial_\mu \delta \phi
}
=
\int d^d x
\partial_\mu \lr{ \frac{\delta \LL}{\delta(\partial_\mu \phi)} \delta \phi }
\end{dmath}
However from \cref{eqn:qftLecture7:180}
\begin{dmath}\label{eqn:qftLecture7:220}
\delta_\epsilon \LL = \partial_\mu J_\epsilon^\mu(\phi, \partial_\mu \phi),
\end{dmath}
so after equating these variations we fine that
\begin{dmath}\label{eqn:qftLecture7:240}
\delta S = \int d^d x \delta_\epsilon \LL = \int d^d x \partial_\mu J_\epsilon^\mu,
\end{dmath}
or
\begin{dmath}\label{eqn:qftLecture7:260}
0 = \int d^d x
\partial_\mu \lr{ \frac{\delta \LL}{\delta(\partial_\mu \phi)} \delta \phi - J_\epsilon^\mu },
\end{dmath}
or \( \partial_\mu j^\mu = 0 \) provided
\boxedEquation{eqn:qftLecture7:280}{
j^\mu =
\frac{\delta \LL}{\delta(\partial_\mu \phi)} \delta_\epsilon \phi  - J_\epsilon^\mu.
}

Integrating the divergence of the current over a space time volume, perhaps that of \cref{fig:spacetimeCylinder:spacetimeCylinderFig1}, is also zero.  That is
%\imageFigure{../figures/phy2403-quantum-field-theory/spacetimeCylinderFig1}{Cylindrical spacetime boundary.}{fig:spacetimeCylinder:spacetimeCylinderFig1}{0.3}
\begin{dmath}\label{eqn:qftLecture7:300}
0 =
\int d^4 x \, \partial_\mu j^\mu
=
\int d^3 \Bx dt \, \partial_\mu j^\mu
=
\int d^3 \Bx dt \, \partial_t j^0 -
\cancel{\int d^3 \Bx dt \spacegrad \cdot \Bj},
\end{dmath}
where the spatial divergence is zero assuming there's no current leaving the volume on the infinite boundary (no \(\Bj\) at spatial infinity.)

We write
\begin{dmath}\label{eqn:qftLecture7:560}
Q = \int d^3x j^0,
\end{dmath}
and call this the on-shell charge associated with the symmetry.
\index{on-shell}

\section{Spacetime translation.}
\index{spacetime translation}

A spacetime translation has the form
\begin{equation}\label{eqn:qftLecture7:320}
x^\mu \rightarrow {x'}^\mu = x^\mu + a^\mu,
\end{equation}
\begin{equation}\label{eqn:qftLecture7:340}
\phi(x) \rightarrow \phi'(x') = \phi(x)
\end{equation}
(contrast this to a Lorentz transformation that had the form \( x^\mu \rightarrow {x'}^\mu = {\Lambda^\mu}_\nu x^\nu \)).

If \(\phi'(x + a) = \phi(x) \), then
\begin{dmath}\label{eqn:qftLecture7:360}
\phi'(x) + a^\mu \partial_\mu \phi'(x) =
\phi'(x) + a^\mu \partial_\mu \phi(x) =
\phi(x),
\end{dmath}
so
\begin{dmath}\label{eqn:qftLecture7:380}
\phi'(x)
= \phi(x) - a^\mu \partial_\mu \phi'(x)
= \phi(x) + \delta_a \phi(x),
\end{dmath}
or
\begin{dmath}\label{eqn:qftLecture7:580}
\delta_a \phi(x) = - a^\mu \partial_\mu \phi(x).
\end{dmath}
Under \( \phi \rightarrow \phi - a^\mu \partial_\mu \phi \), we have
\begin{dmath}\label{eqn:qftLecture7:400}
\LL(\phi) \rightarrow \LL(\phi) - a^\mu \partial_\mu \LL.
\end{dmath}
Let's calculate this with our scalar theory Lagrangian
\begin{dmath}\label{eqn:qftLecture7:420}
\LL = \inv{2} \partial_\mu \phi \partial^\mu \phi - \frac{m^2}{2} \phi^2 - V(\phi)
\end{dmath}
The Lagrangian variation is
\begin{dmath}\label{eqn:qftLecture7:440}
\evalbar{\delta \LL}{\phi \rightarrow \phi + \delta \phi, \delta\phi = - a^\mu \partial_\mu \phi}
=
(\partial_\mu \phi) \delta (\partial^\mu \phi) - m^2 \phi \delta \phi - \PD{\phi}{V} \delta \phi
=
(\partial_\mu \phi)(-a^\nu \partial_\nu \phi \partial^\mu \phi) + m^2 \phi a^\nu \partial_\nu \phi + \PD{\phi}{V} a^\nu \partial_\nu \phi
=
- a^\nu \partial_\nu \lr{ \inv{2} \partial_\mu \partial^\mu \phi - \frac{m^2}{2} \phi^2 - V(\phi) }
=
- a^\nu \partial_\nu \LL.
\end{dmath}

So the current is
\begin{dmath}\label{eqn:qftLecture7:600}
j^\mu
=
(\partial^\mu \phi) (-a^\nu \partial_\nu \phi) + a^\nu \LL
=
-a^\nu \lr{ \partial^\mu \phi \partial_\nu \phi - \LL }
\end{dmath}

We really have a current for each \( \nu \) direction and can make that explicit writing
\begin{dmath}\label{eqn:qftLecture7:460}
\delta_\nu \LL = -\partial_\nu \LL
= - \partial_\mu \lr{ {\delta^\mu}_\nu \LL }
= \partial_\mu {J^\mu}_\nu
\end{dmath}
we write
\begin{dmath}\label{eqn:qftLecture7:480}
{j^\mu}_\nu = \PD{x_\mu}{\phi} \lr{ - \PD{x^\nu}{\phi} } + {\delta^\mu}_\nu \LL,
\end{dmath}
where \( \nu \) are labels which coordinates are translated:
\begin{dmath}\label{eqn:qftLecture7:500}
\begin{aligned}
\partial_\nu \phi &= - \partial_\nu \phi \\
\partial_\nu \LL &= - \partial_\nu \LL.
\end{aligned}
\end{dmath}
We call the conserved quantities elements of the energy-momentum tensor, and write it as
\index{energy momentum tensor}
\boxedEquation{eqn:qftLecture7:520}{
{T^\mu}_\nu = -\PD{x_\mu}{\phi} \PD{x^\nu}{\phi} + {\delta^\mu}_\nu \LL.
}

Incidentally, we picked a non-standard sign convention for the tensor, as an explicit expansion of \( T^{00} \), the energy density component, shows
\begin{dmath}\label{eqn:qftLecture7:540}
{T^0}_0 =
-\PD{t}{\phi}
\PD{t}{\phi}
+\inv{2}
\PD{t}{\phi}
\PD{t}{\phi}
- \inv{2} (\spacegrad \phi) \cdot (\spacegrad \phi)
- \frac{m^2}{2} \phi^2 - V(\phi)
=
-\inv{2} \PD{t}{\phi} \PD{t}{\phi}
- \inv{2} (\spacegrad \phi) \cdot (\spacegrad \phi)
- \frac{m^2}{2} \phi^2 - V(\phi).
\end{dmath}
Had we translated by \( -a^\mu \) we'd have a positive definite tensor instead.

%%\section{Problems.}
%%
%%\makeproblem{Adding a total derivative to the Lagrangian}{problem:qftLecture7:560}{
%%Show that adding a total derivative to the Lagrangian density leaves the equations of motion unchanged.
%%} % problem
%%
%%\makeanswer{problem:qftLecture7:560}{
%%Given
%%\begin{dmath}\label{eqn:qftLecture7:620}
%%\LL' = \LL + \partial_\mu a^\mu
%%\end{dmath}
%%} % answer

%}
%\EndNoBibArticle

      \section{Problems.}
         %
% Copyright � 2016 Peeter Joot.  All Rights Reserved.
% Licenced as described in the file LICENSE under the root directory of this GIT repository.
%
%{
%\input{../latex/blogpost.tex}
%\renewcommand{\basename}{noetherCurrentScalarField}
%\renewcommand{\dirname}{notes/phy1520/}
%%\newcommand{\dateintitle}{}
%%\newcommand{\keywords}{}
%
%\input{../latex/peeter_prologue_print2.tex}
%
%\usepackage{peeters_layout_exercise}
%\usepackage{peeters_braket}
%\usepackage{peeters_figures}
%\usepackage{macros_cal}
%
%\beginArtNoToc
%
%\generatetitle{Energy-momentum tensor for a scalar field}
%\chapter{Energy-momentum tensor for a scalar field}
%\label{chap:noetherCurrentScalarField}
% \citep{sakurai2014modern} pr X.Y
\makeproblem{Energy-momentum tensor for a scalar field}{problem:noetherCurrentScalarField:1}{
It is claimed in \citep{LukeQFT} (3.2.1) that the momentum components of the energy-momentum tensor was found to be

\begin{dmath}\label{eqn:noetherCurrentScalarField:20}
\Be_n \int d^3 x T^{0 n} = \int d^3 k \Bk a_{\Bk}^\dagger a_{\Bk}.
\end{dmath}

\makesubproblem{}{problem:noetherCurrentScalarField:1:a}
Calculate this.

\makesubproblem{}{problem:noetherCurrentScalarField:1:b}
Calculate the other energy-momentum tensor components for the spacelike components.

\makesubproblem{}{problem:noetherCurrentScalarField:1:c}

Calculate the other energy-momentum tensor components for the Hamiltonian component.
} % problem

\makeanswer{problem:noetherCurrentScalarField:1}{
First, from the Noether current for the scalar field Lagrangian in question, what is the energy-momentum tensor explicitly?

\begin{dmath}\label{eqn:noetherCurrentScalarField:40}
T^{\mu \nu}
= \Pi^\mu \partial^\nu \phi - g^{\mu \nu} \LL
= \Pi^\mu \partial^\nu \phi - g^{\mu \nu} \inv{2} \lr{ \partial_\alpha \phi \partial^\alpha \phi - \mu^2 \phi^2 }
= \Pi^\mu \Pi^\nu - g^{\mu \nu} \inv{2} \lr{ \Pi_\alpha \Pi^\alpha - \mu^2 \phi^2 }
= \Pi^\mu \Pi^\nu - \inv{2} g^{\mu \nu} g_{\alpha\beta} \Pi^\beta \Pi^\alpha + \inv{2} g^{\mu \nu} \mu^2 \phi^2.
\end{dmath}

Consider some special cases for the indexes.  For \( \mu = \nu = 0 \), the result is the Hamiltonian density

\begin{dmath}\label{eqn:noetherCurrentScalarField:200}
T^{00}
= \Pi^0 \Pi^0 - \inv{2} g^{0 0} \Pi_\alpha \Pi^\alpha + \inv{2} g^{0 0} \mu^2 \phi^2
= \Pi^0 \Pi^0 - \inv{2} \Pi_\alpha \Pi^\alpha + \inv{2} \mu^2 \phi^2
= \inv{2} \Pi^0 \Pi^0 - \inv{2} \Pi_n \Pi^n + \inv{2} \mu^2 \phi^2
= \inv{2} \Pi^2 + \inv{2} (\spacegrad \phi)^2 + \inv{2} \mu^2 \phi^2,
\end{dmath}

where \( \Pi^2 = (\partial_0 \phi)^2 \ne \partial^2 \phi \).  For any \( \mu \ne \nu \) the off diagonal metric elements are zero, leaving just
\begin{dmath}\label{eqn:noetherCurrentScalarField:220}
T^{\mu\nu} = \Pi^\mu \Pi^\nu.
\end{dmath}

Finally, when \( n \ne 0 \), the remaining diagonal terms are
\begin{dmath}\label{eqn:noetherCurrentScalarField:240}
T^{nn}
= \Pi^n \Pi^n - \inv{2} g^{n n} \Pi_\alpha \Pi^\alpha + \inv{2} g^{n n} n^2 \phi^2
= \Pi^n \Pi^n + \inv{2} \Pi_\alpha \Pi^\alpha - \inv{2} \mu^2 \phi^2
= \inv{2} \Pi^2 + \Pi^n \Pi^n - \inv{2} \Pi^m \Pi^m - \inv{2} \mu^2 \phi^2
= \inv{2} \Pi^2 + \inv{2} \Pi^n \Pi^n - \inv{2} \sum_{m\ne n,0} \Pi^m \Pi^m - \inv{2} \mu^2 \phi^2
= \inv{2} \sum_{m = n,0} \Pi^m \Pi^m - \inv{2} \sum_{m\ne n,0} \Pi^m \Pi^m - \inv{2} \mu^2 \phi^2.
\end{dmath}

The canonical momenta are

\begin{dmath}\label{eqn:noetherCurrentScalarField:60}
\Pi^\mu
=
\partial^\mu
\int \frac{d^3 k}{(2\pi)^{3/2} \sqrt{ 2 \omega_k }} \lr{ a_{\Bk} e^{-i k \cdot x} + a_{\Bk}^\dagger e^{i k \cdot x} },
\end{dmath}

but
\begin{dmath}\label{eqn:noetherCurrentScalarField:80}
\partial^\mu e^{i k \cdot x}
=
\partial^\mu \exp\lr{ i k^\alpha x_\alpha }
=
i k^\mu \exp\lr{ i k \cdot x },
\end{dmath}

so
\begin{dmath}\label{eqn:noetherCurrentScalarField:100}
\Pi^\mu
=
i
\int \frac{d^3 k k^\mu}{(2\pi)^{3/2} \sqrt{ 2 \omega_k }} \lr{ - a_{\Bk} e^{-i k \cdot x} + a_{\Bk}^\dagger e^{i k \cdot x} }
=
i
\int \frac{d^3 k k^\mu}{(2\pi)^{3/2} \sqrt{ 2 \omega_k }} \lr{ - a_{\Bk} e^{-i \omega_k t + \Bk \cdot \Bx} + a_{\Bk}^\dagger e^{i \omega_k t - i \Bk \cdot \Bx} }
=
i
\int \frac{d^3 k k^\mu}{(2\pi)^{3/2} \sqrt{ 2 \omega_k }}
\lr{
- a_{\Bk} e^{-i \omega_k t }
+ a_{-\Bk}^\dagger e^{i \omega_k t }
}
e^{ i \Bk \cdot \Bx}
.
\end{dmath}

This gives
\begin{dmath}\label{eqn:noetherCurrentScalarField:120}
\int d^3 x \Pi^\mu \Pi^\nu
=
-
\inv{2}
\int d^3 x \frac{d^3 k d^3 p}{(2\pi)^{3} } \frac{ k^\mu p^\nu}{\sqrt{ \omega_k \omega_p }}
\lr{
- a_{\Bk} e^{-i \omega_k t }
+ a_{-\Bk}^\dagger e^{i \omega_k t }
}
\lr{
- a_{\Bp} e^{-i \omega_p t }
+ a_{-\Bp}^\dagger e^{i \omega_p t }
}
e^{ i (\Bp + \Bk) \cdot \Bx}
=
-
\inv{2}
\int d^3 k d^3 p \frac{ k^\mu p^\nu}{\sqrt{ \omega_k \omega_p }}
\lr{
- a_{\Bk} e^{-i \omega_k t }
+ a_{-\Bk}^\dagger e^{i \omega_k t }
}
\lr{
- a_{\Bp} e^{-i \omega_p t }
+ a_{-\Bp}^\dagger e^{i \omega_p t }
}
\deltathree( \Bp + \Bk )
=
-
\inv{2}
\int d^3 k d^3 p \frac{ k^\mu p^\nu}{\omega_k}
\lr{
  a_{\Bk} a_{-\Bk} e^{-2 i \omega_k t }
- a_{\Bk} a_{\Bk}^\dagger
- a_{-\Bk}^\dagger a_{-\Bk}
+ a_{-\Bk}^\dagger a_{\Bk}^\dagger e^{2 i \omega_k t }
}
%\lr{
%- a_{\Bk} e^{-i \omega_k t }
%+ a_{-\Bk}^\dagger e^{i \omega_k t }
%}
%\lr{
%- a_{-\Bk} e^{-i \omega_k t }
%+ a_{\Bk}^\dagger e^{i \omega_k t }
%}
\deltathree(\Bp + \Bk)
%%%%
%=
%-\inv{2} \int d^3 x \inv{(2\pi)^3}
%\int d^3 k d^3 j \frac{k^\mu j^\nu}{\sqrt{\omega_k \omega_j}}
%\lr{ - a_{\Bk} e^{-i \omega_k t + \Bk \cdot \Bx} + a_{\Bk}^\dagger e^{i \omega_k t - i \Bk \cdot \Bx} }
%\lr{ - a_j e^{-i \omega_j t + \Bj \cdot \Bx} + a_j^\dagger e^{i \omega_j t - i \Bj \cdot \Bx} }
%=
%-\inv{2} \int d^3 x \inv{(2\pi)^3}
%\int d^3 k d^3 j \frac{k^\mu j^\nu}{\sqrt{\omega_k \omega_j}}
%\lr{
%  a_{\Bk} a_j e^{-i (\omega_j + \omega_k) t + (\Bj + \Bk) \cdot \Bx}
%- a_{\Bk} a_j^\dagger e^{i (\omega_j - \omega_k) t - i (\Bj -\Bk) \cdot \Bx}
%- a_{\Bk}^\dagger a_j e^{-i (\omega_j -\omega_k) t - (\Bk - \Bj) \cdot \Bx}
%+ a_{\Bk}^\dagger a_j^\dagger e^{i (\omega_j + \omega_k) t - i (\Bj + \Bk) \cdot \Bx}
%}
%=
%-\inv{2}
%\int d^3 k d^3 j \frac{k^\mu j^\nu}{\sqrt{\omega_k \omega_j}}
%\lr{
%  a_{\Bk} a_j e^{-i (\omega_j + \omega_k) t } \deltathree(\Bj + \Bk)
%- a_{\Bk} a_j^\dagger e^{i (\omega_j - \omega_k) t } \deltathree(\Bj -\Bk)
%- a_{\Bk}^\dagger a_j e^{-i (\omega_j -\omega_k) t } \deltathree (\Bk - \Bj)
%+ a_{\Bk}^\dagger a_j^\dagger e^{i (\omega_j + \omega_k) t } \deltathree (\Bj + \Bk)
%}
.
\end{dmath}

Further reduction of the leading \( k^\mu p^\nu \) term has a sign that depends on the values of the indices.

\makeSubAnswer{}{problem:noetherCurrentScalarField:1:a}

First consider the momentum case where one of \( \mu \), or \( \nu \) is zero

\begin{dmath}\label{eqn:noetherCurrentScalarField:140}
\int d^3 x \Pi^\mu \Pi^0 =
\int d^3 x \Pi^0 \Pi^\mu
=
-\inv{2}
\int d^3 k k^\mu
\lr{
  a_{\Bk} a_{-\Bk} e^{-2 i \omega_k t }
- a_{\Bk} a_{\Bk}^\dagger
- a_{\Bk}^\dagger a_{\Bk}
+ a_{\Bk}^\dagger a_{-\Bk}^\dagger e^{2 i \omega_k t }
}.
\end{dmath}

For \( \mu \ne 0 \) this can be written as a vector operator
\begin{dmath}\label{eqn:noetherCurrentScalarField:440}
\begin{aligned}
\Be_n \int d^3 x T^{0 n}
&=
-\inv{2}
\int d^3 k \Bk
\lr{
  a_{\Bk} a_{-\Bk} e^{-2 i \omega_k t }
+ a_{\Bk}^\dagger a_{-\Bk}^\dagger e^{2 i \omega_k t }
} \\
&\quad +
\inv{2}
\int d^3 k \Bk
\lr{
  a_{\Bk} a_{\Bk}^\dagger
+ a_{\Bk}^\dagger a_{\Bk}
}
\end{aligned}
\end{dmath}

To get the desired result the time dependent terms have to be made to go away somehow.  Consider a spherical parameterization of the momentum space
\begin{dmath}\label{eqn:noetherCurrentScalarField:460}
\Bk = k \lr{ \sin\theta \cos\phi, \sin\theta \sin\phi, \cos\theta },
\end{dmath}

Note that the volume element is
\begin{dmath}\label{eqn:noetherCurrentScalarField:480}
d^3 k = k^2 \sin\theta dk \wedge d\theta \wedge d\phi,
\end{dmath}
where \( k \in [0, \infty]\), \(\theta \in [0, \pi]\), and \( \phi \in [0, 2\pi] \).  If we map \( \Bk \rightarrow -\Bk \), the volume element becomes
\begin{dmath}\label{eqn:noetherCurrentScalarField:500}
d^3 k = (-k)^2 \sin\theta d(-k) \wedge d\theta \wedge d\phi,
\end{dmath}
over the same angular intervals, but \( k \in [-\infty, 0]\).  Flipping the sign of the time dependent operator products gives
\begin{dmath}\label{eqn:noetherCurrentScalarField:520}
  a_{\Bk} a_{-\Bk} e^{-2 i \omega_k t }
+ a_{\Bk}^\dagger a_{-\Bk}^\dagger e^{2 i \omega_k t }
\rightarrow
  a_{-\Bk} a_{\Bk} e^{-2 i \omega_k t }
+ a_{-\Bk}^\dagger a_{\Bk}^\dagger e^{2 i \omega_k t }
=
  a_{\Bk} a_{-\Bk} e^{-2 i \omega_k t }
+ a_{\Bk}^\dagger a_{-\Bk}^\dagger e^{2 i \omega_k t },
\end{dmath}
which shows that this is an even function in \( \Bk \).  The even characteristics of the volume element and time dependent terms and the odd character of the momentum vector \( \Bk \) can be used to show that these terms integrate out to zero.  Let's compute the integral by averaging the momentum operator using both parameterization sign options.  First write
\begin{dmath}\label{eqn:noetherCurrentScalarField:540}
f(\Bk) =
  a_{\Bk} a_{-\Bk} e^{-2 i \omega_k t }
+ a_{\Bk}^\dagger a_{-\Bk}^\dagger e^{2 i \omega_k t },
\end{dmath}
so
\begin{dmath}\label{eqn:noetherCurrentScalarField:560}
\int d^3 k \Bk f(\Bk)
=
\inv{2} \int d^3 k \Bk f(\Bk)
+
\inv{2} \int d^3 k' \Bk' f(\Bk')
=
\inv{2} \int_0^\infty k^2 dk \int_0^\pi \sin\theta d\theta \int_0^{2\pi}
k \kcap(\theta, \phi) %\lr{ \sin\theta \cos\phi, \sin\theta \sin\phi, \cos\theta }
f(\Bk)
+
\inv{2} \int_{-\infty}^0 k^2 d(-k) \int_0^\pi \sin\theta d\theta \int_0^{2\pi}
(-k) \kcap(\theta, \phi) %\lr{ \sin\theta \cos\phi, \sin\theta \sin\phi, \cos\theta }
f(-\Bk)
=
\inv{2} \int_0^\pi \sin\theta d\theta \int_0^{2\pi} d\phi \kcap
\lr{
\int_0^\infty k^3 dk f(\Bk)
+
\int_{-\infty}^0 k^3 dk f(-\Bk)
}
=
\inv{2} \int_0^\pi \sin\theta d\theta \int_0^{2\pi} d\phi \kcap
\lr{
\int_0^\infty k^3 dk f(\Bk)
-
\int_{0}^\infty k^3 dk f(\Bk)
}
= 0,
\end{dmath}
so the momentum is reduced to
\begin{dmath}\label{eqn:noetherCurrentScalarField:580}
\Be_n \int d^3 x T^{0 n}
=
\inv{2}
\int d^3 k \Bk
\lr{
  a_{\Bk} a_{\Bk}^\dagger
+ a_{\Bk}^\dagger a_{\Bk}
}
=
\inv{2}
\int d^3 k \Bk
\lr{
  2 a_{\Bk}^\dagger a_\Bk
+ \antisymmetric{a_\Bk}{a_{\Bk}^\dagger}
}
=
\int d^3 k \Bk
\lr{
  a_{\Bk}^\dagger a_\Bk
+ \inv{2} \deltathree(0)
}.
\end{dmath}

An argument like that of \citep{peskin1995introduction} can be used to dismiss the unphysical infinity associated with the ground state energy level, leaving just
\boxedEquation{eqn:noetherCurrentScalarField:600}{
\Be_n \int d^3 x T^{0 n}
=
\int d^3 k \Bk
a_{\Bk}^\dagger
  a_{\Bk}
.
}

\makeSubAnswer{}{problem:noetherCurrentScalarField:1:b}

For \( \mu = m \ne 0 \), and \( \nu = n \ne 0 \), we have

\begin{dmath}\label{eqn:noetherCurrentScalarField:620}
\int d^3 x \Pi^m \Pi^n
=
\inv{2}
\int d^3 k \frac{ k^m k^n }{\omega_k}
\lr{
  a_{\Bk} a_{-\Bk} e^{-2 i \omega_k t }
- a_{\Bk} a_{\Bk}^\dagger
- a_{-\Bk}^\dagger a_{-\Bk}
+ a_{-\Bk}^\dagger a_{\Bk}^\dagger e^{2 i \omega_k t }
}.
\end{dmath}

Can the time dependent terms be killed in this case?

\makeSubAnswer{}{problem:noetherCurrentScalarField:1:c}

TODO: some stuff is wrong here.

For \( \nu \ne 0 \)

\begin{dmath}\label{eqn:noetherCurrentScalarField:160}
\int d^3 x \Pi^\mu \Pi^\nu
=
-\inv{2}
\int d^3 k \frac{k^\mu k^\nu}{\omega_k}
\lr{
- a_{\Bk} a_{-\Bk} e^{- 2 i \omega_k t }
- a_{\Bk} a_{\Bk}^\dagger
- a_{\Bk}^\dagger a_{\Bk}
- a_{\Bk}^\dagger a_{-\Bk}^\dagger e^{ 2 i \omega_k t }
}
=
 \inv{2}
\int d^3 k \frac{k^\mu k^\nu}{\omega_k}
\lr{
  a_{\Bk} a_{-\Bk} e^{- 2 i \omega_k t }
+ a_{\Bk} a_{\Bk}^\dagger
+ a_{\Bk}^\dagger a_{\Bk}
+ a_{\Bk}^\dagger a_{-\Bk}^\dagger e^{ 2 i \omega_k t }
}.
\end{dmath}

Here's a summary of these products

\begin{subequations}
\label{eqn:noetherCurrentScalarField:260}
\begin{dmath}\label{eqn:noetherCurrentScalarField:300}
\int d^3 x \Pi^0 \Pi^0
=
-\inv{2}
\int d^3 k \omega_k
\lr{
  a_{\Bk} a_{-\Bk} e^{-2 i \omega_k t }
- a_{\Bk} a_{\Bk}^\dagger
- a_{\Bk}^\dagger a_{\Bk}
+ a_{\Bk}^\dagger a_{-\Bk}^\dagger e^{2 i \omega_k t }
},
\end{dmath}
\begin{dmath}\label{eqn:noetherCurrentScalarField:280}
\int d^3 x \Pi^n \Pi^0
= \int d^3 x \Pi^0 \Pi^n
=
-\inv{2}
\int d^3 k k^n
\lr{
  a_{\Bk} a_{-\Bk} e^{-2 i \omega_k t }
- a_{\Bk} a_{\Bk}^\dagger
- a_{\Bk}^\dagger a_{\Bk}
+ a_{\Bk}^\dagger a_{-\Bk}^\dagger e^{2 i \omega_k t }
},
\end{dmath}
%\begin{dmath}\label{eqn:noetherCurrentScalarField:320}
%\int d^3 x \Pi^n \Pi^n
%=
% \inv{2}
%\int d^3 k \frac{k^n k^n}{\omega_k}
%\lr{
%  a_{\Bk} a_{-\Bk} e^{- 2 i \omega_k t }
%+ a_{\Bk} a_{\Bk}^\dagger
%+ a_{\Bk}^\dagger a_{\Bk}
%+ a_{\Bk}^\dagger a_{-\Bk}^\dagger e^{ 2 i \omega_k t }
%},
%\end{dmath}
\begin{dmath}\label{eqn:noetherCurrentScalarField:340}
\int d^3 x \Pi^m \Pi^n
=
\inv{2}
\int d^3 k \frac{k^m k^n}{\omega_k}
\lr{
  a_{\Bk} a_{-\Bk} e^{- 2 i \omega_k t }
+ a_{\Bk} a_{\Bk}^\dagger
+ a_{\Bk}^\dagger a_{\Bk}
+ a_{\Bk}^\dagger a_{-\Bk}^\dagger e^{ 2 i \omega_k t }
}.
\end{dmath}
\end{subequations}

For the mass term it was previously found that

\begin{dmath}\label{eqn:noetherCurrentScalarField:180}
\inv{2} \int d^3 x \mu^2 \phi^2
=
\frac{\mu^2}{4}
\int
d^3 k
\inv{ \omega_k }
\lr{
 a_{-\Bk} a_{\Bk} e^{- 2 i \omega_k t }
+a_{-\Bk}^\dagger a_{\Bk}^\dagger e^{2 i \omega_k t }
+a_{\Bk} a_{\Bk}^\dagger
+a_{\Bk}^\dagger a_{\Bk}
}.
\end{dmath}

The Hamiltonian component has been previously calculated, and resolves to

\begin{dmath}\label{eqn:noetherCurrentScalarField:360}
\int d^3 x T^{00}
=
\inv{2}
\int d^3 k
\omega_k
\lr{
  a_{\Bk} a_{\Bk}^\dagger
+ a_{\Bk}^\dagger a_{\Bk}
}.
\end{dmath}

The other diagonal components, for \( r \ne s \ne t \) are
\begin{dmath}\label{eqn:noetherCurrentScalarField:380}
\int d^3 x T^{rr}
=
\int d^3 x
\lr{
\inv{2} \sum_{m = r,0} \Pi^m \Pi^m - \inv{2} \sum_{m = s,t} \Pi^m \Pi^m - \inv{2} \mu^2 \phi^2
}
=
\inv{4}
\int d^3 k \frac{(k^r)^2 - (k^s)^2 - (k^t)^2 - \mu^2}{\omega_k}
\lr{
  a_{\Bk} a_{-\Bk} e^{- 2 i \omega_k t }
+ a_{\Bk} a_{\Bk}^\dagger
+ a_{\Bk}^\dagger a_{\Bk}
+ a_{\Bk}^\dagger a_{-\Bk}^\dagger e^{ 2 i \omega_k t }
}
-\inv{4}
\int d^3 k \omega_k
\lr{
  a_{\Bk} a_{-\Bk} e^{-2 i \omega_k t }
- a_{\Bk} a_{\Bk}^\dagger
- a_{\Bk}^\dagger a_{\Bk}
+ a_{\Bk}^\dagger a_{-\Bk}^\dagger e^{2 i \omega_k t }
}
=
\inv{4}
\int d^3 k \frac{(k^r)^2 - (k^s)^2 - (k^t)^2 - \mu^2 - \omega_k^2}{\omega_k}
\lr{
  a_{\Bk} a_{-\Bk} e^{- 2 i \omega_k t }
+ a_{\Bk}^\dagger a_{-\Bk}^\dagger e^{ 2 i \omega_k t }
}
+
\inv{4}
\int d^3 k \frac{(k^r)^2 - (k^s)^2 - (k^t)^2 - \mu^2 + \omega_k^2}{\omega_k}
\lr{
  a_{\Bk} a_{\Bk}^\dagger
+ a_{\Bk}^\dagger a_{\Bk}
}
=
\inv{2}
\int d^3 k \frac{  (k^r)^2 - \omega_k^2}{\omega_k}
\lr{
  a_{\Bk} a_{-\Bk} e^{- 2 i \omega_k t }
+ a_{\Bk}^\dagger a_{-\Bk}^\dagger e^{ 2 i \omega_k t }
}
+
\inv{2}
\int d^3 k \frac{  (k^r)^2}{\omega_k}
\lr{
  a_{\Bk} a_{\Bk}^\dagger
+ a_{\Bk}^\dagger a_{\Bk}
}.
\end{dmath}

This doesn't have the nice cancellation that killed the time dependent terms in the Hamiltonian.  Such cancellation also doesn't appear in the off diagonal energy-momentum tensor components, which are

\begin{dmath}\label{eqn:noetherCurrentScalarField:400}
\int d^3 x T^{n 0}
=
\int d^3 x T^{n 0}
=
-\inv{2}
\int d^3 k k^n
\lr{
  a_{\Bk} a_{-\Bk} e^{-2 i \omega_k t }
- a_{\Bk} a_{\Bk}^\dagger
- a_{\Bk}^\dagger a_{\Bk}
+ a_{\Bk}^\dagger a_{-\Bk}^\dagger e^{2 i \omega_k t }
},
\end{dmath}

and for \( m \ne n \ne 0 \)
\begin{dmath}\label{eqn:noetherCurrentScalarField:420}
\int d^3 x T^{m n}
=
\inv{2}
\int d^3 k \frac{k^m k^n}{\omega_k}
\lr{
  a_{\Bk} a_{-\Bk} e^{- 2 i \omega_k t }
+ a_{\Bk} a_{\Bk}^\dagger
+ a_{\Bk}^\dagger a_{\Bk}
+ a_{\Bk}^\dagger a_{-\Bk}^\dagger e^{ 2 i \omega_k t }
}.
\end{dmath}

The \cref{eqn:noetherCurrentScalarField:400} result has time dependence that the stated result does not (but is linear in \( \Bk \) as desired)?  Did I miss something?
} % answer

%}
%\EndArticle

   \chapter{1st Noether theorem, spacetime translation current, energy momentum tensor, dilatation current.}
      %
% Copyright � 2017 Peeter Joot.  All Rights Reserved.
% Licenced as described in the file LICENSE under the root directory of this GIT repository.
%
%{
%%%\input{../latex/blogpost.tex}
%%%\renewcommand{\basename}{qftLecture8}
%%%\renewcommand{\dirname}{notes/phy2403/}
%%%\newcommand{\keywords}{PHY2403H}
%%%\input{../latex/peeter_prologue_print2.tex}
%%%
%%%%\usepackage{phy2403}
%%%\usepackage{peeters_braket}
%%%\usepackage{peeters_layout_exercise}
%%%\usepackage{peeters_figures}
%%%\usepackage{mathtools}
%%%\usepackage{siunitx}
%%%\usepackage{macros_cal} % LL
%%%
%%%\beginArtNoToc
%%%\generatetitle{PHY2403H Quantum Field Theory.  Lecture 8: 1st Noether theorem, spacetime translation current, energy momentum tensor, dilatation current.  Taught by Prof.\ Erich Poppitz}
\chapter{1st Noether theorem, spacetime translation current, energy momentum tensor, dilatation current.}
\label{chap:qftLecture8}

%%%\paragraph{DISCLAIMER: Very rough notes from class, with some additional side notes.}
%%%
%%%These are notes for the UofT course PHY2403H, Quantum Field Theory I, taught by Prof. Erich Poppitz fall 2018.
%%%%, covering \textchapref{{1}} \citep{peskin1995introduction} content.
%%%
\section{1st Noether theorem.}

Recall that, given a transformation
\begin{dmath}\label{eqn:qftLecture8:20}
\phi(x) \rightarrow \phi(x) + \delta \phi(x),
\end{dmath}
such that the transformation of the Lagrangian is only changed by a total derivative
\begin{dmath}\label{eqn:qftLecture8:40}
\LL(\phi, \partial_\mu \phi) \rightarrow
\LL(\phi, \partial_\mu \phi)
+ \partial_\mu J_\epsilon^\mu,
\end{dmath}
then there is a conserved current
\begin{dmath}\label{eqn:qftLecture8:60}
j^\mu = \PD{(\partial_\mu \phi)}{\LL} \delta_\epsilon \phi - J_\epsilon^\mu.
\end{dmath}
Here \( \epsilon \) is an x-independent quantity (i.e. a \underline{global symmetry}).
This is in contrast to ``gauge symmetries'', which can be more accurately be categorized as a redundancy in the description.

As an example, for \( \LL = (\partial_\mu \phi \partial^\mu \phi - m^2 \phi^2)/2 \), let
\begin{dmath}\label{eqn:qftLecture8:80}
\phi(x) \rightarrow \phi(x) - a^\mu \partial_\mu \phi
\end{dmath}
\begin{dmath}\label{eqn:qftLecture8:100}
\LL(\phi, \partial_\mu \phi) \rightarrow
\LL(\phi, \partial_\mu \phi)
- a^\mu \partial_\mu \LL
=
\LL(\phi, \partial_\mu \phi)
+ \partial_\mu \lr{ -{\delta^\mu}_\nu a^\nu \LL }
\end{dmath}
Here \( J^\mu_\epsilon = \evalbar{J^\mu_\epsilon}{\epsilon = a^\nu} \), and the current is
\begin{dmath}\label{eqn:qftLecture8:120}
J^\mu = (\partial^\mu \phi)(-a^\nu \partial_\nu \phi) + {\delta^{\mu}}_\nu a^\nu \LL.
\end{dmath}
In particular, we have one such current for each \( \nu \), and we write
\begin{dmath}\label{eqn:qftLecture8:140}
{T^\mu}_\nu =
-(\partial^\mu \phi)(\partial_\nu \phi) + {\delta^{\mu}}_\nu \LL.
\end{dmath}
By Noether's theorem, we must have
\begin{dmath}\label{eqn:qftLecture8:160}
\partial_\mu
{T^\mu}_\nu = 0, \quad \forall \nu.
\end{dmath}

\paragraph{Check:}

\begin{dmath}\label{eqn:qftLecture8:1380}
\partial_\mu {T^\mu}_\nu
=
-(\partial_\mu \partial^\mu \phi)(\partial_\nu \phi)
-(\partial^\mu \phi)(\partial_\mu \partial_\nu \phi)
+ {\delta^{\mu}}_\nu
\partial_\mu \lr{
\inv{2} \partial_\alpha \phi \partial^\alpha \phi - \frac{m^2}{2} \phi^2
}
=
-(\partial_\mu \partial^\mu \phi)(\partial_\nu \phi)
-(\partial^\mu \phi)(\partial_\mu \partial_\nu \phi)
+
\inv{2} (\partial_\nu \partial_\mu \phi) (\partial^\mu \phi )
+
\inv{2} (\partial_\mu \phi) (\partial_\nu \partial^\mu \phi )
- m^2 (\partial_\nu \phi) \phi
=
-\lr{ \partial_\mu \partial^\mu \phi + m^2 \phi }(\partial_\nu \phi)
-(\partial_\mu \phi)(\partial^\mu \partial_\nu \phi)
+
\inv{2} (\partial_\nu \partial^\mu \phi) (\partial_\mu \phi )
+
\inv{2} (\partial_\mu \phi) (\partial_\nu \partial^\mu \phi )
= 0.
\end{dmath}

\paragraph{Example: our potential Lagrangian}

\begin{dmath}\label{eqn:qftLecture8:180}
\LL = \inv{2} \partial^\mu \phi \partial_\nu \phi - \frac{m^2}{2} \phi^2 - \frac{\lambda}{4} \phi^4
\end{dmath}
Written with upper indexes
\begin{dmath}\label{eqn:qftLecture8:200}
T^{\mu\nu}
= -(\partial^\mu \phi)(\partial^\nu \phi) + g^{\mu\nu} \LL
= -(\partial^\mu \phi)(\partial^\nu \phi) + g^{\mu\nu} \lr{
\inv{2} \partial^\alpha \phi \partial_\alpha \phi - \frac{m^2}{2} \phi^2 - \frac{\lambda}{4} \phi^4
}
\end{dmath}

There are 4 conserved currents \( J^{\mu(\nu)} = T^{\mu\nu} \).  Observe that this is symmetric (\( T^{\mu\nu} = T^{\nu\mu} \)).

We have four associated charges
\begin{dmath}\label{eqn:qftLecture8:220}
Q^\nu = \int d^3 x T^{0 \nu}.
\end{dmath}
We call
\begin{dmath}\label{eqn:qftLecture8:240}
Q^0 = \int d^3 x T^{0 0},
\end{dmath}
the energy density, and call
\begin{dmath}\label{eqn:qftLecture8:260}
P^i = \int d^3 x T^{0 i},
\end{dmath}
(i = 1,2,3) the momentum density.

writing this out explicitly the energy density is
\begin{dmath}\label{eqn:qftLecture8:280}
T^{00}
= - \dot{\phi}^2 + \inv{2} \lr{ \dot{\phi}^2 - (\spacegrad \phi)^2 - \frac{m^2}{2}\phi^2 - \frac{\lambda}{4} \phi^4}
= -\lr{
\inv{2} \dot{\phi}^2 + \inv{2} (\spacegrad \phi)^2 + \frac{m^2}{2}\phi^2 + \frac{\lambda}{4} \phi^4
},
\end{dmath}
and
\begin{dmath}\label{eqn:qftLecture8:300}
T^{0i} = \partial^0 \phi \partial^i \phi,
\end{dmath}
\begin{dmath}\label{eqn:qftLecture8:320}
P^{i} = -\int d^3 x\partial^0 \phi \partial^i \phi
\end{dmath}
Since the energy density is negative definite (due to an arbitrary choice of translation sign), let's redefine \( T^{\mu\nu} \) to have a positive sign
\begin{dmath}\label{eqn:qftLecture8:340}
T^{00}
\equiv
\inv{2} \dot{\phi}^2 + \inv{2} (\spacegrad \phi)^2 + \frac{m^2}{2} \phi^2 + \frac{\lambda}{4} \phi^4,
\end{dmath}
and
\begin{dmath}\label{eqn:qftLecture8:360}
P^{i} = \int d^3 x\partial^0 \phi \partial^i \phi
\end{dmath}

As an operator we have
\begin{dmath}\label{eqn:qftLecture8:380}
\hatQ = \int d^3 x \hatT^{00} =
\int d^3 x
\lr{
\inv{2} \pihat^2 + \inv{2} (\spacegrad \phihat)^2 + \frac{m^2}{2} \phihat^2 + \frac{\lambda}{4} \phihat^4
}.
\end{dmath}
\begin{dmath}\label{eqn:qftLecture8:400}
\hatP^{i} = \int d^3 x \pihat \partial^i \phi
\end{dmath}

We showed that
\begin{dmath}\label{eqn:qftLecture8:420}
\ddt \hatO = i \antisymmetric{\hatH}{\hatO}
\end{dmath}
This implied that \( \phihat, \pihat \) obey the classical EOMs
\begin{equation}\label{eqn:qftLecture8:440}
\ddt \phihat = i \antisymmetric{\hatH}{\phihat} = \ddt{\pihat}
\end{equation}
\begin{equation}\label{eqn:qftLecture8:460}
\ddt \pihat = i \antisymmetric{\hatH}{\pihat} = ...
\end{equation}

In terms of creation and annihilation operators (for the \( \lambda = 0 \) free field), up to a constant
\begin{dmath}\label{eqn:qftLecture8:480}
\hatH
= \int d^3 x \hatT^{00}
= \int \frac{d^3 p}{(2 \pi)^3} \omega_\Bp \hata_\Bp^\dagger \hata_\Bp
\end{dmath}
Can show that:
% FIXME: do this:
\begin{dmath}\label{eqn:qftLecture8:500}
\hatP^i
= \int d^3 x \pihat \partial^i \phihat
= \cdots
= \int \frac{d^3 p}{(2 \pi)^3} p^i \hata_\Bp^\dagger \hata_\Bp
\end{dmath}
Now we see the energy and momentum as conserved quantities associated with spacetime translation.

\section{Unitary operators.}

In QM we say that \( \hat{\Bp} \) ``generates translations''.

With \( \hat{\Bp} \equiv -i \Hbar \spacegrad \) that translation is
\begin{equation}\label{eqn:qftLecture8:520}
\hatU = e^{i \Ba \cdot \hat{\Bp}} = e^{\Ba \cdot \spacegrad}
\end{equation}

In particular
\begin{dmath}\label{eqn:qftLecture8:540}
\bra{\Bx} \hatU \ket{\psi} = e^{\Ba \cdot \hat{\Bp} } \psi(\Bx) = \psi(\Bx + \Ba).
\end{dmath}

In one dimension
\begin{dmath}\label{eqn:qftLecture8:560}
\hatU \hat{x} \hatU^\dagger
=
e^{\Ba \cdot \hat{p} } \psi(\Bx)
e^{-\Ba \cdot \hat{p} }
= \hat{\Bx} + a \hat{\BOne}.
\end{dmath}
This uses the Baker-Campbell-Hausdorff formula.
\maketheorem{ Baker-Campbell-Hausdorff }{thm:qftLecture8:580}{
\begin{dmath}\label{eqn:qftLecture8:600}
e^{B} A e^{-B} = \sum_{n = 0}^\infty \inv{n!} \antisymmetric{B \cdots}{\antisymmetric{B}{A}},
\end{dmath}
where the n-th commutator is denoted above

\begin{itemize}
\item \( n = 1 \) : \( \antisymmetric{B}{A} \)
\item \( n = 2 \) : \( \antisymmetric{B}{\antisymmetric{B}{A}} \)
\item \( n = 3 \) : \( \antisymmetric{B}{\antisymmetric{B}{\antisymmetric{B}{A}}} \)
\end{itemize}
} % theorem

Proof:

\begin{dmath}\label{eqn:qftLecture8:620}
f(t) = e^{tB} A e^{-tB}
= f(0) + t f'(0) + \frac{t^2}{2} f''(0) + \cdots \frac{t^n}{n!} f^{(n)}(0)
\end{dmath}

\begin{dmath}\label{eqn:qftLecture8:640}
f(0) = A
\end{dmath}
\begin{dmath}\label{eqn:qftLecture8:660}
f'(t)
=
e^{tB} B A e^{-tB}
+
e^{tB} A (-B) e^{-tB}
=
e^{tB} \antisymmetric{B}{A} e^{-tB}
\end{dmath}
\begin{dmath}\label{eqn:qftLecture8:680}
f''(t) =
e^{tB} B \antisymmetric{B}{A} e^{-tB}
+
e^{tB} \antisymmetric{B}{A} (-B) e^{-tB}
=
e^{tB} \antisymmetric{B}{\antisymmetric{B}{A}} e^{-tB}.
\end{dmath}
From
\begin{dmath}\label{eqn:qftLecture8:700}
f(1)
= f(0) + f'(0) + \inv{2} f''(0) + \cdots \inv{n!} f^{(n)}(0)
\end{dmath}
we have
\begin{dmath}\label{eqn:qftLecture8:720}
e^{B} A e^{-B} = A +
\antisymmetric{B}{A} + \inv{2} \antisymmetric{B}{\antisymmetric{B}{A}} + \cdots
\end{dmath}

Example:
\begin{dmath}\label{eqn:qftLecture8:740}
e^{a \partial_x} x e^{-a \partial_x } = x + a \antisymmetric{\partial_x}{x} + \cdots
= x + a.
\end{dmath}

\paragraph{Application:}

\begin{dmath}\label{eqn:qftLecture8:760}
e^{i \text{Hermitian} } = \text{unitary}
\end{dmath}
\begin{dmath}\label{eqn:qftLecture8:860}
e^{i \text{Hermitian} } \times
e^{-i \text{Hermitian} }
= 1
\end{dmath}
So
\begin{dmath}\label{eqn:qftLecture8:780}
\hatU(\Ba) =
e^{i a^j \hat{p}^j }
\end{dmath}
is a unitary operator representing finite translations in a Hilbert space.

\begin{dmath}\label{eqn:qftLecture8:800}
\hatU(\Ba) \phihat(\Bx) \hatU^\dagger(\Ba)
=
e^{i a^j \hat{p}^j }
\phihat(\Bx)
e^{-i a^k \hat{p}^k }
=
\phihat(\Bx)
+ i a^j \antisymmetric{\hatP^j}{\phihat(\Bx)} + \frac{-a^{j_1} a^{j_2}}{2} \antisymmetric{\hatP^{j_1}}{\antisymmetric{\hatP^{j_2}}{\phihat(\Bx)}}
\end{dmath}

\begin{dmath}\label{eqn:qftLecture8:820}
\antisymmetric{\hatP^j}{\phihat(\Bx)}
=
\int d^3 y \antisymmetric{\pihat(\By) \partial^j \phihat(\By)}{\phihat(\Bx)}
=
\int d^3 y \antisymmetric{\pihat(\By)}{\phihat(\Bx} \partial^j \phihat(\By)
=
\int d^3 y (-i ) \delta^3(\By - \Bx) \partial^j \phihat(\By)
=
-i \partial^j \phihat(\Bx).
\end{dmath}

\begin{dmath}\label{eqn:qftLecture8:840}
\hatU(\Ba) \phihat(\Bx) \hatU^\dagger(\Ba)
= \phihat(\Bx) + i a^j (-i) \partial^j \phihat(\Bx) + \cdots
= \phihat(\Bx) + a^j \partial^j \phihat(\Bx) + \cdots
= \phihat(\Bx + \Ba)
\end{dmath}

\section{Continuous symmetries.}

For all infinitesimal transformations, continuous symmetries lead to conserved charges \( Q \).  In QFT we map these charges to Hermitian operators \( Q \rightarrow \hatQ \).  We say that these charges are ``generators of the corresponding symmetry'' through unitary operators
\begin{dmath}\label{eqn:qftLecture8:880}
\hatU = e^{i \text{parameter} \hatQ}.
\end{dmath}
These represent the action of the symmetry in the Hilbert space.

\paragraph{Example: spatial translation}
\begin{dmath}\label{eqn:qftLecture8:900}
\hatU(\Ba) = e^{i \Ba \cdot \hat{\BP}}
\end{dmath}
\paragraph{Example: time translation}
\begin{dmath}\label{eqn:qftLecture8:920}
\hatU(t) = e^{i t \hat{H}}.
\end{dmath}

\section{Classical scalar theory.}

For \( d > 2 \) let's look at
\begin{dmath}\label{eqn:qftLecture8:940}
S =
\int d^d x \lr{
\inv{2} \partial^\mu \phi \partial_\mu \phi - \frac{m^2}{2} \phi^2 - \lambda \phi^{d-2}
}
\end{dmath}
\paragraph{Take \( m^2, \lambda \rightarrow 0 \), the free massless scalar field.}

We have a shift symmetry in this case since \( \phi(x) \rightarrow \phi(x) + \text{constant} \).
The current is just
\begin{dmath}\label{eqn:qftLecture8:960}
j^\mu
= \PD{(\partial_\mu \phi)}{\phi} \delta \phi - \cancel{J^\mu}
= \text{constant} \times \partial^\mu \phi
= \partial^\mu \phi,
\end{dmath}
where the constant factor has been set to one.
This current is clearly conserved since \( \partial_\mu J^\mu = \partial_\mu \partial^\mu \phi = 0\) (the equation of motion).

These are called ``Goldstone bosons'', or ``Nambu-Goldstone bosons''.

\paragraph{With \( m = \lambda = 0, d = 4 \) we have}

NOTE: We did this in class differently with \( d \ne 4, m, \lambda \ne 0\), and then switched to \( m = \lambda = 0, d = 4\), which was confusing.  I've reworked my notes to \( d = 4 \) like the supplemental handout that did the same.

%%%\begin{dmath}\label{eqn:qftLecture8:980}
%%%S =
%%%\int d^d x \lr{
%%%\inv{2} \partial
%%%\inv{2} \partial^\mu \phi \partial_\nu \phi - \tilde{\lambda} \phi^{d-2}
%%%}
%%%\end{dmath}
%%%Here we have a scale or dilatation invariance
%%%\begin{dmath}\label{eqn:qftLecture8:1000}
%%%x \rightarrow e^{\lambda} x
%%%\end{dmath}
%%%\begin{dmath}\label{eqn:qftLecture8:1020}
%%%\phi(x) \rightarrow \phi'(x') = e^{-(d-2) \lambda/2} \phi
%%%\end{dmath}
%%%
%%%\begin{dmath}\label{eqn:qftLecture8:1040}
%%%d^3 x = e^{d\lambda} d^3 x
%%%\end{dmath}
%%%
%%%\begin{dmath}\label{eqn:qftLecture8:1060}
%%%(\partial_\mu \phi)^2 \rightarrow e^{-(d-2)\lambda} e^{-2 \lambda} \lr{ \partial_\mu \phi(x) }^2
%%%\end{dmath}
%%%
%%%\begin{dmath}\label{eqn:qftLecture8:1080}
%%%\phi^{2 d/(d -2)} \rightarrow \lr{\phi'(x')}^{2d/(d-2)}
%%%=
%%%e^{-(d-2)/2 \times 2d/(d-2) \lambda} \lr{ \phi }^{2d/(d-2)}
%%%\end{dmath}
%%%

\begin{dmath}\label{eqn:qftLecture8:980}
S =
\int d^4 x \lr{
\inv{2} \partial^\mu \phi \partial_\mu \phi
}
\end{dmath}
Here we have a scale or dilatation invariance
\begin{equation}\label{eqn:qftLecture8:1000}
x \rightarrow x' = e^{\lambda} x,
\end{equation}
\begin{equation}\label{eqn:qftLecture8:1020}
\phi(x) \rightarrow \phi'(x') = e^{-\lambda} \phi,
\end{equation}
\begin{equation}\label{eqn:qftLecture8:1040}
d^4 x \rightarrow d^4 x' = e^{4\lambda} d^4 x,
\end{equation}

The partials transform as
\begin{dmath}\label{eqn:qftLecture8:1400}
\partial^\mu \rightarrow
\PD{x'_\mu}{}
=
\PD{x'_\mu}{x_\mu}
\PD{x_\mu}{}
=
e^{-\lambda}
\PD{x_\mu}{}
\end{dmath}
% x'_\mu = e^\lambda x_\mu
% x_\mu = e^-\lambda x'_\mu
so the partial of the field transforms as
\begin{equation}\label{eqn:qftLecture8:1420}
\partial^\mu \phi(x) \rightarrow \PD{x'_\mu}{\phi'(x')} = e^{-2\lambda} \partial^\mu \phi(x),
\end{equation}
and finally
\begin{dmath}\label{eqn:qftLecture8:1060}
(\partial_\mu \phi)^2 \rightarrow e^{-4\lambda} \lr{ \partial_\mu \phi(x) }^2.
\end{dmath}

With a \( -4 \lambda \) power in the transformed quadratic term, and \( 4 \lambda \) in the volume element, we see that the action is invariant.
%%%\begin{dmath}\label{eqn:qftLecture8:1080}
%%%\phi^{4} \rightarrow \lr{\phi'(x')}^{4}
%%%=
%%%e^{-3 \times 4 \lambda} \lr{ \phi }^{4}
%%%\end{dmath}
To find Noether current, we need to vary the field and it's derivatives
\begin{dmath}\label{eqn:qftLecture8:1100}
\delta_\lambda \phi
= \phi'(x) - \phi(x)
= \phi'(e^{-\lambda} x') - \phi(x)
\approx \phi'(x' -\lambda x') - \phi(x)
\approx \phi'(x') - \lambda {x'}^\alpha \partial_\alpha \phi'(x') - \phi(x)
\approx (1 - \lambda) \phi(x) - \lambda {x'}^\alpha \partial_\alpha \phi'(x') - \phi(x)
= - \lambda(1 + x^\alpha \partial_\alpha ) \phi,
\end{dmath}
where the last step assumes that \( x' \rightarrow x, \phi' \rightarrow \phi \), effectively weeding out any terms that are quadratic or higher in \( \lambda \).

%%%%In free field with \( \tilde{\lambda} = 0 \)
%%%\begin{dmath}\label{eqn:qftLecture8:1120}
%%%\delta_\lambda \lr{ \inv{2} \partial_\mu \phi \partial^\mu \phi}
%%%= \partial^\mu \delta_\lambda (\partial_\mu \phi)
%%%= \partial^\mu \phi \partial_\mu ( - \lambda( 1 + x^\nu \partial_\nu )\phi )
%%%= -\lambda \lr{ \partial^\mu \phi \partial_\mu \phi + \partial^\mu \phi \partial_\mu \lr{ x^\nu \partial_\nu \phi } }
%%%= -\lambda \lr{ 2 \partial^\mu \phi \partial_\mu \phi + \partial^\mu \phi x^\nu \partial_\nu \phi }
%%%=
%%%\end{dmath}
%%%
%%%Wrong.  Starting over.
%%%
Now we need the variation of the derivatives of \( \phi \)
\begin{dmath}\label{eqn:qftLecture8:1440}
\delta \partial_\mu \phi(x)
=
\partial_\mu' \phi'(x) - \partial_\mu \phi(x),
\end{dmath}
By \cref{eqn:qftLecture8:1420}
\begin{dmath}\label{eqn:qftLecture8:1460}
\partial_\mu' \phi'(x')
=
e^{-2\lambda} \partial_\mu \phi(x)
=
e^{-2\lambda} \partial_\mu \phi(e^{-\lambda} x')
\approx
e^{-2\lambda} \partial_\mu
\lr{
   \phi(x') - \lambda {x'}^\alpha \partial_\alpha \phi(x')
}
\approx
\lr{
   1 - 2 \lambda
}
\partial_\mu
\lr{
   \phi(x') - \lambda {x'}^\alpha \partial_\alpha \phi(x')
},
\end{dmath}
so
\begin{dmath}\label{eqn:qftLecture8:1480}
\delta \partial_\mu \phi =
- \lambda {x}^\alpha \partial_\alpha \partial_\mu \phi(x)
- 2 \lambda \partial_\mu \phi(x) + O(\lambda^2)
=
- \lambda \lr{
   {x}^\alpha \partial_\alpha + 2
}
\partial_\mu \phi(x).
\end{dmath}

%%%\begin{dmath}\label{eqn:qftLecture8:1140}
%%%\inv{2} \partial^\mu \phi \partial_\mu \phi
%%%\rightarrow
%%%\inv{2} {\partial'}^\mu \phi' {\partial'}_\mu \phi'
%%%=
%%%\inv{2} e^{-4\lambda}
%%%\partial^\mu \phi \partial_\mu \phi
%%%\end{dmath}
%%%Let
%%%\begin{dmath}\label{eqn:qftLecture8:1160}
%%%\begin{aligned}
%%%\Phi'(x')
%%%&=
%%%{\partial'}^\mu \phi' {\partial'}_\mu \phi' \\
%%%\Phi(x) &=
%%%\partial^\mu \phi \partial_\mu \phi,
%%%\end{aligned}
%%%\end{dmath}
%%%
%%%so
%%%\begin{dmath}\label{eqn:qftLecture8:1180}
%%%\begin{aligned}
%%%\Phi'(x') &= e^{-4 \lambda} \Phi(x) \\
%%%\Phi'(e^{\lambda} x) &= e^{-4 \lambda} \Phi(x)
%%%\end{aligned}
%%%\end{dmath}
%%%...
%%%so
%%%
\begin{dmath}\label{eqn:qftLecture8:1200}
\delta \LL
=
(\partial^\mu \phi) \delta (\partial_\mu \phi)
= - \lambda \lr{ 2
\partial_\mu \phi
+ x^\alpha \partial_\alpha
\partial_\mu \phi
}
\partial^\mu \phi,
\end{dmath}
or
\begin{dmath}\label{eqn:qftLecture8:1500}
\frac{\delta \LL }{-\lambda}
=
4 \LL + x^\alpha \lr{ \partial_\alpha \partial_\mu \phi } \partial^\mu \phi
=
4 \LL + x^\alpha \partial_\alpha \lr{ \LL }
=
\cancel{4 \LL} + \partial_\alpha \lr{ x^\alpha \LL } - \cancel{\LL \partial_\alpha x^\alpha}.
\end{dmath}
The variation in the Lagrangian density is thus
\begin{equation}\label{eqn:qftLecture8:1520}
\delta \LL = \partial_\mu J^\mu_\lambda = \partial_\mu \lr{ -\lambda x^\mu \LL },
\end{equation}
and the current is
\begin{equation}\label{eqn:qftLecture8:1540}
J^\mu_\lambda = -\lambda x^\mu \LL.
\end{equation}

The Noether current is
\begin{dmath}\label{eqn:qftLecture8:1240}
j^\mu = \PD{(\partial_\mu \phi)}{\LL} \delta \phi - J^\mu
= -\partial^\mu \phi \lr{ 1 + x^\nu \partial_\nu } \phi + \inv{2} x^\mu \partial_\nu \phi \partial^\nu \phi,
\end{dmath}
or after flipping signs
\begin{dmath}\label{eqn:qftLecture8:1280}
j^\mu_{\text{dil}}
= \partial^\mu \phi \lr{ 1 + x^\nu \partial_\nu } \phi - \inv{2} x^\mu
\partial_\nu \phi \partial^\nu \phi
= x_\nu \lr{ \partial^\mu \phi \partial^\nu \phi - \inv{2} {\delta^{\nu}}_\mu \partial_\lambda \phi \partial^\lambda \phi }
+ \inv{2} \partial^\mu (\phi^2),
\end{dmath}

\begin{dmath}\label{eqn:qftLecture8:1300}
j^\mu_{\text{dil}} = -x_\nu T^{\nu \mu} + \inv{2} \partial^\mu (\phi^2),
\end{dmath}

The current and \( T^{\mu\nu} \) can both be redefined \( j^{\mu'} = j^\mu + \partial_\nu C^{\nu\mu} \) adding an antisymmetric \( C^{\mu\nu} = -C^{\nu\mu} \)

\begin{dmath}\label{eqn:qftLecture8:1320}
j^\mu_{\text{dil conformal}} = - x_\nu T^{\nu\mu}_{\text{conformal}}
\end{dmath}

\begin{dmath}\label{eqn:qftLecture8:1340}
\partial_\mu
j^\mu_{\text{dil conformal}} = - {{T_{\text{conformal}}}^\mu}_\mu
\end{dmath}

consequence: \( 0 = T^{00} - T^{11} - T^{22} - T^{33} \), which is essentially
\begin{equation}\label{eqn:qftLecture8:1360}
0 = \rho - 3 p = 0.
\end{equation}

%}
%\EndNoBibArticle

      \section{Problems.}
         %
% Copyright � 2015 Peeter Joot.  All Rights Reserved.
% Licenced as described in the file LICENSE under the root directory of this GIT repository.
%
\makeproblem{Field Lagrangian with a divergence}{qft:problemSet1:5}{
%\makesubproblem{}{qft:problemSet1:5a}

Show that replacing the Lagrange density \( L = L(\phi_a, \partial_\alpha \phi_a ) \) by

\begin{equation}\label{eqn:qftproblemSet1Problem5:20}
L' = L + \partial_\mu \wedge^\mu(x),
\end{equation}

where \( \wedge^\mu(x), \mu = 0,\cdots,3\), are arbitrary functions of the fields \( \phi_a(x) \), does not alter the equations of motion. Thus, when constructing the most general Lagrange density for a field, we do not have to include terms which are total derivatives. This will simplify life.
} % makeproblem

\makeanswer{qft:problemSet1:5}{

I had to start with re-deriving the field equations for a Lagrange density since I'd forgotten what they were.  If a field \( \phi \) is varied \( \phi' = \phi + \overbar{\phi} \), with the zero variation on the boundaries of the action volume element, we have to first order

\begin{dmath}\label{eqn:qftproblemSet1Problem5:40}
\LL'
= \LL( \phi + \overbar{\phi}, \partial_\alpha ( \phi + \overbar{\phi}) )
=
\LL( \phi )
+
\PD{\overbar{\phi}}{\LL} \overbar{\phi}
+
\PD{\partial_\beta \overbar{\phi}}{\LL} \partial_\beta \overbar{\phi}.
\end{dmath}

The variation of the action is

\begin{dmath}\label{eqn:qftproblemSet1Problem5:60}
\delta S
=
\int dV \lr{
\PD{\overbar{\phi}}{\LL} \overbar{\phi}
+
\PD{\partial_\beta \overbar{\phi}}{\LL} \partial_\beta \overbar{\phi}
}
=
\int dV \lr{
\PD{\overbar{\phi}}{\LL} \overbar{\phi}
+
\partial_\beta \lr{
\overbar{\phi}
\PD{\partial_\beta \overbar{\phi}}{\LL}
}
-
\overbar{\phi}
\partial_\beta \lr{
\PD{\partial_\beta \overbar{\phi}}{\LL}
}
}
=
\int d\Omega
\evalbar{
\overbar{\phi}
\PD{\partial_\beta \overbar{\phi}}{\LL}
}
{
\Delta x^\beta
}
+
\int dV
\overbar{\phi}
\lr{
\PD{\overbar{\phi}}{\LL}
-
\partial_\beta \lr{
\PD{\partial_\beta \overbar{\phi}}{\LL}
}
}
\end{dmath}

The first integral vanishes given the boundary condition assumptions, and the second provides the equations for the field.  Generalizing that to multiple fields \( \phi_a \), and evaluating the derivatives at \( \overbar{\phi}_a = \phi_a \) we have

\boxedEquation{eqn:qftProblemSet1Problem5:80}{
\PD{\phi_a}{\LL}
=
\partial_\beta \lr{
\PD{\partial_\beta \phi_a}{\LL}
}.
}

Now we can tackle the problem.  Consider first just two fields, say \( \phi \) and \( \psi\), and consider

\begin{dmath}\label{eqn:qftProblemSet1Problem5:100}
\partial_\beta
\lr{
\PD{\partial_\beta \phi}{} \partial_\mu \wedge^\mu
}
=
\partial_\beta
\lr{
\PD{\partial_\beta \phi}{}
\lr{
\PD{\phi}{ \wedge^\mu } \PD{x^\mu}{ \phi}
+
\PD{\psi}{ \wedge^\mu } \PD{x^\mu}{ \psi}
}
}
=
\partial_\beta \PD{\phi}{ \wedge^\beta }
=
\PD{\phi}{ \partial_\beta \wedge^\beta }.
\end{dmath}

We see that the divergence \( \partial_\mu \wedge^\mu \) also satisfies the field Euler-Lagrange equations for the field \( \phi \).  This will clearly be the case for multiple fields.  Making that explicit, we can generalize the above slightly

\begin{dmath}\label{eqn:qftProblemSet1Problem5:120}
\partial_\beta
\lr{
\PD{\partial_\beta \phi_a}{} \partial_\mu \wedge^\mu
}
=
\partial_\beta
\lr{
\PD{\partial_\beta \phi_a}{}
\PD{\phi_b}{ \wedge^\mu } \PD{x^\mu}{ \phi_b }
}
=
\partial_\beta \PD{\phi_b}{ \wedge^\mu } \delta_{b a} {\delta^\beta}_\mu
=
\PD{\phi_a}{ \partial_\beta \wedge^\beta }.
\end{dmath}

%\makeSubAnswer{}{qft:problemSet1:5a}
}

   \chapter{Unbroken and spontaneously broken symmetries, Higgs Lagrangian, scale invariance, Lorentz invariance, angular momentum quantization.}
      %
% Copyright � 2018 Peeter Joot.  All Rights Reserved.
% Licenced as described in the file LICENSE under the root directory of this GIT repository.
%
%%%\input{../latex/blogpost.tex}
%%%\renewcommand{\basename}{qftLecture9}
%%%\renewcommand{\dirname}{notes/phy2403/}
%%%\newcommand{\keywords}{PHY2403H}
%%%\input{../latex/peeter_prologue_print2.tex}
%%%
%%%%\usepackage{phy2403}
%%%\usepackage{peeters_braket}
%%%%\usepackage{peeters_layout_exercise}
%%%\usepackage{peeters_figures}
%%%\usepackage{mathtools}
%%%\usepackage{siunitx}
%%%\usepackage{enumerate}
%%%\usepackage{macros_cal} % LL
%%%\usepackage{mmacells}
%%%
%%%%\newcommand{\munu}[0]{\mu\nu}
%%%\newcommand{\ultensor}[3]{{{#1}^{#2}}_{#3}}
%%%%s/\\ultensor{\([^}]\+\)}{\([^}]\+\)}{\([^}]\+\)}/{{\1}^\2}_\3/g
%%%%s/ulLambda/ultensor{\\Lambda}/g
%%%\newcommand{\ulLambda}[2]{\ultensor{\Lambda}{#1}{#2}}
%%%\newcommand{\ulDelta}[2]{\ultensor{\delta}{#1}{#2}}
%%%
%%%\beginArtNoToc
%%%\generatetitle{PHY2403H Quantum Field Theory.  Lecture 9: Unbroken and spontaneously broken symmetries, Higgs Lagrangian, scale invariance, Lorentz invariance, angular momentum quantization.  Taught by Prof.\ Erich Poppitz}
%\chapter{Unbroken and spontaneously broken symmetries, Higgs Lagrangian, scale invariance, Lorentz invariance, angular momentum quantization}
\index{unbroken symmetries}
\index{spontaneous symmetry breaking}
\index{symmetries!unbroken}
\index{symmetries!spontaneously broken}
\index{Higgs Lagrangian}
\index{scale invariance}
\index{Lorentz invariance}
\index{angular momentum quantization}
\label{chap:qftLecture9}

%\paragraph{DISCLAIMER: Very rough notes from class, with some additional side notes.}
%
%These are notes for the UofT course PHY2403H, Quantum Field Theory I, taught by Prof. Erich Poppitz fall 2018.
%%, covering \textchapref{{1}} \citep{peskin1995introduction} content.
%
\section{Last time.}

We followed a sequence of operations
\begin{enumerate}
\item
Noether's theorem
\item \( \rightarrow \)
conserved currents
\item \( \rightarrow \)
charges (classical)
\item \( \rightarrow \)
``correspondence principle''
\item \( \rightarrow \hatQ \)
\end{enumerate}

\begin{itemize}
\item Hermitian operators
\item ``generators of symmetry"
\begin{equation}\label{eqn:qftLecture9:20}
\hatU(\alpha) = e^{i \alpha \hatQ}
\end{equation}
We found
\begin{equation}\label{eqn:qftLecture9:40}
\hatU(\alpha) \phihat \hatU^\dagger(\alpha) = \phihat + i \alpha \antisymmetric{\hatQ}{\phihat} + \cdots
\end{equation}
\end{itemize}

\paragraph{Example: internal symmetries:}
\index{internal symmetries}
(non-spacetime), such as \( O(N)\) or \( U(1) \).

In QFT internal symmetries can have different ``\underline{modes of realization}''.

\begin{enumerate}[I]
\item ``Wigner mode''.  These are also called ``unbroken symmetries''.
\begin{equation}\label{eqn:qftLecture9:60}
\hatQ \ket{0} = 0
\end{equation}
i.e. \( \hatU(\alpha) \ket{0} = 0 \).
Ground state invariant.  Formally \( :\hatQ: \) annihilates \( \ket{0} \).
\( \antisymmetric{\hatQ}{\hatH} = 0 \) implies that all eigenstates are eigenstates of \( \hatQ \) in \( U(1) \).  Example from Hw 1
\begin{equation}\label{eqn:qftLecture9:80}
\hatQ = \text{``charge'' under \( U(1) \)}.
\end{equation}
All states have definite charge, just live in QU.
\item ``Nambu-Goldstone mode'' (Landau-Ginsburg).  This is also called a ``spontaneously broken symmetry''\footnote{
First encounter example (HwII, \( \SU{2} \times \SU{2} \rightarrow \SU{2} \)).  Here a \( U(1) \) spontaneous broken symmetry.}.
\( H \) or \( L \) is invariant under symmetry, but ground state is not.
\end{enumerate}

Example:
\begin{equation}\label{eqn:qftLecture9:100}
\LL = \partial_\mu \phi^\conj \partial^\mu \phi - V(\Abs{\phi}),
\end{equation}
where
\begin{equation}\label{eqn:qftLecture9:120}
V(\Abs{\phi}) = m^2 \phi^\conj \phi + \frac{\lambda}{4} \lr{ \phi^\conj \phi }^2.
\end{equation}
When \( m^2 > 0 \) we have a Wigner mode, but when \( m^2 < 0 \) we have an issue: \( \phi = 0 \) is not a minimum of potential.
When \( m^2 < 0 \) we write
\begin{equation}\label{eqn:qftLecture9:140}
\begin{aligned}
V(\phi)
&= - m^2 \phi^\conj \phi + \frac{\lambda}{4} \lr{ \phi^\conj \phi}^2 \\
&= \frac{\lambda}{4} \lr{
\lr{ \phi^\conj \phi}^2 - \frac{4}{\lambda} m^2 } \\
&= \frac{\lambda}{4} \lr{
\phi^\conj \phi - \frac{2}{\lambda} m^2 }^2 - \frac{4 m^4}{\lambda^2},
\end{aligned}
\end{equation}
or simply
\begin{equation}\label{eqn:qftLecture9:780}
V(\phi) = \frac{\lambda}{4} \lr{ \phi^\conj \phi - v^2 }^2 + \text{const}.
\end{equation}
The potential (called the Mexican hat potential) is illustrated in \cref{fig:mexicanHatPotential:mexicanHatPotentialFig1} for non-zero \( v \), and in
\cref{fig:mexicanHatPotential:mexicanHatPotentialFig2} for \( v = 0 \).
The following is a Mathematica code listing that can be used to play with this shape
\begin{mmaCell}[moredefined={potential, v},morepattern={x_, y_, \
v_},morefunctionlocal={x, y}]{Input}
  ClearAll[potential]
  potential[x_, y_, v_] := (\mmaPat{x}^2 + \mmaPat{y}^2 - v^2)^2

  Manipulate[
  Plot3D[ potential[x, y, v], \{x, -5, 5\}, \{y, -5, 5\}, PlotRange\
\(\pmb{\to}\)Full],
  \{\{v,4\}, 0, 10\}
  ]
\end{mmaCell}
\mathImageFigure{../figures/phy2403-quantum-field-theory/mexicanHatPotentialFig1}{Mexican hat potential.}{fig:mexicanHatPotential:mexicanHatPotentialFig1}{0.3}{mexicanHatPotentialManipulate.nb}
\mathImageFigure{../figures/phy2403-quantum-field-theory/mexicanHatPotentialFig2}{Degenerate Mexican hat potential \( v = 0 \).}{fig:mexicanHatPotential:mexicanHatPotentialFig2}{0.3}{mexicanHatPotentialManipulate.nb}
We choose to expand around some point on the minimum ring (it doesn't matter which one). % (P in the figure).
When there is no potential, we call the field massless (i.e. if we are in the minimum ring).
We expand as
\begin{equation}\label{eqn:qftLecture9:160}
\phi(x) = v \lr{ 1 + \frac{\rho(x)}{v} } e^{i \alpha(x)/v },
\end{equation}
so
\begin{equation}\label{eqn:qftLecture9:180}
\begin{aligned}
\frac{\lambda}{4}
\lr{\phi^\conj \phi - v^2}^2 
&=
\lr{
v^2 \lr{ 1 + \frac{\rho(x)}{v} }^2
- v^2
}^2 \\
&=
\frac{\lambda}{4}
v^4 \lr{ \lr{ 1 + \frac{\rho(x)}{v} }^2 - 1 } \\
&=
\frac{\lambda}{4}
v^4
\lr{
   \frac{2 \rho}{v} + \frac{\rho^2}{v^2}
}^2,
\end{aligned}
\end{equation}
and
\begin{equation}\label{eqn:qftLecture9:200}
\partial_\mu \phi =
\lr{
v \lr{ 1 + \frac{\rho(x)}{v} } \frac{i}{v} \partial_\mu \alpha
+ \partial_\mu \rho
} e^{i \alpha}.
\end{equation}
The Lagrangian takes the form
\begin{equation}\label{eqn:qftLecture9:220}
\begin{aligned}
\LL
&= \Abs{\partial \phi^\conj}^2 - \frac{\lambda}{4} \lr{ \Abs{\phi^\conj}^2 - v^2 }^2 \\
&=
\partial_\mu \rho \partial^\mu \rho + \partial_\mu \alpha \partial^\mu \alpha \lr{ 1 + \frac{\rho}{v} }
-
\frac{\lambda v^4}{4} \frac{ 4\rho^2}{v^2} + O(\rho^3) \\
&=
\partial_\mu \rho \partial^\mu \rho
- \lambda v^2\rho^2
+
\partial_\mu \alpha \partial^\mu \alpha \lr{ 1 + \frac{\rho}{v} }.
\end{aligned}
\end{equation}
We have two fields, \( \rho \) : a massive scalar field, the ``Higgs'', and a massless field \( \alpha \) (the Goldstone boson).

\( U(1) \) symmetry acts on \( \phi(x) \rightarrow e^{i \omega } \phi(x) \) i.t.o \( \alpha(x) \rightarrow \alpha(x) + v \omega \).
\( U(1) \) global symmetry (broken) acts on the Goldstone field \( \alpha(x) \) by a constant shift.  (\(U(1)\) is still a symmetry of the Lagrangian.)

The current of the \( U(1) \) symmetry is:
\begin{equation}\label{eqn:qftLecture9:240}
j_\mu = \partial_\mu \alpha \lr{ 1 + \text{higher dimensional \( \rho \) terms} }.
\end{equation}

When we quantize
\begin{equation}\label{eqn:qftLecture9:260}
\alpha(x) =
\int \frac{d^3p}{(2\pi)^3 \sqrt{ 2 \omega_p }} e^{i \omega_p t - i \Bp \cdot \Bx} \hata_\Bp^\dagger +
\int \frac{d^3p}{(2\pi)^3 \sqrt{ 2 \omega_p }} e^{-i \omega_p t + i \Bp \cdot \Bx} \hata_\Bp
\end{equation}
\begin{equation}\label{eqn:qftLecture9:280}
\begin{aligned}
j^\mu(x)
&= \partial^\mu \alpha(x) \\
&=
\int \frac{d^3p}{(2\pi)^3 \sqrt{ 2 \omega_p }} \lr{ i \omega_\Bp - i \Bp } e^{i \omega_p t - i \Bp \cdot \Bx} \hata_\Bp^\dagger  \\
&\quad +
\int \frac{d^3p}{(2\pi)^3 \sqrt{ 2 \omega_p }} \lr{ -i \omega_\Bp + i \Bp } e^{-i \omega_p t + i \Bp \cdot \Bx} \hata_\Bp.
\end{aligned}
\end{equation}
\begin{equation}\label{eqn:qftLecture9:300}
j^\mu(x) \ket{0} \ne 0,
\end{equation}
instead it creates a single particle state.
\section{Examples of symmetries.}
In particle physics, examples of Wigner vs Nambu-Goldstone, ignoring gravity the only exact internal symmetry in the standard module is
\( (B\# - L\#) \), believed to be a \( U(1) \) symmetry in Wigner mode.

Here \(B\#\) is the Baryon number, and \( L\# \) is the Lepton number.  Examples:
\index{Baryon number}
\index{Lepton number}
\index{proton}
\index{quark}
\index{electron}
\index{neutron}
\begin{itemize}
\item \( B(p) = 1 \), proton.
\item \( B(q) = 1/3 \), quark
\item \( B(e) = 1 \), electron
\item \( B(n) = 1 \), neutron.
\item \( L(p) = 1 \), proton.
\item \( L(q) = 0 \), quark.
\item \( L(e) = 0 \), electron.
\end{itemize}

The major use of global internal symmetries in the standard model is as ``approximate'' ones.  They become symmetries when one neglects some effect( ``terms in \( \LL \)'').
There are other approximate symmetries (use of group theory to find the Balmer series).
\paragraph{Example from \cref{qft:problemSet2:2} (Hw2):}
QCD in limit
\begin{equation}\label{eqn:qftLecture9:320}
m_u = m_d = 0.
\end{equation}
\( m_u m_d \ll m_p \) (the products of the up-quark mass and the down-quark mass are much less than a composite one (name?)).
\( \SU{2}_L \times \SU{2}_R \rightarrow \SU{2}_V \)
\paragraph{EWSB (Electro-Weak-Symmetry-Breaking) sector}
\index{EWSB}
\index{Electro-Weak-Symmetry-Breaking}
When the couplings \( g_2, g_1 = 0 \).  (\( g_2 \in \SU{2}, g_1 \in U(1) \)).
\section{Scale invariance.}
\index{scale invariance}
\begin{equation}\label{eqn:qftLecture9:340}
\begin{aligned}
x &\rightarrow e^{\lambda} x \\
\phi &\rightarrow e^{-\lambda} \phi \\
A_\mu &\rightarrow e^{-\lambda} A_\mu
\end{aligned}.
\end{equation}
\index{unitary}
\index{conformal invariance}
Any unitary theory which is scale invariant is also \underline{conformal} invariant.  Conformal invariance means that angles are preserved.
The point here is that there is more than scale invariance.

We have classical internal global continuous symmetries.
These can be either
\begin{enumerate}
\item
``unbroken'' (Wigner mode)
\begin{equation}\label{eqn:qftLecture9:360}
\hatQ\ket{0} = 0.
\end{equation}
\item
``spontaneously broken''
\begin{equation}\label{eqn:qftLecture9:380}
j^\mu(x) \ket{0} \ne 0
\end{equation}
(creates Goldstone modes).
\item ``anomalous''.  Classical symmetries are not a symmetry of QFT.
\index{anomalous symmetry}
Examples:
\begin{itemize}
\item Scale symmetry (to be studied in QFT II), although this is not truly internal.
\item In QCD again when \( \omega_\Bq = 0 \), a \( U(1\) symmetry (chiral symmetry) becomes exact, and cannot be preserved in QFT.
\item In the standard model (E.W sector), the Baryon number and Lepton numbers are not symmetries, but their difference \( B\# - L\# \) is a symmetry.
\end{itemize}
\end{enumerate}

\section{Lorentz invariance.}
\index{Lorentz invariance}
We'd like to study the action of Lorentz symmetries on quantum states.  We are going to ``go by the book'', finding symmetries, currents, quantize, find generators, and so forth.

Under a Lorentz transformation
\index{Lorentz transformation!infinitesimal}
\begin{equation}\label{eqn:qftLecture9:400}
x^\mu \rightarrow {x'}^\mu = {\Lambda^\mu}_\nu x^\nu,
\end{equation}
We are going to consider infinitesimal Lorentz transformations
\begin{equation}\label{eqn:qftLecture9:420}
{\Lambda^\mu}_\nu \approx
{\delta^\mu}_\nu + {\omega^\mu}_\nu
,
\end{equation}
where \( {\omega^\mu}_\nu \) is small.
A Lorentz transformation \( \Lambda \) must satisfy \( \Lambda^\T G \Lambda = G \), or
\begin{equation}\label{eqn:qftLecture9:800}
g_{\mu\nu} = \ulLambda{\alpha}{\mu} g_{\alpha \beta} \ulLambda{\beta}{\nu},
\end{equation}
into which we insert the infinitesimal transformation representation
\begin{equation}\label{eqn:qftLecture9:820}
\begin{aligned}
0 
&=
- g_{\mu\nu} +
\lr{ {\delta^\alpha}_\mu + {\omega^\alpha}_\mu }
g_{\alpha \beta}
\lr{ {\delta^\beta}_\nu + {\omega^\beta}_\nu } \\
&=
- g_{\mu\nu} +
\lr{
   g_{\mu \beta}
   +
   \omega_{\beta\mu}
}
\lr{ {\delta^\beta}_\nu + {\omega^\beta}_\nu } \\
&=
- g_{\mu\nu} +
   g_{\mu \nu}
   +
   \omega_{\nu\mu}
+
\omega_{\mu\nu}
+
   \omega_{\beta\mu}
{\omega^\beta}_\nu.
\end{aligned}
\end{equation}
The quadratic term can be ignored, leaving just
\begin{equation}\label{eqn:qftLecture9:840}
0 =
   \omega_{\nu\mu}
+
\omega_{\mu\nu},
\end{equation}
or
\begin{equation}\label{eqn:qftLecture9:860}
   \omega_{\nu\mu} = - \omega_{\mu\nu}.
\end{equation}
Note that \( \omega \) is a completely antisymmetric tensor, and like \( F_{\mu\nu} \) this has only 6 elements.
This means that the
infinitesimal transformation of the coordinates is
\begin{equation}\label{eqn:qftLecture9:440}
x^\mu \rightarrow {x'}^\mu \approx x^\mu + \omega^{\mu\nu} x_\nu,
\end{equation}
the field transforms as
\begin{equation}\label{eqn:qftLecture9:460}
\phi(x) \rightarrow \phi'(x') = \phi(x)
\end{equation}
or
\begin{equation}\label{eqn:qftLecture9:760}
\phi'(x^\mu + \omega^{\mu\nu} x_\nu) =
\phi'(x) + \omega^{\mu\nu} x_\nu \partial_\mu\phi(x) = \phi(x),
\end{equation}
so
\begin{equation}\label{eqn:qftLecture9:480}
\delta \phi = \phi'(x) - \phi(x) =
-\omega^{\mu\nu} x_\nu \partial_\mu \phi.
\end{equation}

Since \( \LL \) is a scalar
\begin{equation}\label{eqn:qftLecture9:500}
\begin{aligned}
\delta \LL
&=
-\omega^{\mu\nu} x_\nu \partial_\mu \LL \\
&=
-
\partial_\mu \lr{
   \omega^{\mu\nu} x_\nu \LL
}
+
(\partial_\mu x_\nu) \omega^{\mu\nu} \LL \\
&=
\partial_\mu \lr{
-
   \omega^{\mu\nu} x_\nu \LL
},
\end{aligned}
\end{equation}
since \( \partial_\nu x_\mu = g_{\nu\mu} \) is symmetric, and \( \omega \) is antisymmetric.
Our current is
\begin{equation}\label{eqn:qftLecture9:520}
J^\nu_\omega = - \omega^{\mu\nu} x_\mu \LL .
\end{equation}

Our Noether current is
\begin{equation}\label{eqn:qftLecture9:540}
\begin{aligned}
j^\nu_{\omega^{\mu\rho}}
&= \PD{\phi_{,\nu}}{\LL} \delta \phi - J^\mu_\omega \\
&=
\partial^\nu \phi\lr{ - \omega^{\mu\rho} x_\rho \partial_\mu \phi } + \omega^{\nu \rho} x_\rho \LL \\
&=
\omega^{\mu\rho}
\lr{
   \partial^\nu \phi\lr{ - x_\rho \partial_\mu \phi } + {\delta^{\nu}}_\mu x_\rho \LL
} \\
&=
\omega^{\mu\rho} x_\rho
\lr{
   -\partial^\nu \phi \partial_\mu \phi + {\delta^{\nu}}_\mu \LL
}.
\end{aligned}
\end{equation}
We identify
\begin{equation}\label{eqn:qftLecture9:560}
-
{T^\nu}_\mu =
   -\partial^\nu \phi \partial_\mu \phi + {\delta^{\nu}}_\mu \LL,
\end{equation}
so the current is
\begin{equation}\label{eqn:qftLecture9:580}
\begin{aligned}
j^\nu_{\omega_{\mu\rho}}
&=
-\omega^{\mu\rho} x_\rho
{T^\nu}_\mu \\
&=
-\omega_{\mu\rho} x^\rho
T^{\nu\mu}
.
\end{aligned}
\end{equation}
Define
\begin{equation}\label{eqn:qftLecture9:600}
j^{\nu\mu\rho} = \inv{2} \lr{ x^\rho T^{\nu\mu} - x^{\mu} T^{\nu\rho} },
\end{equation}
which retains the antisymmetry in \( \mu \rho \) yet still drops the parameter \( \omega^{\mu\rho} \).
To check that this makes sense, we can contract
\( j^{\nu\mu\rho} \) with \( \omega_{\rho\mu} \)
\begin{equation}\label{eqn:qftLecture9:880}
\begin{aligned}
j^{\nu\mu\rho} \omega_{\rho\mu}
&= -\inv{2} \lr{ x^\rho T^{\nu\mu} - x^{\mu} T^{\nu\rho} }
\omega_{\mu\rho} \\
&=
-\inv{2} x^\rho T^{\nu\mu}
\omega_{\mu\rho}
- \inv{2} x^{\mu} T^{\nu\rho}
\omega_{\rho\mu} \\
&=
-\inv{2} x^\rho T^{\nu\mu}
\omega_{\mu\rho}
- \inv{2} x^{\rho} T^{\nu\mu}
\omega_{\mu\rho} \\
&=
- x^{\rho} T^{\nu\mu}
\omega_{\mu\rho},
\end{aligned}
\end{equation}
which matches \cref{eqn:qftLecture9:580} as desired.

\paragraph{Example.  Rotations \( \mu\rho = ij \)}
\index{rotations}
\begin{equation}\label{eqn:qftLecture9:620}
\begin{aligned}
J^{0 i j} \epsilon_{ijk}
&=
\inv{2} \lr{ x^i T^{0j} - x^{j} T^{0i} } \epsilon_{ijk} \\
&=
x^i T^{0j} \epsilon_{ijk}.
\end{aligned}
\end{equation}
Observe that this has the structure of \( (\Bx \cross \Bp)_k \), where \( \Bp \) is the momentum density of the field.
Let
\begin{equation}\label{eqn:qftLecture9:640}
L_k \equiv Q_k = \int d^3 x J^{0ij} \epsilon_{ijk}.
\end{equation}
We can now quantize and build a generator
\begin{equation}\label{eqn:qftLecture9:660}
\begin{aligned}
\hatU(\Balpha)
&= e^{i \Balpha \cdot \hat{\BL}} \\
&= \exp\lr{i \alpha_k
\int d^3 x x^i \hat{T}^{0j} \epsilon_{ijk}
}
\end{aligned}
\end{equation}
From \cref{eqn:qftLecture9:560} we can quantize with \( T^{0j} = \partial^0 \phi \partial^j \phi \rightarrow \pihat \lr{\spacegrad \phihat}_j\), or
\begin{equation}\label{eqn:qftLecture9:900}
\begin{aligned}
\hatU(\Balpha)
&=
\exp\lr{i \alpha_k
\int d^3 x x^i \pihat (\spacegrad \phihat)_j \epsilon_{ijk}
} \\
&=
\exp\lr{i \Balpha \cdot
\int d^3 x \pihat \spacegrad \phihat \cross \Bx
}
\end{aligned}
\end{equation}
(up to a sign in the exponent which doesn't matter)
\begin{equation}\label{eqn:qftLecture9:680}
\begin{aligned}
\phihat(\By) &\rightarrow \hatU(\alpha) \phihat(\By) \hatU^\dagger(\alpha) \\
&\approx
\phihat(\By) +
i \Balpha \cdot
\antisymmetric{
   \int d^3 x \pihat(\Bx) \spacegrad \phihat(\Bx) \cross \Bx
}
{
   \phihat(\By)
} \\
&=
\phihat(\By) +
i \Balpha \cdot
   \int d^3 x
(-i) \deltathree(\Bx - \By)
\spacegrad \phihat(\Bx) \cross \Bx \\
&=
\phihat(\By) +
\Balpha \cdot \lr{ \spacegrad \phihat(\By ) \cross \By}.
\end{aligned}
\end{equation}
Explicitly, in coordinates, this is
\begin{equation}\label{eqn:qftLecture9:700}
\begin{aligned}
\phihat(\By) 
&\rightarrow
\phihat(\By) +
\alpha^i
\lr{
   \partial^j \phihat(\By) y^k \epsilon_{jki}
} \\
&=
\phihat(\By) -
\epsilon_{ikj} \alpha^i y^k \partial^j \phihat \\
&=
\phihat( y^j - \epsilon^{ikj} \alpha^i y^k ).
\end{aligned}
\end{equation}
This is a rotation.  To illustrate, pick \( \Balpha = (0, 0, \alpha) \), so \( y^j \rightarrow y^j - \epsilon^{ikj} \alpha y^k \delta_{i3} = y^j - \epsilon^{3kj} \alpha y^k \), or
\begin{equation}\label{eqn:qftLecture9:920}
\begin{aligned}
y^1 &\rightarrow y^1 - \epsilon^{3k1} \alpha y^k = y^1 + \alpha y^2 \\
y^2 &\rightarrow y^2 - \epsilon^{3k2} \alpha y^k = y^2 - \alpha y^1 \\
y^3 &\rightarrow y^3 - \epsilon^{3k3} \alpha y^k = y^3,
\end{aligned}
\end{equation}
or in matrix form
\begin{equation}\label{eqn:qftLecture9:720}
\begin{bmatrix}
y^1 \\
y^2 \\
y^3 \\
\end{bmatrix}
\rightarrow
\begin{bmatrix}
1 & \alpha & 0 \\
-\alpha & 1 & 0 \\
0 & 0 & 1
\end{bmatrix}
\begin{bmatrix}
y^1 \\
y^2 \\
y^3 \\
\end{bmatrix}.
\end{equation}

%\EndNoBibArticle

   \chapter{Lorentz boosts, generator of spacetime translation, Lorentz invariant field representation.}
      %
% Copyright � 2017 Peeter Joot.  All Rights Reserved.
% Licenced as described in the file LICENSE under the root directory of this GIT repository.
%
%{
%%%\input{../latex/blogpost.tex}
%%%\renewcommand{\basename}{qftLecture10}
%%%\renewcommand{\dirname}{notes/phy2403/}
%%%\newcommand{\keywords}{PHY2403H}
%%%\input{../latex/peeter_prologue_print2.tex}
%%%
%%%%\usepackage{phy2403}
%%%\usepackage{peeters_braket}
%%%%\usepackage{peeters_layout_exercise}
%%%\usepackage{peeters_figures}
%%%\usepackage{mathtools}
%%%\usepackage{siunitx}
%%%\usepackage{macros_cal} % LL
%%%
%%%\newcommand{\ultensor}[3]{{{#1}^{#2}}_{#3}}
%%%%\newcommand{\ulLambda}[2]{\ultensor{\Lambda}{#1}{#2}}
%%%%\newcommand{\ulDelta}[2]{\ultensor{\delta}{#1}{#2}}
%%%
%%%\beginArtNoToc
%%%%\generatetitle{PHY2403H Quantum Field Theory.  Lecture 10: Lorentz boosts, generator of spacetime translation, Lorentz invariant field representation.  Taught by Prof.\ Erich Poppitz}
\chapter{Lorentz boosts, generator of spacetime translation, Lorentz invariant field representation.}
\label{chap:qftLecture10}

%%%\paragraph{DISCLAIMER: Very rough notes from class, with some additional side notes.}
%%%
%%%These are notes for the UofT course PHY2403H, Quantum Field Theory I, taught by Prof. Erich Poppitz fall 2018.
%%%%, covering \textchapref{{1}} \citep{peskin1995introduction} content.
%%%
\section{Lorentz transform symmetries.}

From last time, recall that an infinitesimal Lorentz transform has the form
\begin{dmath}\label{eqn:qftLecture10:20}
x^\mu \rightarrow x^\mu + \omega^{\mu\nu} x_\nu,
\end{dmath}
where
\begin{dmath}\label{eqn:qftLecture10:40}
\omega^{\mu\nu} = -\omega^{\nu\mu}
\end{dmath}

We showed last time that \( \omega^{ij} \) induces a rotation, and will show today that \( \omega^{0i} \) is a boost.

We introduced a three index current, factoring out explicit dependence on the incremental Lorentz transform tensor \( \omega^{\mu\nu} \) as follows
\begin{dmath}\label{eqn:qftLecture10:80}
J^{\nu \mu\rho} = \inv{2} \lr{ x^\rho T^{\nu\mu} - x^\mu T^{\nu\rho} },
\end{dmath}
and can easily show that this current has the desired zero four-divergence property
\begin{dmath}\label{eqn:qftLecture10:100}
\partial_\nu J^{\nu \mu\rho}
= \inv{2} \lr{
(\partial_\nu x^\rho) T^{\nu\mu}
+
x^\rho \cancel{\partial_\nu T^{\nu\mu} }
- (\partial_\nu x^\mu) T^{\nu\rho}
- x^\mu \cancel{\partial_\nu T^{\nu\rho} }
}
= \inv{2} \lr{
T^{\rho\mu}
+
- T^{\mu\rho}
}
= 0,
\end{dmath}
since the energy-momentum tensor is symmetric.

Defining charge in the usual fashion \( Q = \int d^3 x j^0 \), so we can define a charge for each pair of indexes \( \mu\nu \), and in particular
\begin{dmath}\label{eqn:qftLecture10:120}
Q^{0k} = \int d^3 x J^{0 0 k} = \inv{2} \int d^3 x \lr{ x^k T^{00} - x^0 T^{0k} }
\end{dmath}
\begin{dmath}\label{eqn:qftLecture10:540}
\dot{Q}^{0k}
= \int d^3 x \dot{J}^{0 0k}
= \inv{2} \int d^3 x \lr{ x^k \dot{T}^{00} - x^0 \dot{T}^{0k} }
\end{dmath}

However, since \( 0 = \partial_\mu T^{\mu \nu} = \dot{T}^{0 \nu} + \partial_j T^{j \nu} \),
or \( \dot{T}^{0 \nu} = -\partial_j T^{j \nu} \),
\begin{dmath}\label{eqn:qftLecture10:560}
\dot{Q}^{0k}
= \inv{2} \int d^3 x \lr{ x^k (-\partial_j T^{j0}) - T^{0k} - x^0 (-\partial_j T^{jk}) }
= \inv{2} \int d^3 x \lr{
   \partial_j (-x^k T^{j0}) + (\partial_j x^k) T^{j0}
   - T^{0k} + x^0 \partial_j T^{jk}
}
= \inv{2} \int d^3 x \lr{
   \partial_j (-x^k T^{j0}) + \cancel{T^{k0}}
   - \cancel{T^{0k}} + x^0 \partial_j T^{jk}
}
= \inv{2} \int d^3 x
   \partial_j \lr{
-x^k T^{j0}
+ x^0 T^{jk}
},
\end{dmath}
which leaves just surface terms, so \( \dot{Q}^{0k} = 0 \).

\paragraph{Quantizing:}

From our previous identification \cref{eqn:qftLecture9:560}, we have
\begin{dmath}\label{eqn:qftLecture10:580}
T^{\nu\mu} = \partial^\nu \phi \partial^\mu \phi - g^{\nu\mu} \LL.
\end{dmath}
In particular
\begin{dmath}\label{eqn:qftLecture10:600}
T^{00}
= \partial^0 \phi \partial^0 \phi - \inv{2} \lr{ \partial_0 \phi \partial^0 \phi + \partial_k \phi \partial^k \phi }
= \inv{2} \partial^0 \phi \partial^0 \phi - \inv{2} (\spacegrad \phi)^2,
\end{dmath}
and
\begin{dmath}\label{eqn:qftLecture10:620}
T^{0k} = \partial^0 \phi \partial^k \phi.
\end{dmath}
We may quantize these energy momentum tensor components as
\begin{dmath}\label{eqn:qftLecture10:640}
\begin{aligned}
\hatT^{00} &= \inv{2} \pihat^2 + \inv{2} (\spacegrad \phihat)^2 \\
\hatT^{0k} &= \inv{2} \pihat \partial^k \phihat.
\end{aligned}
\end{dmath}

We can now start computing the commutators associated with the charge operator.  The first of those commutators is
\begin{dmath}\label{eqn:qftLecture10:140}
\antisymmetric{\hatT^{00}(\Bx)}{\phihat(\By)}
=
\inv{2}
\antisymmetric{\pihat^2(\Bx)}{\phihat(\By)},
\end{dmath}
which can be evaluated using the field commutator analogue of \( \antisymmetric{F(p)}{q} = i F' \) which is
\begin{dmath}\label{eqn:qftLecture10:660}
\antisymmetric{F(\pihat(\Bx))}{\phihat(\By)} = -i \frac{dF}{d \pihat} \delta(\Bx - \By),
\end{dmath}
to give
\begin{dmath}\label{eqn:qftLecture10:680}
\antisymmetric{\hatT^{00}(\Bx)}{\phihat(\By)}
= -i \delta^3(\Bx - \By) \pihat(\Bx)
\end{dmath}

The other required commutator is
\begin{dmath}\label{eqn:qftLecture10:160}
\antisymmetric{\hatT^{0i}(\Bx)}{\phihat(\By)}
=
\antisymmetric{\pihat(\Bx)\partial^i \phihat(\Bx)}{\phihat(\By)}
=
\partial^i \phihat(\Bx)
\antisymmetric{\pihat(\Bx)
}{\phihat(\By)}
= -i \delta^3(\Bx - \By) \partial^i \phihat(\Bx),
\end{dmath}

The charge commutator with the field can now be computed
\begin{dmath}\label{eqn:qftLecture10:180}
i \epsilon \antisymmetric{\hatQ^{0k}}{\phihat(\By)}
=
i
\frac{\epsilon}{2} \int d^3 x
\lr{
x^k
\antisymmetric{\hatT^{00}}{\phihat(\By)}
-
x^0
\antisymmetric{\hatT^{0k}}{\phihat(\By)}
}
=
\frac{\epsilon}{2} \lr{ y^k \pihat(\By) - y^0 \partial^k \phihat(\By) }
=
\frac{\epsilon}{2} \lr{ y^k \dot{\phihat}(\By) - y^0 \partial^k \phihat(\By) },
\end{dmath}
so to first order in \( \epsilon \)
\begin{dmath}\label{eqn:qftLecture10:200}
e^{i \epsilon \hatQ^{0k} } \phihat(\By)
e^{-i \epsilon \hatQ^{0k} }
=
\phihat(\By)
+ \frac{\epsilon}{2} y^k \dot{\phihat}(\By)
+ \frac{\epsilon}{2} y^0 \partial_k \phihat(\By)
\end{dmath}

For example, with \( k = 1 \)
\begin{dmath}\label{eqn:qftLecture10:700}
e^{i \epsilon \hatQ^{0k} } \phihat(\By)
e^{-i \epsilon \hatQ^{0k} }
=
\phihat(\By)
+ \frac{\epsilon}{2} \lr{
   y^1 \dot{\phihat}(\By)
   +
   y^0 \PD{y^1}{\phihat}(\By)
}
=
\phihat(y^0 + \frac{\epsilon}{2} y^1,
y^1 + \frac{\epsilon}{2} y^2, y^3).
\end{dmath}

This is a boost.  If we compare explicitly to an infinitesimal Lorentz transformation of the coordinates
\begin{dmath}\label{eqn:qftLecture10:220}
\begin{aligned}
x^0 \rightarrow x^0 + \omega^{01} x_1 &= x^0 - \omega^{01} x^1 \\
x^1 \rightarrow x^1 + \omega^{10} x_0 &= x^1 - \omega^{01} x_0 = x^1 - \omega^{01} x^0
\end{aligned}
\end{dmath}
we can make the identification
\begin{dmath}\label{eqn:qftLecture10:240}
\frac{\epsilon}{2} = - \omega^{01}.
\end{dmath}

We now have the explicit form of the generator of a spacetime translation
%\begin{dmath}\label{eqn:qftLecture10:260}
\boxedEquation{eqn:qftLecture10:260}{
\hatU(\Lambda) = \exp\lr{-i \omega^{0k} \int d^3 x \lr{ \hatT^{00} x^k - \hatT^{0k} x^0 }}
}
%\end{dmath}

%%where
%%
%%\begin{dmath}\label{eqn:qftLecture10:280}
%%\end{dmath}
%%
An explicit boost along the x-axis has the form
\begin{dmath}\label{eqn:qftLecture10:300}
\hatU(\Lambda) \phihat(t, \Bx)
\hatU^\dagger(\Lambda)
=
\phihat\lr{ \frac{t - vx}{\sqrt{1 - v^2}}, \frac{x - vt}{\sqrt{1 - v^2}}, y, z },
\end{dmath}
and more generally
\begin{dmath}\label{eqn:qftLecture10:320}
\hatU(\Lambda) \phihat(x) \hatU^\dagger(\Lambda) =
\phihat(\Lambda x)
\end{dmath}
where
\( x \) is a four vector,
\( (\Lambda x)^\mu = \ultensor{\Lambda}{\mu}{\nu} x^\nu \), and
\(
\ultensor{\Lambda}{\mu}{\nu}
\approx
\ultensor{\delta}{\mu}{\nu}
+
\ultensor{\omega}{\mu}{\nu} \).

\section{Transformation of momentum states}
In the momentum space representation

\begin{dmath}\label{eqn:qftLecture10:340}
\phihat(x)
=
\int \frac{d^3 p}{(2 \pi)^3 \sqrt{2 \omega_\Bp}}  \lr{
   e^{i (\omega_\Bp t - \Bp \cdot \Bx)} \hata_\Bp
+
   e^{-i (\omega_\Bp t - \Bp \cdot \Bx)} \hata^\dagger_\Bp
}
=
\int \frac{d^3 p}{(2 \pi)^3 \sqrt{2 \omega_\Bp}}  \evalbar{
\lr{
   e^{i p^\mu x^\mu } \hata_\Bp
+
   e^{-i p^\mu x^\mu } \hata^\dagger_\Bp
}
}{p_0 = \omega_\Bp}
\end{dmath}
\begin{dmath}\label{eqn:qftLecture10:720}
\hatU(\Lambda) \phihat(x) \hatU^\dagger(\Lambda)
=
\phihat(\Lambda x)
=
\int \frac{d^3 p}{(2 \pi)^3 \sqrt{2 \omega_\Bp}}  \evalbar{
\lr{
   e^{i p^\mu \ultensor{\Lambda}{\mu}{\nu} x^\nu } 
\hata_\Bp
+
   e^{-i p^\mu \ultensor{\Lambda}{\mu}{\nu} x^\nu } \hata^\dagger_\Bp
}
}{p_0 = \omega_\Bp}
\end{dmath}
This can be put into an explicitly Lorentz invariant form
\begin{dmath}\label{eqn:qftLecture10:n}
\phihat(\Lambda x)
=
\int \frac{dp^0 d^3 p}{(2\pi)^3} \delta(p_0^2 - \Bp^2 - m^2) \Theta(p^0) \sqrt{2 \omega_\Bp} 
   e^{i p^\mu \ultensor{\Lambda}{\mu}{\nu} x^\nu } 
\hata_\Bp + \text{h.c.}
=
\int \frac{dp^0 d^3 p}{(2\pi)^3}
\lr{
\frac{\delta(p_0 - \omega_\Bp)}{2 \omega_\Bp}
+
\frac{\delta(p_0 + \omega_\Bp)}{2 \omega_\Bp}
}
\Theta(p^0) \sqrt{2 \omega_\Bp}  \hata_\Bp + \text{h.c.},
\end{dmath}
which recovers \cref{eqn:qftLecture10:720} by making use of the delta function identity
\( \delta(f(x)) = \sum_{f(x_\conj) = 0} \frac{\delta(x - x_\conj)}{f'(x_\conj)} \), since the
\( \Theta(p^0) \) kills the second delta function.

We now have a more explicit Lorentz invariant structure
\begin{dmath}\label{eqn:qftLecture10:380}
\phihat(\Lambda x)
=
\int \frac{dp^0 d^3 p}{(2\pi)^3} \delta(p_0^2 - \Bp^2 - m^2) \Theta(p^0) \sqrt{2 \omega_\Bp}  
   e^{i p^\mu \ultensor{\Lambda}{\mu}{\nu} x^\nu } 
\hata_\Bp + \text{h.c.}
\end{dmath}

Recall that a boost moves a spacetime point along a parabola, such as that of \cref{fig:constantMomentumSurface:constantMomentumSurfaceFig2}, whereas a rotation moves along a constant ``circular'' trajectory of a hyper-paraboloid.  In general, a Lorentz transformation may move a spacetime point along any path on a hyper-paraboloid such as the one depicted (in two spatial dimensions) in
\cref{fig:constantMomentumSurface:constantMomentumSurfaceFig1}.  This paraboloid depict the surfaces of constant energy-momentum \( p^0 = \sqrt{ \Bp^2 + m^2 } \).  Because a Lorentz transformation only shift points along that energy-momentum surface, but cannot change the sign of the energy coordinate \( p^0 \), this means that \( \Theta(p^0) \) is also a Lorentz invariant.
\imageFigure{../figures/phy2403-quantum-field-theory/constantMomentumSurfaceFig2}{One dimensional spacetime surface for constant \( (p^0)^2 - \Bp^2 = m^2 \)}{fig:constantMomentumSurface:constantMomentumSurfaceFig2}{0.3}
\imageFigure{../figures/phy2403-quantum-field-theory/constantMomentumSurfaceFig1}{Surface of constant squared four-momentum.}{fig:constantMomentumSurface:constantMomentumSurfaceFig1}{0.3}

Let's change variables
\begin{dmath}\label{eqn:qftLecture10:400}
p^\lambda = \ultensor{\Lambda}{\lambda}{\rho} {p'}^{\rho}
\end{dmath}
so that
\begin{dmath}\label{eqn:qftLecture10:420}
p_\mu
\ultensor{\Lambda}{\mu}{\nu} x^\nu
=
\ultensor{\Lambda}{\lambda}{\rho} {p'}^\rho g_{\lambda\nu} \ultensor{\Lambda}{\nu}{\sigma} x^{\sigma}
=
{p'}^\rho 
\lr{ \ultensor{\Lambda}{\lambda}{\rho} 
g_{\lambda\nu} \ultensor{\Lambda}{\nu}{\sigma} } x^{\sigma}
=
{p'}^\rho g_{\rho\sigma} x^\sigma
\end{dmath}
which gives
\begin{dmath}\label{eqn:qftLecture10:440}
\phihat(\Lambda x)
=
\int \frac{d{p'}^0 d^3 p'}{(2\pi)^3} \delta({p'}_0^2 - {\Bp'}^2 - m^2) \Theta(p^0) \sqrt{2 \omega_{\Lambda \Bp'}} e^{i p' \cdot x}  \hata_{\Lambda \Bp'} + \text{h.c.}
=
\int \frac{dp^0 d^3 p}{(2\pi)^3} \delta({p}_0^2 - {\Bp}^2 - m^2) \Theta(p^0) \sqrt{2 \omega_{\Lambda \Bp}} e^{i p \cdot x}  \hata_{\Lambda \Bp} + \text{h.c.}
\end{dmath}
Since
\begin{dmath}\label{eqn:qftLecture10:460}
\phihat(x)
=
\int \frac{dp^0 d^3 p}{(2\pi)^3} \delta({p}_0^2 - {\Bp}^2 - m^2) \Theta(p^0) \sqrt{2 \omega_{\Bp}} e^{i p \cdot x}  \hata_{\Bp} + \text{h.c.}
\end{dmath}
we can now conclude that the creation and annihilation operators transform as
%\begin{dmath}\label{eqn:qftLecture10:480}
\boxedEquation{eqn:qftLecture10:480}{
\sqrt{2 \omega_{\Lambda \Bp}} \hata_{\Lambda \Bp}
=
\hatU(\Lambda)
\sqrt{2 \omega_{ \Bp}} \hata_{ \Bp}
\hatU^\dagger(\Lambda)
}
%\end{dmath}

In particular
\begin{dmath}\label{eqn:qftLecture10:500}
\sqrt{2 \omega_{ \Bp}} \hata^\dagger_{ \Bp} \ket{0} = \ket{\Bp}
\end{dmath}
and noting that \( \hatU(\Lambda) \ket{0}  = \ket{0} \) (i.e. the ground state is Lorentz invariant), we have
\begin{dmath}\label{eqn:qftLecture10:520}
\sqrt{2 \omega_{\Lambda \Bp}} \hata^\dagger_{\Lambda \Bp} \ket{0}
=
\hatU(\Lambda) \sqrt{ 2\omega_\Bp} \hata^\dagger_\Bp \hatU^\dagger(\Lambda) \hatU(\Lambda) \ket{0}
=
\hatU(\Lambda) \sqrt{ 2\omega_\Bp} \hata^\dagger_\Bp \ket{0}
=
\hatU(\Lambda) \ket{\Bp}.
\end{dmath}
%}
%%%\EndNoBibArticle

   \chapter{Microcausality, Lorentz invariant measure, retarded time SHO Green's function.}
      %
% Copyright � 2018 Peeter Joot.  All Rights Reserved.
% Licenced as described in the file LICENSE under the root directory of this GIT repository.
%
%{
%%%\input{../latex/blogpost.tex}
%%%\renewcommand{\basename}{qftLecture11}
%%%\renewcommand{\dirname}{notes/phy2403/}
%%%\newcommand{\keywords}{PHY2403H}
%%%\input{../latex/peeter_prologue_print2.tex}
%%%
%%%%\usepackage{phy2403}
%%%\usepackage{peeters_braket}
%%%\usepackage{peeters_layout_exercise}
%%%\usepackage{peeters_figures}
%%%\usepackage{mathtools}
%%%\usepackage{siunitx}
%%%\usepackage{macros_cal} % LL
%%%
%%%\newcommand{\ultensor}[3]{{{#1}^{#2}}_{#3}}
%%%
%%%\beginArtNoToc
%%%\generatetitle{PHY2403H Quantum Field Theory.  Lecture 11: Momentum matrix elements, spacelike surfaces, microcausality, Lorentz invariant measure, wave function Green's function, retarded time contour, advanced time contour.  Taught by Prof.\ Erich Poppitz}
%\chapter{Momentum matrix elements, spacelike surfaces, microcausality, Lorentz invariant measure, wave function Green's function, retarded time contour, advanced time contour.}
%\chapter{Microcausality, Lorentz invariant measure, retarded time SHO Green's function.}
\index{microcausality}
\index{Lorentz invariant measure}
\index{Green's function!retarded time}
\label{chap:qftLecture11}

%%\paragraph{DISCLAIMER: Very rough notes from class, with some additional side notes.}
%%
%%These are notes for the UofT course PHY2403H, Quantum Field Theory I, taught by Prof. Erich Poppitz fall 2018.
%%%, covering \textchapref{{1}} \citep{peskin1995introduction} content.
%%
\section{Relativistic normalization.}

We will continue looking at the generator of spacetime translation \( \hatU(\Lambda) \), which has the property
\begin{dmath}\label{eqn:qftLecture11:40}
\hatU(\Lambda) \ket{0} = \ket{0},
\end{dmath}
%\underline{Lorentz invariance unbroken}
That is
\begin{dmath}\label{eqn:qftLecture11:760}
\hatU(\Lambda) = \BOne + \text{operators that anhillate the vacuum state}.
\end{dmath}

The action on a field was
\begin{dmath}\label{eqn:qftLecture11:60}
\hatU(\Lambda)
\phihat(x) \hatU^\dagger(\Lambda)
= \phihat(\Lambda x),
\end{dmath}
and the action on the anhillation operator was
\begin{dmath}\label{eqn:qftLecture11:300}
\hatU(\Lambda)
\sqrt{ 2 \omega_\Bp } \hata_\Bp
\hatU^\dagger(\Lambda)
=
\sqrt{ 2 \omega_{\Lambda \Bp} } \hata_{\Lambda \Bp}.
\end{dmath}

If \( \ket{\Bp_1} \) is the one particle state with momentum \( \Bp_1 \), then that momentum state can be generated from the ground state with the following normalized creation operation
\begin{dmath}\label{eqn:qftLecture11:780}
\ket{\Bp_1} = \sqrt{ 2 \omega_{\Bp_1} } \hata_{\Bp_1}^\dagger \ket{0}.
\end{dmath}

We can compute the matrix element between two matrix states using the creation operator representation
\begin{dmath}\label{eqn:qftLecture11:80}
\begin{aligned}
\braket{\Bp_2}{\Bp_1}
&=
\sqrt{ 2 \omega_{\Bp_1} }
\sqrt{ 2 \omega_{\Bp_2} }
\bra{0}
\hata_{\Bp_2}
\hata_{\Bp_1}^\dagger
\ket{0} \\
&=
\sqrt{ 2 \omega_{\Bp_1} }
\sqrt{ 2 \omega_{\Bp_2} }
\bra{0}
\lr{
   \hata_{\Bp_1}^\dagger
   \hata_{\Bp_2}
   +
   i (2 \pi)^3 \deltathree(\Bp - \Bq)
} \\
&=
\sqrt{ 2 \omega_{\Bp_1} }
\sqrt{ 2 \omega_{\Bp_2} }
(2 \pi)^3 \deltathree(\Bp_1 - \Bp_2) \\
&=
2 \omega_{\Bp_1}
(2 \pi)^3 \deltathree(\Bp_1 - \Bp_2).
\end{aligned}
\end{dmath}

\section{Spacelike surfaces.}
\index{spacelike surface}

If \( x^\mu, p^\mu \) are four vectors, then \( p^\mu x_\mu = \text{invariant} = {p'}^\mu x'_\mu \).  The light cone is the surface \( p_0^2 = \Bp^2  \), whereas timelike four-momentum form a parabaloid surface \( p_0^2 - \Bp^2 = m^2 \) (i.e. \( E = \sqrt{ m^2 c^4 + \Bp^2 c^2 } \)).
The surface for constant spacelike points (i.e. all related by a Lorentz transformation) is illustrated in \cref{fig:spaceLikeAndLightCone:spaceLikeAndLightConeFig1}.  A boost moves a point up or down that surface along the energy axis.  It is therefore possible to use a sequence of boost and rotation to transform a point \( (E, \Bp) \rightarrow (-E, \Bp) \rightarrow (-E, -\Bp) \).  That is, any spacelike four-vector \( x \) may be transformed to \( -x \) using a Lorentz transformation.
\imageFigure{../figures/phy2403-quantum-field-theory/spaceLikeAndLightConeFig1}{Constant spacelike surface.}{fig:spaceLikeAndLightCone:spaceLikeAndLightConeFig1}{0.3}

\section{Condition on microcausality.}

We defined operators \( \phihat(\Bx) \), which was a Hermitian operator for the real scalar field.  For the complex scalar field we used \( \phihat(\Bx) = (\phihat_1 + \phihat_2)/\sqrt{2} \), where each of \( \phihat_1, \phihat_2 \) were Hermitian operators.  i.e. we can think of these operators as ``observables'', that is \( \phihat(\Bx) = \phihat^\dagger(\Bx) \).

We now want to show that these operators commute at spacelike separations, and see how this relates to the question of causality.  In particular, we want to see that an observation of one operator, will not effect the measurement of the other.

The condition of microcausality is
\begin{equation*}
\antisymmetric{\phihat(x)}{\phihat(y)} = 0
\end{equation*}
if \( x \sim y \), that is \( (x - y)^2 < 0 \).  That is, \( x, y \) are spacelike separated.

We wrote

\begin{dmath}\label{eqn:qftLecture11:160}
\phihat(x)
=
\int \frac{d^3 p}{(2 \pi)^3 \sqrt{2 \omega_\Bp}}
\evalbar{
e^{-i p \cdot x} }{p^0 = \omega_\Bp} \hata_\Bp
+
\int \frac{d^3 p}{(2 \pi)^3 \sqrt{2 \omega_\Bp}}
\evalbar{
e^{i p \cdot x} }{p^0 = \omega_\Bp} \hata^\dagger_\Bp
,
\end{dmath}
or \( \phihat(x) = \phihat_{-}(x) + \phihat_{+}(x) \), where
\begin{dmath}\label{eqn:qftLecture11:180}
\begin{aligned}
\phihat_{-}(x) &=
\int \frac{d^3 p}{(2 \pi)^3 \sqrt{2 \omega_\Bp}}
\evalbar{
e^{-i p \cdot x} }{p^0 = \omega_\Bp} \hata_\Bp \\
\phihat_{+}(x) &=
\int \frac{d^3 p}{(2 \pi)^3 \sqrt{2 \omega_\Bp}}
\evalbar{
e^{i p \cdot x} }{p^0 = \omega_\Bp} \hata^\dagger_\Bp.
\end{aligned}
\end{dmath}

Compute the commutator
\begin{dmath}\label{eqn:qftLecture11:200}
D(x)
= \antisymmetric{\phihat_{-}(x)}{\phihat_{+}(0)}
=
\int \frac{d^3 p}{(2 \pi)^3 \sqrt{2 \omega_\Bp}}
\evalbar{ e^{-i p \cdot x} }{p^0 = \omega_\Bp}
\int \frac{d^3 k}{(2 \pi)^3 \sqrt{2 \omega_\Bk}}
\evalbar{ e^{i k \cdot 0} }{k^0 = \omega_\Bk}
\antisymmetric{\hata_\Bp }{\hata_\Bk^\dagger }
=
\int \frac{d^3 p}{(2 \pi)^3 \sqrt{2 \omega_\Bp}}
\evalbar{ e^{-i p \cdot x} }{p^0 = \omega_\Bp}
\int \frac{d^3 k}{(2 \pi)^3 \sqrt{2 \omega_\Bk}}
(2 \pi)^3 \deltathree(\Bp - \Bk),
\end{dmath}
\boxedEquation{eqn:qftLecture11:800}{
D(x)
=
\int \frac{d^3 p}{(2 \pi)^3 2 \omega_\Bp}
\evalbar{ e^{-i p \cdot x} }{p^0 = \omega_\Bp}.
}

Now about the commutator at two spacetime points
\begin{dmath}\label{eqn:qftLecture11:220}
\antisymmetric{\phihat(x)}{\phihat(y)}
=
\antisymmetric{\phihat_{-}(x) + \phihat_{+}(x)}{\phihat_{-}(y) + \phihat_{+}(y)}
=
\antisymmetric{\phihat_{-}(x)}{\phihat_{+}(y)}
+
\antisymmetric{\phihat_{+}(x)}{\phihat_{-}(y)}
=
-D(y - x) + D(x - y).
\end{dmath}

Find
\begin{dmath}\label{eqn:qftLecture11:240}
\begin{aligned}
\antisymmetric{\phihat(x)}{\phihat(y)} &= D(x - y) - D(y - x) \\
\antisymmetric{\phihat(x)}{\phihat(0)} &= D(x) - D(- x).
\end{aligned}
\end{dmath}

Let's look at \( D(x) \), \cref{eqn:qftLecture11:800}, a bit more closely.

\paragraph{Claim:}
\( D(x) \) is Lorentz invariant (has the same value for all \( x^\mu, {x'}^\mu \)

We can see this by writing this out as
\begin{dmath}\label{eqn:qftLecture11:280}
D(x)
=
\int \frac{d^3 p}{(2 \pi)^3 } dp^0
\delta( p_0^2 - \Bp^2 - m^2) \Theta(p^0)
e^{-i p \cdot x}.
\end{dmath}

The exponential is Lorentz invariant, and the delta function has been put into a Lorentz invariant form.

\paragraph{Claim 1:}
\( D(x) = D(x') \) where \( x^2 = {x'}^2 \).

\paragraph{Claim 2:}
\( x^\mu, -x^\mu \) are related by Lorentz transformations if \( x^2 < 0 \).

From the figure, we see that \( D(x) = D(-x) \) for a spacelike point, which implies that
\(
\antisymmetric{\phihat(x)}{\phihat(0)} = 0 \) for a spacelike point \( x \).

We've shown this for free fields, but later we will see that this is the case for interacting fields too.


   \chapter{Klein-Gordon Green's function, Feynman propagator path deformation, Wightman function, Retarded Green's function.}
      %
% Copyright � 2017 Peeter Joot.  All Rights Reserved.
% Licenced as described in the file LICENSE under the root directory of this GIT repository.
%
%{
\input{../latex/blogpost.tex}
\renewcommand{\basename}{qft12}
\renewcommand{\dirname}{notes/phy2403/}
\newcommand{\keywords}{PHY2403H}
\input{../latex/peeter_prologue_print2.tex}

%\usepackage{phy2403}
\usepackage{peeters_braket}
%\usepackage{peeters_layout_exercise}
\usepackage{peeters_figures}
\usepackage{mathtools}
\usepackage{siunitx}
\usepackage{macros_cal} % LL

\newcommand{\ultensor}[3]{{{#1}^{#2}}_{#3}}

\beginArtNoToc
\generatetitle{PHY2403H Quantum Field Theory.  Lecture 12: XXX.  Taught by Prof.\ Erich Poppitz}
%\chapter{XXX}
\label{chap:qft12}

\paragraph{Disclaimer}

%%Peeter's lecture notes from class.  These may be incoherent and rough.
%%
%%These are notes for the UofT course PHY2403H, Quantum Field Theory, taught by Prof. Erich Poppitz, covering \textchapref{{1}} \citep{peskin1995introduction} content.

\paragraph{DISCLAIMER: Very rough notes from class, with some additional side notes.}

These are notes for the UofT course PHY2403H, Quantum Field Theory, taught by Prof. Erich Poppitz, fall 2018.
%, covering \textchapref{{1}} \citep{peskin1995introduction} content.

\section{YYY}

%}
\EndArticle
%\EndNoBibArticle

   \chapter{Forced Klein-Gordon equation.}
      %
% Copyright � 2017 Peeter Joot.  All Rights Reserved.
% Licenced as described in the file LICENSE under the root directory of this GIT repository.
%
%{
%%%\input{../latex/blogpost.tex}
%%%\renewcommand{\basename}{qftLecture13}
%%%\renewcommand{\dirname}{notes/phy2403/}
%%%\newcommand{\keywords}{PHY2403H}
%%%\input{../latex/peeter_prologue_print2.tex}
%%%
%%%%\usepackage{phy2403}
%%%\usepackage{peeters_braket}
%%%\usepackage{peeters_layout_exercise}
%%%\usepackage{peeters_figures}
%%%\usepackage{mathtools}
%%%\usepackage{siunitx}
%%%\usepackage{macros_cal} % LL
%%%
%%%\newcommand{\ultensor}[3]{{{#1}^{#2}}_{#3}}
%%%
%%%\beginArtNoToc
%%%\generatetitle{PHY2403H Quantum Field Theory.  Lecture 13: Forced Klein-Gordon equation, coherent states, number density, time ordered product, pole shifting, perturbation theory, Heisenberg picture, interaction picture, Dyson's formula.  Taught by Prof.\ Erich Poppitz}
%\chapter{Forced Klein-Gordon equation, coherent states, number density, time ordered product, perturbation theory, Heisenberg picture, interaction picture, Dyson's formula}
\label{chap:qftLecture13}
\index{Klein-Gordon!forced}
\index{number density}
\index{perturbation theory}
\index{Heisenberg picture}
\index{Dyson's formula}

%%Peeter's lecture notes from class.  These may be incoherent and rough.
%%
%%These are notes for the UofT course PHY2403H, Quantum Field Theory, taught by Prof. Erich Poppitz, covering \textchapref{{1}} \citep{peskin1995introduction} content.

%%\paragraph{DISCLAIMER: Very rough notes from class, with some additional side notes.}
%%
%%These are notes for the UofT course PHY2403H, Quantum Field Theory, taught by Prof. Erich Poppitz, fall 2018.
%%%, covering \textchapref{{1}} \citep{peskin1995introduction} content.
%%
\section{Review: ``particle creation problem''.}

We imagined that we have a windowed source function \( j(y^0, \By) \), as sketched in \cref{fig:windowedImpulse:windowedImpulseFig5}, which is acting as a forcing source for the non-homogeneous Klein-Gordon equation
%\imageFigure{../figures/phy2403-quantum-field-theory/windowedImpulseFig5}{Finite window impulse response.}{fig:windowedImpulse:windowedImpulseFig5}{0.2}
\begin{dmath}\label{eqn:qftLecture13:20}
\lr{ \partial_\mu \partial^\mu + m^2 } \phi = j.
\end{dmath}
Our solution was
\begin{dmath}\label{eqn:qftLecture13:40}
\phi(x) = \phi(x_0) + i \int d^4 y D_R( x - y) j(y),
\end{dmath}
where \( \phi(x_0) \) obeys the homogeneous equation, and
\begin{dmath}\label{eqn:qftLecture13:60}
D_R(x - y) = \Theta(x^0 - y^0) \lr{ D(x - y) - D(y - x) },
\end{dmath}
and \( D(x) = \int \frac{d^3 p}{(2\pi)^3 2 \omega_\Bp } \evalbar{ e^{-i p \cdot x} }{p^0 = \omega_\Bp} \) is the Wightman function.

For \( x^0 > t_{\text{after}} \)
\begin{dmath}\label{eqn:qftLecture13:80}
\phi(x)
=
\int \frac{d^3 p}{(2\pi)^3 \sqrt{ 2 \omega_\Bp }}
\evalbar{
   \lr{ e^{-i p \cdot x} a_\Bp + e^{i p \cdot x } a_\Bp^\dagger }
}{
   p^0 = \omega_\Bp
}
+ i
\int \frac{d^3 p}{(2\pi)^3 2 \omega_\Bp }
\evalbar{
   \lr{ e^{-i p \cdot x} \tilde{j}(p) + e^{i p \cdot x} \tilde{j}(p_0, -\Bp) }
}{
   p^0 = \omega_\Bp
}
\end{dmath}
where we have used \( \tilde{j}^\conj(p_0, \Bp) = \tilde{j}(p_0, -\Bp) \).  This gives
\begin{dmath}\label{eqn:qftLecture13:100}
\phi(x) =
\int \frac{d^3 p}{(2\pi)^3 \sqrt{ 2 \omega_\Bp } }
\evalbar{
   \lr{
      e^{-i p \cdot x}
      \lr{ a_\Bp + i \frac{\tilde{j}(p)}{\sqrt{2 \omega_\Bp}} }
      + e^{i p \cdot x }
      \lr{ a_\Bp^\dagger - i \frac{\tilde{j}^\conj(p)}{\sqrt{2 \omega_\Bp}} }
   }
}{
p^0 = \omega_\Bp
}
\end{dmath}

It was left as an exercise to show that given
\begin{dmath}\label{eqn:qftLecture13:120}
H = \int d^3 p \lr{ \inv{2} \pi^2 + \inv{2} \lr{ \spacegrad \phi}^2 + \frac{m^2}{2} \phi^2 },
\end{dmath}
we obtain
\begin{dmath}\label{eqn:qftLecture13:140}
H_{\text{after}} =
\int d^3 p \omega_\Bp
\lr{ a_\Bp^\dagger - i \frac{\tilde{j}^\conj(p)}{\sqrt{2 \omega_\Bp}} }
\lr{ a_\Bp + i \frac{\tilde{j}(p)}{\sqrt{2 \omega_\Bp}} }.
\end{dmath}

System in ground state
\begin{equation}\label{eqn:qftLecture13:160}
\bra{0} \hatH_{\text{before}} \ket{0} = \expectation{E}_{\text{before}} = 0.
\end{equation}
\begin{dmath}\label{eqn:qftLecture13:180}
\bra{0} \hatH_{\text{after}} \ket{0} = \expectation{E}_{\text{after}}
=
\int d^3 p \omega_\Bp
\frac{ \tilde{j}^\conj(p) \tilde{j}(p)}{2 \omega_\Bp}
=
\inv{2} \int d^3 p
\Abs{\tilde{j}(p)}^2.
\end{dmath}
We can identify
\begin{dmath}\label{eqn:qftLecture13:200}
N(\Bp) =
\frac{\Abs{\tilde{j}(p)}^2}{2 \omega_\Bp},
\end{dmath}
as the number density of particles with momentum \( \Bp \).


   \chapter{Coherent states, Number density.}
      %
% Copyright © 2018 Peeter Joot.  All Rights Reserved.
% Licenced as described in the file LICENSE under the root directory of this GIT repository.
%
\section{Digression: coherent states.}
\index{coherent state}
\makedefinition{Coherent state.}{dfn:qftLecture13:220}{
A coherent state is an eigenstate of the destruction operator
\begin{equation*}
a \ket{\alpha} = \alpha \ket{\alpha}.
\end{equation*}
} % definition
For the SHO, if we solve for such a coherent state, we find
\begin{equation}\label{eqn:qftLecture13:240}
\ket{\alpha} = \text{constant} \times \sum_{n = 0}^\infty \frac{\alpha^n}{n!} \lr{ a^\dagger }^n \ket{0}.
\end{equation}
If we assume the existence of a coherent state
\begin{equation}\label{eqn:qftLecture13:260}
a_\Bp \ket{
\frac{\tilde{j}(p)}{\sqrt{2 \omega_\Bp}}
}
=
\frac{\tilde{j}(p)}{\sqrt{2 \omega_\Bp}}
\ket{
\frac{\tilde{j}(p)}{\sqrt{2 \omega_\Bp}}
},
\end{equation}
then the expectation value of the number operator with respect to this state is the number density identified in \cref{eqn:qftLecture13:200}
\begin{equation}\label{eqn:qftLecture13:1200}
\bra{
\frac{\tilde{j}(p)}{\sqrt{2 \omega_\Bp}}
}
a_\Bp^\dagger a_\Bp
\ket{
\frac{\tilde{j}(p)}{\sqrt{2 \omega_\Bp}}
} = \frac{\Abs{\tilde{j}(p)}^2}{2 \omega_\Bp} = N(\Bp).
\end{equation}

      \section{Problems.}
         %
% Copyright � 2015 Peeter Joot.  All Rights Reserved.
% Licenced as described in the file LICENSE under the root directory of this GIT repository.
%
\makeoproblem{Coherent states.}
{qft:LukeProblemSet1:2}
{2015 ps1.2}
{

In a theory of a single harmonic oscillator, define the coherent state \( \ket{ z } \) by

\begin{dmath}\label{eqn:LukeProblemSet1Problem2:20}
\ket{z} = N e^{z a^\dagger} \ket{0}
\end{dmath}

where \( z \) is a complex number and \( N \) is a real positive constant, chosen such that \( \braket{ z}{z} = 1\).
Coherent
states of the SHO are interesting because they smoothly interpolate between the classical and quantum
worlds: for large z they become indistinguishable from classical oscillators. (Similarly, coherent states
of photons correspond to electromagnetic waves in the limit of large numbers of photons). They also
give you good practice at manipulating creation and annihilation operators.
As usual, \( H = \omega (p^2 +q^2 )/2 \) and the raising and lowering operators \( a \) and \( a^\dagger \)are defined as \( a = (q + i p)/\sqrt{2} \), \( a^\dagger = (q - i p)/\sqrt{2}\),
where the usual momentum \( P \) and position \( X \) are \( P = \sqrt{ \mu \omega } p \), \( X = q/\sqrt{\mu \omega} \).
\makesubproblem{}{qft:LukeProblemSet1:2a}
Find N.
\makesubproblem{}{qft:LukeProblemSet1:2b}
Compute \( \braket{z'}{z} \), and \( \bra{ z } H \ket{z} \).
\makesubproblem{}{qft:LukeProblemSet1:2c}
Show that \( \ket{z} \) is an eigenstate of the annihilation operator \( a \) and find its eigenvalue. (Don't be disturbed by finding non-orthogonal eigenstates with complex eigenvalues; \( a \) is not a Hermitian
operator.)
\makesubproblem{}{qft:LukeProblemSet1:2d}
The statement that \( \ket{ z } \) is an eigenstate of a with well-known eigenvalue is, in the \(q\)-representation, a first-order differential equation for \( \braket{q}{z} \), the position-space wave-function of \( \ket{z} \). Solve this equation and find and sketch the wave-function. (Don't bother with normalization factors here).
\makesubproblem{}{qft:LukeProblemSet1:2e}
Consider the time evolution of the system (work in the Heisenberg representation). Show that for
real \( z \) (this just sets the initial conditions) the expectation values of the position and momentum
of the coherent state satisfy

\begin{dmath}\label{eqn:LukeProblemSet1Problem2:40}
\bra{z} X \ket{z} = \sqrt{\frac{2}{\mu\omega}} z \cos\omega t
\end{dmath}
\begin{dmath}\label{eqn:LukeProblemSet1Problem2:60}
\bra{z} P \ket{z} = -\sqrt{2 \mu\omega} z \sin\omega t
\end{dmath}

By contrast, what are the expectation values of \( X \) and \( P \) for an oscillator in any state of definite excitation number \( n \)? Using a sketch, describe the behavior of the wavepacket as a function of
time.
} % makeproblem

\makeanswer{qft:LukeProblemSet1:2}{
\withproblemsetsParagraph{
\makeSubAnswer{}{qft:LukeProblemSet1:2a}

Expanding this definition of \( \ket{z} \) in power series

\begin{dmath}\label{eqn:qftProblemSet1Problem2:80}
\ket{z} = N \sum_{k= 0}^\infty \inv{k!} \lr{ z a^\dagger }^k \ket{k}
\end{dmath}

but

\begin{equation}\label{eqn:qftProblemSet1Problem2:100}
\begin{aligned}
a^\dagger \ket{0} &= \sqrt{1} \ket{1} \\
(a^\dagger)^2 \ket{0} &= \sqrt{2 \times 1} \ket{2} \\
(a^\dagger)^3 \ket{0} &= \sqrt{3 \times 2 \times 1} \ket{3},
\end{aligned}
\end{equation}

or
\begin{equation}\label{eqn:qftProblemSet1Problem2:120}
(a^\dagger)^k = \sqrt{k!} \ket{k}.
\end{equation}

This gives
\begin{equation}\label{eqn:qftProblemSet1Problem2:140}
\ket{z} = N^2 \sum_{k= 0}^\infty \inv{\sqrt{k!}} z^k \ket{k},
\end{equation}
%
from which the braket can be computed
\begin{dmath}\label{eqn:qftProblemSet1Problem2:160}
1
= \braket{z}{z}
= N^2 \sum_{k,m = 0}^\infty \inv{\sqrt{k!}} (z^\conj)^k \bra{k} \inv{\sqrt{m!}} z^m \ket{m}
= N^2 \sum_{k = 0}^\infty \inv{k!} (z^\conj z)^k
= N^2 e^{\Abs{z}^2}.
\end{dmath}

This gives
\boxedEquation{eqn:qftProblemSet1Problem2:180}{
N = e^{-\Abs{z}^2/2}.
}

\makeSubAnswer{}{qft:LukeProblemSet1:2c}

\begin{dmath}\label{eqn:qftProblemSet1Problem2:260}
a \ket{z}
=
a N
\sum_{k = 0}^\infty \inv{\sqrt{k!}} z^k \ket{k}
=
N
\sum_{k = 1}^\infty \inv{\sqrt{k!}} z^k a \ket{k}
=
N
\sum_{k = 1}^\infty \inv{\sqrt{k!}} z^k \sqrt{k} \ket{k-1}
=
z N
\sum_{k = 1}^\infty \inv{\sqrt{(k-1)!}} z^{k-1} \ket{k-1}
=
z \ket{z}
\end{dmath}

\makeSubAnswer{}{qft:LukeProblemSet1:2b}

\begin{dmath}\label{eqn:qftProblemSet1Problem2:200}
\braket{z}{z'}
=
e^{-\Abs{z}^2/2 - \Abs{z'}^2/2 }
\sum_{k,m = 0}^\infty \inv{\sqrt{k!}} (z^\conj)^k \bra{k} \inv{\sqrt{m!}} {z'}^m \ket{m}
=
\exp\lr{
-\Abs{z}^2/2 - \Abs{z'}^2/2 + z^\conj z
}.
\end{dmath}

We also want to put the Hamiltonian into its number operator form by factoring it

\begin{equation}\label{eqn:qftProblemSet1Problem2:220}
\begin{aligned}
H
&= \omega\lr{ p^2 + q^2 }/2 \\
&= \omega\lr{ \inv{2} \lr{ q - i p } \lr{ q + i p } - i \antisymmetric{q}{p}/2 } \\
&= \omega\lr{ a^\dagger a + \inv{2} }.
\end{aligned}
\end{equation}
Having found that \( a \ket{z} = z \ket{z} \), we also have
\begin{dmath}\label{eqn:qftProblemSet1Problem2:280}
\bra{z} a^\dagger
=
\lr{ a \ket{z} }^\dagger
=
\lr{ z \ket{z} }^\dagger
=
\bra{z} z^\conj,
\end{dmath}
so
\begin{dmath}\label{eqn:qftProblemSet1Problem2:240}
\bra{z} H \ket{z}
=
\omega
\bra{z} a^\dagger a + \inv{2} \ket{z}
=
\omega
\lr{ \Abs{z}^2 + \inv{2} }.
\end{dmath}

\makeSubAnswer{}{qft:LukeProblemSet1:2d}

\begin{dmath}\label{eqn:qftProblemSet1Problem2:300}
\bra{q} a \ket{z}
=
\inv{\sqrt{2}} \bra{q} q + i p \ket{z}
=
\inv{\sqrt{2}} \lr{ q + \PD{q}{} } \braket{q}{z},
\end{dmath}
with \( \psi(q) = \braket{q}{z} \), this is
\begin{dmath}\label{eqn:qftProblemSet1Problem2:320}
\lr{ z - \frac{q}{\sqrt{2}} } \psi = \inv{\sqrt{2}} \PD{q}{\psi},
\end{dmath}
which separates into
\begin{dmath}\label{eqn:qftProblemSet1Problem2:340}
\lr{ \sqrt{2} z - q} dq = \frac{d\psi}{\psi}.
\end{dmath}

The solution is of the form
\begin{dmath}\label{eqn:qftProblemSet1Problem2:360}
\psi
\propto \exp\lr{ \sqrt{2} z q - q^2/2 }
= \exp\lr{ -(q^2 - 2 \sqrt{2} z q )/2 }
\propto \exp\lr{ -(q - \sqrt{2} z )^2/2 }.
\end{dmath}

SKETCH: This is a Gaussian, and when \( z \) is real is centred at \( \sqrt{2} z \).

\makeSubAnswer{}{qft:LukeProblemSet1:2e}

Noting that \( 2 q = \sqrt{2} \lr{ a + a^\dagger } \), and \( 2 i p = \sqrt{2}\lr{ a - a^\dagger } \)
\begin{dmath}\label{eqn:qftProblemSet1Problem2:380}
\bra{z} X \ket{z}
=
\inv{\sqrt{\mu \omega}}
\bra{z} q \ket{z}
=
\inv{\sqrt{2 \mu \omega}}
\bra{z}
a e^{-i\omega t} + a^\dagger e^{i \omega t}
\ket{z}
=
\inv{\sqrt{2 \mu \omega}} \lr{
z e^{-i\omega t} + z^\conj e^{i \omega t}
}
=
\sqrt{\frac{2}{\mu \omega}} z \cos(\omega t).
\end{dmath}

\begin{dmath}\label{eqn:qftProblemSet1Problem2:400}
\bra{z} P \ket{z}
=
\sqrt{\mu \omega}
\bra{z} p \ket{z}
=
i\frac{\sqrt{\mu \omega}}{2}
\bra{z}
a^\dagger e^{i\omega t} - a e^{-i \omega t}
\ket{z}
=
i\frac{\sqrt{\mu \omega}}{2} z
\lr{ e^{i\omega t} - e^{-i \omega t}}
=
-\sqrt{2 \mu \omega} z \sin(\omega t).
\end{dmath}

SKETCH: particle expectation values trace an ellipse in phase space.
}
}

   \chapter{Time ordered product, perturbation theory, Heisenberg picture, interaction picture, Dyson's formula.}
      %
% Copyright © 2018 Peeter Joot.  All Rights Reserved.
% Licenced as described in the file LICENSE under the root directory of this GIT repository.
%
%{
\section{Feynman's Green's function.}
\index{Feynman propagator}
\begin{equation}\label{eqn:qftLecture13:280}
\begin{aligned}
D_F(x)
&=
\Theta(x^0) D(x) +
\Theta(-x^0) D(-x)
\\&=
\Theta(x^0) \bra{0} \phi(x) \phi(0) \ket{0}
+\Theta(x^0) \bra{0} \phi(-x) \phi(0) \ket{0}.
\end{aligned}
\end{equation}
Utilizing a translation operation \( U(a) = e^{i a_\mu P^\mu } \), where \( U(a) \phi(y) U^\dagger(a) = \phi(y + a) \), this second operation can be written as
\begin{equation}\label{eqn:qftLecture13:300}
\begin{aligned}
\bra{0} \phi(-x) \phi(0) \ket{0}
&=
\bra{0} U^\dagger(a) U(a) \phi(-x) U^\dagger(a) U(a) \phi(0) U^\dagger(a) U(a) \ket{0}
\\&=
\bra{0} U(a) \phi(-x) U^\dagger(a) U(a) \phi(0) U^\dagger(a) \ket{0}
\\&=
\bra{0} \phi(-x + a) \phi(a) \ket{0},
\end{aligned}
\end{equation}
In particular, with \( a = x \)
\begin{equation}\label{eqn:qftLecture13:320}
\bra{0} \phi(-x) \phi(0) \ket{0} = \bra{0} \phi(0) \phi(x) \ket{0},
\end{equation}
so the Feynman's Green function can be written
\begin{equation}\label{eqn:qftLecture13:340}
\begin{aligned}
D_F(x)
&=
\Theta(x^0) \bra{0} \phi(x) \phi(0) \ket{0}
+\Theta(x^0) \bra{0} \phi(x) \phi(x) \ket{0}
\\&=
\bra{0}
\lr{
\Theta(x^0)
\phi(x) \phi(0)
+
\Theta(-x^0)
\phi(0) \phi(x)
}
\ket{0}.
\end{aligned}
\end{equation}
We define
\index{time ordered product}
\makedefinition{Time ordered product.}{dfn:qftLecture13:360}{
The time ordered product of two operators is defined as
\begin{equation*}
T(\phi(x) \phi(y)) =
\left\{
\begin{array}{l l}
\phi(x)\phi(y) & \quad \mbox{\( x^0 > y^0 \)} \\
\phi(y)\phi(x) & \quad \mbox{\( x^0 < y^0 \)} \\
\end{array}
\right.,
\end{equation*}
or
\begin{equation*}
T(\phi(x) \phi(y)) =
\phi(x)\phi(y) \Theta(x^0 - y^0)
+
\phi(y)\phi(x) \Theta(y^0 - x^0).
\end{equation*}
} % definition

Using this helpful construct, the Feynman's Green function can now be written in a very simple fashion
\boxedEquation{eqn:qftLecture13:380}{
D_F(x) = \bra{0} T(\phi(x) \phi(0)) \ket{0}.
}

\section{Interacting field theory: perturbation theory in QFT.}
\index{perturbation theory}

We perturb the Hamiltonian
\begin{equation}\label{eqn:qftLecture13:500}
H = H_0 + H_{\text{int}},
\end{equation}
where \( H_0 \) is the free Hamiltonian and \( H_{\text{int}} \) is the interaction term (the perturbation).

\paragraph{Example:}
\begin{equation}\label{eqn:qftLecture13:460}
\begin{aligned}
H_0 &= SHO = \frac{p^2}{2} + \frac{\omega^2 q^2}{2} \\
H_{\text{int}} &= \lambda q^4.
\end{aligned}
\end{equation}
i.e.  the anharmonic oscillator.

In QFT
\begin{equation}\label{eqn:qftLecture13:480}
\begin{aligned}
H_0 &=
\int d^3 x \lr{ \inv{2} \pi^2 + \inv{2} \lr{ \spacegrad \phi}^2 + \frac{m^2}{2} \phi^2 } \\
H_{\text{int}} &=
\lambda \int d^3 x \phi^4.
\end{aligned}
\end{equation}

We will expand the interaction in small \( \lambda \).  Perturbation theory is the expansion in a small dimensionless coupling constant, such as
\begin{itemize}
\item \( \lambda \) in \( \lambda \phi^4 \) theory,
\item \( \alpha = e^2/4 \pi \sim \inv{137} \) in QED, and
\item \( \alpha_s \) in QCD.
\end{itemize}

\section{Interaction picture, Dyson formula.}
%\section{Perturbation theory, interaction representation and Dyson formula.}
\index{interaction representation}
\index{Dyson formula}
\begin{equation}\label{eqn:qftLecture13:520}
H = H_0 + H_{\text{int}}
\end{equation}
Example interaction
\begin{equation}\label{eqn:qftLecture13:540}
H_{\text{int}} = \lambda \int d^3 x \phi^4.
\end{equation}

We know all there is to know about \( H_0 \) (decoupled SHOs, ...)
\begin{equation}\label{eqn:qftLecture13:560}
H_0 \ket{0} = \ket{0} E^0_{\text{vac}}
\end{equation}
where \( E^0_{\text{vac}} = 0 \).  Assume
\begin{equation}\label{eqn:qftLecture13:580}
\lr{ H_0 + H_{\text{int}} } \ket{\Omega} = \ket{\Omega} E_{\text{vac}},
\end{equation}
where the ground state energy of the perturbed system is zero when \( \lambda = 0 \).  That is \( E_{\text{vac}}(\lambda = 0 ) = 0 \).

So for
\begin{equation}\label{eqn:qftLecture13:600}
\evalbar{\phi(x) }{x^0 = t_0, \text{some fixed value}}
=
\int \frac{d^3}{(2 \pi)^3 \sqrt{ 2 \omega_\Bp } }
\evalbar{
   \lr{
   e^{-i p \cdot x} a_\Bp
   + e^{i p \cdot x} a_\Bp^\dagger }
   }
{
p^0 = \omega_\Bp
}.
\end{equation}
Let's call \( \phi(\Bx, t_0) \) the free Schr\"{o}dinger operator, where
\( \phi(\Bx, t_0) \) is evaluated at a fixed value of \( t_0 \).  At such a point, the Schr\"{o}dinger and Heisenberg pictures coincide.
\begin{equation}\label{eqn:qftLecture13:620}
\antisymmetric{\phi(\Bx, t_0)}{\pi(\By, t_0)} = i \deltathree(\Bx - \By).
\end{equation}

Normally (QM) one defines the Heisenberg operator as
\begin{equation}\label{eqn:qftLecture13:640}
O_H = e^{i H(t - t_0)} O_S e^{-i H(t - t_0)},
\end{equation}
where \( O_H \) depends on time, and \( O_S \) is defined at a fixed time \( t_0 \), usually 0.
From \cref{eqn:qftLecture13:640} we find
\begin{equation}\label{eqn:qftLecture13:660}
\ddt{O_H} = i \antisymmetric{H}{O_H}.
\end{equation}
The equivalent of \cref{eqn:qftLecture13:640} in QFT is very complicated.  We'd like to develop an intermediate picture.

We will define an intermediate picture, called the ``interaction representation'', which is equivalent to the Heisenberg picture with respect to \( H_0 \).
\index{interaction picture}
\makedefinition{Interaction picture operator.}{dfn:qftLecture13:680}{
\begin{equation*}
\phi_I(t, \Bx) =
e^{i H_0(t - t_0) }
\phi(t_0, \Bx)
e^{-i H_0(t - t_0) }.
\end{equation*}
} % definition

This is familiar, and is the Heisenberg picture operator that we had in free QFT
\begin{equation}\label{eqn:qftLecture13:700}
\phi_I(t, \Bx) =
\int \frac{d^3}{(2 \pi)^3 \sqrt{ 2 \omega_\Bp } }
\evalbar{
   \lr{
   e^{-i p \cdot x} a_\Bp
   + e^{i p \cdot x} a_\Bp^\dagger }
   }
{
p^0 = \omega_\Bp
},
\end{equation}
where \( x_0 = t \).

The Heisenberg picture operator is
\begin{equation}\label{eqn:qftLecture13:720}
\begin{aligned}
\phi_H(t, \Bx)
&=
\phi(t, \Bx)
\\&=
e^{i H(t - t_0) }
e^{-i H_0(t - t_0) }
\lr{
   e^{i H_0(t - t_0) }
   \phi_S(t_0, \Bx)
   e^{-i H_0(t - t_0) }
}
e^{i H_0(t - t_0) }
e^{-i H(t - t_0) }
\\&=
e^{i H(t - t_0) }
e^{-i H_0(t - t_0) }
\phi_I(t, \Bx)
e^{-i H_0(t - t_0) }
e^{i H(t - t_0) }
\end{aligned}
\end{equation}
or
\begin{equation}\label{eqn:qftLecture13:760}
\phi_H(t, \Bx) = U^\dagger(t, t_0) \phi_I(t_0, \Bx) U(t, t_0),
\end{equation}
where
\begin{equation}\label{eqn:qftLecture13:740}
U(t, t_0) = e^{i H_0(t - t_0) } e^{-i H(t - t_0) }.
\end{equation}

We want to apply perturbation techniques to find \( U(t, t_0) \) which is complicated.
\begin{equation}\label{eqn:qftLecture13:780}
\begin{aligned}
i \PD{t}{} U(t, t_0)
&=
i e^{i H_0(t - t_0) } i H_0
e^{-i H(t - t_0) }
+
i e^{i H_0(t - t_0) }
e^{-i H(t - t_0) } (-i H)
\\&=
e^{i H_0(t - t_0) }
\lr{ -H_0 + H }
e^{-i H(t - t_0) }
\\&=
e^{i H_0(t - t_0) }
H_{\text{int}}
e^{-i H_0(t - t_0) }
e^{i H_0(t - t_0) }
e^{-i H(t - t_0) },
\end{aligned}
\end{equation}
so we have
\boxedEquation{eqn:qftLecture13:800}{
i \PD{t}{} U(t, t_0)
=
H_{\text{int}, I}(t) U(t, t_0).
}
For the (Schr\"{o}dinger) interaction \( H_{\text{int}} = \
\lambda \int d^3 x \phi^4(\Bx, t_0)  \), what we really mean by
\( H_{\text{int}, I}(t) \) is
\begin{equation}\label{eqn:qftLecture13:820}
H_{\text{int}, I}(t) = \lambda \int d^3 x \phi_I^4(\Bx, t).
\end{equation}

It will be more convenient to remove the explicit \( \lambda \) factor from the interaction Hamiltonian, and write instead
\begin{equation}\label{eqn:qftLecture13:880}
H_{\text{int}, I}(t) = \int d^3 x \phi_I^4(\Bx, t),
\end{equation}
so the equation to solve is
\begin{equation}\label{eqn:qftLecture13:1220}
i \PD{t}{} U(t, t_0)
=
\lambda H_{\text{int}, I}(t) U(t, t_0).
\end{equation}

We assume that
\begin{equation}\label{eqn:qftLecture13:900}
U(t, t_0)
=
U_0(t, t_0)
+ \lambda U_1(t, t_0)
+ \lambda^2 U_2(t, t_0)
+ \cdots
+ \lambda^n U_n(t, t_0).
\end{equation}

Plugging into \cref{eqn:qftLecture13:880} we have
\begin{equation}\label{eqn:qftLecture13:1160}
\begin{aligned}
  i &\lambda^0 \PD{t}{}U_0(t, t_0)
+ i \lambda^1 \PD{t}{}U_1(t, t_0)
+ i \lambda^2 \PD{t}{}U_2(t, t_0)
+ \cdots
+ i \lambda^n \PD{t}{}U_n(t, t_0) \\
&=
\lambda H_{\text{int}, I}(t)
\lr{
1
+ \lambda U_1(t, t_0)
+ \lambda^2 U_2(t, t_0)
+ \cdots
+ \lambda^n U_n(t, t_0)
},
\end{aligned},
\end{equation}
so
equating equal powers of \( \lambda \) on each side gives a recurrence relation for each \( U_k, k > 0 \)
\begin{equation}\label{eqn:qftLecture13:1180}
\PD{t}{}U_k(t, t_0) = -i H_{\text{int}, I}(t) U_{k-1}(t, t_0).
\end{equation}

Let's consider each power in turn.
\paragraph{\(O(\lambda^0)\):}

Solving \cref{eqn:qftLecture13:800} to \( O(\lambda^0) \) gives
\begin{equation}\label{eqn:qftLecture13:840}
i \PD{t}{} U_0(t, t_0) = 0,
\end{equation}
or
\begin{equation}\label{eqn:qftLecture13:860}
U(t, t_0) = 1 + O(\lambda).
\end{equation}

\paragraph{\(O(\lambda^1)\):}
%%\begin{equation}\label{eqn:qftLecture13:920}
%%\lambda i \PD{t}{U_1(t, t_0)} = \lambda
%%H_{\text{int}, I}(t) U_0(t, t_0)
%%\end{equation}
%%or
\begin{equation}\label{eqn:qftLecture13:940}
\PD{t}{U_1(t, t_0)} = -i H_{\text{int}, I}(t),
\end{equation}
which has solution
\begin{equation}\label{eqn:qftLecture13:960}
U_1(t, t_0) = -i \int_{t_0}^t H_{\text{int}, I}(t') dt'.
\end{equation}

\paragraph{\(O(\lambda^2)\):}
%%\begin{equation}\label{eqn:qftLecture13:980}
%%\lambda^2 i \PD{t}{U_2(t, t_0)} = \lambda^2
%%H_{\text{int}, I}(t) U_1(t, t_0),
%%\end{equation}
%%or
\begin{equation}\label{eqn:qftLecture13:1000}
\begin{aligned}
\PD{t}{U_2(t, t_0)}
  &= -i H_{\text{int}, I}(t) U_1(t, t_0)
\\&= (-i)^2 H_{\text{int}, I}(t)
\int_{t_0}^t H_{\text{int}, I}(t') dt',
\end{aligned}
\end{equation}
which has solution
\begin{equation}\label{eqn:qftLecture13:1020}
\begin{aligned}
U_2(t, t_0)
&= (-i )^2
\int_{t_0}^t H_{\text{int}, I}(t'') dt''
\int_{t_0}^{t''} H_{\text{int}, I}(t') dt'
\\&= (-i )^2
\int_{t_0}^t dt''
\int_{t_0}^{t''}
dt'
H_{\text{int}, I}(t'')
H_{\text{int}, I}(t').
\end{aligned}
\end{equation}
\paragraph{\(O(\lambda^3)\):}
\begin{equation}\label{eqn:qftLecture13:1060}
\PD{t}{U_3(t, t_0)} = -i H_{\text{int}, I}(t) U_2(t, t_0),
\end{equation}
so
\begin{equation}\label{eqn:qftLecture13:1240}
\begin{aligned}
U_3(t, t_0)
&=
-i
\int_{t_0}^t dt'''
H_{\text{int}, I}(t''') U_2(t''', t_0)
\\&=
(-i )^3
\int_{t_0}^t dt'''
H_{\text{int}, I}(t''')
\int_{t_0}^{t'''} dt''
\int_{t_0}^{t''}
dt'
H_{\text{int}, I}(t'')
H_{\text{int}, I}(t')
\\&=
(-i)^3
\int_{t_0}^t dt'''
\int_{t_0}^{t'''} dt''
\int_{t_0}^{t''} dt'
H_{\text{int}, I}(t''')
H_{\text{int}, I}(t'')
H_{\text{int}, I}(t').
\end{aligned}
\end{equation}

\paragraph{Simplifying the integration region.}

For the two fold integral, the integration range is the upper triangular region sketched in \cref{fig:upperTriangleIntegrationRegion:upperTriangleIntegrationRegionFig2}.
\imageFigure{../figures/phy2403-quantum-field-theory/upperTriangleIntegrationRegionFig2}{Upper triangular integration region.}{fig:upperTriangleIntegrationRegion:upperTriangleIntegrationRegionFig2}{0.2}

\paragraph{Claim:} We can integrate over the entire square, and divide by two, provided we keep the time ordering
\begin{equation}\label{eqn:qftLecture13:1040}
U_2(t, t_0)
= \frac{(-i )^2}{2}
\int_{t_0}^t dt''
\int_{t_0}^{t''}
dt'
T(H_{\text{int}, I}(t'') H_{\text{int}, I}(t') ).
\end{equation}

Demonstration:
\begin{equation}\label{eqn:qftLecture13:1100}
\begin{aligned}
\frac{(-i)^2}{2}
&\int_{t_0}^t dt''
\int_{t_0}^t dt'
T( H_I(t'') H_I(t') ) \\
&=
\frac{(-i)^2}{2}
\int_{t_0}^t dt''
\int_{t_0}^t dt'
\Theta(t''- t')
H_I(t'') H_I(t') \\
&\qquad +
\frac{(-i)^2}{2}
\int_{t_0}^t dt''
\int_{t_0}^t dt'
\Theta(t'- t'')
H_I(t') H_I(t''),
\end{aligned}
\end{equation}
but the \( \Theta(t'' - t') \) function is non-zero only for \( t'' - t' > 0 \), or \( t' < t'' \), and the
\( \Theta(t' - t'') \) function is non-zero only for \( t' - t'' > 0 \), or \( t'' < t' \), so we can adjust the integration ranges for
\begin{equation}\label{eqn:qftLecture13:1260}
\begin{aligned}
&\frac{(-i)^2}{2}
\int_{t_0}^t dt''
\int_{t_0}^t dt'
T( H_I(t'') H_I(t') ) \\
&=
\frac{(-i)^2}{2}
\int_{t_0}^t dt''
\int_{t_0}^{t''} dt'
H_I(t'') H_I(t')
+
\frac{(-i)^2}{2}
\int_{t_0}^{t'} dt''
\int_{t_0}^t dt'
H_I(t') H_I(t'') \\
&=
\frac{(-i)^2}{2}
\int_{t_0}^t dt''
\int_{t_0}^{t''} dt'
H_I(t'') H_I(t')
+
\frac{(-i)^2}{2}
\int_{t_0}^t dt''
\int_{t_0}^{t''} dt'
H_I(t'') H_I(t') \\
&=
U_2(t, t_0),
\end{aligned}
\end{equation}
where we swapped integration variables in second integral.  We can clearly do the same thing for the higher order repeated integrals, but instead of a \(1/2 = 1/2!\) adjustment for the number of orderings, we will require a \( 1/n! \) adjustment for an \( n \)-fold integral.

\paragraph{Summary:}
\begin{equation}\label{eqn:qftLecture13:1120}
\begin{aligned}
U_0 &= 1 \\
U_1 &= -i \int_{t_0}^t dt_1 H_I(t_1) \\
U_2 &= \frac{(-i)^2}{2}
\int_{t_0}^t dt_1
\int_{t_0}^t dt_2
T( H_I(t_1)
H_I(t_2) ) \\
U_3 &= \frac{(-i)^3}{3!}
\int_{t_0}^t dt_1
\int_{t_0}^t dt_2
\int_{t_0}^t dt_3
T( H_I(t_1)
H_I(t_2)
H_I(t_3)
) \\
U_n &= \frac{(-i)^n}{n!}
\int_{t_0}^t dt_1
\int_{t_0}^t dt_2
\int_{t_0}^t dt_3
\cdots
\int_{t_0}^t dt_n
T( H_I(t_1)
H_I(t_2)
\cdots
H_I(t_n)
) \\
\end{aligned}
\end{equation}

Summing we find
\begin{equation}\label{eqn:qftLecture13:1140}
\begin{aligned}
U(t, t_0) 
&= T \exp\lr{-i
\int_{t_0}^t dt_1 H_I(t')
}
\\&=
\sum_{n = 0}^\infty
\frac{(-i)^n}{n!} \int_{t_0}^t dt_1 \cdots dt_n T( H_I(t_1) \cdots H_I(t_n) ).
\end{aligned}
\end{equation}
This is called Dyson's formula.
\section{Next time.}
Our goal is to compute: \( \bra{\Omega} T(\phi(x_1) \cdots \phi(x_n)) \ket{\Omega} \).
%}

      %
% Copyright � 2018 Peeter Joot.  All Rights Reserved.
% Licenced as described in the file LICENSE under the root directory of this GIT repository.
%
%{
%%%\input{../latex/blogpost.tex}
%%%\renewcommand{\basename}{nonhomoKGhamiltonian}
%%%%\renewcommand{\dirname}{notes/phy1520/}
%%%\renewcommand{\dirname}{notes/ece1228-electromagnetic-theory/}
%%%%\newcommand{\dateintitle}{}
%%%%\newcommand{\keywords}{}
%%%
%%%\input{../latex/peeter_prologue_print2.tex}
%%%
%%%\usepackage{peeters_layout_exercise}
%%%\usepackage{peeters_braket}
%%%\usepackage{peeters_figures}
%%%\usepackage{siunitx}
%%%\usepackage{verbatim}
%%%%\usepackage{mhchem} % \ce{}
%%%%\usepackage{macros_bm} % \bcM
%%%%\usepackage{macros_qed} % \qedmarker
%%%%\usepackage{txfonts} % \ointclockwise
%%%
%%%\beginArtNoToc
%%%
%%%\generatetitle{Hamiltonian for the non-homogeneous Klein-Gordon equation}
%\chapter{Hamiltonian for the non-homogeneous Klein-Gordon equation}
%\label{chap:nonhomoKGhamiltonian}

\makeproblem{Hamiltonian with forcing term.}{problem:qftLecture13:1280}{
Prove \cref{eqn:qftLecture13:140}.
} % problem

\makeanswer{problem:qftLecture13:1280}{
In class we derived the field for the non-homogeneous Klein-Gordon equation
\begin{dmath}\label{eqn:nonhomoKGhamiltonian:20}
\phi(x)
= \int \frac{d^3 p}{(2\pi)^3} \inv{\sqrt{2 \omega_\Bp}}
\evalbar{
\lr{
   e^{-i p \cdot x} \lr{ a_\Bp + \frac{ i \tilde{j}(p) }{\sqrt{2 \omega_\Bp}} }
   +
   e^{i p \cdot x} \lr{ a_\Bp^\dagger - \frac{ i \tilde{j}^\conj(p) }{\sqrt{2 \omega_\Bp}} }
}
}
{
p^0 = \omega_\Bp
}
= \int \frac{d^3 p}{(2\pi)^3} \inv{\sqrt{2 \omega_\Bp}}
\lr{
   e^{-i \omega_\Bp t + i \Bp \cdot \Bx} \lr{ a_\Bp + \frac{ i \tilde{j}(p) }{\sqrt{2 \omega_\Bp}} }
   +
   e^{i \omega_\Bp t - i \Bp \cdot \Bx} \lr{ a_\Bp^\dagger - \frac{ i \tilde{j}^\conj(p) }{\sqrt{2 \omega_\Bp}} }
}.
\end{dmath}
This means that we have
\begin{equation}\label{eqn:nonhomoKGhamiltonian:40}
\begin{aligned}
\pi = \dot{\phi}
&= \int \frac{d^3 p}{(2\pi)^3} \frac{i \omega_\Bp}{\sqrt{2 \omega_\Bp}}
\lr{
   - e^{-i \omega_\Bp t + i \Bp \cdot \Bx} \lr{ a_\Bp + \frac{ i \tilde{j}(p) }{\sqrt{2 \omega_\Bp}} }
   +
   e^{i \omega_\Bp t - i \Bp \cdot \Bx} \lr{ a_\Bp^\dagger - \frac{ i \tilde{j}^\conj(p) }{\sqrt{2 \omega_\Bp}} }
} \\
(\spacegrad \phi)_k =
&= \int \frac{d^3 p}{(2\pi)^3} \frac{i p_k}{\sqrt{2 \omega_\Bp}}
\lr{
     e^{-i \omega_\Bp t + i \Bp \cdot \Bx} \lr{ a_\Bp + \frac{ i \tilde{j}(p) }{\sqrt{2 \omega_\Bp}} }
   -
   e^{i \omega_\Bp t - i \Bp \cdot \Bx} \lr{ a_\Bp^\dagger - \frac{ i \tilde{j}^\conj(p) }{\sqrt{2 \omega_\Bp}} }
},
\end{aligned}
\end{equation}
and could plug these into the Hamiltonian
\begin{dmath}\label{eqn:nonhomoKGhamiltonian:60}
H = \int d^3 p \lr{ \inv{2} \pi^2 + \inv{2} \lr{ \spacegrad \phi}^2 + \frac{m^2}{2} \phi^2 },
\end{dmath}
to find \( H \) in terms of \( \tilde{j} \) and \( a_\Bp^\dagger, a_\Bp \).  The result was mentioned in class, and it was left as an exercise to verify.

There's an easy way and a dumb way to do this exercise.  I did it the dumb way, and then after suffering through two long pages, where the equations were so long that I had to write on the paper sideways, I realized the way I should have done it.

The easy way is to observe that we've already done exactly this for the case \( \tilde{j} = 0 \), which had the answer
\begin{dmath}\label{eqn:nonhomoKGhamiltonian:80}
H = \inv{2} \int \frac{d^3 p}{(2 \pi)^3} \omega_\Bp \lr{ a_\Bp^\dagger a_\Bp + a_\Bp a_\Bp^\dagger }.
\end{dmath}
To handle this more general case, all we have to do is apply a transformation
\begin{dmath}\label{eqn:nonhomoKGhamiltonian:100}
a_\Bp \rightarrow
a_\Bp + \frac{i \tilde{j}(p)}{\sqrt{2 \omega_\Bp}},
\end{dmath}
to \cref{eqn:nonhomoKGhamiltonian:80}, which gives
\begin{dmath}\label{eqn:nonhomoKGhamiltonian:120}
H
=
\inv{2} \int \frac{d^3 p}{(2 \pi)^3} \omega_\Bp \lr{\lr{ a_\Bp + \frac{i \tilde{j}(p)}{\sqrt{2 \omega_\Bp}} }^\dagger\lr{ a_\Bp + \frac{i \tilde{j}(p)}{\sqrt{2 \omega_\Bp}} } +\lr{ a_\Bp + \frac{i \tilde{j}(p)}{\sqrt{2 \omega_\Bp}} }\lr{ a_\Bp + \frac{i \tilde{j}(p)}{\sqrt{2 \omega_\Bp}} }^\dagger }
=
\inv{2} \int \frac{d^3 p}{(2 \pi)^3} \omega_\Bp \lr{\lr{ a_\Bp^\dagger - \frac{i \tilde{j}^\conj(p)}{\sqrt{2 \omega_\Bp}} } \lr{ a_\Bp + \frac{i \tilde{j}(p)}{\sqrt{2 \omega_\Bp}} } +\lr{ a_\Bp + \frac{i \tilde{j}(p)}{\sqrt{2 \omega_\Bp}} }\lr{ a_\Bp^\dagger - \frac{i \tilde{j}^\conj(p)}{\sqrt{2 \omega_\Bp}} }
}.
\end{dmath}

Like the \( \tilde{j} = 0 \) case, we can use normal ordering.  This is easily seen by direct expansion:
\begin{dmath}\label{eqn:nonhomoKGhamiltonian:140}
\begin{aligned}
\lr{ a_\Bp^\dagger - \frac{i \tilde{j}^\conj(p)}{\sqrt{2 \omega_\Bp}} } \lr{ a_\Bp + \frac{i \tilde{j}(p)}{\sqrt{2 \omega_\Bp}} }
&=
a_\Bp^\dagger a_\Bp
- \frac{i \tilde{j}^\conj(p) a_\Bp}{\sqrt{2 \omega_\Bp}}
+ \frac{ a_\Bp^\dagger i \tilde{j}^\conj(p)}{\sqrt{2 \omega_\Bp}}
+ \frac{\Abs{\tilde{j}}^2}{2 \omega_\Bp} \\
\lr{ a_\Bp + \frac{i \tilde{j}(p)}{\sqrt{2 \omega_\Bp}} }\lr{ a_\Bp^\dagger - \frac{i \tilde{j}^\conj(p)}{\sqrt{2 \omega_\Bp}} }
&=
a_\Bp^\dagger a_\Bp
+ \frac{i \tilde{j}^\conj(p) a_\Bp^\dagger}{\sqrt{2 \omega_\Bp}}
- \frac{ a_\Bp i \tilde{j}^\conj(p)}{\sqrt{2 \omega_\Bp}}
+ \frac{\Abs{\tilde{j}}^2}{2 \omega_\Bp}.
\end{aligned}
\end{dmath}
Because \( \tilde{j} \) is just a complex valued function, it commutes with \( a_\Bp, a_\Bp^\dagger \), and these are equal up to the normal ordering, allowing us to write
\begin{equation}\label{eqn:nonhomoKGhamiltonian:160}
:H: =
\int \frac{d^3 p}{(2 \pi)^3} \omega_\Bp \lr{ a_\Bp^\dagger - \frac{i \tilde{j}^\conj(p)}{\sqrt{2 \omega_\Bp}}} \lr{ a_\Bp + \frac{i \tilde{j}(p)}{\sqrt{2 \omega_\Bp}} },
\end{equation}
which is the result mentioned in class (albeit without the explicit normal ordering syntax.)
} % answer

%}
%\EndNoBibArticle

   \chapter{Time evolution, Hamiltonian pertubation, ground state.}
      %
% Copyright � 2017 Peeter Joot.  All Rights Reserved.
% Licenced as described in the file LICENSE under the root directory of this GIT repository.
%
%{
\input{../latex/blogpost.tex}
\renewcommand{\basename}{qftLecture14}
\renewcommand{\dirname}{notes/phy2403/}
\newcommand{\keywords}{PHY2403H}
\input{../latex/peeter_prologue_print2.tex}

%\usepackage{phy2403}
\usepackage{peeters_braket}
%\usepackage{peeters_layout_exercise}
\usepackage{peeters_figures}
\usepackage{mathtools}
\usepackage{siunitx}
\usepackage{macros_cal} % LL

\newcommand{\ultensor}[3]{{{#1}^{#2}}_{#3}}

\beginArtNoToc
\generatetitle{PHY2403H Quantum Field Theory.  Lecture 14: XXX.  Taught by Prof.\ Erich Poppitz}
%\chapter{XXX}
\label{chap:qftLecture14}

%s/_{\text{I,int}}/_{\\text{I,int}}

%%Peeter's lecture notes from class.  These may be incoherent and rough.
%%
%%These are notes for the UofT course PHY2403H, Quantum Field Theory, taught by Prof. Erich Poppitz, covering \textchapref{{1}} \citep{peskin1995introduction} content.

\paragraph{DISCLAIMER: Very rough notes from class, with some additional side notes.}

These are notes for the UofT course PHY2403H, Quantum Field Theory, taught by Prof. Erich Poppitz, fall 2018.
%, covering \textchapref{{1}} \citep{peskin1995introduction} content.

\section{Review}

Given a field \( \phi(t_0, \Bx) \), satisfying the commutation relations
\begin{dmath}\label{eqn:qftLecture14:20}
\antisymmetric{\pi(t_0, \Bx)}{\phi(t_0, \By)} = -i \delta(\Bx - \By)
\end{dmath}
we introduced an interaction picture field given by
\begin{dmath}\label{eqn:qftLecture14:40}
\phi_I(t, x) = e^{i H_0(t- t_0)} \phi(t_0, \Bx) e^{-iH_0(t - t_0)}
\end{dmath}
related to the Heisenburg picture representation by
\begin{dmath}\label{eqn:qftLecture14:60}
\phi_H(t, x)
= e^{i H(t- t_0)} \phi(t_0, \Bx) e^{-iH(t - t_0)}
= U^\dagger(t, t_0) \phi_I(t, \Bx) U(t, t_0),
\end{dmath}
where \( U(t, t_0) \) is the time evolution operator.
\begin{dmath}\label{eqn:qftLecture14:80}
U(t, t_0) =
e^{i H_0(t - t_0)}
e^{i H(t - t_0)}
\end{dmath}
We argued that
\begin{dmath}\label{eqn:qftLecture14:100}
i \PD{t}{} U(t, t_0) = H_{\text{I,int}}(t) U(t, t_0)
\end{dmath}
We found the glorious expression
\begin{dmath}\label{eqn:qftLecture14:120}
U(t, t_0)
= T \exp{\lr{ -i \int_{t_0}^t H_{\text{I,int}}(t') dt'}}
=
\sum_{n = 0}^\infty \frac{(-i)^n}{n!} \int_{t_0}^t dt_1 dt_2 \cdots dt_n T\lr{ H_{\text{I,int}}(t_1) H_{\text{I,int}}(t_2) \cdots H_{\text{I,int}}(t_n) }
\end{dmath}

However, what we are really after is
\begin{dmath}\label{eqn:qftLecture14:140}
\bra{\Omega} T(\phi(x_1) \cdots \phi(x_n)) \ket{\Omega}
\end{dmath}
Such a product has many labels and names, and we'll describe it as ``vacuum expectation values of time-ordered products of arbitrary \#s of local Heisenberg operators''.

\section{Pertubation}

\begin{dmath}\label{eqn:qftLecture14:n}
\begin{aligned}
H &= \text{exact Hamiltonian} = H_0 + H_int
H_0 &= \text{free Hamiltonian.  We know all about \( H_0 \) and assume that it has a lowest (groundstate) \( \ket{0} \), the ``vacuum'' state of \( H_0 \).}
\end{aligned}
\end{dmath}

H has eigenstates, in particular, assumed to have a unique ground state \( \ket{\Omega} \) satisfying
\begin{dmath}\label{eqn:qftLecture14:n}
H \ket{\Omega}  = \ket{\Omega} E_0,
\end{dmath}
and has states \( \ket{n} \), representing excited (non-vacuum states with energies > \( E_0 \)).
These states are assumed to be a complete basis
\begin{dmath}\label{eqn:qftLecture14:n}
\BOne = \ket{\Omega}\bra{\Omega} + \sum_n \ket{n}\bra{n} + \int dn \ket{n}\bra{n}.
\end{dmath}
The latter is sometimes written with a superimposed sum-integral notation: F1.

For some time \( T \) we have
\begin{dmath}\label{eqn:qftLecture14:n}
e^{-i H T} \ket{0} = e^{-i H T} 
\lr{
   \ket{\Omega}\braket{\Omega}{0} + \sum_n \ket{n}\braket{n}{0} + \int dn \ket{n}\braket{n}{0}.
},
\end{dmath}
and now pull a fast one, considering a formal transformation \( T \rightarrow T(1 - i \epsilon) \), where \( \epsilon \rightarrow 0^+ \), and consider very large \( T \).  This gives
\begin{dmath}\label{eqn:qftLecture14:n}
\lim_{T \rightarrow \infty, \epsilon \rightarrow 0^+} 
e^{-i H T(1 - i \epsilon)} \ket{0} 
= 
\lim_{T \rightarrow \infty, \epsilon \rightarrow 0^+} 
e^{-i H T(1 - i \epsilon)} 
\lr{
   \ket{\Omega}\braket{\Omega}{0} + \sum_n \ket{n}\braket{n}{0} + \int dn \ket{n}\braket{n}{0}.
}
= 
\lim_{T \rightarrow \infty, \epsilon \rightarrow 0^+} 
e^{-i E_0 T - E_0 T \epsilon} 
   \ket{\Omega}\braket{\Omega}{0} + \sumint_n e^{-i E_n T - \epsilon E_n T} \ket{n}\braket{n}{0}
=
\lim_{T \rightarrow \infty, \epsilon \rightarrow 0^+} 
e^{-i E_0 T - E_0 T \epsilon} 
\lr{
   \ket{\Omega}\braket{\Omega}{0} + \sumint_n e^{-i (E_n -E_0) T - \epsilon T (E_n - E_0)} \ket{n}\braket{n}{0}
}
\end{dmath}
The limits are evaluated by first taking \( T \) to infinity, then only after that take \( \epsilon \rightarrow 0^+ \).  Doing this, the sum is dominated by the ground state contribution, since each excited state also has a \( e^{-\epsilon T(E_n - E_0)} \) suppression factor (in addition to the leading suppression factor).

Less formally, we can write
\begin{dmath}\label{eqn:qftLecture14:n}
\sumint \ket{n}\bra{n} \rightarrow 
\sum_k \int \frac{d^3 p}{(2 \pi)^3} \ket{\Bp, k}\bra{\Bp, k}
\end{dmath}
where \( k \) is some unknown quantity that we are summing over.
If we have 
\begin{dmath}\label{eqn:qftLecture14:n}
H \ket{\Bp, k} = E_{\Bp, k} \ket{\Bp, k},
\end{dmath}
then
\begin{dmath}\label{eqn:qftLecture14:n}
e^{-i H T} \sumint \ket{n}\bra{n}
=
\sum_k \int \frac{d^3 p}{(2 \pi)^3} \ket{\Bp, k} e^{-i E_{\Bp, k} \bra{\Bp, k}.
\end{dmath}
If we take matrix elements
\begin{dmath}\label{eqn:qftLecture14:n}
\bra{A} 
e^{-i H T} \sumint \ket{n}\bra{n} \ket{B}
=
\sum_k \int \frac{d^3 p}{(2 \pi)^3} \braket{A}{\Bp, k} e^{-i E_{\Bp, k} \braket{\Bp, k}{B}
=
\sum_k \int \frac{d^3 p}{(2 \pi)^3} e^{-i E_{\Bp, k} f(\Bp).
\end{dmath}
If we assume that \( f(\Bp) \) is a well behaved smooth function, we have ``infinite'' frequency oscillation within a windowed function, as sketched in
F2
The Riemann-Lebescue lemma \citep{wiki:RiemannLebesgue} describes such integrals, the result of which is that such an integral goes to zero.  This is a different hand waving argument to argue that we can examine only the ground state contribution above.
\begin{dmath}\label{eqn:qftLecture14:n}
e^{-i H T} \ket{0} = e^{-i E_0 T} \ket{\Omega}\braket{\Omega}{0}
\end{dmath}
in the \( T \rightarrow \infty(1 - i \epsilon) \) limit.  We can now write the ground state as

\begin{dmath}\label{eqn:qftLecture14:n}
\ket{\Omega}
= 
\evalbar{
\frac{ e^{i E_0 T - i H T } \ket{0} }{
\braket{\Omega}{0} 
}
}{ T \rightarrow \infty(1 - i \epsilon) }
=
\evalbar{
\frac{ e^{i H (T - t_0) } \ket{0} }{
e{-i E_0(T + t_0)} \braket{\Omega}{0} 
}
}{ T \rightarrow \infty(1 - i \epsilon) }
=
\evalbar{
\frac{ e^{i H (t_0 - (-T)) } e^{ -i H_0 (-T - t_0) } \ket{0} }{
e{-i E_0(t_0 - (-T)} \braket{\Omega}{0} 
}
}{ T \rightarrow \infty(1 - i \epsilon) }
\end{dmath}
where we use \( H_0 \ket{0} = 0 \).

With 
\begin{dmath}\label{eqn:qftLecture14:n}
U(t, t_0)
= e^{i H_0(t - t_0)} e^{-i H(t - t_0)}
= T \exp{\lr{ -i \int_{t_0}^t H_{\text{I,int}}(t') dt'}}
\end{dmath}

\paragraph{Claim:} Similarly (DIY) 
\begin{dmath}\label{eqn:qftLecture14:n}
U(t, t')
= e^{i H_0(t - t_0)} e^{-i H(t - t')} e^{-i H_ (t' - t_0)}
= T \exp{\lr{ -i \int_{t'}^t H_{\text{I,int}}(t'') dt''}}
\end{dmath}
Observe that we recover \cref{eqn:qftLecture14:120} when \( t' = t_0 \).

This gives
\boxedEquation{eqn:qftLecture14:n}{
\ket{\Omega}
= \frac{U(t_0, -T) \ket{0} }{e^{-i E_0(t_0 - (-T))} \braket{\Omega}{0}
}.

We are close to where we want to be.  Wednesday we will start scattering and Feynman diagrams.

%}
\EndArticle

   \chapter{Perturbation ground state, time evolution operator, time ordered product, interaction.}
      %
% Copyright � 2017 Peeter Joot.  All Rights Reserved.
% Licenced as described in the file LICENSE under the root directory of this GIT repository.
%
%{
%%\input{../latex/blogpost.tex}
%%\renewcommand{\basename}{qftLecture15}
%%\renewcommand{\dirname}{notes/phy2403/}
%%\newcommand{\keywords}{PHY2403H}
%%\input{../latex/peeter_prologue_print2.tex}
%%
%%%\usepackage{phy2403}
%%\usepackage{peeters_braket}
%%%\usepackage{peeters_layout_exercise}
%%\usepackage{peeters_figures}
%%\usepackage{mathtools}
%%\usepackage{siunitx}
%%\usepackage{macros_cal} % LL
%%
%%\newcommand{\ultensor}[3]{{{#1}^{#2}}_{#3}}
%%
%%\beginArtNoToc
%%\generatetitle{PHY2403H Quantum Field Theory.  Lecture 15 part 1: Perturbation ground state, time evolution operator, time ordered product, interaction.  Taught by Prof.\ Erich Poppitz}
\chapter{Perturbation ground state, time evolution operator, time ordered product, interaction}
\index{ground state}
\index{time evolution operator}
\index{time ordered product}
\index{interaction picture}
\label{chap:qftLecture15}

%%Peeter's lecture notes from class.  These may be incoherent and rough.
%%
%%These are notes for the UofT course PHY2403H, Quantum Field Theory, taught by Prof. Erich Poppitz, covering \textchapref{{1}} \citep{peskin1995introduction} content.

%\paragraph{DISCLAIMER: Very rough notes from class, with some additional side notes.}
%
%These are notes for the UofT course PHY2403H, Quantum Field Theory, taught by Prof. Erich Poppitz, fall 2018.
%%, covering \textchapref{{1}} \citep{peskin1995introduction} content.

\section{Review}

We developed the interaction picture representation, which is really the Heisenberg picture with respect to \( H_0 \).

Recall that we found
\begin{equation}\label{eqn:qftLecture15:20}
U(t, t') = e^{i H_0(t - t_0)} e^{-i H(t - t')} e^{-i H_0(t' - t_0)},
\end{equation}
with solution
\begin{dmath}\label{eqn:qftLecture15:200}
U(t, t')
=
T \exp{\lr{ -i \int_{t'}^t H_{\text{I,int}}(t'') dt''}},
\end{dmath}
\begin{dmath}\label{eqn:qftLecture15:220}
U(t, t')^\dagger
=
T \exp{\lr{ i \int_{t'}^{t} H_{\text{I,int}}(t'') dt''}}
=
T \exp{\lr{ -i \int_{t}^{t'} H_{\text{I,int}}(t'') dt''}}
= U(t', t),
\end{dmath}
and can use this to calculate the time evolution of a field \(
%\begin{dmath}\label{eqn:qftLecture15:40}
\phi(\Bx, t)
=
U^\dagger(t, t_0)
\phi_I(\Bx, t)
U(t, t_0)
\)
%\end{dmath}
and found the ground state ket for \( H \) was
\begin{dmath}\label{eqn:qftLecture15:60}
\ket{\Omega}
=
\evalbar{
\frac{ U(t_0, -T) \ket{0} }
{
e^{-i E_0(T - t_0)} \braket{\Omega}{0}
}
}{T \rightarrow \infty(1 - i \epsilon)}.
\end{dmath}
\paragraph{Question:} What's the point of this, since it is self referential?
\paragraph{Answer:} We will see, and also see that it goes away.  Alternatively, you can write it as
\begin{equation*}
\ket{\Omega} \braket{\Omega}{0}
=
\evalbar{
\frac{ U(t_0, -T) \ket{0} }
{
e^{-i E_0(T - t_0)}
}
}{T \rightarrow \infty(1 - i \epsilon)}.
\end{equation*}

We can also show that
\begin{dmath}\label{eqn:qftLecture15:80}
\bra{\Omega}
=
\evalbar{
\frac{ \bra{0} U(T, t_0) }
{
e^{-i E_0(T - t_0)} \braket{0}{\Omega}
}
}{T \rightarrow \infty(1 - i \epsilon)}.
\end{dmath}

Our goal is still toe calculate
\begin{dmath}\label{eqn:qftLecture15:100}
\bra{\Omega} T \phi(x) \phi(y) \ket{\Omega}.
\end{dmath}
Claim: the ``LSZ'' theorem (a neat way of writing this) relates this to S matrix elements.

Assuming \( x^0 > y^0 \)

\begin{dmath}\label{eqn:qftLecture15:120}
\bra{\Omega} \phi(x) \phi(y) \ket{\Omega}
=
\frac{
\bra{0}
U(T, t_0)
U^\dagger(x^0, t^0)
\phi_I(x)
U(x^0, t^0)
%
U^\dagger(y^0, t^0)
\phi_I(y)
U(y^0, t^0)
U(t_0, -T)
\ket{0}
}
{
e^{-i 2 E_0 T} \Abs{\braket{0}{\Omega}}^2
}.
\end{dmath}
Normalize \( \braket{\Omega}{\Omega} = 1 \), gives
\begin{dmath}\label{eqn:qftLecture15:140}
1
=
\frac{\bra{0} U(T, t_0) U(t_0, -T) \ket{0}}
{
e^{-i 2 E_0 T} \Abs{\braket{0}{\Omega}}^2
}
=
\frac{\bra{0} U(T, -T) \ket{0}}
{
e^{-i 2 E_0 T} \Abs{\braket{0}{\Omega}}^2
},
\end{dmath}
so that
\begin{dmath}\label{eqn:qftLecture15:240}
\bra{\Omega} \phi(x) \phi(y) \ket{\Omega}
=
\frac{
\bra{0}
U(T, t_0)
U^\dagger(x^0, t^0)
\phi_I(x)
U(x^0, t^0)
%
U^\dagger(y^0, t^0)
\phi_I(y)
U(y^0, t^0)
U(t_0, -T)
\ket{0}
}
{
   \bra{0} U(T, -T) \ket{0}
}.
\end{dmath}
For \( t_1 > t_2 > t_3 \)
\begin{dmath}\label{eqn:qftLecture15:280}
U(t_1, t_2) U(t_2, t_3)
=
T e^{-i \int_{t_2}^{t_1} H_I}
T e^{-i \int_{t_3}^{t_2} H_I}
=
T \lr{
e^{-i \int_{t_2}^{t_1} H_I}
e^{-i \int_{t_3}^{t_2} H_I}
}
=
T(
e^{-i \int_{t_3}^{t_1} H_I}
),
\end{dmath}
with an end result of
\begin{dmath}\label{eqn:qftLecture15:320}
U(t_1, t_2) U(t_2, t_3) = U(t_1, t_3).
\end{dmath}
(DIY: work through the details -- this is a problem in \citep{peskin1995introduction})

This gives
\begin{dmath}\label{eqn:qftLecture15:300}
\bra{\Omega} \phi(x) \phi(y) \ket{\Omega}
=
\frac{
\bra{0}
U(T, x^0)
\phi_I(x)
U(x^0, y^0)
\phi_I(y)
U(y^0, -T)
\ket{0}
}
{
   \bra{0} U(T, -T) \ket{0}
}.
\end{dmath}

If \( y^0 > x^0 \) we have the same result, but the \( y \)'s will come first.

\paragraph{Claim:}
\begin{dmath}\label{eqn:qftLecture15:340}
\bra{\Omega} \phi(x) \phi(y) \ket{\Omega}
=
\frac{
   \bra{0}
   T\lr{
      \phi_I(x)
      \phi_I(y)
      e^{-i \int_{-T}^T H_{\text{I,int}}(t') dt'}
   }
   \ket{0}
}
{
   \bra{0}
      T ( e^{-i \int_{-T}^T H_{\text{I,int}}(t') dt'} )
   \ket{0}
}.
\end{dmath}

More generally
\boxedEquation{eqn:qftLecture15:360}{
\bra{\Omega}
\phi_I(x_1) \cdots
\phi_I(x_n)
\ket{\Omega}
=
\frac{
   \bra{0}
   T\lr{
\phi_I(x_1) \cdots
\phi_I(x_n)
      e^{-i \int_{-T}^T H_{\text{I,int}}(t') dt'}
   }
   \ket{0}
}
{
   \bra{0}
      T ( e^{-i \int_{-T}^T H_{\text{I,int}}(t') dt'} )
   \ket{0}
}.
}
This is the holy grail of perturbation theory.

In QFT II you will see this written in a path integral representation
\begin{dmath}\label{eqn:qftLecture15:380}
\bra{\Omega}
\phi_I(x_1) \cdots
\phi_I(x_n)
\ket{\Omega}
=
\frac
{
   \int [\calD \phi] \phi(x_1) \phi(x_2) \cdots \phi(x_n) e^{-i S[\phi]}
}
{
   \int [\calD \phi] e^{-i S[\phi]}
}.
\end{dmath}

\section{Unpacking it.}

\begin{dmath}\label{eqn:qftLecture15:400}
\int_{-T}^T H_{\text{I,int}}(t)
=
\int_{-T}^T
\int d^3 \Bx \frac{\lambda}{4} \lr{ \phi_I(\Bx, t) }^4
=
\int d^4 x
\frac{\lambda}{4} \lr{ \phi_I }^4,
\end{dmath}
so we have
\begin{dmath}\label{eqn:qftLecture15:420}
\frac{
   \bra{0}
   T\lr{
\phi_I(x_1) \cdots
\phi_I(x_n)
      e^{-i \frac{\lambda}{4} \int d^4 x \phi_I^4(x) }
   }
   \ket{0}
}
{
   \bra{0}
      T
      e^{-i \frac{\lambda}{4} \int d^4 x \phi_I^4(x) }
   \ket{0}
}.
\end{dmath}

The numerator expands as
\begin{dmath}\label{eqn:qftLecture15:440}
   \bra{0} T\lr{ \phi_I(x_1) \cdots \phi_I(x_n) } \ket{0}
-i \frac{\lambda}{4} \int d^4 x
   \bra{0} T\lr{ \phi_I(x_1) \cdots \phi_I(x_n) \phi_I^4(x) }
+
\inv{2}
\lr{-i \frac{\lambda}{4}}^2 \int d^4 x d^4 y
   \bra{0} T\lr{ \phi_I(x_1) \cdots \phi_I(x_n)
      \phi_I^4(x)
      \phi_I^4(y)
} \ket{0}
+ \cdots
\end{dmath}
so we see that the problem ends up being the calculation of time ordered products.

\section{Calculating perturbation}
Let's simplify notation, dropping interaction picture suffixes, writing \( \phi(x_i) = \phi_i \).

Let's calculate \(
   \bra{0} T\lr{ \phi_1 \cdots \phi_n } \ket{0}
\).  For \( n = 2 \) we have

\begin{dmath}\label{eqn:qftLecture15:460}
\bra{0} T\lr{ \phi_1 \cdots \phi_n } \ket{0}
= D_F(x_1 - x_2) \equiv D_F(1-2).
\end{dmath}

%\paragraph{TO BE CONTINUED.}
%The rest of the lecture was very visual, and hard to type up.  I'll do so later.
%}
%\EndArticle

      %
% Copyright � 2017 Peeter Joot.  All Rights Reserved.
% Licenced as described in the file LICENSE under the root directory of this GIT repository.
%
%{
\input{../latex/blogpost.tex}
\renewcommand{\basename}{qftLecture15b}
\renewcommand{\dirname}{notes/phy2403/}
\newcommand{\keywords}{PHY2403H}
\input{../latex/peeter_prologue_print2.tex}

%\usepackage{phy2403}
\usepackage{peeters_braket}
%\usepackage{peeters_layout_exercise}
\usepackage{peeters_figures}
\usepackage{mathtools}
\usepackage{siunitx}
\usepackage{macros_cal} % LL

\newcommand{\ultensor}[3]{{{#1}^{#2}}_{#3}}

\beginArtNoToc
\generatetitle{PHY2403H Quantum Field Theory.  Lecture 15b: XXX.  Taught by Prof.\ Erich Poppitz}
%\chapter{XXX}
\label{chap:qftLecture15b}

%%Peeter's lecture notes from class.  These may be incoherent and rough.
%%
%%These are notes for the UofT course PHY2403H, Quantum Field Theory, taught by Prof. Erich Poppitz, covering \textchapref{{1}} \citep{peskin1995introduction} content.

\paragraph{DISCLAIMER: Very rough notes from class, with some additional side notes.}

These are notes for the UofT course PHY2403H, Quantum Field Theory, taught by Prof. Erich Poppitz, fall 2018.
%, covering \textchapref{{1}} \citep{peskin1995introduction} content.

\section{YYY}

%}
\EndArticle
%\EndNoBibArticle

      \section{Problems.}
         %
% Copyright © 2018 Peeter Joot.  All Rights Reserved.
% Licenced as described in the file LICENSE under the root directory of this GIT repository.
%
%{
\makeproblem{\( U(T, t_0) U(t_0, -T) \) }{problem:qftLecture15Problems:1}{
Show that
\begin{equation*}
U(T, t_0) U(t_0, -T) = U(T, -T).
\end{equation*}
} % problem

\makeanswer{problem:qftLecture15Problems:1}{
We can see that from
\begin{equation}\label{eqn:qftLecture15Problems:160}
\begin{aligned}
U(T, t_0) &= e^{i H_0(T - t_0)} e^{-i H(T - t_0)} \cancel{e^{-i H_0(t_0 - t_0)}}  \\
U(t_0, -T) &= \cancel{e^{i H_0(t_0 - t_0)}} e^{-i H(t_0 - -T)} e^{-i H_0(-T - t_0)},
\end{aligned}
\end{equation}
so
\begin{equation}\label{eqn:qftLecture15Problems:180}
\begin{aligned}
U(T, t_0)
U(t_0, -T)
&=
e^{i H_0(T - t_0)} e^{-i H(T - t_0)} e^{-i H(t_0 + T)} e^{-i H_0(-T - t_0)}
\\&=
e^{i H_0(T - t_0)} e^{-i H 2 T } e^{-i H_0(-T - t_0)},
\end{aligned}
\end{equation}
whereas
\begin{equation}\label{eqn:qftLecture15Problems:260}
\begin{aligned}
U(T, -T)
  &= e^{i H_0(T - t_0)} e^{-i H(T - -T)} e^{-i H_0(-T - t_0)}
\\&= e^{i H_0(T - t_0)} e^{-i H 2 T} e^{-i H_0(-T - t_0)}.
\end{aligned}
\end{equation}
} % answer
%}

         %
% Copyright � 2018 Peeter Joot.  All Rights Reserved.
% Licenced as described in the file LICENSE under the root directory of this GIT repository.
%
%{
%\input{../latex/blogpost.tex}
%\renewcommand{\basename}{braOmega}
%%\renewcommand{\dirname}{notes/phy1520/}
%\renewcommand{\dirname}{notes/ece1228-electromagnetic-theory/}
%%\newcommand{\dateintitle}{}
%%\newcommand{\keywords}{}
%
%\input{../latex/peeter_prologue_print2.tex}
%
%\usepackage{peeters_layout_exercise}
%\usepackage{peeters_braket}
%\usepackage{peeters_figures}
%\usepackage{siunitx}
%\usepackage{verbatim}
%%\usepackage{mhchem} % \ce{}
%%\usepackage{macros_bm} % \bcM
%%\usepackage{macros_qed} % \qedmarker
%%\usepackage{txfonts} % \ointclockwise
%
%% qftLecture14:
%% https://tex.stackexchange.com/a/68357/15
%\DeclareMathOperator*{\SumInt}{%
%\mathchoice%
%  {\ooalign{$\displaystyle\sum$\cr\hidewidth$\displaystyle\int$\hidewidth\cr}}
%  {\ooalign{\raisebox{.14\height}{\scalebox{.7}{$\textstyle\sum$}}\cr\hidewidth$\textstyle\int$\hidewidth\cr}}
%  {\ooalign{\raisebox{.2\height}{\scalebox{.6}{$\scriptstyle\sum$}}\cr$\scriptstyle\int$\cr}}
%  {\ooalign{\raisebox{.2\height}{\scalebox{.6}{$\scriptstyle\sum$}}\cr$\scriptstyle\int$\cr}}
%}
%
%\beginArtNoToc
%
%\generatetitle{PHY2403 (QFT I).  Pondering the ground state bra formula.}
%%\chapter{XXX}
%%\label{chap:braOmega}
%

\makeproblem{Pondering the ground state bra formula.}{problem:braOmega:1}{
\index{ground state}
Prove \cref{eqn:qftLecture15:80}.  What is wrong with conjugating
\cref{eqn:qftLecture15:60} to find
\begin{equation*}
\bra{\Omega}
=
\evalbar{
   \frac{ \bra{0} U(-T, t_0) }
   {
   e^{+i E_0(T - t_0)} \braket{0}{\Omega}
   }
}
{
T \rightarrow \infty( 1 - i \epsilon )
}.
\end{equation*}
} % problem

\makeanswer{problem:braOmega:1}{
While there is nothing wrong with stating
\begin{equation}\label{eqn:braOmega:100}
\lr{
   \frac{ U(t_0, -T) \ket{0} }
   {
   e^{-i E_0(T - t_0)} \braket{\Omega}{0}
   }
}^\dagger
=
   \frac{ \bra{0} U(-T, t_0) }
   {
   e^{+i E_0(T - t_0)} \braket{0}{\Omega}
   },
\end{equation}
the limit point \( \infty(1 - i \epsilon) \) also needs to be changed with this conjugation.  So \cref{eqn:braOmega:100} is correct, but it is only part of the story, and should really be stated as
\begin{equation}\label{eqn:braOmega:120}
\bra{\Omega}
=
\evalbar{
   \frac{ \bra{0} U(-T, t_0) }
   {
   e^{+i E_0(T - t_0)} \braket{0}{\Omega}
   }
}{T \rightarrow \infty(1 + i \epsilon)}.
\end{equation}
This is awkward because now our expressions for \( \bra{\Omega} \) and \( \ket{\Omega} \) approach \( T \) from different directions, and we want to evaluate both with a single limiting argument.

To resolve this, we really have to start back with the identity expansion we used in lecture 14, and write
\begin{equation}\label{eqn:braOmega:140}
\begin{aligned}
\bra{0} e^{-i H T}
&=
\lr{
\braket{0}{\Omega}\bra{\Omega}
 + \SumInt_n \braket{0}{n} \bra{n}
}
e^{-i H T}
\\&=
\braket{0}{\Omega}\bra{\Omega}
e^{-i E_0 T}
 + \SumInt_n \braket{0}{n} \bra{n} e^{-i E_n T}.
\end{aligned}
\end{equation}
We argued (as does the text) that approaching to as \( T( 1 - i \epsilon) \) kills off the energetic states since
\begin{equation}\label{eqn:braOmega:160}
\bra{n} e^{-i E_n T}
\rightarrow
\bra{n} e^{-i E_n T} e^{-E_n T \epsilon}
\end{equation}
and the exponential damping factor is smaller for each \( E_n > E_0 \), so it can be neglected in the large \( T \) limit, leaving
\begin{equation}\label{eqn:braOmega:180}
\bra{0} e^{-i H T} = \lim_{T \rightarrow \infty(1 - i \epsilon)}
\braket{0}{\Omega}\bra{\Omega}.
\end{equation}
As we did for \( \ket{\Omega} \) we can shift the large time \( T \) by a small constant (this time \( -t_0 \) instead of \( t_0 \)), to give
\begin{equation}\label{eqn:braOmega:200}
\begin{aligned}
\bra{\Omega}
&=
\lim_{T \rightarrow \infty(1 - i \epsilon)}
\frac{ \bra{0} e^{-i H T} }
{
\braket{0}{\Omega} e^{-i E_0 T}
}
\\&\approx
\lim_{T \rightarrow \infty(1 - i \epsilon)}
\frac{ \bra{0} e^{-i H (T - t_0)} }
{
\braket{0}{\Omega} e^{-i E_0 (T - t_0)}
}
\\&=
\lim_{T \rightarrow \infty(1 - i \epsilon)}
\frac{ \bra{0} e^{i H_0( T - t_0)} e^{-i H (T - t_0)} }
{
\braket{0}{\Omega} e^{-i E_0 (T - t_0)}
}
\\&=
\lim_{T \rightarrow \infty(1 - i \epsilon)}
\frac{ \bra{0} U(T, t_0) }
{
\braket{0}{\Omega} e^{-i E_0 (T - t_0)}
},
\end{aligned}
\end{equation}
where the projective property \( \bra{0} e^{i H_0 \alpha} = \bra{0} \) has been used to insert a no-op (i.e. \( \bra{0} H_0 = 0 \)).  This recovers the result stated in class (also:
\citep{peskin1995introduction}
eq. (4.29).)
} % answer

%}
%\EndArticle
%\EndNoBibArticle

   \chapter{Differential cross section, scattering, pair production, transition amplitude, decay rate, S-matrix, connected and amputated diagrams, vacuum fluctuation, symmetry coefficient.}
      %
% Copyright � 2017 Peeter Joot.  All Rights Reserved.
% Licenced as described in the file LICENSE under the root directory of this GIT repository.
%
%{
\input{../latex/blogpost.tex}
\renewcommand{\basename}{qftLecture16}
\renewcommand{\dirname}{notes/phy2403/}
\newcommand{\keywords}{PHY2403H}
\input{../latex/peeter_prologue_print2.tex}

%\usepackage{phy2403}
\usepackage{peeters_braket}
\usepackage{peeters_layout_exercise}
\usepackage{peeters_figures}
\usepackage{mathtools}
\usepackage{siunitx}
\usepackage{macros_cal} % LL
\usepackage{simplewick}
\usepackage{verbatim}

\newcommand{\ultensor}[3]{{{#1}^{#2}}_{#3}}

\beginArtNoToc
\generatetitle{PHY2403H Quantum Field Theory.  Lecture 16: XXX.  Taught by Prof.\ Erich Poppitz}
%\chapter{XXX}
\label{chap:qftLecture16}

%%Peeter's lecture notes from class.  These may be incoherent and rough.
%%
%%These are notes for the UofT course PHY2403H, Quantum Field Theory, taught by Prof. Erich Poppitz, covering \textchapref{{1}} \citep{peskin1995introduction} content.

\paragraph{DISCLAIMER: Very rough notes from class, with some additional side notes.}

These are notes for the UofT course PHY2403H, Quantum Field Theory, taught by Prof. Erich Poppitz, fall 2018.
%, covering \textchapref{{1}} \citep{peskin1995introduction} content.

\section{Review}

We finished by defining the differential cross section

\makedefinition{Differential cross section.}{dfn:qftLecture16:20}{
\begin{equation*}
\frac{d^3 \sigma}{dp_x dp_y dp_z} = \frac{
\text{number of scattering events with \( \Bp_{\txtf} \) between \( (\Bp_\txtf, \Bp_\txtf + d \Bp_\txtf )\)}
}
{
\text{flux of incoming particles}
}.
\end{equation*}
} % definition

\section{Scattering}

In QFT we typically study \( 2 \rightarrow n \) inelastic scattering.  Most commonly the nature of the final state particles are different from the nature of the incoming state.

For example, we can collide two electrons, and can get muon and anti-muon particles
F1
or pions
F2, or even both
F3

In the \( \lambda \phi^4 \) theory we can have scattering events such as
F4a
F4b

How to calculate in QFT.  Initial state of 2 particles \( A, B \) with initital state
\begin{dmath}\label{eqn:qftLecture16:40}
\ket{\Bk_A, \Bk_B }_{\text{in}, T \rightarrow -\infty}
\end{dmath}
and final n-particle state
\begin{dmath}\label{eqn:qftLecture16:60}
\ket{\Bp_1, \Bp_2, \cdots, \Bp_n }_{\text{out}, T \rightarrow +\infty}
\end{dmath}
The QM transition amplitude from the initial to the final state is
\begin{dmath}\label{eqn:qftLecture16:80}
\prescript{}{\text{out}}{\bra{\Bp_1, \Bp_2, \cdots, \Bp_n }}
\ket{\Bk_A, \Bk_B }_{\text{in}}
=
\bra{\Bp_1, \Bp_2, \cdots, \Bp_n } e^{-2 i H T}
\ket{\Bk_A, \Bk_B }.
\end{dmath}
This is the amplitude for \( A B \rightarrow 1 \cdots n \).
Ultimately, we want the scattering x-section.

We will also be interested in decay rates, as there are unstable particles in QFT that can decay.  This doesn't happen in \( \lambda \phi^4 \) theory.
In a theory with 2 scalar fields \( \Phi, \varphi \) with \( m_\Phi > 2 m_\varphi \).  A possible interaction for such a theory is
\begin{dmath}\label{eqn:qftLecture16:100}
H_{\text{int}} = \mu \Phi \varphi^2,
\end{dmath}
which would permit \( \Phi \rightarrow \varphi \varphi \) decays.
HW4 has a coupling like \( (h/V) \partial_\mu \phi^a \partial^\mu \phi^a \) for which a \( h \rightarrow \phi^a \phi^a \) decay is possible.

\makedefinition{Decay rate.}{dfn:qftLecture16:120}{
The decay rate is defined as
\begin{equation*}
\Gamma =
\frac{
\text{
Number of decays \( \Phi \rightarrow \varphi \varphi \) in unit time
}
}
{
\text{
Number of \( \Phi \) particles present
}
}
\end{equation*}
} % definition

What is the amplitude for such a decay transition?
\begin{dmath}\label{eqn:qftLecture16:140}
\bra{\Bk_\phi}_{\text{in}, T \rightarrow -\infty} \rightarrow
\bra{\Bk_1, \Bk_2}_{\text{out}, T \rightarrow +\infty}.
\end{dmath}
The amplitude for \( \Bk_\phi \rightarrow \Bk_1, \Bk_2 \).
\begin{dmath}\label{eqn:qftLecture16:160}
\bra{\Bk_1, \Bk_2} e^{-i 2 H T } \ket{\Bk_\phi}
=
\prescript{}{\text{out}}{\braket{\Bk_1, \Bk_2}{\Bk_\phi}}
\end{dmath}

\paragraph{mysterious seeming statement something like}: ``The decays are essentially due to interactions with vacuum fluctuations.''

\section{Calculating interactions}

We write
\begin{dmath}\label{eqn:qftLecture16:180}
\begin{aligned}
\prescript{}{\text{out}}{\braket{ \Bp_1, \cdots \Bp_n }{ \Bk_A, \Bk_B }}_{\text{in}}
&=
\lim_{T \rightarrow \infty}
\bra{ \Bp_1, \cdots \Bp_n } e^{-i 2 H T } \ket{ \Bk_A, \Bk_B } \\
&=
\bra{ \Bp_1, \cdots \Bp_n } \hatS \ket{ \Bk_A, \Bk_B } \\
&=
\bra{ \Bp_1, \cdots \Bp_n } \BOne + i \hatT \ket{ \Bk_A, \Bk_B },
\end{aligned}
\end{dmath}
where \( \hatS \) is called the S-matrix or scattering matrix, which is decomposed into a unit portion \( \BOne \) which is a convient way to exclude events with no scattering.  \( \BOne \) contributes for \( n = 2 \) only, but is an \( n \) scattering amplitude.
We are really interested in the \( i \hatT \) portion of this amplitude
\begin{equation}\label{eqn:qftLecture16:200}
\bra{ \Bp_1, \cdots \Bp_n } i \hatT \ket{ \Bk_A, \Bk_B }
=
(2 \pi)^4 \delta^4( \Bk_A + \Bk_B - \sum_{i = 1}^n \Bp_i )
\times
i M( \Bk_A + \Bk_B \rightarrow \Bp_1 \cdots \Bp_n ).
\end{equation}
This amounts to a definition of \( M \).
Recall that we found
\begin{dmath}\label{eqn:qftLecture16:220}
U(T, -T)
= T \lr{ e^{-i \int_{-T}^T H_I(t') dt'} }
=
e^{i H_0(T - t_0)}
e^{-i 2 H T}
e^{-i H_0(-T - t_0)}.
\end{dmath}
We want to replace the \( e^{-i 2 H T} \) in the matrix element above with \( U \).

In perturbation theory, we assume (conjecture) that
\begin{dmath}\label{eqn:qftLecture16:240}
\ket{ \Bk_A, \Bk_B }
\sim
\ket{ \Bk_A, \Bk_B }_\txto
\sim
\text{const}\, a^\dagger_{\Bk_A} a^\dagger_{\Bk_B} \ket{0}
\end{dmath}

Because we'll be squaring the amplitudes, we can assume that the \( e^{i H_0(T-t_0)} \) will result in just phase factors that won't survive, so in \cref{eqn:qftLecture16:180} we can insert \( U \)
\begin{dmath}\label{eqn:qftLecture16:280}
\prescript{}{\text{out}}{\braket{ \Bp_1, \cdots \Bp_n }{ \Bk_A, \Bk_B }}_{\text{in}}
=
\lim_{T \rightarrow \infty}
\bra{ \Bp_1, \cdots \Bp_n } U(T, -T) \ket{ \Bk_A, \Bk_B }
\end{dmath}

\begin{dmath}\label{eqn:qftLecture16:300}
\bra{ \Bp_1, \cdots \Bp_n } i\hatT { \Bk_A, \Bk_B }
=
\lim_{T \rightarrow \infty(1 - i \epsilon) }
\prescript{}{0}{
\bra{ \Bp_1, \cdots \Bp_n }
T( e^{-i \int_{-T}^T H_i(t') dt' } )
 \ket{ \Bk_A, \Bk_B }
}_0
\end{dmath}

These are connected and amputated graphs.

\paragraph{What is ``connected and amputated''?}

Explaining by example.  \( n = 2, \lambda \phi^4/4! \).

\begin{dmath}\label{eqn:qftLecture16:320}
\bra{0}
a_{\Bp_1}
a_{\Bp_2}
\lr{
\cancel{1}
- \frac{i \lambda}{4!} \int d^4 x \phi_I^4(x)
+ \inv{2} \lr{ \frac{i \lambda}{4!}}^2 \int d^4 x d^4 y \phi_I^4(x) \phi_I^4(y)
+ \cdots
}
a_{\Bk_A}^\dagger
a_{\Bk_B}^\dagger
\ket{0}
\end{dmath}
Here time ordering operations are implied, but not written explicitly.
Also, the ``amputated'' indicates that we are going to be dropping the \( 1 \) portion of the exponential expansion (as we've also dropped that in \cref{eqn:qftLecture16:300}).
We will also be using a relativistic normalization so that the \(
a_{\Bk_A}^\dagger
a_{\Bk_B}^\dagger \) terms include
\( \sqrt{
2 \omega_{\Bk_A}
2 \omega_{\Bk_B} } \)
contributions and the \(
a_{\Bp_1}
a_{\Bp_2}
\) include
\( \sqrt{
2 \omega_{\Bp_1}
2 \omega_{\Bp_2} } \) contributions.

\begin{dmath}\label{eqn:qftLecture16:340}
T
\contraction{}{\phi}{{}_I(x_1)}{\phi}
\phi_I(x_1)\phi_I(x_2)
= D_F(x_1 - x_2)
\end{dmath}

When we look at
\begin{dmath}\label{eqn:qftLecture16:360}
\contraction{}{\phi}{{}_I(x_1)}{a}
\phi_I(x_1) a^\dagger_{\Bk}
\sqrt{ 2 \omega_\Bk}
=
\int \frac{d^3 p}{(2 \pi)^3} \frac{e^{-i p \cdot x}}{\sqrt{2 \omega_\Bp}}
\contraction{}{a}{{}_\Bp}{a}
a_\Bp a^\dagger_\Bk
\sqrt{ 2 \omega_\Bk}
=
\int \frac{d^3 p}{(2 \pi)^3} \frac{e^{-i p \cdot x}}{\sqrt{2 \omega_\Bp}}
\delta^3(\Bp - \Bk)
\sqrt{ 2 \omega_\Bk}
= e^{-i k \cdot x}.
\end{dmath}
Similarily
\begin{dmath}\label{eqn:qftLecture16:n}
\contraction{}{a}{{}_\Bp}{\phi}
a_\Bp \phi_I(x_1)
\sqrt{ 2 \omega_\Bp}
=
\int \frac{d^3 k}{(2 \pi)^3} \frac{e^{i k \cdot x}}{\sqrt{2 \omega_\Bk}}
\contraction{}{a}{{}_\Bp}{a}
a_\Bp a^\dagger_\Bk
\sqrt{ 2 \omega_\Bk}
=
\int \frac{d^3 k}{(2 \pi)^3} \frac{e^{i k \cdot x}}{\sqrt{2 \omega_\Bk}}
\delta^3(\Bp - \Bk)
\sqrt{ 2 \omega_\Bk}
= e^{+i p \cdot x}.
\end{dmath}

Summarizing
\begin{dmath}\label{eqn:qftLecture16:380}
\begin{aligned}
\contraction{}{\phi}{{}_I(x_1)}{a}
\phi_I(x_1) a^\dagger_{\Bp}
&= e^{-i p \cdot x} \\
\contraction{}{a}{{}_\Bp}{\phi}
a_\Bp \phi_I(x_1)
&= e^{i p \cdot x}.
\end{aligned}
\end{dmath}

\begin{comment}
\section{junk}
\begin{dmath}\label{eqn:qftLecture15b:260}
%\sideset{_a^b}{'}{x}
\prescript{14}{2}{\mathbf{C}}
\end{dmath}
\begin{dmath}\label{eqn:qftLecture16:420}
\prescript{}{2}{\mathbf{C}}
\end{dmath}
\end{comment}

%}
%\EndArticle
\EndNoBibArticle

   \chapter{Scattering, decay, cross sections in a scalar theory.}
      %
% Copyright � 2017 Peeter Joot.  All Rights Reserved.
% Licenced as described in the file LICENSE under the root directory of this GIT repository.
%
%{
%%\input{../latex/blogpost.tex}
%%\renewcommand{\basename}{qftLecture17}
%%\renewcommand{\dirname}{notes/phy2403/}
%%\newcommand{\keywords}{PHY2403H}
%%\input{../latex/peeter_prologue_print2.tex}
%%
%%%\usepackage{phy2403}
%%\usepackage{peeters_braket}
%%%\usepackage{peeters_layout_exercise}
%%\usepackage{peeters_figures}
%%\usepackage{mathtools}
%%\usepackage{siunitx}
%%\usepackage{macros_cal} % LL
%%
%%\newcommand{\ultensor}[3]{{{#1}^{#2}}_{#3}}
%%\newcommand{\deltathree}[0]{\delta^{(3)}}
%%\newcommand{\deltafour}[0]{\delta^{(4)}}
%%
%%\beginArtNoToc
%%\generatetitle{PHY2403H Quantum Field Theory.  Lecture 17: Scattering, decay, cross sections in a scalar theory.  Taught by Prof.\ Erich Poppitz}
%\chapter{Scattering, decay, cross sections in a scalar theory.}
\label{chap:qftLecture17}
%%
%%%%Peeter's lecture notes from class.  These may be incoherent and rough.
%%%%
%%%%These are notes for the UofT course PHY2403H, Quantum Field Theory, taught by Prof. Erich Poppitz, covering \textchapref{{1}} \citep{peskin1995introduction} content.
%%
%%\paragraph{DISCLAIMER: Very rough notes from class, with some additional side notes.}
%%
%%These are notes for the UofT course PHY2403H, Quantum Field Theory, taught by Prof. Erich Poppitz, fall 2018.
%%%, covering \textchapref{{1}} \citep{peskin1995introduction} content.

\section{Review: S-matrix}

We defined an \( S-\)matrix
\begin{equation}\label{eqn:qftLecture17:20}
\bra{f} S \ket{i} = S_{fi} = \lr{ 2 \pi }^4 \deltafour \lr{ \sum \lr{p_i - \sum_{p_f} } } i M_{fi},
\end{equation}
where
\begin{equation}\label{eqn:qftLecture17:40}
i M_{fi} = \sum \text{ all connected amputated Feynman diagrams }.
\end{equation}
The matrix element \( \bra{f} S \ket{i} \) is the amplitude of the transition from the initial to the final state.
In general this can get very complicated, as the number of terms grows factorially with the order.

We also talked about decays.
\section{Scattering in a scalar theory}
Suppose that we have a scalar theory with a light field \( \Phi, M \) and a heavy field \( \varphi, m \), where \( m > 2 M \).  Perhaps we have an interaction with a \( z^2 \) symmetry so that the interaction potential is quadratic in \( \Phi \)
\begin{dmath}\label{eqn:qftLecture17:60}
V_{\text{int}} = \mu \varphi \Phi \Phi.
\end{dmath}
We may have \( \Phi \Phi \rightarrow \Phi \Phi \) scattering.

We will denote diagrams using a double line for \( \phi \) and a single line for \( \Phi \), as sketched in
\cref{fig:qftLecture17:qftLecture17Fig1}.
\imageFigure{../figures/phy2403-quantum-field-theory/qftLecture17Fig1}{Particle line convention.}{fig:qftLecture17:qftLecture17Fig1}{0.2}

There are three possible diagrams:
\imageThreeFiguresOneLine{../figures/phy2403-quantum-field-theory/qftLecture17Fig2a}{../figures/phy2403-quantum-field-theory/qftLecture17Fig2b}{../figures/phy2403-quantum-field-theory/qftLecture17Fig2c}{Possible diagrams.}{fig:qftLecture17:qftLecture17Fig2}{scale=0.3}
%\cref{fig:qftLecture17:qftLecture17Fig2a}.
%\imageFigure{../figures/phy2403-quantum-field-theory/qftLecture17Fig2a}{CAPTION: qftLecture17Fig2a}{fig:qftLecture17:qftLecture17Fig2a}{0.3}
%\cref{fig:qftLecture17:qftLecture17Fig2b}.
%\imageFigure{../figures/phy2403-quantum-field-theory/qftLecture17Fig2b}{CAPTION: qftLecture17Fig2b}{fig:qftLecture17:qftLecture17Fig2b}{0.3}
%\cref{fig:qftLecture17:qftLecture17Fig2c}.
%\imageFigure{../figures/phy2403-quantum-field-theory/qftLecture17Fig2c}{CAPTION: qftLecture17Fig2c}{fig:qftLecture17:qftLecture17Fig2c}{0.3}

The first we will call the s-channel, which has amplitude

\begin{dmath}\label{eqn:qftLecture17:80}
A(\text{s-channel}) \sim \frac{i}{p^2 - m^2 + i \epsilon} =
\frac{i}{s - m^2 + i \epsilon}
\end{dmath}

\begin{dmath}\label{eqn:qftLecture17:100}
(p_1 + p_2)^2 = s
\end{dmath}
In the centre of mass frame
\begin{dmath}\label{eqn:qftLecture17:120}
\Bp_1 = - \Bp_2,
\end{dmath}
so
\begin{dmath}\label{eqn:qftLecture17:140}
s = \lr{ p_1^0 + p_2^0 }^2 = E_{\text{cm}}^2.
\end{dmath}

To the next order we have a diagram like
\cref{fig:qftLecture17:qftLecture17Fig3}.
\imageFigure{../figures/phy2403-quantum-field-theory/qftLecture17Fig3}{Higher order.}{fig:qftLecture17:qftLecture17Fig3}{0.2}
and can have additional virtual particles created, with diagrams like \cref{fig:qftLecture17:qftLecture17Fig4}.
\imageFigure{../figures/phy2403-quantum-field-theory/qftLecture17Fig4}{More virtual particles.}{fig:qftLecture17:qftLecture17Fig4}{0.1}

We will see (QFT II) that this leads to an addition imaginary \( i \Gamma \) term in the propagator
\begin{dmath}\label{eqn:qftLecture17:160}
\frac{i}{s - m^2 + i \epsilon}
\rightarrow
\frac{i}{s - m^2 - i m \Gamma + i \epsilon}.
\end{dmath}
If we choose to zoom into the such a figure, as sketched in
\cref{fig:qftLecture17:qftLecture17Fig5},
we find that it contains the interaction of interest for our diagram, so we can
(looking forward to currently unknown material) know that our diagram also has such an imaginary \( i \Gamma \) term in its
propagator.
\imageFigure{../figures/phy2403-quantum-field-theory/qftLecture17Fig5}{Zooming into the diagram for a higher order virtual particle creation event.}{fig:qftLecture17:qftLecture17Fig5}{0.3}

Assuming such a term, the squared amplitude becomes
\begin{dmath}\label{eqn:qftLecture17:180}
\evalbar{\sigma}{s \text{near} m^2}
\sim
\Abs{A_s}^2 \sim \inv{(s - m)^2 + m^2 \Gamma^2}
\end{dmath}

This is called a resonance (name?), and is sketched in
\cref{fig:qftLecture17:qftLecture17Fig6}.
\imageFigure{../figures/phy2403-quantum-field-theory/qftLecture17Fig6}{Resonance.}{fig:qftLecture17:qftLecture17Fig6}{0.3}

Where are the poles of the modified propagator?

\begin{dmath}\label{eqn:qftLecture17:220}
\frac{i}{s - m^2 - i m \Gamma + i \epsilon}
=
\frac{i}{p_0^2 - \Bp^2 - m^2 - i m \Gamma + i \epsilon}
\end{dmath}

The pole is found, neglecting \( i \epsilon \), is found at
\begin{dmath}\label{eqn:qftLecture17:200}
p_0 = \sqrt{ \omega_\Bp^2 + i m \Gamma }
= \omega_\Bp \sqrt{ 1 + \frac{i m \Gamma }{\omega_\Bp^2} }
\approx \omega_\Bp + \frac{i m \Gamma }{2 \omega_\Bp}
\end{dmath}

\section{Decay rates.}

We have an initial state
\begin{dmath}\label{eqn:qftLecture17:240}
\ket{i} = \ket{k},
\end{dmath}
and final state
\begin{dmath}\label{eqn:qftLecture17:260}
\ket{f} = \ket{p_1^f, p_2^f \cdots p_n^f}.
\end{dmath}
We defined decay rate as the ratio of the number of initial particles to the number of final particles.

The probability is
\begin{dmath}\label{eqn:qftLecture17:280}
\rho \sim \Abs{\bra{f} S \ket{i}}^2
=
(2 \pi)^4 \deltafour( p_{\text{in}} - \sum p_f )
(2 \pi)^4 \deltafour( p_{\text{in}} - \sum p_f )
\times \Abs{ M_{fi} }^2
\end{dmath}

Saying that \( \delta(x) f(x) = \delta(x) f(0) \) we can set the argument of one of the delta functions to zero, which gives us a vacuum volume element factor
\begin{dmath}\label{eqn:qftLecture17:300}
(2 \pi)^4
\deltafour( p_{\text{in}} - \sum p_f )  =
(2 \pi)^4
\deltafour( 0 )
= V_3 T,
\end{dmath}
so
\begin{dmath}\label{eqn:qftLecture17:320}
\frac{\text{probability for \( i \rightarrow f\)}}{\text{unit time}}
\sim
(2 \pi)^4 \deltafour( p_{\text{in}} - \sum p_f )
V_3
\times \Abs{ M_{fi} }^2
\end{dmath}

\begin{dmath}\label{eqn:qftLecture17:340}
\braket{\Bk}{\Bk} = 2 \omega_\Bk V_3
\end{dmath}

coming from

\begin{dmath}\label{eqn:qftLecture17:360}
\braket{k}{p} = (2 \pi)^3 2 \omega_\Bp \deltathree(\Bp - \Bk)
\end{dmath}
so
\begin{dmath}\label{eqn:qftLecture17:380}
\braket{k}{k} = 2 \omega_\Bp V_3
\end{dmath}

\begin{dmath}\label{eqn:qftLecture17:400}
\frac{\text{probability for \(i \rightarrow f\)}}{\text{unit time}}
\sim
\frac{
(2 \pi)^4 \deltafour( p_{\text{in}} - \sum p_f )
\Abs{ M_{fi} }^2 V_3
}
{
2 \omega_\Bk V_3
2 \omega_{\Bp_1}
\cdots
2 \omega_{\Bp_n} V_3^n
}
\end{dmath}

If we multiply the number of final states with \( p_i^f \in (p_i^f, p_i^f + dp_i^f) \) for a particle in a box
\begin{dmath}\label{eqn:qftLecture17:420}
p_x = \frac{ 2 \pi n_x}{L}
\end{dmath}

\begin{dmath}\label{eqn:qftLecture17:440}
\Delta p_x = \frac{ 2 \pi }{L} \Delta n_x
\end{dmath}

\begin{dmath}\label{eqn:qftLecture17:460}
\Delta n_x
=
\frac{L}{2 \pi} \Delta p_x
\end{dmath}

and

\begin{dmath}\label{eqn:qftLecture17:480}
\Delta n_x
\Delta n_y
\Delta n_z
= \frac{V_3}{(2 \pi)^3 }
\Delta p_x
\Delta p_y
\Delta p_z
\end{dmath}

\begin{dmath}\label{eqn:qftLecture17:500}
\Gamma
=
\frac{\text{number of events \( i \rightarrow f \)}}{\text{unit time}}
=
\prod_{f} \frac{ d^3 p}{(2 \pi)^3 2 \omega_{\Bp^f} }
 \frac{ (2 \pi)^4 \deltafour( k - \sum_f p^f ) \Abs{M_{fi}}^2 }
{
2 \omega_{\Bk}
}
\end{dmath}

Note that everything here is Lorentz invariant except for the denominator of the second term ( \(2 \omega_{\Bk}\)).  This is a well known result (the decay rate changes in different frames).

\section{Cross section.}

For \( 2 \rightarrow \text{many} \) transitions

\begin{dmath}\label{eqn:qftLecture17:520}
\frac{\text{probability \( i \rightarrow f \)}}{\text{unit time}}
\times \lr{
\text{ number of final states with \( p_f \in (p_f, p_f + dp_f) \)
}
}
=
 \frac{ (2 \pi)^4 \deltafour( \sum p_i - \sum_f p^f ) \Abs{M_{fi}}^2 \cancel{V_3} }
{
2 \omega_{\Bk_1} V_3
2 \omega_{\Bk_2} \cancel{V_3 }
}
\prod_{f} \frac{ d^3 p}{(2 \pi)^3 2 \omega_{\Bp^f} }
\end{dmath}

We need to divide by the flux

In the CM frame, as sketched in
\cref{fig:qftLecture17:qftLecture17Fig7},
the current is
\begin{dmath}\label{eqn:qftLecture17:540}
\Bj = n \Bv_1 - n \Bv_2,
\end{dmath}
so if the density is
\begin{dmath}\label{eqn:qftLecture17:560}
n = \inv{V_3},
\end{dmath}
(one particle in \(V_3\)), then
\begin{dmath}\label{eqn:qftLecture17:580}
\Bj = \frac{\Bv_1 - \Bv_2}{V_3}.
\end{dmath}
\imageFigure{../figures/phy2403-quantum-field-theory/qftLecture17Fig7}{Centre of mass frame.}{fig:qftLecture17:qftLecture17Fig7}{0.1}

This is where \citep{peskin1995introduction} stop,
\begin{dmath}\label{eqn:qftLecture17:640}
\sigma
=
 \frac{ (2 \pi)^4 \deltafour( \sum p_i - \sum_f p^f ) \Abs{M_{fi}}^2 \cancel{V_3} }
{
2 \omega_{\Bk_1}
2 \omega_{\Bk_2}
\Abs{\Bv_1 - \Bv_2}
}
\prod_{f} \frac{ d^3 p}{(2 \pi)^3 2 \omega_{\Bp^f} }
\end{dmath}

There is, however, a nice Lorentz invariant generalization
\begin{dmath}\label{eqn:qftLecture17:600}
j = \inv{ V_3 \omega_{k_A} \omega_{k_B}} \sqrt{ (k_A - k_B)^2 - m_A^2 m_B^2 }
\end{dmath}

(Claim: DIY)
\begin{dmath}\label{eqn:qftLecture17:620}
\evalbar{j}{CM} =
\inv{V_3}
\lr{
   \frac{\Abs{\Bk}}{\omega_{k_A}}
   +
   \frac{\Abs{\Bk}}{\omega_{k_B}}
}
=
\inv{V_3} \lr{ \Abs{\Bv_A} + \Abs{\Bv_B} }
=
\inv{V_3} \Abs{\Bv_1 - \Bv_2 }
\end{dmath}

\begin{dmath}\label{eqn:qftLecture17:660}
\sigma
=
 \frac{ (2 \pi)^4 \deltafour( \sum p_i - \sum_f p^f ) \Abs{M_{fi}}^2 \cancel{V_3} }
{
4 \sqrt{ (k_A - k_B)^2 - m_A^2 m_B^2 }
}
\prod_{f} \frac{ d^3 p}{(2 \pi)^3 2 \omega_{\Bp^f} }.
\end{dmath}

%}
%\EndArticle


   \chapter{Problem Set 1.}

      %
% Copyright � 2018 Peeter Joot.  All Rights Reserved.
% Licenced as described in the file LICENSE under the root directory of this GIT repository.
%
\makeoproblem{
%Back to classics: relativistic electrodynamics and variational principle.
Electrodynamics and the variational principle.
}{qft:problemSet1:1}{2018 Hw1.I}{
Given the action
In terms of the four-vector potential \(A\), the Lagrangian density of the electromagnetic field,
interacting with a charged particle of mass m can be written as follows:
\index{Lagrangian density!electromagnetic}
\begin{dmath}\label{eqn:ProblemSet1Problem1:20}
S =
\int_{\text{all spacetime}}
d^4 x
\lr{
-\inv{4}
F_{\mu\nu}
F^{\mu\nu}
- A_\mu j^\mu
}
- m
\int_{\text{worldline}}
ds.
\end{dmath}
\index{field strength tensor}
Here, \( F_{\mu\nu} \equiv \partial_\mu A_\nu - \partial_\nu A_\mu \) field strength tensor. The current \( j^\mu \) is the current corresponding to the particle which can be written as:
\begin{dmath}\label{eqn:ProblemSet1Problem1:40}
j^\mu(x) = e
\int_{\text{worldline}}
dX^\mu (\tau) \deltafour ( x - X(\tau) ),
\end{dmath}
where \( \deltafour(x) \) is a four-dimensional delta function.  All indices are raised and lowered by means of the
metric tensor \( g_{\mu\nu} \) and its inverse \( g^{\mu\nu} \).

The last term in \cref{eqn:ProblemSet1Problem1:20} is the relativistic kinetic energy of the particle and the integral is over the particle's worldline, $X^\mu(\tau)$. Note that $\tau$ is a parameter used to describe the particle's location along the worldline. One can  take this parameter be equal to $x^0$, so that  $X^\mu(\tau)$ means ($X^0 = x^0$, $X^i = X^i(x^0)$), where $\BX(x^0)$ is simply the trajectory of the particle (such a choice of parametrization can be useful, but is not required).
Notice also that the term involving the current in \cref{eqn:ProblemSet1Problem1:20}, after substitution of \cref{eqn:ProblemSet1Problem1:40} simply becomes
\begin{dmath}\label{eqn:ProblemSet1Problem1:860}
- e \int\limits_{worldline} d X^\mu(\tau) A_\mu(X(\tau))~  ,
\end{dmath}
which is the usual   coupling of a charged particle to the electromagnetic field (choose the $\tau = x^0$ parameterization of the worldline to see this). Whether you use this form of the one of \cref{eqn:ProblemSet1Problem1:20} depends on the problem you're solving (this is a hint).

The dynamical degrees of freedom in the action \cref{eqn:ProblemSet1Problem1:20} are the four-vector potential $A_\mu$ and the particle position $X^\mu(\tau)$.
\makesubproblem{}{qft:problemSet1:1a}
Use the identification \( A^0 = \phi \),
the scalar potential, and \((A^1,A^2,A^3) = \BA\), the vector potential,
to convince yourself that
\( F_{01} = E_x, F_{02} = E_y, F_{03} = E_z\), and that \(F_{12} = - B_z, F_{31} = -B_y, F_{23} = - B_x \).
\makesubproblem{}{qft:problemSet1:1b}
Prove the identity
\begin{dmath}\label{eqn:ProblemSet1Problem1:120}
\epsilon^{\mu\nu\alpha\beta} \partial_\nu F_{\alpha \beta} = 0,
\end{dmath}
and use this to show that
the source free Maxwell's equations can be recovered directly from the
definition of \( F_{ij} \).
\makesubproblem{}{qft:problemSet1:1c}
Write the Euler-Lagrange equations obtained when varying
\cref{eqn:ProblemSet1Problem1:20}
with respect to \( A_\mu \).  Show
that they can be cast in terms of the field strength tensor \( F \) and \( j \). Note that when varying
with respect to \( A_\mu \), the current is kept fixed. Using the \( \BE \) and \( \BB \) fields as the appropriate
components of \( F \), show that the Euler-Lagrange equations for \( A_\mu \)
 from
\cref{eqn:ProblemSet1Problem1:20}
reduce to the
Maxwell equations familiar to you from electrodynamics.
\makesubproblem{}{qft:problemSet1:1d}
Finally, write the Euler-Lagrange equation varying with respect to the worldline of the particle.
Show that they give \( m dU^\mu/ds = e F^{\mu\nu} U_\nu \), where \( U^\mu = dX^\mu/ds \) is the four velocity of the
particle and \( F \) is, of course, taken at the particle's position. Convince yourself that this is
the relativistic Lorentz force equation.
\index{Lorentz force equation}
\index{Maxwell's equations}

\paragraph{The} point of this problem is to make sure you remember/learn how the action principle works in  electrodynamics. The two coupled equations, obtained by varying w.r.t. $A_\mu$ and $X^\mu$ complete the equations of classical electrodynamics. Feel free to use \citep{landau1980classical}, or \citep{poppitzphy450} while solving this problem.
} % makeproblem

\makeanswer{qft:problemSet1:1}{
\withproblemsetsParagraph{
\makeSubAnswer{}{qft:problemSet1:1a}
With \( k = \setlr{1, 2,3} \),
\begin{dmath}\label{eqn:ProblemSet1Problem1:60}
\sum_k F_{0k} \Be_k
= \sum_k \lr{ \partial_0 A_k - \partial_k A_0 } \Be_k
= - \sum_k \lr{ \PD{t}{A^k} - \PD{x^k}{\phi} } \Be_k
= - \PD{t}{\BA} - \spacegrad \phi
= \BE.
\end{dmath}
which is the conventional scalar, plus vector potential definition of the electric field in natural units.
For the magnetic field, it's easier to work backwards
\begin{dmath}\label{eqn:ProblemSet1Problem1:80}
\BB
= \spacegrad \cross \BA
= \epsilon_{ijk} \partial_i A^j \Be_k,
\end{dmath}
or, for each cyclic permutation of \( i j k = \setlr{1,2,3}\)
\begin{dmath}\label{eqn:ProblemSet1Problem1:100}
B^i
= \partial_j A^k - \partial_k A^j
= -\partial_j A_k + \partial_k A_j
= F_{kj}
= -F_{jk},
\end{dmath}

\makeSubAnswer{}{qft:problemSet1:1b}
To prove \cref{eqn:ProblemSet1Problem1:120}, we use explicit expansion and an index exchange
\begin{dmath}\label{eqn:ProblemSet1Problem1:140}
=
\epsilon^{\mu\nu\alpha\beta} \partial_\nu \lr{ \partial_\alpha A_\beta - \partial_\beta A_\alpha}
=
\epsilon^{\mu\nu\alpha\beta} \partial_\nu \partial_\alpha A_\beta
-\epsilon^{\mu\nu\beta\alpha} \partial_\nu \partial_\beta A_\alpha
=
2 \epsilon^{\mu\nu\alpha\beta} \partial_\nu \partial_\alpha A_\beta,
\end{dmath}
but because the partials are symmetric in \( \nu \alpha \) (assuming sufficient continuity of the fields components), and because the sum is antisymmetric in the same indexes, the result is zero as claimed.

Expanding \cref{eqn:ProblemSet1Problem1:120} explicitly for \( \nu = 0 \), we find Gauss's law for the magnetic field
\begin{dmath}\label{eqn:ProblemSet1Problem1:160}
0
=
\epsilon^{ijk} \partial_i F_{jk}
=
-\partial_i B^i
= -\spacegrad \cdot \BB,
\end{dmath}
For \( \nu = 1 \) % 2 3 0
\begin{dmath}\label{eqn:ProblemSet1Problem1:180}
0 = \partial_2 F_{30} + \partial_3 F_{02} + \partial_0 F_{23}
= -\partial_2 E^3 + \partial_3 E^2 - \PD{t}{B^1}
= - (\spacegrad \cross \BE)_x - \PD{t}{B_x},
\end{dmath}
and for \( \nu = 2 \) % 3 0 1
\begin{dmath}\label{eqn:ProblemSet1Problem1:300}
0 = \partial_3 F_{01} + \partial_0 F_{13} + \partial_1 F_{30}
= \partial_3 E^1
+ \PD{t}{B^2}
- \partial_1 E^3
= (\spacegrad \cross \BE)_y + \PD{t}{B_y},
\end{dmath}
and for \( \nu = 3 \) % 0 1 2
\begin{dmath}\label{eqn:ProblemSet1Problem1:200}
0 = \partial_0 F_{12} + \partial_1 F_{20} + \partial_2 F_{01}
=
- \PD{t}{B^3}
- \partial_1 E^2
+ \partial_2 E^1
= - (\spacegrad \cross \BE)_z - \PD{t}{B_z},
\end{dmath}
so
\begin{dmath}\label{eqn:ProblemSet1Problem1:220}
0 = \spacegrad \cross \BE + \PD{t}{\BB},
\end{dmath}
which is Faraday's law.

\makeSubAnswer{}{qft:problemSet1:1c}
For the source dependent Maxwell's equations we vary the action.
Recall that for a single field Lagrangian density \( \LL = \LL(\phi, \partial_\mu \phi) \) the variation of the action \( S = \int \LL \) can be found by Taylor expansion
\begin{dmath}\label{eqn:ProblemSet1Problem1:240}
\delta S
= \int d^4 x \delta \LL
= \int d^4 x \PD{\phi}{\LL} \delta \phi + \int d^4 x \PD{(\partial_\nu \phi)}{\LL} \delta (\partial_\nu \phi)
= \int d^4 x \PD{\phi}{\LL} \delta \phi + \int d^4 x \PD{(\partial_\nu \phi)}{\LL} \partial_\nu \delta \phi
=
  \int d^4 x \PD{\phi}{\LL} \delta \phi
+ \int d^4 x \partial_\nu \lr{ \PD{(\partial_\nu \phi)}{\LL} \delta \phi }
- \int d^4 x \partial_\nu \lr{ \PD{(\partial_\nu \phi)}{\LL} } \delta \phi
=
  \int d^4 x \delta \phi \lr{
\PD{\phi}{\LL}
- \partial_\nu \lr{ \PD{(\partial_\nu \phi)}{\LL} }
}
\end{dmath}
Assuming that \( \delta \phi \) is stationary at the boundaries killed the second integral in the second last step.  Setting \( \delta S = 0 \) gives the Euler-Lagrange equations for a Lagrangian density that is dependent on a single field and its first derivatives
\begin{dmath}\label{eqn:ProblemSet1Problem1:260}
0 =
\PD{\phi}{\LL}
- \partial_\nu \lr{ \PD{(\partial_\nu \phi)}{\LL} }.
\end{dmath}
For a multiple particle field we must Taylor expand around each field variable, so we have one equation for each field
\begin{dmath}\label{eqn:ProblemSet1Problem1:280}
0 =
\PD{A_\mu}{\LL}
- \partial_\nu \lr{ \PD{(\partial_\nu A_\mu)}{\LL} }.
\end{dmath}
We wish to apply \cref{eqn:ProblemSet1Problem1:280} to the field Lagrangian density
\begin{dmath}\label{eqn:ProblemSet1Problem1:320}
\LL =
-\inv{4}
F_{\mu\nu}
F^{\mu\nu}
- A_\mu j^\mu,
\end{dmath}
and vary with respect to the fields \( A_{\mu} \) (or \( A^{\mu} \)).

The first order partials are trivial
\begin{dmath}\label{eqn:ProblemSet1Problem1:340}
\PD{A_\mu}{\LL}
=
- j^\mu,
\end{dmath}
but we have to do a bit more work for the rest
\begin{dmath}\label{eqn:ProblemSet1Problem1:360}
\PD{(\partial_\nu A_\mu)}{\LL}
=
-\inv{2}
F^{\alpha\beta} \PD{(\partial_\nu A_\mu)}{} F_{\alpha\beta}
=
-\inv{2}
F^{\alpha\beta} \PD{(\partial_\nu A_\mu)}{} \lr{
\partial_\alpha A_\beta -
\partial_\beta A_\alpha}
=
-\inv{2}
F^{\nu\mu}
+
\inv{2}
F^{\mu\nu}
=
F^{\mu\nu}.
\end{dmath}
Putting the pieces together, we have
\begin{dmath}\label{eqn:ProblemSet1Problem1:380}
0 = -j^\mu - \partial_\nu F^{\mu\nu},
\end{dmath}
or
%\begin{boxed}\label{eqn:ProblemSet1Problem1:400}
\boxedEquation{eqn:ProblemSet1Problem1:420}{
\partial_\mu F^{\mu\nu} = j^\nu.
}
%\end{boxed}

For \( \nu = 0 \) this is
\begin{dmath}\label{eqn:ProblemSet1Problem1:440}
\partial_\mu F^{\mu 0} = j^0,
\end{dmath}
or
\begin{dmath}\label{eqn:ProblemSet1Problem1:460}
\rho
=
\partial_k F^{k 0}
=
-\partial_k F_{k 0}
=
\partial_k F_{0 k}
=
\spacegrad \cdot \BE,
\end{dmath}
which is Gauss's law.

% for the other indexes \( \nu \) we have
\begin{dmath}\label{eqn:ProblemSet1Problem1:480}
j^1
=
\partial_\mu F^{\mu 1}
=
\partial_0 F^{0 1}
+
\partial_2 F^{2 1}
+
\partial_3 F^{3 1}
=
- \PD{t}{E_x}
+ \partial_2 B_z
- \partial_3 B_y
= \lr{ -\BE + \spacegrad \cross \BB } \cdot \Be_1
\end{dmath}
\begin{dmath}\label{eqn:ProblemSet1Problem1:500}
j^2
=
\partial_\mu F^{\mu 2}
=
\partial_3 F^{3 2}
+
\partial_0 F^{0 2}
+
\partial_1 F^{1 2}
=
  \partial_3 B_x
- \PD{t}{E_y}
- \partial_1 B_z
= \lr{ -\BE + \spacegrad \cross \BB } \cdot \Be_2
\end{dmath}
\begin{dmath}\label{eqn:ProblemSet1Problem1:520}
j^3
=
\partial_\mu F^{\mu 3}
=
\partial_0 F^{0 3}
+
\partial_1 F^{1 3}
+
\partial_2 F^{2 3}
=
- \PD{t}{E_z}
+ \partial_1 B_y
- \partial_2 B_x
= \lr{ -\BE + \spacegrad \cross \BB } \cdot \Be_3,
\end{dmath}
so
\begin{dmath}\label{eqn:ProblemSet1Problem1:540}
\BJ = -\PD{t}{\BE} + \spacegrad \cross \BB,
\end{dmath}
which recovers the Ampere-Maxwell equation.

\makeSubAnswer{}{qft:problemSet1:1d}
The portion of the action that is dependent on the worldline is
%in terms of a parameterization \( X(s) \) is
\begin{dmath}\label{eqn:ProblemSet1Problem1:560}
S =
%\int_{\text{worldline}} ds \lr{ - m - e A_\mu \frac{dX^\mu}{ds} }
\int_{\text{worldline}} \lr{ - m ds - e A_\mu dX^\mu }
\end{dmath}

Let's consider the variation of each of these terms separately, starting with \( \delta ds \)
\begin{dmath}\label{eqn:ProblemSet1Problem1:580}
\delta \int ds
=
\delta \int \sqrt{ dX^\mu dX_\mu }
=
\int \inv{2 ds} 2 dX^\mu \delta dX_\mu
=
\int \frac{dX^\mu}{ds} d \delta X_\mu
=
\int d \lr{ \frac{dX^\mu}{ds} \delta X_\mu } - d \lr{ \frac{dX^\mu}{ds} } \delta X_\mu
=
\evalbar{ \frac{dX^\mu}{ds} \delta X_\mu }{\Delta s} - \int d \lr{ \frac{dX^\mu}{ds} } \delta X_\mu.
\end{dmath}
The endpoints of the worldline are presumed to be stationary, which kills the boundary term, leaving just
\begin{dmath}\label{eqn:ProblemSet1Problem1:600}
\delta \int ds = - \int d U^\mu \delta X_\mu.
\end{dmath}
Now let's compute the variation of the potential term
\begin{dmath}\label{eqn:ProblemSet1Problem1:620}
\delta \int A_\mu dX^\mu
=
\int (\delta A_\mu) dX^\mu
+
\int A_\mu \delta dX^\mu
=
\int \partial_\nu A_\mu \delta X^\nu dX^\mu
-
\int d A_\mu \delta X^\mu
=
\int \partial_\nu A_\mu \delta X^\nu U^\mu ds
-
\int \partial_\nu A_\mu dX^\nu \delta X^\mu
=
\int \lr{ \partial_\nu A_\mu U^\mu \delta X^\nu
-
\partial_\nu A_\mu U^\nu \delta X^\mu
}
ds
=
\int \lr{ \partial_\nu A_\mu - \partial_\mu A_\nu } U^\mu \delta X^\nu ds
=
\int F_{\nu\mu} U^\mu \delta X^\nu ds
=
\int F^{\nu\mu} U_\mu \delta X_\nu ds.
\end{dmath}
Here the boundary term has been dropped again after integration by parts, and an index switcheroo was done to factor out a common
\( U^\mu \delta X^\nu ds \) term from the integrand, and we finish off with a set of raising and lowering operations on all the matched indexes.  Putting the pieces back together we have
\begin{dmath}\label{eqn:ProblemSet1Problem1:640}
\delta S
=
  \int
\lr{
-m \dot{U}^\nu
-
e
F^{\nu\mu} U_\mu
}
\delta X_\nu ds
=
  \int
\lr{
m \dot{U}^\mu
-
e
F^{\mu\nu} U_\nu
}
\delta X_\mu ds
.
\end{dmath}
Requiring \( \delta S = 0 \) for all worldline path variations \( \delta X_\mu \) means that the equations of motion are
%\begin{dmath}\label{eqn:ProblemSet1Problem1:660}
\boxedEquation{eqn:ProblemSet1Problem1:660}{
m \frac{dU^\mu}{ds}
=
e
F^{\mu\nu} U_\nu,
}
%\end{dmath}
as expected.

To unpack this and obtain the conventional Lorentz force equation we need to relate the proper time derivatives to the time of a stationary observer
\begin{dmath}\label{eqn:ProblemSet1Problem1:680}
\frac{d}{ds} =
\frac{dt}{ds}
\frac{d}{dt},
% = (1, \Bv) \frac{dt}{ds},
\end{dmath}
The stationary observer's world line is \( X^\mu = (t, \Bx) \), and the spacetime interval on that worldline is
\begin{dmath}\label{eqn:ProblemSet1Problem1:700}
ds^2 = dt^2 - d\Bx^2,
\end{dmath}
or
\begin{equation}\label{eqn:ProblemSet1Problem1:720}
\lr{\frac{ds}{dt}}^2 = 1 - {\frac{dx}{dt}}^2 = 1 - \Bv^2.
\end{equation}
\Cref{eqn:ProblemSet1Problem1:680} can now be written as
\begin{equation}\label{eqn:ProblemSet1Problem1:740}
\frac{d}{ds} =
\inv{\sqrt{ 1 - \Bv^2 }}
\frac{d}{dt}
\equiv \gamma
\frac{d}{dt}.
\end{equation}
In particular, the proper velocity is
\begin{dmath}\label{eqn:ProblemSet1Problem1:760}
U^\mu = \gamma \lr{ 1, \Bv }.
\end{dmath}

First inserting \( \mu = 0 \) into \cref{eqn:ProblemSet1Problem1:660} now gives
\begin{dmath}\label{eqn:ProblemSet1Problem1:780}
\frac{d}{ds} \frac{m}{\sqrt{1 - \Bv^2}}
= e F^{0 k} U_k
= (-1)^2 e F_{0 k} U^k
= e \BE \cdot \Bv \gamma,
\end{dmath}
or
\begin{dmath}\label{eqn:ProblemSet1Problem1:800}
\frac{d}{dt} \frac{m}{\sqrt{1 - \Bv^2}} = e \BE \cdot \Bv.
\end{dmath}
This is the timelike portion of the Lorentz force equation in non-covariant form and natural units (cf. \citep{landau1980classical} eq. (17.7).)

For the \( \mu \ne 0 \) case, we find
\begin{dmath}\label{eqn:ProblemSet1Problem1:820}
\gamma \frac{d}{dt} \frac{m \Bv}{\sqrt{1 - \Bv^2}}
= e F^{j \nu} U_\nu \Be_j
= e F^{j 0} \Be_j
- e \sum_{1 \le (j \ne k) \le 3} F^{j k} v^k \Be_j \gamma
= e \BE + e \epsilon_{jki} B^i v^k \Be_j \gamma
= e \BE + e \Bv \cross \BB \gamma,
\end{dmath}
or
\begin{dmath}\label{eqn:ProblemSet1Problem1:840}
\frac{d\Bp}{dt} = e \BE + e \Bv \cross \BB,
\end{dmath}
which is the Lorentz force equation in natural units in terms of \( \Bp = d(\gamma m \Bv)/dt \), the relativistically correct momentum from the viewpoint of a stationary observer.
=
}}

      %
% Copyright � 2018 Peeter Joot.  All Rights Reserved.
% Licenced as described in the file LICENSE under the root directory of this GIT repository.
%
%{
\makeoproblem{Complex scalar field.}{qft:problemSet1:2}{2018 Hw1.II (from \citep{peskin1995introduction} pr. 2.2)}{
Consider a complex scalar field with action
\begin{equation}\label{eqn:ProblemSet1Problem2:1400}
S = \int d^4x\lr{\partial_\mu \phi^\dagger \partial^\mu \phi - m^2 \phi^\dagger \phi}. 
\end{equation}
When doing the variational principle consider \( \phi \) and \(\phi^\dagger \) as independent, rather than their real and imaginary parts (this is equivalent, but more convenient).

\makesubproblem{}{qft:problemSet1:2a}
Show that \( H = \int d^3x \lr{ \pi^\dagger \pi + \spacegrad \phi^\dagger \cdot \spacegrad \phi + m^2 \phi^\dagger \phi } \) and that the Klein-Gordon equation is obeyed by \( \phi \) and \( \phi^\dagger\).
\makesubproblem{}{qft:problemSet1:2b}
Introduce complex amplitudes, diagonalize the Hamiltonian, and quantize the theory. Show that the theory has now two sets of particles.
\makesubproblem{}{qft:problemSet1:2c}
Write the charge conserved due to the global \( U(1) \) symmetry,
\begin{equation}\label{eqn:ProblemSet1Problem2:580}
Q = \int d^3 x \frac{i}{2} \lr{ \phi^\dagger \pi^\dagger - \pi \phi },
\end{equation}
in
terms of creation and annihilation operators and find the charge of the particles of each type.
\index{complex scalar field}
} % makeproblem

\makeanswer{qft:problemSet1:2}{
\withproblemsetsParagraph{
\makeSubAnswer{}{qft:problemSet1:2a}
Classically, evaluating the Euler-Lagrange equations gives us
\begin{equation}\label{eqn:ProblemSet1Problem2:20}
\begin{aligned}
\PD{\phi}{\LL} &= -m^2 \phi^\dagger \\
\PD{(\partial_\mu \phi)}{\LL} &= \partial^\mu \phi^\dagger \\
\PD{\phi^\dagger}{\LL} &= -m^2 \phi       \\
\PD{(\partial_\mu \phi^\dagger)}{\LL} &= \partial^\mu \phi,
\end{aligned}
\end{equation}
so the equations of the field are respectively
\begin{equation}\label{eqn:ProblemSet1Problem2:40}
%\boxedEquation{eqn:ProblemSet1Problem2:40}{
\begin{aligned}
\partial_\mu \partial^\mu \phi^\dagger &=  -m^2 \phi^\dagger \\
\partial_\mu \partial^\mu \phi &=  -m^2 \phi.
\end{aligned}
%}
\end{equation}
These are Klein-Gordon equations for each field variable \( \phi, \phi^\dagger \) as expected, although this can be made more explicit written out explicitly in the stationary observer frame
\boxedEquation{eqn:ProblemSet1Problem2:340}{
\begin{aligned}
\lr{ \partial_{tt} - \spacegrad^2 + m^2 } \phi^\dagger &= 0 \\
\lr{ \partial_{tt} - \spacegrad^2 + m^2 } \phi &= 0 \\
\end{aligned}
}
To find the Hamiltonian, note that the Lagrangian density written out explicitly is
\begin{equation}\label{eqn:ProblemSet1Problem2:60}
\LL = \partial_0 \phi^\dagger \partial_0 \phi - (\spacegrad \phi^\dagger) \cdot (\spacegrad \phi) - m^2 \phi^\dagger \phi,
\end{equation}
so the conjugate momentum densities are
\begin{equation}\label{eqn:ProblemSet1Problem2:80}
\begin{aligned}
\pi(\Bx, t) &= \PD{(\partial_0 \phi)}{\LL} = \partial_0 \phi^\dagger \\
\pi^\dagger(\Bx, t) &= \PD{(\partial_0 \phi^\dagger)}{\LL} = \partial_0 \phi \\
\end{aligned}
\end{equation}

The Hamiltonian (including a ``\(p \dot{q}\)'' term for each of \( \phi, \phi^\dagger \)) is
\begin{equation}\label{eqn:ProblemSet1Problem2:100}
H
= \int d^3 x \lr{ \pi \partial_0 \phi + \pi^\dagger \partial_0 \phi^\dagger - \LL }
=
\int d^3 x \lr{ \pi \pi^\dagger + \pi^\dagger \pi - \pi \pi^\dagger +
(\spacegrad \phi^\dagger) \cdot (\spacegrad \phi) + m^2 \phi^\dagger \phi
 }
=
\int d^3 x \lr{ \pi^\dagger \pi +
(\spacegrad \phi^\dagger) \cdot (\spacegrad \phi) + m^2 \phi^\dagger \phi
 }
\end{equation}
\makeSubAnswer{}{qft:problemSet1:2b}
To canonically quantize the fields, we promote the fields to operators, demand that we have commutators for conjugate pairs of operators
\begin{equation}\label{eqn:ProblemSet1Problem2:120}
\antisymmetric{\phi(\Bx)}{\pi(\By)}
=
\antisymmetric{\phi^\dagger(\Bx)}{\pi^\dagger(\By)}
= i \deltathree(\Bx - \By),
\end{equation}
and require all the other operator pairs \( \phi \phi^\dagger, \pi \pi^\dagger, \phi^\dagger\pi, \phi\pi^\dagger \) commute.

Before diagonalizing the Hamiltonian, let's verify that applying computing Hamilton's equations using such quantized operators recovers the Klein-Gordon equations we expect.
\begin{subequations}
\label{eqn:ProblemSet1Problem2:700}
\begin{equation}\label{eqn:ProblemSet1Problem2:720}
\PD{t}{\phi}(\Bx, t)
= i \antisymmetric{H}{\phi(\Bx)}
=
i \int d^3 y \lr{
   \antisymmetric{\pi^\dagger(\By) \pi(\By)}{\phi(\Bx)}
   +
   \cancel{\antisymmetric{ \spacegrad_\By \phi^\dagger(\By) \cdot \spacegrad_\By \phi(\By) }{\phi(\Bx)}}
   +
   \cancel{\antisymmetric{ \phi^\dagger(\By) \phi(\By)  }{\phi(\Bx)}}
}
=
i \int d^3 y \pi^\dagger(\By) \antisymmetric{\pi(\By)}{\phi(\Bx)}
=
i \int d^3 y \pi^\dagger(\By) (-i) \deltathree(\By - \Bx)
=
\pi^\dagger(\Bx)
\end{equation}
\begin{equation}\label{eqn:ProblemSet1Problem2:740}
\PD{t}{\phi^\dagger}(\Bx, t)
= i \antisymmetric{H}{\phi^\dagger(\Bx)}
=
i \int d^3 y \lr{
   \antisymmetric{\pi^\dagger(\By) \pi(\By)}{\phi^\dagger(\Bx)}
   +
   \cancel{\antisymmetric{ \spacegrad_\By \phi^\dagger(\By) \cdot \spacegrad_\By \phi(\By) }{\phi^\dagger(\Bx)}}
   +
   m^2 \cancel{\antisymmetric{ \phi^\dagger(\By) \phi(\By)  }{\phi^\dagger(\Bx)}}
}
=
i \int d^3 y \pi(\By) \antisymmetric{\pi^\dagger(\By)}{\phi^\dagger(\Bx)}
=
i \int d^3 y \pi(\By) (-i) \deltathree(\By - \Bx)
=
\pi(\Bx)
\end{equation}
\begin{equation}\label{eqn:ProblemSet1Problem2:760}
\PD{t}{\pi}(\Bx, t)
=
i \antisymmetric{H}{\pi(\Bx)}
=
i \int d^3 y \lr{
   \cancel{\antisymmetric{\pi^\dagger(\By) \pi(\By)}{\pi(\Bx)}}
   +
   \antisymmetric{ \spacegrad_\By \phi^\dagger(\By) \cdot \spacegrad_\By \phi(\By) }{\pi(\Bx)}
   +
   m^2 \antisymmetric{ \phi^\dagger(\By) \phi(\By)  }{\pi(\Bx)}
}
=
i \int d^3 y \lr{
   \spacegrad_\By \phi^\dagger(\By) \cdot \spacegrad_\By \antisymmetric{ \phi(\By) }{\pi(\Bx)}
   +
   m^2 \phi^\dagger(\By) \antisymmetric{ \phi(\By)  }{\pi(\Bx)}
}
=
i \int d^3 y \lr{
   \spacegrad_\By \phi^\dagger(\By) \cdot \spacegrad_\By (i \deltathree(\By - \Bx))
   +
   m^2 \phi^\dagger(\By) i \deltathree(\By - \Bx)
}
=
- \int d^3 y \lr{
   \spacegrad_\By \cdot \lr{ \deltathree(\By - \Bx) \spacegrad_\By \phi^\dagger(\By) }
   -
   \deltathree(\By - \Bx) \spacegrad_\By^2 \phi^\dagger(\By)
}
- m^2 \phi^\dagger(\Bx)
=
   \spacegrad^2 \phi^\dagger(\Bx)
- m^2 \phi^\dagger(\Bx).
\end{equation}
\begin{equation}\label{eqn:ProblemSet1Problem2:780}
\PD{t}{\pi^\dagger}(\Bx, t)
=
i \antisymmetric{H}{\pi^\dagger(\Bx)}
=
i \int d^3 y \lr{
   \cancel{\antisymmetric{\pi^\dagger(\By) \pi(\By)}{\pi^\dagger(\Bx)}}
   +
   \antisymmetric{ \spacegrad_\By \phi^\dagger(\By) \cdot \spacegrad_\By \phi(\By) }{\pi^\dagger(\Bx)}
   +
   m^2 \antisymmetric{ \phi^\dagger(\By) \phi(\By)  }{\pi^\dagger(\Bx)}
}
=
i \int d^3 y \lr{
   \spacegrad_\By \phi(\By) \cdot \spacegrad_\By \antisymmetric{ \phi^\dagger(\By) }{\pi^\dagger(\Bx)}
   +
   m^2 \phi(\By) \antisymmetric{ \phi^\dagger(\By)  }{\pi^\dagger(\Bx)}
}
=
i \int d^3 y \lr{
   \spacegrad_\By \phi(\By) \cdot \spacegrad_\By (i \deltathree(\By - \Bx))
   +
   m^2 \phi(\By) i \deltathree(\By - \Bx)
}
=
- \int d^3 y \lr{
   \spacegrad_\By \cdot \lr{ \deltathree(\By - \Bx) \spacegrad_\By \phi(\By) }
   -
   \deltathree(\By - \Bx) \spacegrad_\By^2 \phi(\By)
}
- m^2 \phi(\Bx)
=
   \spacegrad^2 \phi(\Bx)
- m^2 \phi(\Bx).
\end{equation}
\end{subequations}
This recovers the Klein-Gordon equations
\begin{equation}\label{eqn:ProblemSet1Problem2:820}
\begin{aligned}
\lr{ \PDSq{t}{} - \spacegrad^2 + m^2 } &\phi(\Bx, t) = 0 \\
\lr{ \PDSq{t}{} - \spacegrad^2 + m^2 } &\phi^\dagger(\Bx, t) = 0,
\end{aligned}
\end{equation}
consistent with \cref{eqn:ProblemSet1Problem2:340} found by evaluating the classical Euler-Lagrange equations.

Somewhat cavalierly, the divergence integrals of the delta function above were assumed to be zero.
One possible justification for killing the delta function divergence integrals above first transforms those into surface integrals
\begin{equation}\label{eqn:ProblemSet1Problem2:800}
\int_V d^3 y \spacegrad_\By \cdot \lr{ \deltathree(\By - \Bx) \spacegrad_\By f(\By) }
=
\int_{\partial V} dA_\By \deltathree(\By - \Bx) \ncap_\By \cdot \spacegrad_\By f(\By),
\end{equation}
after which one argue that this is non-zero only when \( \Bx \) is on the boundary, so if we let the boundary go to infinity, it is zero everywhere, regardless of the normal derivative of the function being operated on\footnote{This was Prof. Poppitz's argument.  It's not completely convincing to me, as it requires integrating a delta function that may sit on the boundary.  However, what is the meaning of such a boundary integral, such as \( \int_0^\infty \delta(x) dx \)?  Apparently, such integrals are considered well defined in field theory, and we'll end up encountering these later too, and one of the future problems will help us understand an interpretation.}.
%\citep{DavidMayrhofer} where the author starts with separate real and imaginary fields and builds the complex representation systematically.

\paragraph{Diagonal basis for the Hamiltonian.}
In class we saw that a momentum space representation of \( \phi, \pi \) for the scalar single field Lagrangian simplified the Hamiltonian considerably.
Let's assume a similar momentum space representation of our field operator
\begin{equation}\label{eqn:ProblemSet1Problem2:840}
\wtilde{\phi}(\Bp, t) = \inv{\sqrt{2 \omega_\Bp}} \lr{ e^{-i \omega_\Bp t} a_\Bp + e^{i \omega_\Bp t} b^\dagger_\Bp },
\end{equation}
but will not make any a-priori assumption that the quantized field operator \( \phi \) is Hermitian.
We find the following spatial representation of the operator \( \phi \) and it's relations
\begin{subequations}
\label{eqn:ProblemSet1Problem2:860}
\begin{equation}\label{eqn:ProblemSet1Problem2:880}
\phi(\Bx, t) =
\int \frac{d^3 p}{(2\pi)^3} e^{i \Bp \cdot \Bx}
\inv{\sqrt{2 \omega_\Bp}} \lr{ e^{-i \omega_\Bp t} a_\Bp + e^{i \omega_\Bp t} b^\dagger_\Bp }
\end{equation}
\begin{equation}\label{eqn:ProblemSet1Problem2:900}
\phi^\dagger(\Bx, t) =
\int \frac{d^3 p}{(2\pi)^3} e^{-i \Bp \cdot \Bx}
\inv{\sqrt{2 \omega_\Bp}} \lr{ e^{i \omega_\Bp t} a_\Bp^\dagger + e^{-i \omega_\Bp t} b_\Bp }
\end{equation}
\begin{equation}\label{eqn:ProblemSet1Problem2:1140}
\spacegrad \phi(\Bx, t) =
\int \frac{d^3 p}{(2\pi)^3} e^{i \Bp \cdot \Bx}
\frac{i \Bp}{\sqrt{2 \omega_\Bp}} \lr{ e^{-i \omega_\Bp t} a_\Bp + e^{i \omega_\Bp t} b^\dagger_\Bp }
\end{equation}
\begin{equation}\label{eqn:ProblemSet1Problem2:1160}
\spacegrad \phi^\dagger(\Bx, t) =
\int \frac{d^3 p}{(2\pi)^3} e^{-i \Bp \cdot \Bx}
\frac{-i \Bp}{\sqrt{2 \omega_\Bp}} \lr{ e^{i \omega_\Bp t} a_\Bp^\dagger + e^{-i \omega_\Bp t} b_\Bp }
\end{equation}
\begin{equation}\label{eqn:ProblemSet1Problem2:920}
\pi(\Bx, t) = \PD{t}{\phi^\dagger}
=
\int \frac{d^3 p}{(2\pi)^3} e^{-i \Bp \cdot \Bx}
\frac{i \omega_\Bp}{\sqrt{2 \omega_\Bp}} \lr{ e^{i \omega_\Bp t} a_\Bp^\dagger - e^{-i \omega_\Bp t} b_\Bp }
\end{equation}
\begin{equation}\label{eqn:ProblemSet1Problem2:940}
\pi^\dagger(\Bx, t) = \PD{t}{\phi}
=
\int \frac{d^3 p}{(2\pi)^3} e^{i \Bp \cdot \Bx}
\frac{i\omega_\Bp}{\sqrt{2 \omega_\Bp}} \lr{ -e^{-i \omega_\Bp t} a_\Bp + e^{i \omega_\Bp t} b^\dagger_\Bp }.
\end{equation}
\end{subequations}
By inspection, we may read off the Fourier transform of \( \wtilde{\pi^\dagger} \), which is
\begin{equation}\label{eqn:ProblemSet1Problem2:960}
\wtilde{\pi^\dagger}(\Bp, t)
=
\frac{i\omega_\Bp}{\sqrt{2 \omega_\Bp}} \lr{ -e^{-i \omega_\Bp t} a_\Bp + e^{i \omega_\Bp t} b^\dagger_\Bp },
\end{equation}
which allows, with \cref{eqn:ProblemSet1Problem2:840}, inversion for operators \( a_\Bp, b^\dagger_\Bp \)
\begin{equation}\label{eqn:ProblemSet1Problem2:980}
\begin{aligned}
a_\Bp &= e^{i\omega_\Bp t} \sqrt{\frac{\omega_\Bp}{2}} \lr{ \wtilde{\phi} - \inv{i \omega_\Bp} \wtilde{\pi^\dagger} } \\
b_\Bp^\dagger &= e^{-i\omega_\Bp t} \sqrt{\frac{\omega_\Bp}{2}} \lr{ \wtilde{\phi} + \inv{i \omega_\Bp} \wtilde{\pi^\dagger} },
\end{aligned}
\end{equation}
or, in terms of spatial operators
\begin{equation}\label{eqn:ProblemSet1Problem2:1000}
\begin{aligned}
a_\Bp &=
\int d^3 x e^{-i \Bp \cdot \Bx}
e^{i\omega_\Bp t} \sqrt{\frac{\omega_\Bp}{2}} \lr{ \phi(\Bx, t) - \inv{i \omega_\Bp} \pi^\dagger(\Bx, t) } \\
a_\Bp^\dagger &=
\int d^3 x e^{i \Bp \cdot \Bx}
e^{-i\omega_\Bp t} \sqrt{\frac{\omega_\Bp}{2}} \lr{ \phi^\dagger(\Bx, t) + \inv{i \omega_\Bp} \pi(\Bx, t) } \\
b_\Bp &=
\int d^3 x e^{i \Bp \cdot \Bx}
e^{i\omega_\Bp t} \sqrt{\frac{\omega_\Bp}{2}} \lr{ \phi^\dagger(\Bx, t) - \inv{i \omega_\Bp} \pi(\Bx, t) } \\
b_\Bp^\dagger &=
\int d^3 x e^{-i \Bp \cdot \Bx}
e^{-i\omega_\Bp t} \sqrt{\frac{\omega_\Bp}{2}} \lr{ \phi(\Bx, t) + \inv{i \omega_\Bp} \pi^\dagger(\Bx, t) }.
\end{aligned}
\end{equation}
%a_\Bq^\dagger &=
%\int d^3 y e^{i \Bq \cdot \By}
%e^{-i\omega_\Bq t} \sqrt{\frac{\omega_\Bq}{2}} \lr{ \phi^\dagger(\By, t) + \inv{i \omega_\Bq} \pi(\By, t) } \\
%b_\Bq &=
%\int d^3 y e^{i \Bq \cdot \By}
%e^{i\omega_\Bq t} \sqrt{\frac{\omega_\Bq}{2}} \lr{ \phi^\dagger(\By, t) - \inv{i \omega_\Bq} \pi(\By, t) }
%b_\Bq^\dagger &=
%\int d^3 y e^{-i \Bq \cdot \By}
%e^{-i\omega_\Bq t} \sqrt{\frac{\omega_\Bq}{2}} \lr{ \phi(\By, t) + \inv{i \omega_\Bq} \pi^\dagger(\By, t) }.
We seek the commutators of all the \cref{eqn:ProblemSet1Problem2:1000} Fourier coefficient operators, which we expect to behave like creation and annihilation operators.  By inspection
\( 0 =
\antisymmetric{a_\Bp}{b_\Bq^\dagger}
=
\antisymmetric{a_\Bp}{a_\Bq}
=
\antisymmetric{b_\Bp}{a_\Bq^\dagger}
=
\antisymmetric{b_\Bp}{b_\Bq}
\), but the rest require evaluation.  We expect \( 0
= \antisymmetric{a_\Bp}{b_\Bq}
= \antisymmetric{a_\Bp^\dagger}{b_\Bq^\dagger} \) and explicit expansion confirms this
\begin{subequations}
\label{eqn:ProblemSet1Problem2:1020}
\begin{equation}\label{eqn:ProblemSet1Problem2:1040}
\begin{aligned}
\antisymmetric{a_\Bp}{b_\Bq} 
&=
\int d^3 x
d^3 y
e^{-i \Bp \cdot \Bx}
e^{i \Bq \cdot \By}
e^{i\omega_\Bp t}
e^{i\omega_\Bq t}
\sqrt{\frac{\omega_\Bp}{2}}
\sqrt{\frac{\omega_\Bq}{2}} \,\times \\
&\qquad
\antisymmetric{ \phi(\Bx, t) - \inv{i \omega_\Bp} \pi^\dagger(\Bx, t) }
{ \phi^\dagger(\By, t) - \inv{i \omega_\Bq} \pi(\By, t) } \\
&=
\int
d^3 x
d^3 y
e^{-i \Bp \cdot \Bx}
e^{i \Bq \cdot \By}
e^{i\omega_\Bp t}
e^{i\omega_\Bq t}
\sqrt{\frac{\omega_\Bp}{2}}
\sqrt{\frac{\omega_\Bq}{2}} \,\times \\
&\qquad
\lr{
   -\inv{i \omega_\Bq} i \deltathree(\Bx - \By)
   -\inv{i \omega_\Bp} (-i) \deltathree(\Bx - \By)
} \\
&=
\inv{2} \int d^3 x e^{i (\Bq - \Bp) \cdot \Bx}
e^{i\omega_\Bp t}
e^{i\omega_\Bq t}
\sqrt{\omega_\Bp \omega_\Bq}
\lr{
   -\inv{ \omega_\Bq}
   +\inv{ \omega_\Bp}
} \\
&=
(2 \pi)^3 \delta(\Bq - \Bp)
y^{i\omega_\Bp t}
e^{i\omega_\Bq t}
\sqrt{\omega_\Bp \omega_\Bq}
\lr{
   -\inv{ \omega_\Bq}
   +\inv{ \omega_\Bp}
} \\
&= 0,
\end{aligned}
\end{equation}
\begin{equation}\label{eqn:ProblemSet1Problem2:1060}
\begin{aligned}
\antisymmetric{a_\Bp^\dagger}{b_\Bq^\dagger} 
&=
\int
d^3 x
d^3 y
e^{i \Bp \cdot \Bx}
e^{-i \Bq \cdot \By}
e^{-i\omega_\Bp t}
e^{-i\omega_\Bq t}
\sqrt{\frac{\omega_\Bp}{2}}
\sqrt{\frac{\omega_\Bq}{2}} \,\times \\
&\qquad
\antisymmetric{ \phi^\dagger(\Bx, t) + \inv{i \omega_\Bp} \pi(\Bx, t) }
{ \phi(\By, t) + \inv{i \omega_\Bq} \pi^\dagger(\By, t) } \\
&=
\int
d^3 x
d^3 y
e^{i \Bp \cdot \Bx}
e^{-i \Bq \cdot \By}
e^{-i\omega_\Bp t}
e^{-i\omega_\Bq t}
\sqrt{\frac{\omega_\Bp}{2}}
\sqrt{\frac{\omega_\Bq}{2}} \,\times \\
&\qquad
\lr{
   +\inv{i \omega_\Bq} i \deltathree(\Bx - \By)
   +\inv{i \omega_\Bp} (-i) \deltathree(\Bx - \By)
} \\
&=
\inv{2} \int d^3 x e^{i (\Bp - \Bq) \cdot \Bx}
e^{-i\omega_\Bp t}
e^{-i\omega_\Bq t}
\sqrt{\omega_\Bp \omega_\Bq}
\lr{
   \inv{ \omega_\Bq}
   -\inv{ \omega_\Bp}
} \\
&=
(2 \pi)^3 \delta(\Bp - \Bq)
e^{i\omega_\Bp t}
e^{i\omega_\Bq t}
\sqrt{\omega_\Bp \omega_\Bq}
\lr{
   \inv{ \omega_\Bq}
   - \inv{ \omega_\Bp}
}
&= 0.
\end{aligned}
\end{equation}
\end{subequations}
Finally, we expect that there are two pairs of non-zero commutators
\begin{subequations}
\label{eqn:ProblemSet1Problem2:1080}
\begin{equation}\label{eqn:ProblemSet1Problem2:1100}
\begin{aligned}
\antisymmetric{a_\Bp}{a_\Bq^\dagger} 
&=
\int d^3 x
d^3 y
e^{-i \Bp \cdot \Bx}
e^{i \Bq \cdot \By}
e^{i\omega_\Bp t}
e^{-i\omega_\Bq t}
\sqrt{\frac{\omega_\Bp}{2}}
\sqrt{\frac{\omega_\Bq}{2}} \,\times \\
&\qquad \antisymmetric{ \phi(\Bx, t) - \inv{i \omega_\Bp} \pi^\dagger(\Bx, t) }
{ \phi^\dagger(\By, t) + \inv{i \omega_\Bq} \pi(\By, t) } \\
&=
\inv{2} \int d^3 x
d^3 y
e^{-i \Bp \cdot \Bx}
e^{i \Bq \cdot \By}
e^{i\omega_\Bp t}
e^{-i\omega_\Bq t} 
\sqrt{\omega_\Bp \omega_\Bq}
\,\times \\
&\qquad \lr{
   \inv{i \omega_\Bq} i \deltathree(\Bx - \By)
   -\inv{i \omega_\Bp} (-i) \deltathree(\Bx - \By)
} \\
&=
\inv{2} \int d^3 x
e^{i (\Bq -\Bp)\cdot \Bx}
e^{i\omega_\Bp t}
e^{-i\omega_\Bq t}
\sqrt{\omega_\Bp \omega_\Bq}
\lr{
   \inv{ \omega_\Bq}
   +\inv{ \omega_\Bp}
} \\
&=
\inv{2} (2 \pi)^3 \delta(\Bq - \Bp)
e^{i\omega_\Bp t}
e^{-i\omega_\Bq t}
\sqrt{\omega_\Bp \omega_\Bq}
\lr{
   \inv{ \omega_\Bq}
   +\inv{ \omega_\Bp}
} \\
&=
(2 \pi)^3 \delta(\Bq - \Bp),
\end{aligned}
\end{equation}
\begin{equation}\label{eqn:ProblemSet1Problem2:1120}
\begin{aligned}
\antisymmetric{b_\Bp}{b_\Bq^\dagger} 
&=
\int d^3 x
d^3 y
e^{i \Bp \cdot \Bx}
e^{-i \Bq \cdot \By}
e^{i\omega_\Bp t}
e^{-i\omega_\Bq t}
\sqrt{\frac{\omega_\Bp}{2}}
\sqrt{\frac{\omega_\Bq}{2}} \, \times \\
&\qquad \antisymmetric
{ \phi^\dagger(\Bx, t) - \inv{i \omega_\Bp} \pi(\Bx, t) }
{ \phi(\By, t) + \inv{i \omega_\Bq} \pi^\dagger(\By, t) } \\
&=
\int d^3 x
d^3 y
e^{i \Bp \cdot \Bx}
e^{-i \Bq \cdot \By}
e^{i\omega_\Bp t}
e^{-i\omega_\Bq t}
\sqrt{\frac{\omega_\Bp}{2}}
\sqrt{\frac{\omega_\Bq}{2}} \,\times \\
&\qquad \lr{
   \inv{i \omega_\Bq} i \deltathree(\Bx - \By)
   -\inv{i \omega_\Bp} (-i) \deltathree(\Bx - \By)
} \\
&=
\inv{2} \int d^3 x
e^{i (\Bp -\Bq)\cdot \Bx}
e^{i\omega_\Bp t}
e^{-i\omega_\Bq t}
\sqrt{\omega_\Bp \omega_\Bq}
\lr{
   \inv{ \omega_\Bq}
   +\inv{ \omega_\Bp}
} \\
&=
\inv{2}
(2 \pi)^3
\deltathree(\Bp - \Bq)
e^{i\omega_\Bp t}
e^{-i\omega_\Bq t}
\sqrt{\omega_\Bp \omega_\Bq}
\lr{
   \inv{ \omega_\Bq}
   +\inv{ \omega_\Bp}
} \\
&=
(2 \pi)^3
\deltathree(\Bp - \Bq).
\end{aligned}
\end{equation}
\end{subequations}
%Other than the \( (2 \pi)^3 \) scale factor we see from 
The \( \antisymmetric{a_\Bp^\dagger}{a_\Bq^\dagger}, \antisymmetric{b_\Bp^\dagger}{b_\Bq^\dagger} \) commutators show that the fields may be represented as a pair of independent creation and annihilation operators.

Let's compute the Hamiltonian representation next to verify that it diagonalizes nicely with this representation.
We use \cref{eqn:ProblemSet1Problem2:860} to find
\begin{subequations}
\label{eqn:ProblemSet1Problem2:1180}
\begin{equation}\label{eqn:ProblemSet1Problem2:1200}
\begin{aligned}
\int &d^3 x \pi^\dagger \pi \\
&=
\int d^3 x
\frac{d^3 p}{(2\pi)^3}
\frac{d^3 q}{(2\pi)^3}
e^{i \Bp \cdot \Bx}
e^{-i \Bq \cdot \Bx}
\frac{i \omega_\Bp}{\sqrt{2 \omega_\Bp}}
\frac{i \omega_\Bq}{\sqrt{2 \omega_\Bq}} \,\times \\
&\qquad \lr{ -e^{-i \omega_\Bp t} a_\Bp + e^{i \omega_\Bp t} b^\dagger_\Bp }
\lr{ e^{i \omega_\Bq t} a_\Bq^\dagger - e^{-i \omega_\Bq t} b_\Bq } \\
&=
\inv{2} \int
\frac{d^3 p}{(2\pi)^3}
\omega_\Bp
\lr{ e^{-i \omega_\Bp t} a_\Bp - e^{i \omega_\Bp t} b^\dagger_\Bp }
\lr{ e^{i \omega_\Bp t} a_\Bp^\dagger - e^{-i \omega_\Bp t} b_\Bp } \\
&=
\inv{2} \int
\frac{d^3 p}{(2\pi)^3}
\omega_\Bp
\lr{
   a_\Bp a_\Bp^\dagger + b_\Bp^\dagger b_\Bp
   + e^{ 2 i \omega_\Bp t} (-b_\Bp^\dagger a_\Bp^\dagger) +
   + e^{ -2 i \omega_\Bp t} (-a_\Bp b_\Bp )
},
\end{aligned}
\end{equation}
\begin{equation}\label{eqn:ProblemSet1Problem2:1220}
\begin{aligned}
\int &d^3 x \lr{ \spacegrad \phi^\dagger \cdot \spacegrad \phi + m^2 \phi^\dagger \phi } \\
&=
\inv{2} \int d^3 x
\frac{d^3 p}{(2\pi)^3}
\frac{d^3 q}{(2\pi)^3}
e^{i (\Bq -\Bp) \cdot \Bx} \,\times \\
&\qquad \frac{ (\Bp \cdot \Bq + m^2) }{\sqrt{\omega_\Bp \omega_\Bq}}
\lr{ e^{i \omega_\Bp t} a_\Bp^\dagger + e^{-i \omega_\Bp t} b_\Bp }
\lr{ e^{-i \omega_\Bq t} a_\Bq + e^{i \omega_\Bq t} b^\dagger_\Bq } \\
&=
\inv{2} \int
d^3 p
\frac{d^3 q}{(2\pi)^3}
\deltathree(\Bq - \Bp)
\frac{ (\Bp \cdot \Bq + m^2) }{\sqrt{\omega_\Bp \omega_\Bq}} \,\times \\
&\qquad \lr{ e^{i \omega_\Bp t} a_\Bp^\dagger + e^{-i \omega_\Bp t} b_\Bp }
\lr{ e^{-i \omega_\Bq t} a_\Bq + e^{i \omega_\Bq t} b^\dagger_\Bq } \\
&=
\inv{2} \int
\frac{d^3 p}{(2\pi)^3}  \omega_\Bp
\lr{ e^{i \omega_\Bp t} a_\Bp^\dagger + e^{-i \omega_\Bp t} b_\Bp }
\lr{ e^{-i \omega_\Bp t} a_\Bp + e^{i \omega_\Bp t} b^\dagger_\Bp } \\
&=
\inv{2} \int
\frac{d^3 p}{(2\pi)^3}  \omega_\Bp
\lr{
   a_\Bp^\dagger
   a_\Bp
   +
   b_\Bp
   b_\Bp^\dagger
   +
   e^{2 i \omega_\Bp t} \lr{ a_\Bp^\dagger b_\Bp^\dagger }
   +
   e^{-2 i \omega_\Bp t} \lr{ b_\Bp a_\Bp }
}
\end{aligned}
\end{equation}
\end{subequations}
Summing \cref{eqn:ProblemSet1Problem2:1180}, we find the
Hamiltonian has the expected diagonal representation
\begin{equation}\label{eqn:ProblemSet1Problem2:1240}
H =
\inv{2}
\int
\frac{d^3 p}{(2\pi)^3}
\omega_\Bp
\lr{
   a_\Bp^\dagger a_\Bp
+
   a_\Bp a_\Bp^\dagger
+ b_\Bp^\dagger b_\Bp
+
   b_\Bp b_\Bp^\dagger
},
\end{equation}
or in normal form
\boxedEquation{eqn:ProblemSet1Problem2:1260}{
:H: =
\int
\frac{d^3 p}{(2\pi)^3}
\omega_\Bp
\lr{
   a_\Bp^\dagger a_\Bp
+ b_\Bp^\dagger b_\Bp
}.
}

\makeSubAnswer{}{qft:problemSet1:2c}

Before diving into computation, it is worth deriving \cref{eqn:ProblemSet1Problem2:580} manually, since the naive calculation using the current as derived in class differs slightly.  We can find the current/charge as stated in the problem if our variation maintains the order of the conjugate pairs.  The symmetry is that imposed by the transformation
%
\begin{equation}\label{eqn:ProblemSet1Problem2:1300}
\begin{aligned}
\phi(x) &\rightarrow e^{-i\theta/2} \phi(x) \approx (1 - i\theta/2)\phi(x) \\
\phi^\dagger(x) &\rightarrow e^{i\theta/2} \phi^\dagger(x) \approx (1 + i\theta/2)\phi^\dagger(x),
\end{aligned}
\end{equation}
or
\begin{equation}\label{eqn:ProblemSet1Problem2:1320}
\begin{aligned}
\delta \phi(x) &= -\frac{i}{2} \theta \phi(x) \\
\delta \phi^\dagger(x) &= \frac{i}{2} \theta \phi^\dagger(x).
\end{aligned}
\end{equation}

The Lagrangian is left unchanged by this transformation, so we can determine the current directly by varying the action, but do so leaving the order of the \(\phi^\dagger\) and \( \phi\) terms in the Lagrangian unchanged
\begin{equation}\label{eqn:ProblemSet1Problem2:1340}
\delta S
= \int d^4 x \delta \lr{ \partial_\mu \phi^\dagger \partial^\mu \phi - m^2 \phi^\dagger \phi }
= \int d^4 x
\lr{
   \delta\lr{ \partial_\mu \phi^\dagger } \partial^\mu \phi
  + \partial^\mu \phi^\dagger \delta \lr{ \partial_\mu \phi }
   - m^2 \lr{\delta \phi^\dagger} \phi
  - m^2 \phi^\dagger \lr{\delta \phi}
}
= \int d^4 x
\lr{
   \partial_\mu \lr{ \lr{ \delta \phi^\dagger} \partial^\mu \phi }
-
   \lr{\delta \phi^\dagger} \lr{\partial_\mu \partial^\mu \phi}
 + \partial_\mu \lr{ \partial^\mu \phi^\dagger \delta \phi }
 - \lr{ \partial_\mu \partial^\mu \phi^\dagger } \delta \phi
   - m^2 \lr{ \delta \phi^\dagger} \phi - m^2 \phi^\dagger \lr{ \delta \phi}
}
=
\int d^4 x
\partial_\mu
\lr{
   \delta \phi^\dagger \partial^\mu \phi
 + \partial^\mu \phi^\dagger \delta \phi
}
-
\int d^4 x
\delta \phi^\dagger
\lr{
   \lr{\partial_\mu \partial^\mu \phi}
+ m^2 \phi
}
-
\int d^4 x
\lr{
 \partial_\mu \partial^\mu \phi^\dagger
+ m^2 \phi^\dagger
}
\delta \phi
=
\int d^4 x
\partial_\mu
\lr{
   \delta \phi^\dagger \partial^\mu \phi
 + \partial^\mu \phi^\dagger \delta \phi
},
\end{equation}
where the Euler-Lagrange equations for each of the fields has been imposed to kill off the last two integrals.  We are left with a current
\begin{equation}\label{eqn:ProblemSet1Problem2:1360}
j^\mu =
   \delta \phi^\dagger \partial^\mu \phi
 + \partial^\mu \phi^\dagger \delta \phi
=
\frac{i\theta}{2} \lr{
   \phi^\dagger \lr{ \partial^\mu \phi }
 - \lr{ \partial^\mu \phi^\dagger} \phi
}.
\end{equation}
In particular
\begin{equation}\label{eqn:ProblemSet1Problem2:1380}
\evalbar{j^0}{\theta = 1}
=
\frac{i}{2} \lr{
   \phi^\dagger \lr{ \partial^0 \phi }
 - \lr{ \partial^0 \phi^\dagger} \phi
}
=
\frac{i}{2} \lr{
   \phi^\dagger \pi^\dagger - \pi \phi
}.
\end{equation}
This recovers \cref{eqn:ProblemSet1Problem2:580}, and we are now set to compute the charge by plugging in
\cref{eqn:ProblemSet1Problem2:580}
\begin{equation}\label{eqn:ProblemSet1Problem2:600}
Q =
\frac{i}{2}
\int d^3 x
\lr{ \phi^\dagger \pi^\dagger - \pi \phi }
=
\frac{i}{4}
\int d^3 x
\frac{d^3 p}{(2\pi)^3}
\frac{d^3 q}{(2\pi)^3}
e^{i (\Bq -\Bp) \cdot \Bx}
\frac{i\omega_\Bq}{\sqrt{\omega_\Bp\omega_\Bq}}
\lr{ e^{i \omega_\Bp t} a_\Bp^\dagger + e^{-i \omega_\Bp t} b_\Bp }
\lr{ -e^{-i \omega_\Bq t} a_\Bq + e^{i \omega_\Bq t} b^\dagger_\Bq }
-
\frac{i}{4}
\int d^3 x
\frac{d^3 p}{(2\pi)^3}
\frac{d^3 q}{(2\pi)^3}
e^{i (\Bp -\Bq)\cdot \Bx}
\frac{i \omega_\Bp}{\sqrt{\omega_\Bp\omega_\Bq}}
\lr{ e^{i \omega_\Bp t} a_\Bp^\dagger - e^{-i \omega_\Bp t} b_\Bp }
\lr{ e^{-i \omega_\Bq t} a_\Bq + e^{i \omega_\Bq t} b^\dagger_\Bq }
=
\frac{1}{4}
\int
\frac{d^3 p}{(2\pi)^3}
\lr{
   \lr{ e^{i \omega_\Bp t} a_\Bp^\dagger - e^{-i \omega_\Bp t} b_\Bp }
   \lr{ e^{-i \omega_\Bp t} a_\Bp + e^{i \omega_\Bp t} b^\dagger_\Bp }
   -
   \lr{ e^{i \omega_\Bp t} a_\Bp^\dagger + e^{-i \omega_\Bp t} b_\Bp }
   \lr{ -e^{-i \omega_\Bp t} a_\Bp + e^{i \omega_\Bp t} b^\dagger_\Bp }
}
=
\frac{1}{4}
\int
\frac{d^3 p}{(2\pi)^3}
\lr{
   a_\Bp^\dagger a_\Bp - b_\Bp b_\Bp^\dagger + a_\Bp^\dagger a_\Bp - b_\Bp b_\Bp^\dagger
   + e^{2 i \omega_\Bp t} \lr{ b_\Bp^\dagger a_\Bp^\dagger  -a_\Bp^\dagger b_\Bp^\dagger }
   + e^{-2 i \omega_\Bp t} \lr{ -b_\Bp a_\Bp + b_\Bp a_\Bp }
}
=
\frac{1}{2}
\int
\frac{d^3 p}{(2\pi)^3}
\lr{
   a_\Bp^\dagger a_\Bp - b_\Bp b_\Bp^\dagger
},
\end{equation}
or, in normal order
\boxedEquation{eqn:ProblemSet1Problem2:1280}{
:Q:
=
\frac{1}{2}
\int
\frac{d^3 p}{(2\pi)^3}
\lr{
   a_\Bp^\dagger a_\Bp - b_\Bp^\dagger b_\Bp
}
}
To understand the action of the charge operator (a set of number operators) we may apply it to the states corresponding to each creation operator.  With
\begin{equation}\label{eqn:ProblemSet1Problem2:640}
\begin{aligned}
\ket{\Bk}_a &= a^\dagger_\Bk \ket{0} \\
\ket{\Bk}_b &= b^\dagger_\Bk \ket{0},
\end{aligned}
\end{equation}
we find
\begin{equation}\label{eqn:ProblemSet1Problem2:660}
Q \ket{\Bk}_a =
\frac{1}{2} \int
\frac{d^3 p}{(2\pi)^3}
\lr{
a_\Bp^\dagger a_\Bp - b_\Bp^\dagger b_\Bp
}
a^\dagger_\Bk \ket{0}
=
\frac{1}{2} \int
\frac{d^3 p}{(2\pi)^3}
a_\Bp^\dagger a_\Bp
a^\dagger_\Bk \ket{0}
=
\frac{1}{2} \int
\frac{d^3 p}{(2\pi)^3}
a_\Bp^\dagger
\lr{
a^\dagger_\Bk
a_\Bp
+ (2\pi)^3 \deltathree(\Bk - \Bp)
}
\ket{0}
=
\inv{2}
a_\Bk^\dagger \ket{0}
=
\inv{2}
\ket{\Bk}_a,
\end{equation}
and
\begin{equation}\label{eqn:ProblemSet1Problem2:680}
Q \ket{\Bk}_b =
\frac{1}{2} \int
\frac{d^3 p}{(2\pi)^3}
\lr{
a_\Bp^\dagger a_\Bp - b_\Bp^\dagger b_\Bp
}
b^\dagger_\Bk \ket{0}
=
-\frac{1}{2} \int
\frac{d^3 p}{(2\pi)^3}
b_\Bp^\dagger b_\Bp
b^\dagger_\Bk \ket{0}
=
-\frac{1}{2} \int
\frac{d^3 p}{(2\pi)^3}
b_\Bp^\dagger
\lr{
b^\dagger_\Bk
b_\Bp
+ (2\pi)^3 \deltathree(\Bk - \Bp)
}
\ket{0}
=
-\inv{2}
b_\Bk^\dagger \ket{0}
=
-\inv{2}
\ket{\Bk}_b.
\end{equation}
So, we could say that the particles associated with creation operator \( a_\Bp^\dagger \) have a (1/2) charge and
particles associated with creation operator \( b_\Bp^\dagger \) have a (-1/2) charge.
However, the \( 1/2 \), as well as the sign itself, was arbitrary, coming from the value of \( \theta \) used in the transformation of the field.   Therefore,
it is probably more accurate to say that the
\( a_\Bp^\dagger a_\Bp \) portion of the charge operator is associated with some unit of charge whereas the
\( b_\Bp^\dagger b_\Bp \) portion of the charge operator is associated with a unit of charge that has an opposite sign.
}}
%}

      %
% Copyright � 2018 Peeter Joot.  All Rights Reserved.
% Licenced as described in the file LICENSE under the root directory of this GIT repository.
%
\makeproblem{
Zero point energy, an exercise in unit conversion, and scales related to the ``cosmological constant problem''
}{qft:problemSet1:3}{
In class, we showed that the zero-point energy of the quantized massless scalar field (we are taking this case, because in the physically relevant case of electrodynamics, the number of degrees of freedom and the associated vacuum energy is the same as that of two massless scalar fields) can be written as:
\begin{dmath}\label{eqn:ProblemSet1Problem3:20}
E_{\text{vac}} = V_3 \int \frac{d^3 k}{(2\pi)^3} \frac{\omega_k}{2}.
\end{dmath}
where \( V_3 \) is the (large, i.e., almost infinite) volume of space. This expression diverges, because we assume that electromagnetic fields and photons of arbitrarily large momenta exist. There's no justification to this, as particle physicists have only probed the Standard Model up to energies of order a few \( \si{TeV} \). Assume, then, that the integral above is cut off at some maximum value of the momentum \( \Lambda \) (called the ``UV cutoff''), say of order \( 10 \,\si{TeV} \).
\makesubproblem{}{qft:problemSet1:3a}
What is the value of the vacuum energy density \( \rho_{\text{vac}} \), in units of \( \si{g/cm^3} \).
\makesubproblem{}{qft:problemSet1:3b}
What value should \( \Lambda \) have in order that \( \rho_{\text{vac}} \) matches the observed value of the ``dark energy'', of order
\( \rho_{\text{dark}} \sim 10^{-29} \, \si{g/cm^3} \).
Express \( \Lambda \) both as a high-energy scale cutoff and as a short-distance cutoff.
\makesubproblem{}{qft:problemSet1:3c}
What is the ratio of \( \rho_{\text{vac}} \) for \( \Lambda \sim M_{\text{Planck}} \) to \( \rho_{\text{dark}} \)?
\makesubproblem{}{qft:problemSet1:3d}
Note that the zero-point energies of phonons -- the zero point energies of the quantized collective sound oscillations of nuclei in a crystal -- are given, up to simple numerical factors counting the numbers of polarizations (which we won't worry about here) by an expression similar to the above.
This is because phonons are massless scalar fields propagating with the speed of sound instead of speed of light.
Notice that this difference is irrelevant as \( c \) appears in \( E_{\text{vac}} \) simply: \( k \) is a wavevector and \( \omega_k = c k \) -- a frequency (secretly multiplied by \( \Hbar \), of course).
In the case of phonons, however, we are well aware that a cutoff scale exists and we understand well its nature: it is given by the interatomic separation, as the notion of phonons does not make sense for shorter wavelengths.
Now take \( k_{\text{max}} = \Lambda \sim 1/a_0 \), with \( a_0 \) of order the Bohr radius and estimate the energy density of the zero point fluctuations in a crystal.

Compare your result to the typical rest energy (i.e. mass) density of crystals.
The results from the first three items above lead to a puzzle commonly referred to as the ``cosmological constant problem''.
There are various proposals for its solution, ranging from cancellations between the contributions of high and low momentum oscillators, anthropic principle (multiverse) considerations, modifications of gravity at long distances, to name a few.
The issue awaits your input!
} % makeproblem

\makeanswer{qft:problemSet1:3}{
\makeSubAnswer{}{qft:problemSet1:3a}
To make a bit more sense of the unit conversions required, let's insert factors of \( \Hbar, c \) back into the mix temporarily
\begin{dmath}\label{eqn:ProblemSet1Problem3:40}
E_{\text{vac}}
= V_3 \int \frac{d^3 k}{(2\pi)^3} \frac{\Hbar \omega_k}{2}.
= V_3 \frac{\Hbar (4 \pi)}{(2\pi)^3 2} \int_0^k k^2 dk \omega_k
= V_3 \frac{\Hbar }{(2\pi)^2 c^3 } \int_0^\omega \omega^3 d\omega
= V_3 \frac{\Hbar \omega^4}{4 (2\pi)^2 c^3 },
\end{dmath}
so
%\begin{equation}\label{eqn:ProblemSet1Problem3:60}
\boxedEquation{eqn:ProblemSet1Problem3:60}{
\rho_{\text{vac}}
=
\frac{E_{\text{vac}} }{V_3}
=
\inv{16 \pi^2} (\Hbar \omega) \lr{\frac{\omega}{c}}^3
}
%\end{equation}
Observe that \( [\omega/c] = 1/L \) so we have \( \text{energy}/\text{L}^3\) as desired.
With the following conversion factors (\citep{wiki:naturalUnits})
\begin{dmath}\label{eqn:ProblemSet1Problem3:80}
\begin{aligned}
1 \,\si{eV} &= 1.78 \times 10^{-33} \,\si{g} \\
1 \,\si{(eV)^{-1}} &= 1.97 \times 10^{-5} \,\si{cm} \\
\end{aligned}
\end{dmath}
we have
\begin{equation}\label{eqn:ProblemSet1Problem3:220}
(1 eV)^4 = 1.78 \times 10^{-33} \lr{ \inv{1.97 \times 10^{-5}} }^3 \,\si{g/(cm)^3}
=
2.3 \times 10^{-19} \,\si{g/(cm)^3},
\end{equation}
and
\begin{equation}\label{eqn:ProblemSet1Problem3:240}
1 \,\si{g/(cm)^3}
= \inv{ 2.3 \times 10^{-19} }
\,\si{(eV)^4}
= 4.3 \times 10^{18}
\,\si{(eV)^4}
\end{equation}
The vacuum energy density at the \( 10 \,\si{TeV} \) cutoff is therefore
\begin{dmath}\label{eqn:ProblemSet1Problem3:120}
\rho_{\text{vac}} = \inv{16 \pi^2} (10^{13} eV)^4 \times
2.3 \times 10^{-19} \,\si{g/(cm)^3/(eV)^4}
=
1.4 \times 10^{31} \,\si{g/(cm)^3}.
\end{dmath}
This seems extraordinarily large to me, especially given the intuitive description of vacuum as empty.

\makeSubAnswer{}{qft:problemSet1:3b}
The equivalent cutoff associated with the dark energy density is
\begin{dmath}\label{eqn:ProblemSet1Problem3:140}
\Lambda
= \lr{ 16 \pi^2 \rho }^{1/4}
= \lr{ 16 \pi^2 \times 10^{-29} \,\si{g/(cm)^3} }^{1/4}
= \lr{ 16 \pi^2 \times 10^{-29} \,\si{g/(cm)^3}
\times
4.3 \times 10^{18}
\,\si{(eV)^4/(g/(cm)^3)}
}^{1/4}
=
9.1 \times 10^{-3} \,\si{eV}.
\end{dmath}
(In contrast with the vacuum energy density, this seems extraordinarily small.)

As a distance scale (wavelength), this is
\begin{dmath}\label{eqn:ProblemSet1Problem3:160}
\lambda = \frac{2 \pi}{k} = \frac{2 \pi}{9.1 \times 10^{-3} \,\si{eV}} \times 1.97 \times 10^{-5} (\si{eV})(\si{cm})
= 1.4 \times 10^{-2} \,\si{cm}.
\end{dmath}

\makeSubAnswer{}{qft:problemSet1:3c}

The Planck mass is
\begin{dmath}\label{eqn:ProblemSet1Problem3:180}
M_{\text{Planck}}
=
2.2 \times 10^{-5} \,\si{g} \times \frac{1 \,\si{eV}}{1.78 \times 10^{-33} \,\si{g} }
= 1.2 \times 10^{28} eV,
\end{dmath}
so the energy density ratio is
\begin{dmath}\label{eqn:ProblemSet1Problem3:200}
\frac{\rho_{\text{vac (Planck)}}}{
\rho_{\text{dark}}}
= \frac
{\lr{ 10^{28} \,\si{eV} }^4}
{\lr{ 10^{-2} \,\si{eV} }^4}
= 10^{120}.
\end{dmath}
This is an extraordinary difference, but what it means is not clear to me.

\makeSubAnswer{}{qft:problemSet1:3d}
Mathematica workbook attached.
}

      %
% Copyright � 2018 Peeter Joot.  All Rights Reserved.
% Licenced as described in the file LICENSE under the root directory of this GIT repository.
%
\makeoproblem{Scale invariance and conserved charge.}{qft:problemSet1:4}{2018 HW1.IV}{
Consider classical electrodynamics with the Lagrangian
\begin{dmath}\label{eqn:ProblemSet1Problem4:20}
S = \int d^4 x \lr{ -\inv{4} F_{\mu\nu} F^{\mu\nu} }.
\end{dmath}
Consider the following ``dilatation'' (or ``scale'') transformation:
\begin{dmath}\label{eqn:ProblemSet1Problem4:40}
\begin{aligned}
x_\mu &\rightarrow x'_\mu = e^d x_\mu \\
A_\mu(x) &\rightarrow A'_\mu(x') = e^{-d} A_\mu(x),
\end{aligned}
\end{dmath}
where \( d \) is a constant, called the dilatation parameter.

Dilatation invariance in QED (and QCD) is perhaps the simplest example of a symmetry, where the classical action is invariant, but the quantum theory is not (as you will learn later, in the spring class). Broken scale invariance arises because one has to introduce a short-distance cutoff (a UV ``regulator'') to define the quantum theory. (We already saw an indication of the need for a regulator when we considered the divergent zero point energy of the free quantum scalar field.)

\makesubproblem{}{qft:problemSet1:4a}
Show that the action is invariant under dilatations.
\makesubproblem{}{qft:problemSet1:4b}
Find the corresponding Noether current.
\makesubproblem{}{qft:problemSet1:4c}
Show that -- perhaps, after a redefinition of \( j_\mu \) ; notice that any conserved current \( j_\mu \) can be
redefined by adding to it \( \partial^\nu C_{\mu\nu} \), where \( C_{\mu\nu} \) is antisymmetric, without spoiling its conservation
(in this case \( C \) can depend on \( x^\mu, \partial^\mu \) and \( A^\mu \), of course) the dilatation current is simply related
to the energy-momentum tensor: \( j^{\text{con f}}_\mu = x_\nu {{T^\nu}_\mu}^{\text{con f}}\), where the symbol con f indicates that
these are the conformal energy-momentum tensor and dilatation current. Notice that this problem, secretly, requires you to also derive \( T^{\mu\nu} \) for the electromagnetic field.
\makesubproblem{}{qft:problemSet1:4d}
Show, then, that conservation of \(  j^{\text{con f}}_\mu  \) implies that the energy-momentum tensor of classical
electrodynamics is traceless (the trace of the tensor is defined as usual to be \( g_{\mu\nu} T^{\mu\nu}\)).
\makesubproblem{}{qft:problemSet1:4e}
Finally, open your classical electrodynamics books and recall the interpretation of the \( T^{00}, T^{xx},T^{yy} \), etc., components of the energy momentum tensor as energy density and pressure. Show that the tracelessness of \( T^{\mu\nu} \) is equivalent to the familiar relation
\begin{dmath}\label{eqn:ProblemSet1Problem4:680}
p = \rho/3
\end{dmath}
between the energy density and pressure of isotropic radiation -- the equation of state of blackbody radiation.\footnote{In class, I   promised you some finite-temperature problem, but this homework got long. For now, this will remain the only connection. I'll try to keep my promise... may be in the final?}
\index{scale invariance}
\index{conserved charge}
\index{Noether current}
} % makeproblem

\makeanswer{qft:problemSet1:4}{
\withproblemsetsParagraph{
\makeSubAnswer{}{qft:problemSet1:4a}
With \( {x'}^\mu = e^{d} x^\mu \), the volume element transforms as
\begin{dmath}\label{eqn:ProblemSet1Problem4:60}
d^4 x' \rightarrow e^{4d} d^4 x.
\end{dmath}
The components of the four-gradient transform as
\begin{dmath}\label{eqn:ProblemSet1Problem4:80}
\PD{x'_\mu}{}
=
\PD{x'_\mu}{x_\mu}
\PD{x_\mu}{}
=
e^{-d}
\PD{x_\mu}{},
\end{dmath}
so
\begin{dmath}\label{eqn:ProblemSet1Problem4:100}
F'_{\mu\nu} =
\partial'_{\mu} A'_\nu
-
\partial'_{\nu} A'_\mu
=
e^{-2d} F_{\mu\nu}.
\end{dmath}
The action is therefore invariant
\begin{dmath}\label{eqn:ProblemSet1Problem4:120}
S'
= -\inv{4} \int d^4 x' F'_{\mu\nu} {F'}^{\mu\nu}
= -\inv{4} \int e^{4d} d^4 x e^{2d} F_{\mu\nu} e^{2d} F^{\mu\nu}
= -\inv{4} \int d^4 x F_{\mu\nu} F^{\mu\nu}
= S.
\end{dmath}
\makeSubAnswer{}{qft:problemSet1:4b}
We need the variation of the potential
\begin{dmath}\label{eqn:ProblemSet1Problem4:180}
\delta A_{\nu}
= A'_{\nu}(x) - A_{\nu}(x)
= A'_{\nu}(e^{-d} x') - A_{\nu}(x)
\approx A'_{\nu}((1 -d) x') - A_{\nu}(x)
= e^{-d} A_{\nu}((1 -d) x') - A_{\nu}(x)
\approx (1-d) \lr{ A_{\nu} - d x^\alpha \partial_\alpha A_{\nu}} - A_{\nu}
=
- d x^\alpha \partial_\alpha A_{\nu}
- d \lr{ A_{\nu} - d x^\alpha \partial_\alpha A_{\nu}}
\approx
-d(1+ x^\alpha \partial_\alpha )A_{\nu},
\end{dmath}
and the variation of the field
\begin{dmath}\label{eqn:ProblemSet1Problem4:140}
\delta F_{\mu\nu}
= F'_{\mu\nu}(x) - F_{\mu\nu}(x)
= F'_{\mu\nu}(e^{-d} x') - F_{\mu\nu}(x)
\approx F'_{\mu\nu}((1 -d) x') - F_{\mu\nu}(x)
= e^{-2d} F_{\mu\nu}((1 -d) x') - F_{\mu\nu}(x)
\approx (1-2d) \lr{ F_{\mu\nu} - d x^\alpha \partial_\alpha F_{\mu\nu}} - F_{\mu\nu}
=
- d x^\alpha \partial_\alpha F_{\mu\nu}
-2 d \lr{ F_{\mu\nu} - d x^\alpha \partial_\alpha F_{\mu\nu}}
\approx
-d(2+ x^\alpha \partial_\alpha )
F_{\mu\nu},
\end{dmath}
so the variation of the Lagrangian is
\begin{dmath}\label{eqn:ProblemSet1Problem4:160}
\delta \LL
=
-\inv{2} (\delta F_{\mu\nu}) F^{\mu\nu}
=
-\inv{2}
(-d)
F^{\mu\nu}
(2+ x^\alpha \partial_\alpha )F_{\mu\nu}
=
(d)
F^{\mu\nu}
F_{\mu\nu}
+
\frac{d}{2}
F^{\mu\nu} x^\alpha \partial_\alpha F_{\mu\nu}
=
(d)
F^{\mu\nu}
F_{\mu\nu}
+
\frac{d}{4}
x^\alpha \partial_\alpha \lr{ F_{\mu\nu} F^{\mu\nu} }
=
-4 (d) \LL - (d) x^\alpha \partial_\alpha \LL
=
-4 (d) \LL - (d) \lr{ \partial_\alpha (x^\alpha \LL) - \LL \partial_\alpha x^\alpha }
=
- (d) \partial_\alpha (x^\alpha \LL),
\end{dmath}
so the variational current (what is this called?) is
\begin{dmath}\label{eqn:ProblemSet1Problem4:200}
J^\mu_d
=
- (d) x^\mu \LL.
\end{dmath}
Finally, we need
\begin{dmath}\label{eqn:ProblemSet1Problem4:220}
\PD{(\partial_\mu A_\nu)}{\LL}
=
-\inv{2}
F^{\alpha\beta}
\PD{(\partial_\mu A_\nu)}{}\lr{ \partial_\alpha A_\beta - \partial_\beta A_\alpha}
=
-\inv{2}
\lr{
   F^{\mu\nu}
   -
   F^{\nu\mu}
}
=
-F^{\mu\nu}.
\end{dmath}
Combining \cref{eqn:ProblemSet1Problem4:220}, \cref{eqn:ProblemSet1Problem4:200}, and \cref{eqn:ProblemSet1Problem4:180} we can calculate the conserved current, which is (for \( d = 1 \)) is
\begin{dmath}\label{eqn:ProblemSet1Problem4:240}
j^\mu_{\text{dil}}
=
\PD{(\partial_\mu A_\nu)}{\LL} \delta A_\nu - J^\mu_d
=
F^{\mu\nu}
\lr{
   A_\nu + x^\alpha \partial_\alpha A_{\nu}
}
+
x^\mu \LL.
\end{dmath}
This can be put into a slightly nicer form
\begin{dmath}\label{eqn:ProblemSet1Problem4:260}
j^\mu_{\text{dil}}
=
F^{\mu\nu}
   A_\nu
+
F^{\mu\nu} x^\alpha
F_{\alpha\nu}
+
F^{\mu\nu} x^\alpha
\partial_\nu A_{\alpha}
+
x^\mu \LL
=
\cancel{
F^{\mu\nu}
   A_\nu
}
+
F^{\mu\nu} x^\alpha
F_{\alpha\nu}
+
\partial_\nu \lr{
   F^{\mu\nu} x^\alpha
   A_{\alpha}
}
-
A_{\alpha}
x^\alpha \cancel{\partial_\nu F^{\mu\nu} }
-
\cancel{
A_{\alpha}
F^{\mu\nu}
\partial_\nu x^\alpha
}
+
x^\mu \LL
=
F^{\mu\nu} x^\alpha
F_{\alpha\nu}
+
\partial_\nu \lr{
   F^{\mu\nu} x^\alpha
   A_{\alpha}
}
+
x^\mu \LL,
\end{dmath}
or
\boxedEquation{eqn:ProblemSet1Problem4:280}{
   j^\mu_{\text{dil}}
   =
   x^\alpha
   \lr{
      F^{\mu\nu}
      F_{\alpha\nu}
      +
      {\delta^\mu}_\alpha \LL
   }
   +
   \partial_\nu \lr{
      F^{\mu\nu} x^\alpha
      A_{\alpha}
   }
}
It was hinted that the complete derivative of an antisymmetric tensor may be dropped from the current, that's because
\begin{dmath}\label{eqn:ProblemSet1Problem4:560}
\partial_\mu \lr{ j^\mu + \partial_\nu C^{\mu\nu} }
=
\partial_\mu j^\mu + \partial_\mu \partial_\nu C^{\mu\nu}
=
\partial_\mu j^\mu,
\end{dmath}
since
the derivative operator \( \partial_\mu\partial_\nu \) is symmetric, and the sum of the contraction of symmetric and antisymmetric tensors is zero.  Since the
complete derivative term  \(
   F^{\mu\nu} x^\alpha
   A_{\alpha} \) is antisymmetric in \( \mu\nu \) so we may drop it from the current, leaving only dependence on the electromagnetic field \( F \).

\makeSubAnswer{}{qft:problemSet1:4c}
Having been given the secret that we have to calculate the energy momentum tensor, let's start with calculation of the conserved current associated with a spacetime translation
\begin{subequations}
\label{eqn:ProblemSet1Problem4:320}
\begin{equation}\label{eqn:ProblemSet1Problem4:300}
x_\mu \rightarrow x_\mu' = x_\mu + a_\mu.
\end{equation}
\begin{equation}\label{eqn:ProblemSet1Problem4:340}
%\phi(x) \rightarrow \phi'(x') = \phi'(x + a) \approx \phi(x) + a^\alpha \partial_\alpha \phi.
A_\nu(x) \rightarrow A_\nu'(x') = A_\nu(x) + a^\alpha \partial_\alpha A_\nu.
\end{equation}
\end{subequations}
The gradient \( \partial_\mu \) and volume element \( d^4 x\) are unchanged by a translation transformation.  The potential transforms as
\begin{dmath}\label{eqn:ProblemSet1Problem4:360}
\delta A_\nu
= A_\nu'(x) - A_\nu(x)
= A_\nu'(x' - a) - A_\nu(x)
\approx A_\nu(x) - a^\alpha \partial_\alpha A_\nu - A_\nu(x)
=
- a^\alpha \partial_\alpha A_\nu.
\end{dmath}
The field transforms as
\begin{dmath}\label{eqn:ProblemSet1Problem4:380}
\delta F_{\mu\nu}
= F_{\mu\nu}'(x) - F_{\mu\nu}(x)
= F_{\mu\nu}'(x' - a) - F_{\mu\nu}(x)
\approx F_{\mu\nu}(x) - a^\alpha \partial_\alpha F_{\mu\nu} - F_{\mu\nu}(x)
=
- a^\alpha \partial_\alpha F_{\mu\nu}.
\end{dmath}
Finally the Lagrangian density transforms as
\begin{dmath}\label{eqn:ProblemSet1Problem4:400}
\delta \LL
=
-\inv{2} (\delta F_{\mu\nu}) F^{\mu\nu}
=
\inv{2} a^\alpha \lr{ \partial_\alpha F_{\mu\nu} } F^{\mu\nu}
=
\inv{4} a^\alpha \partial_\alpha \lr{ F_{\mu\nu} F^{\mu\nu} }
=
- \partial_\alpha \lr{ a^\alpha \LL }.
\end{dmath}
That is
\begin{dmath}\label{eqn:ProblemSet1Problem4:420}
J^\mu_a = - a^\mu \LL.
\end{dmath}
The conserved current associated with spacetime translation is
\begin{dmath}\label{eqn:ProblemSet1Problem4:440}
j^\mu_a
=
\PD{(\partial_\mu A_\nu)}{\LL} \delta A_\nu - J^\mu_a
=
- F^{\mu\nu} (-a^\alpha \partial_\alpha A_\nu) + a^\mu \LL.
\end{dmath}
As was the case in \cref{eqn:ProblemSet1Problem4:280} we are able to put group all the explicit potential dependence in a discardable package
\begin{dmath}\label{eqn:ProblemSet1Problem4:460}
=
a^\alpha
F^{\mu\nu}
F_{\alpha\nu}
+
a^\alpha
F^{\mu\nu}
\partial_\nu A_\alpha
 + a^\mu \LL
=
a^\alpha
F^{\mu\nu}
F_{\alpha\nu}
+
a^\alpha
\partial_\nu
\lr{
F^{\mu\nu}
A_\alpha
}
-
a^\alpha
\cancel{ \partial_\nu F^{\mu\nu} }
A_\alpha
 + a^\mu \LL
\end{dmath}
or
\boxedEquation{eqn:ProblemSet1Problem4:480}{
   j^\mu_a
   =
   a^\alpha \lr{
      F^{\mu\nu}
      F_{\alpha\nu}
      +
      {\delta^\mu}_\alpha \LL
   }
   +
   \partial_\nu
   \lr{
      F^{\mu\nu}
      a^\alpha
      A_\alpha
   }
}
The factor \(
F^{\mu\nu}
a^\alpha
A_\alpha \) is completely antisymmetric in \( \mu\nu \) so we may drop it from the current.  From
\cref{eqn:ProblemSet1Problem4:280}, \cref{eqn:ProblemSet1Problem4:480} we can introduce
(conformal) dilatation \( \tilde{j_{\text{dil}}} \) and translation conservation
\( \tilde{j}_a \)
currents
\begin{dmath}\label{eqn:ProblemSet1Problem4:660}
\begin{aligned}
   \tilde{j}^\mu_{\text{dil}}
   &=
   -j^\mu_{\text{dil}}
   +
   \partial_\nu \lr{
      F^{\mu\nu} x^\alpha
      A_{\alpha}
   } \\
   \tilde{j}^\mu_a
   &=
   -j^\mu_a
   +
   \partial_\nu
   \lr{
      F^{\mu\nu}
      a^\alpha
      A_\alpha
   },
\end{aligned}
\end{dmath}
effectively dropping the complete derivative terms (also changing signs to match the literature \citep{jackson1975cew}).  That is
\boxedEquation{eqn:ProblemSet1Problem4:500}{
\begin{aligned}
\tilde{j}^\mu_{\text{dil}} &= x^\nu {\Theta^\mu}_\nu \\
\tilde{j}^\mu_{a} &= a^\nu {\Theta^\mu}_\nu \\
{\Theta^\mu}_\nu &=
   F^{\mu\sigma}
   F_{\sigma\nu}
   -
   {\delta^\mu}_\nu \LL.
\end{aligned}
}
Here we've factored out the common (conformal) energy momentum tensor \( {\Theta^\mu}_\nu \), which may also be written with upper indexes
\begin{dmath}\label{eqn:ProblemSet1Problem4:520}
\Theta^{\mu\nu} =
   F^{\mu\sigma}
   F_{\sigma\alpha}
   g^{\alpha\nu}
   -
   g^{\mu\nu} \LL,
\end{dmath}
which is symmetric with respect to index interchange
\begin{dmath}\label{eqn:ProblemSet1Problem4:540}
\Theta^{\nu\mu}
=
   F^{\nu\sigma}
   F_{\sigma\alpha}
   g^{\alpha\mu}
   -
   g^{\nu\mu} \LL
=
   g^{\beta\nu}
   F_{\beta\sigma}
   F^{\sigma\mu}
   -
   g^{\mu\nu} \LL
=
   F^{\mu\sigma}
   F_{\sigma\beta}
   g^{\beta\nu}
   -
   g^{\mu\nu} \LL
= \Theta^{\mu\nu}.
\end{dmath}
\makeSubAnswer{}{qft:problemSet1:4d}
We require the divergence of a Noether current to be zero, so for the dilatation current
\begin{dmath}\label{eqn:ProblemSet1Problem4:580}
0 = \partial_\mu \tilde{j}^\mu_{\text{dil}}
=
\lr{ \partial_\mu x^\nu }
{\Theta^\mu}_\nu
+
x^\nu
\partial_\mu {\Theta^\mu}_\nu
=
{\Theta^\mu}_\mu
+
x^\nu
\partial_\mu {\Theta^\mu}_\nu.
\end{dmath}
In particular for \( x = 0 \) we must have \(
{\Theta^\mu}_\mu  = 0 \).  Incidentally, given \( {\Theta^\mu}_\mu  = 0 \), then
for non-zero \( x \) we must also have \( \partial_\mu {\Theta^\mu}_\nu = 0 \).  That can be demonstrated directly utilizing the zero divergence of the Noether current for a spacetime translation
\begin{dmath}\label{eqn:ProblemSet1Problem4:600}
0
= \partial_\mu \tilde{j}^\mu_{a}
= a^\nu \partial_\mu {\Theta^\mu}_\nu.
\end{dmath}
As this is zero for all \( a \) we must have \( \partial_\mu {\Theta^\mu}_\nu = 0 \).

\makeSubAnswer{}{qft:problemSet1:4e}
The trace written out explicitly is
\begin{equation}\label{eqn:ProblemSet1Problem4:620}
0 =
{\Theta^\mu}_\mu
=
{\Theta^0}_0
+
{\Theta^1}_1
+
{\Theta^2}_2
+
{\Theta^3}_3
=
\Theta^{00}
-
\Theta^{11}
-
\Theta^{22}
-
\Theta^{33},
\end{equation}
Since \( \Theta^{00} = \inv{2}( \BE^2 + \BB^2 ) = \rho \), and \( -\Theta^{kj} = T^{(M)}_{kj} = E_k E_j + B_k B_j - \inv{2} \delta_{kj}(\BE^2 + \BB^2) \), where \( T^{(M)}_{kj} \) is the electromagnetic stress tensor (borrowing notation from \citep{jackson1975cew} again), we have
\begin{dmath}\label{eqn:ProblemSet1Problem4:700}
\rho = -\sum_{k = 1}^3 T^{(M)}_{kk}.
\end{dmath}
In \citep{griffiths1999introduction} \( T^{(M)}_{ij} \) is described as ``the force (per unit area) in the ith direction action on an element of surface oriented in the jth direction -- diagonal elements represent pressures, and off-diagonal elements are shears''.
Integration of the stress tensor over a cube, as sketched in \cref{fig:outwards_normal_cube:outwards_normal_cubeFig1}, serves to illustrate this nicely, as only the diagonal elements contribute to such an integral.  If the total cubic face area is \( A = 6 \Delta A \), the total force of on the surface is
\begin{dmath}\label{eqn:ProblemSet1Problem4:720}
\BF
=
\int \lrT \cdot \Ba
=
\Be_1 \int \delta_{1k} \lr{ \evalbar{T^{(M)}_{k1}}{+} - \evalbar{T^{(M)}_{k1}}{-}}
+
\Be_2 \int \delta_{2k} \lr{ \evalbar{T^{(M)}_{k2}}{+} - \evalbar{T^{(M)}_{k2}}{-}}
+
\Be_3 \int \delta_{3k} \lr{ \evalbar{T^{(M)}_{k3}}{+} - \evalbar{T^{(M)}_{k3}}{-}}
=
\Delta A \Be_1 \lr{ \evalbar{T^{(M)}_{k1}}{+} - \evalbar{T^{(M)}_{k1}}{-}}
+
\Delta A \Be_2 \lr{ \evalbar{T^{(M)}_{k2}}{+} - \evalbar{T^{(M)}_{k2}}{-}}
+
\Delta A \Be_3 \lr{ \evalbar{T^{(M)}_{k3}}{+} - \evalbar{T^{(M)}_{k3}}{-}}
\end{dmath}
\imageFigure{../figures/phy2403-quantum-field-theory/outwards_normal_cubeFig1}{Cubic surface and outwards normals.}{fig:outwards_normal_cube:outwards_normal_cubeFig1}{0.3}
Assuming isotropic fields, the total pressure of the fields on the surface is
\begin{dmath}\label{eqn:ProblemSet1Problem4:740}
p = \Abs{ \frac{2 \Delta A \sum_{k = 1}^3 T^{(M)}_{kk}}{6 \Delta A} }
= \inv{3} \rho,
\end{dmath}
which recovers \cref{eqn:ProblemSet1Problem4:680}.
}}

      %
% Copyright � 2018 Peeter Joot.  All Rights Reserved.
% Licenced as described in the file LICENSE under the root directory of this GIT repository.
%
\makeproblem{Observability of the zero point energy: the Casimir force.}{qft:problemSet1:5}{
In class, when discussing the quantization of the real scalar field, we found the sum of zero
point energies of the harmonic oscillators (one per each \( \Bk \) ) into which we decomposed the field:
\begin{dmath}\label{eqn:ProblemSet1Problem5:20}
E_{\text{zero point}} =
V_3 \int \frac{d^3 k}{(2\pi)^3} \frac{\omega_\Bk}{2}.
\end{dmath}
\makesubproblem{}{qft:problemSet1:5a}
Show that the boundary conditions on the plates impose a quantization condition on the allowed values of field momentum perpendicular to the plates, i.e. \( k_x = n\pi/a, n = 0, \pm 1,  \pm 2, \cdots \) [e.g., recall your waveguide physics].
\makesubproblem{}{qft:problemSet1:5b}
Consider now the contribution to the energy of the vacuum fluctuations of the field in the space between the plates and find the zero point energy per unit area of the plates. To do this, replace the integral over \( k_x \) in
\cref{eqn:ProblemSet1Problem5:20}
by a sum over \( n \), \( \int dk_x = (\pi/a)\sum_n\) [Hint: to save work, use the fact that the correct expression should have the property that as the plates are removed, \( a \rightarrow \infty \), the energy (per unit volume) should give back
\cref{eqn:ProblemSet1Problem5:20}
]. Does the resulting expression for the zero point energy still diverge?
\makesubproblem{}{qft:problemSet1:5c}
Show now, starting from
\cref{eqn:ProblemSet1Problem5:20}
, with integral replaced by sum, that the difference between the zero point energies per unit area, in the space between the plates in the presence of the plates and without the plates is:
\begin{dmath}\label{eqn:ProblemSet1Problem5:40}
\Delta E_{\text{vac}}(a) = \int_0^\infty \frac{dk}{2\pi} k \lr{ \frac{k}{4}
+ \inv{2} \sum_{n = 1}^\infty \sqrt{ k^2 + \frac{n^2 \pi^2}{a^2} }
- \inv{2} \int_0^\infty \sqrt{ k^2 + \frac{n^2 \pi^2}{a^2} }
}.
\end{dmath}
where, obviously, \( k \) is radial wave vector in \(y, z\)-directions.
\makesubproblem{}{qft:problemSet1:5d}
COPY.
\makesubproblem{}{qft:problemSet1:5e}
COPY.
\makesubproblem{}{qft:problemSet1:5f}
COPY.
\makesubproblem{}{qft:problemSet1:5g}
COPY.
\makesubproblem{}{qft:problemSet1:5h}
COPY.
} % makeproblem

\makeanswer{qft:problemSet1:5}{
\makeSubAnswer{}{qft:problemSet1:5a}
TODO.
\makeSubAnswer{}{qft:problemSet1:5b}
TODO.
\makeSubAnswer{}{qft:problemSet1:5c}
TODO.
\makeSubAnswer{}{qft:problemSet1:5d}
TODO.
\makeSubAnswer{}{qft:problemSet1:5e}
TODO.
\makeSubAnswer{}{qft:problemSet1:5f}
TODO.
\makeSubAnswer{}{qft:problemSet1:5g}
TODO.
\makeSubAnswer{}{qft:problemSet1:5h}
TODO.
}


   \chapter{Problem Set 2.}

      %
% Copyright � 2018 Peeter Joot.  All Rights Reserved.
% Licenced as described in the file LICENSE under the root directory of this GIT repository.
%
\makeoproblem{Spacetime behaviour of various Green's functions.}{qft:problemSet2:1}{2018 HW2.I}{
\index{Green's function!spacelike separation}
\index{principle value integration}
\index{half delta function}
\index{light cone}
\index{Compton wavelength}
Here, you'll study some properties of
\begin{equation}\label{eqn:ProblemSet2Problem1:20}
D(x) \equiv \antisymmetric{\phihat_{-}(x)}{\phihat_{+}(x)} = \int \frac{d^3 p}{(2\pi)^3 2 \omega_p} e^{-i \omega_p t + i \Bp \cdot \Bx}.
\end{equation}
\makesubproblem{}{qft:problemSet2:1a}
For m = 0 (``photon''), show that:
\begin{dmath}\label{eqn:ProblemSet2Problem1:40}
D(x) = -\inv{2 \pi^2} \calP \inv{t^2 - r^2} - \frac{i}{8 \pi} \lr{
\frac{\delta(t - r)}{r}
-\frac{\delta(t + r)}{r}
},
\end{dmath}
where \( r = \Norm{\Bx} \). Notice that \( D(x) \) is singular on the light cone \( t = r\). Does it vanish for spacelike separations?

Hint: Please recall that (and why!)
\begin{dmath}\label{eqn:ProblemSet2Problem1:60}
\inv{a \pm i \epsilon} = \calP \inv{a} \mp i \pi \delta(a)
\end{dmath}
(here \( \calP \) denotes ``principal value integration'',
as this relation is to be understood in terms of generalized functions, i.e. in the back of your mind it always needs to be integrated over a with suitable smooth and integrable ``test functions''). Note
also that what looks like a ``half-delta-function integral'' \( \int_0^\infty dy e^{i x y} \)
should really be understood as
\(
\lim_{\epsilon\rightarrow 0} \int_0^\infty dy e^{-\epsilon y + i x y}
\)
\makesubproblem{}{qft:problemSet2:1b}
For \( m^2 > 0 \), study the behaviour of \( D(x) \) for spacelike \( x \) and find the asymptotic behaviour for
\( -x^2 \gg 1/m^2 \) (i.e., at spacelike separations larger than the particle's Compton wavelength).
} % makeproblem

\makeanswer{qft:problemSet2:1}{
\withproblemsetsParagraph{
\makeSubAnswer{}{qft:problemSet2:1a}
Let's evaluate the integral in spherical polar coordinates
\begin{dmath}\label{eqn:ProblemSet2Problem1:80}
\begin{aligned}
\Bp &= p (\sin\theta \cos\phi, \sin\theta \sin\phi, \cos\theta) \\
\Bx &= r(0, 0, 1) \\
d^3 p &= p^2 dp \sin\theta d\theta d\phi \\
\omega_\Bp &= \sqrt{\Bp^2 + \cancel{m^2}} = \Norm{\Bp} = p.
\end{aligned},
\end{dmath}
which gives
\begin{dmath}\label{eqn:ProblemSet2Problem1:100}
D(x)
=
\inv{(2\pi)^2} \int_{p = 0}^\infty dp\, p^2 \int_{\theta = 0}^\pi -d(\cos\theta) \inv{2 p} e^{-i p t + i p r \cos\theta}
=
-\inv{8 \pi^2} \int_{p = 0}^\infty dp\, p
e^{-i p t}
\evalrange{\frac{e^{i p r u}}{i p r}}{u = \cos\theta = 1}{-1}
=
\frac{i}{8 \pi^2 r} \int_{p = 0}^\infty dp \,
e^{-i p t}
\lr{
   e^{- i p r}
   -e^{i p r}
}
=
\frac{i}{8 \pi^2 r} \int_{p = 0}^\infty dp \,
\lr{
   e^{- i p (r+t)}
   -e^{i p (r-t)}
}
\end{dmath}
As hinted, this half delta function should be interpreted offset slightly
\begin{dmath}\label{eqn:ProblemSet2Problem1:120}
D(x)
=
\frac{i}{8 \pi^2 r} \int_{p = 0}^\infty dp \,
\lr{
   e^{- i p (r+t) + p \epsilon}
   -e^{i p (r-t) - p \epsilon}
}
=
\frac{i}{8 \pi^2 r}
\evalrange{
   \lr{
      \frac{e^{- i p (r+t) - p \epsilon}}{-i(r+t) - \epsilon}
      - \frac{e^{i p (r-t) - p \epsilon}}{i(r-t) - \epsilon}
   }
}
{0}{\infty}
=
\frac{i}{8 \pi^2 r}
   \lr{
        \frac{1}{ i(r+t) + \epsilon }
      + \frac{1}{ i(r-t) - \epsilon }
   }
=
\frac{1}{8 \pi^2 r}
   \lr{
        \frac{1}{ r + t - \epsilon }
      + \frac{1}{ r - t + \epsilon }
   }
\end{dmath}
Employing the hint
\cref{eqn:ProblemSet2Problem1:60}\footnote{A nice explanation of this second hint can be found in \citep{wiki:Sokhotski} under ``Proof of the real version''.},
\cref{eqn:ProblemSet2Problem1:120}
can be cast into delta function form
\begin{dmath}\label{eqn:ProblemSet2Problem1:140}
D(x)
=
\frac{1}{8 \pi^2 r}
   \lr{
        \calP \frac{1}{ r+t } + i \pi \delta( r + t )
      +
        \calP \frac{1}{ r-t } - i \pi \delta( r - t )
   }
=
\frac{1}{8 \pi^2 r}
   \lr{
        \calP \frac{2 r }{ r^2 - t^2 } + i \pi \lr{ \delta( r + t ) - \delta( r - t ) }
   },
\end{dmath}
which, after cosmetic rearrangement, is \cref{eqn:ProblemSet2Problem1:40}.

\makeSubAnswer{}{qft:problemSet2:1b}
Let's evaluate \( D(x) \) function at spacelike point \( x = (0, r\zcap) \), and switch to polar momentum space coordinates \( \Bp = p (\sin\theta \cos\phi, \sin\theta \sin\phi, \cos\theta) \).
This gives
\begin{dmath}\label{eqn:ProblemSet2Problem1:160}
D(0, r \zcap)
=
\inv{(2\pi)^3}
\int_0^\infty dp \int_0^{2\pi} d\phi \int_0^\pi d\theta p^2 \sin\theta \frac{e^{i p r \cos\theta}}{2 \sqrt{p^2 + m^2}}
=
\inv{2 (2\pi)^2}
\int_0^\infty dp \frac{p^2}{\sqrt{p^2 + m^2}} \int_{-1}^{1} du e^{i p r u}
=
\inv{2 (2\pi)^2}
\int_0^\infty dp \frac{p^2}{\sqrt{p^2 + m^2}} \frac{e^{i p r} - e^{-i p r}}{i p r}
=
\frac{-i}{2 (2\pi)^2 r}
\int_0^\infty dp \frac{p}{\sqrt{p^2 + m^2}} e^{i p r}
-
\frac{-i}{2 (2\pi)^2 r}
\int_0^{-\infty} (-dp') \frac{(-p')}{\sqrt{(-p')^2 + m^2}} e^{i p' r}
=
\frac{-i}{2 (2\pi)^2 r}
\int_{-\infty}^\infty dp \frac{p}{\sqrt{p^2 + m^2}} e^{i p r},
\end{dmath}
where a \( u = \cos\theta \) substitution was made, followed by \( p = -p' \) in the second integral.

This integral can be evaluated with the half dogbone contour sketched in \cref{fig:halfDogBoneContour:halfDogBoneContourFig1}.  The exponential \( e^{i p r} \) vanishes on the infinite radial contours \( D, E \) since the real part of \( i p r \) is negative in the upper half plane.  We are left with
\begin{dmath}\label{eqn:ProblemSet2Problem1:180}
\int_A = -\int_B - \int_C.
\end{dmath}
\imageFigure{../figures/phy2403-quantum-field-theory/halfDogBoneContourFig1}{Contour for branch at \( p = i m \).}{fig:halfDogBoneContour:halfDogBoneContourFig1}{0.3}

That doesn't look like a particularly helpful transformation at first, but because we are integrating around the branch cut running from \( [i m, i\infty] \), the square root differs by an \( 2 \pi i \) argument on each side of the cut.  Along contour \( B \) that square root is
\begin{dmath}\label{eqn:ProblemSet2Problem1:200}
(p^2 + m^2)^{1/2}
=
\lr{ e^{i \pi} \lr{ -p^2 - m^2} }^{1/2}
=
e^{i \pi/2} \sqrt{ -p^2 - m^2 }
=
i \sqrt{ -p^2 - m^2 }
\end{dmath}
and along contour \( D \)
\begin{dmath}\label{eqn:ProblemSet2Problem1:220}
(p^2 + m^2)^{1/2}
=
\lr{ e^{3 i \pi} \lr{ -p^2 - m^2} }^{1/2}
=
e^{3 i \pi/2} \sqrt{ -p^2 - m^2 }
=
-i \sqrt{ -p^2 - m^2 }.
\end{dmath}
Specifically
\begin{dmath}\label{eqn:ProblemSet2Problem1:240}
\int_A
= -\int_B - \int_C
=
\frac{i}{2 (2\pi)^2 r}
\int_{i \infty}^{i m} dp \frac{p}{i \sqrt{-p^2 - m^2}} e^{i p r}
+
\frac{i}{2 (2\pi)^2 r}
\int_{i m}^{i \infty} dp \frac{p}{-i \sqrt{-p^2 - m^2}} e^{i p r}
=
-\frac{1}{(2\pi)^2 r}
\int_{i m}^{i \infty} dp \frac{p}{\sqrt{-p^2 - m^2}} e^{i p r}.
\end{dmath}
Changing the integration variable to \( q \in [m, \infty] \) (i.e. \( p = i q \)), we have
\begin{dmath}\label{eqn:ProblemSet2Problem1:260}
D(0, r \zcap)
=
\frac{1}{(2\pi)^2 r}
\int_{m}^{\infty} dq \frac{q}{\sqrt{q^2 - m^2}} e^{-q r}
=
\frac{1}{(2\pi)^2 r^2}
\int_{m}^{\infty} r dq \frac{r q}{r \sqrt{q^2 - m^2}} e^{-q r}
=
\frac{1}{(2\pi)^2 r^2}
\int_{r m}^{\infty} dx \frac{x}{\sqrt{x^2 - (r m)^2}} e^{-x}
=
\frac{1}{(2\pi)^2 r} K_1(r m)
\sim
r^{-3/2}
e^{-r m }.
\end{dmath}
where the integral was evaluated with Mathematica, and the asymptotic approximation is from \citep{abramowitz1964handbook} \S 9.7.2 (\( K_\nu(z) \sim \sqrt{\pi/2z} e^{-z} \).)  For \( r \gg 1/m \), this goes to zero quickly.
}}

      %
% Copyright � 2018 Peeter Joot.  All Rights Reserved.
% Licenced as described in the file LICENSE under the root directory of this GIT repository.
%
\makeproblem{A model with \(SU(2)_L \times SU(2)_R\) internal global symmetry: chiral symmetry and the Higgs}{qft:problemSet2:2}{
This problem introduces a model to describe the symmetry realization of the nonabelian chiral symmetry in QCD (quantum chromodynamics).
The word ``chiral'' should become clear later in this class, but the ``nonabelian'' part will be clear below.
\(SU(2)_L \times SU(2)_R\) is an exact symmetry of QCD in the limit when the ``current masses'' of the \( u \) and \( d \) quark, \( m_u \) and \( m_d \), are taken to vanish.
In the real world, it is an approximate symmetry, in the sense that \( m_u \) and \( m_d \) are small compared to the intrinsic scale of QCD, given, say, by the proton mass (\( m_{u,d} \sim \,\si{MeV} \ll 1 \,\si{GeV} \)).
This is, thus, an example of an ``approximate symmetry''.

Closer to the theory you will study below, the scalar model with \(SU(2)_L \times SU(2)_R\) symmetry, is really the same as the Higgs sector in the Standard Model, in the limit when the electromagnetic and weak interactions are turned off.
\(SU(2)_L \times SU(2)_R\) becomes a symmetry in this limit.
It is only an approximate symmetry, as the electromagnetic and weak couplings (which explicitly break it) are dimensionless numbers smaller then unity.

Finally, to end the preaching preamble, the notion of approximate symmetries is not new and you have, for sure, been exposed to its usefulness when studying the hydrogen atom spectrum in quantum mechanics.

\makesubproblem{}{qft:problemSet2:2a}
The Lagrangian you will study is that of two complex scalar fields, assembled into a column \( \Phi = (\phi_1,\phi_2)^\T \) (the \( \T \) is here so I do not have to go through the trouble to write a column instead of a row).
It is given by:
\begin{dmath}\label{eqn:ProblemSet2Problem2:10} %(1)
\LL = \partial_\mu \Phi^\dagger \partial_\mu \Phi - m^2 \Phi^\dagger \Phi - \lambda \lr{\Phi^\dagger\Phi}^2.
\end{dmath}
Show that
\cref{eqn:ProblemSet2Problem2:10}
%(1)
is invariant under an \(SU(2)_L\) global symmetry transformation \( \Phi \rightarrow  U_L \Phi \), where \( U_L^\dagger U_L = 1 \) is a \( 2 \times 2 \) unitary matrix of unit determinant.
In addition, the Lagrangian has a \( U(1) \) symmetry, not part of \(SU(2)_L\), acting as \( \Phi \rightarrow  e^{i\alpha}\Phi\).
Find the currents and conserved charges under these symmetries.

Hint: recall that an infinitesimal \(SU(2)_L\) transformation can be written as \( U_L \approx \sigma^0 + i\omega_a \frac{\sigma^a}{2} \), where \( \sigma^0 \) is the unit \( 2 \times 2 \) matrix, \( \sigma^a, a = 1, 2, 3 \) are the Pauli matrices, and \( \omega_a \) are the three parameters of infinitesimal \(SU(2)_L\) transformations.

\makesubproblem{}{qft:problemSet2:2b}
Show that the charge operators, \( \hatQ^L_a, a = 1,2,3\), conserved due to \(SU(2)_L\) invariance, obey the angular momentum algebra, i.e., \( \antisymmetric{\hatQ^L_1}{ \hatQ^L_2 } = i \hatQ^L_3 \) (plus cyclic permutations).

\makesubproblem{}{qft:problemSet2:2c}
The Lagrangian \cref{eqn:ProblemSet2Problem2:10} has, however, a larger symmetry than simply the above \(SU(2)_L\).
To begin seeing this, instead of using \( \Phi = (\phi^1,\phi^2)^\T \) introduce the real and imaginary parts of \( \phi^{1,2} \).
Use \( \phi^1 = \psi^1 +i \psi^2, \phi^2 = \psi^3 +i \psi^4 \), and introducing \( \Psi = (\psi^1,\psi^2,\psi^3,\psi^4)^\T\), show that \cref{eqn:ProblemSet2Problem2:10} can be written as:
\begin{dmath}\label{eqn:ProblemSet2Problem2:20} % , (2)
\LL = a \partial_\mu \Psi^\T \partial^\mu \Psi -b m^2 \Psi^\T \Psi - c \lambda(\Psi^\T\Psi)^2
\end{dmath}
on the way determining the (pure numbers) \(a, b, c\).
The Lagrangian \cref{eqn:ProblemSet2Problem2:20} has, clearly, an \( O(4) \) symmetry, i.e., is invariant under \( \Psi \rightarrow  O \Psi\), where \( O \) is a \( 4 \times 4 \) orthogonal matrix, \( O^\T O = 1\).
Is there a continuous U(1) allowed in this case?

Comment: I will spare you finding the currents for \(SO(4)\) (\(SO(4)\) matrices are the restriction of \(O(4)\) matrices to the ones with unit determinant).
What you will do next, instead, is to use the equivalence of Lie algebras \(SO(4) \approx SU(2)_L \times SU(2)_R\), which will come about by another change of variables (see below).
Notice also that, as it comes, \(SO(4)\) happens to be the Euclidean version of \(SO(1,3)\).

\makesubproblem{}{qft:problemSet2:2d}
To expose the \(SU(2)_L \times SU(2)_R\) symmetry of \cref{eqn:ProblemSet2Problem2:10}, now use the following change of variables.
Consider, instead of \(\Phi\) in \cref{eqn:ProblemSet2Problem2:10} the \(2 \times 2\) matrix \(H\) made up by components of \(\Phi\) as follows:
\begin{equation}\label{eqn:ProblemSet2Problem2:30}
H \equiv \inv{\sqrt{2}} (i\sigma_2\Phi^\conj,\Phi) = \inv{\sqrt{2}}
\begin{bmatrix}
\phi_2^\conj & \phi_1 \\
-\phi_1^\conj & \phi_2
\end{bmatrix}
\end{equation}
Show that under \(SU(2) \) transformations, \( H \rightarrow  \inv{\sqrt{2}} (i\sigma_2 (U_L\Phi)^\conj, U_L \Phi) = \inv{\sqrt{2}} (U_L i \sigma_2 \Phi^\conj, U_L\Phi) = U_L H\).

Hint: the tricky part is to show that \( i\sigma_2(U_L\Phi)^\conj = i\sigma_2 U_L^\conj \Phi^\conj = U_L i \sigma_2 \Phi^\conj\).
What you need to show, then, is that \( \sigma_2 U_L \sigma_2 = U_L^\conj \) (this fact will be very useful in our future studies of spinors, so make sure you understand it).

\makesubproblem{}{qft:problemSet2:2e}
Using the change of variables \cref{eqn:ProblemSet2Problem2:30}, show that
\begin{dmath}\label{eqn:ProblemSet2Problem2:40}
H^\dagger H = \inv{2}
\begin{bmatrix}
\Abs{\phi_1}^2 + \Abs{\phi_2}^2 & 0 \\
0 & \Abs{\phi_1}^2 + \Abs{\phi_2}^2
\end{bmatrix},
\end{dmath}
% (4)
and, hence, that
\cref{eqn:ProblemSet2Problem2:10} %(1)
can be written as
\begin{dmath}\label{eqn:ProblemSet2Problem2:50}
\LL = \trace{
   \lr{
      \partial \mu H^\dagger \partial^\mu H
   }
}
- m^2 \trace{
   \lr{
      H^\dagger H
   }
}
- \lambda \lr{
   \trace{
      H^\dagger H
   }
}^2
\end{dmath}
%(5)
where \( \trace{}\) denotes the matrix trace.
Show that now \cref{eqn:ProblemSet2Problem2:50} has \(SU(2)_L \times SU(2)_R\) symmetry,
acting on \( H \) as
\begin{dmath}\label{eqn:ProblemSet2Problem2:60}
H \rightarrow U_L H U_R^\dagger,
\end{dmath}
% (6)
where the action of \( U_R^\dagger \) on the right is pure convention (we could have taken \( U_R \) instead).
\( U_L \) and \( U_R \) are two sets of independent \(SU(2) \) transformations.
The \( L \) and \( R \) (left and right) names are self-evident in the way \cref{eqn:ProblemSet2Problem2:60} is written.
Show that under \(SU(2)_L \times SU(2)_R\), we have \( \delta H = i \omega_a^L \frac{\sigma^a}{2} H - i\omega_b^R H \frac{\sigma^b}{2}\).

Hint: clearly, the only thing you need to show is \(SU(2)_R\) invariance, as \(SU(2)_L\) was already shown.

\makesubproblem{}{qft:problemSet2:2f}
Show that the left and right \(SU(2) \) conserved currents can be written as
\begin{dmath}\label{eqn:ProblemSet2Problem2:80}
\begin{aligned}
j^{\mu,a}_L &= \frac{i}{2} \trace{
\lr{
   \partial_\mu H^\dagger \sigma^a H - H^\dagger \sigma^a \partial_\mu H
}
}  \\
j^{\mu,b}_R &= \frac{i}{2} \trace{
\lr{
   \partial_\mu H \sigma^b H^\dagger - H \sigma^b \partial_\mu H^\dagger
}
}
\end{aligned}
\end{dmath}
and that the corresponding generators \( \hatQ^{L,R}_a \) obey the commutation relations of two commuting angular momentum algebras.

Hint: notice that both currents are Hermitean and that the left is obtained from the right by interchanging \( H \) with \( H^\dagger\).
} % makeproblem

\makeanswer{qft:problemSet2:2}{
\makeSubAnswer{}{qft:problemSet2:2a}
Let's consider the \( SU(2)_L \) case first.  Noting that \( (\sigma^a)^\dagger = \sigma^a \), the transformed fields are
\begin{dmath}\label{eqn:ProblemSet2Problem2:100}
\begin{aligned}
\Phi' &= e^{i \Bsigma \cdot \Bomega/2} \Phi \\
{\Phi'}^\dagger &= \Phi^\dagger e^{-i \Bsigma \cdot \Bomega/2},
\end{aligned}
\end{dmath}
so \( {\Phi'}^\dagger \Phi' = \Phi^\dagger \Phi \), and
so \( \partial_\mu {\Phi'}^\dagger \partial^\mu \Phi' = \partial_\mu \Phi^\dagger \partial^\mu \Phi \).
This shows that the Lagrangian density is invariant under this transformation.

The variation of the field is
\begin{dmath}\label{eqn:ProblemSet2Problem2:120}
\delta \Phi
= \Phi' - \Phi
\approx \lr{ 1 + i \Bsigma \cdot \Bomega/2} \Phi - \Phi
=
\frac{i}{2} \Bsigma \cdot \Bomega \Phi,
\end{dmath}
so
\begin{dmath}\label{eqn:ProblemSet2Problem2:140}
\delta (\Phi^\dagger \Phi)
=
(\delta \Phi^\dagger) \Phi + \Phi^\dagger \delta \Phi
=
\frac{i}{2} \lr{
-\Phi^\dagger \Bsigma \cdot \Bomega \Phi
+ \Phi^\dagger \Bsigma \cdot \Bomega \Phi
}
=
0,
\end{dmath}
and
\begin{dmath}\label{eqn:ProblemSet2Problem2:160}
\delta (\partial_\mu \Phi^\dagger \partial^\mu \Phi)
=
\partial_\mu (\delta \Phi^\dagger) \partial^\mu \Phi
+
\partial_\mu \Phi^\dagger \partial^\mu (\delta \Phi)
=
\frac{i}{2}
\lr{
   - \partial_\mu \Phi^\dagger \Bsigma \cdot \Bomega \partial^\mu \Phi
   + \partial_\mu \Phi^\dagger \Bsigma \cdot \Bomega \partial^\mu \Phi
}
=
0,
\end{dmath}
so \( \delta \LL = 0 \).  To calculate the conserved current, we have to be slightly careful with the order of operations so that the matrix products are compatible
\begin{dmath}\label{eqn:ProblemSet2Problem2:180}
j^\mu_\Bomega
=
\PD{(\partial_\mu \Phi)}{\LL} \delta \Phi
+
\delta \Phi^\dagger
\PD{(\partial_\mu \Phi^\dagger)}{\LL}
=
\frac{i}{2}
\lr{
   \partial^\mu \Phi^\dagger (\Bsigma \cdot \Bomega) \Phi
   -
   \Phi^\dagger (\Bsigma \cdot \Bomega) \partial^\mu \Phi
},
\end{dmath}
or
\begin{dmath}\label{eqn:ProblemSet2Problem2:200}
j^{\mu a} =
\frac{i}{2}
\lr{
   \partial^\mu \Phi^\dagger \sigma^a \Phi
   -
   \Phi^\dagger \sigma^a \partial^\mu \Phi
},
\end{dmath}
where \( j^\mu_\Bomega = \omega_a j^{\mu a} \).

For the \( U(1) \) case we clearly have \( \LL' = \LL \).  The variation is
\begin{dmath}\label{eqn:ProblemSet2Problem2:220}
\delta \Phi
= \Phi' - \Phi
\approx (1 + i\alpha) \Phi - \Phi
=
i\alpha \Phi,
\end{dmath}
so
\begin{dmath}\label{eqn:ProblemSet2Problem2:240}
\delta (\Phi^\dagger \Phi)
=
(\delta \Phi^\dagger) \Phi
+
\Phi^\dagger (\delta \Phi)
=
i \alpha
\lr{
   -\Phi^\dagger \Phi
   +
   \Phi^\dagger \Phi
}
=
0,
\end{dmath}
and
\begin{dmath}\label{eqn:ProblemSet2Problem2:260}
\delta (\partial_\mu \Phi^\dagger \partial^\mu \Phi)
=
\partial_\mu (\delta \Phi^\dagger) \partial^\mu \Phi
+
\partial_\mu \Phi^\dagger \partial^\mu (\delta \Phi)
=
i \alpha
\lr{
   -\partial_\mu \Phi^\dagger \partial^\mu \Phi
   +
   \partial_\mu \Phi^\dagger \partial^\mu \Phi
}
=
0,
\end{dmath}
so \( \delta \LL = 0 \).
The conserved current, again being careful of the order, is
\begin{dmath}\label{eqn:ProblemSet2Problem2:280}
j^\mu_\alpha
=
\PD{(\partial_\mu \Phi)}{\LL} \delta \Phi
+
\delta \Phi^\dagger
\PD{(\partial_\mu \Phi^\dagger)}{\LL}
=
i \alpha
\lr{
   (\partial^\mu \Phi^\dagger) \Phi
   -
   \Phi^\dagger (\partial^\mu \Phi)
}.
\end{dmath}

\makeSubAnswer{}{qft:problemSet2:2b}
Let's work with the individual fields so the commutators can be computed more easily.  Expanding out the matrices, we have
\begin{dmath}\label{eqn:ProblemSet2Problem2:300}
Q^a
= \frac{i}{2} \int d^3 x
\lr{
   \partial^0 \Phi^\dagger \sigma^a \Phi
   -
   \Phi^\dagger \sigma^a \partial^0 \Phi
}
=
\frac{i}{2} \int d^3 x
\lr{
   \Pi^\dagger \sigma^a \Phi
   -
   \Phi^\dagger \sigma^a \Pi
}
=
\frac{i}{2} \int d^3 x
\lr{
   \pi^\dagger_r \sigma^a_{rs} \phi_s
   -
   \phi^\dagger_r \sigma^a_{rs} \pi_s
}.
\end{dmath}
To simplify the commutator expansion, assume that
\( r,s \) indexed functions are functions of \( \Bx \) and
\( m,n \) indexed functions are functions of \( \By \), for
\begin{dmath}\label{eqn:ProblemSet2Problem2:320}
\antisymmetric{Q^a}{Q^b}
=
-\frac{1}{4} \int d^3 x d^3 y
\sigma^a_{rs}
\sigma^b_{mn}
\antisymmetric
{
   \pi^\dagger_r \phi_s
   -
   \phi^\dagger_r \pi_s
}
{
   \pi^\dagger_m \phi_n
   -
   \phi^\dagger_m \pi_n
}
=
\frac{1}{4} \int d^3 x d^3 y
\sigma^a_{rs}
\sigma^b_{mn}
\lr{
   \antisymmetric
   {
      \pi^\dagger_r \phi_s
   }
   {
      \phi^\dagger_m \pi_n
   }
   +
   \antisymmetric
   {
      \phi^\dagger_r \pi_s
   }
   {
      \pi^\dagger_m \phi_n
   }
}
=
\frac{1}{4} \int d^3 x d^3 y
\sigma^a_{rs}
\sigma^b_{mn}
\lr{
      \pi^\dagger_r
      \phi^\dagger_m
      \phi_s
      \pi_n
   -
      \phi^\dagger_m
      \pi^\dagger_r
      \pi_n
      \phi_s
   +
      \phi^\dagger_r
      \pi^\dagger_m
      \pi_s
      \phi_n
   -
      \pi^\dagger_m
      \phi^\dagger_r
      \phi_n
      \pi_s
}
=
\frac{1}{4} \int d^3 x d^3 y
\sigma^a_{rs}
\sigma^b_{mn}
\lr{
   \lr{
      \phi^\dagger_m
      \pi^\dagger_r
+
      \antisymmetric{ \pi^\dagger_r }{ \phi^\dagger_m}
   }
      \phi_s
      \pi_n
   -
      \phi^\dagger_m
      \pi^\dagger_r
      \pi_n
      \phi_s
   +
   \lr{
         \pi^\dagger_m
         \phi^\dagger_r
      +  \antisymmetric{ \phi^\dagger_r}{ \pi^\dagger_m}
   }
      \pi_s
      \phi_n
   -
      \pi^\dagger_m
      \phi^\dagger_r
      \phi_n
      \pi_s
}
=
\frac{1}{4} \int d^3 x d^3 y
\sigma^a_{rs}
\sigma^b_{mn}
\lr{
     \phi^\dagger_m \pi^\dagger_r \antisymmetric{ \phi_s}{ \pi_n}
   + \antisymmetric{ \pi^\dagger_r }{ \phi^\dagger_m} \phi_s \pi_n
   + \pi^\dagger_m \phi^\dagger_r \antisymmetric{ \pi_s}{ \phi_n}
   + \antisymmetric{ \phi^\dagger_r}{ \pi^\dagger_m} \pi_s \phi_n
}.
\end{dmath}
Each of these commutators has a \( \delta(\Bx - \By) \) term, leaving
\begin{dmath}\label{eqn:ProblemSet2Problem2:n}
\antisymmetric{Q^a}{Q^b}
=
\frac{i}{4} \int d^3 x
\sigma^a_{rs}
\sigma^b_{mn}
\lr{
     \phi^\dagger_m \pi^\dagger_r \delta_{sn}
   - \delta_{rm} \phi_s \pi_n
   - \pi^\dagger_m \phi^\dagger_r \delta_{sn}
   + \delta_{rm} \pi_s \phi_n
}
=
\frac{i}{4} \int d^3 x
   \sigma^a_{rs}
\lr{
   \sigma^b_{ms}
   \lr{
        \phi^\dagger_m \pi^\dagger_r
      - \pi^\dagger_m \phi^\dagger_r
   }
+
   \sigma^b_{rn}
   \lr{
        \pi_s \phi_n
      - \phi_s \pi_n
   }
}
=
\frac{i}{4} \int d^3 x
\lr{
   (\phi^\dagger_m \sigma^b_{ms})
   (\pi^\dagger_r \sigma^a_{rs})
-
   (\pi^\dagger_m \sigma^b_{ms})
   (\phi^\dagger_r \sigma^a_{rs})
+
   (\sigma^a_{rs} \pi_s )
   (\sigma^b_{rn} \phi_n)
-
   (\sigma^a_{rs} \phi_s )
   (\sigma^b_{rn} \pi_n)
}
=
\frac{i}{4} \int d^3 x
\lr{
   \Phi^\dagger \sigma^b \sigma^a \Pi
-
   \Pi^\dagger \sigma^b \sigma^a \Phi
+
   \Pi^\dagger \sigma^a \sigma^b \Phi
-
   \Phi^\dagger \sigma^a \sigma^b \Pi
}
=
\frac{i}{4} \int d^3 x
\lr{
   \Pi^\dagger \antisymmetric{\sigma^a}{\sigma^b} \Phi
   -\Phi^\dagger \antisymmetric{\sigma^a}{\sigma^b} \Pi
}
=
\frac{i}{4} \int d^3 x
\lr{
   \Pi^\dagger \antisymmetric{\sigma^a}{\sigma^b} \Phi
   -\Phi^\dagger \antisymmetric{\sigma^a}{\sigma^b} \Pi
}
=
-\frac{1}{2} \int d^3 x
\epsilon^{a b c}
\lr{
   \Pi^\dagger \sigma^c \Phi
   -\Phi^\dagger \sigma^c \Pi
}
=
i \epsilon^{a b c} Q^c,
\end{dmath}
as desired.
\makeSubAnswer{}{qft:problemSet2:2c}
TODO.
\makeSubAnswer{}{qft:problemSet2:2d}
TODO.
\makeSubAnswer{}{qft:problemSet2:2e}
TODO.
\makeSubAnswer{}{qft:problemSet2:2f}
TODO.
}

      %
% Copyright � 2018 Peeter Joot.  All Rights Reserved.
% Licenced as described in the file LICENSE under the root directory of this GIT repository.
%
\makeoproblem{\( SU(2)_L \times SU(2)_R\), realized in the Wigner and Nambu-Goldstone modes.}{qft:problemSet2:3}{2018 HW2.III}{
Consider now our Lagrangian
\cref{eqn:ProblemSet2Problem2:50}
and imagine that \( m^2 < 0\), for whatever reason (nobody knows, really), while \( \lambda\) is still positive. This now becomes the Higgs Lagrangian of the Standard Model.

\makesubproblem{}{qft:problemSet2:3a}
Show that the classical potential in
\cref{eqn:ProblemSet2Problem2:50}
now becomes:
\begin{dmath}\label{eqn:ProblemSet2Problem3:20}
V = -\Abs{m^2}
\trace{H^\dagger H}
+ \lambda
\lr{ \trace{H^\dagger H} }^2
= \lambda \lr{
   \Abs{\phi_1}^2
   +
   \Abs{\phi_2}^2
   -
   \frac{\Abs{m^2}}{2 \lambda}
}^2
+ \text{const}.
\end{dmath}
\makesubproblem{}{qft:problemSet2:3b}
Clearly, there are extrema of the potential when
\(
   \Abs{\phi_1}^2
   +
   \Abs{\phi_2}^2
= 0 \)
and when
\(
   \Abs{\phi_1}^2
   +
   \Abs{\phi_2}^2
=
   \frac{\Abs{m^2}}{2 \lambda}
 \)
The second one has, clearly, smaller energy density. To quantize the theory, we now have to choose which classical minimum to expand around. Show that, if we expand around
\(
   \Abs{\phi_1}^2
   +
   \Abs{\phi_2}^2
= 0 \)
, we will find that the \( \phi _{1,2} \) excitations are tachyons, even classically. This signals an instability, rather than a faster-than-light propagation and shows that we have chosen the wrong value of \( \Phi \) to build our quantum theory.
\makesubproblem{}{qft:problemSet2:3c}
Thus, consider the
\(
   \Abs{\phi_1}^2
   +
   \Abs{\phi_2}^2
=
   \frac{\Abs{m^2}}{2 \lambda}
 \)
minimum of \( V \). This is really a set of minima. In fact
the set parameterized by
\(
   \Abs{\phi_1}^2
   +
   \Abs{\phi_2}^2
=
   \text{const}
 \)
is also known as a three sphere (\(S^3\), embedded in a four-dimensional space parameterized by
\(\psi^{1\cdots4}\) - not the spacetime!). To build the quantum theory, we will choose a point on this three sphere (a.k.a. the ``vacuum manifold'' - the set of field values that minimize the potential). We will now study the small fluctuations around the chosen point and the spectrum of the theory in this vacuum. There is an infinite number of parameterizations that can be used to do this, but I will suggest one that makes the symmetries the clearest.
 Thus, use the \(H\)-representation and take
\begin{dmath}\label{eqn:ProblemSet2Problem3:40}
H(x) = \frac{\Abs{m}}{2\sqrt{\lambda}} ( 1 + h(x) ) e^{i \phi^a(x) \sigma^a }
\end{dmath}
The logic here is as follows. When \( h(x)\) and \( \phi^a(x) \) vanish (i.e. there are no excitations), the
parameterization
\cref{eqn:ProblemSet2Problem3:40}
is equivalent, by
\cref{eqn:ProblemSet2Problem2:40}
, to taking a specific point on the vacuum manifold,
i.e. the one where \( \phi_1 = 0 \) and \( \phi_2 = \Abs{m}/\sqrt{2\lambda} \).
The fields \( h(x) \) and \( \phi^a(x) \) parameterize the
fluctuations around this ground state (for sure, they can be mapped - the map is nonlinear - to the fluctuations of the fields
\( \phi_{1,2} \) around the chosen vacuum value for \( \phi_2\).\footnote{As in classical mechanics, which variables one uses to describe physics is a matter of choice and convenience. The Euler-Lagrange equations have the property that they are invariant under changes of variables, so long as no singularity occurs in the process. In fact, one of the main motivations of using Lagrangians in classical mechanics is that the change of variables is much easier to do. In other words, it is much easier to first transform the Lagrangian to spherical coordinates and then find the Euler-Lagrange equations then to transform the equations found in Cartesian coordinates to spherical coordinates (in the latter case you need to differentiate twice...). Invariance of physics under nonsingular changes of variables in the Lagrangian is, of course, inherited in field theory.}
What you will do now is take the form
\cref{eqn:ProblemSet2Problem3:40}
, plug it into the Lagrangian
\cref{eqn:ProblemSet2Problem2:50}
with
\( m^2 = -\Abs{m^2}\),
and expand what you find to second order in the fields \( h(x) \) and \( \phi^a(x)\).
Show that the field \( h(x) \) has a mass and find an expression for it.
Show that the fields \( \phi^a(x) \) remain massless and that their Lagrangian (not just to quadratic order) only contains derivatives.

The latter point can be seen pretty simply by noting that \( H(x) \) from
\cref{eqn:ProblemSet2Problem3:40}
can be written as
\begin{dmath}\label{eqn:ProblemSet2Problem3:60}
H(x) = \frac{\Abs{m}}{2 \sqrt{ \lambda } }\Omega(x) ( 1 + h(x) ),
\end{dmath}
with \( \Omega^\dagger \Omega = 1 \) and \( \det(\Omega(x)) = 1 \).
In this parameterization \( \Omega(x) \) fluctuations
  correspond to going around the vacuum manifold \( S^3 \), while the \( h(x) \) fluctuations are along the ``radial'' directions away from the minimum.
The latter cost energy, hence \( h \) is massive (the Higgs field!), while the \( \Omega(x) \) only cost energy if the x-dependence is nontrivial.
The \( \phi^a(x)\) (or \( \Omega(x) \)) are equivalent parameterizations of the Goldstone fields. What you found here is an example of a general story: if a theory has a continuous symmetry, which is not a symmetry of the ground state, there is a number of massless Goldstone (or Nambu-Goldstone) modes. For internal symmetries like the ones we are considering here, their number is equal to the number of broken generators.

In the Standard Model, \(h(x)\) is indeed the Higgs field. The fields \(\phi^a(x)\) actually become the longitudinal components of the W and Z-bosons (one usually says that they are ``eaten'', a manifestation of the Landau-Anderson-Higgs-Brout-Englert-Guralnik-Hagen-... mechanism).

\makesubproblem{}{qft:problemSet2:3d}
One question that was not discussed and remained a bit obscure is that of the unbroken part of the symmetry. The original Lagrangian has \( SU(2)_L \times SU(2)_R \) symmetry. The value of
\( H(x) \) in the vacuum, denoted by \( \expectation{H}\), is given by
\cref{eqn:ProblemSet2Problem3:40}
with \( h = \phi^a = 0 \) and is
 \( \expectation{H} \sim\)
unit matrix.
Show that, while  \( \expectation{H} \) is not invariant under \( SU(2)_L \times SU(2)_R \) for arbitrary \( SU(2)_L \) and \( SU(2)_R \) transformations, it is invariant under
\cref{eqn:ProblemSet2Problem2:60}
with
\( U_L = U_R\). Such \( SU(2)_L \times SU(2)_R \) transformations with \( U_L = U_R \) are called ``diagonal'' or ``vector'' \( SU(2)_V \) transformations.
These remain unbroken in the vacuum. In the electroweak theory, the third component of \( SU(2)_V \) is identified with electromagnetic \( U(1)\).
Show that the current associated with \( SU(2)_V \) transformations has the form:
\begin{dmath}\label{eqn:ProblemSet2Problem3:80}
j_\mu^{V,a} = \frac{i}{2} \trace{\lr{
\partial_\mu H^\dagger \antisymmetric{\sigma^a}{H}
+
\partial_\mu H \antisymmetric{\sigma^a}{H^\dagger}
}}
\end{dmath}
Show also that the other ``linear'' combination of \( SU(2)_L \) and \( SU(2)_R \),
\cref{eqn:ProblemSet2Problem2:60}
with \( U_R = U_L^\dagger \) corresponds to the current (not conserved!) usually called the ``axial current''
\begin{dmath}\label{eqn:ProblemSet2Problem3:120}
j_\mu^{A,a} = \frac{i}{2} \trace{
\partial_\mu H^\dagger \symmetric{\sigma^a}{H} - \partial_\mu H \symmetric{\sigma^a}{H^\dagger},
}
\end{dmath}
where \( \symmetric{A}{B} = AB + BA \) denotes the anticommutator.

\makesubproblem{}{qft:problemSet2:3e}
Show that to linear order in the fields \( h(x),\phi^a(x) \), the a-th axial current is simply
\begin{dmath}\label{eqn:ProblemSet2Problem3:100}
j^{A,a} \sim \expectation{H} \partial_\mu \phi^a,
\end{dmath}
and find the constant in front. Thus, when the quantum operator corresponding to
\cref{eqn:ProblemSet2Problem3:100} % (12)
acts on the vacuum, it creates a quantum of the Goldstone boson (times the momentum and the ``Goldstone boson decay constant'' which is really equal to  \( \expectation{H} \)).

Show also that, to leading nontrivial order in the fields, the conserved vector current \( j^{V,a} \) is
quadratic in the fields \( \phi^a\).

In QCD, the relation
\cref{eqn:ProblemSet2Problem3:100} % (12)
and the algebra of the currents \( j^{V,A} \) constitute the basis of an approach to
soft-pion physics (soft means low energy) known as ``current algebra''.

Here, we studied the Nambu-Goldstone mode. In the Wigner mode, when \( m^2 > 0\), there are no massless particles, as is easy to convince yourselves.
\index{Nambu-Goldstone}
\index{Wigner mode}
\index{Higgs}
\index{W-boson}
\index{Z-boson}
} % makeproblem

\makeanswer{qft:problemSet2:3}{
\withproblemsetsParagraph{
\makeSubAnswer{}{qft:problemSet2:3a}
To expand the potential note that
\begin{dmath}\label{eqn:ProblemSet2Problem3:140}
\trace{\lr{
H^\dagger H
}}
=
\inv{2}
\trace{\lr{
\begin{bmatrix}
-i \Phi^\T \sigma^2 \\
\Phi^\dagger
\end{bmatrix}
\begin{bmatrix}
i \sigma^2 \Phi^\conj & \Phi
\end{bmatrix}
}}
=
\inv{2}\lr{ \Phi^\T \Phi^\dagger + \Phi^\dagger \Phi }
=
\inv{2}\lr{
   \phi_1 \phi_1^\conj + \phi_2 \phi_2^\conj + \phi_1^\conj \phi_1 + \phi_2^\conj \phi^2
}
=
\Abs{\phi_1}^2 + \Abs{\phi_2}^2,
\end{dmath}
so we have
\begin{dmath}\label{eqn:ProblemSet2Problem3:160}
V = -\Abs{m}^2
\trace{\lr{
H^\dagger H
}}
+ \lambda\lr{
\trace{\lr{
H^\dagger H
}}}^2
=
-
\Abs{m}^2
\lr{
   \Abs{\phi_1}^2 + \Abs{\phi_2}^2
}
+ \lambda
\lr{
   \Abs{\phi_1}^2 + \Abs{\phi_2}^2
}^2
=
\lambda\lr{
   \lr{
      \Abs{\phi_1}^2 + \Abs{\phi_2}^2
   }^2
   -
   \frac{\Abs{m}^2 }{\lambda}
   \lr{
      \Abs{\phi_1}^2 + \Abs{\phi_2}^2
   }
}.
\end{dmath}
Completing the square gives
\begin{dmath}\label{eqn:ProblemSet2Problem3:180}
V
=
\lambda\lr{
      \Abs{\phi_1}^2 + \Abs{\phi_2}^2
   -
   \frac{\Abs{m}^2 }{2\lambda}
}^2
   -
\lambda
\lr{
   \frac{\Abs{m}^2 }{2\lambda}
}^2,
\end{dmath}
which proves the result and shows that the constant is \( - \frac{\Abs{m}^4 }{4\lambda} \).
\makeSubAnswer{}{qft:problemSet2:3b}
From \cref{eqn:ProblemSet2Problem3:160} the first order expansion, ignoring constant terms, around \( \Abs{\phi_1}^2 + \Abs{\phi_2}^2 = 0 \) is
\begin{equation}\label{eqn:ProblemSet2Problem3:360}
V = -\Abs{m^2} \lr{ \Abs{\phi_1}^2 + \Abs{\phi_2}^2 } = -\Abs{m^2} \Phi^\dagger \Phi.
\end{equation}
The Lagrangian density, to first order, may be written in the compact form
\begin{dmath}\label{eqn:ProblemSet2Problem3:200}
\LL = \partial_\mu \Phi^\dagger \partial^\mu \Phi + \Abs{m}^2 \Phi^\dagger \Phi.
\end{dmath}
The equations of motion are
\begin{dmath}\label{eqn:ProblemSet2Problem3:220}
\begin{aligned}
\partial_\mu \partial^\mu \Phi &= \Abs{m}^2 \Phi \\
\partial_\mu \partial^\mu \Phi^\dagger &= \Abs{m}^2 \Phi^\dagger
\end{aligned},
\end{dmath}
or, \( \partial_\mu \partial^\mu \psi = \Abs{m}^2 \psi \) for any \( \psi \in \phi_1, \phi_2, \phi_1^\conj, \phi_2^\conj \).

Suppose that one of these wave functions has a Fourier transform representation
\begin{dmath}\label{eqn:ProblemSet2Problem3:240}
\psi(x) = \int \frac{d^4 p}{(2\pi)^4} e^{i p \cdot x} \tilde{\psi}.
\end{dmath}
Such a solution must satisfy the equations of motion
\begin{dmath}\label{eqn:ProblemSet2Problem3:260}
0
=
\lr{
   \partial_{tt} - \spacegrad^2 - \Abs{m^2}
}
\psi
=
\lr{
   \partial_{tt} - \spacegrad^2 - \Abs{m^2}
}
\int \frac{d^4 p}{(2\pi)^4} e^{i \omega t - i \Bp \cdot \Bx} \tilde{\psi}.
=
\int \frac{d^4 p}{(2\pi)^4}
\lr{
   (i\omega)^2 - (-i \Bp)^2 -\Abs{m}^2
}
e^{i \omega t - i \Bp \cdot \Bx} \tilde{\psi},
\end{dmath}
so
\begin{dmath}\label{eqn:ProblemSet2Problem3:280}
0 = -\omega^2 + \Bp^2 - \Abs{m}^2,
\end{dmath}
or
\begin{dmath}\label{eqn:ProblemSet2Problem3:300}
\omega = \sqrt{\Bp^2 - \Abs{m}^2}.
\end{dmath}
Any \( \Norm{\Bp} < \Abs{m} \) results in an imaginary angular frequency.  For example, at \( \Bp = 0 \), we have
\begin{dmath}\label{eqn:ProblemSet2Problem3:320}
\omega = \pm i \Abs{m}.
\end{dmath}
In particular
\begin{dmath}\label{eqn:ProblemSet2Problem3:340}
p_0 x^0
=
\omega t
= \pm i \Abs{m} t
= \pm \Abs{m} ( i t ).
\end{dmath}
We see that the angular momentum constraint on the system \cref{eqn:ProblemSet2Problem3:280} results in the imaginary time that is characteristic of tachonic solutions.

\makeSubAnswer{}{qft:problemSet2:3c}
It seems reasonable that we can assume that \( h(x) \) and \( \phi^a(x) \) in
\cref{eqn:ProblemSet2Problem3:40} are all real valued scalar (non-matrix) functions.  That is \( h(x) \) has the role of radial extension or compression of the field magnitude, and the exponential is of the form \( e^{i \Bsigma \cdot \Bphi(x) } \), a matrix valued rotation operator, where \( \Bphi = (\phi^1, \phi^2, \phi^3)\).  Given that assumption, \( H^\dagger H \) can be computed with relative ease, and has only radial dependence
\begin{dmath}\label{eqn:ProblemSet2Problem3:380}
\trace{\lr{H^\dagger H}}
=
\frac{\Abs{m}^2}{4 \lambda} (1 + h(x))^2 \trace{\lr{ e^{-i \Bsigma \cdot \Bphi} e^{i \Bsigma \cdot \Bphi} }}
=
\frac{\Abs{m}^2}{4 \lambda} (1 + h(x))^2 \trace{\BOne}
=
\frac{\Abs{m}^2}{2 \lambda} (1 + h)^2.
\end{dmath}
For the derivative quadratic form, it is expedient to use the form \cref{eqn:ProblemSet2Problem3:60}, which gives
\begin{dmath}\label{eqn:ProblemSet2Problem3:400}
\partial_\mu H^\dagger \partial^\mu H
=
\frac{\Abs{m}^2}{4 \lambda}
\lr{
   \partial_\mu h \Omega^\dagger
   + (1 + h) \partial_\mu \Omega^\dagger
}
\lr{
   \partial^\mu h \Omega
   + (1 + h)
\partial^\mu \Omega
}
=
\frac{\Abs{m}^2}{4 \lambda}
\lr{
   \partial_\mu h \Omega^\dagger \partial^\mu h \Omega
   + (1 + h)
      \lr{
         \partial_\mu h
         \Omega^\dagger (\partial^\mu \Omega)
       +
         \partial^\mu h
         (\partial_\mu \Omega^\dagger) \Omega
      }
   + (1 + h)^2 \partial_\mu \Omega^\dagger \partial^\mu \Omega
}
\end{dmath}
where we have made the usual assumptions that the independent fields \((h, \Omega)\) commute.
Because \( \Omega^\dagger \Omega = 1 \), we have
\begin{dmath}\label{eqn:ProblemSet2Problem3:480}
\partial_\mu h
\Omega^\dagger (\partial^\mu \Omega)
 +
\partial^\mu h
(\partial_\mu \Omega^\dagger) \Omega
=
\partial_\mu h
\lr{
   \Omega^\dagger (\partial^\mu \Omega)
    +
   (\partial^\mu \Omega^\dagger) \Omega
}
=
\partial_\mu h
\lr{
   \partial^\mu (\Omega^\dagger \Omega) - (\partial^\mu \Omega^\dagger) \Omega
    +
   (\partial^\mu \Omega^\dagger) \Omega
}
=
   \partial^\mu (1)
= 0.
\end{dmath}
All the cross terms with both \( h \) and \( \Omega \) derivatives are zero (to all orders, not just quadratic).

Taking traces (and using cyclic permutation of the matrices in the trace operations),
the Lagrangian density is now determined to quadratic order
\begin{dmath}\label{eqn:ProblemSet2Problem3:500}
\LL =
\frac{\Abs{m}^2}{2 \lambda}
   \partial_\mu h \partial^\mu h
+
\frac{\Abs{m}^2}{4 \lambda}
   \trace{\lr{
      \partial_\mu \Omega^\dagger \partial^\mu \Omega
   }}
+ \Abs{m}^2
\frac{\Abs{m}^2}{2 \lambda} \lr{ 1 + h }^2
- \lambda
\lr{\frac{\Abs{m}^2}{2 \lambda}}^2
\lr{ 1 + h }^4.
\end{dmath}
Observe that the Lagrangian density can be split into two independent parts, one for the radial field \( h \), and another for the rotation field \( \Omega \).  Rescaling to drop the common constant factor \( \Abs{m}^2/2\lambda \), the radial Lagrangian is
\begin{dmath}\label{eqn:ProblemSet2Problem3:560}
\LL_h
=
\partial_\mu h \partial^\mu h
+ \Abs{m}^2
\lr{ 1 + h }^2
-
\frac{\Abs{m}^2}{2}
\lr{ 1 + h }^4
=
\partial_\mu h \partial^\mu h
-\frac{\Abs{m}^2}{2}
\lr{
   \lr{ 1 + h }^4
   -2 \lr{ 1 + h }^2
}
=
\partial_\mu h \partial^\mu h
-\frac{\Abs{m}^2}{2}
\lr{
   \lr{ 1 + h }^2 - 1
}^2
+ \cancel{\text{const.}}
=
\partial_\mu h \partial^\mu h
-\frac{\Abs{m}^2}{2}
\lr{ 2 h + h^2
}^2
=
\partial_\mu h \partial^\mu h
- \frac{\Abs{m}^2}{2} h^2
\lr{ 2 + h }^2
=
\partial_\mu h \partial^\mu h
- 2 \Abs{m}^2 h^2
+ O(h^3).
\end{dmath}
This shows that the mass of the \( h \) field is \( \sqrt{2} \Abs{m} \).

The only remaining task is to express the Lagrangian density for \( \phi^a \) in terms of those field instead of \( \Omega \).  To evaluate those derivatives, we can utilize a first order Taylor expansion

\begin{dmath}\label{eqn:ProblemSet2Problem3:580}
\partial_\mu \Omega
=
\partial_\mu \lr{ \BOne + i \Bsigma \cdot \Bphi }
=
i \Bsigma \cdot \partial_\mu \Bphi,
\end{dmath}
so the rotation Lagrangian density is
\begin{dmath}\label{eqn:ProblemSet2Problem3:600}
\LL_\Bphi =
\inv{2} \trace{\lr{
(-i \Bsigma \cdot \partial_\mu \Bphi)
(i \Bsigma \cdot \partial^\mu \Bphi)
}}
=
(\partial_\mu \Bphi) \cdot (\partial^\mu \Bphi)
=
(\partial_\mu \phi^a) (\partial^\mu \phi^a),
\end{dmath}
where we use the fact that \( \trace{\lr{(\Bsigma \cdot \Bx)(\Bsigma \cdot \By)}} = 2 \Bx \cdot \By \).

The full Lagrangian density, to quadratic order, is
\boxedEquation{eqn:ProblemSet2Problem3:620}{
\LL
= \LL_h + \LL_\Bphi
=
\partial_\mu h \partial^\mu h
- 2 \Abs{m}^2 h^2
+
\partial_\mu \phi^a \partial^\mu \phi^a.
}

\makeSubAnswer{}{qft:problemSet2:3d}
\paragraph{Problem statement inconsistency.}
In the problem statement \( \expectation{H} \) is defined as a \( 2 \times 2 \) unit matrix scaled by \( \Abs{m}/2 \sqrt{\lambda} \), but later when used in the statement of the axial current, it appears as a number (since the current is a number, and not a matrix).  In this solution I've used \( \expectation{H} \) as just the numeric factor, and dropped the identity matrix factor.

\paragraph{Setup.}
This problem is easiest if we can work directly with in matrix notation, but first need to know how to express the current.  Given matrix elements \( H_{ab}, H^\conj_{ab} \), that current is
\begin{dmath}\label{eqn:ProblemSet2Problem3:640}
j^\mu
=
\PD{(\partial_\mu H_{ij})}{\LL} \delta H_{ij}
+
\PD{(\partial_\mu H^\conj_{ij})}{\LL} \delta H^\conj_{ij}.
\end{dmath}
The trace of a matrix product in terms of the respective matrix elements is
\begin{equation}\label{eqn:ProblemSet2Problem3:660}
\trace{\lr{ A B }}
=
A_{ik} B_{kj} \delta_{ij}
=
A_{ij} B_{ji},
\end{equation}
so the Kinetic portion of the Lagrangian density expands as
\begin{equation}\label{eqn:ProblemSet2Problem3:680}
\trace{\lr{
\partial_\mu H^\dagger \partial^\mu H
}}
=
\partial_\mu (H^\dagger)_{ji} \partial^\mu H_{ij}
=
\partial_\mu H^\conj_{ij} \partial^\mu H_{ij}.
\end{equation}
We can now put the current \cref{eqn:ProblemSet2Problem3:640} into matrix form
\begin{dmath}\label{eqn:ProblemSet2Problem3:700}
j^\mu
=
\partial^\mu H^\conj_{ij} \delta H_{ij}
+
\delta H^\conj_{ij} \partial^\mu H_{ij}
=
\partial^\mu (H^\dagger)_{ji} \delta H_{ij}
+
\delta (H^\dagger)_{ji} \partial^\mu H_{ij}
=
\trace{\lr{
\partial^\mu H^\dagger \delta H
+
\delta H^\dagger \partial^\mu H
}}.
\end{dmath}

\paragraph{Vector current.}

With \( H \rightarrow U_L H U_L^\dagger \), the \( H \) variation is
\begin{dmath}\label{eqn:ProblemSet2Problem3:720}
\delta H
=
H' - H
\approx
\lr{ 1 + \frac{i}{2} \Bsigma \cdot \Bomega } H
\lr{ 1 - \frac{i}{2} \Bsigma \cdot \Bomega }
- H
=
\frac{i}{2} (\Bsigma \cdot \Bomega) H
 - \frac{i}{2} H (\Bsigma \cdot \Bomega)
+ O(\Bomega^2)
=
\frac{i}{2} \antisymmetric{\Bsigma \cdot \Bomega}{H},
\end{dmath}
and its conjugate is
\begin{equation}\label{eqn:ProblemSet2Problem3:740}
\delta H^\dagger
=
-\frac{i}{2} \antisymmetric{H^\dagger}{\Bsigma \cdot \Bomega}
=
\frac{i}{2} \antisymmetric{\Bsigma \cdot \Bomega}{H^\dagger}.
\end{equation}

Putting the pieces together gives
\begin{dmath}\label{eqn:ProblemSet2Problem3:860}
j_\mu^{V, \Bomega}
= \frac{i}{2}
\trace{\lr{
\partial_\mu H^\dagger
\antisymmetric{\Bsigma \cdot \Bomega}{H}
+
\antisymmetric{\Bsigma \cdot \Bomega}{H^\dagger}
\partial_\mu H
}}
= \frac{i \omega^a}{2}
\trace{\lr{
\partial_\mu H^\dagger
\antisymmetric{\sigma^a}{H}
+
\partial_\mu H
\antisymmetric{\sigma^a}{H^\dagger}
}},
\end{dmath}
so setting \( j_\mu^{V, \Bomega} = \omega^a j_\mu^{V,a} \) to factor out the \( \omega^a \)'s, provides the desired result.

\paragraph{Axial current.}
This is only cosmetically different from the Vector current.

With \( H \rightarrow U_L H U_L \), the \( H \) variation is
\begin{dmath}\label{eqn:ProblemSet2Problem3:780}
\delta H
=
H' - H
\approx
\lr{ 1 + \frac{i}{2} \Bsigma \cdot \Bomega } H
\lr{ 1 + \frac{i}{2} \Bsigma \cdot \Bomega }
- H
=
\frac{i}{2} (\Bsigma \cdot \Bomega) H
 + \frac{i}{2} H (\Bsigma \cdot \Bomega)
+ O(\Bomega^2)
=
\frac{i}{2} \symmetric{\Bsigma \cdot \Bomega}{H},
\end{dmath}
and its conjugate is
\begin{equation}\label{eqn:ProblemSet2Problem3:800}
\delta H^\dagger
=
-\frac{i}{2} \symmetric{\Bsigma \cdot \Bomega}{H^\dagger}.
\end{equation}

Putting the pieces together gives
\begin{dmath}\label{eqn:ProblemSet2Problem3:880}
j_\mu^{A, \Bomega}
= \frac{i}{2}
\trace{\lr{
\partial_\mu H^\dagger
\symmetric{\Bsigma \cdot \Bomega}{H}
-
\symmetric{\Bsigma \cdot \Bomega}{H^\dagger}
\partial_\mu H
}}
= \frac{i \omega^a}{2}
\trace{\lr{
\partial_\mu H^\dagger
\symmetric{\sigma^a}{H}
-
\partial_\mu H
\symmetric{\sigma^a}{H^\dagger}
}},
\end{dmath}
so setting \( j_\mu^{A, \Bomega} = \omega^a j_\mu^{A,a} \) to factor out the \( \omega^a \)'s, provides the desired result.

\makeSubAnswer{}{qft:problemSet2:3e}
\paragraph{Axial current to first order.}

To first order the \( H \) partial is
\begin{dmath}\label{eqn:ProblemSet2Problem3:820}
\partial_\mu H
=
\expectation{H}
\lr{
   \partial_\mu h \lr{ 1 + i \Bsigma \cdot \Bphi }
+
   (1 + h) i \Bsigma \cdot \partial_\mu \Bphi
}
=
\expectation{H}
\lr{
   \partial_\mu h
+
   i \Bsigma \cdot \partial_\mu \Bphi
}
+ O(2).
\end{dmath}
Because this has no zero order terms, we need only the zeroth order parts of the anticommutators
\begin{dmath}\label{eqn:ProblemSet2Problem3:840}
\symmetric{\sigma^a}{H}
=
\expectation{H}
(1 + h) \symmetric{\sigma^a}{1 + i \Bsigma \cdot \Bphi}
=
\expectation{H} \symmetric{\sigma^a}{1}
+ O(1)
=
2 \expectation{H} \sigma^a.
\end{dmath}
To first order
\begin{dmath}\label{eqn:ProblemSet2Problem3:900}
j^{A,a}_\mu
=
i \expectation{H}^2 \trace{\lr{
   \lr{
      \partial_\mu h
   -
      i \Bsigma \cdot \partial_\mu \Bphi
   }
   \sigma^a
-
   \lr{
      \partial_\mu h
   +
      i \Bsigma \cdot \partial_\mu \Bphi
   } \sigma^a
}}
=
2 \expectation{H}^2 \trace{\lr{ \sigma^b \partial_\mu \phi^b \sigma^a }}.
\end{dmath}
Since \( \trace{\lr{ \sigma^a \sigma^b }} = 2 \delta_{ab} \), this reduces to
\begin{dmath}\label{eqn:ProblemSet2Problem3:920}
j^{A,a}_\mu
=
\expectation{H} \lr{ 4 \expectation{H} } \partial_\mu \phi^a,
\end{dmath}
so the ``constant in front'' is \( 4 \expectation{H} = 2 \Abs{m}/\sqrt{\lambda} \).

\paragraph{Vector current to second order.}
%%For this part of the problem, we are to show that, to leading order in the fields, the conserved vector current is quadratic in \( \phi^a \).  Let's assume that we can drop any \( O(3) \) terms in the current, and look at what terms are left over of lesser order.
%%
To make life less messy, let's write
\begin{dmath}\label{eqn:ProblemSet2Problem3:940}
H = \expectation{H} (1 + h) \Omega,
\end{dmath}
so that
\begin{equation}\label{eqn:ProblemSet2Problem3:960}
\antisymmetric{\sigma^a}{H}
=
\expectation{H}
\antisymmetric{\sigma^a}{(1 + h)\Omega}
=
\expectation{H} h \antisymmetric{\sigma^a}{\Omega}.
\end{equation}
We also have, also to all orders,
\begin{dmath}\label{eqn:ProblemSet2Problem3:980}
\partial_\mu H
=
\expectation{H} \lr{ \partial_\mu h \Omega + (1 + h)\partial_\mu \Omega }
%=
%\expectation{H} \lr{
%   \partial_\mu h \Omega + \partial_\mu \Omega + h \partial_\mu \Omega
%}
\end{dmath}
The current is
\begin{dmath}\label{eqn:ProblemSet2Problem3:1000}
j^{V,a}_\mu
=
\frac{i}{2}
\trace{\lr{
   \partial_\mu H^\dagger \antisymmetric{\sigma^a}{H}
   +
   \partial_\mu H \antisymmetric{\sigma^a}{H^\dagger}
}}
=
\frac{i}{2}
\expectation{H}^2
\trace{\lr{
   \lr{
      \partial_\mu h \Omega^\dagger
         + (1 + h) \partial_\mu \Omega^\dagger
   }
   h \antisymmetric{\sigma^a}{\Omega}
+
   \lr{
      \partial_\mu h \Omega
      + (1 + h )\partial_\mu \Omega
   }
   h \antisymmetric{\sigma^a}{\Omega^\dagger}
}}
=
\frac{i}{2}
\expectation{H}^2
\lr{
   (\partial_\mu h) h
   \trace{\lr{
         \Omega^\dagger \antisymmetric{\sigma^a }{\Omega}
         +
         \Omega \antisymmetric{\sigma^a }{\Omega^\dagger}
   }}
   +
   h(1+h)
   \trace{\lr{
         \partial_\mu \Omega^\dagger \antisymmetric{\sigma^a}{\Omega}
      +
         \partial_\mu \Omega \antisymmetric{\sigma^a}{\Omega^\dagger}
   }}
}
=
\frac{i}{2}
\expectation{H}^2
\lr{
   (\partial_\mu h) h A
   +
   h(1+h) B
},
\end{dmath}
where
\begin{dmath}\label{eqn:ProblemSet2Problem3:1100}
\begin{aligned}
A &=
   \trace{\lr{
         \Omega^\dagger \antisymmetric{\sigma^a }{\Omega}
         +
         \Omega \antisymmetric{\sigma^a }{\Omega^\dagger}
   }} \\
B &=
   \trace{\lr{
         \partial_\mu \Omega^\dagger \antisymmetric{\sigma^a}{\Omega}
      +
         \partial_\mu \Omega \antisymmetric{\sigma^a}{\Omega^\dagger}
   }}.
\end{aligned}
\end{dmath}
The first trace \( A \) is easily shown to be zero
\begin{dmath}\label{eqn:ProblemSet2Problem3:1020}
A
=
\trace{\lr{
      \Omega^\dagger \sigma^a \Omega
      -\Omega^\dagger \Omega \sigma^a
      +
      \Omega \sigma^a \Omega^\dagger
      -
      \Omega \Omega^\dagger \sigma^a
}}
=
\trace{\lr{
   \lr{
        \Omega \Omega^\dagger
      - \Omega^\dagger \Omega
      + \Omega^\dagger \Omega
      - \Omega \Omega^\dagger
   }  \sigma^a
}}
= 0,
\end{dmath}
where cyclic permutation within the trace was used to arrange the terms for easy cancellation \(1 - 1 + 1 - 1 = 0\).

Expanding commutators, and using cyclic permutation in the trace, we have for \( B \)
\begin{dmath}\label{eqn:ProblemSet2Problem3:1040}
B
=
\trace{\lr{
      \partial_\mu \Omega^\dagger \antisymmetric{\sigma^a}{\Omega}
   +
      \partial_\mu \Omega \antisymmetric{\sigma^a}{\Omega^\dagger}
}}
=
\trace{\lr{
     (\partial_\mu \Omega^\dagger) \sigma^a \Omega
   -
     (\partial_\mu \Omega^\dagger) \Omega \sigma^a
   +
     (\partial_\mu \Omega) \sigma^a \Omega^\dagger
   -
     (\partial_\mu \Omega) \Omega^\dagger \sigma^a
}}
=
\trace{\lr{
   \lr{
        \Omega (\partial_\mu \Omega^\dagger )
      - (\partial_\mu \Omega^\dagger) \Omega
      + \Omega^\dagger (\partial_\mu \Omega )
      - (\partial_\mu \Omega) \Omega^\dagger
   } \sigma^a
}}
\end{dmath}
This can be simplified using
\begin{dmath}\label{eqn:ProblemSet2Problem3:1060}
\begin{aligned}
\Omega (\partial_\mu \Omega^\dagger) &= -(\partial_\mu \Omega) \Omega^\dagger \\
\Omega^\dagger (\partial_\mu \Omega) &= -(\partial_\mu \Omega^\dagger) \Omega
\end{aligned},
\end{dmath}
so
\begin{dmath}\label{eqn:ProblemSet2Problem3:1080}
B
=
2
\trace{\lr{
   \lr{
        \Omega^\dagger (\partial_\mu \Omega )
      - (\partial_\mu \Omega) \Omega^\dagger
   } \sigma^a
}}.
\end{dmath}
The derivative \( \partial_\mu \Omega \) has no \( O(0) \) terms, so let's expand the rotation matrix only to \( O(1) \), and then drop any \( O(2) \) terms from \( \partial_\mu \Omega \).  This gives
\begin{dmath}\label{eqn:ProblemSet2Problem3:1120}
B
=
2
\trace{\lr{
   \lr{
        (1 - i \Bsigma \cdot \Bphi) (\partial_\mu \Omega )
      - (\partial_\mu \Omega) (1 - i \Bsigma \cdot \Bphi)
   } \sigma^a
}}
+ O(3)
=
- 2 i
\trace{\lr{
   \lr{
        (\Bsigma \cdot \Bphi) (\partial_\mu \Omega )
      - (\partial_\mu \Omega) (\Bsigma \cdot \Bphi)
   } \sigma^a
}}
=
- 2 i^2
\trace{\lr{
   \lr{
      \sigma^c \phi^c \sigma^b \partial_\mu \phi^b
      -
      \sigma^b \partial_\mu \phi^b \sigma^c \phi^c
   } \sigma^a
}}
+ O(3)
=
2 \phi^c (\partial_\mu \phi^b) \trace{\lr{ \sigma^c \sigma^b \sigma^a }}
-2 (\partial_\mu \phi^b) \phi^c \trace{\lr{ \sigma^b \sigma^c \sigma^a }}
=
-2
\lr{ \phi^c (\partial_\mu \phi^b) + (\partial_\mu \phi^b) \phi^c }
\trace{\lr{ \sigma^1 \sigma^2 \sigma^3 }} \epsilon^{abc}
= - 4 i
\symmetric{ \phi^c}{\partial_\mu \phi^b} \epsilon^{abc},
\end{dmath}
to quadratic order in \( \phi^a \).  The final steps above used the fact that the trace of three Pauli matrices is zero unless they are all different, and \( \trace{\lr{ \sigma^1 \sigma^2 \sigma^3 }} = 2 i \).

The current, to lowest order in \( \phi^a \), and all orders in \( h \), is
\begin{dmath}\label{eqn:ProblemSet2Problem3:1140}
j^{V,a}_\mu
=
2
\expectation{H}^2 h(1 + h) \symmetric{ \phi^c}{\partial_\mu \phi^b} \epsilon^{abc},
\end{dmath}
which is quadratic in \( \phi^a \) as claimed.
}}

      %
% Copyright � 2018 Peeter Joot.  All Rights Reserved.
% Licenced as described in the file LICENSE under the root directory of this GIT repository.
%
\makeproblem{Playing with the non-relativistic limit}{qft:problemSet2:4}{
Consider a real scalar relativistic field theory of mass m with \( \lambda \phi^4 \) interaction. Let there be \( N \) particles of momenta labeled by \( p_1,\cdots, p_N\), whose energies are such that they are insufficient to create any new particles. Nevertheless, the particles can scatter and exchange momenta. In what follows you will study this N-particle nonrelativistic limit in some detail.
\makesubproblem{}{qft:problemSet2:4a}
Write down the Hamiltonian of the field theory, including the interaction term, restricted to the N-particle sector of Hilbert space. (Use the creation and annihilation operator representation, i.e. write the result as sums of products of creation and annihilation operators of particles of various momenta.)
\makesubproblem{}{qft:problemSet2:4b}
Does the resulting Hamiltonian preserve particle number? Is there an associated symmetry? What is the operator that generates it?
\makesubproblem{}{qft:problemSet2:4c}
Consider now the interaction term in your reduced (to the N-particle sector of Hilbert space) Hamiltonian. How does a typical interaction term (for given configurations of momenta) act on an N-particle state? What kinds of scattering processes does it describe?
\makesubproblem{}{qft:problemSet2:4d}
What do you think is the potential, in x-space, that allows the various particles to scatter and exchange momentum? How would you describe the resulting nonrelativistic quantum system to friends who never took QFT but are well-versed in quantum mechanics?
} % makeproblem

\makeanswer{qft:problemSet2:4}{
\makeSubAnswer{}{qft:problemSet2:4a}
The Lagrangian density of a massive scalar field with a \( \lambda \phi^4 \) interaction has the form
\begin{dmath}\label{eqn:ProblemSet2Problem4:20}
\LL = \inv{2} \partial_\mu \phi \partial^\mu \phi - \inv{2} m^2 \phi^2 - \lambda \phi^4.
\end{dmath}
The corresponding Hamiltonian is
\begin{dmath}\label{eqn:ProblemSet2Problem4:40}
H = \inv{2} \int d^3x \lr{ \pi^2 + \frac{m^2}{2} (\spacegrad \phi)^2 + m^2 \phi^2 } + \lambda \int d^3 x \phi^4.
\end{dmath}
In terms of creation and annihilation operators, we know the form of the non-interaction portion of the Hamiltonian, which in normal order is
\begin{dmath}\label{eqn:ProblemSet2Problem4:60}
H_0 = \int \frac{d^3 p}{(2 \pi)^3} \omega_\Bp a_\Bp^\dagger a_\Bp,
\end{dmath}
but the interaction contribution is much messier
\begin{dmath}\label{eqn:ProblemSet2Problem4:80}
H_{\text{int}}
=
\lambda \int d^3 x \frac{ d^3 p d^3 k d^3 q d^3 s}{4 (2 \pi)^{12} \sqrt{
\omega_\Bp  \omega_\Bk  \omega_\Bq  \omega_\Bs
} }
\lr{ a_\Bp e^{-i p \cdot x} + a_\Bp e^{i p \cdot x} }
\lr{ a_\Bk e^{-i k \cdot x} + a_\Bk e^{i k \cdot x} }
\lr{ a_\Bq e^{-i q \cdot x} + a_\Bq e^{i q \cdot x} }
\lr{ a_\Bs e^{-i s \cdot x} + a_\Bs e^{i s \cdot x} }
=
\lambda \int d^3 x \frac{ d^3 p d^3 k d^3 q d^3 s}{4 (2 \pi)^{12} \sqrt{
\omega_\Bp  \omega_\Bk  \omega_\Bq  \omega_\Bs
} }
\lr{ a_\Bp e^{-i \omega_\Bp t + i \Bp \cdot \Bx} + a_\Bp e^{i \omega_\Bp t - i \Bp \cdot \Bx} }
\lr{ a_\Bk e^{-i \omega_\Bk t + i \Bk \cdot \Bx} + a_\Bk e^{i \omega_\Bk t - i \Bk \cdot \Bx} }
\lr{ a_\Bq e^{-i \omega_\Bq t + i \Bq \cdot \Bx} + a_\Bq e^{i \omega_\Bq t - i \Bq \cdot \Bx} }
\lr{ a_\Bs e^{-i \omega_\Bs t + i \Bs \cdot \Bx} + a_\Bs e^{i \omega_\Bs t - i \Bs \cdot \Bx} }
=
\lambda \int \frac{ d^3 p d^3 k d^3 q d^3 s}{4 (2 \pi)^{9} \sqrt{
\omega_\Bp  \omega_\Bk  \omega_\Bq  \omega_\Bs
} }
\lr{
   a_\Bp a_\Bk a_\Bq a_\Bs e^{-i (\omega_\Bp + \omega_\Bk + \omega_\Bq + \omega_\Bs)t} \delta^3( \Bp + \Bk + \Bq + \Bs )
   +
   a_\Bp a_\Bk a_\Bq a_\Bs^\dagger e^{-i (\omega_\Bp + \omega_\Bk + \omega_\Bq - \omega_\Bs)t} \delta^3( \Bp + \Bk + \Bq - \Bs )
   +
   \cdots
   +
   a_\Bp^\dagger a_\Bk^\dagger a_\Bq^\dagger a_\Bs^\dagger e^{-i (-\omega_\Bp - \omega_\Bk - \omega_\Bq - \omega_\Bs)t} \delta^3( -\Bp - \Bk - \Bq - \Bs )
}
=
\lambda \int \frac{ d^3 p d^3 k d^3 q }{4 (2 \pi)^{9}
}
\lr{
   \inv{\sqrt{
      \omega_\Bp  \omega_\Bk  \omega_\Bq  \omega_{-\Bp - \Bk - \Bq}
   }}
   a_\Bp a_\Bk a_\Bq a_{-\Bp -\Bk - \Bq} e^{-i (\omega_\Bp + \omega_\Bk + \omega_\Bq + \omega_{-\Bp -\Bk -\Bq})t}
+
   \inv{\sqrt{
      \omega_\Bp  \omega_\Bk  \omega_\Bq  \omega_{\Bp + \Bk + \Bq}
   }}
   a_\Bp a_\Bk a_\Bq a_{\Bp + \Bk + \Bq}^\dagger e^{-i (\omega_\Bp + \omega_\Bk + \omega_\Bq - \omega_{\Bp + \Bk + \Bq})t}
+
   \cdots
+
   \inv{\sqrt{
      \omega_\Bp  \omega_\Bk  \omega_\Bq  \omega_{-\Bp - \Bk - \Bq}
   }}
   a_\Bp^\dagger a_\Bk^\dagger a_\Bq^\dagger a_{-\Bp -\Bk -\Bq}^\dagger e^{-i (-\omega_\Bp - \omega_\Bk - \omega_\Bq - \omega_{-\Bp -\Bk -\Bq})t}
}.
\end{dmath}
Assuming we can normal order these terms as in \( H_0 \), we can rewrite the interaction as
\begin{dmath}\label{eqn:ProblemSet2Problem4:100}
H_{\text{int}}
=
\lambda \int \frac{ d^3 p d^3 k d^3 q }{4 (2 \pi)^{9} }
\lr{
   \binom{4}{0}
      \inv{\sqrt{
         \omega_\Bp  \omega_\Bk  \omega_\Bq  \omega_{-\Bp - \Bk - \Bq}
      }}
      a_\Bp a_\Bk a_\Bq a_{-\Bp -\Bk - \Bq} e^{-i (\omega_\Bp + \omega_\Bk + \omega_\Bq + \omega_{-\Bp -\Bk -\Bq})t}
   +
   \binom{4}{1}
      \inv{\sqrt{
         \omega_\Bp  \omega_\Bk  \omega_\Bq  \omega_{\Bp - \Bk - \Bq}
      }}
      a_\Bp^\dagger a_\Bk a_\Bq a_{\Bp - \Bk - \Bq} e^{-i (-\omega_\Bp + \omega_\Bk + \omega_\Bq + \omega_{\Bp - \Bk - \Bq})t}
   +
   \binom{4}{2}
      \inv{\sqrt{
         \omega_\Bp  \omega_\Bk  \omega_\Bq  \omega_{\Bp + \Bk - \Bq}
      }}
      a_\Bp^\dagger a_\Bk^\dagger a_\Bq a_{\Bp + \Bk - \Bq} e^{-i (-\omega_\Bp - \omega_\Bk + \omega_\Bq + \omega_{\Bp + \Bk - \Bq})t}
   +
   \binom{4}{3}
      \inv{\sqrt{
         \omega_\Bp  \omega_\Bk  \omega_\Bq  \omega_{\Bp + \Bk _ \Bq}
      }}
      a_\Bp^\dagger a_\Bk^\dagger a_\Bq^\dagger a_{\Bp + \Bk + \Bq} e^{-i (-\omega_\Bp - \omega_\Bk - \omega_\Bq + \omega_{\Bp + \Bk + \Bq})t}
   +
   \binom{4}{4}
      \inv{\sqrt{
         \omega_\Bp  \omega_\Bk  \omega_\Bq  \omega_{-\Bp - \Bk - \Bq}
      }}
      a_\Bp^\dagger a_\Bk^\dagger a_\Bq^\dagger a_{-\Bp -\Bk -\Bq}^\dagger e^{-i (-\omega_\Bp - \omega_\Bk - \omega_\Bq - \omega_{-\Bp -\Bk -\Bq})t}
}
\end{dmath}
If we restrict the allowed momenta to the discrete set \( \Bp \in \setlr{ \Bp_1, \Bp_2, \cdots \Bp_N} \), the total Hamiltonian including the interaction term
takes the form
\begin{dmath}\label{eqn:ProblemSet2Problem4:120}
\text{\(:H:\)} =
\sum_{i = 1}^N \omega_{\Bp_i} a_{\Bp_i}^\dagger a_{\Bp_i}
+
\frac{
\lambda
}{4 }
\sum_{j,m,n = 1}^N
\lr{
   \binom{4}{0}
      \inv{\sqrt{
         \omega_{\Bp_j}  \omega_{\Bp_m}  \omega_{\Bp_n}  \omega_{-{\Bp_j} - {\Bp_m} - {\Bp_n}}
      }}
      a_{\Bp_j} a_{\Bp_m} a_{\Bp_n} a_{-\Bp -\Bk - \Bq} e^{-i (\omega_{\Bp_j} + \omega_{\Bp_m} + \omega_{\Bp_n} + \omega_{-{\Bp_j} -{\Bp_m} -{\Bp_n}})t}
   +
   \binom{4}{1}
      \inv{\sqrt{
         \omega_{\Bp_j}  \omega_{\Bp_m}  \omega_{\Bp_n}  \omega_{{\Bp_j} - {\Bp_m} - {\Bp_n}}
      }}
      a_{\Bp_j}^\dagger a_{\Bp_m} a_{\Bp_n} a_{{\Bp_j} - {\Bp_m} - {\Bp_n}} e^{-i (-\omega_{\Bp_j} + \omega_{\Bp_m} + \omega_{\Bp_n} + \omega_{{\Bp_j} - {\Bp_m} - {\Bp_n}})t}
   +
   \binom{4}{2}
      \inv{\sqrt{
         \omega_{\Bp_j}  \omega_{\Bp_m}  \omega_{\Bp_n}  \omega_{{\Bp_j} + {\Bp_m} - {\Bp_n}}
      }}
      a_{\Bp_j}^\dagger a_{\Bp_m}^\dagger a_{\Bp_n} a_{{\Bp_j} + {\Bp_m} - {\Bp_n}} e^{-i (-\omega_{\Bp_j} - \omega_{\Bp_m} + \omega_{\Bp_n} + \omega_{{\Bp_j} + {\Bp_m} - {\Bp_n}})t}
   +
   \binom{4}{3}
      \inv{\sqrt{
         \omega_{\Bp_j}  \omega_{\Bp_m}  \omega_{\Bp_n}  \omega_{{\Bp_j} + {\Bp_m} - {\Bp_n}}
      }}
      a_{\Bp_j}^\dagger a_{\Bp_m}^\dagger a_{\Bp_n}^\dagger a_{{\Bp_j} + {\Bp_m} + {\Bp_n}} e^{-i (-\omega_{\Bp_j} - \omega_{\Bp_m} - \omega_{\Bp_n} + \omega_{{\Bp_j} + {\Bp_m} + {\Bp_n}})t}
   +
   \binom{4}{4}
      \inv{\sqrt{
         \omega_{\Bp_j}  \omega_{\Bp_m}  \omega_{\Bp_n}  \omega_{-{\Bp_j} - {\Bp_m} - {\Bp_n}}
      }}
      a_{\Bp_j}^\dagger a_{\Bp_m}^\dagger a_{\Bp_n}^\dagger a_{-{\Bp_j} -{\Bp_m} -{\Bp_n}}^\dagger e^{-i (-\omega_{\Bp_j} - \omega_{\Bp_m} - \omega_{\Bp_n} - \omega_{-{\Bp_j} -{\Bp_m} -{\Bp_n}})t}
}.
\end{dmath}
When we did the same sort of calculation for \( (\spacegrad \phi)^2 + m^2 \phi^2 \) all the time dependent terms cancelled nicely, but that isn't obviously the case here.
However, we haven't used the non-relativistic (low energy) constraint.  That constraint can be expressed as \( \Bp^2 \ll m^2 \), in which case \( \omega_\Bp = \sqrt{ \Bp^2 + m^2 } \sim m \), the mass of each of the particles.  Incorporating that into our N-particle Hamiltonian, we have
\begin{dmath}\label{eqn:ProblemSet2Problem4:140}
\text{\(:H:\)} =
\sum_{i = 1}^N \omega_{\Bp_i} a_{\Bp_i}^\dagger a_{\Bp_i}
+
\frac{
\lambda
}{4 m^2 }
\sum_{j,m,n = 1}^N
\lr{
   \binom{4}{0}
      a_{\Bp_j} a_{\Bp_m} a_{\Bp_n} a_{-\Bp -\Bk - \Bq} e^{- 4 i m t}
   +
   \binom{4}{1}
      a_{\Bp_j}^\dagger a_{\Bp_m} a_{\Bp_n} a_{{\Bp_j} - {\Bp_m} - {\Bp_n}} e^{-3 i m t}
   +
   \binom{4}{2}
      a_{\Bp_j}^\dagger a_{\Bp_m}^\dagger a_{\Bp_n} a_{{\Bp_j} + {\Bp_m} - {\Bp_n}}
   +
   \binom{4}{3}
      a_{\Bp_j}^\dagger a_{\Bp_m}^\dagger a_{\Bp_n}^\dagger a_{{\Bp_j} + {\Bp_m} + {\Bp_n}} e^{ 3 i m t }
   +
   \binom{4}{4}
      a_{\Bp_j}^\dagger a_{\Bp_m}^\dagger a_{\Bp_n}^\dagger a_{-{\Bp_j} -{\Bp_m} -{\Bp_n}}^\dagger e^{4 i m t}
}.
\end{dmath}
Presuming there's a good argument to kill off the time dependent terms, the N-sector Hamiltonian is reduced to just
\begin{dmath}\label{eqn:ProblemSet2Problem4:160}
\text{\(:H:\)} =
\sum_{i = 1}^N \omega_{\Bp_i} a_{\Bp_i}^\dagger a_{\Bp_i}
+
\frac{
3 \lambda
}{2 m^2 }
\sum_{j,m,n = 1}^N
      a_{\Bp_j}^\dagger a_{\Bp_m}^\dagger a_{\Bp_n} a_{{\Bp_j} + {\Bp_m} - {\Bp_n}}.
\end{dmath}

The only annoying aspect to this Hamiltonian is the \( a_{{\Bp_j} + {\Bp_m} - {\Bp_n}} \) operator in the interaction term, which is not clear to me how to interpret.  That seems to imply that it is possible to create particles with linear combinations of momentum that may not be in the original set of \( N \) particle momenta.  I think that this can be further fudged by invoking the non-relativisitic constraint again, and decreeing that each of the uniquely indexed creation and anhillation operators are distinguishable only by index, so we can write the N-particle non-relativisitic sector Hamiltonian as
\begin{dmath}\label{eqn:ProblemSet2Problem4:170}
\text{\(:H:\)} =
\sum_{i = 1}^N \omega_{\Bp_i}
a_{i}^\dagger a_{i}
+
\frac{
3 \lambda
}{2 m^2 }
\sum_{r,s,t,u = 1}^N
      a_{r}^\dagger a_{s}^\dagger a_{t} a_{u}.
\end{dmath}

\makeSubAnswer{}{qft:problemSet2:4b}
Yes, with the number of creation and anhillation operators matched, this Hamiltonian preserves particle number (one particle is created for each particle destroyed).
The symmetry appears to be one associated with a permutation operation in the interaction.

%The hint seems to suggest that particle number is conserved, even though the interaction term does not have the structure of a number operator.  I have to conclude (too late for problem set submission) that I don't really understand what is meant by preservation of particle number in this case, and will need to see the problem set solution or discuss this in office hours to understand what is being asked for.
\makeSubAnswer{}{qft:problemSet2:4c}
Continued freehand, time allowing.
%If we designate an N-particle momentum state by
%\begin{dmath}\label{eqn:ProblemSet2Problem4:180}
%\ket{\Bp_1, \Bp_2, \cdots \Bp_N} =
%a_{\Bp_1}^\dagger
%a_{\Bp_2}^\dagger
%\cdots
%a_{\Bp_N}^\dagger \ket{0, 0, \cdots, 0},
%\end{dmath}
%then the interaction terms action on such a state is
%\begin{dmath}\label{eqn:ProblemSet2Problem4:200}
%      a_{\Bp_j}^\dagger a_{\Bp_m}^\dagger a_{\Bp_n} a_{{\Bp_j} + {\Bp_m} - {\Bp_n}}
%\ket{\Bp_1, \Bp_2, \cdots \Bp_N}.
%\end{dmath}
%I'm not sure if this is meaningful, or how to interpret it, and think that I'm going to have to get explanation about what this abstraction means.  I'm also not sure what is meant by the question ``What kinds of scattering processes does it describe.''
\makeSubAnswer{}{qft:problemSet2:4d}
Also continued freehand, time allowing.
%Not attempted.
}


\ifthenelse{\boolean{redacted}}%
{%
\relax%
}%
{%
   \chapter{Problem Set 3.}

      %
% Copyright � 2018 Peeter Joot.  All Rights Reserved.
% Licenced as described in the file LICENSE under the root directory of this GIT repository.
%
\makeproblem{description}{qft:problemSet3:1}{
\makesubproblem{}{qft:problemSet3:1a}
} % makeproblem

\makeanswer{qft:problemSet3:1}{
\makeSubAnswer{}{qft:problemSet3:1a}

TODO.
}

      %
% Copyright � 2018 Peeter Joot.  All Rights Reserved.
% Licenced as described in the file LICENSE under the root directory of this GIT repository.
%
\makeoproblem{Perturbation, and particle creation.}{qft:problemSet3:2}{2018 Hw3.II}{
\index{particle creation}

{\flushleft{In}} class, the problem of creation of particles by an external source in quantum mechanics was discussed. Let us now study this using QFT and Feynman diagrams. Consider a massive scalar free field interacting with a classical source $j(x)$ via:
\begin{equation}
\label{h1}
H = H_0 + \int d^3 x (- j(x) \phi(x))~.
\end{equation}
The classical source $j(x)$ is nonzero only for a finite amount of time, i.e. it is turned on and off, is assumed localized in space, and thus has a well-defined four-dimensional Fourier transform (thus the source is not itself a generalized function).

\makesubproblem{}{qft:problemSet3:2a}
Argue---e.g. using our expressions for overlap of $\vert 0 \rangle$ and $\vert \Omega \rangle$ from class, as well as their meaning---that the probability that the source $j(x)$ creates no particles is
\begin{equation}
\label{p1}
P(0) = \bigg\vert \langle 0 \vert T \left\{e^{ i \int d^4 x j(x) \phi_I(x)} \right\} \vert 0 \rangle \bigg\vert^2~.
\end{equation}
\makesubproblem{}{qft:problemSet3:2b}
Find the order-$j^2$ term in $P(0)$ and show that $P(0) = 1 - \lambda + {\cal{O}} (j^4)$, where
\begin{equation}
\label{p2}
\lambda = \int {d^3 p \over (2 \pi)^3} {1 \over 2 \omega_{\Bp}} \vert \tilde{j}(p) \vert^2~,~{\rm where} ~~ \tilde{j}(p) \equiv \int d^4 y e^{i p \cdot y} j(y)~.
\end{equation}
\makesubproblem{}{qft:problemSet3:2c}
Represent the term computed above as a Feynman diagram. Now represent the entire series for $P(0)$ in terms of Feynman diagrams. Show that the series exponentiates and, therefore, $P(0) = e^{- \lambda}$.
\makesubproblem{}{qft:problemSet3:2d}
Find the probability that the source creates one particle of momentum $\Bk$. First, compute this to order $j$ and then to all orders, using the trick above to sum the series.
\makesubproblem{}{qft:problemSet3:2e}
Show that the probability of producing $n$ particles is $P(n) = {1 \over n!} \lambda^n e^{- \lambda}$, the Poisson distribution.
\makesubproblem{}{qft:problemSet3:2f}
Show that $\sum\limits_{n=0}^\infty P(n) = 1$ and that $\langle N \rangle = \sum\limits_{n=0}^\infty n P(n) = \lambda$, where $\lambda$ is given in (\ref{p2}). Notice that the expression for the mean particle number $\langle N \rangle$ created exactly reproduces (when dimensionally reduced to $d=1$) the one from quantum mechanics given in class. Finally, compute the mean square fluctuation $\langle (N - \langle N \rangle)^2 \rangle$.
} % makeproblem

\makeanswer{qft:problemSet3:2}{
\withproblemsetsParagraph{
\makeSubAnswer{}{qft:problemSet3:2a}
The amplitude for a transition for the evolution of an initial state \( \ket{i} \) to a final state \( \ket{f} \) is
\begin{dmath}\label{eqn:ProblemSet3Problem2:720}
\bra{f} U \ket{i}
=
\bra{f} T e^{-i \int dt H_I(t) } \ket{i}
\end{dmath}
Given the ground state \( \ket{0} \) for the system before the interaction takes effect, the amplitude for production of particles with momenta \( \Bk_1, \cdots \Bk_n \) is
\begin{dmath}\label{eqn:ProblemSet3Problem2:740}
\bra{\Bk, \cdots \Bk_n}
T e^{-i \int dt \int d^3 x (-j(x) \phi(x) } \ket{0}
=
\bra{\Bk, \cdots \Bk_n}
T e^{i \int d^4 x j(x) \phi(x) } \ket{0}.
\end{dmath}
Similarly, the amplitude for a final state that contains no particles is just
\begin{dmath}\label{eqn:ProblemSet3Problem2:760}
\bra{0} T e^{i \int d^4 x j(x) \phi(x) } \ket{0}.
\end{dmath}
The absolute square of this amplitude is \cref{p1}, the probability that no particles are created.

\makeSubAnswer{}{qft:problemSet3:2b}
Expanding matrix element in powers of \( j \) we have
\begin{dmath}\label{eqn:ProblemSet3Problem2:100}
\bra{0} T\lr{ \exp\lr{ i \int d^4 x j(x) \phi_I(x) }} \ket{0}
=
\bra{0} 1 \ket{0}
+
i \bra{0} T\lr{ \int d^4 a j(a) \phi_I(a) } \ket{0}
+
\frac{i^2}{2!} \bra{0} T\lr{ \int d^4 a d^4 b j(a) \phi_I(a) j(b) \phi_I(b) } \ket{0}
+
\frac{i^3}{3!} \bra{0} T\lr{ \int d^4 a d^4 b d^4 c j(a) \phi_I(a) j(b) \phi_I(b) j(c) \phi_I(c) } \ket{0}
+
\cdots
\end{dmath}
Using Wick's theorem to evaluate the integrals, all the odd powers of \( j \) are zero.  We may evaluate the first non-zero integral by contracting the two fields
\begin{dmath}\label{eqn:ProblemSet3Problem2:120}
\bra{0} T\lr{ \int d^4 a d^4 b j(a) \phi_I(a) j(b) \phi_I(b) } \ket{0}
=
\contraction{\int d^4 a d^4 b j(a) }{\phi}{{}_I(a) j(b) }{\phi}
\int d^4 a d^4 b j(a) \phi_I(a) j(b) \phi_I(b)
=
\int d^4 a d^4 b j(a) D_F(a - b) j(b)
=
i \int d^4 a d^4 b \frac{d^4 p}{(2 \pi)^4} j(a) j(b) \frac{ e^{-i p \cdot (a - b) } }{p^2 - m^2 + i \epsilon}
=
i \int \frac{ d^4 p }{(2 \pi)^4} \inv{ p^2 - m^2 + i \epsilon}
\int d^4 a j(a) e^{-i p \cdot a}
\int d^4 b j(b) e^{i p \cdot b}
=
i \int \frac{ d^4 p }{(2 \pi)^4} \inv{ p^2 - m^2 + i \epsilon }
\tilde{j}(p)
\tilde{j}(-p).
\end{dmath}
Assuming that \( j(x) \) is real, this is
\begin{dmath}\label{eqn:ProblemSet3Problem2:140}
\bra{0} T\lr{ \int d^4 a d^4 b j(a) \phi_I(a) j(b) \phi_I(b) } \ket{0}
=
i \int \frac{ d^4 p }{(2 \pi)^4} \frac{
\Abs{\tilde{j}(p)}^2
}{p^2 - m^2 + i \epsilon}
=
\frac{i}{2 \pi} \int dp_0 \int \frac{d^3 p}{(2 \pi)^3} \frac{
\Abs{\tilde{j}(p)}^2
 }{p_0^2 - \Bp^2 - m^2 + i \epsilon}
=
\frac{i}{2 \pi} \int dp_0 \int \frac{d^3 p}{(2 \pi)^3} \frac{
\Abs{\tilde{j}(p)}^2
 }{p_0^2 - \omega_\Bp^2 + i \epsilon}.
\end{dmath}
Integrating \( p_0 \) over the lower half plane contour of \cref{fig:ps3p2:ps3p2Fig1}, which encloses the pole at \( p_0 \approx \omega_\Bp - i \epsilon \) we have
\imageFigure{../figures/phy2403-quantum-field-theory/ps3p2LowerHalfFeynmandContourFig2}{Feynman propagator contour in lower half plane.}{fig:ps3p2:ps3p2Fig1}{0.2}
\begin{dmath}\label{eqn:ProblemSet3Problem2:160}
\bra{0} T\lr{ \int d^4 a d^4 b j(a) \phi_I(a) j(b) \phi_I(b) } \ket{0}
=
\frac{-2 \pi i^2}{2 \pi} \int \frac{d^3 p}{(2 \pi)^3} \evalbar{ \frac{ \Abs{\tilde{j}(p)}^2 }{2 \omega_\Bp}
}{p_0 = \omega_\Bp}
=
\lambda,
\end{dmath}
where it has been assumed that \( \tilde{j}(p_0, \Bp) \rightarrow 0 \) along the infinite circular arc of the integration contour.
To second order in \( j \) our matrix element is
\begin{dmath}\label{eqn:ProblemSet3Problem2:180}
\bra{0} T\lr{ \exp\lr{ i \int d^4 x j(x) \phi_I(x) }} \ket{0}
=
1 - \frac{\lambda}{2}.
\end{dmath}
We can now use this as an initial approximation for the probability
\begin{dmath}\label{eqn:ProblemSet3Problem2:200}
P(0) =
\lr{ 1 - \frac{\lambda}{2} + O(\lambda^2) }^2
=
1 - \lambda + O(\lambda^2),
\end{dmath}
as desired.

\makeSubAnswer{}{qft:problemSet3:2c}
The diagram for the integral just computed is a single line segment as sketched in \cref{fig:ps3OneEdgeDiagram:ps3OneEdgeDiagramFig3}.
\imageFigure{../figures/phy2403-quantum-field-theory/ps3OneEdgeDiagramFig3}{\( j^2 \) diagram.}{fig:ps3OneEdgeDiagram:ps3OneEdgeDiagramFig3}{0.1}
The diagram for the \( j^4 \) integrals are sketched in
\cref{fig:ps3jfourthDiagram:ps3jfourthDiagramFig4}.
\imageFigure{../figures/phy2403-quantum-field-theory/ps3jfourthDiagramFig4}{\( j^4 \) diagrams.}{fig:ps3jfourthDiagram:ps3jfourthDiagramFig4}{0.2}
After that the diagrams get messier to enumerate.  We can see the pattern by considering a non-trivial example such as the \( j^6 \) integral.  For that each diagram has three edges, where all possible combinations \( ab, ac, ad, ae, af, bc, bd, be, bf, cd, ce, cf, de, df, ef \) are found for the ``first'' edge in each diagram.  This is a total of \( \binom{6}{2} = 15 \) edges.  For each such diagram, there are \( \binom{4}{2} \) choices for the next edge in the diagram (example: \( ab, cd, ef \) would be one diagram). The coefficient of this integral is therefore
\begin{dmath}\label{eqn:ProblemSet3Problem2:220}
\frac{i^6}{3!} \inv{6!} \binom{6}{2} \binom{4}{2}
=
\frac{(-1)^3}{3!} \inv{\cancel{6!}} \frac{\cancel{6!}}{\cancel{4!}2!} \frac{\cancel{4!}}{2!2!}
=
\frac{(-1)^3}{3! 2^3}.
\end{dmath}
Here the \( 3! \) downstairs is to compensate for the fact that there are \( 3 \times 2 \) possible orderings of each distinct pair of endpoints.  Example:
\begin{dmath}\label{eqn:ProblemSet3Problem2:280}
 \setlr{ ab, cd, ef }, \setlr{ ab, ef, cd}, \setlr{ cd, ab, ef }, \setlr{ cd, ef, ab }, \setlr{ ef, ab, cd }, \setlr{ ef, cd, ab }.
\end{dmath}
The pattern of perfect cancellation is clear.  The \( j^{2n} \) order integral is
\begin{dmath}\label{eqn:ProblemSet3Problem2:240}
\frac{(-1)^n}{n!} \inv{(2n)!} \frac{(2n)!}{(2n -2)!2!} \cdots \frac{4!}{2!2!} \lambda^n
=
\frac{(-\lambda/2)^n}{n!},
\end{dmath}
so we find that
\begin{dmath}\label{eqn:ProblemSet3Problem2:260}
P(0) =
\lr{
\sum_{n = 0}^\infty
\frac{(-\lambda/2)^n}{n!}
}^2
=
\lr{ e^{-\lambda/2} }^2
=
e^{-\lambda},
\end{dmath}
as desired.

\makeSubAnswer{}{qft:problemSet3:2d}
The probability for a single particle of momentum \( \Bk \) is
\begin{dmath}\label{eqn:ProblemSet3Problem2:300}
P_\Bk
=
\Abs{ \bra{0} T a_\Bk e^{i \int d^4 x j(x) \phi(x)} \ket{0} }^2
=
\sum_{n,m = 1}^\infty \frac{ i^n (-i)^m }{ n! m! }
\bra{0} T a_\Bk \lr{ \int d^4 x j(x) \phi(x)}^n \ket{0}
\lr{ \bra{0} T a_\Bk \lr{ \int d^4 x j(x) \phi(x)}^m \ket{0} }^\dagger
=
\sum_{r,s = 1}^\infty \frac{ i^{2 r + 1} (-i)^{2 s + 1} }{ (2r + 1)! (2 s + 1)! }
\bra{0} T a_\Bk \lr{ \int d^4 x j(x) \phi(x)}^{2 r + 1} \ket{0}
\lr{ \bra{0} T a_\Bk \lr{ \int d^4 x j(x) \phi(x)}^{2 s + 1} \ket{0} }^\dagger
,
\end{dmath}
where we've accounted for the fact that these matrix elements are zero for any even powers \( m, n \).

For \( n = 1 \) we want to evaluate
\begin{dmath}\label{eqn:ProblemSet3Problem2:320}
\bra{0}
T a_\Bk \int d^4 x j(x) \phi(x)
\ket{0},
\end{dmath}
which has the diagram \cref{fig:ps3Problem3d:ps3Problem3dFig1}, which is
\imageFigure{../figures/phy2403-quantum-field-theory/ps3Problem3dFig1}{\( n = 1 \) diagram.}{fig:ps3Problem3d:ps3Problem3dFig1}{0.1}
\begin{dmath}\label{eqn:ProblemSet3Problem2:340}
\bra{0}
T a_\Bk \int d^4 x j(x) \phi(x)
\ket{0}
=
\contraction{\int d^4 x j(x) }{a}{{}_\Bk }{\phi}
\int d^4 x j(x) a_\Bk \phi(x)
=
\int d^4 x j(x) e^{i k \cdot x}
=
j^\conj(k).
\end{dmath}
The next diagram is sketched in \cref{fig:ps3Problem3d:ps3Problem3dFig2}, which, temporarily ignoring symmetry factors, gives
\begin{dmath}\label{eqn:ProblemSet3Problem2:360}
\bra{0}
T a_\Bk \int d^4 a d^4 b d^4 c j(a) \phi(a) j(b) \phi(b) j(c) \phi(c)
\ket{0}
=
\int d^4 a d^4 b d^4 c j(a) e^{i k \cdot a} D_F(b - c) j(b) j(c)
=
j^\conj(k) \lambda
\end{dmath}
\imageFigure{../figures/phy2403-quantum-field-theory/ps3Problem3dFig2}{\( n = 3 \) diagrams.}{fig:ps3Problem3d:ps3Problem3dFig2}{0.15}
To compute the symmetry factors consider the \( n = 5 \) diagram sketched in
\cref{fig:ps3Problem3d:ps3Problem3dFig3},
which is instructive.
\imageFigure{../figures/phy2403-quantum-field-theory/ps3Problem3dFig3}{\( n = 5 \) diagrams.}{fig:ps3Problem3d:ps3Problem3dFig3}{0.4}
We have 5 ways to contract with the first \( \phi \), and \( \binom{4}{2} \) diagrams for each such selection, which has a \( 2! \) redundancy factor since each pair of nodes (say \( bc, de \)) can be ordered in either order.  The symmetry factor for \( n = 5 = 2 (2) + 1 \) is therefore
\begin{dmath}\label{eqn:ProblemSet3Problem2:380}
\inv{3!} 5 \times \binom{4}{2}
=
\inv{3!} 5 \times \frac{4!}{2^2}
\end{dmath}
For \( n = 7 \) that factor is
\begin{dmath}\label{eqn:ProblemSet3Problem2:400}
\inv{3!} 7 \times \binom{6}{2} \binom{4}{2}
=
\inv{3!} 7 \times \frac{6!}{2^3},
\end{dmath}
and in general for \( n = 2 r + 1 \)
\begin{dmath}\label{eqn:ProblemSet3Problem2:420}
\inv{r!} (2 r + 1 ) \times \frac{(2r)!}{2^r}.
\end{dmath}
This gives
\begin{dmath}\label{eqn:ProblemSet3Problem2:440}
\inv{(2 r + 1)!}
\bra{0}
T a_\Bk \lr{ \int d^4 s j(x) \phi(x) }^{2 r + 1}
\ket{0}
=
\inv{r!} \frac{2 r + 1 }{(2 r + 1)!} \times \frac{(2r)!}{2^r} j^\conj(k) \lambda^r
=
\inv{r!} j^\conj(k) \lr{\frac{\lambda}{2}}^r.
\end{dmath}
Plugging this into \cref{eqn:ProblemSet3Problem2:300} we find
\begin{dmath}\label{eqn:ProblemSet3Problem2:460}
P_\Bk
=
\Abs{j(k)}^2
\sum_{r,s = 1}^\infty \frac{ i^{2 r + 1} (-i)^{2 s + 1} }{ r!s! }
\lr{\frac{\lambda}{2}}^{r + s}
=
\Abs{j(k)}^2
\sum_{r,s = 1}^\infty \frac{ i^{2 (r + s + 1)} (-1)^{2 s + 1} }{ r!s! }
\lr{\frac{\lambda}{2}}^{r + s}
=
\Abs{j(k)}^2
\sum_{r,s = 1}^\infty \frac{ (-1)^{r + s} }{ r!s! }
\lr{\frac{\lambda}{2}}^{r + s}
=
\Abs{j(k)}^2
\lr{ e^{-\lambda/2} }^2
=
\Abs{j(k)}^2 e^{-\lambda}.
\end{dmath}

\makeSubAnswer{}{qft:problemSet3:2e}
Summing \cref{eqn:ProblemSet3Problem2:460} over all momentum states, we find the probability to create one particle is
\begin{dmath}\label{eqn:ProblemSet3Problem2:480}
P(1)
= \int \frac{d^3 k}{(2 \pi)^3} \inv{2 \omega_\Bk}
\Abs{j(k)}^2 e^{-\lambda}
=
\lambda e^{-\lambda}.
\end{dmath}

For \( P(2) \) we want
\begin{dmath}\label{eqn:ProblemSet3Problem2:500}
P(2) = \inv{2!} \int \frac{d^3 k d^3 p}{(2 \pi)^6 2 \omega_\Bk 2 \omega_\Bp}
\Abs{
\bra{0} a_\Bk a_\Bp T e^{i \int d^4 x j(x) \phi(x)} \ket{0}
}^2,
\end{dmath}
where an inverse \( 2! \) factor has been added for all the possible orderings of the annihilation operators.
The zero and first order terms in the matrix element \( \bra{0} a_\Bk a_\Bp T e^{i \int d^4 x j(x) \phi(x)} \ket{0} \) are zero.  After this we want to compute all the contractions of
\begin{dmath}\label{eqn:ProblemSet3Problem2:520}
\frac{i^2}{2!} a_\Bk a_\Bp \int d^4 x d^4 y j_x \phi_x j_y \phi_y,
\end{dmath}
which have diagrams sketched in \cref{fig:ps3p2e:ps3p2eFig1}.
\imageFigure{../figures/phy2403-quantum-field-theory/ps3p2eFig1}{\( n = 2 \) diagrams.}{fig:ps3p2e:ps3p2eFig1}{0.15}
The symmetry factor (times the leading inverse factorial) is \( 2! \), so the leading term is
\begin{dmath}\label{eqn:ProblemSet3Problem2:540}
2! \times \frac{i^2}{2!} \int d^4 x d^4 y e^{i k \cdot x + i p \cdot y } j(x) j(y)
=
-j^\conj(k) j^\conj(p).
\end{dmath}
For \( n = 4 \) the diagrams are sketched in \cref{fig:ps3p2e:ps3p2eFig2}.  The coefficient symmetry factor is \( 2 \times \binom{4}{2} \), so the next order term in the matrix element is
\begin{dmath}\label{eqn:ProblemSet3Problem2:560}
\binom{4}{2} \times (2!) \times
\frac{i^4}{4!} \int d^4 a d^4 b d^4 c d^4 d e^{i k \cdot a + i p \cdot b } j(a) j(b) D_F(c - d)
=
j^\conj(k) j^\conj(p) \frac{\lambda}{2}.
\end{dmath}
\imageFigure{../figures/phy2403-quantum-field-theory/ps3p2eFig2}{\( n = 4 \) diagrams.}{fig:ps3p2e:ps3p2eFig2}{0.2}
For \( n = 2 r \) we have
\begin{dmath}\label{eqn:ProblemSet3Problem2:580}
(-1)^r 2 \inv{(2r)!} \binom{2 r}{2} \cdots \binom{4}{2} j^\conj(k) j^\conj(p) \lambda^r
=
\inv{(r-1)!} \lr{ \frac{-\lambda}{2}}^r
j^\conj(k) j^\conj(p)
\end{dmath}
so
\begin{dmath}\label{eqn:ProblemSet3Problem2:600}
\bra{0} a_\Bk a_\Bp T e^{i \int d^4 x j(x) \phi(x)} \ket{0}
=
-j^\conj(k) j^\conj(p) e^{-\lambda/2}.
\end{dmath}
Plugging back into \cref{eqn:ProblemSet3Problem2:500} we have
\begin{dmath}\label{eqn:ProblemSet3Problem2:620}
P(2) = \inv{2!} \int \frac{d^3 k d^3 p}{(2 \pi)^6 2 \omega_\Bk 2 \omega_\Bp}
\lr{ -j^\conj(k) j^\conj(p) e^{-\lambda/2} }
\lr{ j(k) j(p) e^{-\lambda/2} }
=
\inv{2!} \lambda^2 e^{-\lambda}.
\end{dmath}
We are left to generalize this to \( n > 2 \).  Considering the first couple diagrams for \( n = 3 \) as sketched in
\cref{fig:ps3p2e:ps3p2eFig3},
exposes the pattern, namely
\begin{dmath}\label{eqn:ProblemSet3Problem2:640}
\bra{0} a_\Bk a_\Bp a_\Bq T e^{i \int d^4 x j(x) \phi(x)} \ket{0}
=
3! \frac{i^3}{3!} j^\conj(k) j^\conj(p) j^\conj(q)
+
3! \binom{5}{3} \frac{i^5}{5!} j^\conj(k) j^\conj(p) j^\conj(q) \lambda
+
\cdots
=
i^3 j^\conj(k) j^\conj(p) j^\conj(q) e^{-\lambda/2}.
\end{dmath}
\imageFigure{../figures/phy2403-quantum-field-theory/ps3p2eFig3}{\( n = 3, 5 \) diagrams for three particle creation.}{fig:ps3p2e:ps3p2eFig3}{0.2}
The total probability is therefore
\begin{dmath}\label{eqn:ProblemSet3Problem2:660}
P(3)
=
\inv{3!} \int \frac{d^3 k d^3 p d^3 q}{(2 \pi)^9} \inv{8 \omega_\Bk \omega_\Bp \omega_\Bq}
\Abs{ i^3 j^\conj(k) j^\conj(p) j^\conj(q) e^{-\lambda/2} }^2
=
\inv{3!} \lambda^3 e^{-\lambda}.
\end{dmath}
For \( m \) particles the matrix element expands as
\begin{dmath}\label{eqn:ProblemSet3Problem2:680}
\bra{0} a_{\Bk_1} \cdots a_{\Bk_m} T e^{i \int d^4 x j(x) \phi(x)} \ket{0}
=
m! \times \frac{i^m}{m!} j^\conj(k_1) \cdots j^\conj(k_m)
+
m! \times \frac{i^{m+2}}{(m+2)!} \binom{m+2}{m} j^\conj(k_1) \cdots j^\conj(k_m) \lambda
+
m! \times \frac{i^{m+4}}{(m+4)!} \binom{m+4}{m} \binom{m+2}{m} j^\conj(k_1) \cdots j^\conj(k_m) \lambda^2
+ \cdots
=
i^m
j^\conj(k_1) \cdots j^\conj(k_m) e^{-\lambda/2},
\end{dmath}
so
\begin{dmath}\label{eqn:ProblemSet3Problem2:700}
P(m) = \inv{m!}
\int
\frac{d^3 k_1}{(2\pi)^3 2 \omega_{\Bk_1} }
\cdots
\frac{d^3 k_m}{(2\pi)^3 2 \omega_{\Bk_m} }
\Abs{
i^m
j^\conj(k_1) \cdots j^\conj(k_m) e^{-\lambda/2}
}^2
=
\inv{m!} \lambda^m e^{-\lambda}.
\end{dmath}

\makeSubAnswer{}{qft:problemSet3:2f}
The sum of the probabilities is easy to compute
\begin{equation}\label{eqn:ProblemSet3Problem2:20}
\begin{aligned}
\sum_{n = 0}^\infty P(n)
&=
e^{-\lambda} \sum_{n = 0}^\infty \inv{n!} \lambda^n \\
&=
e^{-\lambda} e^{\lambda} \\
&= 1.
\end{aligned}
\end{equation}
The mean is
\begin{equation}\label{eqn:ProblemSet3Problem2:40}
\begin{aligned}
\expectation{N}
&=
\sum_{n = 0}^\infty n P(n) \\
&=
e^{-\lambda} \sum_{n = 1}^\infty \frac{n }{n!} \lambda^n \\
&=
e^{-\lambda} \lambda \sum_{n = 1}^\infty \frac{1 }{(n-1)!} \lambda^{n-1} \\
&=
e^{-\lambda} \lambda e^{\lambda} \\
&=
\lambda.
\end{aligned}
\end{equation}
For the mean square we first compute
\begin{equation}\label{eqn:ProblemSet3Problem2:60}
\begin{aligned}
\expectation{N^2}
\sum_{n = 0}^\infty n^2 P(n)
&=
e^{-\lambda} \sum_{n = 1}^\infty \frac{n^2 }{n!} \lambda^n \\
&=
e^{-\lambda} \lambda \sum_{n = 1}^\infty \frac{n }{(n-1)!} \lambda^{n-1} \\
&=
e^{-\lambda} \lambda \sum_{n = 0}^\infty \frac{n+1 }{n!} \lambda^{n} \\
&=
e^{-\lambda} \lambda \lr{ e^\lambda + \sum_{n = 0}^\infty \frac{n}{n!} \lambda^{n} } \\
&=
\lambda +
e^{-\lambda} \lambda^2 e^{\lambda} \\
&=
\lambda + \lambda^2,
\end{aligned}
\end{equation}
so
\begin{dmath}\label{eqn:ProblemSet3Problem2:80}
\expectation{(N - \expectation{N})^2}
=
\expectation{N^2 - 2 N \expectation{N} + \expectation{N}^2}
=
\expectation{N^2} - 2 \expectation{N}^2 + \expectation{N}^2
=
\expectation{N^2} - \expectation{N}^2
=
\lambda + \lambda^2 - \lambda^2
= \lambda.
\end{dmath}
}
}

      %
% Copyright � 2018 Peeter Joot.  All Rights Reserved.
% Licenced as described in the file LICENSE under the root directory of this GIT repository.
%
\makeoproblem{
%``Radiation" by accelerated source and ``IR catastrophe''.
Radiation and the IR catastrophe.
}{qft:problemSet3:3}{2018 Hw3.III}{
\index{IR catastrophe}
\index{radiation}

{\flushleft{This}} is a baby problem having to do with radiation of scalar particles. (As we will not have too much time to study the radiation of electromagnetic fields this term, it is a good opportunity.) Consider the coupling of a classical particle to a scalar field (remember Hw 1, Problem 1, where a similar coupling to the electromagnetic field was considered):
\begin{equation}
\label{scalar1}
S_{int} = e \int\limits_{worldline} ds \phi(x(s))~,
\end{equation}
where $x(s)$ is the worldline of the particle and $e$ is its scalar charge (what is its dimension?). The coupling (\ref{scalar1}) corresponds to a ``current" $j(x)$ coupling to $\phi$ as in Problem {\bf II.} above:
\begin{equation}
\label{scalar2}
S_{int} = e \int\limits_{worldline} ds \phi(x(s) =   \int d^4 x j(x) \phi(x)~,
\end{equation}
where
\begin{equation}\label{eqn:ProblemSet3Problem3:n}
j(x) = e \int\limits_{worldline} ds \delta^{(4)}(x - x(s))~,
\end{equation}
is the  current.

\makesubproblem{}{qft:problemSet3:3a}
Consider a particle of mass $M$, whose worldline is given by:
\begin{equation}
x^\mu(s) = {p^\mu_{i} \over M} s, ~{\rm for} ~~ s<0 ~ {\rm and} ~~ x^\mu(s) = {p^\mu_{f} \over M} s, ~{\rm for} ~~ s>0~,
\label{scalar3}
\end{equation}
where $p^\mu_i$ and $p^\mu_f$ are the initial and final four-momenta of the particle (both obeying $p^\mu p_\mu = M^2$,  with $p^0 > 0$, of course). The physical meaning of this trajectory is that the particle undergoes acceleration at $x^0=0$, suddenly changing its four-momentum from $p_i$ to $p_f$. Show that the Fourier transform of the current, as defined in (\ref{p2}) above, is given by:
\begin{equation}
\label{current1}
\tilde{j}(p) ={ i e M \over p \cdot p_f} - { i e M \over p \cdot p_i}
\end{equation}
To make the TA's life (and yours) easier, in getting (\ref{current1}), consider without loss of generality, trajectories with $p_i = (M,0,0,0)$ and $p_f = (\sqrt{M^2 + q^2},q,0,0)$. \footnote{Recall  the ``half-delta function" integrals from Homework 2, Problem 1 and ignore the $i \epsilon$ factors which should be present in the denominators in (\ref{current1}) as they will not be important for what follows.}

\makesubproblem{}{qft:problemSet3:3b}
Now study the expression for the average number of particles produced, $\lambda$, or $\langle N \rangle$,  of eq.~(\ref{p2}), as well as the average energy $\langle E \rangle$, which you can easily come up with, from (\ref{p2}). From now on, consider the case where the mass of the produced particles ($\phi$-quanta) is zero. This has two advantages:  simplifications in the various formulae as well as giving us the feeling that we are actually looking at something close to radiation of photons.

Show that the integrals over the momenta of the emitted ``photons" in $\langle N\rangle$ and $\langle E\rangle$ diverge at large $p$.

{\small  {\flushleft{T}}his is because our trajectory has a sudden change of momentum at $s=0$. We expect that the formulae for the radiated ``photons" is still valid for sufficiently small momenta where the nature of the kink is not relevant (presumably for momenta less than the inverse time during which a smooth change of momenta occurs, i.e. momenta smaller than the reciprocal of the scattering time). Thus, we now
 suppose there is a high-momentum cutoff. }

 Let us then study  the convergence of the small-$p$ integrals over the momenta of the emitted particles in $\langle N\rangle$ and $\langle E\rangle$. This counts the number or energy  of ``soft" photons emitted.
 Show what while $\langle E\rangle$ is finite, the expression for $\langle N \rangle$ diverges for small $\Bp$.

{\small \flushleft{T}his divergence in the number of soft photons radiated by a classical source is called the ``infrared catastrophe", in the case of QED. A similar answer is obtained using a tree-level QFT calculation of the radiation of soft photons. Note one interesting fact: the divergence of the integral determining $\langle N \rangle$ is logarithmic: $ \langle N \rangle \sim e^2 \log {k_{max} \over k_{min}}$, where the IR cutoff $k_{min}$ is introduced to make the integral finite. You see now that $e^2$ (really, the fine structure constant $\alpha \sim 1/137$, in QED) is multiplied by a large $\log$, which can be bigger than $137$. This is a first indication that perturbation theory breaks down and some resummation  of diagrams may be  needed. Indeed, in QED, the infrared divergence is cancelled after adding ``loop" effects, see Section 6.5 of Peskin and Schroeder.  }

 {\small {\flushleft{T}}he point of this problem was to illustrate two things. First, it shows (within  this classical calculation of the overlap between free and interacting vacua) how the two vacua can be orthogonal (in the case of massless $\phi$, due to infrared (small momenta) problems). Second, it points toward something---the infrared divergences in QED, and the resulting ``Sudakov double logs"---that you will study later, either in QFT2 or by yourselves.}
} % makeproblem

\makeanswer{qft:problemSet3:3}{
\withproblemsetsParagraph{
\makeSubAnswer{}{qft:problemSet3:3a}
Our Fourier transform is
\begin{dmath}\label{eqn:ProblemSet3Problem3:20}
\tilde{j}(p)
=
\int d^4 y e^{i p \cdot y} j(y)
=
e \int d^4 y e^{i p \cdot y} \int ds \deltafour(y - y(x))
=
e \int_0^\infty ds \int d^4 y e^{i p \cdot y} \deltafour(y - \ifrac{p_\txtf s}{M})
+
e \int_{-\infty}^0 ds \int d^4 y e^{i p \cdot y} \deltafour(y - \ifrac{p_i s}{M}).
\end{dmath}
Writing \( p \cdot p_f = p_\mu p_f^\mu \) and \( p \cdot p_i = p_\mu p_i^\mu \), and
using the half delta function representation from Hw2, this reduces to
\begin{dmath}\label{eqn:ProblemSet3Problem3:40}
\tilde{j}(p)
=
e \int_0^\infty ds e^{i p \cdot \ifrac{p_f s}{M} - \epsilon s}
+
e \int_{-\infty}^0 ds e^{i p \cdot \ifrac{p_i s}{M} + \epsilon s}
=
e
\evalrange{
\frac{e^{i p \cdot p_f s/M - \epsilon s}}{ i p \cdot p_f/M - \epsilon }
}{0}{\infty}
+
e
\evalrange{
\frac{e^{i p \cdot p_i s / M + \epsilon s}}{ i p \cdot p_i/M + \epsilon }
}{-\infty}{0}
\rightarrow
i e M \lr{ \inv{p \cdot p_f} - \inv{p \cdot p_i} },
\end{dmath}
as desired.  While it was suggested that we use specific values of \( p_i, p_f \) to make life easier,
it isn't clear how that would have helped.
\makeSubAnswer{}{qft:problemSet3:3b}
Observing that \( \Abs{\tilde{j}(p)}^2/(2 \omega_p) \) is the number density, our energy is given by
\begin{dmath}\label{eqn:ProblemSet3Problem3:60}
\expectation{E} = \inv{2} \int \frac{d^3 p}{(2 \pi)^3} \Abs{\tilde{j}(p)}^2.
\end{dmath}
Utilizing the ``make the TA's life easier'' representation of \( p_f, p_i \), the
absolute squared momentum space current is
\begin{dmath}\label{eqn:ProblemSet3Problem3:80}
\Abs{\tilde{j}(p)}^2
=
e^2 M^2 \lr{ \inv{ p_0 \sqrt{ M^2 + q^2 } - p_1 q } - \inv{ p_0 M } }^2.
\end{dmath}
This gives
\begin{dmath}\label{eqn:ProblemSet3Problem3:100}
\begin{aligned}
\expectation{N} &= \frac{e^2 M^2}{2 (2 \pi)^3} \int dp_2 dp_3 \int \frac{dp_1}{\omega_\Bp} \lr{ \inv{ \omega_\Bp \sqrt{ M^2 + q^2 } - p_1 q } - \inv{ \omega_\Bp M } }^2 \\
\expectation{E} &= \frac{e^2 M^2}{2 (2 \pi)^3} \int dp_2 dp_3 \int dp_1 \lr{ \inv{ \omega_\Bp \sqrt{ M^2 + q^2 } - p_1 q } - \inv{ \omega_\Bp M } }^2,
\end{aligned}
\end{dmath}
where \( \omega_\Bp = \sqrt{ \Bp^2 + M^2 } \).  The \( p_1 \) integrals can both be evaluated (using Mathematica), and we find
\begin{dmath}\label{eqn:ProblemSet3Problem3:120}
\begin{aligned}
%\expectation{N} &= \frac{e^2 M^2}{(2 \pi)^3 M^2 q} \lr{ 2 q - 2 M \tanh^{-1} \lr{ \frac{ q }{\sqrt{M^2 + q^2}}} }
%\int dp_2 dp_3
%\inv{p_2^2 + p_3^2 + M^2} \\
\expectation{N} &= \frac{e^2}{4 \pi^3 q} \lr{ q - M \tanh^{-1} \lr{ \frac{ q }{\sqrt{M^2 + q^2}}} }
\int dp_2 dp_3
\inv{p_2^2 + p_3^2 + M^2} \\
%\expectation{E} &= \frac{e^2 M^2}{(2 \pi)^3}
%\frac{\pi  \left(\sqrt{M^2+q^2}-M\right)}{2 M^3}
%\int dp_2 dp_3
%\inv{ \sqrt{p_2^2 + p_3^2 + M^2}}
\expectation{E} &= \frac{e^2 \lr{\sqrt{M^2+q^2}-M } }{16 \pi^2 M}
\int dp_2 dp_3
\inv{ \sqrt{p_2^2 + p_3^2 + M^2}}
\end{aligned}
\end{dmath}
Neither of the \( dp_2 dp_3 \) integrals converge for \( p_2, p_3 \in [-\infty, \infty] \), so both the average number of particles and energy diverge in the large \( \Bp \) limit.

We can evaluate these in the small \( \Bp \) limit by imposing a limit on the range of the \( dp_2 dp_3 \) integrands, and find
\begin{dmath}\label{eqn:ProblemSet3Problem3:140}
\iint_{-\epsilon}^\epsilon dp_2 dp_3
\inv{ \sqrt{p_2^2 + p_3^2 + M^2}}
= 8 \epsilon \sinh^{-1}(1),
\end{dmath}
so the average energy in the small \( \Bp \) limit is
\begin{dmath}\label{eqn:ProblemSet3Problem3:160}
\expectation{E} = \frac{e^2 \lr{\sqrt{M^2+q^2}-M } \epsilon \sinh^{-1}(1)}{8 \pi^2 M}.
\end{dmath}
However, for the average number of particles in the small \( \Bp \) limit, the integral
\begin{dmath}\label{eqn:ProblemSet3Problem3:180}
\iint_{-\epsilon}^\epsilon dp_2 dp_3
\inv{p_2^2 + p_3^2 + M^2},
\end{dmath}
does not converge, so we find that the average number of particles associated with this current diverges in both the small and large \( \Bp \) limit.
}
}

      %
% Copyright � 2018 Peeter Joot.  All Rights Reserved.
% Licenced as described in the file LICENSE under the root directory of this GIT repository.
%
\makeoproblem{Where is the particle?}{qft:problemSet3:4}{2018 Hw3.IV}{
\index{particle localization}
{\flushleft{In}} class, we did mention that, by analogy with non relativistic quantum mechanics, the state $\hat\phi(\Bx,t=0) \vert 0\rangle$ allows us to say something along the lines that {\it ``the operator $\hat\phi(\Bx)_+$ creates a particle at $\Bx$"}.
These words are based on noticing  that in QM, we have
$$\vert\Bx\rangle \sim \sum_{\Bp} e^{ i \Bp \cdot \Bx} \vert \Bp \rangle,$$
 where $\vert\Bx\rangle$ is an eigenstate of the position operator with eigenvalue $\Bx$ and $\Bp$ is, likewise, an eigenstate of momentum. On the other hand, in free massive scalar theory, the state $\hat\phi(\Bx,t=0) \vert 0\rangle$ can be similarly expressed as $$\hat\phi(\Bx,t=0) \vert 0\rangle = \int {d^3 p \over (2 \pi)^3 \sqrt{2 \omega_{\Bp}}} e^{ - i \Bp \cdot \Bx} \hat{a}^\dagger_{\Bp} \vert 0 \rangle =  \int {d^3 p \over (2 \pi)^3 2 \omega_{\Bp}} e^{ - i \Bp \cdot \Bx}   \vert \Bp \rangle,$$ where $\vert \Bp \rangle$
  is the relativistically normalized momentum eigenstate. Comparing the above two equations, reading from left to right, we are compelled to utter the words quoted in the beginning.


 Accepting this interpretation literally, we are next faced with explaining the following. Consider the state $\vert \Bzero,0 \rangle = \hat\phi(\Bzero,t=0) \vert 0\rangle$, interpreted (as per the above discussion) as a particle created at $\Bx=0$ at $t=0$. Similarly, the state $$\vert \By,t \rangle = \hat\phi(\By,t) \vert 0 \rangle$$ is that of  a particle at $\By$ at $t$. Notice that these are free fields so their time evolution is trivial. Then, by the usual Born rule  of quantum mechanics (which we accept in QFT), the inner product
    $$
    \langle \By,t \vert \Bzero,0 \rangle $$
    would be ``{\it the amplitude that the particle created at $\Bzero$ at $t=0$
    is found at $\By$ at $t$}". Notice that this is exactly the kind of answer that the quantum-mechanical propagator,  often denoted precisely by  $\langle \By,t \vert \Bzero,0 \rangle$, gives.
    A problem with this arises when one realizes that
     $$
    \langle \By,t \vert \Bzero,0 \rangle  =\langle 0\vert \hat{\phi}(\By,t) \hat{\phi}(\Bzero,0)\vert0 \rangle = D(\By,t) \ne 0 ~ {\rm for} ~ (\By,t) \sim (\Bzero,0)~.$$
In other words, this amplitude is nonzero for spacelike separations (as you explicitly showed in Homework 2, Problem 1, Part 2).
The point of the simple exercise below is to argue that the above interpretation of this amplitude should be taken with a grain of salt, i.e. not too literally, as far as relativity is concerned, of course.

The question we will ask is: to what extent is this particle at $\Bx=0$ localized? In quantum mechanics, we answer this question by pointing out that for an eigenstate of $\hat{x}$, whose wave function is $\delta(x -x')$,  the probability to find the particle anywhere but at $x=x'$ is zero. Trying to pursue this in QFT, a conundrum that arises is that we do not have wave functions for particles. Recall that we have wave functionals, which determine the probability that {\it the field} has this or that value. The coordinate, on the other hand, is an argument,   not an operator (hence ``observable") in the theory---just like time in QM, which is also not an operator; after all we
said ``QM=QFT in $d=1$".
The best we can do is to consider the state $\vert \By,0 \rangle$ and ask where its properties identifiable in QFT---energy or momentum---are localized.

 Thus,  consider the expectation value of $\hat{T}_{00}(\Bx,t)$ (assumed normal-ordered) in this state:
$$
\rho(\By, \Bx, t) \equiv \langle \By,0 \vert T_{00}(\Bx,t)\vert \By,0 \rangle~.
$$
From the Born rule, the natural interpretation of the above quantity is the value of the energy density of the state $\vert \By,0 \rangle$ observed at $(\Bx,t)$---spacelike or not w.r.t. $(\By,0)$.


\makesubproblem{}{qft:problemSet3:4a}
Show, using the translation operator, that $\rho(\By, \Bx, t) =  \rho(0, \Bx - \By, t) \equiv \tilde\rho(\Bx-\By,t)$, where the last equality defines the new energy density $\tilde\rho(\Bx,t)$.
\makesubproblem{}{qft:problemSet3:4b}
Using Wick's theorem---really, a baby-version thereof---express $\tilde\rho(\Bx,t)$ in terms of $D(\Bx,t)$ and its derivatives.
\makesubproblem{}{qft:problemSet3:4c}
Using the knowledge acquired from Homework 2, study how well is the particle's energy localized, already at $t=0$.

Are you surprised by the result?  Are you comforted?

 {\small {\flushleft W}e didn't have time, apart from Problem 4 of Homework 2, to dwell much on the nonrelativisic limit.
This limit can be achieved by forgetting the antiparticles and then defining non-relativistic fields. This is very well described in either Tong's or Luke's notes. For those of you studying cold atoms, it is
definitely a must-read! }

\bigskip

My final comment is that the most concise formulation of causality that goes beyond simply stating that the commutators vanish for spacelike separations is the one first due to Stueckelberg (1940's) and then finessed by Bogoljubov (1950's).
\index{Stueckelberg}

They consider the expectation value of an operator $\hat{O}(x)$   in a state prepared by the action of an operator $U[g]\vert 0\rangle$.
$U[g]$ is an evolution operator (see below) which is a functional of some classical fields $g(y)$ used to prepare the state of the field (e.g. external e.m. fields using to focus, accelerate, etc., the particles; $g(y)$ could also be used to turn on and off the interactions in different space time regions). Thus the object of study is:
$$ \langle \hat{O}(x) \rangle =  \langle 0 \vert U^\dagger[g] \hat{O}(x) U[g] \vert 0 \rangle~.$$
The causality condition, then, is that
$$ {\delta \langle \hat{O}(x) \rangle \over \delta g[y]} = 0 ~ {\rm for } ~ x\sim y~.$$
Now, recalling the form of the evolution operator,
\begin{equation}\label{eqn:ProblemSet3Problem4:17}
U[g] = T e^{  i \int dt d^3 x L_I(t, \Bx, g(\Bx,t))},
\end{equation}
and the Baker-Campbell-Hausdorf formula, it should be clear how the vanishing of the commutators outside the light cone becomes  relevant for the above condition  to hold.
For Bogoljubov, the vanishing commutators are a {\it consequence} of the causality condition given in terms of variational derivatives, as expressed above; he derives the $S$-matrix expansion from that requirement along with a few others (locality and Lorentz invariance, basically).
\index{Baker-Campbell-Hausdorf formula}
\index{Bogoljubov}


{\flushleft{T}he reason to include this comment was to close the loop on something that I mentioned in class, now that we've seen what $U[g]$ may look like.
 }

%\begin{comment}
%   {\flushleft {\bf II.}} {\it Is gravity scalar?  }
%
%
%  \smallskip
%{\flushleft{Here,}} you are going to study, using the results from {\bf I.}, the question whether gravity can be described by a relativistic massless scalar field. After all, the Yukawa potential is $\sim 1/r$, in the limit when the mass goes to zero, which is just like the Newtonian potential.
%
%We'll take gravity to be described by a massless free scalar $\phi$, the ``scalar graviton",
%just like the previous problem but with zero mass. The matter fields will have mass and will be described by another scalar field, $\psi$, this time massive. Now we have to decide how to couple these two. We shall attempt to do it this way: we shall couple the field $\phi$ to the field $\psi$ as follows: $H_{int} = g \int d^3 x T^\mu_{(\psi) \; \mu}(x) \phi(x)$. Here, $T^\mu_{(\psi) \; \mu}(x)$ is the trace of the energy-momentum tensor of the $\psi$ field and $g$ is a coupling constant.
%\begin{enumerate}
%\item
%We shall not dwell much on the dynamics of the $\psi$ field and will just consider its $T^\mu_{(\psi) \; \mu}$ an external source of $\phi$. Still, we need to find a form for  $T^\mu_{(\psi) \; \mu}$ appropriate for a non relativistic static particle.
% Consider...
% \item Find the constant $g$ such that the Newton law between two static masses is correctly reproduced by (\ref{three}).
% \end{enumerate}
% {\small \flushleft{Th}e reason scalar gravity does not work is that the coupling $T^\mu_{(\psi) \; \mu}(x) \phi(x)$ means that the ``scalar graviton" $\phi$ does not couple to the electromagnetic field (remember that $T^\mu_\mu =0$ there). On the other hand, the bending of light in the gravitational field is an experimental fact, so there should be a coupling between the two, not contained in our model. }
%
%\bigskip
%\end{comment}
%
%
} % makeproblem

\makeanswer{qft:problemSet3:4}{
\withproblemsetsParagraph{
\makeSubAnswer{}{qft:problemSet3:4a}
In class we defined the time translation operator as \( U(\Ba) = e^{i \Ba \cdot \hat{\BP} } \), which satisfies the relations\footnote{There is some variation in at least some of the literature.  In particular
\citep{desai2009quantum} defines the translation operator as \( D(\Ba) = e^{-i \Ba \cdot \hat{\BP}/\Hbar} \) defined by the property \( D(\Ba) \ket{\Bx} = \ket{\Bx + \Ba} \).}
\begin{equation}\label{eqn:ProblemSet3Problem4:20}
\begin{aligned}
U(\Ba) \phi(\Bx) U^\dagger(\Ba) &= \phi(\Bx - \Ba) \\
U^\dagger(\Ba) \ket{\Bx} &= \ket{\Bx + \Ba}.
\end{aligned}
\end{equation}
In particular \( \bra{\Bzero} U(\By) = \bra{\By} \) and \( U^\dagger(\By) \ket{\Bzero} = \ket{\By} \).  As \( T^{00} \) is composed entirely of products of \( \phi(\Bx) \) or its derivatives, clearly
\begin{equation}\label{eqn:ProblemSet3Problem4:40}
U(\By) T^{00}(\Bx, t) U^\dagger(\By)
=
T^{00}(\Bx - \By, t),
\end{equation}
so
\begin{equation}\label{eqn:ProblemSet3Problem4:60}
\begin{aligned}
\rho(0, \Bx - \By, t)
&=
\bra{\Bzero, 0} T^{00}(\Bx - \By, t) \ket{\Bzero, 0}
\\&=
\bra{\Bzero, 0} U(\By) T^{00}(\Bx, t) U^\dagger(\By) \ket{\Bzero, 0}
\\&=
\bra{\By, 0} T^{00}(\Bx, t) \ket{\By, 0}
\\&=
\rho(\By, \Bx, t).
\end{aligned}
\end{equation}

\makeSubAnswer{}{qft:problemSet3:4b}
Let's start by computing the energy-momentum tensor
\begin{equation}\label{eqn:ProblemSet3Problem4:80}
\begin{aligned}
T^{00}
&= \partial^0 \phi \partial^0 \phi - g^{00}\LL
\\&= \partial^0 \phi \partial^0 \phi - \inv{2} \lr{
\partial_0 \phi \partial^0 \phi - (\spacegrad \phi)^2 - m^2 \phi^2
}
\\&=
\inv{2}
\lr{
   \partial_0 \phi \partial_0 \phi + (\spacegrad \phi)^2 + m^2 \phi^2
}
\\&=
\inv{2}
\int \frac{d^3 p\, d^3 q}{(2 \pi)^6 2 \sqrt{\omega_\Bp \omega_\Bq}}
\Biglr{
   \partial_0
   \lr{
      a_\Bp e^{-i p \cdot x} + a_\Bp^\dagger e^{i p \cdot x}
   }
   \partial_0
   \lr{
      a_\Bq e^{-i q \cdot x} + a_\Bq^\dagger e^{i q \cdot x}
   }
\\&\quad\quad
+
   \partial_k
   \lr{
      a_\Bp e^{-i p \cdot x} + a_\Bp^\dagger e^{i p \cdot x}
   }
   \partial_k
   \lr{
      a_\Bq e^{-i q \cdot x} + a_\Bq^\dagger e^{i q \cdot x}
   }
\\&\quad\quad
+ m^2
   \lr{
      a_\Bp e^{-i p \cdot x} + a_\Bp^\dagger e^{i p \cdot x}
   }
   \lr{
      a_\Bq e^{-i q \cdot x} + a_\Bq^\dagger e^{i q \cdot x}
   }
}.
\end{aligned}
\end{equation}
For the derivatives, we have
\begin{equation}\label{eqn:ProblemSet3Problem4:100}
\begin{aligned}
\partial_\nu e^{\pm i p \cdot x}
&=
\partial_\nu e^{\pm i p_\mu x^\mu}
\\&=
\pm i p_\nu e^{\pm i p \cdot x},
\end{aligned}
\end{equation}
so
\begin{equation}\label{eqn:ProblemSet3Problem4:120}
\begin{aligned}
T^{00}
&=
\inv{4}
\int \frac{d^3 p\, d^3 q}{(2 \pi)^6 \sqrt{\omega_\Bp \omega_\Bq}} \\
&\quad
\Biglr{
-
   \lr{
      \omega_\Bp \omega_\Bq + \Bp \cdot \Bq
   }
   \lr{
      -a_\Bp e^{-i p \cdot x} + a_\Bp^\dagger e^{i p \cdot x}
   }
   \lr{
      -a_\Bq e^{-i q \cdot x} + a_\Bq^\dagger e^{i q \cdot x}
   }
\\&\quad\quad
+ m^2
   \lr{
      a_\Bp e^{-i p \cdot x} + a_\Bp^\dagger e^{i p \cdot x}
   }
   \lr{
      a_\Bq e^{-i q \cdot x} + a_\Bq^\dagger e^{i q \cdot x}
   }
} \\
&=
\inv{4}
\int \frac{d^3 p\, d^3 q}{(2 \pi)^6 \sqrt{\omega_\Bp \omega_\Bq}} \\
&\quad
\Biglr{
   \lr{
      -\omega_\Bp \omega_\Bq - \Bp \cdot \Bq + m^2
   }
   \lr{
      a_\Bp a_\Bq e^{-i (p + q) \cdot x}
   +
      a_\Bp^\dagger a_\Bq^\dagger e^{i (p + q) \cdot x}
   }
\\&\quad\quad
+
   \lr{
      \omega_\Bp \omega_\Bq + \Bp \cdot \Bq + m^2
   }
   \lr{
      a_\Bp a_\Bq^\dagger e^{i (q - p) \cdot x}
   +
      a_\Bp^\dagger a_\Bq e^{i (p - q) \cdot x}
   }
}
\end{aligned}
\end{equation}
We can justify dropping the \( a_\Bp a_\Bq \) and \( a_\Bp^\dagger a_\Bq^\dagger \) terms in this integral since we are computing \( \tilde{\rho}(\Bx, t) = \bra{\Bzero, 0} T^{00}(\Bx, t) \ket{\Bzero, 0} \), where
\begin{equation}\label{eqn:ProblemSet3Problem4:140}
\begin{aligned}
\bra{\Bzero, 0} &= \bra{0} \int \frac{d^3 r}{(2 \pi)^3 \sqrt{ 2 \omega_\Br }} a_\Br \\
\ket{\Bzero, 0} &= \int \frac{d^3 s}{(2 \pi)^3 \sqrt{ 2 \omega_\Bs }} a_\Bs^\dagger \ket{0},
\end{aligned}
\end{equation}
so those terms only contribute zeros
\begin{equation}\label{eqn:ProblemSet3Problem4:300}
\begin{aligned}
0 &= \bra{0} a_\Br a_\Bp a_\Bq a_\Bs^\dagger \ket{0} \\
0 &= \bra{0} a_\Br a_\Bp^\dagger a_\Bq^\dagger a_\Bs^\dagger \ket{0}.
\end{aligned}
\end{equation}
These zeros are easily computed by commutation, but also by the Wick's corollary mentioned in class (expectations of odd numbers of creation or annihilation operators are zero).
With those same sign \((p,q)\) exponential terms eliminated and a \( p, q \) swap in the \( a_\Bp a_\Bq^\dagger \) term, we are left with
\begin{equation}\label{eqn:ProblemSet3Problem4:320}
T^{00} =
\inv{4}
\int \frac{d^3 p\, d^3 q}{(2 \pi)^6 \sqrt{\omega_\Bp \omega_\Bq}}
   \lr{
      \omega_\Bp \omega_\Bq + \Bp \cdot \Bq + m^2
   }
   \lr{
      a_\Bq a_\Bp^\dagger
   +
      a_\Bp^\dagger a_\Bq
   }
   e^{i (p - q) \cdot x}.
\end{equation}
Normal ordered, we have
\begin{equation}\label{eqn:ProblemSet3Problem4:340}
\normalorder{ T^{00} } =
\inv{2}
\int \frac{d^3 p\, d^3 q}{(2 \pi)^6 \sqrt{\omega_\Bp \omega_\Bq}}
   \lr{
      \omega_\Bp \omega_\Bq + \Bp \cdot \Bq + m^2
   }
      a_\Bp^\dagger a_\Bq
   e^{i (p - q) \cdot x}.
\end{equation}
We expect this to equal the Hamiltonian density, and can check that as a quick sanity check
\begin{equation}\label{eqn:ProblemSet3Problem4:360}
\begin{aligned}
\int d^3 x \normalorder{ T^{00} }
&=
\inv{2}
\int \frac{d^3 x\, d^3 p\, d^3 q}{(2 \pi)^6 \sqrt{\omega_\Bp \omega_\Bq}}
   \lr{
      \omega_\Bp \omega_\Bq + \Bp \cdot \Bq + m^2
   }
      a_\Bp^\dagger a_\Bq
   e^{i (\omega_\Bp - \omega_\Bq) t} e^{-i (\Bp - \Bq) \cdot \Bx}
   \\&=
\inv{2}
\int \frac{d^3 p\, d^3 q}{(2 \pi)^3 \sqrt{\omega_\Bp \omega_\Bq}}
   \lr{
      \omega_\Bp \omega_\Bq + \Bp \cdot \Bq + m^2
   }
      a_\Bp^\dagger a_\Bq
   e^{i (\omega_\Bp - \omega_\Bq) t} \delta(\Bq - \Bp)
   \\&=
\inv{2}
\int \frac{d^3 p}{(2 \pi)^3 \omega_\Bp }
   \lr{
      \omega_\Bp^2 + \Bp^2 + m^2
   }
   a_\Bp^\dagger a_\Bp
   \\&=
\inv{2}
\int \frac{d^3 p}{(2 \pi)^3 \omega_\Bp }
   2 \omega_\Bp^2
   a_\Bp^\dagger a_\Bp
   \\&=
H.
\end{aligned}
\end{equation}

We are now ready to complete the computation of \( \tilde{\rho}(x) \), which is
\begin{equation}\label{eqn:ProblemSet3Problem4:180}
\tilde{\rho}(x)
=
\inv{4}
\int \frac{d^3 r\, d^3 p\, d^3 q\, d^3 s}{(2 \pi)^{12} \sqrt{\omega_\Br \omega_\Bp \omega_\Bq \omega_\Bs}}
   \lr{
      \omega_\Bp \omega_\Bq + \Bp \cdot \Bq + m^2
   }
   \bra{0} a_\Br a_\Bp^\dagger a_\Bq a_\Bs^\dagger \ket{0}
   e^{i (p - q) \cdot x}.
\end{equation}
%%%\begin{equation}\label{eqn:ProblemSet3Problem4:200}
%%%\bra{0} a_\Br a_\Bp a_\Bq a_\Bs^\dagger \ket{0}
%%%=
%%%\bra{0} a_\Br a_\Bp \lr{ a_\Bs^\dagger + (2 \pi)^3 \delta(\Bq - \Bs)} \ket{0}
%%%= 0,
%%%\end{equation}
%%%and
%%%\begin{equation}\label{eqn:ProblemSet3Problem4:220}
%%%\bra{0} a_\Br a_\Bp^\dagger a_\Bq^\dagger a_\Bs^\dagger \ket{0}
%%%=
%%%\bra{0} \lr{ a_\Bp^\dagger a_\Br + (2 \pi)^3 \delta( \Br - \Bp ) } a_\Bq^\dagger a_\Bs^\dagger \ket{0}
%%%= 0,
%%%\end{equation}
Evaluating this matrix element with Wick's theorem, we have
%%%\begin{equation}\label{eqn:ProblemSet3Problem4:240}
%%%\bra{0} a_\Br a_\Bp a_\Bq^\dagger a_\Bs^\dagger \ket{0}
%%%=
%%%\contraction{}{a}{{}_\Br a_\Bp}{a}{}
%%%\contraction[2ex]{}{a}{{}_\Br a_\Bp a_\Bq^\dagger}{a}
%%%\bcontraction{a_\Br}{a}{{}_\Bp}{a}{}
%%%\bcontraction[2ex]{a_\Br}{a}{{}_\Bp a_\Bq^\dagger}{a}
%%%a_\Br a_\Bp a_\Bq^\dagger a_\Bs^\dagger
%%%=
%%%(2\pi)^6 \delta(\Br - \Bq) \delta(\Bp - \Bs)
%%%+
%%%(2\pi)^6 \delta(\Br - \Bs) \delta(\Bp - \Bq),
%%%\end{equation}
%%%and
\begin{equation}\label{eqn:ProblemSet3Problem4:260}
\begin{aligned}
\bra{0} a_\Br a_\Bp^\dagger a_\Bq a_\Bs^\dagger \ket{0}
&=
\contraction{}{a}{{}_\Br}{a}
\contraction{ a_\Br a_\Bp^\dagger}{a}{{}_\Bq}{a}
a_\Br a_\Bp^\dagger a_\Bq a_\Bs^\dagger
\\&=
(2 \pi)^6 \delta(\Br - \Bp) \delta(\Bq - \Bs),
\end{aligned}
\end{equation}
so
\begin{equation}\label{eqn:ProblemSet3Problem4:380}
\begin{aligned}
\tilde{\rho}(x)
&=
\inv{4}
\int \frac{d^3 p\, d^3 q}{(2 \pi)^{6} \omega_\Bp \omega_\Bq}
   \lr{
      \omega_\Bp \omega_\Bq + \Bp \cdot \Bq + m^2
   }
   e^{i (p - q) \cdot x}
\\&=
\int \frac{d^3 p}{(2 \pi)^{3} 2 \omega_\Bp} \omega_\Bp e^{i p \cdot x}
\int \frac{d^3 q}{(2 \pi)^{3} 2 \omega_\Bq} \omega_\Bq e^{i q \cdot (-x)}
\\&\quad
+
\int \frac{d^3 p}{(2 \pi)^{3} 2 \omega_\Bp} \Bp e^{i p \cdot x}
\cdot
\int \frac{d^3 q}{(2 \pi)^{3} 2 \omega_\Bq} \Bq e^{i q \cdot (-x)}
\\&\quad
+
m^2
\int \frac{d^3 p}{(2 \pi)^{3} 2 \omega_\Bp} e^{i p \cdot x}
\int \frac{d^3 q}{(2 \pi)^{3} 2 \omega_\Bq} e^{i q \cdot (-x)},
\end{aligned}
\end{equation}
which is just
\boxedEquation{eqn:ProblemSet3Problem4:400}{
\tilde{\rho}(x)
=
\partial_t D(x) \partial_t D(-x)
+
\lr{ \spacegrad D(x) } \cdot \lr{ \spacegrad D(-x) }
+
m^2
D(x) D(-x).
}

\makeSubAnswer{}{qft:problemSet3:4c}
In homework 2 we found that at a spacelike distance \( x = (0, r \rcap) \) the Wightman function had the form
\begin{equation}\label{eqn:ProblemSet3Problem4:420}
D(r, 0) \sim e^{-m r},
\end{equation}
where \( \rcap \) is the unit vector directed along the line from the origin to \( \Bx \).
We wish to evaluate the gradients of \( D(\Bx, 0) \) and \( D(-\Bx, 0) \), and may do so by evaluating each with respect to oppositely oriented coordinate systems.
\begin{equation}\label{eqn:ProblemSet3Problem4:440}
\begin{aligned}
\spacegrad D(\Bx, 0)
&=
\rcap \PD{r}{} e^{-m r}
\\&=
-m \rcap e^{-m r},
\end{aligned}
\end{equation}
and
\begin{equation}\label{eqn:ProblemSet3Problem4:460}
\begin{aligned}
\spacegrad D(-\Bx, 0)
&=
(-\rcap) \PD{r}{} e^{-m r}
\\&=
m \rcap e^{-m r},
\end{aligned}
\end{equation}
so
\begin{equation}\label{eqn:ProblemSet3Problem4:480}
\begin{aligned}
\tilde{\rho}(\Bx, 0)
&=
\lr{ -m \rcap e^{-m r} } \cdot
\lr{ m \rcap e^{-m r} }
+ m^2 e^{-2 mr}
\\&=
0.
\end{aligned}
\end{equation}
We have perfect cancellation at spacelike separations.

I am comforted and not surprised that we don't find observable effects at spacelike separations where we don't expect to find them.
}
}


   \chapter{Problem Set 4.}

      %
% Copyright � 2018 Peeter Joot.  All Rights Reserved.
% Licenced as described in the file LICENSE under the root directory of this GIT repository.
%
\makeproblem{The Wick theorem(s)}{qft:problemSet4:1}{
 
 \begin{enumerate}
\item  {\bf The mother of all Wick theorem(s): }
Let $A_1, A_2,... $ and $B$ denote a set of either creation or annihilation operators. In other words, $A_i = a_{k_i}$ or $a_{k_i}^\dagger$ (as well as $B$; $B$ is just like one of the $A$'s, but we'll use the letter $B$ to denote an operator which is singled out, as it is needed in the proof). Next, define a contraction $A_i A_k$ as follows:
\begin{equation}
\label{c1}
\contraction{O_1}{A_i}{O_2}{A_k}
O_1 A_i O_2 A_k =   O_1 O_2 \contraction{}{A_i}{} {A_k}  A_i A_k ~,
 \end{equation}
 where $O_1$, $O_2$ are arbitrary strings of operators. The above equation signifies the fact that the ``contraction" is a $c$-number, i.e. commutes with all operators. It is defined as follows:
  \begin{equation}
  \contraction{}{A_i}{} {A_j}  A_i A_j = \left\{ \begin{array}l 0, \; {\rm if } \; \; A_i = a_{k_i}, A_j = a_{k_j} \;\; {\rm or} \;\; A_i = a_{k_i}^\dagger, A_j = a_{k_j}^\dagger\cr 0, \; {\rm if } \; \; A_i = a_{k_i}^\dagger \; \; {\rm and} \; \; A_j = a_{k_j} \cr
(2 \pi)^3  \delta^{(3)}(k_i - k_j),  \; {\rm if } \; \; A_i = a_{k_i} \; \; {\rm and} \; \; A_j = a_{k_j}^\dagger  
 \end{array}~ \right.
\label{c2}
 \end{equation}
 Put in words, the contraction vanishes if both $A$'s are creation (or both are annihilation operators), as indicated in the first line in (\ref{c2}). The contraction is also zero if the operator to the right is an annihilation one, as per the second line in (\ref{c2}). Finally, the contraction is equal to the commutator of $a_{k_i}$ with $a_{k_j}^\dagger$ in the case when the creation operator is to the left of the annihilation operator. 
 
 Finally, we use $:A \ldots B:$ to denote the expression where all annihilation operators appear to the right of all creation operators, i.e. the usual normal ordered expression. Then, 
 Wick's theorem---as used in many body physics---is formulated as follows:
 \begin{eqnarray}
 \label{w1}
 A_1 \ldots A_n &=& \;\;\; :A_1 \ldots A_n:\nonumber \\ 
 & & + :\contraction{}{A_1}{}{A_2} A_1 A_2 A_3\ldots A_n: + \ldots + :\contraction{}{A_1}{\ldots}{A_{n-1}} A_1 \ldots A_{n-1} A_n: + :\contraction{}{A_1}{\ldots}{A_{n}} A_1 \ldots A_{n}: \\ 
 && + :\contraction{}{A_1}{}{A_2} A_1 A_2 \contraction{}{A_3}{}{A_4} A_3 A_4 \ldots A_n: + \ldots \nonumber~.
 \end{eqnarray}
 The first line contains the normal-ordered product of all operators without contractions, the second line---all possible terms with one contraction (not involving only $A_1$ of course, but all single-contraction terms, which would be painful to indicate), the third line has all possible two-contraction terms, etc. 

Now, you will prove (\ref{w1}) in steps. 
\begin{enumerate}
\item Prove the following Lemma:
 \begin{equation}
 \label{lemma}
  :A_1 A_2 \ldots A_n: B =  :A_1 A_2 \ldots A_n  B: + \sum\limits_{1 \le k \le n}:A_1 \ldots \contraction{}{A_k}{\ldots A_n}{B} A_k \ldots A_n B:
 \end{equation} Argue that if $B$ is an annihilation operator, the Lemma is trivial. Thus, consider $B$ to be a creation operator.
 Notice  that if any of the $A_{1,...,n}$ are creation operators, they can be taken to the left of the normal products in (\ref{lemma}) (because all their contractions with $B$ are zero). Thus, if the (\ref{lemma}) is proven for arbitrary $n$ for the case when all $A_i$'s are annihilation operators, the general case is obtained by multiplying on the left with the desired number of creation operators. Thus, it suffices to prove the Lemma for the case when all $A_i$'s are annihilation operators. Also
notice 
  Thus, after proving the Lemma for $n=1$, use induction to show that it holds for any $n$. Assuming it holds for some number $n$, go to the case $n+1$ by multiplying (\ref{lemma})  by some annihilation operator $A_0$ on the left and show that the Lemma holds for $n+1$ operators.   
  
  By the chain of logic described above, you have proven (\ref{lemma}).

Notice also that the lemma (\ref{lemma}) holds also if the product $$:A_1 A_2 \ldots A_n:$$ is replaced by $$:A_1 \contraction{}{A_2}{\ldots}{A_p} A_2 \ldots  A_p \ldots A_n:,$$
  i.e. with the product of operators with an arbitrary number of contractions (one, as written above), with a trivial modification of the last term (since, obviously, you can not contract $B$ with contractions).
  \item Now prove the actual Wick theorem (\ref{w1}). Assuming that it holds for $n=2$. Imagine that (\ref{w1}) holds for $n$ operators and prove that it holds for $n+1$, using (\ref{lemma}).
  \end{enumerate}
  \item {\bf An intermediate step:} Let now $A_i$ and $B$ be operators expressed as some linear combinations of creation and annihilation operators. In particular the subscripts $i$ may now indicate spatial dependence, rather than momentum eigenvalues. Now, define the contraction as follows:
   \begin{equation}
  \contraction{}{A_i}{} {A_j}  A_i A_j = \langle 0 \vert A_i A_j \vert 0 \rangle~,
    \label{c5}
 \end{equation}
 where $\vert 0 \rangle$ is the Fock vacuum.
Notice that (\ref{c5}) is equivalent to (\ref{c2}) when $A_i$'s are either creation or annihilation operators. Argue that (\ref{w1}) holds verbatim. 

  \item {\bf The time-ordered Wick theorem:} Use the above Wick theorem to prove the time-ordered version. Notice that, despite appearances, there is not much left to do. Now, we have space-time rather than momentum space arguments and the theorem is now formulated as follows:\begin{eqnarray}
 \label{w2}
T( A_1 \ldots A_n ) &=& \;\;\; :A_1 \ldots A_n:\nonumber \\ 
 & & + :\contraction{}{A_1}{}{A_2} A_1 A_2 A_3\ldots A_n: + \ldots + :\contraction{}{A_1}{\ldots}{A_{n-1}} A_1 \ldots A_{n-1} A_n: + :\contraction{}{A_1}{\ldots}{A_{n}} A_1 \ldots A_{n}: \\ 
 && + :\contraction{}{A_1}{}{A_2} A_1 A_2 \contraction{}{A_3}{}{A_4} A_3 A_4 \ldots A_n: + \ldots \nonumber~,
 \end{eqnarray}
 with the difference that $A_i$ are fields (we are considering real scalar fields),  $1 \ldots n$ denote space-time points, and the contraction is now the Feynman propagator, e.g. $D_F(x_1-x_2)$, etc. 
 
 Notice that  to prove (\ref{w2}) one can consider a particular time ordering. Then the $T$ product becomes the normal product of operators (as they are assumed ordered). The space-time dependence can be taken out by Fourier transform which multiplies every term.  Every operator is a sum of creation and annihilation operators. Their commutators are exactly the ones giving rise to the contraction in (\ref{c2}), on one hand, and to the function $D(x_i -x_j)$ after Fourier transform, on the other (recall that this function appears in the Feynman propagator).
Convince yourselves, using (\ref{c5}),  that this proves the theorem.
 
 \item For extra bonus, generalize all theorems above to anti commuting fields.
\end{enumerate}
%\makesubproblem{}{qft:problemSet4:1a}
%\makesubproblem{}{qft:problemSet4:1b}
%\makesubproblem{}{qft:problemSet4:1c}
%\makesubproblem{}{qft:problemSet4:1d}
} % makeproblem

\makeanswer{qft:problemSet4:1}{
%\makeSubAnswer{}{qft:problemSet4:1a}
%TODO.
%\makeSubAnswer{}{qft:problemSet4:1b}
%TODO.
%\makeSubAnswer{}{qft:problemSet4:1c}
%TODO.
%\makeSubAnswer{}{qft:problemSet4:1d}
%TODO.
}

      %
% Copyright � 2018 Peeter Joot.  All Rights Reserved.
% Licenced as described in the file LICENSE under the root directory of this GIT repository.
%
\makeproblem{The ``$h \rightarrow WW, ZZ$" Higgs-decay width.}{qft:problemSet4:2}{
From the $SU(2)_L \times SU(2)_R$ model of Homework 2---really, the Higgs Lagrangian of the Standard Model, find the coupling of the $h$-particle (the Higgs boson) to the $\phi^a$ particles (these are now Goldstone bosons, in the electroweak theory, they become the longitudinal components of the $W$ and $Z$ particles). Canonically normalizing $h$ and $\phi^a$, this coupling has the form
\begin{equation}
\label{gg1}
const. \; h \; \partial_\mu \phi^a \partial^\mu \phi^a~.
\end{equation}

\makesubproblem{}{qft:problemSet4:2a}
 Determine the value of $const.$ for canonically normalized $h$ and $\phi^a$.
\makesubproblem{}{qft:problemSet4:2b}
Use this coupling to compute the width  $\Gamma(h \rightarrow \phi^3 \phi^3)$ of the Higgs particle to decay to two longitudinal (say) $Z$-bosons (hence the index $3$).
\makesubproblem{}{qft:problemSet4:2c}
 Plug in some numbers. Use the fact that the vacuum expectation value $|m|/\sqrt{\lambda}  = 246$ GeV  and the fact that $m_h = 125$GeV to get a number for the lifetime. Compare to the total width of the Higgs from \url{http://pdg.lbl.gov/2012/reviews/rpp2012-rev-higgs-boson.pdf}, see figure 5 there, as well to the partial width to $WW$ given in Figure 4 there.
\makesubproblem{}{qft:problemSet4:2d}
 At the same time, determine the values of $|m|$ and $\lambda$ separately. Is $\lambda \ll 1$ (i.e. perturbative)?

{\flushleft {\small Notice that this calculation would have been physically relevant had the Higgs been heavy, $m_h \gg m_W \sim 100$ GeV. This is because the $h\rightarrow WW$ decay then is dominated (in this limit) by the decay into the longitudinal component, which is really the Goldstone boson field $\phi^a$ (in this limit, the result is independent of the gauge couplings $g_{1,2}$ of the Standard Model). Nonetheless, having some real numbers in this class is good.}}
} % makeproblem

\makeanswer{qft:problemSet4:2}{
\makeSubAnswer{}{qft:problemSet4:2a}
Here's a reminder and summary of the Higgs Lagrangian we will be working with in this problem
\begin{dmath}\label{eqn:ProblemSet4Problem2:640}
\LL = \trace{
   \lr{
      \partial_\mu H^\dagger \partial^\mu H
   }
}
- V,
\end{dmath}
where
\begin{dmath}\label{eqn:ProblemSet4Problem2:660}
V =
-\Abs{m}^2 \trace{
   \lr{
      H^\dagger H
   }
}
+ \lambda
\lr{
   \trace{
      H^\dagger H
   }
}^2.
\end{dmath}
It was postulated that the field had a radial component \( h \), the Higgs field, and an rotational component \( \Omega \), where the total field was given by
\begin{dmath}\label{eqn:ProblemSet4Problem2:680}
H(x) = \frac{\Abs{m}}{2 \sqrt{ \lambda } }\Omega(x) ( 1 + h(x) ),
\end{dmath}
where
\begin{equation}\label{eqn:ProblemSet4Problem2:700}
\Omega = e^{ i \Bsigma \cdot \Bphi } = e^{i \phi^a(x) \sigma^a }.
\end{equation}

Assuming that \( h(x) \) and \( \phi^a(x) \) commute, \( H^\dagger H \) can be computed with relative ease, and has only radial dependence
\begin{dmath}\label{eqn:ProblemSet4Problem2:380}
\trace{\lr{H^\dagger H}}
=
\frac{\Abs{m}^2}{4 \lambda} (1 + h(x))^2 \trace{\lr{ e^{-i \Bsigma \cdot \Bphi} e^{i \Bsigma \cdot \Bphi} }}
=
\frac{\Abs{m}^2}{4 \lambda} (1 + h(x))^2 \trace{\BOne}
=
\frac{\Abs{m}^2}{2 \lambda} (1 + h)^2.
\end{dmath}
For the derivative quadratic form, we find
\begin{dmath}\label{eqn:ProblemSet4Problem2:400}
\partial_\mu H^\dagger \partial^\mu H
=
\frac{\Abs{m}^2}{4 \lambda}
\lr{
   \partial_\mu h \Omega^\dagger
   + (1 + h) \partial_\mu \Omega^\dagger
}
\lr{
   \partial^\mu h \Omega
   + (1 + h)
\partial^\mu \Omega
}
=
\frac{\Abs{m}^2}{4 \lambda}
\lr{
   \partial_\mu h \Omega^\dagger \partial^\mu h \Omega
   + (1 + h)
      \lr{
         \partial_\mu h
         \Omega^\dagger (\partial^\mu \Omega)
       +
         \partial^\mu h
         (\partial_\mu \Omega^\dagger) \Omega
      }
   + (1 + h)^2 \partial_\mu \Omega^\dagger \partial^\mu \Omega
}.
\end{dmath}
Because \( \Omega^\dagger \Omega = 1 \), we have
\begin{dmath}\label{eqn:ProblemSet4Problem2:480}
\partial_\mu h
\Omega^\dagger (\partial^\mu \Omega)
 +
\partial^\mu h
(\partial_\mu \Omega^\dagger) \Omega
=
\partial_\mu h
\lr{
   \Omega^\dagger (\partial^\mu \Omega)
    +
   (\partial^\mu \Omega^\dagger) \Omega
}
=
\partial_\mu h
\lr{
   \partial^\mu (\Omega^\dagger \Omega) - (\partial^\mu \Omega^\dagger) \Omega
    +
   (\partial^\mu \Omega^\dagger) \Omega
}
=
   \partial^\mu (1)
= 0.
\end{dmath}
All the cross terms with both \( h \) and \( \Omega \) derivatives are zero (to all orders, not just quadratic).

Taking traces (and using cyclic permutation of the matrices in the trace operations),
the Lagrangian density is now determined
\begin{dmath}\label{eqn:ProblemSet4Problem2:500}
\LL =
\frac{\Abs{m}^2}{2 \lambda}
   \partial_\mu h \partial^\mu h
+
\frac{\Abs{m}^2}{4 \lambda} ( 1 + h )^2
   \trace{\lr{
      \partial_\mu \Omega^\dagger \partial^\mu \Omega
   }}
+ \Abs{m}^2
\frac{\Abs{m}^2}{2 \lambda} \lr{ 1 + h }^2
- \lambda
\lr{\frac{\Abs{m}^2}{2 \lambda}}^2
\lr{ 1 + h }^4
=
\frac{\Abs{m}^2}{\lambda} \LL',
\end{dmath}
where \( \LL' \) is the ``canonically normalized'' \footnote{Canonically normalized is assumed to mean that there's a one-half factor on the kinetic terms} Lagrangian
\begin{dmath}\label{eqn:ProblemSet4Problem2:720}
\LL' =
   \inv{2} \partial_\mu h \partial^\mu h
+
\inv{4}
( 1 + h )^2
   \trace{\lr{
      \partial_\mu \Omega^\dagger \partial^\mu \Omega
   }}
+
\inv{2}
\Abs{m}^2
\lr{ 1 + h }^2
-
\frac{\Abs{m}^2}{4}
\lr{ 1 + h }^4.
\end{dmath}

The coupling, let's call it \( c \), is to first order in \( h \) is
\begin{dmath}\label{eqn:ProblemSet4Problem2:740}
c = \inv{4} 2 h
   \trace{\lr{
      \partial_\mu \Omega^\dagger \partial^\mu \Omega
   }}.
\end{dmath}
Looking at these derivatives, to first order, we have
\begin{dmath}\label{eqn:ProblemSet4Problem2:580}
\partial_\mu \Omega
=
\partial_\mu \lr{ \BOne + i \Bsigma \cdot \Bphi }
=
i \Bsigma \cdot \partial_\mu \Bphi,
\end{dmath}
so
\begin{dmath}\label{eqn:ProblemSet4Problem2:760}
c
=
\inv{2} h
\trace{\lr{
   (-i \Bsigma \cdot \partial_\mu \Bphi^\dagger)
   (i \Bsigma \cdot \partial^\mu \Bphi)
}}
=
\inv{2}
h
\trace{\lr{
   (\Bsigma \cdot \partial_\mu \Bphi)
   (\Bsigma \cdot \partial^\mu \Bphi)
}},
\end{dmath}
where the real nature of each of the \( \phi^a \)'s has been used to eliminate the \( \dagger\).
The structure of this trace is that of
\begin{dmath}\label{eqn:ProblemSet4Problem2:780}
\trace{\lr{
   (\Bsigma \cdot \Bx)
   (\Bsigma \cdot \By)
}}
=
x^a y^b
\trace{\lr{
   \sigma^a \sigma^b
}}
=
x^a y^b
\left\{
\begin{array}{l l}
2 & \quad \mbox{\( a = b \)} \\
0 & \quad \mbox{\( a \ne b \)} \\
\end{array}
\right.
=
2 \Bx \cdot \By,
\end{dmath}
so the coupling is
\begin{dmath}\label{eqn:ProblemSet4Problem2:800}
c = h
\partial_\mu \phi^a \partial^\mu \phi^a.
\end{dmath}
This answers the question of the constant, which we find is just \( 1 \) after canonical normalization.

\makeSubAnswer{}{qft:problemSet4:2b}

The scattering calculation machine that was presented in class, assumes that the final states scattering process can be related to a scattering matrix with the following structure
\begin{equation}\label{eqn:ProblemSet4Problem2:820}
\prescript{}{\text{out}}{\braket{\Bp_1, \Bp_2}{\Bk}}_{\text{in}}
=
\bra{ \Bp_1, \Bp_2 } \hat{S} \ket{ \Bk}
=
\bra{ \Bp_1, \Bp_2 } T\lr{ e^{-i\int dt H_{\text{int}}(t)} } \ket{ \Bk}
\end{equation}
where \( \Bk \) is the momentum of the initial higgs particle, \( \Bp_i \) are the momenta of the Z-boson disinegration products, and we evaluate the amplitude by summing ``connected amputated diagrams''.

With the plus-minus decomposition of the field \( h(z) = h^{+}(z) + h^{-}(z) \), contracting the field with the initial momentum state gives
\begin{dmath}\label{eqn:ProblemSet4Problem2:840}
h(z) \ket{\Bk}
=
h^{+}(z) \ket{\Bk}
=
\int \frac{d^3 q}{(2 \pi)^3 \sqrt{2 \omega_\Bq} } e^{-i q \cdot z} a_\Bq \ket{\Bk}
=
\int \frac{d^3 q}{(2 \pi)^3 \sqrt{2 \omega_\Bq} } e^{-i q \cdot z} a_\Bq \sqrt{ 2 \omega_\Bk } a_\Bk^\dagger \ket{0}
=
\int d^3 q e^{-i q \cdot z} \deltathree( \Bq - \Bk ) \ket{0}
=
e^{-i k \cdot z}.
\end{dmath}
Similarly,
\begin{dmath}\label{eqn:ProblemSet4Problem2:860}
\bra{\Bp} \phi^3(z)
=
e^{i p \cdot z}.
\end{dmath}

Apparently\footnote{According to a ``trust-me, it's a long story'' kind of statement related to a classmate from Professor Poppitz.}, the interaction Hamiltonian density that we want to use for this problem is \( H_{\text{int}} = -\LL_{\text{int}} \).  Given that, the
\begin{dmath}\label{eqn:ProblemSet4Problem2:880}
-i \int dt H_{\text{int}}(t)
=
i \int dt \int d^3 x h(x) \partial_\mu \phi^a(x) \partial^\mu \phi^a(x).
=
i \int d^4 x h(x) \partial_\mu \phi^a(x) \partial^\mu \phi^a(x).
\end{dmath}
so the
first order expansion of the scattering amplitude is
\begin{dmath}\label{eqn:ProblemSet4Problem2:900}
i \int d^4 x \bra{ \Bp_1, \Bp_2 } T\lr{
   h(x) \partial_\mu \phi^a(x) \partial^\mu \phi^a(x)
} \ket{ \Bk}.
\end{dmath}
There are two possible diagrams associated with this amplitude, sketched in \cref{fig:hw4p2}, but only the first qualifies as ``connected amputated''.
\imageTwoFigures
{../figures/phy2403-quantum-field-theory/hw4p2decayFig1}
{../figures/phy2403-quantum-field-theory/hw4p2VirtualProcessFig2}
{Possible figures}{fig:hw4p2}{scale=0.2}

Algebraically, in terms of contractions the first diagram is
\begin{dmath}\label{eqn:ProblemSet4Problem2:920}
i \int d^4 x
\contraction{\langle }{\Bp}{{}_1, \Bp_2 \rvert h(x) \partial_\mu }{\phi}
\contraction[2ex]{\langle \Bp_1, }{\Bp}{{}_2 \rvert h(x) \partial_\mu \phi^a(x) \partial^\mu }{\phi}
\bcontraction{\langle \Bp_1, \Bp_2 \rvert }{h}{(x) \partial_\mu \phi^a(x) \partial^\mu \phi^a(x) \lvert}{\Bk}
\langle \Bp_1, \Bp_2 \rvert h(x) \partial_\mu \phi^a(x) \partial^\mu \phi^a(x) \lvert \Bk \rangle,
\end{dmath}
however, since \( \Bp_i \) are the momenta for \( \phi^3 \) particles, only the \( a = 3 \) terms above contribute, leaving
\begin{dmath}\label{eqn:ProblemSet4Problem2:940}
i \int d^4 x
\contraction{\langle }{\Bp}{{}_1, \Bp_2 \rvert h(x) \partial_\mu }{\phi}
\contraction[2ex]{\langle \Bp_1, }{\Bp}{{}_2 \rvert h(x) \partial_\mu \phi^3(x) \partial^\mu }{\phi}
\bcontraction{\langle \Bp_1, \Bp_2 \rvert }{h}{(x) \partial_\mu \phi^3(x) \partial^\mu \phi^3(x) \lvert}{\Bk}
\langle \Bp_1, \Bp_2 \rvert h(x) \partial_\mu \phi^3(x) \partial^\mu \phi^3(x) \lvert \Bk \rangle
=
i \int d^4 x \bra{0} \partial_\mu e^{i p_1 \cdot x} \partial^\mu e^{i p_2 \cdot x} e^{-i k \cdot x} \ket{0}
=
i \int d^4 x \bra{0} (i (p_1)_\mu)(i (p_2)^\mu) e^{i (p_1 + p_2 - k) \cdot x} \ket{0}
=
-i \int d^4 x \bra{0} p_1 \cdot p_2 e^{i (p_1 + p_2 - k) \cdot x} \ket{0}
=
-i p_1 \cdot p_2 (2 \pi)^4 \deltafour(p_1 + p_2 - k).
\end{dmath}
This equals \( i \calM_{fi} (2 \pi)^4 \deltafour(p_1 + p_2 - k) \), so
\begin{dmath}\label{eqn:ProblemSet4Problem2:960}
\calM_{fi} = -p_1 \cdot p_2.
\end{dmath}
We can how start plugging this into our decay rate formula
\begin{dmath}\label{eqn:ProblemSet4Problem2:980}
\Gamma = \inv{2 M} \int d(LIPS)_2 \Abs{\calM_{fi}}^2,
\end{dmath}
where
\begin{dmath}\label{eqn:ProblemSet4Problem2:1000}
d(LIPS)_2
=
(2 \pi)^4 \deltafour(p_1 + p_2 - k)
\frac{d^3 p_1}{(2 \pi)^3 2 \omega_{\Bp_1} }
\frac{d^3 p_2}{(2 \pi)^3 2 \omega_{\Bp_2} }
=
(2 \pi)^4 \deltathree(\Bp_1 + \Bp_2 - \Bk) \delta(\omega_1 + \omega_2 - \omega_I)
\frac{d^3 p_1}{(2 \pi)^3 2 \omega_{\Bp_1} }
\frac{d^3 p_2}{(2 \pi)^3 2 \omega_{\Bp_2} }
\end{dmath}
(Prof. Luke's notes call this last factor \( D \).)
\begin{dmath}\label{eqn:ProblemSet4Problem2:1020}
\Gamma
= \inv{2 M} \int
\frac{d^3 p_1}{(2 \pi)^3 2 \omega_{\Bp_1} }
\frac{d^3 p_2}{(2 \pi)^3 2 \omega_{\Bp_2} }
(2 \pi)^4 \deltathree(\Bp_1 + \Bp_2 - \Bk) \delta(\omega_{\Bp_1} + \omega_{\Bp_2} - \omega_I)
(-p_1 \cdot p_2)^2.
\end{dmath}
This is simplest to evaluate in the CM frame, as sketched in \cref{fig:hw4p2CenterOfMassFrame:hw4p2CenterOfMassFrameFig3}, where \( \Bk = 0 \).  This leaves
\imageFigure{../figures/phy2403-quantum-field-theory/hw4p2CenterOfMassFrameFig3}{Center of mass frame.}{fig:hw4p2CenterOfMassFrame:hw4p2CenterOfMassFrameFig3}{0.3}
\begin{dmath}\label{eqn:ProblemSet4Problem2:1040}
\Gamma
= \inv{2 M} \int
\frac{d^3 p_1}{(2 \pi)^2 4 \omega_{\Bp_1}^2}
\delta(\omega_{\Bp_1} + \omega_{\Bp_2} - \omega_I)
\evalbar{(p_1 \cdot p_2)^2}{\Bp_2 = -\Bp_1}.
\end{dmath}
If
\begin{dmath}\label{eqn:ProblemSet4Problem2:1060}
\begin{aligned}
p_1 &= (\omega_{\Bp_1}, \Bp_1) \\
p_2 &= (\omega_{\Bp_2}, \Bp_2),
\end{aligned}
\end{dmath}
then
\begin{dmath}\label{eqn:ProblemSet4Problem2:1080}
p_1 \cdot p_2
= \omega_{\Bp_1} \omega_{\Bp_2} - \Bp_1 \cdot \Bp_2,
\end{dmath}
and
\begin{dmath}\label{eqn:ProblemSet4Problem2:1100}
\evalbar{p_1 \cdot p_2}{\Bp_2 = -\Bp_1}
= \omega_{\Bp_1} \omega_{\Bp_1} + \Bp_1^2
\end{dmath}

\makeSubAnswer{}{qft:problemSet4:2c}
TODO.
\makeSubAnswer{}{qft:problemSet4:2d}
TODO.
}

      %
% Copyright � 2018 Peeter Joot.  All Rights Reserved.
% Licenced as described in the file LICENSE under the root directory of this GIT repository.
%
% original title:
% The Goldstone boson scattering cross-section, its growth with $E_{c.m.}$, and the Higgs.
\makeoproblem{Goldstone boson X-section, and Higgs.}{qft:problemSet4:3}{2018 Hw4.III}{
\index{Higgs}
\index{scattering cross section!Goldstone boson}

{\flushleft{This}} problem has:
\begin{itemize}
\item A great historical significance, for giving an argument in favor of the  existence of a Higgs particle. The strongest argument for the Higgs particle's existence was that it was required---within the  weakly coupled scenario of electroweak symmetry breaking---to tame the growth of the $WW$ scattering amplitude and restore unitarity of the electroweak theory.  Unitarity is a sacred thing and we don't want to easily give it up.
\item A great future significance: measurements of $WW$ scattering at the LHC (and future colliders) will test the Higgs model precisely, in particular the hypothesis that the Higgs particle that was found last year {\it completely} restores unitarity and there is no other state required. Current measurements of $WW$ scattering at the LHC are not just not complete, they are nonexistent (and are very difficult, I am told), hence the  question of whether ``the Higgs is {\it the} Higgs" is still open.
\end{itemize}

{\flushleft{N}}ow, to the concrete stuff:

\makesubproblem{}{qft:problemSet4:3a}
You will calculate the scattering amplitude of Goldstone boson quanta via Higgs exchange, due to the coupling you found in eq. (1) of Problem 2. To be definite, study the amplitude ${M}(\phi^1 \phi^1 \rightarrow \phi^3 \phi^3)$ (I am being very nice here, as I let you only look at the $s$-channel process!).

For energies of the $\phi^a$ quanta greater than the mass of the $W$ and $Z$ bosons (roughly $100$ GeV), this scattering amplitude via $h$-exchange can be shown [you got to believe me here] to be the same as the scattering of {\it longitudinal} $W, Z$-bosons.

Show that
\begin{equation}
\label{g2}
{M}(\phi^1 \phi^1 \rightarrow \phi^3 \phi^3)\big\vert_{ h-exchange} = const. {s^2 \over v^2 (s - m_h^2)}~,
\end{equation}
where $s$ is the appropriate Mandelstam variable (the square of the c.m. energy), $m_h$ is the mass of $h$, $v= |m|/\sqrt\lambda$, and you will determine the constant.
\index{Mandelstam variable}
What you found is that the scattering amplitude (\ref{g2}) grows with the c.m. energy, without bound. It should intuitively clear that this may violate unitarity by leading to probabilities greater then unity at sufficiently high energies.\footnote{Showing this more precisely---and putting bounds on the mass on the Higgs from unitarity---requires study of partial wave decomposition (which is also widely used in quantum mechanics; while the idea is the same, it gets technically a bit more messy in QFT), which is left for future studies.}
\makesubproblem{}{qft:problemSet4:3b}
Now, the interesting thing about the Higgs model is that the growth of (\ref{g2}) with center of mass energy is actually cancelled by the same amplitude, but now due to the direct coupling between $\phi^a$ quanta. To find these interactions, go to eq.~(9) of Hw 2 and study the coupling of $\phi^a$: substitute $H(x)$ of eq. (9) into eq. (5) and find the coupling between four $\phi$-quanta that gives the leading  contribution to the ${M}(\phi^1 \phi^1 \rightarrow \phi^3 \phi^3)\big\vert_{local \; \phi-interaction}$ scattering amplitude. Show that it has the form:
\begin{equation}
\label{g3}
const \;  \phi^c \phi^d  \partial_\mu \phi^a \partial^\mu \phi^b \; {\rm Tr}\left(\sigma^c \sigma^d \sigma^a \sigma^b\right)~,
\end{equation}
and determine the constant.
\makesubproblem{}{qft:problemSet4:3c}
Finally, use (\ref{g3}) to calculate ${M}(\phi^1 \phi^1 \rightarrow \phi^3 \phi^3)\big\vert_{local \; \phi-interaction}$ and show that, when added to ${M}(\phi^1 \phi^1 \rightarrow \phi^3 \phi^3)\big\vert_{ h-exchange}$, the various constants combine such that the  amplitude ${M}(\phi^1 \phi^1 \rightarrow \phi^3 \phi^3)\big\vert_{ h-exchange +local \; \phi-interaction }$ does not grow with the center of mass energy. Hence, in the Higgs model of Homework 2  unitary (as expected) rules.

{\flushleft{T}}he discovery of the Higgs, which is expected from such theoretical arguments, is a strong evidence in favor of the recent statement:

\smallskip

{\tt "Quantum field theory is how the world works." -Ed Witten} (NYT, August 12 2013)
} % makeproblem

\makeanswer{qft:problemSet4:3}{
\withproblemsetsParagraph{
\makeSubAnswer{}{qft:problemSet4:3a}
TODO.
\makeSubAnswer{}{qft:problemSet4:3b}
TODO.
\makeSubAnswer{}{qft:problemSet4:3c}
TODO.
}
}

      %
% Copyright � 2018 Peeter Joot.  All Rights Reserved.
% Licenced as described in the file LICENSE under the root directory of this GIT repository.
%
\makeproblem{Lorentz transforms of spinors---some useful identities}{qft:problemSet4:4}{

{\flushleft{C}}onsider the matrix $$\Lambda_{1\over 2} = e^{- {i\over 2} \omega_{\mu\nu} S^{\mu\nu}}~.$$
Here, $S^{\mu\nu} = {i\over 4} [\gamma^\mu, \gamma^\nu ]$ is as defined in class, in terms of the four $\gamma$-matrices (notice that, when using the representation of the $\gamma$ matrices in terms of Pauli matrices, the matrix $\Lambda_{1\over 2}$ looks like two sets of $M$ (and $M^*$) matrices discussed in class, now combined into one four-by-four object).

\makesubproblem{}{qft:problemSet4:4a}
Show that $\Lambda_{1\over 2}^{-1} \gamma^\mu \Lambda_{1\over 2} = \Lambda^\mu_{\; \nu} \gamma^\nu$, where $ \Lambda^\mu_{\; \nu}$ is the usual Lorentz transformation acting on vectors. (Feel free to show this for the infinitesimal form of the transformations, but then argue that the finite form holds as well.)
\makesubproblem{}{qft:problemSet4:4b}
Show that $\Lambda_{1\over 2}^{\dagger} \gamma^0 \Lambda_{1\over 2} =  \gamma^0$.
\makesubproblem{}{qft:problemSet4:4c}
Consider the fermion bilinear $\bar\psi \gamma^\mu \gamma^\nu \psi ={1\over 2} \bar\psi \{ \gamma^\mu ,\gamma^\nu \} \psi + {1\over 2} \bar\psi [\gamma^\mu, \gamma^\nu] \psi$, where $\{A,B\} = AB + BA$ is the anticommutator. Show that the two terms on the right transform as a scalar and a second-rank tensor, respectively, under Lorentz transformations.
} % makeproblem

\makeanswer{qft:problemSet4:4}{
\makeSubAnswer{}{qft:problemSet4:4a}
TODO.
\makeSubAnswer{}{qft:problemSet4:4b}
TODO.
\makeSubAnswer{}{qft:problemSet4:4c}
TODO.
}

%
}%

   \chapter{Independent study problems.}

      %
% Copyright � 2016 Peeter Joot.  All Rights Reserved.
% Licenced as described in the file LICENSE under the root directory of this GIT repository.
%
%{
%\input{../latex/blogpost.tex}
%\renewcommand{\basename}{scalarFieldCreationOpCommutator}
%\renewcommand{\dirname}{notes/phy2403/}
%%\newcommand{\dateintitle}{}
%%\newcommand{\keywords}{}
%
%\input{../latex/peeter_prologue_print2.tex}
%
%\usepackage{peeters_layout_exercise}
%\usepackage{peeters_braket}
%\usepackage{peeters_figures}
%
%\beginArtNoToc
%
%\generatetitle{Scalar field creation operator commutator}
%\chapter{Scalar field creation operator commutator}
\label{chap:scalarFieldCreationOpCommutator}

\makeproblem{Scalar field creation operator commutator.}{problem:scalarFieldCreationOpCommutator:1}{
In \citep{qftLectureNotes} it is stated that the creation operators of eq. 2.78

\begin{dmath}\label{eqn:scalarFieldCreationOpCommutator:20}
\alpha_k = \inv{2} \int \frac{d^3k}{(2\pi)^3} \lr{
\phi(x,0) + \frac{i}{\omega_k} \partial_0 \phi(x,0)
}
e^{-i \Bk \cdot \Bx }
\end{dmath}

associated with field operator \( \phi \) commute.  Verify that.

} % problem

\makeanswer{problem:scalarFieldCreationOpCommutator:1}{

\begin{dmath}\label{eqn:scalarFieldCreationOpCommutator:40}
\antisymmetric{\alpha_k}{\alpha_m}
=
\inv{4}
\frac{1}{(2\pi)^6}
\int d^3 x d^3 y
e^{-i \Bk \cdot \Bx }
e^{-i \Bm \cdot \By }
\antisymmetric
{
\phi(x,0) + \frac{i}{\omega_k} \partial_0 \phi(x,0)
}
{
\phi(y,0) + \frac{i}{\omega_m} \partial_0 \phi(y,0)
}
=
\frac{i}{4}
\frac{1}{(2\pi)^6}
\int d^3 x d^3 y
e^{-i \Bk \cdot \Bx }
e^{-i \Bm \cdot \By }
\lr{
\antisymmetric{\phi(x,0)}{\inv{\omega_m} \partial_0 \phi(y,0)}
+
\antisymmetric{\inv{\omega_k} \partial_0 \phi(x,0)}{\phi(y,0)}
}
=
\frac{i}{4}
\frac{1}{(2\pi)^6}
\int d^3 x d^3 y
e^{-i \Bk \cdot \Bx }
e^{-i \Bm \cdot \By }
\lr{
\frac{i}{\omega_m} \deltathree(\Bx - \By)
-
\frac{i}{\omega_k} \deltathree(\Bx - \By)
}
=
-\frac{1}{4}
\frac{1}{(2\pi)^6}
\int d^3 x
e^{ -i (\Bk + \Bm) \cdot \Bx }
\lr{
\frac{1}{\omega_m}
-
\frac{1}{\omega_k}
}
=
-\frac{1}{4}
\frac{1}{(2\pi)^3}
\lr{
\frac{1}{\omega_m}
-
\frac{1}{\omega_k}
}
\deltathree(\Bk + \Bm)
=
-\frac{1}{4}
\frac{1}{(2\pi)^3}
\lr{
\frac{1}{\omega_{\Abs{-\Bk}}}
-
\frac{1}{\omega_{\Abs{\Bk}}}
}
\deltathree(\Bk + \Bm)
=
0.
\end{dmath}
} % answer

%}
%\EndArticle

      %
% Copyright � 2016 Peeter Joot.  All Rights Reserved.
% Licenced as described in the file LICENSE under the root directory of this GIT repository.
%
%{
%\input{../latex/blogpost.tex}
%\renewcommand{\basename}{scalarFieldHamiltonian}
%\renewcommand{\dirname}{notes/phy2403/}
%%\newcommand{\dateintitle}{}
%%\newcommand{\keywords}{}
%
%\input{../latex/peeter_prologue_print2.tex}
%
%\usepackage{peeters_layout_exercise}
%\usepackage{peeters_braket}
%\usepackage{peeters_figures}
%
%\beginArtNoToc
%
%\generatetitle{Hamiltonian for a scalar field}
%\chapter{Hamiltonian for a scalar field}

\makeproblem{}{problem:scalarFieldHamiltonian:1}{
In \citep{qftLectureNotes} it is left as an exercise to expand the scalar field Hamiltonian in terms of the raising and lowering operators.  Let's do that.
} % problem

\makeanswer{problem:scalarFieldHamiltonian:1}{
The field operator expanded in terms of the raising and lowering operators is

\begin{dmath}\label{eqn:scalarFieldHamiltonian:20}
\phi(x)
=
\int \frac{ d^3 k}{ (2 \pi)^{3/2} \sqrt{ 2 \omega_k } } \lr{
a_\Bk e^{-i k \cdot x}
+ a_\Bk^\dagger e^{i k \cdot x}
}
=
\int \frac{ d^3 k}{ (2 \pi)^{3/2} \sqrt{ 2 \omega_k } } \lr{
a_\Bk e^{-i \omega_k t + i \Bk \cdot \Bx}
+a_\Bk^\dagger e^{i \omega_k t - i \Bk \cdot \Bx}
}
=
\int \frac{ d^3 k}{ (2 \pi)^{3/2} \sqrt{ 2 \omega_k } } \lr{
a_\Bk e^{-i \omega_k t + i \Bk \cdot \Bx}
+a_{-\Bk}^\dagger e^{i \omega_k t + i \Bk \cdot \Bx}
}
=
\int \frac{ d^3 k}{ (2 \pi)^{3/2} \sqrt{ 2 \omega_k } } \lr{
a_\Bk e^{-i \omega_k t } +a_{-\Bk}^\dagger e^{i \omega_k t}
}
e^{ i \Bk \cdot \Bx}
.
\end{dmath}

Note that \( x \) and \( k \) here are both four-vectors, so this field is dependent on a spacetime point, but the integration is over a spatial volume.  This is discussed in the class notes but also justified nicely in \citep{peskin1995introduction} using the structure of the raising and lower operators.  The trick of reversing the sign above is also from that text.

The Hamiltonian in terms of the fields was
\begin{dmath}\label{eqn:scalarFieldHamiltonian:40}
H = \inv{2} \int d^3 x \lr{ \Pi^2 + \lr{ \spacegrad \phi }^2 + \mu^2 \phi^2 }.
\end{dmath}

The field derivatives are

\begin{dmath}\label{eqn:scalarFieldHamiltonian:60}
\Pi
= \partial_0 \phi
\int \frac{ d^3 k}{ (2 \pi)^{3/2} \sqrt{ 2 \omega_k } } \lr{
a_\Bk e^{-i \omega_k t } +a_{-\Bk}^\dagger e^{i \omega_k t}
}
e^{ i \Bk \cdot \Bx}
=
i
\int \frac{ d^3 k}{ (2 \pi)^{3/2}} \sqrt{ \frac{\omega_k}{2} } \lr{
-a_\Bk e^{-i \omega_k t } +a_{-\Bk}^\dagger e^{i \omega_k t}
}
e^{ i \Bk \cdot \Bx}
,
\end{dmath}

and

\begin{dmath}\label{eqn:scalarFieldHamiltonian:80}
\partial_n \phi
= \partial_n
\int \frac{ d^3 k}{ (2 \pi)^{3/2} \sqrt{ 2 \omega_k } } \lr{
a_\Bk e^{-i \omega_k t } +a_{-\Bk}^\dagger e^{i \omega_k t}
}
e^{ i \Bk \cdot \Bx}
=
i
\int \frac{ d^3 k}{ (2 \pi)^{3/2}} \frac{k^n}{ \sqrt{ 2 \omega_k } } \lr{
a_\Bk e^{-i \omega_k t } +a_{-\Bk}^\dagger e^{i \omega_k t}
}
e^{ i \Bk \cdot \Bx}
.
\end{dmath}

Introducing a second set of momentum variables \( \Bj \), the momentum portion of the Hamiltonian is

\begin{dmath}\label{eqn:scalarFieldHamiltonian:100}
\inv{2} \int d^3 x \Pi^2
=
-\inv{2}
\inv{(2 \pi)^{3}}
\int d^3 x
\int
d^3 j
d^3 k
\frac{
\omega_j
\omega_k
}{2}
\lr{
-a_\Bj e^{-i \omega_j t }
+a_{-\Bj}^\dagger e^{i \omega_j t }
}
 \lr{
-a_\Bk e^{-i \omega_k t }
+a_{-\Bk}^\dagger e^{i \omega_k t }
}
e^{ i \Bk \cdot \Bx}
e^{ i \Bj \cdot \Bx}
=
-\inv{2}
\int
d^3 j
d^3 k
\frac{
\omega_j
\omega_k
}{2}
\lr{
-a_\Bj e^{-i \omega_j t }
+a_{-\Bj}^\dagger e^{i \omega_j t }
}
 \lr{
-a_\Bk e^{-i \omega_k t }
+a_{-\Bk}^\dagger e^{i \omega_k t }
}
\deltathree( \Bk + \Bj )
=
-\inv{2}
\int
d^3 k
\frac{
\omega_k^2
}{2}
\lr{
-a_{-\Bk} e^{-i \omega_k t }
+a_{\Bk}^\dagger e^{i \omega_k t }
}
 \lr{
-a_\Bk e^{-i \omega_k t }
+a_{-\Bk}^\dagger e^{i \omega_k t }
}
%=
%-\inv{4}
%\inv{(2 \pi)^{3}}
%\int d^3 x
%\int
%d^3 j
%d^3 k
%\sqrt{
%\omega_j
%\omega_k}
%\biglr{
%  a_\Bj^\dagger a_\Bk^\dagger e^{i (\omega_k + \omega_j) t - i (\Bk + \Bj) \cdot \Bx}
%+ a_\Bj a_\Bk e^{-i (\omega_j + \omega_k) t + i (\Bj + \Bk) \cdot \Bx}
% - a_\Bj^\dagger a_\Bk e^{-i (\omega_k -\omega_j) t - i (\Bj - \Bk) \cdot \Bx}
%- a_\Bj a_\Bk^\dagger e^{-i (\omega_j - \omega_k) t - i (\Bk - \Bj) \cdot \Bx}
%}
%=
%-\inv{4}
%\int
%d^3 j
%d^3 k
%\sqrt{
%\omega_j
%\omega_k}
%\biglr{
%  a_\Bj^\dagger a_\Bk^\dagger e^{i (\omega_k + \omega_j) t } \deltathree(\Bk + \Bj)
%+ a_\Bj a_\Bk e^{-i (\omega_j + \omega_k) t } \deltathree(-\Bj - \Bk)
% - a_\Bj^\dagger a_\Bk e^{-i (\omega_k -\omega_j) t } \deltathree(\Bj - \Bk)
%- a_\Bj a_\Bk^\dagger e^{-i (\omega_j - \omega_k) t } \deltathree(\Bk - \Bj)
%}
=
-\inv{4}
\int
d^3 k
\omega_k
\lr{
  a_{-\Bk}^\dagger a_\Bk^\dagger e^{2 i \omega_k t }
+ a_{-\Bk} a_\Bk e^{- 2 i \omega_k t }
- a_\Bk^\dagger a_\Bk
- a_{-\Bk} a_{-\Bk}^\dagger
}.
\end{dmath}

For the gradient portion of the Hamiltonian we have

\begin{dmath}\label{eqn:scalarFieldHamiltonian:120}
\inv{2} \int d^3 x \lr{ \spacegrad \phi }^2
=
-\inv{2}
\inv{(2 \pi)^{3}}
\int d^3 x
\int
d^3 j
d^3 k
\inv{ \sqrt{ 4 \omega_j \omega_k } }
\lr{ \sum_{n=1}^3 j^n k^n }
\lr{
 a_\Bj e^{-i \omega_j t }
+a_{-\Bj}^\dagger e^{i \omega_j t }
}
 \lr{
 a_\Bk e^{-i \omega_k t }
+a_{-\Bk}^\dagger e^{i \omega_k t }
}
e^{ i \Bj \cdot \Bx}
e^{ i \Bk \cdot \Bx}
=
-\inv{2}
\int
d^3 j
d^3 k
\inv{ \sqrt{ 4 \omega_j \omega_k } }
\Bj \cdot \Bk
\lr{
 a_\Bj e^{-i \omega_j t }
+a_{-\Bj}^\dagger e^{i \omega_j t }
}
 \lr{
 a_\Bk e^{-i \omega_k t }
+a_{-\Bk}^\dagger e^{i \omega_k t }
}
\deltathree(\Bj + \Bk)
=
\inv{2}
\int
d^3 k
\inv{ \sqrt{ 4 \omega_k \omega_k } }
\Bk^2
\lr{
 a_{-\Bk} e^{-i \omega_k t }
+a_{\Bk}^\dagger e^{i \omega_k t }
}
 \lr{
 a_\Bk e^{-i \omega_k t }
+a_{-\Bk}^\dagger e^{i \omega_k t }
}
=
\inv{4}
\int
d^3 k
\frac{\Bk^2}{ \omega_k }
\biglr{
   a_{-\Bk}^\dagger a_\Bk^\dagger e^{2 i \omega_k t }
 + a_{-\Bk} a_\Bk e^{- 2 i \omega_k t }
 + a_\Bk^\dagger a_\Bk
 + a_{-\Bk} a_{-\Bk}^\dagger
}.
\end{dmath}

Finally, for the mass term, we have

\begin{dmath}\label{eqn:scalarFieldHamiltonian:140}
\inv{2} \int d^3 x \mu^2 \phi^2
=
\frac{\mu^2}{2}
\inv{(2 \pi)^{3}}
\int d^3 x
\int
d^3 j
d^3 k
\inv{ \sqrt{ 4 \omega_j \omega_k } }
\biglr{
 a_\Bj e^{-i \omega_j t }
+a_{-\Bj}^\dagger e^{i \omega_j t }
}
 \lr{
 a_\Bk e^{-i \omega_k t }
+a_{-\Bk}^\dagger e^{i \omega_k t }
}
e^{ i \Bj \cdot \Bx}
e^{ i \Bk \cdot \Bx}
=
\frac{\mu^2}{2}
\int
d^3 j
d^3 k
\inv{ \sqrt{ 4 \omega_j \omega_k } }
\biglr{
 a_\Bj e^{-i \omega_j t }
+a_{-\Bj}^\dagger e^{i \omega_j t }
}
 \lr{
 a_\Bk e^{-i \omega_k t }
+a_{-\Bk}^\dagger e^{i \omega_k t }
}
\deltathree(\Bj + \Bk)
=
\frac{\mu^2}{2}
\int
d^3 k
\inv{ 2 \omega_k }
\biglr{
 a_{-\Bk} e^{-i \omega_k t }
+a_{\Bk}^\dagger e^{i \omega_k t }
}
 \lr{
 a_\Bk e^{-i \omega_k t }
+a_{-\Bk}^\dagger e^{i \omega_k t }
}
%=
%\frac{\mu^2}{2}
%\inv{(2 \pi)^{3}}
%\int d^3 x
%\int
%d^3 j
%d^3 k
%\inv{ \sqrt{ 4 \omega_j \omega_k } }
%\biglr{
% a_\Bj a_\Bk e^{-i (\omega_k + \omega_j) t + i (\Bk + \Bj) \cdot \Bx}
%+a_\Bj^\dagger a_\Bk^\dagger e^{i (\omega_j + \omega_k) t - i (\Bk + \Bj) \cdot \Bx}
%+a_\Bj a_\Bk^\dagger e^{i (\omega_k - \omega_j) t - i (\Bk - \Bj) \cdot \Bx}
%+a_\Bj^\dagger a_\Bk e^{-i (\omega_k + \omega_j) t - i (\Bj - \Bk) \cdot \Bx}
%}
%=
%\frac{\mu^2}{2}
%\int
%d^3 j
%d^3 k
%\inv{ \sqrt{ 4 \omega_j \omega_k } }
%\biglr{
% a_\Bj a_\Bk e^{-i (\omega_k + \omega_j) t } \deltathree(- \Bk - \Bj)
%+a_\Bj^\dagger a_\Bk^\dagger e^{i (\omega_j + \omega_k) t } \deltathree( \Bk + \Bj)
%+a_\Bj a_\Bk^\dagger e^{i (\omega_k - \omega_j) t } \deltathree (\Bk - \Bj)
%+a_\Bj^\dagger a_\Bk e^{-i (\omega_k + \omega_j) t } \deltathree (\Bj - \Bk)
%}
=
\frac{\mu^2}{4}
\int
d^3 k
\inv{ \omega_k }
\lr{
 a_{-\Bk} a_\Bk e^{- 2 i \omega_k t }
+a_{-\Bk}^\dagger a_\Bk^\dagger e^{2 i \omega_k t }
+a_{-\Bk} a_{-\Bk}^\dagger
+a_\Bk^\dagger a_\Bk
}.
\end{dmath}

Now all the pieces can be put back together again

\begin{dmath}\label{eqn:scalarFieldHamiltonian:160}
\begin{aligned}
H
&=
\inv{4}
\int d^3 k
\inv{\omega_k}
\biglr{ \\
&\qquad -\omega_k^2
\lr{
  a_{-\Bk}^\dagger a_\Bk^\dagger e^{2 i \omega_k t }
+ a_{-\Bk} a_\Bk e^{- 2 i \omega_k t }
- a_\Bk^\dagger a_\Bk
- a_{-\Bk} a_{-\Bk}^\dagger
} \\
&\qquad +
\Bk^2
\lr{
  a_{-\Bk}^\dagger a_\Bk^\dagger e^{2 i \omega_k t }
+ a_{-\Bk} a_\Bk e^{- 2 i \omega_k t }
+ a_\Bk^\dagger a_\Bk
+ a_{-\Bk} a_{-\Bk}^\dagger
} \\
&\qquad +
\mu^2
\lr{
 a_{-\Bk} a_\Bk e^{- 2 i \omega_k t }
+a_{-\Bk}^\dagger a_\Bk^\dagger e^{2 i \omega_k t }
+a_{-\Bk} a_{-\Bk}^\dagger
+a_\Bk^\dagger a_\Bk
}
} \\
&=
\inv{4}
\int d^3 k
\inv{\omega_k}
\biglr{
a_{-\Bk}^\dagger a_\Bk^\dagger e^{2 i \omega_k t }
\lr{
-\omega_k^2
+ \Bk^2
+
\mu^2
} \\
&\qquad + a_{-\Bk} a_\Bk e^{- 2 i \omega_k t }
\lr{
-\omega_k^2
+ \Bk^2
+
\mu^2
} \\
&\qquad + a_\Bk a_\Bk^\dagger
\lr{
 \omega_k^2
+ \Bk^2
+
\mu^2
} \\
&\qquad + a_\Bk^\dagger a_\Bk
\lr{
 \omega_k^2
+ \Bk^2
+
\mu^2
}
}.
\end{aligned}
\end{dmath}

With \( \omega_k^2 = \Bk^2 + \mu^2 \), the time dependent terms are killed leaving
\begin{dmath}\label{eqn:scalarFieldHamiltonian:180}
H
=
\inv{2}
\int d^3 k
\omega_k
\lr{
  a_\Bk a_\Bk^\dagger
+ a_\Bk^\dagger a_\Bk
}.
\end{dmath}
} % answer

%}
%\EndArticle

