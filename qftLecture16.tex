%
% Copyright � 2018 Peeter Joot.  All Rights Reserved.
% Licenced as described in the file LICENSE under the root directory of this GIT repository.
%
%{
%%\input{../latex/blogpost.tex}
%%\renewcommand{\basename}{qftLecture16}
%%\renewcommand{\dirname}{notes/phy2403/}
%%\newcommand{\keywords}{PHY2403H}
%%\input{../latex/peeter_prologue_print2.tex}
%%
%%%\usepackage{phy2403}
%%\usepackage{peeters_braket}
%%\usepackage{peeters_layout_exercise}
%%\usepackage{peeters_figures}
%%\usepackage{mathtools}
%%\usepackage{siunitx}
%%\usepackage{macros_cal} % LL
%%\usepackage{simplewick}
%%\usepackage{verbatim}
%%
%%\newcommand{\ultensor}[3]{{{#1}^{#2}}_{#3}}
%%\newcommand{\deltathree}[0]{\delta^{(3)}}
%%\newcommand{\deltafour}[0]{\delta^{(4)}}
%%
%%\beginArtNoToc
%%\generatetitle{PHY2403H Quantum Field Theory.  Lecture 16: Differential cross section, scattering, pair production, transition amplitude, decay rate, S-matrix, connected and amputated diagrams, vacuum fluctuation, symmetry coefficient.  Taught by Prof.\ Erich Poppitz}
%\chapter{Differential cross section, scattering, pair production, transition amplitude, decay rate, S-matrix, connected and amputated diagrams, vacuum fluctuation, symmetry coefficient}
\index{differential cross section}
\index{scattering}
\index{pair production}
\index{transition amplitude}
\index{S-matrix}
\index{connected diagrams}
\index{vacuum fluctuation}
\index{symmetry coefficient}
\label{chap:qftLecture16}
%%
%%%%Peeter's lecture notes from class.  These may be incoherent and rough.
%%%%
%%%%These are notes for the UofT course PHY2403H, Quantum Field Theory, taught by Prof. Erich Poppitz, covering \textchapref{{1}} \citep{peskin1995introduction} content.
%%
%%\paragraph{DISCLAIMER: Very rough notes from class, with some additional side notes.}
%%
%%These are notes for the UofT course PHY2403H, Quantum Field Theory, taught by Prof. Erich Poppitz, fall 2018.
%%%, covering \textchapref{{1}} \citep{peskin1995introduction} content.
%%
%%\section{Review}
%%
%%We finished by defining the differential cross section \cref{dfn:qftLecture15b:520}
%%
%%\makedefinition{Differential cross section.}{dfn:qftLecture16:20}{
%%\begin{equation*}
%%\frac{d^3 \sigma}{dp_x dp_y dp_z} = \frac{
%%\text{number of scattering events with \( \Bp_{\txtf} \) between \( (\Bp_\txtf, \Bp_\txtf + d \Bp_\txtf )\)}
%%}
%%{
%%\text{flux of incoming particles}
%%}.
%%\end{equation*}
%%} % definition
%%
%\section{Definitions and motivation (cont.)}
In QFT we typically study \( 2 \rightarrow n \) inelastic scattering.  Most commonly the nature of the final state particles are different from the nature of the incoming state.

For example, we can collide an electron and anti-electron, and can get muon and anti-muon particles as sketched in
\cref{fig:qftLecture16:qftLecture16Fig1},
or pions as sketched in
\cref{fig:qftLecture16:qftLecture16Fig2},
or even both as sketched in
\cref{fig:qftLecture16:qftLecture16Fig3}.
\imageFigure{../figures/phy2403-quantum-field-theory/qftLecture16Fig1}{Muon pair production.}{fig:qftLecture16:qftLecture16Fig1}{0.15}
\imageFigure{../figures/phy2403-quantum-field-theory/qftLecture16Fig2}{Pion pair production.}{fig:qftLecture16:qftLecture16Fig2}{0.15}
\imageFigure{../figures/phy2403-quantum-field-theory/qftLecture16Fig3}{Muon and pion pair production.}{fig:qftLecture16:qftLecture16Fig3}{0.15}

In the \( \lambda \phi^4 \) theory we can have scattering events such as \( 2 \rightarrow 2 \) and \( 2 \rightarrow 2 n \) production as sketched in \cref{fig:qftLecture16:qftLecture16Fig4}.
\imageTwoFigures{../figures/phy2403-quantum-field-theory/qftLecture16Fig4a}{../figures/phy2403-quantum-field-theory/qftLecture16Fig4b}{lambda fourth scattering events.}{fig:qftLecture16:qftLecture16Fig4}{scale=0.2}
%\cref{fig:qftLecture16:qftLecture16Fig4a}.
%\imageFigure{../figures/phy2403-quantum-field-theory/qftLecture16Fig4a}{CAPTION: qftLecture16Fig4a}{fig:qftLecture16:qftLecture16Fig4a}{0.3}
%\cref{fig:qftLecture16:qftLecture16Fig4b}.
%\imageFigure{../figures/phy2403-quantum-field-theory/qftLecture16Fig4b}{CAPTION: qftLecture16Fig4b}{fig:qftLecture16:qftLecture16Fig4b}{0.3}

How to calculate in QFT.  Initial state of 2 particles \( A, B \) with initial state
\begin{equation}\label{eqn:qftLecture16:40}
\ket{\Bk_A, \Bk_B }_{\text{in}, T \rightarrow -\infty}
\end{equation}
and final n-particle state
\begin{equation}\label{eqn:qftLecture16:60}
\ket{\Bp_1, \Bp_2, \cdots, \Bp_n }_{\text{out}, T \rightarrow +\infty}
\end{equation}
The QM transition amplitude from the initial to the final state is
\begin{equation}\label{eqn:qftLecture16:80}
\prescript{}{\text{out}}{\bra{\Bp_1, \Bp_2, \cdots, \Bp_n }}
\ket{\Bk_A, \Bk_B }_{\text{in}}
=
\bra{\Bp_1, \Bp_2, \cdots, \Bp_n } e^{-2 i H T}
\ket{\Bk_A, \Bk_B }.
\end{equation}
This is the amplitude for \( A B \rightarrow 1 \cdots n \).
Ultimately, we want the scattering x-section.

We will also be interested in decay rates, as there are unstable particles in QFT that can decay.  This doesn't happen in \( \lambda \phi^4 \) theory.
In a theory with 2 scalar fields \( \Phi, \varphi \) with \( m_\Phi > 2 m_\varphi \).  A possible interaction for such a theory is
\begin{equation}\label{eqn:qftLecture16:100}
H_{\text{int}} = \mu \Phi \varphi^2,
\end{equation}
which would permit \( \Phi \rightarrow \varphi \varphi \) decays.
Hw4 has a coupling like \( (h/V) \partial_\mu \phi^a \partial^\mu \phi^a \) for which a \( h \rightarrow \phi^a \phi^a \) decay is possible.

\makedefinition{Decay rate.}{dfn:qftLecture16:120}{
\index{decay rate}
The decay rate is defined as
\begin{equation*}
\Gamma =
\frac{
\text{
Number of decays \( \Phi \rightarrow \varphi \varphi \) in unit time
}
}
{
\text{
Number of \( \Phi \) particles present
}
}
\end{equation*}
} % definition

What is the amplitude for such a decay transition?
\begin{equation}\label{eqn:qftLecture16:140}
\bra{\Bk_\phi}_{\text{in}, T \rightarrow -\infty} \rightarrow
\bra{\Bk_1, \Bk_2}_{\text{out}, T \rightarrow +\infty}.
\end{equation}
The amplitude for \( \Bk_\phi \rightarrow \Bk_1, \Bk_2 \).
\begin{equation}\label{eqn:qftLecture16:160}
\bra{\Bk_1, \Bk_2} e^{-i 2 H T } \ket{\Bk_\phi}
=
\prescript{}{\text{out}}{\braket{\Bk_1, \Bk_2}{\Bk_\phi}}
\end{equation}

\paragraph{mysterious seeming statement something like}: ``The decays are essentially due to interactions with vacuum fluctuations.''

\section{Calculating interactions.}

We write
\begin{equation}\label{eqn:qftLecture16:180}
\begin{aligned}
\prescript{}{\text{out}}{\braket{ \Bp_1, \cdots \Bp_n }{ \Bk_A, \Bk_B }}_{\text{in}}
&=
\lim_{T \rightarrow \infty}
\bra{ \Bp_1, \cdots \Bp_n } e^{-i 2 H T } \ket{ \Bk_A, \Bk_B } \\
&=
\bra{ \Bp_1, \cdots \Bp_n } \hatS \ket{ \Bk_A, \Bk_B } \\
&=
\bra{ \Bp_1, \cdots \Bp_n } \BOne + i \hatT \ket{ \Bk_A, \Bk_B },
\end{aligned}
\end{equation}
where \( \hatS \) is called the S-matrix or scattering matrix, which is decomposed into a unit portion \( \BOne \) which is a convenient way to exclude events with no scattering.  \( \BOne \) contributes for \( n = 2 \) only, but is an \( n \) scattering amplitude.
We are really interested in the \( i \hatT \) portion of this amplitude
\begin{equation}\label{eqn:qftLecture16:200}
\begin{aligned}
&\bra{ \Bp_1, \cdots \Bp_n } i \hatT \ket{ \Bk_A, \Bk_B } \\
&\qquad =
(2 \pi)^4 \deltafour( \Bk_A + \Bk_B - \sum_{i = 1}^n \Bp_i )
\times
i M( \Bk_A + \Bk_B \rightarrow \Bp_1 \cdots \Bp_n ).
\end{aligned}
\end{equation}
This amounts to a definition of \( M \).
Recall that we found
\begin{equation}\label{eqn:qftLecture16:220}
U(T, -T)
= T \lr{ e^{-i \int_{-T}^T H_I(t') dt'} }
=
e^{i H_0(T - t_0)}
e^{-i 2 H T}
e^{-i H_0(-T - t_0)}.
\end{equation}
We want to replace the \( e^{-i 2 H T} \) in the matrix element above with \( U \).

In perturbation theory, we assume (conjecture) that
\begin{equation}\label{eqn:qftLecture16:240}
\ket{ \Bk_A, \Bk_B }
\sim
\ket{ \Bk_A, \Bk_B }_\txto
\sim
\text{const}\, a^\dagger_{\Bk_A} a^\dagger_{\Bk_B} \ket{0}
\end{equation}

Because we'll be squaring the amplitudes, we can assume that the \( e^{i H_0(T-t_0)} \) will result in just phase factors that won't survive, so in \cref{eqn:qftLecture16:180} we can insert \( U \)
\begin{equation}\label{eqn:qftLecture16:280}
\prescript{}{\text{out}}{\braket{ \Bp_1, \cdots \Bp_n }{ \Bk_A, \Bk_B }}_{\text{in}}
=
\lim_{T \rightarrow \infty}
\bra{ \Bp_1, \cdots \Bp_n } U(T, -T) \ket{ \Bk_A, \Bk_B }
\end{equation}
\begin{equation}\label{eqn:qftLecture16:300}
\bra{ \Bp_1, \cdots \Bp_n } i\hatT \ket{ \Bk_A, \Bk_B }
=
\lim_{T \rightarrow \infty(1 - i \epsilon) }
\prescript{}{0}{
\bra{ \Bp_1, \cdots \Bp_n }
T( e^{-i \int_{-T}^T H_i(t') dt' } )
 \ket{ \Bk_A, \Bk_B }
}_0
\end{equation}

We will see that evaluating this beastie amounts to summing all the
connected and amputated diagrams, a subset of all the possible graphs.  Why this is true is not covered in this course (i.e. see QFT II and/or \S 7.2 \citep{peskin1995introduction}).  Using the \( \phi^4 \)th theory, \citep{tobiasQFTL10FeynmanDiagrams} provides a very nice example that shows how the first order disconnected diagrams happen to cancel out perfectly (even though they both represent infinities!).
\paragraph{What is ``connected and amputated''?}
Explaining by example.  \( n = 2, \lambda \phi^4/4! \).
\begin{equation}\label{eqn:qftLecture16:320}
\bra{0}
a_{\Bp_1}
a_{\Bp_2}
\lr{
\cancel{1}
- \frac{i \lambda}{4!} \int d^4 x \phi_I^4(x)
+ \inv{2} \lr{ \frac{i \lambda}{4!}}^2 \int d^4 x d^4 y \phi_I^4(x) \phi_I^4(y)
+ \cdots
}
a_{\Bk_A}^\dagger
a_{\Bk_B}^\dagger
\ket{0}
\end{equation}
Here time ordering operations are implied, but not written explicitly.
Also, the ``amputated'' indicates that we are going to be dropping the \( 1 \) portion of the exponential expansion (as we've also dropped that in \cref{eqn:qftLecture16:300}).
We will also be using a relativistic normalization so that the \(
a_{\Bk_A}^\dagger
a_{\Bk_B}^\dagger \) terms include
\( \sqrt{
2 \omega_{\Bk_A}
2 \omega_{\Bk_B} } \)
contributions and the \(
a_{\Bp_1}
a_{\Bp_2}
\) include
\( \sqrt{
2 \omega_{\Bp_1}
2 \omega_{\Bp_2} } \) contributions.
\begin{equation}\label{eqn:qftLecture16:340}
T
\contraction{}{\phi}{{}_I(x_1)}{\phi}
\phi_I(x_1)\phi_I(x_2)
= D_F(x_1 - x_2)
\end{equation}

When we look at
\begin{equation}\label{eqn:qftLecture16:360}
\contraction{}{\phi}{{}_I(x_1)}{a}
\phi_I(x_1) a^\dagger_{\Bk}
\sqrt{ 2 \omega_\Bk}
=
\int \frac{d^3 p}{(2 \pi)^3} \frac{e^{-i p \cdot x}}{\sqrt{2 \omega_\Bp}}
\contraction{}{a}{{}_\Bp}{a}
a_\Bp a^\dagger_\Bk
\sqrt{ 2 \omega_\Bk}
=
\int \frac{d^3 p}{(2 \pi)^3} \frac{e^{-i p \cdot x}}{\sqrt{2 \omega_\Bp}}
\deltathree(\Bp - \Bk)
\sqrt{ 2 \omega_\Bk}
= e^{-i k \cdot x}.
\end{equation}
Similarly
\begin{equation}\label{eqn:qftLecture16:440}
\contraction{}{a}{{}_\Bp}{\phi}
a_\Bp \phi_I(x_1)
\sqrt{ 2 \omega_\Bp}
=
\int \frac{d^3 k}{(2 \pi)^3} \frac{e^{i k \cdot x}}{\sqrt{2 \omega_\Bk}}
\contraction{}{a}{{}_\Bp}{a}
a_\Bp a^\dagger_\Bk
\sqrt{ 2 \omega_\Bk}
=
\int \frac{d^3 k}{(2 \pi)^3} \frac{e^{i k \cdot x}}{\sqrt{2 \omega_\Bk}}
\deltathree(\Bp - \Bk)
\sqrt{ 2 \omega_\Bk}
= e^{+i p \cdot x}.
\end{equation}

Summarizing
\boxedEquation{eqn:qftLecture16:380}{
%\begin{equation}\label{eqn:qftLecture16:380}
\begin{aligned}
\contraction{}{\phi}{{}_I(x)}{a}
\phi_I(x) a^\dagger_{\Bp}
&= e^{-i p \cdot x} \\
\contraction{}{a}{{}_\Bp}{\phi}
a_\Bp \phi_I(x)
&= e^{i p \cdot x}.
\end{aligned}
%\end{equation}
}

\section{Example diagrams.}
\index{Feynman diagrams!examples}

We want to examine the relevant diagrams corresponding to a transition amplitudes for the \( \phi^4 \) theory.  Contractions such as
\begin{equation}\label{eqn:qftLecture16:460}
\contraction{\langle }{a}{{}_{\Bp_1} a_{\Bp_2} \vert }{a}
\contraction[2ex]{\langle a_{\Bp_1} }{a}{{}_{\Bp_2} \vert a_{\Bk_A}^\dagger }{a}
\bcontraction[2ex]{\langle}{a}{{}_{\Bp_1} a_{\Bp_2} \vert a_{\Bk_A}^\dagger }{a}
\bcontraction{\langle a_{\Bp_1} }{a}{{}_{\Bp_2} \vert }{a}
{\langle a_{\Bp_1} a_{\Bp_2} \vert a_{\Bk_A}^\dagger a_{\Bk_B}^\dagger \rangle}_0.
%{\left\langle a_{\Bp_1} a_{\Bp_2} \middle \vert a_{\Bk_A}^\dagger a_{\Bk_B}^\dagger \right\rangle}_0
\end{equation}
result in diagrams that are \underline{not connected} as sketched in
\cref{fig:qftLecture16:qftLecture16Fig6}.
\imageFigure{../figures/phy2403-quantum-field-theory/qftLecture16Fig6}{Not connected diagrams.}{fig:qftLecture16:qftLecture16Fig6}{0.2}

There are no other possibilities for the first order (and these ones are not interesting).  For the second order transition amplitudes we want to sum of all the contractions for the expectation
\begin{equation}\label{eqn:qftLecture16:480}
\expectation{ a_{\Bp_1} a_{\Bp_2} \phi_I^4(x) a_{\Bk_A}^\dagger a_{\Bk_B}^\dagger }
=
 - i \frac{\lambda}{4!} \sum \text{all contractions}.
\end{equation}
Our diagrams include \cref{fig:qftLecture16:qftLecture16Fig7}, which are not connected.  The figure eight is a vacuum fluctuation that represents virtual processes.
\imageFigure{../figures/phy2403-quantum-field-theory/qftLecture16Fig7}{Not connected second order interactions, including vacuum fluctuations.}{fig:qftLecture16:qftLecture16Fig7}{0.2}
Another diagram is
\cref{fig:qftLecture16:qftLecture16Fig8}, also not connected.
\imageFigure{../figures/phy2403-quantum-field-theory/qftLecture16Fig8}{Another second order diagram.}{fig:qftLecture16:qftLecture16Fig8}{0.2}

We want diagrams that we will describe as ``connected and amputated''.  We are clearly discarding non-connected diagrams like those above, but will need to demonstrate what is meant by amputated, and will continue to consider examples to make that clear.

Here's another diagram \cref{fig:qftLecture16:qftLecture16Fig9} that is also not connected.
\imageFigure{../figures/phy2403-quantum-field-theory/qftLecture16Fig9}{Another not-connected diagram.}{fig:qftLecture16:qftLecture16Fig9}{0.18}
From the diagrams we can construct the functionals that they represent.  The single line in this one is a \( \deltathree( \Bp_1 - \Bk_A ) \) whereas the balloon with two strings is
\begin{equation}\label{eqn:qftLecture16:500}
\int d^4 x e^{-i k_B \cdot x } D_F( x - x ) e^{ i p_2 \cdot x }.
\end{equation}
There are similar not-connected variations of the possible diagrams that we will also discard.  The connected diagrams all come from contractions such as
\begin{equation}\label{eqn:qftLecture16:520}
\contraction[2ex]{\bra{0} }{a}{{}_{\Bp_1} a_{\Bp_2} }{\phi}
\contraction{\bra{0} a_{\Bp_1} }{a}{{}_{\Bp_2} }{\phi}
\bcontraction{\bra{0} a_{\Bp_1} a_{\Bp_2} }{\phi}{{}_I^4(x) }{a}
\bcontraction[2ex]{\bra{0} a_{\Bp_1} a_{\Bp_2} }{\phi}{{}_I^4(x) a_{\Bk_A}^\dagger }{a}
\bra{0} a_{\Bp_1} a_{\Bp_2} \phi_I^4(x) a_{\Bk_A}^\dagger a_{\Bk_B}^\dagger \ket{0}
\end{equation}
The diagram for this interaction now has a vertex representing the contractions with \( \phi_I^4(x) \) with four edges from that vertex as sketched in
\cref{fig:qftLecture16:qftLecture16Fig10}.
\imageFigure{../figures/phy2403-quantum-field-theory/qftLecture16Fig10}{Not non-connected diagram.}{fig:qftLecture16:qftLecture16Fig10}{0.2}
The algebraic expression for this diagram is
\begin{equation}\label{eqn:qftLecture16:540}
4! \lr{ \frac{-i \lambda}{4!} } \int d^4 x e^{-i(k_A + k_B) \cdot x} e^{i (p_1 + p_2) \cdot x}
=
-i \lambda (2 \pi)^4 \deltafour(p_1 + p_2 - k_A - k_B).
\end{equation}
Such a diagram has the general form
\begin{equation}\label{eqn:qftLecture16:560}
(2 \pi)^4 \deltafour\lr{ \sum \text{in} - \sum \text{final} } \times i M( A,B \rightarrow 1,2 ),
\end{equation}
so
\begin{equation}\label{eqn:qftLecture16:580}
M( A,B \rightarrow 1,2 )
=
-\lambda.
\end{equation}
Here the ``symmetry factor'' \( 4! \) was added in to count all possible ways of constructing such a diagram.

\paragraph{Next order}

How about an amplitude like
\begin{equation}\label{eqn:qftLecture16:600}
\bcontraction{\bra{0} a_{\Bp_1} a_{\Bp_2} \inv{2} \lr{ \frac{-i \lambda}{4!} }^2 \int d^4 x \int d^4 y }{\phi}{{}_I}{}
\bcontraction[2ex]{\bra{0} a_{\Bp_1} a_{\Bp_2} \inv{2} \lr{ \frac{-i \lambda}{4!} }^2 \int d^4 x \int d^4 y }{\phi}{{}_I^4(x) }{\phi}
\contraction[2ex]{\bra{0} }{a}{{}_{\Bp_1} a_{\Bp_2} \inv{2} \lr{ \frac{-i \lambda}{4!} }^2 \int d^4 x \int d^4 y }{\phi}
\bra{0} a_{\Bp_1} a_{\Bp_2} \inv{2} \lr{ \frac{-i \lambda}{4!} }^2 \int d^4 x \int d^4 y \phi_I^4(x) \phi_I^4(y) a_{\Bk_A}^\dagger a_{\Bk_B}^\dagger \ket{0}
\end{equation}
Disconnected diagrams include \cref{fig:qftLecture16:qftLecture16Fig11}.
\imageFigure{../figures/phy2403-quantum-field-theory/qftLecture16Fig11}{Disconnected third order interaction.}{fig:qftLecture16:qftLecture16Fig11}{0.2}
However, we have connected diagrams like \cref{fig:qftLecture16:qftLecture16Fig12}.
\imageFigure{../figures/phy2403-quantum-field-theory/qftLecture16Fig12}{Connected diagram.}{fig:qftLecture16:qftLecture16Fig12}{0.2}
The loop in this diagram represents an interaction with ``vacuum fluctuation''.  Such an interaction is not relevant to scattering, and we may consider just the portion of the diagram that leaves off this vacuum fluctuation.  This is what is meant by amputated.  Amputated diagrams do not include such factors.  Other example interactions that may also be amputated include \cref{fig:qftLecture16:qftLecture16Fig13}.
\imageFigure{../figures/phy2403-quantum-field-theory/qftLecture16Fig13}{Other amputatable diagrams.}{fig:qftLecture16:qftLecture16Fig13}{0.15}

At the next order we can have fun interactions like that of \cref{fig:qftLecture16:qftLecture16Fig14}, which is not amputatable (it connects branches), and must be considered.
\imageFigure{../figures/phy2403-quantum-field-theory/qftLecture16Fig14}{Fun interaction.}{fig:qftLecture16:qftLecture16Fig14}{0.2}

At the \( \lambda^2 \) order, the relevant diagrams are sketched in \cref{fig:qftLecture16:qftLecture16Fig15}
\imageThreeFiguresOneLine{../figures/phy2403-quantum-field-theory/qftLecture16Fig15a}{../figures/phy2403-quantum-field-theory/qftLecture16Fig15b}{../figures/phy2403-quantum-field-theory/qftLecture16Fig15c}{Second order connected amputated diagrams.}{fig:qftLecture16:qftLecture16Fig15}{scale=0.15}
%%\cref{fig:qftLecture16:qftLecture16Fig15a}.
%%\imageFigure{../figures/phy2403-quantum-field-theory/qftLecture16Fig15a}{CAPTION: qftLecture16Fig15a}{fig:qftLecture16:qftLecture16Fig15a}{0.2}
%%\cref{fig:qftLecture16:qftLecture16Fig15b}.
%%\imageFigure{../figures/phy2403-quantum-field-theory/qftLecture16Fig15b}{CAPTION: qftLecture16Fig15b}{fig:qftLecture16:qftLecture16Fig15b}{0.2}
%%\cref{fig:qftLecture16:qftLecture16Fig15c}.
%%\imageFigure{../figures/phy2403-quantum-field-theory/qftLecture16Fig15c}{CAPTION: qftLecture16Fig15c}{fig:qftLecture16:qftLecture16Fig15c}{0.2}
%%
At this order \( \phi^4(x), \phi^4(y) \) each contribute a vertex with 4 edges.
\makedefinition{Amputated}{dfn:qftLecture16:620}{
\index{amputated diagrams}
Omit anything that only effects input or output lines.
} % definition

\section{The recipe.}
The general transition amplitude for a \( 2 \rightarrow n \) event has the form
\begin{equation}\label{eqn:qftLecture16:640}
\braket{ p_1 \cdots p_n }{ k_A k_B }
=
(2 \pi)^4 \deltafour \lr{ \sum k_{\text{in}} - \sum p_{\text{out}} } i M( A,B \rightarrow 1, \cdots n ).
\end{equation}
Our recipe is
\begin{enumerate}
\item \( i M = \sum \text{of all connected amputated diagrams, lines and vertices} \).
\item to every internal line (not connected to input or final particle)
\item associate a propagator
\begin{equation}\label{eqn:qftLecture16:660}
\frac{i}{p^2 - m^2 - i \epsilon},
\end{equation}
where \( p \) is the 4-momentum of the line.  External lines are \( \equiv 1 \).
\item Impose non-conservation with every vertex.
\item integrate \( \int d^4 p/(2 \pi)^4 \) over all momenta not fixed.
\item symmetry factors
\item vertex: \( (-i \lambda) \).
\end{enumerate}

\section{Back to our scalar theory.}

Applying these rules to the diagram \cref{fig:qftLecture16:qftLecture16Fig16}, we get
\begin{equation}\label{eqn:qftLecture16:680}
-i \lambda = i M,
\end{equation}
or
\begin{equation}\label{eqn:qftLecture16:700}
M = -\lambda.
\end{equation}
\imageFigure{../figures/phy2403-quantum-field-theory/qftLecture16Fig16}{First order interaction.}{fig:qftLecture16:qftLecture16Fig16}{0.15}

For the second order diagrams
\imageThreeFiguresOneLine{../figures/phy2403-quantum-field-theory/qftLecture16Fig17a}{../figures/phy2403-quantum-field-theory/qftLecture16Fig17b}{../figures/phy2403-quantum-field-theory/qftLecture16Fig17c}{Second order diagrams.}{fig:qftLecture16:qftLecture16Fig17}{scale=0.15}
The first diagram gives
\begin{equation}\label{eqn:qftLecture16:720}
\lr{ -i \lambda }^2
\frac{ i}{q_1^2 -m^2 -i \epsilon}
\frac{ i}{q_2^2 -m^2 -i \epsilon},
\end{equation}
where \( q_1 + q_2 = k_A + k_B \), so we can let \( q_2 = k_A + k_B - q_1 \), which gives
\begin{equation}\label{eqn:qftLecture16:740}
\int \frac{d^4 q_1}{(2 \pi)^4}
\lr{ -i \lambda }^2
\frac{ i}{q_1^2 -m^2 -i \epsilon}
\frac{ i}{\lr{
k_A + k_B - q_1
}^2 -m^2 -i \epsilon}
\end{equation}
Calculating the symmetry coefficients is a counting game, illustrated roughly in
\cref{fig:qftLecture16:qftLecture16Fig18}, where the \( 1/2 \) factor was eliminated by the two choices, and the rest by factorial counting (4 ways to pick first, leaving 3 ways for the next choice, two for the next, until the last.)
\imageFigure{../figures/phy2403-quantum-field-theory/qftLecture16Fig18}{Symmetry coefficient counting.}{fig:qftLecture16:qftLecture16Fig18}{0.2}
In the end we have a symmetry factor of \( (4 \times 3) \times 2 \times (4 \times 3) \).
%\section{figures. unsorted}
%\cref{fig:qftLecture16:qftLecture16Fig17a}.
%\imageFigure{../figures/phy2403-quantum-field-theory/qftLecture16Fig17a}{CAPTION: qftLecture16Fig17a}{fig:qftLecture16:qftLecture16Fig17a}{0.2}
%\cref{fig:qftLecture16:qftLecture16Fig17b}.
%\imageFigure{../figures/phy2403-quantum-field-theory/qftLecture16Fig17b}{CAPTION: qftLecture16Fig17b}{fig:qftLecture16:qftLecture16Fig17b}{0.2}
%\cref{fig:qftLecture16:qftLecture16Fig17c}.
%\imageFigure{../figures/phy2403-quantum-field-theory/qftLecture16Fig17c}{CAPTION: qftLecture16Fig17c}{fig:qftLecture16:qftLecture16Fig17c}{0.2}

%}
%\EndNoBibArticle
