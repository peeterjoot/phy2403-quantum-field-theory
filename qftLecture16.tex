%
% Copyright � 2017 Peeter Joot.  All Rights Reserved.
% Licenced as described in the file LICENSE under the root directory of this GIT repository.
%
%{
\input{../latex/blogpost.tex}
\renewcommand{\basename}{qftLecture16}
\renewcommand{\dirname}{notes/phy2403/}
\newcommand{\keywords}{PHY2403H}
\input{../latex/peeter_prologue_print2.tex}

%\usepackage{phy2403}
\usepackage{peeters_braket}
\usepackage{peeters_layout_exercise}
\usepackage{peeters_figures}
\usepackage{mathtools}
\usepackage{siunitx}
\usepackage{macros_cal} % LL
\usepackage{simplewick}
\usepackage{verbatim}

\newcommand{\ultensor}[3]{{{#1}^{#2}}_{#3}}
\newcommand{\deltathree}[0]{\delta^{(3)}}
\newcommand{\deltafour}[0]{\delta^{(4)}}

\beginArtNoToc
\generatetitle{PHY2403H Quantum Field Theory.  Lecture 16: XXX.  Taught by Prof.\ Erich Poppitz}
%\chapter{XXX}
\label{chap:qftLecture16}

%%Peeter's lecture notes from class.  These may be incoherent and rough.
%%
%%These are notes for the UofT course PHY2403H, Quantum Field Theory, taught by Prof. Erich Poppitz, covering \textchapref{{1}} \citep{peskin1995introduction} content.

\paragraph{DISCLAIMER: Very rough notes from class, with some additional side notes.}

These are notes for the UofT course PHY2403H, Quantum Field Theory, taught by Prof. Erich Poppitz, fall 2018.
%, covering \textchapref{{1}} \citep{peskin1995introduction} content.

\section{Review}

We finished by defining the differential cross section

\makedefinition{Differential cross section.}{dfn:qftLecture16:20}{
\begin{equation*}
\frac{d^3 \sigma}{dp_x dp_y dp_z} = \frac{
\text{number of scattering events with \( \Bp_{\txtf} \) between \( (\Bp_\txtf, \Bp_\txtf + d \Bp_\txtf )\)}
}
{
\text{flux of incoming particles}
}.
\end{equation*}
} % definition

\section{Scattering}

In QFT we typically study \( 2 \rightarrow n \) inelastic scattering.  Most commonly the nature of the final state particles are different from the nature of the incoming state.

For example, we can collide two electrons, and can get muon and anti-muon particles
F1
or pions
F2, or even both
F3

In the \( \lambda \phi^4 \) theory we can have scattering events such as
F4a
F4b

How to calculate in QFT.  Initial state of 2 particles \( A, B \) with initital state
\begin{dmath}\label{eqn:qftLecture16:40}
\ket{\Bk_A, \Bk_B }_{\text{in}, T \rightarrow -\infty}
\end{dmath}
and final n-particle state
\begin{dmath}\label{eqn:qftLecture16:60}
\ket{\Bp_1, \Bp_2, \cdots, \Bp_n }_{\text{out}, T \rightarrow +\infty}
\end{dmath}
The QM transition amplitude from the initial to the final state is
\begin{dmath}\label{eqn:qftLecture16:80}
\prescript{}{\text{out}}{\bra{\Bp_1, \Bp_2, \cdots, \Bp_n }}
\ket{\Bk_A, \Bk_B }_{\text{in}}
=
\bra{\Bp_1, \Bp_2, \cdots, \Bp_n } e^{-2 i H T}
\ket{\Bk_A, \Bk_B }.
\end{dmath}
This is the amplitude for \( A B \rightarrow 1 \cdots n \).
Ultimately, we want the scattering x-section.

We will also be interested in decay rates, as there are unstable particles in QFT that can decay.  This doesn't happen in \( \lambda \phi^4 \) theory.
In a theory with 2 scalar fields \( \Phi, \varphi \) with \( m_\Phi > 2 m_\varphi \).  A possible interaction for such a theory is
\begin{dmath}\label{eqn:qftLecture16:100}
H_{\text{int}} = \mu \Phi \varphi^2,
\end{dmath}
which would permit \( \Phi \rightarrow \varphi \varphi \) decays.
HW4 has a coupling like \( (h/V) \partial_\mu \phi^a \partial^\mu \phi^a \) for which a \( h \rightarrow \phi^a \phi^a \) decay is possible.

\makedefinition{Decay rate.}{dfn:qftLecture16:120}{
The decay rate is defined as
\begin{equation*}
\Gamma =
\frac{
\text{
Number of decays \( \Phi \rightarrow \varphi \varphi \) in unit time
}
}
{
\text{
Number of \( \Phi \) particles present
}
}
\end{equation*}
} % definition

What is the amplitude for such a decay transition?
\begin{dmath}\label{eqn:qftLecture16:140}
\bra{\Bk_\phi}_{\text{in}, T \rightarrow -\infty} \rightarrow
\bra{\Bk_1, \Bk_2}_{\text{out}, T \rightarrow +\infty}.
\end{dmath}
The amplitude for \( \Bk_\phi \rightarrow \Bk_1, \Bk_2 \).
\begin{dmath}\label{eqn:qftLecture16:160}
\bra{\Bk_1, \Bk_2} e^{-i 2 H T } \ket{\Bk_\phi}
=
\prescript{}{\text{out}}{\braket{\Bk_1, \Bk_2}{\Bk_\phi}}
\end{dmath}

\paragraph{mysterious seeming statement something like}: ``The decays are essentially due to interactions with vacuum fluctuations.''

\section{Calculating interactions}

We write
\begin{dmath}\label{eqn:qftLecture16:180}
\begin{aligned}
\prescript{}{\text{out}}{\braket{ \Bp_1, \cdots \Bp_n }{ \Bk_A, \Bk_B }}_{\text{in}}
&=
\lim_{T \rightarrow \infty}
\bra{ \Bp_1, \cdots \Bp_n } e^{-i 2 H T } \ket{ \Bk_A, \Bk_B } \\
&=
\bra{ \Bp_1, \cdots \Bp_n } \hatS \ket{ \Bk_A, \Bk_B } \\
&=
\bra{ \Bp_1, \cdots \Bp_n } \BOne + i \hatT \ket{ \Bk_A, \Bk_B },
\end{aligned}
\end{dmath}
where \( \hatS \) is called the S-matrix or scattering matrix, which is decomposed into a unit portion \( \BOne \) which is a convient way to exclude events with no scattering.  \( \BOne \) contributes for \( n = 2 \) only, but is an \( n \) scattering amplitude.
We are really interested in the \( i \hatT \) portion of this amplitude
\begin{equation}\label{eqn:qftLecture16:200}
\bra{ \Bp_1, \cdots \Bp_n } i \hatT \ket{ \Bk_A, \Bk_B }
=
(2 \pi)^4 \deltafour( \Bk_A + \Bk_B - \sum_{i = 1}^n \Bp_i )
\times
i M( \Bk_A + \Bk_B \rightarrow \Bp_1 \cdots \Bp_n ).
\end{equation}
This amounts to a definition of \( M \).
Recall that we found
\begin{dmath}\label{eqn:qftLecture16:220}
U(T, -T)
= T \lr{ e^{-i \int_{-T}^T H_I(t') dt'} }
=
e^{i H_0(T - t_0)}
e^{-i 2 H T}
e^{-i H_0(-T - t_0)}.
\end{dmath}
We want to replace the \( e^{-i 2 H T} \) in the matrix element above with \( U \).

In perturbation theory, we assume (conjecture) that
\begin{dmath}\label{eqn:qftLecture16:240}
\ket{ \Bk_A, \Bk_B }
\sim
\ket{ \Bk_A, \Bk_B }_\txto
\sim
\text{const}\, a^\dagger_{\Bk_A} a^\dagger_{\Bk_B} \ket{0}
\end{dmath}

Because we'll be squaring the amplitudes, we can assume that the \( e^{i H_0(T-t_0)} \) will result in just phase factors that won't survive, so in \cref{eqn:qftLecture16:180} we can insert \( U \)
\begin{dmath}\label{eqn:qftLecture16:280}
\prescript{}{\text{out}}{\braket{ \Bp_1, \cdots \Bp_n }{ \Bk_A, \Bk_B }}_{\text{in}}
=
\lim_{T \rightarrow \infty}
\bra{ \Bp_1, \cdots \Bp_n } U(T, -T) \ket{ \Bk_A, \Bk_B }
\end{dmath}

\begin{dmath}\label{eqn:qftLecture16:300}
\bra{ \Bp_1, \cdots \Bp_n } i\hatT { \Bk_A, \Bk_B }
=
\lim_{T \rightarrow \infty(1 - i \epsilon) }
\prescript{}{0}{
\bra{ \Bp_1, \cdots \Bp_n }
T( e^{-i \int_{-T}^T H_i(t') dt' } )
 \ket{ \Bk_A, \Bk_B }
}_0
\end{dmath}

These are connected and amputated graphs.

\paragraph{What is ``connected and amputated''?}

Explaining by example.  \( n = 2, \lambda \phi^4/4! \).

\begin{dmath}\label{eqn:qftLecture16:320}
\bra{0}
a_{\Bp_1}
a_{\Bp_2}
\lr{
\cancel{1}
- \frac{i \lambda}{4!} \int d^4 x \phi_I^4(x)
+ \inv{2} \lr{ \frac{i \lambda}{4!}}^2 \int d^4 x d^4 y \phi_I^4(x) \phi_I^4(y)
+ \cdots
}
a_{\Bk_A}^\dagger
a_{\Bk_B}^\dagger
\ket{0}
\end{dmath}
Here time ordering operations are implied, but not written explicitly.
Also, the ``amputated'' indicates that we are going to be dropping the \( 1 \) portion of the exponential expansion (as we've also dropped that in \cref{eqn:qftLecture16:300}).
We will also be using a relativistic normalization so that the \(
a_{\Bk_A}^\dagger
a_{\Bk_B}^\dagger \) terms include
\( \sqrt{
2 \omega_{\Bk_A}
2 \omega_{\Bk_B} } \)
contributions and the \(
a_{\Bp_1}
a_{\Bp_2}
\) include
\( \sqrt{
2 \omega_{\Bp_1}
2 \omega_{\Bp_2} } \) contributions.

\begin{dmath}\label{eqn:qftLecture16:340}
T
\contraction{}{\phi}{{}_I(x_1)}{\phi}
\phi_I(x_1)\phi_I(x_2)
= D_F(x_1 - x_2)
\end{dmath}

When we look at
\begin{dmath}\label{eqn:qftLecture16:360}
\contraction{}{\phi}{{}_I(x_1)}{a}
\phi_I(x_1) a^\dagger_{\Bk}
\sqrt{ 2 \omega_\Bk}
=
\int \frac{d^3 p}{(2 \pi)^3} \frac{e^{-i p \cdot x}}{\sqrt{2 \omega_\Bp}}
\contraction{}{a}{{}_\Bp}{a}
a_\Bp a^\dagger_\Bk
\sqrt{ 2 \omega_\Bk}
=
\int \frac{d^3 p}{(2 \pi)^3} \frac{e^{-i p \cdot x}}{\sqrt{2 \omega_\Bp}}
\deltathree(\Bp - \Bk)
\sqrt{ 2 \omega_\Bk}
= e^{-i k \cdot x}.
\end{dmath}
Similarily
\begin{dmath}\label{eqn:qftLecture16:440}
\contraction{}{a}{{}_\Bp}{\phi}
a_\Bp \phi_I(x_1)
\sqrt{ 2 \omega_\Bp}
=
\int \frac{d^3 k}{(2 \pi)^3} \frac{e^{i k \cdot x}}{\sqrt{2 \omega_\Bk}}
\contraction{}{a}{{}_\Bp}{a}
a_\Bp a^\dagger_\Bk
\sqrt{ 2 \omega_\Bk}
=
\int \frac{d^3 k}{(2 \pi)^3} \frac{e^{i k \cdot x}}{\sqrt{2 \omega_\Bk}}
\deltathree(\Bp - \Bk)
\sqrt{ 2 \omega_\Bk}
= e^{+i p \cdot x}.
\end{dmath}

Summarizing
\begin{dmath}\label{eqn:qftLecture16:380}
\begin{aligned}
\contraction{}{\phi}{{}_I(x_1)}{a}
\phi_I(x_1) a^\dagger_{\Bp}
&= e^{-i p \cdot x} \\
\contraction{}{a}{{}_\Bp}{\phi}
a_\Bp \phi_I(x_1)
&= e^{i p \cdot x}.
\end{aligned}
\end{dmath}

\begin{comment}
\section{junk}
\begin{dmath}\label{eqn:qftLecture15b:260}
%\sideset{_a^b}{'}{x}
\prescript{14}{2}{\mathbf{C}}
\end{dmath}
\begin{dmath}\label{eqn:qftLecture16:420}
\prescript{}{2}{\mathbf{C}}
\end{dmath}
\end{comment}

%}
%\EndArticle
\EndNoBibArticle
