%
% Copyright � 2018 Peeter Joot.  All Rights Reserved.
% Licenced as described in the file LICENSE under the root directory of this GIT repository.
%
%{
\input{../latex/blogpost.tex}
\renewcommand{\basename}{lukenotetypos}
%\renewcommand{\dirname}{notes/phy1520/}
\renewcommand{\dirname}{notes/ece1228-electromagnetic-theory/}
%\newcommand{\dateintitle}{}
%\newcommand{\keywords}{}

\input{../latex/peeter_prologue_print2.tex}

\usepackage{peeters_layout_exercise}
\usepackage{peeters_braket}
\usepackage{peeters_figures}
\usepackage{siunitx}
\usepackage{verbatim}
%\usepackage{mhchem} % \ce{}
%\usepackage{macros_bm} % \bcM
%\usepackage{macros_qed} % \qedmarker
%\usepackage{txfonts} % \ointclockwise

\newcommand{\deltathree}[0]{\delta^{(3)}}
\newcommand{\deltafour}[0]{\delta^{(4)}}

\beginArtNoToc

\generatetitle{Typo notes.}
%\chapter{XXX}
%
% \citep{sakurai2014modern} pr X.Y
% \citep{pozar2009microwave}
% \citep{qftLectureNotes}
% \citep{doran2003gap}
% \citep{jackson1975cew}
% \citep{griffiths1999introduction}

\begin{itemize}
\item Page 8.  brace missing in last entry of the table: \( (Z boson \rightarrow (Z boson) \).
\item Page 62. Third paragraph ``is and arbitrary'' \( \rightarrow \) ``is an arbitrary''.
\item Page 62. Notation clarification: it looks like you are using \( \lambda_\mu \) as the four vector basis, and I don't see any reference to a four-vector basis notation prior to this (but may have missed it).  Isn't \( \gamma_\mu \) typically used for that when the basis is made explicit?
\item Page 66. (V.8) looks like it has a sign error relative to (V.5) and should be:
\begin{dmath*}
\phi(x) = \phi_0(x) - i \int d^4 y\, D_R(x - y) \rho(y).
\end{dmath*}
Rationale:
\begin{dmath*}
\lr{ \partial_\mu \partial^\mu + \mu^2 }
\phi(x) = 
\lr{ \partial_\mu \partial^\mu + \mu^2 }
\phi_0(x)
 - i \int d^4 y\, 
\lr{ \partial_\mu \partial^\mu + \mu^2 }
D_R(x - y) \rho(y)
=
0 
 - i \int d^4 y\, 
(-i) \deltafour(x - y) \rho(y)
=
- \rho(y).
\end{dmath*}
\end{itemize}

%}
\EndArticle
%\EndNoBibArticle
