%
% Copyright � 2017 Peeter Joot.  All Rights Reserved.
% Licenced as described in the file LICENSE under the root directory of this GIT repository.
%
%{
%%%\input{../latex/blogpost.tex}
%%%\renewcommand{\basename}{qftLecture10}
%%%\renewcommand{\dirname}{notes/phy2403/}
%%%\newcommand{\keywords}{PHY2403H}
%%%\input{../latex/peeter_prologue_print2.tex}
%%%
%%%%\usepackage{phy2403}
%%%\usepackage{peeters_braket}
%%%%\usepackage{peeters_layout_exercise}
%%%\usepackage{peeters_figures}
%%%\usepackage{mathtools}
%%%\usepackage{siunitx}
%%%\usepackage{macros_cal} % LL
%%%
%%%\newcommand{\ultensor}[3]{{{#1}^{#2}}_{#3}}
%%%%\newcommand{\ulLambda}[2]{\ultensor{\Lambda}{#1}{#2}}
%%%%\newcommand{\ulDelta}[2]{\ultensor{\delta}{#1}{#2}}
%%%
%%%\beginArtNoToc
%%%%\generatetitle{PHY2403H Quantum Field Theory.  Lecture 10: Lorentz boosts, generator of spacetime translation, Lorentz invariant field representation.  Taught by Prof.\ Erich Poppitz}
\chapter{Lorentz boosts, generator of spacetime translation, Lorentz invariant field representation.}
\label{chap:qftLecture10}
\index{Lorentz boost}
\index{Lorentz invariance}

%%%\paragraph{DISCLAIMER: Very rough notes from class, with some additional side notes.}
%%%
%%%These are notes for the UofT course PHY2403H, Quantum Field Theory I, taught by Prof. Erich Poppitz fall 2018.
%%%%, covering \textchapref{{1}} \citep{peskin1995introduction} content.
%%%
\section{Lorentz transform symmetries.}
\index{Lorentz!symmetry}

From last time, recall that an infinitesimal Lorentz transform has the form
\begin{dmath}\label{eqn:qftLecture10:20}
x^\mu \rightarrow x^\mu + \omega^{\mu\nu} x_\nu,
\end{dmath}
where
\begin{dmath}\label{eqn:qftLecture10:40}
\omega^{\mu\nu} = -\omega^{\nu\mu}.
\end{dmath}

We showed last time that \( \omega^{ij} \) induces a rotation, and will show today that \( \omega^{0i} \) is a boost.

We introduced a three index current, factoring out explicit dependence on the incremental Lorentz transform tensor \( \omega^{\mu\nu} \) as follows
\begin{dmath}\label{eqn:qftLecture10:80}
J^{\nu \mu\rho} = \inv{2} \lr{ x^\rho T^{\nu\mu} - x^\mu T^{\nu\rho} },
\end{dmath}
and can easily show that this current has the desired zero four-divergence property
\begin{dmath}\label{eqn:qftLecture10:100}
\partial_\nu J^{\nu \mu\rho}
= \inv{2} \lr{
(\partial_\nu x^\rho) T^{\nu\mu}
+
x^\rho \cancel{\partial_\nu T^{\nu\mu} }
- (\partial_\nu x^\mu) T^{\nu\rho}
- x^\mu \cancel{\partial_\nu T^{\nu\rho} }
}
= \inv{2} \lr{
T^{\rho\mu}
+
- T^{\mu\rho}
}
= 0,
\end{dmath}
since the energy-momentum tensor is symmetric.

Defining charge in the usual fashion \( Q = \int d^3 x j^0 \), so we can define a charge for each pair of indexes \( \mu\nu \), and in particular
\begin{dmath}\label{eqn:qftLecture10:120}
Q^{0k} = \int d^3 x J^{0 0 k} = \inv{2} \int d^3 x \lr{ x^k T^{00} - x^0 T^{0k} }
\end{dmath}
\begin{dmath}\label{eqn:qftLecture10:540}
\dot{Q}^{0k}
= \int d^3 x \dot{J}^{0 0k}
= \inv{2} \int d^3 x \lr{ x^k \dot{T}^{00} - x^0 \dot{T}^{0k} }.
\end{dmath}

However, since \( 0 = \partial_\mu T^{\mu \nu} = \dot{T}^{0 \nu} + \partial_j T^{j \nu} \),
or \( \dot{T}^{0 \nu} = -\partial_j T^{j \nu} \),
\begin{dmath}\label{eqn:qftLecture10:560}
\dot{Q}^{0k}
= \inv{2} \int d^3 x \lr{ x^k (-\partial_j T^{j0}) - T^{0k} - x^0 (-\partial_j T^{jk}) }
= \inv{2} \int d^3 x \lr{
   \partial_j (-x^k T^{j0}) + (\partial_j x^k) T^{j0}
   - T^{0k} + x^0 \partial_j T^{jk}
}
= \inv{2} \int d^3 x \lr{
   \partial_j (-x^k T^{j0}) + \cancel{T^{k0}}
   - \cancel{T^{0k}} + x^0 \partial_j T^{jk}
}
= \inv{2} \int d^3 x
   \partial_j \lr{
-x^k T^{j0}
+ x^0 T^{jk}
},
\end{dmath}
which leaves just surface terms, so \( \dot{Q}^{0k} = 0 \).

\paragraph{Quantizing:}

From our previous identification \cref{eqn:qftLecture9:560}, we have
\begin{dmath}\label{eqn:qftLecture10:580}
T^{\nu\mu} = \partial^\nu \phi \partial^\mu \phi - g^{\nu\mu} \LL.
\end{dmath}
In particular
\begin{dmath}\label{eqn:qftLecture10:600}
T^{00}
= \partial^0 \phi \partial^0 \phi - \inv{2} \lr{ \partial_0 \phi \partial^0 \phi + \partial_k \phi \partial^k \phi }
= \inv{2} \partial^0 \phi \partial^0 \phi - \inv{2} (\spacegrad \phi)^2,
\end{dmath}
and
\begin{dmath}\label{eqn:qftLecture10:620}
T^{0k} = \partial^0 \phi \partial^k \phi.
\end{dmath}
We may quantize these energy momentum tensor components as
\begin{dmath}\label{eqn:qftLecture10:640}
\begin{aligned}
\hatT^{00} &= \inv{2} \pihat^2 + \inv{2} (\spacegrad \phihat)^2 \\
\hatT^{0k} &= \inv{2} \pihat \partial^k \phihat.
\end{aligned}
\end{dmath}

We can now start computing the commutators associated with the charge operator.  The first of those commutators is
\begin{dmath}\label{eqn:qftLecture10:140}
\antisymmetric{\hatT^{00}(\Bx)}{\phihat(\By)}
=
\inv{2}
\antisymmetric{\pihat^2(\Bx)}{\phihat(\By)},
\end{dmath}
which can be evaluated using the field commutator analogue of \( \antisymmetric{F(p)}{q} = i F' \) which is
\begin{dmath}\label{eqn:qftLecture10:660}
\antisymmetric{F(\pihat(\Bx))}{\phihat(\By)} = -i \frac{dF}{d \pihat} \delta(\Bx - \By),
\end{dmath}
to give
\begin{dmath}\label{eqn:qftLecture10:680}
\antisymmetric{\hatT^{00}(\Bx)}{\phihat(\By)}
= -i \delta^3(\Bx - \By) \pihat(\Bx).
\end{dmath}

The other required commutator is
\begin{dmath}\label{eqn:qftLecture10:160}
\antisymmetric{\hatT^{0i}(\Bx)}{\phihat(\By)}
=
\antisymmetric{\pihat(\Bx)\partial^i \phihat(\Bx)}{\phihat(\By)}
=
\partial^i \phihat(\Bx)
\antisymmetric{\pihat(\Bx)
}{\phihat(\By)}
= -i \delta^3(\Bx - \By) \partial^i \phihat(\Bx).
\end{dmath}

The charge commutator with the field can now be computed
\begin{dmath}\label{eqn:qftLecture10:180}
i \epsilon \antisymmetric{\hatQ^{0k}}{\phihat(\By)}
=
i
\frac{\epsilon}{2} \int d^3 x
\lr{
x^k
\antisymmetric{\hatT^{00}}{\phihat(\By)}
-
x^0
\antisymmetric{\hatT^{0k}}{\phihat(\By)}
}
=
\frac{\epsilon}{2} \lr{ y^k \pihat(\By) - y^0 \partial^k \phihat(\By) }
=
\frac{\epsilon}{2} \lr{ y^k \dot{\phihat}(\By) - y^0 \partial^k \phihat(\By) },
\end{dmath}
so to first order in \( \epsilon \)
\begin{dmath}\label{eqn:qftLecture10:200}
e^{i \epsilon \hatQ^{0k} } \phihat(\By)
e^{-i \epsilon \hatQ^{0k} }
=
\phihat(\By)
+ \frac{\epsilon}{2} y^k \dot{\phihat}(\By)
+ \frac{\epsilon}{2} y^0 \partial_k \phihat(\By).
\end{dmath}

For example, with \( k = 1 \)
\begin{dmath}\label{eqn:qftLecture10:700}
e^{i \epsilon \hatQ^{0k} } \phihat(\By)
e^{-i \epsilon \hatQ^{0k} }
=
\phihat(\By)
+ \frac{\epsilon}{2} \lr{
   y^1 \dot{\phihat}(\By)
   +
   y^0 \PD{y^1}{\phihat}(\By)
}
=
\phihat(y^0 + \frac{\epsilon}{2} y^1,
y^1 + \frac{\epsilon}{2} y^2, y^3).
\end{dmath}

This is a boost.  If we compare explicitly to an infinitesimal Lorentz transformation of the coordinates
\begin{dmath}\label{eqn:qftLecture10:220}
\begin{aligned}
x^0 \rightarrow x^0 + \omega^{01} x_1 &= x^0 - \omega^{01} x^1 \\
x^1 \rightarrow x^1 + \omega^{10} x_0 &= x^1 - \omega^{01} x_0 = x^1 - \omega^{01} x^0,
\end{aligned}
\end{dmath}
we can make the identification
\begin{dmath}\label{eqn:qftLecture10:240}
\frac{\epsilon}{2} = - \omega^{01}.
\end{dmath}

We now have the explicit form of the generator of a spacetime translation
\index{generator!spacetime translation}
%\begin{dmath}\label{eqn:qftLecture10:260}
\boxedEquation{eqn:qftLecture10:260}{
\hatU(\Lambda) = \exp\lr{-i \omega^{0k} \int d^3 x \lr{ \hatT^{00} x^k - \hatT^{0k} x^0 }}.
}
%\end{dmath}

%%where
%%
%%\begin{dmath}\label{eqn:qftLecture10:280}
%%\end{dmath}
%%
An explicit boost along the x-axis has the form
\begin{dmath}\label{eqn:qftLecture10:300}
\hatU(\Lambda) \phihat(t, \Bx)
\hatU^\dagger(\Lambda)
=
\phihat\lr{ \frac{t - vx}{\sqrt{1 - v^2}}, \frac{x - vt}{\sqrt{1 - v^2}}, y, z },
\end{dmath}
and more generally
\begin{dmath}\label{eqn:qftLecture10:320}
\hatU(\Lambda) \phihat(x) \hatU^\dagger(\Lambda) =
\phihat(\Lambda x),
\end{dmath}
where
\( x \) is a four vector,
\( (\Lambda x)^\mu = \ultensor{\Lambda}{\mu}{\nu} x^\nu \), and
\(
\ultensor{\Lambda}{\mu}{\nu}
\approx
\ultensor{\delta}{\mu}{\nu}
+
\ultensor{\omega}{\mu}{\nu} \).

\section{Transformation of momentum states.}
\index{momentum states!transformation}
In the momentum space representation

\begin{dmath}\label{eqn:qftLecture10:340}
\phihat(x)
=
\int \frac{d^3 p}{(2 \pi)^3 \sqrt{2 \omega_\Bp}}  \lr{
   e^{i (\omega_\Bp t - \Bp \cdot \Bx)} \hata_\Bp
+
   e^{-i (\omega_\Bp t - \Bp \cdot \Bx)} \hata^\dagger_\Bp
}
=
\int \frac{d^3 p}{(2 \pi)^3 \sqrt{2 \omega_\Bp}}  \evalbar{
\lr{
   e^{i p^\mu x^\mu } \hata_\Bp
+
   e^{-i p^\mu x^\mu } \hata^\dagger_\Bp
}
}{p_0 = \omega_\Bp}.
\end{dmath}
\begin{dmath}\label{eqn:qftLecture10:720}
\hatU(\Lambda) \phihat(x) \hatU^\dagger(\Lambda)
=
\phihat(\Lambda x)
=
\int \frac{d^3 p}{(2 \pi)^3 \sqrt{2 \omega_\Bp}}  \evalbar{
\lr{
   e^{i p^\mu \ultensor{\Lambda}{\mu}{\nu} x^\nu }
\hata_\Bp
+
   e^{-i p^\mu \ultensor{\Lambda}{\mu}{\nu} x^\nu } \hata^\dagger_\Bp
}
}{p_0 = \omega_\Bp}.
\end{dmath}
This can be put into an explicitly Lorentz invariant form
\begin{dmath}\label{eqn:qftLecture10:740}
\phihat(\Lambda x)
=
\int \frac{dp^0 d^3 p}{(2\pi)^3} \delta(p_0^2 - \Bp^2 - m^2) \Theta(p^0) \sqrt{2 \omega_\Bp}
   e^{i p^\mu \ultensor{\Lambda}{\mu}{\nu} x^\nu }
\hata_\Bp + \text{h.c.}
=
\int \frac{dp^0 d^3 p}{(2\pi)^3}
\lr{
\frac{\delta(p_0 - \omega_\Bp)}{2 \omega_\Bp}
+
\frac{\delta(p_0 + \omega_\Bp)}{2 \omega_\Bp}
}
\Theta(p^0) \sqrt{2 \omega_\Bp}  \hata_\Bp + \text{h.c.},
\end{dmath}
which recovers \cref{eqn:qftLecture10:720} by making use of the delta function identity
\( \delta(f(x)) = \sum_{f(x_\conj) = 0} \frac{\delta(x - x_\conj)}{f'(x_\conj)} \), since the
\( \Theta(p^0) \) kills the second delta function.

We now have a more explicit Lorentz invariant structure
\begin{dmath}\label{eqn:qftLecture10:380}
\phihat(\Lambda x)
=
\int \frac{dp^0 d^3 p}{(2\pi)^3} \delta(p_0^2 - \Bp^2 - m^2) \Theta(p^0) \sqrt{2 \omega_\Bp}
   e^{i p^\mu \ultensor{\Lambda}{\mu}{\nu} x^\nu }
\hata_\Bp + \text{h.c.}
\end{dmath}

Recall that a boost moves a spacetime point along a parabola, such as that of \cref{fig:constantMomentumSurface:constantMomentumSurfaceFig2}, whereas a rotation moves along a constant ``circular'' trajectory of a hyper-paraboloid.  In general, a Lorentz transformation may move a spacetime point along any path on a hyper-paraboloid such as the one depicted (in two spatial dimensions) in
\cref{fig:constantMomentumSurface:constantMomentumSurfaceFig1}.  This paraboloid depict the surfaces of constant energy-momentum \( p^0 = \sqrt{ \Bp^2 + m^2 } \).  Because a Lorentz transformation only shift points along that energy-momentum surface, but cannot change the sign of the energy coordinate \( p^0 \), this means that \( \Theta(p^0) \) is also a Lorentz invariant.
\imageFigure{../figures/phy2403-quantum-field-theory/constantMomentumSurfaceFig2}{One dimensional spacetime surface for constant \( (p^0)^2 - \Bp^2 = m^2 \)}{fig:constantMomentumSurface:constantMomentumSurfaceFig2}{0.3}
\imageFigure{../figures/phy2403-quantum-field-theory/constantMomentumSurfaceFig1}{Surface of constant squared four-momentum.}{fig:constantMomentumSurface:constantMomentumSurfaceFig1}{0.3}

Let's change variables
\begin{dmath}\label{eqn:qftLecture10:400}
p^\lambda = \ultensor{\Lambda}{\lambda}{\rho} {p'}^{\rho},
\end{dmath}
so that
\begin{dmath}\label{eqn:qftLecture10:420}
p_\mu
\ultensor{\Lambda}{\mu}{\nu} x^\nu
=
\ultensor{\Lambda}{\lambda}{\rho} {p'}^\rho g_{\lambda\nu} \ultensor{\Lambda}{\nu}{\sigma} x^{\sigma}
=
{p'}^\rho
\lr{ \ultensor{\Lambda}{\lambda}{\rho}
g_{\lambda\nu} \ultensor{\Lambda}{\nu}{\sigma} } x^{\sigma}
=
{p'}^\rho g_{\rho\sigma} x^\sigma,
\end{dmath}
which gives
\begin{dmath}\label{eqn:qftLecture10:440}
\phihat(\Lambda x)
=
\int \frac{d{p'}^0 d^3 p'}{(2\pi)^3} \delta({p'}_0^2 - {\Bp'}^2 - m^2) \Theta(p^0) \sqrt{2 \omega_{\Lambda \Bp'}} e^{i p' \cdot x}  \hata_{\Lambda \Bp'} + \text{h.c.}
=
\int \frac{dp^0 d^3 p}{(2\pi)^3} \delta({p}_0^2 - {\Bp}^2 - m^2) \Theta(p^0) \sqrt{2 \omega_{\Lambda \Bp}} e^{i p \cdot x}  \hata_{\Lambda \Bp} + \text{h.c.}
\end{dmath}
Since
\begin{dmath}\label{eqn:qftLecture10:460}
\phihat(x)
=
\int \frac{dp^0 d^3 p}{(2\pi)^3} \delta({p}_0^2 - {\Bp}^2 - m^2) \Theta(p^0) \sqrt{2 \omega_{\Bp}} e^{i p \cdot x}  \hata_{\Bp} + \text{h.c.}
\end{dmath}
we can now conclude that the creation and annihilation operators transform as
%\begin{dmath}\label{eqn:qftLecture10:480}
\boxedEquation{eqn:qftLecture10:480}{
\sqrt{2 \omega_{\Lambda \Bp}} \hata_{\Lambda \Bp}
=
\hatU(\Lambda)
\sqrt{2 \omega_{ \Bp}} \hata_{ \Bp}
\hatU^\dagger(\Lambda).
}
%\end{dmath}

If the desired normalization for a momentum state is assumed to be
\begin{dmath}\label{eqn:qftLecture10:500}
\sqrt{2 \omega_{ \Bp}} \hata^\dagger_{ \Bp} \ket{0} = \ket{\Bp},
\end{dmath}
then by
noting that \( \hatU(\Lambda) \ket{0}  = \ket{0} \) (i.e. the ground state is Lorentz invariant), we have
\begin{dmath}\label{eqn:qftLecture10:520}
\sqrt{2 \omega_{\Lambda \Bp}} \hata^\dagger_{\Lambda \Bp} \ket{0}
=
\hatU(\Lambda) \sqrt{ 2\omega_\Bp} \hata^\dagger_\Bp \hatU^\dagger(\Lambda) \hatU(\Lambda) \ket{0}
=
\hatU(\Lambda) \sqrt{ 2\omega_\Bp} \hata^\dagger_\Bp \ket{0}
=
\hatU(\Lambda) \ket{\Bp}.
\end{dmath}
The normalization \cref{eqn:qftLecture10:500} means that the Lorentz transformation of a momentum state, takes a particularly simple form
\begin{dmath}\label{eqn:qftLecture10:760}
\hatU(\Lambda) \ket{\Bp} = \ket{\Lambda \Bp}.
\end{dmath}

In \citep{peskin1995introduction}, this is argued differently.  In particular, it's argued that
\cref{eqn:qftLecture10:500} must be the required normalization based on a requirement for Lorentz invariant measure, and then demands \( \hatU(\Lambda) \ket{\Bp} = \ket{\Lambda \Bp} \).  After this
\cref{eqn:qftLecture10:480} follows as a consequence (albeit, how to conclude that is not spelled out in detail).

%}
%%%\EndNoBibArticle
