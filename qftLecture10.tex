%
% Copyright � 2017 Peeter Joot.  All Rights Reserved.
% Licenced as described in the file LICENSE under the root directory of this GIT repository.
%
%{
\input{../latex/blogpost.tex}
\renewcommand{\basename}{qftLecture10}
\renewcommand{\dirname}{notes/phy2403/}
\newcommand{\keywords}{PHY2403H}
\input{../latex/peeter_prologue_print2.tex}

%\usepackage{phy2403}
\usepackage{peeters_braket}
%\usepackage{peeters_layout_exercise}
\usepackage{peeters_figures}
\usepackage{mathtools}
\usepackage{siunitx}

\newcommand{\ultensor}[3]{{{#1}^{#2}}_{#3}}
%\newcommand{\ulLambda}[2]{\ultensor{\Lambda}{#1}{#2}}
%\newcommand{\ulDelta}[2]{\ultensor{\delta}{#1}{#2}}

\beginArtNoToc
\generatetitle{PHY2403H Quantum Field Theory.  Lecture 10: XXX.  Taught by Prof.\ Erich Poppitz}
%\chapter{XXX}
\label{chap:qftLecture10}

\paragraph{DISCLAIMER: Very rough notes from class, with some additional side notes.}

These are notes for the UofT course PHY2403H, Quantum Field Theory I, taught by Prof. Erich Poppitz fall 2018.
%, covering \textchapref{{1}} \citep{peskin1995introduction} content.

\section{Lorentz transform symmetries.}

\begin{dmath}\label{eqn:qftLecture10:20}
x^\mu \rightarrow x^\mu + \omega^{\mu\nu} x_\nu
\end{dmath}

\begin{dmath}\label{eqn:qftLecture10:40}
\omega^{\mu\nu} = -\omega^{\nu\mu}
\end{dmath}

\begin{dmath}\label{eqn:qftLecture10:60}
\begin{aligned}
\omega^{ij} &= \text{rotations} \\
\omega^{0i} &= \text{boosts}
\end{aligned}
\end{dmath}

\begin{dmath}\label{eqn:qftLecture10:80}
J^{\nu (\mu\rho)} = \inv{2} \lr{ x^\rho T^{\nu\mu} - x^\mu T^{\nu\rho} }
\end{dmath}

\begin{dmath}\label{eqn:qftLecture10:100}
\partial_nu J^{\nu (\mu\rho)}
= \inv{2} \lr{
\partial_\nu x^\rho T^{\nu\mu}
+
x^\rho \cancel{\partial_\nu T^{\nu\mu} }
- \partial_\nu x^\mu T^{\nu\rho}
- x^\mu \cancel{\partial_\nu T^{\nu\rho} }
}
=
...
= 0.
\end{dmath}

\begin{dmath}\label{eqn:qftLecture10:120}
Q^{0i} = \int d^3 x J^{0 (oi)} = \inv{2} \int d^3 x \lr{ x^i T^{00} - x^0 T^{0i} }
\dot{Q}^{0i}
= \int d^3 x J^{0 (oi)}
= \inv{2} \int d^3 x \lr{ x^i \dot{T}^{00} - x^0 \dot{T}^{0i} }
= \inv{2} \int d^3 x \lr{ x^i (-\partial_j T^{j0}) - T^{0i} - x^0 (-\partial_j T^{ji}) }
= \inv{2} \int d^3 x \lr{ \partial_j (-x^i T^{j0}) + (\partial_j x^i) T^{j0}
- T^{0i} - x^0 (-\partial_j T^{ji}) }
= \text{surface terms}
\end{dmath}

so \( \dot{Q}^{0i} = 0 \).

Quantizing

\begin{dmath}\label{eqn:qftLecture10:140}
\antisymmetric{\hatT^{00}(\Bx)}{\phihat(\By)}
=
\inv{2}
\antisymmetric{\pihat^2(\Bx)}{\phihat(\By)}
=
\inv{2}
\antisymmetric{\pihat^2(\Bx)}{\phihat(\By)}
=
\pihat(\Bx)
\antisymmetric{\pihat(\Bx)}{\phihat(\By)}
= -i \delta^3(\Bx - \By) \pihat(\Bx)
\end{dmath}

Also:
\begin{dmath}\label{eqn:qftLecture10:160}
\antisymmetric{\hatT^{0i}(\Bx)}{\phihat(\By)}
=
\antisymmetric{\pihat(\Bx)\partial^i \phihat(\Bx)}{\phihat(\By)}
= -i \delta^3(\Bx - \By) \partial^i \phihat(\Bx)
\end{dmath}

\begin{dmath}\label{eqn:qftLecture10:180}
i \epsilon \antisymmetric{\hatQ^{0i}}{\phihat(\By)}
=
i
\frac{\epsilon}{2} \int d^3 x
\lr{
x^i
\antisymmetric{\hatT^{00}}{\phihat(\By)}
-
x^0
\antisymmetric{\hatT^{0i}}{\phihat(\By)}
}
=
\frac{\epsilon}{2} \lr{ y^i \pihat(\By) - y^0 \partial^i \phihat(\By) }
=
\frac{\epsilon}{2} \lr{ y^i \dot{\phihat}(\By) - y^0 \partial^i \phihat(\By) }
\end{dmath}

\begin{dmath}\label{eqn:qftLecture10:200}
e^{i \epsilon \hatQ^{0k} } \phihat(\By)
e^{-i \epsilon \hatQ^{0k} }
=
\phihat(\By)
+ \frac{\epsilon}{2} y^k \dot{\phihat}(\By)
- \frac{\epsilon}{2} y^0 \partial^k \phihat(\By)
+ \cdots
=
\phihat(\By)
+ \frac{\epsilon}{2} \lr{
   y^1 \dot{\phihat}(\By)
   -
   y^0 \PD{y^1}{\phihat}(\By)
}
=
\phihat(y^0 + \frac{\epsilon}{2} y^1,
y^1 + \frac{\epsilon}{2} y^2, y^3)
\end{dmath}

This is a boost.  Compare to

\begin{dmath}\label{eqn:qftLecture10:220}
\begin{aligned}
x^0 \rightarrow x^0 + \omega^{01} x_1 &= x^0 - \omega^{01} x^1 \\
x^1 \rightarrow x^1 + \omega^{10} x_0 &= x^1 - \omega^{01} x_0 = x^1 - \omega^{01} x^0
\end{aligned}
\end{dmath}

identification:

\begin{dmath}\label{eqn:qftLecture10:240}
\frac{\epsilon}{2} = - \omega^{01}.
\end{dmath}

\begin{dmath}\label{eqn:qftLecture10:260}
\hatU(\Lambda) = \exp\lr{-i \omega^{01} \int d^3 x \lr{ \hatT^{00} x^i - \hatT^{0i} x^0 }}
\end{dmath}

where

\begin{dmath}\label{eqn:qftLecture10:280}
\ultensor{\Lambda}{\mu}{\nu}
\approx
\ultensor{\delta}{\mu}{\nu}
+
\ultensor{\omega}{\mu}{\nu}
\end{dmath}

\begin{dmath}\label{eqn:qftLecture10:300}
\hatU(\Lambda) \phihat(t, \Bx)
\hatU^\dagger(\Lambda)
=
\phihat\lr{ \gamma (t - vx), \gamma(x - vt), y, z }.
\end{dmath}

Or

\begin{dmath}\label{eqn:qftLecture10:320}
\hatU(\Lambda) \phihat(x) \hatU^\dagger(\Lambda) =
\phihat(\Lambda x)
\end{dmath}
where \( (\Lambda x)^\mu = \ultensor{\Lambda}{\mu}{\nu} x^\nu \) and \( x \) is a four vector.

\section{Transformation of momentum states}
In the momentum space representation

\begin{dmath}\label{eqn:qftLecture10:340}
\phihat(x)
=
\int \frac{d^3 p}{(2 \pi)^3 \sqrt{2 \omega_\Bp}}  \lr{
   e^{i (\omega_\Bp t - \Bp \cdot \Bx)} \hata_\Bp
+
   e^{-i (\omega_\Bp t - \Bp \cdot \Bx)} \hata^\dagger_\Bp
}
=
\int \frac{d^3 p}{(2 \pi)^3 \sqrt{2 \omega_\Bp}}  \evalbar{
\lr{
   e^{i p^\mu x^\mu } \hata_\Bp
+
   e^{-i p^\mu x^\mu } \hata^\dagger_\Bp
}
}{p_0 = \omega_\Bp}
\hatU(\Lambda) \phihat(x) \hatU^\dagger(\Lambda)
=
\phihat(\Lambda x)
=
\int \frac{d^3 p}{(2 \pi)^3 \sqrt{2 \omega_\Bp}}  \evalbar{
\lr{
   e^{i p^\mu \ultensor{\Lambda}{\mu}{\nu} x^\nu } \hata_\Bp
+
   e^{-i p^\mu \ultensor{\Lambda}{\mu}{\nu} x^\nu } \hata^\dagger_\Bp
}
}{p_0 = \omega_\Bp}
=
\int \frac{dp^0 d^3 p}{(2\pi)^3} \delta(p_0^2 - \Bp^2 - m^2) \Theta(p^0) \sqrt{2 \omega_\Bp} e^{...} \hata_\Bp + hc
\end{dmath}

using \( \delta(f(x)) = \sum_{f(x_\conj) = 0} \frac{\delta(x - x_\conj)}{f'(x_\conj)} \)

this is
\begin{dmath}\label{eqn:qftLecture10:360}
=
\int \frac{dp^0 d^3 p}{(2\pi)^3}
\lr{
\frac{\delta(p_0 - \omega_\Bp)}{2 \omega_\Bp}
+
\frac{\delta(p_0 + \omega_\Bp)}{2 \omega_\Bp}
}
\Theta(p^0) \sqrt{2 \omega_\Bp}  \hata_\Bp + hc
\end{dmath}

but the \( \Theta(p^0) \) kills the second delta function.

We now have a more explicit Lorentz invariant structure

\begin{dmath}\label{eqn:qftLecture10:380}
\phihat(\Lambda x)
=
\int \frac{dp^0 d^3 p}{(2\pi)^3} \delta(p_0^2 - \Bp^2 - m^2) \Theta(p^0) \sqrt{2 \omega_\Bp}  \hata_\Bp + hc
\end{dmath}

In \cref{fig:constantMomentumSurface:constantMomentumSurfaceFig1}, the paraboloid depict the surfaces of constant energy-momentum \( p^0 = \sqrt{ \Bp^2 + m^2 } \).  Lorentz transformations shift points in the energy-momentum space along the paraboloid, but cannot change the sign of the energy coordinate, so \( \Theta(p^0) \) is Lorentz invariant, since the sign of the energy \( p^0 \) does not change under such a transformation.
\imageFigure{../figures/phy2403-quantum-field-theory/constantMomentumSurfaceFig1}{Surface of constant squared four-momentum.}{fig:constantMomentumSurface:constantMomentumSurfaceFig1}{0.3}

Let's change variables

\begin{dmath}\label{eqn:qftLecture10:400}
p^\lambda = \ultensor{\Lambda}{\lambda}{\rho} {p'}^{\rho}
\end{dmath}

so that

\begin{dmath}\label{eqn:qftLecture10:420}
p_\mu
\ultensor{\Lambda}{\mu}{\nu} x^\nu
=
\ultensor{\Lambda}{\lambda}{\rho} {p'}^\rho g_{\lambda\nu} \ultensor{\Lambda}{\nu}{\sigma} x^{\sigma}
=
{p'}^\rho g_{\rho\sigma} x^\sigma
\end{dmath}

which gives

\begin{dmath}\label{eqn:qftLecture10:440}
\phihat(\Lambda x)
=
\int \frac{d{p'}^0 d^3 p'}{(2\pi)^3} \delta({p'}_0^2 - {\Bp'}^2 - m^2) \Theta(p^0) \sqrt{2 \omega_{\Lambda \Bp'}} e^{i \Bp' \cdot \Bx}  \hata_{\Lambda \Bp'} + hc
=
\int \frac{dp^0 d^3 p}{(2\pi)^3} \delta({p}_0^2 - {\Bp}^2 - m^2) \Theta(p^0) \sqrt{2 \omega_{\Lambda \Bp}} e^{i \Bp \cdot \Bx}  \hata_{\Lambda \Bp} + hc
\end{dmath}

but

\begin{dmath}\label{eqn:qftLecture10:460}
\phihat(\Lambda x)
=
\int \frac{dp^0 d^3 p}{(2\pi)^3} \delta({p}_0^2 - {\Bp}^2 - m^2) \Theta(p^0) \sqrt{2 \omega_{\Bp}} e^{i \Bp \cdot \Bx}  \hata_{\Bp} + hc
\end{dmath}

We can now conclude that
\begin{dmath}\label{eqn:qftLecture10:480}
\sqrt{2 \omega_{\Lambda \Bp}} \hata_{\Lambda \Bp}
=
\hatU(\Lambda)
\sqrt{2 \omega_{ \Bp}} \hata_{ \Bp}
\hatU^\dagger(\Lambda)
\end{dmath}

In particular
\begin{dmath}\label{eqn:qftLecture10:500}
\sqrt{2 \omega_{ \Bp}} \hata^\dagger_{ \Bp} \ket{0} = \ket{\Bp}
\end{dmath}
and noting that \( \hatU(\Lambda) \ket{0}  = \ket{0} \) (i.e. the ground state is Lorentz invariant), we have
\begin{dmath}\label{eqn:qftLecture10:n}
\sqrt{2 \omega_{\Lambda \Bp}} \hata^\dagger_{\Lambda \Bp} \ket{0}
=
\hatU(\Lambda) \sqrt{ 2\omega_\Bp} \hata^\dagger_\Bp \hatU^\dagger(\Lambda) \hatU(\Lambda) \ket{0}
=
\hatU(\Lambda) \sqrt{ 2\omega_\Bp} \hata^\dagger_\Bp \ket{0}
=
\hatU(\Lambda) \ket{\Bp}.
\end{dmath}

%}
%\EndArticle
\EndNoBibArticle
