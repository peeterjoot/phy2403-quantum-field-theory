%
% Copyright � 2018 Peeter Joot.  All Rights Reserved.
% Licenced as described in the file LICENSE under the root directory of this GIT repository.
%
\makeoproblem{Interaction energy between static external charges}{qft:problemSet3:1}{2018 Hw3.I}{
\makesubproblem{}{qft:problemSet3:1a}
 Calculate the vacuum expectation value of the time ordered exponential
\begin{equation}
\label{one}
\langle 0 \vert T e^{\;i \int d^4 x\;  g \; j(x)\;  \phi(x)} \vert 0 \rangle
\end{equation}
for the case of a massive free real scalar field. Here, $g$ is a coupling constant, which we shall call the ``Yukawa coupling". Show, e.g. using Wick's theorem, that the answer is
 \begin{equation}
\label{two}
  e^{\;-{g^2\over 2} \int d^4 x d^4 y\;  j(x)  D_F(x-y) j(y) } ~,
   \end{equation}
   which is really the exponential of the second order term and $D_F$ is the Feynman propagator.
\makesubproblem{}{qft:problemSet3:1b}
Consider the case where $j(t,\Bx) = \theta(T - t) \theta(T+t) \left( \delta^{(3)}(\Bx) - \delta^{(3)}(\Bx - \BR)\right)$. This source term represents two external opposite ``charges"\footnote{In other words, classical particles linearly coupled to $\phi$ (if $\phi$ was the electrostatic potential $A^0$, this would really be the electromagnetic charge.) For a discussion of whether an interaction like you will study can arise from a realistic QFT, see comment in 2. below.} a distance $R = |\BR|$ apart, created at $t=-T$ and existing for time $2 T$.  Show that, in the limit $T \gg R \gg 1/m$, eq.~(\ref{two}) {\it is proportional to}:
   \begin{equation}
   \label{three}
   e^{ - i 2T V(R)},
   \end{equation}
   where $V(R)$ is the Yukawa potential.

    {\flushleft{Hint:}} {\small Recall that $\lim\limits_{T \rightarrow \infty} \int\limits_{-T}^T d x e^{i p x} = 2 \pi \delta(p)$ as well as the usual relation $(2 \pi \delta(p))^2 = 2 \pi \delta(p) 2 T$.}

      {\small  {\flushleft{T}he} result (\ref{three}) means that ``{\it two static sources of scalar field a distance $R$ apart interact via the Yukawa potential}." This is because (\ref{three}) is  the evolution operator (it is  $ \sim e^{- i H t}$, for  $t=2T$)  of the field theory
   in the presence of the static external sources (or, more appropriately, (\ref{three}) is the contribution to the evolution operator that has to do with the interaction between the sources induced by the field). Thus, it is natural to call the quantity multiplying $- i 2 T$  and depending on $R$, the interaction potential $V(R)$ between the sources.}

 Do opposite-sign ``charges" attract or repel? How about same-sign?

 {\small  Notice that when the ``charges" are also considered as part of a QFT and, therefore, $j(x)$  in (\ref{one}) is replaced by an appropriate QFT expression, one finds more interesting results. Namely,  the Yukawa interaction between two fermions is always attractive---whether it is between two particles, two anti-particles, or between a particle and an anti-particle. The way to establish this, as well an alternative derivation of the expression for $V(R)$ you found in (\ref{three}), is to start  from the scattering of (anti)fermions via scalar exchange and then take the nonrelativistic limit. A comparison with quantum-mechanical Born scattering yields then an expression for $V(R)$.

 This result quoted above is of great interest in nuclear physics, where single-pion exchange   operates via $V(R)$, and turns out to be attractive between nucleons and between nucleons and anti-nucleons.  }

\makesubproblem{}{qft:problemSet3:1c}
What do you think is the significance of the various limits $T \gg R \gg 1/m$? Also, what is the  meaning of the factors you omitted upon going from (\ref{two}) to (\ref{three})?
} % makeproblem

\makeanswer{qft:problemSet3:1}{
\withproblemsetsParagraph{
\makeSubAnswer{}{qft:problemSet3:1a}
\begin{dmath}\label{eqn:ProblemSet3Problem1:20}
\bra{0} T e^{i \int d^4 x  g  j(x)  \phi(x)} \ket{0}
=
\bra{0} T (1) \ket{0}
+
i g
\bra{0} T \int d^4 x j(x) \phi(x) \ket{0}
+
- \frac{g^2}{2}
\int d^4 x d^4 y
\bra{0} T
\contraction{ j(x) }{\phi}{(x) j(y) }{\phi}
j(x) \phi(x) j(y) \phi(y)
\ket{0}
+ \cdots
\end{dmath}
Using Wick's theorem, the first order term is zero (odd number of creation and annihilation operators), so to first order, we have
\begin{dmath}\label{eqn:ProblemSet3Problem1:40}
\bra{0} T e^{i \int d^4 x  g  j(x)  \phi(x)} \ket{0}
=
1 - \frac{g^2}{2} \int d^4 x d^4 y j(x) D_F(x - y) j(y) + \cdots
\approx
\exp\lr{ - \frac{g^2}{2} \int d^4 x d^4 y j(x) D_F(x - y) j(y) }
\end{dmath}

\makeSubAnswer{}{qft:problemSet3:1b}
We wish to evaluate the integral in the exponential argument
\begin{dmath}\label{eqn:ProblemSet3Problem1:60}
\begin{aligned}
\int &d^4 x d^4 y j(x) D_F(x - y) j(y)  \\
&=
\int dt d^3 x dt' d^3 y  \\
%j(x)
&\quad \theta(T - t) \theta(T+t) \left( \delta^{(3)}(\Bx) - \delta^{(3)}(\Bx - \BR)\right)
D_F(\Bx - \By, t - t')
%j(y)
\theta(T - t') \theta(T+t') \left( \delta^{(3)}(\By) - \delta^{(3)}(\By - \BR)\right) \\
&=
\int_{-T}^T dt
\int_{-T}^T dt'
\int
d^3 x d^3 y
\left( \delta^{(3)}(\Bx) - \delta^{(3)}(\Bx - \BR)\right)
D_F(\Bx - \By, t - t')
\left( \delta^{(3)}(\By) - \delta^{(3)}(\By - \BR)\right) \\
&=
\int_{-T}^T dt
\int_{-T}^T dt'
\int
d^3 y
\lr{
   D_F(- \By, t - t')
   -
   D_F(\BR - \By, t - t')
}
\left( \delta^{(3)}(\By) - \delta^{(3)}(\By - \BR)\right) \\
&=
\int_{-T}^T dt
\int_{-T}^T dt'
\lr{
   D_F(\Bzero, t - t')
   -
   D_F(\BR, t - t')
-
   D_F(- \BR, t - t')
   +
   D_F(\Bzero, t - t')
} \\
&=
\int_{-T}^T dt
\int_{-T}^T dt'
\lr{
2  D_F(\Bzero, t - t')
   -
   D_F(\BR, t - t')
-
   D_F(- \BR, t - t')
}.
\end{aligned}
\end{dmath}
The propagator, written in space and time coordinates is
\begin{dmath}\label{eqn:ProblemSet3Problem1:80}
D_F(\Bx, t)
=
i \int \frac{dp_0 d^3 p}{(2 \pi)^4} \frac{e^{-i p_0 t} e^{i \Bp \cdot \Bx} }{p_0^2 - \Bp^2 - m^2 + i \epsilon },
\end{dmath}
so we have
\begin{dmath}\label{eqn:ProblemSet3Problem1:100}
\int d^4 x d^4 y j(x) D_F(x - y) j(y)
=
i
\int_{-T}^T dt
\int_{-T}^T dt'
\frac{dp_0 d^3 p}{(2 \pi)^4} \frac{e^{-i p_0 (t-t')} }{p_0^2 - \Bp^2 - m^2 + i \epsilon }
\lr{
2
-
e^{-i \Bp \cdot \BR}
-
e^{i \Bp \cdot \BR}
}
\end{dmath}
The time integrals can be done first
\begin{dmath}\label{eqn:ProblemSet3Problem1:120}
\int_{-T}^T dt
e^{-i p_0 t}
\int_{-T}^T dt'
e^{ i p_0 t}
=
(2 \pi)^2
\delta(-p_0)
\delta(p_0).
\end{dmath}
Following the supplied hint we write
\begin{dmath}\label{eqn:ProblemSet3Problem1:140}
(2 \pi)^2
\delta(p_0)
\delta(-p_0)
=
(2 \pi)^2
\delta(p_0)
\delta(0)
=
2 \pi
\delta(p_0)
\int_{-T}^T dt'
=
(2 \pi)(2 T)
\delta(p_0),
\end{dmath}
which gives
\begin{dmath}\label{eqn:ProblemSet3Problem1:160}
\int d^4 x d^4 y j(x) D_F(x - y) j(y)
=
2 T i
\int
\frac{dp_0 d^3 p}{(2 \pi)^3} \frac{ \delta(p_0) }{p_0^2 - \Bp^2 - m^2 + i \epsilon }
\lr{
2
-
e^{-i \Bp \cdot \BR}
-
e^{i \Bp \cdot \BR}
}
=
-2 T i
\int
\frac{d^3 p}{(2 \pi)^2} \frac{ 1 }{ \Bp^2 + m^2 - i \epsilon }
\lr{
2
-
e^{-i \Bp \cdot \BR}
-
e^{i \Bp \cdot \BR}
}.
\end{dmath}
We can now make the usual spherical coordinate change of variables
\begin{dmath}\label{eqn:ProblemSet3Problem1:180}
\begin{aligned}
d^3 p &= \calp^2 d\calp \sin\theta d\theta d\phi \\
\Bp &= \calp \lr{ \sin\theta \cos\phi, \sin\theta \sin\phi, \cos\theta } \\
\BR &= R\lr{ 0, 0, 1},
\end{aligned}
\end{dmath}
so the integral factor of the coupling exponential
\cref{two}
is reduced to
\begin{dmath}\label{eqn:ProblemSet3Problem1:200}
\int d^4 x d^4 y j(x) D_F(x - y) j(y)
=
-2 T i
\int_0^\infty d \calp \frac{\calp^2}{(2 \pi)^2} \int_0^\pi d\theta \sin\theta \frac{ 1 }{ \calp^2 + m^2 - i \epsilon }
\lr{
   2
   -
   e^{-i \calp R \cos\theta}
   -
   e^{i \calp R \cos\theta}
}
=
2 T i
\int_0^\infty d \calp \frac{\calp^2}{(2 \pi)^2} \int_1^{-1} du \frac{ 1 }{ \calp^2 + m^2 - i \epsilon }
\lr{
   2
   -
   e^{-i \calp R u}
   -
   e^{i \calp R u}
}
=
2 T i
\int_0^\infty d \calp \frac{\calp^2}{(2 \pi)^2} \frac{ 1 }{ \calp^2 + m^2 - i \epsilon }
\lr{
   2 (-2)
   -
   \frac{
      e^{i \calp R }
      -
      e^{-i \calp R }
   }
   {
      -i \calp R
   }
   -
   \frac{
      e^{-i \calp R }
      -
      e^{i \calp R }
   }
   {
      i \calp R
   }
}
=
\frac{2 T i}{\pi^2}
\int_0^\infty d \calp \frac{ \calp^2 }{ \calp^2 + m^2 - i \epsilon }
\lr{
   \sinc( \calp R ) - 1
}.
\end{dmath}
With \( p_0 \) integrated out, we don't need the \( i \epsilon \) factor for pole avoidance, and find (using Mathematica) that
\begin{dmath}\label{eqn:ProblemSet3Problem1:220}
-\int_0^\infty d \calp \frac{ \calp^2 }{ \calp^2 + m^2 }
\lr{
   1 \mp \sinc( \calp R )
}
=
\frac{\pi}{2} \lr{ m \pm \frac{e^{-m R}}{R} }.
\end{dmath}
The \( m \pi/2 \) term contributes only a phase adjustment, and can be ignored.  This leaves
\begin{dmath}\label{eqn:ProblemSet3Problem1:240}
  e^{\;-{g^2\over 2} \int d^4 x d^4 y\;  j(x)  D_F(x-y) j(y) }
=
\exp\lr{
\frac{-g^2}{2}
\frac{2 T i}{\pi^2}
\frac{1}{2} \pi \frac{e^{-m R}}{R}
}
=
\exp\lr{
   -2 T i V(R)
},
\end{dmath}
where
\begin{dmath}\label{eqn:ProblemSet3Problem1:260}
V(R)
=
%-
\frac{g^2}{4 \pi} \frac{e^{-m R}}{R}.
\end{dmath}
which is a positive (repulsive) variation of the Yukawa potential as defined in \citep{peskin1995introduction} (eq. 4.127).
%Comparing to \citep{wiki:yukawaPotential}, this is off by a factor of \( 4 \pi \).  There are a few possible explainations for that
%\begin{enumerate}
%\item Typo in the problem statement.
%\item The constant \( 4 \pi \) scaling factor wasn't considered interesting, and was ignored in the problem statement.
%\item Mistake in my work above.
%\item There are different conventions for \( V(R) \), and the desired form includes a factor of \( 4 \pi \) downstairs.
%\end{enumerate}
%\begin{mmaCell}[morefunctionlocal={p}]{Input}
%  ClearAll[r, m, e]
%  e = Integrate[(Sinc[p r] - 1)p^2/(p^2 + m^2), {p, 0, Infinity}];
%  Assuming[ r > 0 && m > 0, Simplify[e]]
%\end{mmaCell}
%\begin{mmaCell}{Output}
%  \mmaFrac{1}{2} $\pi$ (m+\mmaFrac{\mmaSup{e}{-m r}}{r})
%\end{mmaCell}
%The negative potential indicates that opposite charges attract.

\paragraph{Like charges.}

For like charges the modification of \cref{eqn:ProblemSet3Problem1:60} is
\begin{dmath}\label{eqn:ProblemSet3Problem1:280}
\begin{aligned}
\int &d^4 x d^4 y j(x) D_F(x - y) j(y)  \\
&=
\int_{-T}^T dt
\int_{-T}^T dt'
\int
d^3 x d^3 y
\left( \delta^{(3)}(\Bx) + \delta^{(3)}(\Bx - \BR)\right)
D_F(\Bx - \By, t - t')
\left( \delta^{(3)}(\By) + \delta^{(3)}(\By - \BR)\right) \\
&=
\int_{-T}^T dt
\int_{-T}^T dt'
\int
d^3 y
\lr{
   D_F(- \By, t - t')
   +
   D_F(\BR - \By, t - t')
}
\left( \delta^{(3)}(\By) + \delta^{(3)}(\By - \BR)\right) \\
&=
\int_{-T}^T dt
\int_{-T}^T dt'
\lr{
   D_F(\Bzero, t - t')
   +
   D_F(\BR, t - t')
+
   D_F(- \BR, t - t')
   +
   D_F(\Bzero, t - t')
} \\
&=
\int_{-T}^T dt
\int_{-T}^T dt'
\lr{
   2 D_F(\Bzero, t - t')
   +
   D_F(\BR, t - t')
+
   D_F(- \BR, t - t')
}.
\end{aligned}
\end{dmath}
Applying this sign adjustment to the calculation of \cref{eqn:ProblemSet3Problem1:200} we find
\begin{dmath}\label{eqn:ProblemSet3Problem1:300}
\int d^4 x d^4 y j(x) D_F(x - y) j(y)
=
-\frac{2 T i}{\pi^2}
\int_0^\infty d \calp \frac{ \calp^2 }{ \calp^2 + m^2 - i \epsilon }
\lr{
   \sinc( \calp R ) + 1
}.
\end{dmath}
Using \cref{eqn:ProblemSet3Problem1:220} again, we find for equal charges an opposite sign potential
\begin{dmath}\label{eqn:ProblemSet3Problem1:320}
V(R)
=
-\frac{g^2}{4 \pi} \frac{e^{-m R}}{R}.
\end{dmath}
As this is a negative potential, it appears to indicate that like charges attract in the scalar theory.

\makeSubAnswer{}{qft:problemSet3:1c}
I did not have any reason to utilize the \( T \gg R \gg 1/m \) limits in the derivation above (unless neglecting higher powers of \( g \) relies on such a limit implicitly).  However, these limits do effect the form of the potential and the resulting matrix element.

In the \( T \gg R \) limit the exponential dies off more slowly, as illustrated in \cref{fig:ps3p1TR:ps3p1TRFig1}.  This means that the potential acts more strongly away from the origin than for \( T \sim R \).
\mathImageFigure{../figures/phy2403-quantum-field-theory/ps3p1TRFig1}{Plots of \( T e^{-mR}/R \).}{fig:ps3p1TR:ps3p1TRFig1}{0.3}{ps3p1TRFig1.nb}

For \( R \gg 1/m \), or \( m R \gg 1 \) we have \( e^{-m R} \sim 0 \).  In this limit we have no interaction, as the potential is effectively zero for all \( R \).
%Given both \( T \gg 1/m \) we have
%\begin{dmath}\label{eqn:ProblemSet3Problem1:340}
%-2 i T V(R)
%=
%+2 i T \frac{g^2}{4 \pi} e^{-m R}/R
%\approx
%i T \frac{g^2}{2 \pi} \lr{ \inv{R} - m }
%\approx
%- i T m \frac{g^2}{2 \pi},
%\end{dmath}
%In this case, we also have no interaction to speak of, as the radial dependence is eliminated.

We omitted a factor of $\pi m/2$, which adds a pure phase factor
\begin{dmath}\label{eqn:ProblemSet3Problem1:360}
e^{-2 i T \frac{g^2}{4 \pi} \frac{\pi m}{2} }
=
e^{- g^2 i T m / 4 }.
\end{dmath}
We can use the time ordered exponential to compute the probability that there is no scattering (as in \cref{qft:problemSet3:2}), but the amplitude squared operation required for that probability kills such a phase factor, so it does not seem physically meaningful.
}
}
