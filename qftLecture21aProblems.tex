%
% Copyright © 2018 Peeter Joot.  All Rights Reserved.
% Licenced as described in the file LICENSE under the root directory of this GIT repository.
%
%{

\makeproblem{Verify \( v(p) \) solution.}{problem:qftLecture21:1}{
Prove \cref{thm:qftLecture21:101}.
%Show that \cref{eqn:qftLecture21:280} is a solution of the Dirac equation.
} % problem

\makeanswer{problem:qftLecture21:1}{
Let \( D = \lr{ i \gamma^\mu \partial_\mu - m } \) represent the Dirac operator.  Applying to \( e^{i p \cdot x} \) we have
\begin{equation}\label{eqn:qftLecture21:1040}
\begin{aligned}
D e^{i p \cdot x}
&=
\lr{ i \gamma^\mu \partial_\mu - m } e^{i p_\mu x^\mu}
\\&=
-\lr{ \gamma^\mu p_\mu + m } e^{i p \cdot x}
\\&=
-\lr{
   m \ITwo
   + p_0
   \DiracGammaZero
   + p_k
   \DiracGammaK{k}
}
e^{i p \cdot x}
\\&=
-
\begin{bmatrix}
m & p_0 \sigma^0 + p_k \sigma^k \\
p_0 \sigma^0 - p_k \sigma^k & m
\end{bmatrix}
e^{i p \cdot x}
\\&=
-
\begin{bmatrix}
m & p \cdot \sigma \\
p \cdot \osigma & m
\end{bmatrix}
e^{i p \cdot x}.
\end{aligned}
\end{equation}
We are now set to apply the Dirac operator to the claimed solution from \cref{thm:qftLecture21:101}.
%\cref{eqn:qftLecture21:280}
\begin{equation}\label{eqn:qftLecture21:1060}
\begin{aligned}
D v(p)
&=
\begin{bmatrix}
m & p \cdot \sigma \\
p \cdot \osigma & m
\end{bmatrix}
\begin{bmatrix}
\sqrt{p \cdot \sigma} \eta^s \\
-\sqrt{p \cdot \osigma} \eta^s \\
\end{bmatrix},
e^{i p \cdot x}
\\&=
-
\begin{bmatrix}
\lr{ m \sqrt{ p \cdot \sigma} - p \cdot \sigma \sqrt{ p \cdot \osigma } } \eta \\
\lr{ p \cdot \osigma \sqrt{ p \cdot \sigma } -m \sqrt{ p \cdot \osigma } } \eta \\
\end{bmatrix}
e^{i p \cdot x}
\\&=
\begin{bmatrix}
\sqrt{ p \cdot \sigma} \lr{ m - \sqrt{ p \cdot \sigma  p \cdot \osigma } } \eta \\
\sqrt{ p \cdot \osigma} \lr{ \sqrt{ p \cdot \osigma p \cdot \sigma } -m } \eta \\
\end{bmatrix}
e^{i p \cdot x}
\\&=
\begin{bmatrix}
\sqrt{ p \cdot \sigma} \lr{ m - \sqrt{ m^2 } } \eta \\
\sqrt{ p \cdot \osigma} \lr{ \sqrt{ m^2 } -m } \eta \\
\end{bmatrix}
e^{i p \cdot x}
\\&=
0.
\end{aligned}
\end{equation}
} % answer

\makeproblem{\(v(p)\) normalization.}{problem:qftLecture21:2}{
Prove \cref{thm:qftLecture21:13}.
} % problem

\makeanswer{problem:qftLecture21:2}{
Expanding the matrices gives
\begin{equation}\label{eqn:qftLecture21:1080}
\begin{aligned}
\vbar^r v^s
&= v^{r\dagger} \gamma^0 v^s
\\&=
\begin{bmatrix}
\eta^{r \T} \sqrt{ p \cdot \sigma} & - \eta^{r \T} \sqrt{ p \cdot \osigma}
\end{bmatrix}
\DiracGammaZero
\begin{bmatrix}
\sqrt{ p \cdot \sigma } \eta^s \\
-\sqrt{ p \cdot \osigma } \eta^s
\end{bmatrix}
\\&=
\begin{bmatrix}
\eta^{r \T} \sqrt{ p \cdot \sigma} & - \eta^{r \T} \sqrt{ p \cdot \osigma}
\end{bmatrix}
\begin{bmatrix}
-\sqrt{ p \cdot \osigma } \eta^s \\
\sqrt{ p \cdot \sigma } \eta^s \\
\end{bmatrix}
\\&=
-\eta^{r \T} \sqrt{ p \cdot \sigma} \sqrt{ p \cdot \osigma } \eta^s
- \eta^{r \T} \sqrt{ p \cdot \osigma} \sqrt{ p \cdot \sigma } \eta^s
\\&= \delta^{rs} 2 \sqrt{ m^2 }
\\&= 2 m \delta^{rs},
\end{aligned}
\end{equation}
and
\begin{equation}\label{eqn:qftLecture21:1100}
\begin{aligned}
v^{r\dagger} v^s
&=
\begin{bmatrix}
\eta^{r \T} \sqrt{ p \cdot \sigma} & - \eta^{r \T} \sqrt{ p \cdot \osigma}
\end{bmatrix}
\begin{bmatrix}
\sqrt{ p \cdot \sigma } \eta^s \\
-\sqrt{ p \cdot \osigma } \eta^s
\end{bmatrix}
\\&=
\eta^{r \T} (p \cdot \sigma) \eta^s
+ \eta^{r \T} (p \cdot \osigma) \eta^s
\\&= \delta^{rs} \lr{ p_0 - \Bp \cdot \Bsigma + p_0 + \Bp \cdot \Bsigma }
\\&= 2 p_0 \delta^{rs}.
\end{aligned}
\end{equation}
} % answer

\makeproblem{\( \ubar v, \vbar u \) relations.}{problem:qftLecture21:4}{
Prove \cref{thm:qftLecture21:1260}.
} % problem

\makeanswer{problem:qftLecture21:4}{
We need only expand the matrix products
\begin{equation}\label{eqn:qftLecture21:1220}
\begin{aligned}
\ubar^r v^s
&=
\begin{bmatrix}
\zeta^{r\T} \sqrt{ p \cdot \osigma } & \zeta^{r\T} \sqrt{ p \cdot \sigma }
\end{bmatrix}
\begin{bmatrix}
\sqrt{ p \cdot \sigma } \eta^s  \\
-\sqrt{ p \cdot \osigma } \eta^s
\end{bmatrix}
\\&=
m \zeta^{r\T} \eta^s
- m \zeta^{r\T} \eta^s
\\&= 0,
\end{aligned}
\end{equation}
and
\begin{equation}\label{eqn:qftLecture21:1240}
\begin{aligned}
\vbar^r u^s
&=
\begin{bmatrix}
-\eta^{r\T} \sqrt{ p \cdot \osigma } & \eta^{r\T} \sqrt{ p \cdot \sigma }
\end{bmatrix}
\begin{bmatrix}
\sqrt{ p \cdot \sigma } \zeta^s  \\
\sqrt{ p \cdot \osigma } \zeta^s
\end{bmatrix}
\\&=
- m \eta^{r\T} \zeta^s
+ m \eta^{r\T} \zeta^s
\\&= 0,
\end{aligned}
\end{equation}
} % answer

\makeproblem{Dagger orthonormality conditions.}{problem:qftLecture21:5}{
Prove \cref{thm:qftLecture21:1261}.
} % problem

\makeanswer{problem:qftLecture21:5}{
\begin{equation}\label{eqn:qftLecture21:1260}
\begin{aligned}
u^{r\dagger}(\Bp)
v^{s}(-\Bp)
&=
\evalbar{
\begin{bmatrix}
\zeta^{r\T} \sqrt{ p \cdot \sigma } &\zeta^{r\T} \sqrt{ p \cdot \osigma }
\end{bmatrix}
\begin{bmatrix}
\sqrt{ q \cdot \sigma } \eta^s \\
-\sqrt{ q \cdot \osigma } \eta^s
\end{bmatrix}
}{p = (0, \Bp), q = (0, -\Bp)}
\\&=
\begin{bmatrix}
\zeta^{r\T} \sqrt{ -\Bp \cdot \Bsigma } &\zeta^{r\T} \sqrt{ \Bp \cdot \Bsigma }
\end{bmatrix}
\begin{bmatrix}
\sqrt{ \Bp \cdot \Bsigma } \eta^s \\
-\sqrt{ -\Bp \cdot \Bsigma } \eta^s
\end{bmatrix}
\\&=
\zeta^{r\T} \sqrt{ -\Bp \cdot \Bsigma } \sqrt{ \Bp \cdot \Bsigma } \eta^s
-
\zeta^{r\T} \sqrt{ \Bp \cdot \Bsigma } \sqrt{ -\Bp \cdot \Bsigma } \eta^s
%=
%\delta^{rs} \sqrt{ -(\Bp \cdot \Bsigma)^2 }
%-
%\delta^{rs} \sqrt{ -(\Bp \cdot \Bsigma)^2 }
\\&=
0,
\end{aligned}
\end{equation}
and
\begin{equation}\label{eqn:qftLecture21:1280}
\begin{aligned}
v^{s\dagger}(-\Bp)
u^{r}(\Bp)
&=
\evalbar{
\begin{bmatrix}
\eta^{s\dagger}
\sqrt{ q \cdot \sigma } &
-
\eta^{s\dagger}
\sqrt{ q \cdot \osigma }
\end{bmatrix}
\begin{bmatrix}
\sqrt{ p \cdot \sigma } \zeta^r \\
\sqrt{ p \cdot \osigma }\zeta^r
\end{bmatrix}
}{p = (0, \Bp), q = (0, -\Bp)}
\\&=
\begin{bmatrix}
\eta^{s\T} \sqrt{ \Bp \cdot \Bsigma } & -\eta^{s} \sqrt{ -\Bp \cdot \Bsigma }
\end{bmatrix}
\begin{bmatrix}
\sqrt{ -\Bp \cdot \Bsigma } \zeta^r \\
\sqrt{ \Bp \cdot \Bsigma } \zeta^r
\end{bmatrix}
\\&=
\eta^{s\T} \sqrt{ (-\Bp \cdot \Bsigma)(\Bp \cdot \Bsigma) } \zeta^r
-
\eta^{s\T} \sqrt{ (\Bp \cdot \Bsigma)(-\Bp \cdot \Bsigma) } \zeta^r
\\&=
0.
\end{aligned}
\end{equation}
} % answer
\makeproblem{Direct product relation for the \(u\)'s.}{problem:qftLecture21:3}{
Prove the \( u \) direct product relations of \cref{thm:qftLecture21:17}.
} % problem
\makeanswer{problem:qftLecture21:3}{
\begin{equation}\label{eqn:qftLecture21:1180}
\begin{aligned}
\sum u^s \otimes \ubar^s
&=
\begin{bmatrix}
\sqrt{ p \cdot \sigma } \zeta^s \\
\sqrt{ p \cdot \osigma } \zeta^s
\end{bmatrix}
\otimes
\begin{bmatrix}
\zeta^{s\T} \sqrt{ p \cdot \osigma } &
\zeta^{s\T} \sqrt{ p \cdot \sigma }
\end{bmatrix}
\\&=
\sum
\begin{bmatrix}
\sqrt{ p \cdot \sigma } \zeta^s \otimes \zeta^{s\T} \sqrt{ p \cdot \osigma } & \sqrt{ p \cdot \sigma } \zeta^s \otimes \zeta^{s\T} \sqrt{ p \cdot \sigma } \\
\sqrt{ p \cdot \osigma } \zeta^s \otimes \zeta^{s\T} \sqrt{ p \cdot \osigma } & \sqrt{ p \cdot \osigma } \zeta^s \otimes \zeta^{s\T} \sqrt{ p \cdot \sigma } \\
\end{bmatrix}
\\&=
\begin{bmatrix}
m & p \cdot \sigma \\
p \cdot \osigma & m
\end{bmatrix}
\\&=
m + p \cdot \gamma.
\end{aligned}
\end{equation}
} % answer
%}
