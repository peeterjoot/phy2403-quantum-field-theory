%
% Copyright � 2018 Peeter Joot.  All Rights Reserved.
% Licenced as described in the file LICENSE under the root directory of this GIT repository.
%
%\input{../latex/blogpost.tex}
%\renewcommand{\basename}{qftLecture3}
%\renewcommand{\dirname}{notes/phy2403/}
%\newcommand{\keywords}{PHY2403H}
%\input{../latex/peeter_prologue_print2.tex}
%
%%%\usepackage{phy2403}
%%\usepackage{peeters_braket}
%%%\usepackage{peeters_layout_exercise}
%%\usepackage{peeters_figures}
%%\usepackage{mathtools}
%%\usepackage{siunitx}
%%\usepackage{macros_cal}
%%
%%\beginArtNoToc
%%\generatetitle{PHY2403H Quantum Field Theory.  Lecture 3: Lorentz transformations and a scalar action.  Taught by Prof.\ Erich Poppitz}
%\chapter{Lorentz transformations and a scalar action.}
\index{Lorentz transformations}
\index{action}
\label{chap:qftLecture3}

%%\paragraph{DISCLAIMER: Very rough notes from class.  Some additional side notes, but otherwise barely edited.}
%%
%%These are notes for the UofT course PHY2403H, Quantum Field Theory, taught by Prof. Erich Poppitz.
%%%, covering \textchapref{{1}} \citep{peskin1995introduction} content.
%%
\section{Determinant of Lorentz transformations.}
\index{Lorentz transformation!determinant}
%SO(1,d-1),d=3

We require that Lorentz transformations leave the dot product invariant, that is \( x \cdot y = x' \cdot y' \), or
\begin{dmath}\label{eqn:qftLecture3:20}
x^\mu g_{\mu\nu} y^\nu = {x'}^\mu g_{\mu\nu} {y'}^\nu.
\end{dmath}
Explicitly, with coordinate transformations
\begin{dmath}\label{eqn:qftLecture3:40}
\begin{aligned}
{x'}^\mu &= {\Lambda^\mu}_\rho x^\rho \\
{y'}^\mu &= {\Lambda^\mu}_\rho y^\rho
\end{aligned}
\end{dmath}
such a requirement is equivalent to demanding that
\begin{dmath}\label{eqn:qftLecture3:500}
x^\mu g_{\mu\nu} y^\nu
=
{\Lambda^\mu}_\rho x^\rho
g_{\mu\nu}
{\Lambda^\nu}_\kappa y^\kappa
=
x^\mu
{\Lambda^\alpha}_\mu
g_{\alpha\beta}
{\Lambda^\beta}_\nu
y^\nu,
\end{dmath}
or
\begin{dmath}\label{eqn:qftLecture3:60}
g_{\mu\nu}
%= {\Lambda^k}_\mu g_{\kappa \rho} {\Lambda^\rho}_\nu
=
{\Lambda^\alpha}_\mu
g_{\alpha\beta}
{\Lambda^\beta}_\nu
\end{dmath}

multiplying by the inverse we find
\begin{dmath}\label{eqn:qftLecture3:200}
g_{\mu\nu}
{\lr{\Lambda^{-1}}^\nu}_\lambda
=
%{\Lambda^\kappa}_\mu g_{\kappa \rho} {\Lambda^\rho}_\nu
{\Lambda^\alpha}_\mu
g_{\alpha\beta}
{\Lambda^\beta}_\nu
{\lr{\Lambda^{-1}}^\nu}_\lambda
=
%{\Lambda^\kappa}_\mu g_{\kappa \lambda}
{\Lambda^\alpha}_\mu
g_{\alpha\lambda}
=
g_{\lambda\alpha}
{\Lambda^\alpha}_\mu.
\end{dmath}
This is now amenable to expressing in matrix form
\begin{dmath}\label{eqn:qftLecture3:220}
(G \Lambda^{-1})_{\mu\lambda}
=
(G \Lambda)_{\lambda\mu}
=
((G \Lambda)^\T)_{\mu\lambda}
=
(\Lambda^\T G)_{\mu\lambda},
\end{dmath}
or
\begin{dmath}\label{eqn:qftLecture3:80}
G \Lambda^{-1}
=
(G \Lambda)^\T.
\end{dmath}

Taking determinants (using the normal identities for products of determinants, determinants of transposes and inverses), we find
\begin{dmath}\label{eqn:qftLecture3:100}
\cancel{det(G) }
det(\Lambda^{-1})
=
\cancel{det(G)} det(\Lambda),
\end{dmath}
or
\begin{dmath}\label{eqn:qftLecture3:120}
det(\Lambda)^2 = 1,
\end{dmath}
or
\( det(\Lambda)^2 = \pm 1 \).  We will generally ignore the case of reflections in spacetime that have a negative determinant.

Smart-alec Peeter pointed out after class last time that we can do the same thing easier in matrix notation
\begin{dmath}\label{eqn:qftLecture3:140}
\begin{aligned}
x' &= \Lambda x \\
y' &= \Lambda y
\end{aligned}
\end{dmath}
where
\begin{dmath}\label{eqn:qftLecture3:160}
x' \cdot y'
=
(x')^\T G y'
=
x^\T \Lambda^\T G \Lambda y,
\end{dmath}
which we require to be \( x \cdot y = x^\T G y \) for all four vectors \( x, y \), that is
\begin{dmath}\label{eqn:qftLecture3:180}
\Lambda^\T G \Lambda = G.
\end{dmath}
We can find the result \cref{eqn:qftLecture3:120} immediately without having to first translate from index notation to matrices.

%\EndArticle
