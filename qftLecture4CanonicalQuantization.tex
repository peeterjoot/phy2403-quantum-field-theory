%
% Copyright © 2018 Peeter Joot.  All Rights Reserved.
% Licenced as described in the file LICENSE under the root directory of this GIT repository.
%
%{
\section{Canonical quantization.}
\index{canonical quantization}

The harmonic oscillator described by
\begin{equation}\label{eqn:qftLecture4:740}
\LL = \inv{2} \dot{q}^2 - \frac{\omega^2}{2} q^2,
\end{equation}
which has solution \(\ddot{q} = - \omega^2 q\).
With
\begin{equation}\label{eqn:qftLecture4:760}
p = \PD{\dot{q}}{\LL} = \dot{q},
\end{equation}
the Hamiltonian is given by
\begin{equation}\label{eqn:qftLecture4:780}
\begin{aligned}
H(p,q)
&= \evalbar{p \dot{q} - \LL}{\dot{q}(p, q)} \\
&= p p - \inv{2} p^2 + \frac{\omega^2}{2} q^2 \\
&= \frac{p^2}{2} + \frac{\omega^2}{2} q^2.
\end{aligned}
\end{equation}

In QM we quantize by mapping Poisson brackets to commutators.
\begin{equation}\label{eqn:qftLecture4:800}
\antisymmetric{\hatp}{\hatq} = -i
\end{equation}
One way to represent is to say that states are \( \Psi(\hatq) \), a wave function, \( \hatq \) acts by \( q \)
\begin{equation}\label{eqn:qftLecture4:820}
\hatq \Psi = q \Psi(q)
\end{equation}
With
\begin{equation}\label{eqn:qftLecture4:840}
\hatp = -i \PD{q}{},
\end{equation}
so
\begin{equation}\label{eqn:qftLecture4:860}
\antisymmetric{ -i \PD{q}{} } { q} = -i
\end{equation}
\paragraph{Returning to the field Lagrangian.}
Let's introduce an explicit space time split.  We'll write
\begin{equation}\label{eqn:qftLecture4:880}
L = \int d^3 x \lr{
\inv{2} (\partial_0 \phi(\Bx, t))^2 - \inv{2} \lr{ \spacegrad \phi(\Bx, t) }^2 - \frac{m^2}{2} \phi^2
},
\end{equation}
so that the action is
\begin{equation}\label{eqn:qftLecture4:900}
S = \int dt L.
\end{equation}
The dynamical variables are \( \phi(\Bx) \).  We define
\begin{equation}\label{eqn:qftLecture4:920}
\begin{aligned}
\pi(\Bx, t)
&= \frac{\delta L}{\delta (\partial_0 \phi(\Bx, t))} \\
&= \partial_0 \phi(\Bx, t) \\
&= \dot{\phi}(\Bx, t),
\end{aligned}
\end{equation}
called the canonical momentum, or the momentum conjugate to \( \phi(\Bx, t) \).
Why \( \delta \)?  Has to do with an implicit Dirac function to eliminate the integral?

\begin{equation}\label{eqn:qftLecture4:940}
\begin{aligned}
H
&= \int d^3 x \evalbar{\lr{ \pi(\bar{\Bx}, t) \dot{\phi}(\bar{\Bx}, t) - L }}{\dot{\phi}(\bar{\Bx}, t) = \pi(x, t) } \\
&= \int d^3 x \lr{ (\pi(\Bx, t))^2 - \inv{2} (\pi(\Bx, t))^2 + \inv{2} (\spacegrad \phi)^2 + \frac{m^2}{2} \phi^2 },
\end{aligned}
\end{equation}
or
\begin{equation}\label{eqn:qftLecture4:960}
H = \int d^3 x \lr{ \inv{2} (\pi(\Bx, t))^2 + \inv{2} (\spacegrad \phi(\Bx, t))^2 + \frac{m^2}{2} (\phi(\Bx, t))^2 }
\end{equation}

In analogy to the momentum, position commutator in QM
\begin{equation}\label{eqn:qftLecture4:1000}
\antisymmetric{\hat{p}_i}{\hat{q}_j} = -i \delta_{ij},
\end{equation}
we ``quantize'' the scalar field theory by promoting \( \pi, \phi \) to operators and insisting that they also obey a commutator relationship
\begin{equation}\label{eqn:qftLecture4:980}
\antisymmetric{\pi(\Bx, t)}{\phi(\By, t)} = -i \deltathree(\Bx - \By).
\end{equation}
Note that in this commutator, the fields are evaluated at different spatial points, but at the same time.
%}
