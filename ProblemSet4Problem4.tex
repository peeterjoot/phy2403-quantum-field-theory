%
% Copyright � 2018 Peeter Joot.  All Rights Reserved.
% Licenced as described in the file LICENSE under the root directory of this GIT repository.
%
\makeproblem{Lorentz transforms of spinors---some useful identities}{qft:problemSet4:4}{

{\flushleft{C}}onsider the matrix $$\Lambda_{1\over 2} = e^{- {i\over 2} \omega_{\mu\nu} S^{\mu\nu}}~.$$
Here, $S^{\mu\nu} = {i\over 4} [\gamma^\mu, \gamma^\nu ]$ is as defined in class, in terms of the four $\gamma$-matrices (notice that, when using the representation of the $\gamma$ matrices in terms of Pauli matrices, the matrix $\Lambda_{1\over 2}$ looks like two sets of $M$ (and $M^*$) matrices discussed in class, now combined into one four-by-four object).

\makesubproblem{}{qft:problemSet4:4a}
Show that $\Lambda_{1\over 2}^{-1} \gamma^\mu \Lambda_{1\over 2} = \Lambda^\mu_{\; \nu} \gamma^\nu$, where $ \Lambda^\mu_{\; \nu}$ is the usual Lorentz transformation acting on vectors. (Feel free to show this for the infinitesimal form of the transformations, but then argue that the finite form holds as well.)
\makesubproblem{}{qft:problemSet4:4b}
Show that $\Lambda_{1\over 2}^{\dagger} \gamma^0 \Lambda_{1\over 2} =  \gamma^0$.
\makesubproblem{}{qft:problemSet4:4c}
Consider the fermion bilinear $\bar\psi \gamma^\mu \gamma^\nu \psi ={1\over 2} \bar\psi \{ \gamma^\mu ,\gamma^\nu \} \psi + {1\over 2} \bar\psi [\gamma^\mu, \gamma^\nu] \psi$, where $\{A,B\} = AB + BA$ is the anticommutator. Show that the two terms on the right transform as a scalar and a second-rank tensor, respectively, under Lorentz transformations.
} % makeproblem

\makeanswer{qft:problemSet4:4}{
\makeSubAnswer{}{qft:problemSet4:4a}
For infinesimal transformations we can show this using the BCH theorem.  First let
\begin{dmath}\label{eqn:ProblemSet4Problem4:20}
B
= -\frac{i}{2} \omega_{\mu\nu} S^{\mu\nu}
= -\frac{i}{2} \omega_{\mu\nu} \frac{i}{4} \antisymmetric{\gamma^\mu}{\gamma^\nu}
= \frac{1}{8} \omega_{\mu\nu} \lr{ \gamma^\mu \gamma^\nu - \gamma^\nu \gamma^\mu}.
= \frac{1}{4} \omega_{\mu\nu} \gamma^\mu \gamma^\nu,
\end{dmath}
where we've made use
\( \omega_{\mu\nu} = - \omega_{\nu\mu} \) to eliminate any \( \mu = \nu \) terms in the sum,
and \( \gamma^\mu \gamma^\nu = -\gamma^\nu \gamma^\mu \) for \( \mu \ne \nu \).
With this substitution, we have
\begin{dmath}\label{eqn:ProblemSet4Problem4:40}
\Lambda^{-1}_{1/2}
\gamma^\mu
\Lambda_{1/2}
=
e^{-B}
\gamma^\mu
e^{B}
=
\gamma^\mu
+
\antisymmetric{-B}{\gamma^\mu}
+
O(\omega^2),
\end{dmath}
and can now compute the commutator
\begin{dmath}\label{eqn:ProblemSet4Problem4:60}
\antisymmetric{-B}{\gamma^\mu}
=
-\frac{1}{4} \omega_{\alpha\beta}
\antisymmetric{
\gamma^\alpha \gamma^\beta
}{\gamma^\mu}
=
-\frac{1}{4} \omega^{\alpha\beta}
\antisymmetric{
\gamma_\alpha \gamma_\beta
}{\gamma^\mu}
=
\frac{1}{4} \omega_{\alpha\beta}
\lr{
   \gamma^\mu \gamma_\alpha \gamma^\beta
   -
   \gamma_\alpha \gamma_\beta \gamma^\mu
}.
\end{dmath}
When \( \mu \ne \alpha, \beta \), \( \gamma^\mu \) commutes with both \( \gamma_\alpha,  \gamma_\beta \),
so the matrices cancel, leaving just the \( \mu = \alpha, \mu = \beta \) contributions to the sum
\begin{dmath}\label{eqn:ProblemSet4Problem4:80}
\antisymmetric{-B}{\gamma^\mu}
=
\frac{1}{4} \omega_{\alpha\beta}
\lr{
   \ultensor{\delta}{\mu}{\alpha} \gamma^\beta
   -\ultensor{\delta}{\mu}{\beta} \gamma^\alpha
   - \gamma^\alpha \ultensor{\delta}{\mu}{\beta}
   + \gamma^\beta \ultensor{\delta}{\mu}{\alpha}
}
=
\frac{1}{2} \omega_{\alpha\beta}
\lr{
   \ultensor{\delta}{\mu}{\alpha} \gamma^\beta
   -\ultensor{\delta}{\mu}{\beta} \gamma^\alpha
}
=
\frac{1}{2}
\lr{
   \omega_{\alpha\beta}
   \ultensor{\delta}{\mu}{\alpha} \gamma^\beta
   -
   \omega_{\beta\alpha}
   \ultensor{\delta}{\mu}{\alpha} \gamma^\beta
}
=
   \omega_{\alpha\beta}
   \ultensor{\delta}{\mu}{\alpha} \gamma^\beta
=
   \omega^{\mu\beta}
\gamma_\beta.
\end{dmath}
The transformation to first order is therefore
\begin{dmath}\label{eqn:ProblemSet4Problem4:100}
\Lambda^{-1}_{1/2}
\gamma^\mu
\Lambda_{1/2}
=
\gamma^\mu
+
   \omega^{\mu\beta}
\gamma_\beta
=
\lr{
   \ultensor{\delta}{\mu}{\beta}
+
   \ultensor{\omega}{\mu}{\beta}
}
\gamma^\beta
=
\ultensor{\Lambda}{\mu}{\beta} \gamma^\beta.
\end{dmath}

\makeSubAnswer{}{qft:problemSet4:4b}
TODO.
\makeSubAnswer{}{qft:problemSet4:4c}
TODO.
}
