%
% Copyright � 2018 Peeter Joot.  All Rights Reserved.
% Licenced as described in the file LICENSE under the root directory of this GIT repository.
%
\makeoproblem{Lorentz transforms of spinors---some useful identities}{qft:problemSet4:4}{2018 HW4.IV}{

{\flushleft{C}}onsider the matrix $$\Lambda_{1\over 2} = e^{- {i\over 2} \omega_{\mu\nu} S^{\mu\nu}}~.$$
Here, $S^{\mu\nu} = {i\over 4} [\gamma^\mu, \gamma^\nu ]$ is as defined in class, in terms of the four $\gamma$-matrices (notice that, when using the representation of the $\gamma$ matrices in terms of Pauli matrices, the matrix $\Lambda_{1\over 2}$ looks like two sets of $M$ (and $M^*$) matrices discussed in class, now combined into one four-by-four object).

\makesubproblem{}{qft:problemSet4:4a}
Show that $\Lambda_{1\over 2}^{-1} \gamma^\mu \Lambda_{1\over 2} = \Lambda^\mu_{\; \nu} \gamma^\nu$, where $ \Lambda^\mu_{\; \nu}$ is the usual Lorentz transformation acting on vectors. (Feel free to show this for the infinitesimal form of the transformations, but then argue that the finite form holds as well.)
\makesubproblem{}{qft:problemSet4:4b}
Show that $\Lambda_{1\over 2}^{\dagger} \gamma^0 \Lambda_{1\over 2} =  \gamma^0$.
\makesubproblem{}{qft:problemSet4:4c}
Consider the fermion bilinear $\bar\psi \gamma^\mu \gamma^\nu \psi ={1\over 2} \bar\psi \{ \gamma^\mu ,\gamma^\nu \} \psi + {1\over 2} \bar\psi [\gamma^\mu, \gamma^\nu] \psi$, where $\{A,B\} = AB + BA$ is the anticommutator. Show that the two terms on the right transform as a scalar and a second-rank tensor, respectively, under Lorentz transformations.
} % makeproblem

\makeanswer{qft:problemSet4:4}{
\withproblemsetsParagraph{
\makeSubAnswer{}{qft:problemSet4:4a}
For infinesimal transformations we can show this using the BCH theorem.  First let
\begin{dmath}\label{eqn:ProblemSet4Problem4:20}
B
= -\frac{i}{2} \omega_{\mu\nu} S^{\mu\nu}
= -\frac{i}{2} \omega_{\mu\nu} \frac{i}{4} \antisymmetric{\gamma^\mu}{\gamma^\nu}
= \frac{1}{8} \omega_{\mu\nu} \lr{ \gamma^\mu \gamma^\nu - \gamma^\nu \gamma^\mu}.
= \frac{1}{4} \omega_{\mu\nu} \gamma^\mu \gamma^\nu,
\end{dmath}
where we've made use
\( \omega_{\mu\nu} = - \omega_{\nu\mu} \) to eliminate any \( \mu = \nu \) terms in the sum,
and \( \gamma^\mu \gamma^\nu = -\gamma^\nu \gamma^\mu \) for \( \mu \ne \nu \).
With this substitution, we have
\begin{dmath}\label{eqn:ProblemSet4Problem4:40}
\Lambda^{-1}_{1/2}
\gamma^\mu
\Lambda_{1/2}
=
e^{-B}
\gamma^\mu
e^{B}
=
\gamma^\mu
+
\antisymmetric{-B}{\gamma^\mu}
+
O(\omega^2),
\end{dmath}
and can now compute the commutator
\begin{dmath}\label{eqn:ProblemSet4Problem4:60}
\antisymmetric{-B}{\gamma^\mu}
=
-\frac{1}{4} \omega_{\alpha\beta}
\antisymmetric{
\gamma^\alpha \gamma^\beta
}{\gamma^\mu}
=
-\frac{1}{4} \omega^{\alpha\beta}
\antisymmetric{
\gamma_\alpha \gamma_\beta
}{\gamma^\mu}
=
\frac{1}{4} \omega^{\alpha\beta}
\lr{
   \gamma^\mu \gamma_\alpha \gamma_\beta
   -
   \gamma_\alpha \gamma_\beta \gamma^\mu
}.
\end{dmath}
When \( \mu \ne \alpha, \beta \), \( \gamma^\mu \) commutes with both \( \gamma_\alpha,  \gamma_\beta \),
so the matrices cancel, leaving just the \( \mu = \alpha, \mu = \beta \) contributions to the sum
\begin{dmath}\label{eqn:ProblemSet4Problem4:80}
\antisymmetric{-B}{\gamma^\mu}
=
\frac{1}{4} \omega^{\alpha\beta}
\lr{
   \ultensor{\delta}{\mu}{\alpha} \gamma_\beta
   -\ultensor{\delta}{\mu}{\beta} \gamma_\alpha
   - \gamma_\alpha \ultensor{\delta}{\mu}{\beta}
   + \gamma_\beta \ultensor{\delta}{\mu}{\alpha}
}
=
\frac{1}{2} \omega^{\alpha\beta}
\lr{
   \ultensor{\delta}{\mu}{\alpha} \gamma_\beta
   -\ultensor{\delta}{\mu}{\beta} \gamma_\alpha
}
=
\frac{1}{2}
\lr{
   \omega^{\alpha\beta}
   \ultensor{\delta}{\mu}{\alpha} \gamma_\beta
   -
   \omega^{\beta\alpha}
   \ultensor{\delta}{\mu}{\alpha} \gamma_\beta
}
=
   \omega^{\alpha\beta}
   \ultensor{\delta}{\mu}{\alpha} \gamma_\beta
=
   \omega^{\mu\beta}
\gamma_\beta.
\end{dmath}
The transformation to first order is therefore
\begin{dmath}\label{eqn:ProblemSet4Problem4:100}
\Lambda^{-1}_{1/2}
\gamma^\mu
\Lambda_{1/2}
=
\gamma^\mu
+
   \omega^{\mu\beta}
\gamma_\beta
=
\lr{
   \ultensor{\delta}{\mu}{\beta}
+
   \ultensor{\omega}{\mu}{\beta}
}
\gamma^\beta
=
\ultensor{\Lambda}{\mu}{\beta} \gamma^\beta.
\end{dmath}

To extend the argument to finite angles we use the usual argument.  For example, for a finite rotation \( e^{i \theta} \), we may decompose such a rotation into \( n \) small pieces, \( e^{i\theta/n} \) and compound those rotations by applying the small ones in sequence \( \lr{ e^{i \theta/n} }^n \).  Given the block matrix structure of \( \Lambda_{1/2} \)
\begin{dmath}\label{eqn:ProblemSet4Problem4:120}
\Lambda_{1/2}
=
\begin{bmatrix}
e^{-\inv{2} \omega_{0k} \sigma^k - \frac{i}{4} \omega_{jk} e^{jkl}  \sigma^l } & 0 \\
0 & e^{\inv{2} \omega_{0k} \sigma^k - \frac{i}{4} \omega_{jk} e^{jkl} \sigma^l }  \\
\end{bmatrix},
\end{dmath}
(as found in class), where we have \( 2 \times 2 \) exponentials on the diagonals, the same argument applies.

\makeSubAnswer{}{qft:problemSet4:4b}
As \( \Lambda_{1/2} \) is a diagonal matrix, we can compute \( \gamma^0 A^\dagger \gamma^0 \) in the block matrix representation
\begin{dmath}\label{eqn:ProblemSet4Problem4:140}
\gamma^0
\begin{bmatrix}
a & 0 \\
0 & b
\end{bmatrix}
\gamma^0
=
\begin{bmatrix}
0 & 1 \\
1 & 0
\end{bmatrix}
\begin{bmatrix}
a^\dagger & 0 \\
0         & b^\dagger
\end{bmatrix}
\begin{bmatrix}
0 & 1 \\
1 & 0
\end{bmatrix}
=
\begin{bmatrix}
0 & 1 \\
1 & 0
\end{bmatrix}
\begin{bmatrix}
0         & a^\dagger \\
b^\dagger & 0
\end{bmatrix}
=
\begin{bmatrix}
b^\dagger & 0 \\
0         & a^\dagger
\end{bmatrix}.
\end{dmath}

Using this (and \( (\sigma^k)^\dagger = \sigma^k \)), we have
\begin{dmath}\label{eqn:ProblemSet4Problem4:160}
\begin{aligned}
\gamma^0 \Lambda_{1/2}^\dagger \gamma^0
&=
\begin{bmatrix}
\lr{ e^{-\inv{2} \omega_{0k} \sigma^k - \frac{i}{4} \omega_{jk} e^{jkl}  \sigma^l } }^\dagger & 0 \\
0 & \lr{ e^{\inv{2} \omega_{0k} \sigma^k - \frac{i}{4} \omega_{jk} e^{jkl} \sigma^l } }^\dagger \\
\end{bmatrix} \\
&=
\begin{bmatrix}
e^{-\inv{2} \omega_{0k} \sigma^k + \frac{i}{4} \omega_{jk} e^{jkl}  \sigma^l } & 0 \\
0 & e^{\inv{2} \omega_{0k} \sigma^k + \frac{i}{4} \omega_{jk} e^{jkl} \sigma^l } \\
\end{bmatrix}.
\end{aligned}
\end{dmath}
Comparing to \cref{eqn:ProblemSet4Problem4:120}, we see that
\begin{dmath}\label{eqn:ProblemSet4Problem4:180}
\gamma^0 \Lambda_{1/2}^\dagger \gamma^0
=
\Lambda^{-1}_{1/2},
\end{dmath}
so
\begin{dmath}\label{eqn:ProblemSet4Problem4:200}
\gamma^0 \Lambda_{1/2}^\dagger \gamma^0 \Lambda_{1/2} = 1.
\end{dmath}
Multiplying by \( \gamma^0 \) on the left using \( (\gamma^0)^2 = 1 \), completes the proof.

\makeSubAnswer{}{qft:problemSet4:4c}
First recall that a second rank tensor transforms as
\begin{dmath}\label{eqn:ProblemSet4Problem4:220}
A^{\mu\nu} \rightarrow
\ulLambda{\mu}{\alpha}
\ulLambda{\nu}{\beta}
A^{\mu\nu}.
\end{dmath}

The transformation of the anticommutator term is just
\begin{dmath}\label{eqn:ProblemSet4Problem4:240}
\inv{2} \overbar{\Psi} \symmetric{\gamma^\mu}{\gamma^\nu} \Psi
=
\overbar{\Psi} g^{\mu\nu} \Psi
\rightarrow
\lr{ \overbar{\Psi}
\Lambda^{-1}_{1/2}
}
\lr{
\ulLambda{\mu}{\alpha}
\ulLambda{\nu}{\beta}
g^{\alpha\beta}
}
\lr{
\Lambda_{1/2}
\Psi
}
=
\overbar{\Psi}
g^{\mu\nu}
\Psi
=
\inv{2} \overbar{\Psi} \symmetric{\gamma^\mu}{\gamma^\nu} \Psi,
\end{dmath}
showing that the anticommutator transforms as a Lorentz scalar.
%where we've used the identity \( \ulLambda{\mu}{\alpha} \ulLambda{\nu}{\beta} g^{\alpha\beta} = g^{\mu\nu} \).

For the commutator term we have
\begin{dmath}\label{eqn:ProblemSet4Problem4:260}
\begin{aligned}
\inv{2} \overbar{\Psi}
\antisymmetric{\gamma^\mu}{\gamma^\nu}
\Psi
&\rightarrow
\inv{2} \lr{ \overbar{\Psi}
\Lambda^{-1}_{1/2}
}
\antisymmetric{\gamma^\mu}{\gamma^\nu}
\lr{
\Lambda_{1/2}
\Psi
} \\
&=
\inv{2}
\overbar{\Psi}
\Lambda^{-1}_{1/2}
\lr{
   \gamma^\mu \gamma^\nu
   -
   \gamma^\nu \gamma^\mu
}
\Lambda_{1/2}
\Psi \\
&=
\inv{2}
\overbar{\Psi}
\Biglr{
   \lr{
   \Lambda^{-1}_{1/2}
      \gamma^\mu
   \Lambda_{1/2}
   }
   \lr{
   \Lambda^{-1}_{1/2}
   \gamma^\nu
   \Lambda_{1/2}
   } \\
&\qquad-
   \lr{
   \Lambda^{-1}_{1/2}
      \gamma^\nu
   \Lambda_{1/2}
   }
   \lr{
   \Lambda^{-1}_{1/2}
   \gamma^\mu
   \Lambda_{1/2}
   }
}
\Psi \\
&=
\inv{2}
\overbar{\Psi}
\lr{
   \lr{\ulLambda{\mu}{\alpha} \gamma^\alpha}
   \lr{\ulLambda{\nu}{\beta} \gamma^\beta}
      -
   \lr{\ulLambda{\nu}{\beta} \gamma^\beta}
   \lr{\ulLambda{\mu}{\alpha} \gamma^\alpha}
}
\Psi \\
&=
\ulLambda{\mu}{\alpha}
\ulLambda{\nu}{\beta}
\lr{
   \inv{2}
   \overbar{\Psi}
   \antisymmetric{\gamma^\alpha}{\gamma^\beta}
   \Psi
},
\end{aligned}
\end{dmath}
which is the transformation property \cref{eqn:ProblemSet4Problem4:220} for second rank tensors noted above.
}
}
