%
% Copyright � 2015 Peeter Joot.  All Rights Reserved.
% Licenced as described in the file LICENSE under the root directory of this GIT repository.
%
\makeoproblem{Field Lagrangian with a divergence}
{qft:LukeProblemSet1:5}
{2015 ps1.5}
{
%\makesubproblem{}{qft:LukeProblemSet1:5a}

Show that replacing the Lagrange density \( L = L(\phi_a, \partial_\alpha \phi_a ) \) by

\begin{equation}\label{eqn:LukeProblemSet1Problem5:20}
L' = L + \partial_\mu \wedge^\mu(x),
\end{equation}

where \( \wedge^\mu(x), \mu = 0,\cdots,3\), are arbitrary functions of the fields \( \phi_a(x) \), does not alter the equations of motion. Thus, when constructing the most general Lagrange density for a field, we do not have to include terms which are total derivatives. This will simplify life.
} % makeproblem

\makeanswer{qft:LukeProblemSet1:5}{
\withproblemsetsParagraph{

%%I had to start with re-deriving the field equations for a Lagrange density since I'd forgotten what they were.  If a field \( \phi \) is varied \( \phi' = \phi + \overbar{\phi} \), with the zero variation on the boundaries of the action volume element, we have to first order
%%
%%\begin{dmath}\label{eqn:LukeProblemSet1Problem5:40}
%%\LL'
%%= \LL( \phi + \overbar{\phi}, \partial_\alpha ( \phi + \overbar{\phi}) )
%%=
%%\LL( \phi )
%%+
%%\PD{\overbar{\phi}}{\LL} \overbar{\phi}
%%+
%%\PD{\partial_\beta \overbar{\phi}}{\LL} \partial_\beta \overbar{\phi}.
%%\end{dmath}
%%
%%The variation of the action is
%%
%%\begin{dmath}\label{eqn:LukeProblemSet1Problem5:60}
%%\delta S
%%=
%%\int dV \lr{
%%\PD{\overbar{\phi}}{\LL} \overbar{\phi}
%%+
%%\PD{\partial_\beta \overbar{\phi}}{\LL} \partial_\beta \overbar{\phi}
%%}
%%=
%%\int dV \lr{
%%\PD{\overbar{\phi}}{\LL} \overbar{\phi}
%%+
%%\partial_\beta \lr{
%%\overbar{\phi}
%%\PD{\partial_\beta \overbar{\phi}}{\LL}
%%}
%%-
%%\overbar{\phi}
%%\partial_\beta \lr{
%%\PD{\partial_\beta \overbar{\phi}}{\LL}
%%}
%%}
%%=
%%\int d\Omega
%%\evalbar{
%%\overbar{\phi}
%%\PD{\partial_\beta \overbar{\phi}}{\LL}
%%}
%%{
%%\Delta x^\beta
%%}
%%+
%%\int dV
%%\overbar{\phi}
%%\lr{
%%\PD{\overbar{\phi}}{\LL}
%%-
%%\partial_\beta \lr{
%%\PD{\partial_\beta \overbar{\phi}}{\LL}
%%}
%%}
%%\end{dmath}
%%
%%The first integral vanishes given the boundary condition assumptions, and the second provides the equations for the field.  Generalizing that to multiple fields \( \phi_a \), and evaluating the derivatives at \( \overbar{\phi}_a = \phi_a \) we have
%%
%%\boxedEquation{eqn:qftProblemSet1Problem5:80}{
%%\PD{\phi_a}{\LL}
%%=
%%\partial_\beta \lr{
%%\PD{\partial_\beta \phi_a}{\LL}
%%}.
%%}
%%
%%Now we can tackle the problem.  
%%XX
Consider first just two fields, say \( \phi \) and \( \psi\), and consider

\begin{dmath}\label{eqn:qftProblemSet1Problem5:100}
\partial_\beta
\lr{
\PD{\partial_\beta \phi}{} \partial_\mu \wedge^\mu
}
=
\partial_\beta
\lr{
\PD{\partial_\beta \phi}{}
\lr{
\PD{\phi}{ \wedge^\mu } \PD{x^\mu}{ \phi}
+
\PD{\psi}{ \wedge^\mu } \PD{x^\mu}{ \psi}
}
}
=
\partial_\beta \PD{\phi}{ \wedge^\beta }
=
\PD{\phi}{ \partial_\beta \wedge^\beta }.
\end{dmath}

We see that the divergence \( \partial_\mu \wedge^\mu \) also satisfies the field Euler-Lagrange equations for the field \( \phi \).  This will clearly be the case for multiple fields.  Making that explicit, we can generalize the above slightly

\begin{dmath}\label{eqn:qftProblemSet1Problem5:120}
\partial_\beta
\lr{
\PD{\partial_\beta \phi_a}{} \partial_\mu \wedge^\mu
}
=
\partial_\beta
\lr{
\PD{\partial_\beta \phi_a}{}
\PD{\phi_b}{ \wedge^\mu } \PD{x^\mu}{ \phi_b }
}
=
\partial_\beta \PD{\phi_b}{ \wedge^\mu } \delta_{b a} {\delta^\beta}_\mu
=
\PD{\phi_a}{ \partial_\beta \wedge^\beta }.
\end{dmath}

%\makeSubAnswer{}{qft:LukeProblemSet1:5a}
}
}
