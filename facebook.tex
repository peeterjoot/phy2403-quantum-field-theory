%
% Copyright � 2018 Peeter Joot.  All Rights Reserved.
% Licenced as described in the file LICENSE under the root directory of this GIT repository.
%
%{
\input{../latex/blogpost.tex}
\renewcommand{\basename}{facebook}
%\renewcommand{\dirname}{notes/phy1520/}
\renewcommand{\dirname}{notes/ece1228-electromagnetic-theory/}
%\newcommand{\dateintitle}{}
%\newcommand{\keywords}{}

\input{../latex/peeter_prologue_print2.tex}

\usepackage{peeters_layout_exercise}
\usepackage{peeters_braket}
\usepackage{peeters_figures}
\usepackage{siunitx}
\usepackage{verbatim}
%\usepackage{mhchem} % \ce{}
%\usepackage{macros_bm} % \bcM
%\usepackage{macros_qed} % \qedmarker
%\usepackage{txfonts} % \ointclockwise

\beginArtNoToc

\generatetitle{Images for the group chat}
%\chapter{Images for the group chat}
%\label{chap:facebook}

%\begin{equation*}
%j^\mu = \trace{\lr{ \delta H^\dagger \partial^\mu H + \partial^\mu H^\dagger \delta H }}
%\end{equation*}
%
i.e.
\begin{equation*}
\begin{aligned}
H &= \inv{\sqrt{2}}
\begin{bmatrix}
i \sigma^2 \Phi^\conj & \Phi
\end{bmatrix} \\
H^\dagger &= \inv{\sqrt{2}}
\begin{bmatrix}
-i \Phi^\T \sigma^2 \\
\Phi^\dagger
\end{bmatrix}
\end{aligned}
\end{equation*}
so
\begin{dmath}\label{eqn:facebook:20}
j^{\mu, a}_L
=
\frac{i}{2}
   \trace{\lr{
- H^\dagger \sigma^a
\partial^\mu H
   +
   \partial^\mu H^\dagger
\sigma^a H
   }}
=
\frac{i}{4}
   \trace{\lr{
-
\begin{bmatrix}
-i \Phi^\T \sigma^2 \\
\Phi^\dagger
\end{bmatrix}
\sigma^a
\begin{bmatrix}
i \sigma^2 \partial^\mu \Phi^\conj & \partial^\mu \Phi
\end{bmatrix}
   +
\begin{bmatrix}
-i \partial^\mu \Phi^\T \sigma^2 \\
\partial^\mu \Phi^\dagger
\end{bmatrix}
\sigma^a
\begin{bmatrix}
i \sigma^2 \Phi^\conj & \Phi
\end{bmatrix}
  }}
=
\cdots
=
\frac{i}{2} \lr{ \partial^\mu \Phi^\dagger \sigma^a \Phi - \Phi^\dagger \sigma^a \partial^\mu \Phi}
\end{dmath}

Actually, it looks like I have one cross term after the expansion to second order

\begin{equation*}
i \lr{ \partial_\mu h \sigma^a \partial^\mu \phi^a
-
\sigma^a \partial^\mu \phi^a \partial_\mu h }
\end{equation*}

\section{blah}

The hint seems to suggest that particle number is conserved, even though the interaction term does not have the structure of a number operator.  I have to conclude (too late for problem set submission) that I don't really understand what is meant by preservation of particle number in this case, and will need to see the problem set solution or discuss this in office hours to understand what is being asked for.
%\makeSubAnswer{}{qft:problemSet2:4c}
If we designate an N-particle momentum state by
\begin{dmath}\label{eqn:ProblemSet2Problem4:180}
\ket{\Bp_1, \Bp_2, \cdots \Bp_N} =
a_{\Bp_1}^\dagger
a_{\Bp_2}^\dagger
\cdots
a_{\Bp_N}^\dagger \ket{0, 0, \cdots, 0},
\end{dmath}
then the interaction terms action on such a state is
\begin{dmath}\label{eqn:ProblemSet2Problem4:200}
      a_{\Bp_j}^\dagger a_{\Bp_m}^\dagger a_{\Bp_n} a_{{\Bp_j} + {\Bp_m} - {\Bp_n}}
\ket{\Bp_1, \Bp_2, \cdots \Bp_N}.
\end{dmath}
I'm not sure if this is meaningful, or how to interpret it, and think that I'm going to have to get explanation about what this abstraction means.  I'm also not sure what is meant by the question ``What kinds of scattering processes does it describe.''
%\makeSubAnswer{}{qft:problemSet2:4d}
Not attempted.
%}
\EndArticle
%\EndNoBibArticle
