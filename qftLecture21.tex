%
% Copyright � 2018 Peeter Joot.  All Rights Reserved.
% Licenced as described in the file LICENSE under the root directory of this GIT repository.
%
%{
%\input{../latex/blogpost.tex}
%\renewcommand{\basename}{qftLecture21}
%\renewcommand{\dirname}{notes/phy2403/}
%\newcommand{\keywords}{PHY2403H}
%\input{../latex/peeter_prologue_print2.tex}
%
%%\usepackage{phy2403}
%\usepackage{peeters_braket}
%\usepackage{peeters_layout_exercise}
%\usepackage{peeters_figures}
%\usepackage{mathtools}
%\usepackage{siunitx}
%\usepackage{macros_cal} % LL
%
%\newcommand{\ultensor}[3]{{{#1}^{#2}}_{#3}}
%\newcommand{\oPsi}[0]{\overbar{\Psi}}
%\newcommand{\osigma}[0]{\overbar{\sigma}}
%\newcommand{\ubar}[0]{\overbar{u}}
%\newcommand{\vbar}[0]{\overbar{v}}
%\newcommand{\deltathree}[0]{\delta^{(3)}}
%\newcommand{\deltafour}[0]{\delta^{(4)}}
%\newcommand{\ITwo}[0]{{\begin{bmatrix} 1 & 0 \\ 0 & 1 \end{bmatrix}}}
%\newcommand{\DiracGammaZero}[0]{{\begin{bmatrix} 0 & 1 \\ 1 & 0 \end{bmatrix}}}
%\newcommand{\DiracGammaK}[1]{{\begin{bmatrix} 0 & \sigma^{#1} \\ -\sigma^{#1} & 0 \end{bmatrix}}}
%
%\beginArtNoToc
%\generatetitle{PHY2403H Quantum Field Theory.  Lecture 21, Part I: Dirac equation solutions, orthogonality conditions, direct products.  Taught by Prof.\ Erich Poppitz}
%%\chapter{Dirac equation solutions, orthogonality conditions, direct products}
\label{chap:qftLecture21}

%%Peeter's lecture notes from class.  These may be incoherent and rough.
%%
%%These are notes for the UofT course PHY2403H, Quantum Field Theory, taught by Prof. Erich Poppitz, covering \textchapref{{1}} \citep{peskin1995introduction} content.

%\paragraph{DISCLAIMER: Rough notes from class, with some additional side notes.}
%
%These are notes for the UofT course PHY2403H, Quantum Field Theory, taught by Prof. Erich Poppitz, fall 2018.
%%, covering \textchapref{{1}} \citep{peskin1995introduction} content.
%
\section{Review.}

We were studying the Dirac Lagrangian
\begin{dmath}\label{eqn:qftLecture21:20}
\LL_{\text{Dirac}} = \oPsi \lr{ i \gamma^\mu \partial_\mu -m } \Psi,
\end{dmath}
from which we find
\begin{dmath}\label{eqn:qftLecture21:40}
\lr{ i \gamma^\mu \partial_\mu -m } \Psi = 0,
\end{dmath}
the Dirac equation, and saw that solutions to this equation satisfies the KG equation.  We found solution
\begin{dmath}\label{eqn:qftLecture21:60}
\Psi(x) = u(p) e^{-i p \cdot x},
\end{dmath}
which is automatically a solution to the KG equation.  There are actually two linearly independent solutions
\begin{dmath}\label{eqn:qftLecture21:80}
u^s(p) =
\begin{bmatrix}
\sqrt{ p \cdot \sigma } \zeta^s \\
\sqrt{ p \cdot \osigma } \zeta^s \\
\end{bmatrix},
\end{dmath}
where \( \zeta^1 = (1,0)^\T, \zeta^2 = (0,1)^\T \).

\section{Normalization.}

\maketheorem{\( u^\dagger u \)}{thm:qftLecture21:1}{
\begin{equation*}
u^{r \dagger} u^{s}
= 2 p_0 \delta^{r s}.
\end{equation*}
} % theorem

Proof:
\begin{dmath}\label{eqn:qftLecture21:120}
u^{s \dagger} u^r
=
\begin{bmatrix}
\zeta^{s \dagger} \sqrt{ p \cdot \sigma }
&
\zeta^{s \dagger} \sqrt{ p \cdot \osigma }
\end{bmatrix}
\begin{bmatrix}
\sqrt{ p \cdot \sigma } \zeta \\
\sqrt{ p \cdot \osigma } \zeta \\
\end{bmatrix}
=
\zeta^{s \dagger} \lr{
\sqrt{ p \cdot \sigma }
\sqrt{ p \cdot \sigma }
+
\sqrt{ p \cdot \osigma }
\sqrt{ p \cdot \osigma }
}
\zeta^r
=
\zeta^{s \dagger} \lr{
 p \cdot \sigma
+
 p \cdot \osigma
}
\zeta^r
=
\zeta^{s \dagger}
\lr{
p_0 - \Bp \cdot \Bsigma
+
p_0 + \Bp \cdot \Bsigma
}
\zeta^r
=
2 p_0
\zeta^{s \dagger} \zeta^r.
\end{dmath}
%Putting back in the suffixes, we have
%\begin{dmath}\label{eqn:qftLecture21:1000}
%u^{s \dagger} u^{r}
%=
%2 p_0
%\zeta^{s\dagger} \zeta^r.
%\end{dmath}
We can easily see that \( \zeta^{s\dagger} \zeta^r = \delta^{rs} \) by writing out those products
\begin{dmath}\label{eqn:qftLecture21:1020}
\begin{aligned}
\zeta^{1\dagger} \zeta^1
&=
\begin{bmatrix}
1 & 0
\end{bmatrix}
\begin{bmatrix}
1 \\
0
\end{bmatrix}
&=
1 \\
\zeta^{1\dagger} \zeta^2
&=
\begin{bmatrix}
1 & 0
\end{bmatrix}
\begin{bmatrix}
0 \\
1
\end{bmatrix}
&=
0 \\
\zeta^{2\dagger} \zeta^1
&=
\begin{bmatrix}
0 & 1
\end{bmatrix}
\begin{bmatrix}
1 \\
0
\end{bmatrix}
&=
0 \\
\zeta^{2\dagger} \zeta^2
&=
\begin{bmatrix}
0 & 1
\end{bmatrix}
\begin{bmatrix}
0 \\
1
\end{bmatrix}
&=
1,
\end{aligned}
\end{dmath}
which completes the proof.

%\begin{dmath}\label{eqn:qftLecture21:140}
%u^{s \dagger} u^{s'}
%= 2 p_0 \zeta^{s \dagger} \zeta^{s'}
%= 2 p_0 \delta^{s s'}
%\end{dmath}

We also want to compute \( \ubar u \), but need a couple intermediate results.
\makelemma{Products of \( p \cdot \sigma, p \cdot \osigma \).}{lemma:qftLecture21:2}{
\begin{equation*}
(p \cdot \sigma) (p \cdot \osigma)
=
(p \cdot \osigma) (p \cdot \sigma)
= m^2.
\end{equation*}
} % lemma

Proof:
\begin{dmath}\label{eqn:qftLecture21:180}
(p \cdot \sigma) (p \cdot \osigma)
=
\lr{ p^0 - \Bp \cdot \Bsigma}
\lr{ p^0 + \Bp \cdot \Bsigma}
=
(p^0)^2
-
(\Bp \cdot \Bsigma)^2
=
(p^0)^2
-
\Bp^2
= m^2
\end{dmath}
and
\begin{dmath}\label{eqn:qftLecture21:200}
(p \cdot \osigma) (p \cdot \sigma)
=
\lr{ p^0 + \Bp \cdot \Bsigma}
\lr{ p^0 - \Bp \cdot \Bsigma}
=
(p^0)^2
-
(\Bp \cdot \Bsigma)^2
=
(p^0)^2
-
\Bp^2
= m^2.
\end{dmath}

\maketheorem{\( \ubar u \).}{thm:qftLecture21:3}{
\begin{equation*}
\ubar^r(\Bp) u^{s}(\Bp) = 2 m \delta^{r s}.
\end{equation*}
} % theorem

Proof:
\begin{dmath}\label{eqn:qftLecture21:160}
\ubar^r u^s
=
u^{r\dagger} \gamma^0 u^s
=
\begin{bmatrix}
\zeta^{r \dagger} \sqrt{ p \cdot \sigma }
&
\zeta^{r \dagger} \sqrt{ p \cdot \osigma }
\end{bmatrix}
\begin{bmatrix}
0 & 1 \\
1 & 0
\end{bmatrix}
\begin{bmatrix}
\sqrt{p \cdot \sigma} \zeta^s \\
\sqrt{p \cdot \osigma} \zeta^s
\end{bmatrix}
=
\zeta^{r\dagger} \lr{
\sqrt{ p \cdot \sigma }
\sqrt{ p \cdot \osigma }
+
\sqrt{ p \cdot \osigma }
\sqrt{ p \cdot \sigma }
}
\zeta^s
=
2 m \zeta^{r\dagger} \zeta^s
=
2 m \delta^{rs},
\end{dmath}
which completes the proof.

\section{Other solution.}

Now we seek the other plane wave solution
\begin{dmath}\label{eqn:qftLecture21:260}
\Psi(x) = v(p) e^{i p \cdot x}.
\end{dmath}
\maketheorem{\(v\) solution to the Dirac equation.}{thm:qftLecture21:101}{
\Cref{eqn:qftLecture21:260} is a solution to the Dirac equation, provided
%\boxedEquation{eqn:qftLecture21:280}{
\begin{equation*}
v^s(p)
=
\begin{bmatrix}
\sqrt{p \cdot \sigma} \eta^s \\
-\sqrt{p \cdot \osigma} \eta^s \\
\end{bmatrix},
\end{equation*}
where \( \eta^1 = (1,0)^\T, \eta^2 = (0,1)^\T \).
} % theorem
Proof is left to \cref{problem:qftLecture21:1}.

\maketheorem{\(v\) normalization.}{thm:qftLecture21:13}{
%\label{eqn:qftLecture21:300}
\begin{equation*}
\begin{aligned}
\vbar^r(p) v^s(p) &= - 2 m \delta^{rs} \\
v^{r \dagger}(p) v^s(p) &= 2 p^0 \delta^{rs}.
\end{aligned}
\end{equation*}
} % theorem
\Cref{thm:qftLecture21:13} is proven in \cref{problem:qftLecture21:2}.
%and
%\begin{dmath}\label{eqn:qftLecture21:480}
%\vbar^s
%= v^{s\dagger} \gamma_0.
%\end{dmath}

It will also be useful to restate the \( 2 \delta^{rs} p_0 \) normalization conditions as
\begin{dmath}\label{eqn:qftLecture21:1200}
\begin{aligned}
u^{r\dagger}(\Bp) u^s(\Bp) &= 2 \omega_\Bp \delta^{sr} \\
v^{r\dagger}(\Bp) v^s(\Bp) &= 2 \omega_\Bp \delta^{sr}.
\end{aligned}
\end{dmath}

Various orthogonality conditions exist between the \( u \)'s and \( v\)'s
\maketheorem{Dirac adjoint orthogonality conditions.}{thm:qftLecture21:1260}{
\begin{equation*}
\begin{aligned}
\ubar^r(p) v^s(p) &= 0 \\
\vbar^r(p) u^s(p) &= 0.
\end{aligned}
\end{equation*}
} % theorem
Proof left to \cref{problem:qftLecture21:4}.

\maketheorem{Dagger orthogonality conditions.}{thm:qftLecture21:1261}{
\begin{equation*}
\begin{aligned}
v^{r \dagger}(-\Bp) u^s(\Bp) &= 0 \\
u^{r\dagger}(\Bp) v^s(-\Bp) &= 0.
\end{aligned}
\end{equation*}
} % theorem
Proof left to \cref{problem:qftLecture21:5}.

Finally, there are a couple tensor products of interest.
\makedefinition{Tensor product.}{dfn:qftLecture21:7}{
Given a pair of vectors
\begin{equation*}
x =
\begin{bmatrix}
x_1 \\
\vdots \\
x_n \\
\end{bmatrix},
\qquad
y =
\begin{bmatrix}
y_1 \\
\vdots \\
y_n \\
\end{bmatrix},
\end{equation*}
the tensor product is the matrix of all elements \( x_i y_j \)
%There are two possible ways that we can assemble a matrix product of two vectors, the first of which is just the dot product
%\begin{equation}\label{eqn:qftLecture21:380}
%y^\T x = \text{scalar} = \sum_{i = 1}^n x^i y^i,
%\end{equation}
%and the second is a tensor (or direct) product
\begin{equation*}
x \otimes y^\T =
\begin{bmatrix}
x_1 \\
\vdots \\
x_n \\
\end{bmatrix}
\otimes
\begin{bmatrix}
y_1 \cdots y_n
\end{bmatrix}
=
\begin{bmatrix}
x_1 y_1 & x_1 y_2 & \cdots & x_1 y_n \\
x_2 y_1 & x_2 y_2 & \cdots & x_2 y_n \\
x_3 y_1 & \ddots  &        &         \\
\vdots  &         &        &         \\
x_n y_1 & \cdots  &        & x_n y_n
\end{bmatrix}.
%=
%\Norm{ x_i y_j }
\end{equation*}
} % definition

% I don't see how this would be:
%It was claimed that such a tensor product can be likened to the completeness relation that we express in QM as
%\begin{dmath}\label{eqn:qftLecture21:460}
%\sum_n \ket{n} \bra{n} \equiv 1.
%\end{dmath}
%
\maketheorem{Direct product relations.}{thm:qftLecture21:17}{
\begin{equation*}
\begin{aligned}
\sum_{s = 1}^2 u^s(p) \otimes \ubar^s(p) &= \gamma \cdot p + m \\
\sum_{s = 1}^2 v^s(p) \otimes \vbar^s(p) &= \gamma \cdot p - m \\
\end{aligned}
\end{equation*}
} % theorem

For the \( v \)'s
\begin{dmath}\label{eqn:qftLecture21:420}
\sum_{s = 1,2}
\begin{bmatrix}
\sqrt{p \cdot \sigma} \eta^s \\
-\sqrt{p \cdot \osigma} \eta^s \\
\end{bmatrix}
\otimes
\begin{bmatrix}
(\eta^s)^\T \sqrt{ p \cdot \sigma } &
-(\eta^s)^\T \sqrt{ p \cdot \sigma }
\end{bmatrix}
\begin{bmatrix}
0 & 1 \\
1 & 0
\end{bmatrix}
=
\sum_{s = 1,2}
\begin{bmatrix}
\sqrt{p \cdot \sigma} \eta^s \\
\sqrt{p \cdot \osigma} \eta^s \\
\end{bmatrix}
\otimes
\begin{bmatrix}
-(\eta^s)^\T \sqrt{ p \cdot \sigma } &
(\eta^s)^\T \sqrt{ p \cdot \sigma }
\end{bmatrix}
=
\sum_{s = 1,2}
\begin{bmatrix}
-\sqrt{p \cdot \sigma} \eta^s \otimes (\eta^s)^\T \sqrt{ p \cdot \sigma } & \sqrt{p \cdot \sigma} \eta^s \otimes (\eta^s)^\T \sqrt{ p \cdot \sigma }  \\
\sqrt{p \cdot \osigma} \eta^s \otimes (\eta^s)^\T \sqrt{ p \cdot \sigma } & -\sqrt{p \cdot \osigma} \eta^s \otimes (\eta^s)^\T \sqrt{ p \cdot \sigma }
\end{bmatrix},
\end{dmath}
but
\begin{dmath}\label{eqn:qftLecture21:1120}
\eta^1 \otimes \eta^{1T}
=
\begin{bmatrix}
1 \\
0
\end{bmatrix}
\begin{bmatrix}
1 &
0
\end{bmatrix}
=
\begin{bmatrix}
1 & 0 \\
0 & 0
\end{bmatrix},
\end{dmath}
and
\begin{dmath}\label{eqn:qftLecture21:1140}
\eta^2 \otimes \eta^{2T}
=
\begin{bmatrix}
0 \\
1
\end{bmatrix}
\begin{bmatrix}
0 &
1
\end{bmatrix}
=
\begin{bmatrix}
0 & 0 \\
0 & 1
\end{bmatrix},
\end{dmath}
so \( \sum_{s = 1,2} \eta^s \otimes \eta^{s\T} = 1 \), leaving
\begin{dmath}\label{eqn:qftLecture21:1160}
\sum_{s = 1}^2 v^s(p) \otimes \vbar^s(p)
=
\begin{bmatrix}
-\sqrt{ p \cdot \sigma } \sqrt{ p \cdot \osigma } & \sqrt{ p \cdot \sigma } \sqrt{ p \cdot \sigma } \\
\sqrt{ p \cdot \osigma } \sqrt{ p \cdot \osigma } & -\sqrt{ p \cdot \osigma } \sqrt{ p \cdot \osigma }
\end{bmatrix}
=
\begin{bmatrix}
-\sqrt{ p \cdot \sigma p \cdot \osigma } & \sqrt{ p \cdot \sigma p \cdot \sigma } \\
\sqrt{ p \cdot \osigma p \cdot \osigma } & -\sqrt{ p \cdot \osigma p \cdot \osigma }
\end{bmatrix}
=
\begin{bmatrix}
-m & p \cdot \sigma \\
p \cdot \osigma & - m
\end{bmatrix}
=
-m \BOne
+ p^0
\begin{bmatrix}
0 & 1 \\
1 & 0
\end{bmatrix}
+ \Bp
\cdot
\begin{bmatrix}
0 & -\Bsigma \\
\Bsigma & 0
\end{bmatrix}
=
-m + p^\mu \gamma_\mu,
\end{dmath}
as stated.
Proof for the \( u\)'s is left to \cref{problem:qftLecture21:3}.

%}
%\EndNoBibArticle
