%
% Copyright � 2017 Peeter Joot.  All Rights Reserved.
% Licenced as described in the file LICENSE under the root directory of this GIT repository.
%
%{
\input{../latex/blogpost.tex}
\renewcommand{\basename}{qft21}
\renewcommand{\dirname}{notes/phy2403/}
\newcommand{\keywords}{PHY2403H}
\input{../latex/peeter_prologue_print2.tex}

%\usepackage{phy2403}
\usepackage{peeters_braket}
\usepackage{peeters_layout_exercise}
\usepackage{peeters_figures}
\usepackage{mathtools}
\usepackage{siunitx}
\usepackage{macros_cal} % LL

\newcommand{\ultensor}[3]{{{#1}^{#2}}_{#3}}
\newcommand{\oPsi}[0]{\overbar{\Psi}}
\newcommand{\osigma}[0]{\overbar{\sigma}}
\newcommand{\ubar}[0]{\overbar{u}}
\newcommand{\vbar}[0]{\overbar{v}}
\newcommand{\deltathree}[0]{\delta^{(3)}}
\newcommand{\deltafour}[0]{\delta^{(4)}}

\beginArtNoToc
\generatetitle{PHY2403H Quantum Field Theory.  Lecture 21: XXX.  Taught by Prof.\ Erich Poppitz}
%\chapter{XXX}
\label{chap:qft21}

\paragraph{Disclaimer}

%%Peeter's lecture notes from class.  These may be incoherent and rough.
%%
%%These are notes for the UofT course PHY2403H, Quantum Field Theory, taught by Prof. Erich Poppitz, covering \textchapref{{1}} \citep{peskin1995introduction} content.

\paragraph{DISCLAIMER: Very rough notes from class, with some additional side notes.}

These are notes for the UofT course PHY2403H, Quantum Field Theory, taught by Prof. Erich Poppitz, fall 2018.
%, covering \textchapref{{1}} \citep{peskin1995introduction} content.

\section{Review.}

We were studying the Dirac Lagrangian
\begin{dmath}\label{eqn:qftLecture21:20}
\LL_{\text{Dirac}} = \oPsi \lr{ i \gamma^\mu \partial_\mu -m } \Psi,
\end{dmath}
from which we find
\begin{dmath}\label{eqn:qftLecture21:40}
\lr{ i \gamma^\mu \partial_\mu -m } \Psi = 0,
\end{dmath}
the Dirac equation, and saw that solutions to this equation satisfies the KG equation.  We found solution
\begin{dmath}\label{eqn:qftLecture21:60}
\Psi(x) = u(p) e^{-i p \cdot x},
\end{dmath}
which is automatically a solution to the KG equation.  There are actually two linearly independent solutions
\begin{dmath}\label{eqn:qftLecture21:80}
u^s(p) =
\begin{bmatrix}
\sqrt{ p \cdot \sigma } \zeta^s \\
\sqrt{ p \cdot \osigma } \zeta^s \\
\end{bmatrix},
\end{dmath}
where \( \zeta^1 = (1,0)^\T, \zeta^2 = (0,1)^\T \).

\section{Normalization.}

\maketheorem{\( u^\dagger u \)}{thm:qftLecture21:1}{
\begin{equation*}
u^{s \dagger} u^{s'}
= 2 p_0 \delta^{s s'}
\end{equation*}
} % theorem

Proof:
\begin{dmath}\label{eqn:qftLecture21:120}
u^{s \dagger} u^r
=
\begin{bmatrix}
\zeta^{s \dagger} \sqrt{ p \cdot \sigma }
&
\zeta^{s \dagger} \sqrt{ p \cdot \osigma }
\end{bmatrix}
\begin{bmatrix}
\sqrt{ p \cdot \sigma } \zeta \\
\sqrt{ p \cdot \osigma } \zeta \\
\end{bmatrix}
=
\zeta^{s \dagger} \lr{
\sqrt{ p \cdot \sigma }
\sqrt{ p \cdot \sigma }
+
\sqrt{ p \cdot \osigma }
\sqrt{ p \cdot \osigma }
}
\zeta^r
=
\zeta^{s \dagger} \lr{
 p \cdot \sigma
+
 p \cdot \osigma
}
\zeta^r
=
\zeta^{s \dagger}
\lr{
p_0 - \Bp \cdot \Bsigma
+
p_0 + \Bp \cdot \Bsigma
}
\zeta^r
=
2 p_0
\zeta^{s \dagger} \zeta^r.
\end{dmath}
%Putting back in the suffixes, we have
%\begin{dmath}\label{eqn:qftLecture21:1000}
%u^{s \dagger} u^{r}
%=
%2 p_0
%\zeta^{s\dagger} \zeta^r.
%\end{dmath}
We can easily see that \( \zeta^{s\dagger} \zeta^r = \delta^{rs} \) by writing out those products
\begin{dmath}\label{eqn:qftLecture21:1020}
\begin{aligned}
\zeta^{1\dagger} \zeta^1
&=
\begin{bmatrix}
1 & 0
\end{bmatrix}
\begin{bmatrix}
1 \\
0
\end{bmatrix}
&=
1 \\
\zeta^{1\dagger} \zeta^2
&=
\begin{bmatrix}
1 & 0
\end{bmatrix}
\begin{bmatrix}
0 \\
1
\end{bmatrix}
&=
0 \\
\zeta^{2\dagger} \zeta^1
&=
\begin{bmatrix}
0 & 1
\end{bmatrix}
\begin{bmatrix}
1 \\
0
\end{bmatrix}
&=
0 \\
\zeta^{2\dagger} \zeta^2
&=
\begin{bmatrix}
0 & 1
\end{bmatrix}
\begin{bmatrix}
0 \\
1
\end{bmatrix}
&=
1,
\end{aligned}
\end{dmath}
which completes the proof.

%\begin{dmath}\label{eqn:qftLecture21:140}
%u^{s \dagger} u^{s'}
%= 2 p_0 \zeta^{s \dagger} \zeta^{s'}
%= 2 p_0 \delta^{s s'}
%\end{dmath}

We also want to compute \( \ubar u \), but need a couple intermediate results.
\makelemma{Products of \( p \cdot \sigma, p \cdot \osigma \).}{lemma:qftLecture21:2}{
\begin{equation*}
(p \cdot \sigma) (p \cdot \osigma)
=
(p \cdot \osigma) (p \cdot \sigma)
= m^2.
\end{equation*}
} % lemma

Proof:
\begin{dmath}\label{eqn:qftLecture21:180}
(p \cdot \sigma) (p \cdot \osigma)
=
\lr{ p^0 - \Bp \cdot \Bsigma}
\lr{ p^0 + \Bp \cdot \Bsigma}
=
(p^0)^2
-
(\Bp \cdot \Bsigma)^2
=
(p^0)^2
-
\Bp^2
= m^2
\end{dmath}
and
\begin{dmath}\label{eqn:qftLecture21:200}
(p \cdot \osigma) (p \cdot \sigma)
=
\lr{ p^0 + \Bp \cdot \Bsigma}
\lr{ p^0 - \Bp \cdot \Bsigma}
=
(p^0)^2
-
(\Bp \cdot \Bsigma)^2
=
(p^0)^2
-
\Bp^2
= m^2.
\end{dmath}

\maketheorem{\( \ubar u \).}{thm:qftLecture21:3}{
\begin{equation*}
\ubar^s(\Bp) u^{s'}(\Bp) = 2 m \delta^{s s'}.
\end{equation*}
} % theorem

Proof:
\begin{dmath}\label{eqn:qftLecture21:160}
\ubar^r u^s
=
u^{r\dagger} \gamma^0 u^s
=
\begin{bmatrix}
\zeta^{r \dagger} \sqrt{ p \cdot \sigma }
&
\zeta^{r \dagger} \sqrt{ p \cdot \osigma }
\end{bmatrix}
\begin{bmatrix}
0 & 1 \\
1 & 0
\end{bmatrix}
\begin{bmatrix}
\sqrt{p \cdot \sigma} \zeta^s \\
\sqrt{p \cdot \osigma} \zeta^s
\end{bmatrix}
=
\zeta^{r\dagger} \lr{
\sqrt{ p \cdot \sigma }
\sqrt{ p \cdot \osigma }
+
\sqrt{ p \cdot \osigma }
\sqrt{ p \cdot \sigma }
}
\zeta^s
=
2 m \zeta^{r\dagger} \zeta^s
=
2 m \delta^{rs},
\end{dmath}
which completes the proof.

\section{Other solution.}

Now we seek the other plane wave solution
\begin{dmath}\label{eqn:qftLecture21:260}
\Psi(x) = v(p) e^{i p \cdot x}.
\end{dmath}
The answer is
\begin{dmath}\label{eqn:qftLecture21:280}
v^s(p)
=
\begin{bmatrix}
\sqrt{p \cdot \sigma} \eta^s \\
-\sqrt{p \cdot \osigma} \eta^s \\
\end{bmatrix},
\end{dmath}
where \( \eta^1 = (1,0)^\T, \eta^2 = (0,1)^\T \).
In this case the normalization is
\begin{dmath}\label{eqn:qftLecture21:300}
\begin{aligned}
\vbar^r(p) v^s(p) &= - 2 m \delta^{rs} \\
\vbar^{r \dagger}(p) v^s(p) &= 2 p^0 \delta^{rs}.
\end{aligned}
\end{dmath}
It can be shown that
\begin{dmath}\label{eqn:qftLecture21:320}
\begin{aligned}
\ubar^r(p) v^s(p) &= 0 \\
\vbar^r(p) u^s(p) &= 0 \\
v^\dagger(-\Bp) u(\Bp) &= 0 \\
\end{aligned}
\end{dmath}

It can also be shown that
\begin{dmath}\label{eqn:qftLecture21:340}
\begin{aligned}
u^{r\dagger}(\Bp) u^s(\Bp) &= 2 \omega_\Bp \delta^{sr} \\
v^{r\dagger}(\Bp) v^s(\Bp) &= 2 \omega_\Bp \delta^{sr} \\
u^{r\dagger}(\Bp) v^s(\Bp) &= 0 \\
v^{r\dagger}(-\Bp) u^s(\Bp) &= 0.
\end{aligned}
\end{dmath}

Given

\begin{equation}\label{eqn:qftLecture21:360}
x =
\begin{bmatrix}
x_1 \\
\vdots \\
x_n \\
\end{bmatrix},
\qquad
y =
\begin{bmatrix}
y_1 \\
\vdots \\
y_n \\
\end{bmatrix}.
\end{equation}

\begin{dmath}\label{eqn:qftLecture21:380}
y^\T x = \text{scalar} = \sum_{i = 1}^n x^i y^i
\end{dmath}

We define
\begin{dmath}\label{eqn:qftLecture21:440}
x \otimes y^\T =
\begin{bmatrix}
x_1 \\
\vdots \\
x_n \\
\end{bmatrix}
\otimes
\begin{bmatrix}
y_1 \cdots y_n
\end{bmatrix}
=
\begin{bmatrix}
x_1 y_1 & x_1 y_2 & \cdots & x_1 y_n \\
x_2 y_1 & x_2 y_2 & \cdots & x_2 y_n \\
x_3 y_1 & \ddots  &        &         \\
\vdots  &         &        &         \\
x_n y_1 & \cdots  &        & x_n y_n
\end{bmatrix}
=
\Norm{ x_i y_j }
\end{dmath}

Such a tensor product can be likened to the completeness relation that we express in QM as
\begin{dmath}\label{eqn:qftLecture21:460}
\sum_n \ket{n} \bra{n} \equiv 1.
\end{dmath}

Can also show that
\begin{dmath}\label{eqn:qftLecture21:400}
\begin{aligned}
\sum_{s = 1}^2 u^s(p) \otimes \ubar^s(p) &= \gamma \cdot p + m \\
\sum_{s = 1}^2 v^s(p) \otimes \vbar^s(p) &= \gamma \cdot p - m \\
\end{aligned}
\end{dmath}

\begin{dmath}\label{eqn:qftLecture21:480}
\vbar^s
= v^\dagger \gamma_0
\end{dmath}

Let's show this for the \( v \)'s
\begin{dmath}\label{eqn:qftLecture21:420}
\sum_{s = 1,2}
\begin{bmatrix}
\sqrt{p \cdot \sigma} \eta^s \\
-\sqrt{p \cdot \osigma} \eta^s \\
\end{bmatrix}
\otimes
\begin{bmatrix}
(\eta^s)^\T \sqrt{ p \cdot \sigma } &
-(\eta^s)^\T \sqrt{ p \cdot \sigma }
\end{bmatrix}
\begin{bmatrix}
0 & 1 \\
1 & 0
\end{bmatrix}
=
\sum_{s = 1,2}
\begin{bmatrix}
\sqrt{p \cdot \sigma} \eta^s \\
\sqrt{p \cdot \osigma} \eta^s \\
\end{bmatrix}
\otimes
\begin{bmatrix}
-(\eta^s)^\T \sqrt{ p \cdot \sigma } &
(\eta^s)^\T \sqrt{ p \cdot \sigma }
\end{bmatrix}
=
\sum_{s = 1,2}
\begin{bmatrix}
-\sqrt{p \cdot \sigma} \eta^s \otimes (\eta^s)^\T \sqrt{ p \cdot \sigma } & \sqrt{p \cdot \sigma} \eta^s \otimes (\eta^s)^\T \sqrt{ p \cdot \sigma }  \\
\sqrt{p \cdot \osigma} \eta^s \otimes (\eta^s)^\T \sqrt{ p \cdot \sigma } & -\sqrt{p \cdot \osigma} \eta^s \otimes (\eta^s)^\T \sqrt{ p \cdot \sigma }
\end{bmatrix}
=
\begin{bmatrix}
-\sqrt{ p \cdot \sigma } \sqrt{ p \cdot \osigma } & \sqrt{ p \cdot \sigma } \sqrt{ p \cdot \sigma } \\
\sqrt{ p \cdot \osigma } \sqrt{ p \cdot \osigma } & -\sqrt{ p \cdot \osigma } \sqrt{ p \cdot \osigma }
\end{bmatrix}
=
\begin{bmatrix}
-\sqrt{ p \cdot \sigma p \cdot \osigma } & \sqrt{ p \cdot \sigma p \cdot \sigma } \\
\sqrt{ p \cdot \osigma p \cdot \osigma } & -\sqrt{ p \cdot \osigma p \cdot \osigma }
\end{bmatrix}
=
\begin{bmatrix}
-m & p \cdot \sigma \\
p \cdot \osigma & - m
\end{bmatrix}
=
-m \BOne
+ p^0
\begin{bmatrix}
0 & 1 \\
1 & 0
\end{bmatrix}
+ \Bp
\cdot
\begin{bmatrix}
0 & -\Bsigma \\
\Bsigma & 0
\end{bmatrix}
=
-m + p^\mu \gamma_\mu
\end{dmath}

\section{Lagrangian.}

\begin{dmath}\label{eqn:qftLecture21:500}
\LL_{\text{Dirac}}
= \oPsi i \gamma^0 \partial_0 \Psi + i \oPsi \gamma^j \partial_j \Psi - m \oPsi \Psi
= \Psi^\dagger \gamma^0 i \gamma^0 \partial_0 \Psi + i \Psi \gamma^0 \gamma^j \partial_j \Psi - m \Psi^\dagger \gamma^0 \Psi
= \Psi^\dagger i \dot{\Psi} + i \Psi^\dagger \gamma^0 \gamma^j \partial_j \Psi - m \Psi^\dagger \gamma^0 \Psi
\end{dmath}

\begin{dmath}\label{eqn:qftLecture21:520}
\pi_\psi = \PD{\dot{\Psi}}{\LL} = i \Psi^\dagger
\end{dmath}

Our Hamiltonian density is
\begin{dmath}\label{eqn:qftLecture21:540}
H_{\text{Dirac}}
= \pi_\Psi \dot{\Psi} - \LL
= i \Psi^\dagger \dot{\Psi} - \lr{
\Psi^\dagger i \dot{\Psi} + i \Psi^\dagger \gamma^0 \gamma^j \partial_j \Psi - m \Psi^\dagger \gamma^0 \Psi
}
\end{dmath}

so the Dirac Hamiltonian is
\begin{dmath}\label{eqn:qftLecture21:560}
H
=
\int d^3 x
\lr{
- i \Psi^\dagger \gamma^0 \gamma^j \partial_j \Psi + m \Psi^\dagger \gamma^0 \Psi
}
=
\int d^3 x
\Psi^\dagger
\lr{
- i \gamma^0 \gamma^j \partial_j \Psi + m \gamma^0
}
\Psi.
\end{dmath}

Now we use linear combinations of the solutions we have found
\begin{dmath}\label{eqn:qftLecture21:580}
\Psi(\Bx,0) = \int \frac{d^3 p}{(2 \pi)^3 \sqrt{2 \omega_\Bp} } e^{i \Bp \cdot \Bx} \sum_{s = 1,2} \lr{
   u^s(\Bp) a^s_\Bp
   +
   v^s(-\Bp) b^s_{-\Bp}
},
\end{dmath}
where \( t = 0 \) for convience.

Given the Dirac equation
\begin{dmath}\label{eqn:qftLecture21:600}
\lr{ i \gamma^\mu \partial_\mu - m } \Psi = 0.
\end{dmath}
When we plug in \( \Psi_u = u(p) e^{-i p \cdot x} \) we get \( (\gamma^\mu p_\mu - m) u = 0 \), so
\begin{dmath}\label{eqn:qftLecture21:640}
\gamma^i p_i u = (-\gamma^0 p^0 + m)u
\end{dmath}
whereas when we plug in
\( \Psi_v = v(p) e^{i p \cdot x} \) we get \( (-\gamma^\mu p_\mu - m) u = 0 \), so
\begin{dmath}\label{eqn:qftLecture21:660}
\gamma^j p_j v = -(m + \gamma^0 p_0) v
\end{dmath}

What is the action of \( -i \gamma^0 \gamma^j \partial_j + m \gamma^0 \) on \(\Psi_u \)?  We get

\begin{dmath}\label{eqn:qftLecture21:620}
(-i \gamma^0 \gamma^j \partial_j + m \gamma^0 ) \Psi_u
=
- \gamma^0 \lr{ - \gamma^0 p^0 + m }
u e^{-i p \cdot x}
+ m \gamma^0
u e^{-i p \cdot x}
=
(\gamma^0)^2 p_0 u^{-i p \cdot x}
\end{dmath}

\begin{dmath}\label{eqn:qftLecture21:630}
(-i \gamma^0 \gamma^j \partial_j + m \gamma^0 ) \Psi_u
=
 \gamma^0 \lr{ m + \gamma^0 p_0 }
v e^{i p \cdot x}
+ m \gamma^0
v e^{i p \cdot x}
=
(\gamma^0)^2 p_0 u^{-i p \cdot x}
\end{dmath}

restart
\begin{dmath}\label{eqn:qftLecture21:680}
\lr{ -i \gamma^0 \gamma^j \partial_j + m \gamma^0 } \Psi = - \gamma^0
\lr{ i \gamma^j \partial_j - m } \Psi
\end{dmath}

acting on
\begin{dmath}\label{eqn:qftLecture21:700}
\Psi_u = u(p)
e^{-i p_\mu x^\mu }
\end{dmath}

we get

\begin{dmath}\label{eqn:qftLecture21:720}
\lr{ i \gamma^j \partial_j - m }
e^{-i p_\mu x^\mu }
=
\lr{ i \gamma^j \partial_j - m }
e^{-i p_\mu x^\mu }
=
\lr{ \gamma^j p_j - m } e^{-i p\cdot x}
\end{dmath}

used EOM for u

\begin{dmath}\label{eqn:qftLecture21:740}
\Psi_v = v(p)
e^{i p_\mu x^\mu }
\end{dmath}

\begin{dmath}\label{eqn:qftLecture21:880}
\lr{ i \gamma^j \partial_j - m }
e^{-i p_k x^k }
=
\lr{ -i \gamma^j p_j - m } e^{i p \cdot x}
\end{dmath}

We find

\begin{dmath}\label{eqn:qftLecture21:760}
-\gamma^0 \lr{ i \gamma^j \partial_j - m } \Psi_v = -\gamma^0 \gamma^0 p_0 \Psi_v = -p_0 \Psi_v
\end{dmath}


\boxedEquation{eqn:qftLecture21:780}{
\begin{aligned}
-\gamma^0 \lr{ i \gamma^j \partial_j - m } \Psi_u &= p_0 \Psi_u \\
-\gamma^0 \lr{ i \gamma^j \partial_j - m } \Psi_v &= -p_0 \Psi_v
\end{aligned}
}

Adding time back in the mix the most general wave function is
\begin{dmath}\label{eqn:qftLecture21:800}
\Psi(\Bx, t)
=
\int \frac{d^3 p}{(2 \pi)^3 \sqrt{ 2 \omega_\Bp } }
\lr{
   \sum_s e^{-i p \cdot x} u^s_\Bp a_\Bp^s
+
   \sum_s e^{i p \cdot x} v^s_\Bp b_\Bp^s
},
\end{dmath}

\begin{dmath}\label{eqn:qftLecture21:820}
\Psi(\Bx,0)
=
\int \frac{d^3 p}{(2 \pi)^3 \sqrt{2 \omega_\Bp} } e^{i \Bp \cdot x} \sum_{s = 1,2} \lr{
   u^s(\Bp) a^s_\Bp
   +
   v^s(-\Bp) b^s_{-\Bp}
}
\end{dmath}

\begin{dmath}\label{eqn:qftLecture21:840}
\Psi^\dagger(\Bx,0)
= \int \frac{d^3 q}{(2 \pi)^3 \sqrt{2 \omega_\Bq} } e^{-i \Bq \cdot x} \sum_{r = 1,2} \lr{
   u^{r \dagger}(\Bq) a^{r \dagger}_\Bq
   +
   v^{r \dagger}(-\Bq) b^{r \dagger}_{-\Bq}
}
\end{dmath}

We'll get the following

\begin{dmath}\label{eqn:qftLecture21:860}
H_{\text{Dirac}} =
\int \frac{d^3 p}{(2 \pi)^3 2 \omega_\Bp }
\sum_{r = 1,2} \lr{
   u^r(\Bp) a^r_\Bp
   +
   v^r(-\Bp) b^r_{-\Bp}
}
\times
\sum_{s = 1,2} \lr{
   \omega_\Bp
   u^s(\Bp) a^s_\Bp
   +
   \omega_\Bp
   v^s(-\Bp) b^s_{-\Bp}
}
=
\int \frac{d^3 p}{(2 \pi)^3 2 \omega_\Bp }
\omega_\Bp
\sum_{r,s = 1,2} \lr{
a^{r \dagger}_\Bp a^s_\Bp \underbrace{u^{r \dagger}(\Bp) u^s(\Bp) }_{\delta^{rs} 2 \omega_\Bp}
-
\cancel{a^{r \dagger}_\Bp b^s_{-\Bp} u^{r \dagger}(\Bp) v^s(\Bp) }
+
\cancel{b^{r \dagger}_{-\Bp} a^s_\Bp v^{r \dagger}(-\Bp) u^s(\Bp) }
-
b^{r \dagger}_{-\Bp} b^s_{-\Bp} \underbrace{v^{r \dagger}(\Bp) v^s(\Bp) }_{\delta^{rs} 2 \omega_\Bp}
}
=
\int \frac{d^3 p}{(2 \pi)^3 }
\omega_\Bp
\sum_{r = 1}
\lr{
(a_\Bp^{r \dagger} a_\Bp^s - b_\Bp^{r \dagger} b_\Bp^s
}.
\end{dmath}
Note that we have a minus sign in the Hamiltonian, so there is no bound to the energy from below!  This makes interpretation of the \( a_p \)'s and \( b_p \)'s as the familiar raising and lowering operators that we know, we end up in trouble.

We can save the day, making the ``Dirac sea'' argument\footnote{There was a long discussion of this topic that I was not able to capture in my notes.}.

It will turn out that our operators are Fermions.
Let
\begin{dmath}\label{eqn:qftLecture21:900}
\begin{aligned}
b^{s \dagger}_p &= tb_p \\
b^s &= \tilde{b^{s \dagger}}_p
\end{aligned}
\end{dmath}

some properties are
\begin{dmath}\label{eqn:qftLecture21:920}
\begin{aligned}
(a^s_p)^2 &= 0 \\
(a^{s \dagger}_p)^2 &= 0 \\
(b^s_p)^2 &= 0 \\
(b^{s \dagger}_p)^2 &= 0
\end{aligned}
\end{dmath}

\begin{dmath}\label{eqn:qftLecture21:940}
\begin{aligned}
\symmetric{a^s_\Bp}{a^{r \dagger}_\Bq} &= \delta^{sr} \deltathree(\Bp - \Bq) \\
\symmetric{b^s_\Bp}{b^{r \dagger}_\Bq} &= \delta^{sr} \deltathree(\Bp - \Bq)
\end{aligned}
\end{dmath}

and all other anticommutators are zero
\begin{dmath}\label{eqn:qftLecture21:960}
\symmetric{a^r}{b^s} =
\symmetric{a^r}{b^{s\dagger}} =
\symmetric{a^{r\dagger}}{b^s} =
\symmetric{a^{r\dagger}}{b^{s\dagger}} = 0.
\end{dmath}

\begin{dmath}\label{eqn:qftLecture21:980}
H_{\text{Dirac}}
=
\int \frac{d^3 p}{(2 \pi)^3 }
\omega_\Bp
\sum_{r = 1}
\lr{
\tilde{a}_\Bp^r a_\Bp^s - \tilde{b}_\Bp^r b_\Bp^s
}
=
\int \frac{d^3 p}{(2 \pi)^3 }
\omega_\Bp
\sum_{s = 1}
\lr{
a_\Bp^{r \dagger} a_\Bp^s
+ \tilde{b}_\Bp^s \tilde{b}^{s\dagger}_\Bp + \delta^{ss} \deltathree( \Bp - \Bp )
}
=
\int \frac{d^3 p}{(2 \pi)^3 }
\lr{
\omega_\Bp
\sum_{s = 1}
\lr{
a_\Bp^{ r \dagger} a_\Bp^s
+ \tilde{b}_\Bp^s \tilde{b}^{s\dagger}_\Bp
}
- 4 V_3 \frac{\omega_\Bp}{2}
}
\end{dmath}

We'll end up dropping the vacuum energy term.  We'll end up labelling the \( a \)'s as the operators associated with electrons, and the \( b \)'s with antielectrons.

%}
%\EndArticle
\EndNoBibArticle
