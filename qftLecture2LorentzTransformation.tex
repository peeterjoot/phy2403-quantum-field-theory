%
% Copyright © 2018 Peeter Joot.  All Rights Reserved.
% Licenced as described in the file LICENSE under the root directory of this GIT repository.
%
%{
\section{Lorentz transformations.}
\index{Lorentz transformations}

The goal, perhaps not for today, is to study the simplest (relativistic) scalar field theory.  First studied classically, and then consider such a quantum field theory.
How is relativity implemented when we write the Lagrangian and action?

Our first step must be to consider Lorentz transformations and the Lorentz group.

Spacetime (Minkowski space) is \R{3,1} (or \R{d-1,1}).  Our coordinates are
\begin{equation}\label{eqn:qftLecture2:340}
(c t, x^1, x^2, x^3) = (c t, \Br).
\end{equation}

Here, we've scaled the time scale by \( c \) so that we measure time and space in the same dimensions.  We write this as
\begin{equation}\label{eqn:qftLecture2:360}
x^\mu = (x^0, x^1, x^2, x^3),
\end{equation}
%
where \( \mu = 0, 1, 2, 3 \), and call this a ``4-vector''.  These are called the space-time coordinates of an event, which tell us where and when an event occurs.

For two events whose spacetime coordinates differ by \( dx^0, dx^1, dx^2, dx^3 \) we introduce the notion of a space time \underline{interval}
\begin{equation}\label{eqn:qftLecture2:380}
\begin{aligned}
ds^2
&= c^2 dt^2
- (dx^1)^2
- (dx^2)^2
- (dx^3)^2 \\
&=
\sum_{\mu, \nu = 0}^3 g_{\mu\nu} dx^\mu dx^\nu
\end{aligned}
\end{equation}

Here \( g_{\mu\nu} \) is the Minkowski space metric, an object with two indexes that run from 0-3.  i.e. this is a diagonal matrix
\begin{equation}\label{eqn:qftLecture2:400}
g_{\mu\nu} \sim
\begin{bmatrix}
1 & 0 & 0 & 0 \\
0 & -1 & 0 & 0 \\
0 & 0 & -1 & 0 \\
0 & 0 & 0 & -1 \\
\end{bmatrix}
\end{equation}

i.e.
\begin{equation}\label{eqn:qftLecture2:420}
\begin{aligned}
g_{00} &= 1 \\
g_{11} &= -1 \\
g_{22} &= -1 \\
g_{33} &= -1 \\
\end{aligned}
\end{equation}

We will use the Einstein summation convention, where any repeated upper and lower indexes are considered summed over.  That is \cref{eqn:qftLecture2:380} is written with an implied sum
\begin{equation}\label{eqn:qftLecture2:440}
ds^2 = g_{\mu\nu} dx^\mu dx^\nu.
\end{equation}

Explicit expansion:
\begin{equation}\label{eqn:qftLecture2:460}
\begin{aligned}
ds^2 = g_{\mu\nu} dx^\mu dx^\nu
&=
g_{00} dx^0 dx^0
+g_{11} dx^1 dx^1
+g_{22} dx^2 dx^2
+g_{33} dx^3 dx^3 \\
&=
(1) dx^0 dx^0
+ (-1) dx^1 dx^1
+ (-1) dx^2 dx^2
+ (-1) dx^3 dx^3.
\end{aligned}
\end{equation}

Recall that rotations (with orthogonal matrix representations) are transformations that leave the dot product unchanged, that is
\begin{equation}\label{eqn:qftLecture2:480}
\begin{aligned}
(R \Bx) \cdot (R \By)
&= \Bx^\T R^\T R \By \\
&= \Bx^\T \By \\
&= \Bx \cdot \By,
\end{aligned}
\end{equation}
%
where \( R \) is a rotation orthogonal 3x3 matrix.  The set of such transformations that leave the dot product unchanged have orthonormal matrix representations \( R^\T R = 1 \).  We call the set of such transformations that have unit determinant the \(\SO{3}\) group.

We call a Lorentz transformation, if it is a linear transformation acting on 4 vectors that leaves the spacetime interval (i.e. the inner product of 4 vectors) invariant.  That is, a transformation that leaves
\begin{equation}\label{eqn:qftLecture2:500}
x^\mu y^\nu g_{\mu\nu} = x^0 y^0 - x^1 y^1 - x^2 y^2 - x^3 y^3
\end{equation}
unchanged.

Suppose that transformation has a 4x4 matrix form
\begin{equation}\label{eqn:qftLecture2:520}
{x'}^\mu = {\Lambda^\mu}_\nu x^\nu
\end{equation}

For an example of a possible \( \Lambda \), consider the transformation sketched in
\cref{fig:Lecture2:Lecture2Fig3}.
\imageFigure{../figures/phy2403-quantum-field-theory/Lecture2Fig3}{Boost transformation.}{fig:Lecture2:Lecture2Fig3}{0.2}
We know that boost has the form
\begin{equation}\label{eqn:qftLecture2:540}
\begin{aligned}
x &= \frac{x' + v t'}{\sqrt{1 - v^2/c^2}} \\
y &= y' \\
z &= z' \\
t &= \frac{t' + (v/c^2) x'}{\sqrt{1 - v^2/c^2}} \\
\end{aligned}
\end{equation}
(this is a boost along the x-axis, not y as I'd drawn),
or
\begin{equation}\label{eqn:qftLecture2:560}
\begin{bmatrix}
c t \\
x \\
y \\
z
\end{bmatrix}
=
\begin{bmatrix}
\inv{\sqrt{1 - v^2/c^2}} & \frac{v/c}{\sqrt{1 - v^2/c^2}} & 0 & 0 \\
\frac{v/c}{\sqrt{1 - v^2/c^2}} & \frac{1}{\sqrt{1 - v^2/c^2}} & 0 & 0 \\
0 & 0 & 1 & 0 \\
0 & 0 & 0 & 1 \\
\end{bmatrix}
\begin{bmatrix}
c t' \\
x' \\
y' \\
z'
\end{bmatrix}
\end{equation}

Other examples include rotations (\({\lambda^0}_0 = 1\) zeros in \( {\lambda^0}_k, {\lambda^k}_0 \), and a rotation matrix in the remainder.)
% submatrix:
%\begin{equation}\label{eqn:qftLecture2:580}
%\begin{bmatrix}
%1 & 0 & 0 & 0
%0 &
%0 &    R
%0 &
%\end{bmatrix}
%\end{equation}

Back to Lorentz transformations (\(\text{SO}(1,3)^+\)), let
\begin{equation}\label{eqn:qftLecture2:600}
\begin{aligned}
{x'}^\mu    &= {\Lambda^\mu}_\nu x^\nu \\
{y'}^\kappa &= {\Lambda^\kappa}_\rho y^\rho
\end{aligned}
\end{equation}

The dot product
\begin{equation}\label{eqn:qftLecture2:620}
\begin{aligned}
g_{\mu \kappa}
{x'}^\mu
{y'}^\kappa
&=
g_{\mu \kappa}
{\Lambda^\mu}_\nu
{\Lambda^\kappa}_\rho
x^\nu
y^\rho \\
&=
g_{\nu\rho}
x^\nu
y^\rho,
\end{aligned}
\end{equation}
where the last step introduces the invariance requirement of the transformation.  That is
%\begin{equation}\label{eqn:qftLecture2:640}
\boxedEquation{eqn:qftLecture2:640}{
g_{\nu\rho}
=
g_{\mu \kappa}
{\Lambda^\mu}_\nu
{\Lambda^\kappa}_\rho.
}
%\end{equation}
\paragraph{Upper and lower indexes}
\index{upper indexes}
\index{lower indexes}
We've defined
\begin{equation}\label{eqn:qftLecture2:660}
x^\mu = (t, x^1, x^2, x^3).
\end{equation}

We could also define a four vector with lower indexes
\begin{equation}\label{eqn:qftLecture2:680}
x_\nu = g_{\nu\mu} x^\mu = (t, -x^1, -x^2, -x^3).
\end{equation}
That is
\begin{equation}\label{eqn:qftLecture2:700}
\begin{aligned}
x_0 &= x^0 \\
x_1 &= -x^1 \\
x_2 &= -x^2 \\
x_3 &= -x^3,
\end{aligned}
\end{equation}
which allows us to write the dot product as simply \( x^\mu y_\mu \).

We can also define a metric tensor with upper indexes
\begin{equation}\label{eqn:qftLecture2:401}
g^{\mu\nu} \sim
\begin{bmatrix}
1 & 0 & 0 & 0 \\
0 & -1 & 0 & 0 \\
0 & 0 & -1 & 0 \\
0 & 0 & 0 & -1 \\
\end{bmatrix}
\end{equation}
This is the inverse matrix of \( g_{\mu\nu} \), and it satisfies
\begin{equation}\label{eqn:qftLecture2:720}
g^{\mu \nu} g_{\nu\rho} = {\delta^\mu}_\rho
\end{equation}

Exercise: Check:
\begin{equation}\label{eqn:qftLecture2:740}
g_{\mu\nu} x^\mu y^\nu = x_\nu y^\nu = x^\nu y_\nu
= g^{\mu\nu} x_\mu y_\nu = {\delta^\mu}_\nu x_\mu y^\nu
\end{equation}

Class ended around this point, but it appeared that we were heading this direction:

Returning to the Lorentz invariant and multiplying both sides of
\cref{eqn:qftLecture2:640} with an inverse Lorentz transformation \( \Lambda^{-1} \), we find
\begin{equation}\label{eqn:qftLecture2:760}
\begin{aligned}
g_{\nu\rho}
{\lr{\Lambda^{-1}}^\rho}_\alpha
&=
g_{\mu \kappa}
{\Lambda^\mu}_\nu
{\Lambda^\kappa}_\rho
{\lr{\Lambda^{-1}}^\rho}_\alpha \\
&=
g_{\mu \kappa}
{\Lambda^\mu}_\nu
{\delta^\kappa}_\alpha \\
&=
g_{\mu \alpha}
{\Lambda^\mu}_\nu,
\end{aligned}
\end{equation}
or
\begin{equation}\label{eqn:qftLecture2:780}
\lr{\Lambda^{-1}}_{\nu \alpha} = \Lambda_{\alpha \nu}.
\end{equation}
This is clearly analogous to \( R^\T = R^{-1} \), although the index notation obscures things considerably.  Prof. Poppitz said that next week this would all lead to showing that the determinant of any Lorentz transformation was \( \pm 1 \).

For what it's worth, it seems to me that this index notation makes life a lot harder than it needs to be, at least for a matrix related question (i.e. determinant of the transformation).  In matrix/column-(4)-vector notation, let \(x' = \Lambda x, y' = \Lambda y\) be two four vector transformations, then
\begin{equation}\label{eqn:qftLecture2:800}
x' \cdot y' = {x'}^T G y' = (\Lambda x)^T G \Lambda y = x^T ( \Lambda^T G \Lambda) y = x^T G y.
\end{equation}
so
\boxedEquation{eqn:qftLecture2:820}{
\Lambda^T G \Lambda = G.
}
Taking determinants of both sides gives
\begin{equation}\label{eqn:qftLecture2LorentzTransformation:n}
-(\det(\Lambda))^2 = -1,
\end{equation}
so \(\det(\Lambda) = \pm 1\).
%}
