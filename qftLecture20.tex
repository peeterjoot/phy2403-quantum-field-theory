%
% Copyright � 2018 Peeter Joot.  All Rights Reserved.
% Licenced as described in the file LICENSE under the root directory of this GIT repository.
%
%{
%\input{../latex/blogpost.tex}
%\renewcommand{\basename}{qftLecture20}
%\renewcommand{\dirname}{notes/phy2403/}
%\newcommand{\keywords}{PHY2403H}
%\input{../latex/peeter_prologue_print2.tex}
%
%%\usepackage{phy2403}
%\usepackage{peeters_braket}
%\usepackage{peeters_layout_exercise}
%\usepackage{peeters_figures}
%\usepackage{mathtools}
%\usepackage{siunitx}
%\usepackage{macros_cal} % LL
%
%\newcommand{\ultensor}[3]{{{#1}^{#2}}_{#3}}
%\newcommand{\deltathree}[0]{\delta^{(3)}}
%\newcommand{\deltafour}[0]{\delta^{(4)}}
%\newcommand{\ulLambda}[2]{\ultensor{\Lambda}{#1}{#2}}
%\newcommand{\ulDelta}[2]{\ultensor{\delta}{#1}{#2}}
%
%\beginArtNoToc
%\generatetitle{PHY2403H Quantum Field Theory.  Lecture 20: Dirac Lagrangian, spinor solutions to the KG equation, Dirac matrices, plane wave solution, helicity.  Taught by Prof.\ Erich Poppitz}
%%\chapter{Dirac Lagrangian, spinor solutions to the KG equation, Dirac matrices, plane wave solution, helicity}
\index{Dirac Lagrangian}
\index{spinor}
\index{Dirac matrix}
\index{plane wave}
\index{helicity}
\label{chap:qftLecture20}

%%Peeter's lecture notes from class.  These may be incoherent and rough.
%%
%%These are notes for the UofT course PHY2403H, Quantum Field Theory, taught by Prof. Erich Poppitz, covering \textchapref{{1}} \citep{peskin1995introduction} content.

%\paragraph{DISCLAIMER: Very rough notes from class, with some additional side notes.}
%
%These are notes for the UofT course PHY2403H, Quantum Field Theory, taught by Prof. Erich Poppitz, fall 2018.
%%, covering \textchapref{{1}} \citep{peskin1995introduction} content.
%
\section{Review.}

Last time we
\begin{itemize}
\item
introduced the Clifford algebra Dirac matrix (gamma matrices) elements satisfying
\begin{equation}\label{eqn:qftLecture20:40}
\symmetric{\gamma^\mu}{\gamma^\nu} = 2 g^{\mu\nu},
\end{equation}
where we use the Weyl representation
\begin{equation}\label{eqn:qftLecture20:60}
\begin{aligned}
\gamma^0 &=
\begin{bmatrix}
0 & 1 \\
1 & 0
\end{bmatrix} \\
\gamma^k &=
\begin{bmatrix}
0 & \sigma^k \\
-\sigma^k & 0
\end{bmatrix}.
\end{aligned}
\end{equation}
In particular \( (\gamma^0)^2 = 1, (\gamma^0)^\dagger = \gamma^0 \).
\item and left off after
showing that the Dirac Lagrangian
\begin{equation}\label{eqn:qftLecture20:20}
\LL = \overbar{\Psi}\lr{ i \gamma^\mu \partial_\mu - m } \Psi,
\end{equation}
where \( \overbar{\Psi} = \Psi^\dagger \gamma^0 \),
is Lorentz invariant.  We argued that a single spinor can only describe a massless field, and that a two spinor construction can be used for a massive field.  We skipped from there to the Dirac Lagrangian above.
\end{itemize}

\section{Dirac equation.}
\index{Dirac equation}

Varying the Dirac action (\cref{problem:qftLecture20:10}), we find the Dirac equation
\boxedEquation{eqn:qftLecture20:80}{
\lr{ i \gamma^\mu \partial_\mu - m } \Psi = 0.
}

\maketheorem{Dirac equations as Klein-Gordon solutions.}{thm:qftLecture20:760}{
If \( \Psi \) obeys \cref{eqn:qftLecture20:80}, the Dirac equation, then
\( \Psi \) is a solution to the Klein-Gordon equation.
} % theorem

\begin{proof}
\Cref{thm:qftLecture20:760} follows by pre-multiplying by a sort of ``conjugate'' operator\footnote{Q: Is there a name for such a conjugation operation?} \( i \gamma^\mu + m \) to find
\begin{equation}\label{eqn:qftLecture20:100}
\begin{aligned}
0
&=
\lr{ i \gamma^\mu \partial_\mu + m }
\lr{ i \gamma^\mu \partial_\mu - m } \Psi
\\&=
\lr{ - \gamma^\mu \gamma_\nu \partial_{\mu} \partial_\nu -m^2 } \Psi
\\&=
\lr{
   - \inv{2} \lr{
      \gamma^\mu \gamma_\nu +
      \gamma^\nu \gamma_\mu
   }
   \partial_{\mu} \partial_\nu -m^2
} \Psi
\\&=
\lr{
   - g^{\mu\nu} \partial_{\mu} \partial_\nu -m^2
} \Psi
\\&=
-\lr{
   \partial_{\mu} \partial^\mu + m^2
} \Psi
\\&=
-\lr{ \partial_{00} - \spacegrad^2 + m^2 } \Psi,
\end{aligned}
\end{equation}
which is a Klein-Gordon equation for \( \Psi \).
\end{proof}

\paragraph{Goal:}

Expand \( \Psi(\Bx, t) \) in a basis of solutions of the Dirac equation.  Call the coefficients \( a, b, \cdots \).  This will be like
\begin{equation}\label{eqn:qftLecture20:120}
\phi \sim \int \frac{d^3 p}{(2 \pi)^3 2 \omega_\Bp} \lr{
e^{i p \cdot x} a_\Bp^\dagger
+
e^{-i p \cdot x} a_\Bp
}
\end{equation}

As with the scalar field, let's look for plane wave solutions.  We'll first look for solutions of the form
\begin{equation}\label{eqn:qftLecture20:140}
\Psi(x) = u(p) e^{-i p \cdot x},
\end{equation}
where \( p^2 = m^2 \), \( p^0 > 0\, \forall p \).  Plugging into
\cref{eqn:qftLecture20:80}
we find
\begin{equation}\label{eqn:qftLecture20:160}
\lr{ \gamma^\mu p_\mu - m } u(p) = 0.
\end{equation}

Let's write this out explicitly for exposition, first noting that
\begin{equation}\label{eqn:qftLecture20:860}
\begin{aligned}
\gamma^\mu p_\mu
&= \gamma^0 p_0 + \gamma^k p_k
\\&=
p_0
\begin{bmatrix}
0 & \sigma^0 \\
\sigma^0 & 0
\end{bmatrix}
+
p_k
\begin{bmatrix}
0 & \sigma^k \\
-\sigma^k & 0
\end{bmatrix}
\\&=
\begin{bmatrix}
0 & p^0 \sigma^0 - \Bsigma \cdot \Bp \\
p^0 \sigma^0 + \Bsigma \cdot \Bp & 0
\end{bmatrix},
\end{aligned}
\end{equation}
so \cref{eqn:qftLecture20:160} becomes
\begin{equation}\label{eqn:qftLecture20:880}
\begin{aligned}
0
&=
\begin{bmatrix}
-m & p^0 \sigma^0 - \Bsigma \cdot \Bp \\
p^0 \sigma^0 + \Bsigma \cdot \Bp & -m
\end{bmatrix}
\begin{bmatrix}
u_1(p) \\
u_2(p) \\
\end{bmatrix}
\\&=
\begin{bmatrix}
-m & p \cdot \sigma \\
p \cdot \overbar{\sigma} & -m
\end{bmatrix}
\begin{bmatrix}
u_1(p) \\
u_2(p) \\
\end{bmatrix},
\end{aligned}
\end{equation}
where the following handy shorthand\footnote{Assuming I wrote this down correctly, this follows the usual convention \( x \cdot p = x^\mu p_\mu = x^0 p^0 - \Bx \cdot \Bp \).
I had some doubt that I got the signs right in my notes from class, since a peek at
\citep{peskin1995introduction} seemingly showed the opposite sign convention where \( \sigma \cdot x \) was first defined, namely eq. 3.41/3.43.  There they write \( \sigma \cdot \partial = \partial_0 + \Bsigma \cdot \spacegrad \), not \( \sigma \cdot \partial = \partial_0 - \Bsigma \cdot \spacegrad\).  What explains this is the fact that the four-gradient in coordinate form should really considered a lower index quantity (\(\partial_\mu\)), so in the scalar+vector tuple form, we should write \( \partial^\mu = (\partial^0, -\spacegrad) \), or \( \partial_\mu = (\partial_0, \spacegrad) \).  This means that \( \sigma \cdot \partial = \sigma^0 \partial^0 - \Bsigma \cdot (-\spacegrad) = \partial_0 + \Bsigma \cdot \spacegrad \).  Having a tuple notation that can be used to represent either lower or upper index quantities is very confusing, and probably justifies avoiding that notation for any lower index quantity whenever possible for clarity!} has been used to group the momentum related block matrices
\begin{equation}\label{eqn:qftLecture20:300}
\begin{aligned}
p \cdot \sigma &= p^0 \sigma^0 - \Bp \cdot \Bsigma \\
p \cdot \overbar{\sigma} &= p^0 \sigma^0 + \Bp \cdot \Bsigma.
\end{aligned}
\end{equation}

Note that these \( p \cdot \sigma, p \cdot \overbar{\sigma} \)'s are both block matrices.  In particular
\begin{equation}\label{eqn:qftLecture20:320}
\begin{aligned}
p \cdot \sigma
&= p_\mu \sigma^\mu
\\&=
\begin{bmatrix}
p_0 + p_3   & p_1 - i p_2 \\
p_1 + i p_2 & p_0 - p_3
\end{bmatrix}.
\end{aligned}
\end{equation}

The question is what \( u's \) obey such an equation.

We can gain some insight by first considering the rest frame, where \( \Bp = 0, p^0 = m \).  Going back to \cref{eqn:qftLecture20:160}, the rest frame Dirac equation becomes
\begin{equation}\label{eqn:qftLecture20:180}
\begin{aligned}
0
&=
\lr{ \gamma^0 p_0 - m } u
\\&=
m \lr{ \gamma^0 - 1 } u.
\end{aligned}
\end{equation}
Our block matrix equation is now reduced to a set of \( 2 \times 2 \) identity matrices
\begin{equation}\label{eqn:qftLecture20:220}
0 =
\begin{bmatrix}
-1 & 1 \\
1 & -1
\end{bmatrix}
u(\Bp = 0).
\end{equation}
The solution space is given by
\begin{equation}\label{eqn:qftLecture20:240}
\begin{bmatrix}
-1 & 1 \\
1 & -1
\end{bmatrix}
\begin{bmatrix}
\zeta \\
\zeta \\
\end{bmatrix}
=0,
\end{equation}
where \( \zeta \) is itself a \( 2 \times 1 \) column matrix, say
\begin{equation}\label{eqn:qftLecture20:380}
\zeta =
\begin{bmatrix}
\zeta_1 \\
\zeta_2
\end{bmatrix}.
\end{equation}
so our solutions are all proportional to column
\begin{equation}\label{eqn:qftLecture20:260}
u(\Bp = 0) \sim
\sqrt{m}
\begin{bmatrix}
\zeta \\
\zeta
\end{bmatrix}.
\end{equation}
We'll figure out the desired normalization later\footnote{\citep{peskin1995introduction} says of this that we pick a normalization with \( \zeta^\dagger \zeta = 1 \).}, and have added a \( \sqrt{m} \) factor into the mix for later convenience.
\Cref{eqn:qftLecture20:260} is a solution of the Dirac equation
in the rest frame where \( \Bp = 0 \).  A solution in a frame where \( \Bp \ne 0 \) can be found using a boost.
We won't work that out explicitly here, but instead show the answer and argue that it must be valid, but the interested student can find that boost calculated explicitly in \citep{peskin1995introduction}.  A nice treatment of such a boost can also be found in \citep{tobiasQFTL15Dirac} supplemented by \cref{problem:squarerootpsigma:1}.

\paragraph{Claim:} in a boosted frame where \( \Bp \ne 0 \) solution is
\begin{equation}\label{eqn:qftLecture20:280}
u(p) =
\begin{bmatrix}
\sqrt{p \cdot \sigma} \zeta \\
\sqrt{p \cdot \overbar{\sigma}} \zeta
\end{bmatrix}
\end{equation}
What do we mean by these square roots?  Since \(
p \cdot \sigma,
p \cdot \overbar{\sigma} \) are both Hermitian \( 2 \times 2 \) matrices, we can define the square root as the matrix of the square roots of the eigenvalues.
\paragraph{Check:}
In the rest frame
\begin{equation}\label{eqn:qftLecture20:340}
\begin{aligned}
\evalbar{
\sqrt{ p \cdot \sigma }
}{ \Bp = 0 }
&=
\evalbar{
\sqrt{ p \cdot \overbar{\sigma} }
}{ \Bp = 0 }
\\&=
\begin{bmatrix}
\sqrt{p_0} & 0 \\
0 & \sqrt{p_0}
\end{bmatrix}
\\&=
\begin{bmatrix}
\sqrt{m} & 0 \\
0 & \sqrt{m}
\end{bmatrix},
\end{aligned}
\end{equation}
so
\begin{equation}\label{eqn:qftLecture20:940}
\begin{aligned}
u(p)
&=
\sqrt{m}
\begin{bmatrix}
\sigma^0 \zeta \\
\sigma^0 \zeta
\end{bmatrix}
\\&=
\sqrt{m}
\begin{bmatrix}
\zeta \\
\zeta
\end{bmatrix},
\end{aligned}
\end{equation}
as we already found.

We claim that the structure of the boost is
\begin{equation}\label{eqn:qftLecture20:360}
u(\Bp) =
\begin{bmatrix}
\frac{\sqrt{p \cdot \sigma}}{m} & 0 \\
0 & \frac{\sqrt{p \cdot \overbar{\sigma}}}{m}
\end{bmatrix}
\mathLabelBox[ labelstyle={below of=m\themathLableNode, below of=m\themathLableNode} ]
{
   \sqrt{m}
   \begin{bmatrix}
   \zeta \\
   \zeta
   \end{bmatrix}
}
{
\(u(\Bp = 0)\)
}
\end{equation}
We'd like to check that this is an element of \(SL{2}\).
We'll also see in the end that we don't have to calculate these square roots, since we always end up with two spinors and when all is said we end up with products of these roots.

\makelemma{Determinant of square root.}{lemma:qftLecture20:3}{
If matrix \( A \) is diagonalizable, then \( \det \sqrt{A} = \sqrt{ \det A } \).
} % lemma

\begin{proof}
Suppose that
\begin{equation}\label{eqn:qftLecture20:900}
A = U \diag( \lambda_1, \cdots \lambda_n ) U^\dagger,
\end{equation}
then
\begin{equation}\label{eqn:qftLecture20:920}
\begin{aligned}
\det \sqrt{A}
&=
\det \lr{
U \diag( \sqrt{\lambda_1}, \cdots, \sqrt{\lambda_n} ) U^\dagger
}
\\&=
\prod_j \sqrt{\lambda_j}
\\&=
\sqrt{
\prod_j \lambda_j
}
\\&=
\sqrt{ \det A }.
\end{aligned}
\end{equation}
\end{proof}

\makelemma{Determinant of \( p \cdot \sigma \).}{lemma:qftLecture20:7}{
\begin{equation*}
\det \frac{\sqrt{p \cdot \sigma}}{\sqrt{m}} =1
\end{equation*}
} % lemma

\begin{proof}
\begin{equation}\label{eqn:qftLecture20:420}
\begin{aligned}
\det \frac{\sqrt{p \cdot \sigma}}{\sqrt{m}}
&=
\sqrt{
\det \frac{(p \cdot \sigma)}{m}
},
\\&=
\sqrt{
   \det
   \inv{m}
   \begin{bmatrix}
   p^0 + p^3 & -p_1 + i p_2 \\
   -p_1 -i p_2 & p^0 + p^3
   \end{bmatrix}
}
\\&=
\sqrt{ \inv{m^2} \lr{ (p^0)^2 - \Bp^2 } }
\\&=
\sqrt{ \frac{m^2}{m^2} }
\\&=
1.
\end{aligned}
\end{equation}
\end{proof}

\makelemma{\((p \cdot \sigma)(p \cdot \overbar{\sigma})\).}{lemma:qftLecture20:17}{
\begin{equation*}
(p \cdot \sigma)(p \cdot \overbar{\sigma}) = m^2.
\end{equation*}
} % lemma

\begin{proof}
\begin{equation}\label{eqn:qftLecture20:480}
\begin{aligned}
(p \cdot \sigma)(p \cdot \overbar{\sigma}) = \lr{ p^0 - \Bp \cdot \Bsigma }\lr{ p^0 + \Bp \cdot \Bsigma }
&=
(p^0)^2 - \lr{ \Bp \cdot \Bsigma }^2
\\&=
(p^0)^2 - \Bp^2
\\&=
m^2.
\end{aligned}
\end{equation}
\end{proof}

\maketheorem{\( u(p) \) is a solution to the Dirac equation.}{thm:qftLecture20:13}{
\Cref{eqn:qftLecture20:280} is a solution of \cref{eqn:qftLecture20:80}, the Dirac equation.
} % theorem

\begin{proof}
\begin{equation}\label{eqn:qftLecture20:460}
\begin{aligned}
\lr{ \gamma^\mu p_\mu -m } u(p)
&=
\begin{bmatrix}
-m & p \cdot \sigma \\
p \cdot \overbar{\sigma} & -m
\end{bmatrix}
\begin{bmatrix}
\sqrt{ p \cdot \sigma } \zeta \\
\sqrt{ p \cdot \overbar{\sigma} \zeta }
\end{bmatrix}
\\&=
\begin{bmatrix}
\lr{ -m \sqrt{p \cdot \sigma} + p \cdot \sigma \sqrt{p \cdot \overbar{\sigma}} } \zeta \\
\lr{ p \cdot \overbar{\sigma} \sqrt{ p \cdot \sigma } - m \sqrt{p \cdot \overbar{\sigma}} } \zeta \\
\end{bmatrix}
\\&=
\begin{bmatrix}
\sqrt{p \cdot \sigma}
\lr{ -m + \sqrt{p \cdot \sigma} \sqrt{p \cdot \overbar{\sigma}} } \zeta \\
\sqrt{ p \cdot \overbar{\sigma}}
\lr{ p \sqrt{\cdot \overbar{\sigma}} \sqrt{ p \cdot \sigma } - m } \zeta \\
\end{bmatrix}
\\&=
\begin{bmatrix}
\sqrt{p \cdot \sigma}
\lr{ -m + \sqrt{m^2} } \zeta \\
\sqrt{ p \cdot \overbar{\sigma}}
\lr{ p \sqrt{ m^2 } - m } \zeta \\
\end{bmatrix}
\\&= 0.
\end{aligned}
\end{equation}
\end{proof}

\paragraph{Summary:}
For \( p^0 > 0, p^2  = m^2 \)
\begin{equation}\label{eqn:qftLecture20:500}
\Psi(x) = e^{-i p \cdot x } u(p)
\end{equation}
\begin{equation}\label{eqn:qftLecture20:520}
u(p) =
\begin{bmatrix}
\sqrt{p \cdot \sigma} \zeta \\
\sqrt{p \cdot \overbar{\sigma}} \zeta \\
\end{bmatrix}
\end{equation}

\paragraph{Example:}
\begin{equation}\label{eqn:qftLecture20:540}
p = \lr{ E, 0, 0, p^3 }
\end{equation}

We have
\begin{equation}\label{eqn:qftLecture20:960}
\begin{aligned}
\sqrt{\sigma \cdot p}
&=
\sqrt{
   E - p^3 \sigma^3
}
\\&=
\sqrt{\lr{
\begin{bmatrix}
E - p^3 & 0 \\
0       & E + p^3
\end{bmatrix}
}}
\\&=
\begin{bmatrix}
\sqrt{E - p^3} & 0 \\
0       & \sqrt{E + p^3}
\end{bmatrix}.
\end{aligned}
\end{equation}
Similarly
\begin{equation}\label{eqn:qftLecture20:1000}
\sqrt{\overbar{\sigma} \cdot p}
=
\begin{bmatrix}
\sqrt{E + p^3} & 0 \\
0       & \sqrt{E - p^3}
\end{bmatrix},
\end{equation}
so
\begin{equation}\label{eqn:qftLecture20:980}
u(p) =
\begin{bmatrix}
\begin{bmatrix}
\sqrt{E - p^3} & 0 \\
0       & \sqrt{E + p^3}
\end{bmatrix} \zeta \\
\begin{bmatrix}
\sqrt{E + p^3} & 0 \\
0       & \sqrt{E - p^3}
\end{bmatrix} \zeta
\end{bmatrix}
\end{equation}

Suppose we let
\begin{equation}\label{eqn:qftLecture20:740}
\zeta =
\begin{bmatrix}
1 \\
0
\end{bmatrix}
\end{equation}
we are left with
\begin{equation}\label{eqn:qftLecture20:580}
u(p) =
\begin{bmatrix}
\sqrt{ E - p^3}
\begin{bmatrix}
1 \\
0
\end{bmatrix}
\\
\sqrt{ E + p^3}
\begin{bmatrix}
1 \\
0
\end{bmatrix}
\end{bmatrix}
\end{equation}

Alternatively for \( \zeta =
\begin{bmatrix}
0 \\ 1
\end{bmatrix} \)
\begin{equation}\label{eqn:qftLecture20:600}
u(p) =
\begin{bmatrix}
\sqrt{ E + p^3}
\begin{bmatrix}
0 \\
1
\end{bmatrix}
\\
\sqrt{ E - p^3}
\begin{bmatrix}
0 \\
1
\end{bmatrix}
\end{bmatrix}
\end{equation}

If we pick \( p_3 = E  \)\footnote{we can boost a massive particle to be arbitrarily close to \( p_3 = E \).}, then we find two solutions
\begin{equation}\label{eqn:qftLecture20:620}
\evalbar{
u(p)
}{\zeta = (1,0)^\T, p_3 = E}
=
\begin{bmatrix}
0
\begin{bmatrix}
1 \\ 0
\end{bmatrix} \\
\sqrt{2 E}
\begin{bmatrix}
1 \\ 0
\end{bmatrix}
\end{bmatrix}
\end{equation}
and
\begin{equation}\label{eqn:qftLecture20:640}
\evalbar{
u(p)
}{\zeta = (0,1)^\T, p_3 = E}
=
\begin{bmatrix}
\sqrt{2 E}
\begin{bmatrix}
0 \\ 1
\end{bmatrix} \\
0
\begin{bmatrix}
0 \\ 1
\end{bmatrix} \\
\end{bmatrix}
\end{equation}
\section{Helicity.}
\index{helicity}
Let \( h \) (the helicity) be
\begin{equation}\label{eqn:qftLecture20:660}
\begin{aligned}
h
&= \inv{2}
\begin{bmatrix}
\hat{\Bp} \cdot \Bsigma & 0 \\
0 & \hat{\Bp} \cdot \Bsigma
\end{bmatrix}
\\&=
\hat{\Bp} \cdot \BS,
\end{aligned}
\end{equation}
where
\begin{equation}\label{eqn:qftLecture20:680}
\BS =
\begin{bmatrix}
\frac{\Bsigma}{2} & 0 \\
0 & \frac{\Bsigma}{2}
\end{bmatrix}.
\end{equation}
\( h \) has eigenvalues \( \pm 1/2 \).

It turns out that \cref{eqn:qftLecture20:620}, and \cref{eqn:qftLecture20:640} are both eigenstates of the helicity operator.
\begin{equation}\label{eqn:qftLecture20:700}
h u^{(1)} = \inv{2} u^{(1)}
\end{equation}
\begin{equation}\label{eqn:qftLecture20:720}
h u^{(2)} = -\inv{2} u^{(2)},
\end{equation}
corresponding to momentum aligned with and opposing the spin directions as sketched in \cref{fig:l20:l20Fig1}.
\imageFigure{../figures/phy2403-quantum-field-theory/l20Fig1}{Helicity orientation.}{fig:l20:l20Fig1}{0.2}

\section{Next time.}

We found \( \Psi = u e^{-i p \cdot x} \).  Next time we will seek another solution \( \Psi = v e^{+ i p \cdot x} \), and we will also figure out how to normalize things.


%}
%\EndArticle
