%
% Copyright � 2018 Peeter Joot.  All Rights Reserved.
% Licenced as described in the file LICENSE under the root directory of this GIT repository.
%
%{
%%\input{../latex/blogpost.tex}
%%\renewcommand{\basename}{qftLecture14}
%%\renewcommand{\dirname}{notes/phy2403/}
%%\newcommand{\keywords}{PHY2403H}
%%\input{../latex/peeter_prologue_print2.tex}
%%
%%%\usepackage{phy2403}
%%\usepackage{peeters_braket}
%%%\usepackage{peeters_layout_exercise}
%%\usepackage{peeters_figures}
%%\usepackage{mathtools}
%%\usepackage{siunitx}
%%\usepackage{macros_cal} % LL
%%
%%\newcommand{\ultensor}[3]{{{#1}^{#2}}_{#3}}
%%% https://tex.stackexchange.com/a/68357/15
%%\DeclareMathOperator*{\SumInt}{%
%%\mathchoice%
%%  {\ooalign{$\displaystyle\sum$\cr\hidewidth$\displaystyle\int$\hidewidth\cr}}
%%  {\ooalign{\raisebox{.14\height}{\scalebox{.7}{$\textstyle\sum$}}\cr\hidewidth$\textstyle\int$\hidewidth\cr}}
%%  {\ooalign{\raisebox{.2\height}{\scalebox{.6}{$\scriptstyle\sum$}}\cr$\scriptstyle\int$\cr}}
%%  {\ooalign{\raisebox{.2\height}{\scalebox{.6}{$\scriptstyle\sum$}}\cr$\scriptstyle\int$\cr}}
%%}
%%
%%\beginArtNoToc
%%\generatetitle{PHY2403H Quantum Field Theory.  Lecture 14: Time evolution, Hamiltonian perturbation, ground state.  Taught by Prof.\ Erich Poppitz}
%\chapter{Time evolution, Hamiltonian pertubation, ground state}
\index{time evolution}
\index{pertubation!Hamiltonian}
\index{ground state}
\label{chap:qftLecture14}

%,$ s/_Iint/_{\\text{I,int}}
%,$ s/_int/_{\\text{int}}

%%Peeter's lecture notes from class.  These may be incoherent and rough.
%%
%%These are notes for the UofT course PHY2403H, Quantum Field Theory, taught by Prof. Erich Poppitz, covering \textchapref{{1}} \citep{peskin1995introduction} content.

%\paragraph{DISCLAIMER: Very rough notes from class, with some additional side notes.}
%
%These are notes for the UofT course PHY2403H, Quantum Field Theory, taught by Prof. Erich Poppitz, fall 2018.
%%, covering \textchapref{{1}} \citep{peskin1995introduction} content.
%
\section{Review.}

Given a field \( \phi(t_0, \Bx) \), satisfying the commutation relations
\begin{equation}\label{eqn:qftLecture14:20}
\antisymmetric{\pi(t_0, \Bx)}{\phi(t_0, \By)} = -i \deltathree(\Bx - \By),
\end{equation}
we introduced an interaction picture field given by
\begin{equation}\label{eqn:qftLecture14:40}
\phi_I(t, x) = e^{i H_0(t- t_0)} \phi(t_0, \Bx) e^{-iH_0(t - t_0)},
\end{equation}
related to the Heisenberg picture representation by
\begin{equation}\label{eqn:qftLecture14:60}
\begin{aligned}
\phi_H(t, x)
&= e^{i H(t- t_0)} \phi(t_0, \Bx) e^{-iH(t - t_0)} \\
&= U^\dagger(t, t_0) \phi_I(t, \Bx) U(t, t_0),
\end{aligned}
\end{equation}
where \( U(t, t_0) \) is the time evolution operator.
\begin{equation}\label{eqn:qftLecture14:80}
U(t, t_0) =
e^{i H_0(t - t_0)}
e^{-i H(t - t_0)}
\end{equation}
We argued that
\begin{equation}\label{eqn:qftLecture14:100}
i \PD{t}{} U(t, t_0) = H_{\text{I,int}}(t) U(t, t_0),
\end{equation}
and found the ``glorious expression''
\boxedEquation{eqn:qftLecture14:120}{
\begin{aligned}
U(t, t_0)
&= T \exp{\lr{ -i \int_{t_0}^t H_{\text{I,int}}(t') dt'}} \\
&=
\sum_{n = 0}^\infty \frac{(-i)^n}{n!} \int_{t_0}^t dt_1 dt_2 \cdots dt_n T\lr{ H_{\text{I,int}}(t_1) H_{\text{I,int}}(t_2) \cdots H_{\text{I,int}}(t_n) }.
\end{aligned}
}

However, what we are really after is
\begin{equation}\label{eqn:qftLecture14:140}
\bra{\Omega} T(\phi(x_1) \cdots \phi(x_n)) \ket{\Omega}.
\end{equation}
Such a product has many labels and names, and we'll describe it as ``vacuum expectation values of time-ordered products of arbitrary \#s of local Heisenberg operators''.

\section{Perturbation.}
\index{perturbation}
Following \S 4.2, \citep{peskin1995introduction}.

\begin{equation}\label{eqn:qftLecture14:160}
\begin{aligned}
H &= \text{exact Hamiltonian} = H_0 + H_{\text{int}}
\\
H_0 &= \text{free Hamiltonian.
}
\end{aligned}
\end{equation}
We know all about \( H_0 \) and assume that it has a lowest (ground state) \( \ket{0} \), the ``vacuum'' state of \( H_0 \).

\( H \) has eigenstates, in particular \( H \) is assumed to have a unique ground state \( \ket{\Omega} \) satisfying
\begin{equation}\label{eqn:qftLecture14:180}
H \ket{\Omega}  = \ket{\Omega} E_0,
\end{equation}
and has states \( \ket{n} \), representing excited (non-vacuum states with energies > \( E_0 \)).
These states are assumed to be a complete basis
\begin{equation}\label{eqn:qftLecture14:200}
\BOne = \ket{\Omega}\bra{\Omega} + \sum_n \ket{n}\bra{n} + \int dn \ket{n}\bra{n}.
\end{equation}
The latter terms may be written with a superimposed sum-integral notation as
\begin{equation}\label{eqn:qftLecture14:440}
\sum_n + \int dn
=
\SumInt_n,
\end{equation}
so the identity operator takes the more compact form
\begin{equation}\label{eqn:qftLecture14:460}
\BOne = \ket{\Omega}\bra{\Omega} + \SumInt_n \ket{n}\bra{n}.
\end{equation}

For some time \( T \) we have
\begin{equation}\label{eqn:qftLecture14:220}
e^{-i H T} \ket{0}
= e^{-i H T}
\lr{
   \ket{\Omega}\braket{\Omega}{0} + \SumInt_n \ket{n}\braket{n}{0}
}.
\end{equation}

We now wish to argue that the \( \SumInt_n \) term can be ignored.
\paragraph{Argument 1:}

This is something of a fast one, but one can consider a formal transformation \( T \rightarrow T(1 - i \epsilon) \), where \( \epsilon \rightarrow 0^+ \), and consider very large \( T \).  This gives
\begin{equation}\label{eqn:qftLecture14:240}
\begin{aligned}
&\lim_{T \rightarrow \infty, \epsilon \rightarrow 0^+}
e^{-i H T(1 - i \epsilon)} \ket{0} \\
&=
\lim_{T \rightarrow \infty, \epsilon \rightarrow 0^+}
e^{-i H T(1 - i \epsilon)}
\lr{
   \ket{\Omega}\braket{\Omega}{0} + \SumInt_n \ket{n}\braket{n}{0}
}
\\&=
\lim_{T \rightarrow \infty, \epsilon \rightarrow 0^+}
e^{-i E_0 T - E_0 \epsilon T}
   \ket{\Omega}\braket{\Omega}{0} + \SumInt_n e^{-i E_n T - \epsilon E_n T} \ket{n}\braket{n}{0}
   \\&=
\lim_{T \rightarrow \infty, \epsilon \rightarrow 0^+}
e^{-i E_0 T - E_0 \epsilon T}
\lr{
   \ket{\Omega}\braket{\Omega}{0} + \SumInt_n e^{-i (E_n -E_0) T - \epsilon T (E_n - E_0)} \ket{n}\braket{n}{0}
}.
\end{aligned}
\end{equation}
The limits are evaluated by first taking \( T \) to infinity, then only after that take \( \epsilon \rightarrow 0^+ \).  Doing this, the sum is dominated by the ground state contribution, since each excited state also has a \( e^{-\epsilon T(E_n - E_0)} \) suppression factor (in addition to the leading suppression factor).

\paragraph{Argument 2:}
With the hand waving required for the argument above, it's worth pointing other (less formal) ways to arrive at the same result.  We can write
\begin{equation}\label{eqn:qftLecture14:260}
\SumInt \ket{n}\bra{n} \rightarrow
\sum_k \int \frac{d^3 p}{(2 \pi)^3} \ket{\Bp, k}\bra{\Bp, k},
\end{equation}
where \( k \) is some unknown quantity that we are summing over.
If we have
\begin{equation}\label{eqn:qftLecture14:280}
H \ket{\Bp, k} = E_{\Bp, k} \ket{\Bp, k},
\end{equation}
then
\begin{equation}\label{eqn:qftLecture14:300}
e^{-i H T} \SumInt \ket{n}\bra{n}
=
\sum_k \int \frac{d^3 p}{(2 \pi)^3} \ket{\Bp, k} e^{-i E_{\Bp, k}} \bra{\Bp, k}.
\end{equation}
If we take matrix elements
\begin{equation}\label{eqn:qftLecture14:320}
\begin{aligned}
\bra{A}
e^{-i H T} \SumInt \ket{n}\bra{n} \ket{B}
&=
\sum_k \int \frac{d^3 p}{(2 \pi)^3} \braket{A}{\Bp, k} e^{-i E_{\Bp, k}} \braket{\Bp, k}{B}
\\&=
\sum_k \int \frac{d^3 p}{(2 \pi)^3} e^{-i E_{\Bp, k}} f(\Bp).
\end{aligned}
\end{equation}
If we assume that \( f(\Bp) \) is a well behaved smooth function, we have ``infinite'' frequency oscillation within the envelope provided by the amplitude of that function, as depicted in \cref{fig:RiemannLebesque:RiemannLebesqueFig1}.
The Riemann-Lebesgue lemma \citep{wiki:RiemannLebesgue} describes such integrals, the result of which is that such an integral goes to zero.  This is a different sort of hand waving argument, but either way, we can argue that only the ground state contributes to the sum
\cref{eqn:qftLecture14:220}
 above.
\mathImageFigure{../figures/phy2403-quantum-field-theory/RiemannLebesqueFig1}{High frequency oscillations within envelope of well behaved function.}{fig:RiemannLebesque:RiemannLebesqueFig1}{0.25}{RiemannLebesgueFig1.nb}

\paragraph{Ground state of the perturbed Hamiltonian.}
\index{ground state}

With the excited states ignored, we are left with
\begin{equation}\label{eqn:qftLecture14:340}
e^{-i H T} \ket{0} = e^{-i E_0 T} \ket{\Omega}\braket{\Omega}{0},
\end{equation}
in the \( T \rightarrow \infty(1 - i \epsilon) \) limit.  We can now write the ground state as
\begin{equation}\label{eqn:qftLecture14:360}
\begin{aligned}
\ket{\Omega}
&=
\evalbar{
\frac{ e^{i E_0 T - i H T } \ket{0} }{
\braket{\Omega}{0}
}
}{ T \rightarrow \infty(1 - i \epsilon) }
\\&=
\evalbar{
   \frac{ e^{- i H T } \ket{0} }{
   e^{-i E_0 T} \braket{\Omega}{0}
   }
}{ T \rightarrow \infty(1 - i \epsilon) }.
\end{aligned}
\end{equation}
Shifting the very large \( T \rightarrow T + t_0 \) shouldn't change things, so
\begin{equation}\label{eqn:qftLecture14:480}
\ket{\Omega}
=
\evalbar{
   \frac{ e^{- i H (T + t_0) } \ket{0} }{
   e^{-i E_0 (T + t_0) } \braket{\Omega}{0}
   }
}{ T \rightarrow \infty(1 - i \epsilon) }.
\end{equation}
%so we may insert an \( e^{-i H_0(-T -t_0)} \) factor without effect
%\begin{equation}\label{eqn:qftLecture14:500}
%\ket{\Omega}
%%=
%%\evalbar{
%%\frac{ e^{i H (t_0 - (-T)) } e^{ -i H_0 (-T - t_0) } \ket{0} }{
%%e^{-i E_0(t_0 - (-T))} \braket{\Omega}{0}
%%}
%%}{ T \rightarrow \infty(1 - i \epsilon) }.
%%\end{equation}

A bit of manipulation shows that the operator in the numerator has the structure of a time evolution operator.
%%With
%%\begin{equation}\label{eqn:qftLecture14:380}
%%U(t, t_0)
%%= e^{i H_0(t - t_0)} e^{-i H(t - t_0)}
%%= T \exp{\lr{ -i \int_{t_0}^t H_{\text{I,int}}(t') dt'}}
%%\end{equation}
%%
\paragraph{Claim: (DIY):}
\Cref{eqn:qftLecture14:80}, \cref{eqn:qftLecture14:120} may be generalized to
\begin{equation}\label{eqn:qftLecture14:400}
U(t, t') = e^{i H_0(t - t_0)} e^{-i H(t - t')} e^{-i H_0(t' - t_0)} =
T \exp{\lr{ -i \int_{t'}^t H_{\text{I,int}}(t'') dt''}}.
\end{equation}
Observe that we recover \cref{eqn:qftLecture14:120} when \( t' = t_0 \).
Using \cref{eqn:qftLecture14:400} we find
\begin{equation}\label{eqn:qftLecture14:520}
\begin{aligned}
U(t_0, -T) \ket{0}
  &= e^{i H_0(t_0 - t_0)} e^{-i H(t_0 + T)} e^{-i H_0(-T - t_0)} \ket{0}
\\&= e^{-i H(t_0 + T)} e^{-i H_0(-T - t_0)} \ket{0}
\\&= e^{-i H(t_0 + T)} \ket{0},
\end{aligned}
\end{equation}
where we use the fact that \(
%\begin{equation}\label{eqn:qftLecture14:540}
e^{i H_0 \tau} \ket{0} = \lr{ 1 + i H_0 \tau + \cdots } \ket{0} = 1 \ket{0},
%\end{equation}
\)
since \( H_0 \ket{0} = 0 \).

We are left with
\index{ground state}
\boxedEquation{eqn:qftLecture14:420}{
\ket{\Omega}
= \frac{U(t_0, -T) \ket{0} }{e^{-i E_0(t_0 - (-T))} \braket{\Omega}{0}}.
}

We are close to where we want to be.  Wednesday we finish off, and then start scattering and Feynman diagrams.

%}
%\EndArticle
