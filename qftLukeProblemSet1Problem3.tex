%
% Copyright � 2015 Peeter Joot.  All Rights Reserved.
% Licenced as described in the file LICENSE under the root directory of this GIT repository.
%
\makeoproblem{One dimensional string.}
{qft:LukeProblemSet1:3}
{2015 ps1.3}
{

A string of length \( a \), mass per unit length \( \sigma \) and under tension \(T\) is fixed at each end. The Lagrangian governing the time evolution of the transverse displacement \( y(x,t) \) is
\begin{equation}\label{eqn:LukeProblemSet1Problem3:20}
L = \int_0^a dx \lr{ \frac{\sigma}{2} \lr{ \PD{t}{y} }^2 - \frac{T}{2} \lr{ \PD{x}{y} }^2 },
\end{equation}
where \( x \) identifies position along the string from one end point.
\makesubproblem{}{qft:LukeProblemSet1:3a}
By expressing the displacement as a sine series Fourier expansion of the form
\begin{equation}\label{eqn:LukeProblemSet1Problem3:40}
y(x,t) = \sqrt{\frac{2}{a}} \sum_{n=1}^\infty \sin\lr{ \frac{ n \pi x }{a} } q_n(t).
\end{equation}

Show that the Lagrangian becomes
\begin{equation}\label{eqn:LukeProblemSet1Problem3:60}
L = \sum_{n=1}^\infty \lr{ \frac{\sigma}{2} \dot{q}_n^2 - \frac{T}{2} \lr{ \frac{ n \pi }{2} }^2 q_n^2 } .
\end{equation}
\makesubproblem{}{qft:LukeProblemSet1:3b}
Derive the equations of motion. Hence, show that the string is equivalent to an infinite set of decoupled
harmonic oscillators, and find their frequencies.
} % makeproblem
\makeanswer{qft:LukeProblemSet1:3}{
\withproblemsetsParagraph{
\makeSubAnswer{}{qft:LukeProblemSet1:3a}
First observe that the functions \( \braket{x}{n} = \sqrt{\frac{2}{a}} \sin\lr{ n \pi x/a } \) are orthonormal over the \( [0,a] \) domain.
\begin{equation}\label{eqn:qftProblemSet1Problem3:80}
\begin{aligned}
\braket{ n }{n}
&=
\frac{2}{a}
\int_0^a \sin^2\lr{ n \pi x/a } dx \\
&=
2
\int_0^1 \sin^2\lr{ n \pi u } du \\
&=
\int_0^1 \lr{ 1 - \cos\lr{ 2 n \pi u } } du \\
&=
1,
\end{aligned}
\end{equation}
and for \( n \ne m \)
\begin{equation}\label{eqn:qftProblemSet1Problem3:100}
\begin{aligned}
\braket{n}{m}
&=
\frac{2}{a}
\int_0^a \sin\lr{ n \pi x/a } \sin\lr{ m \pi x/a } dx \\
&=
2
\int_0^1 \sin\lr{ n \pi u } \sin\lr{ m \pi u } du \\
&=
-\inv{2}
\int_0^1
\lr{ e^{i n \pi u} - e^{-i n \pi u} }
\lr{ e^{i m \pi u} - e^{-i m \pi u} }
du \\
&=
-
\int_0^1
du
\lr{
\cos( ( n + m) \pi u ) - \cos( (m - n) \pi u )
} \\
&= 0,
\end{aligned}
\end{equation}
so
\begin{equation}\label{eqn:qftProblemSet1Problem3:120}
\begin{aligned}
L
&=
\int_0^a dx
\frac{2}{a}
\sum_{m,n = 1}^\infty
\sin\lr{ \frac{ n \pi x }{a} }
\sin\lr{ \frac{ m \pi x }{a} }
\lr{ \frac{\sigma}{2} \dot{q}_n \dot{q}_m
- \frac{T}{2}
\lr{ \frac{n \pi}{a} }
\lr{ \frac{m \pi}{a} } q_n q_m
}
\\&=
\sum_{m,n = 1}^\infty \delta_{nm}
\lr{ \frac{\sigma}{2} \dot{q}_n \dot{q}_m
- \frac{T}{2}
\lr{ \frac{n \pi}{a} }
\lr{ \frac{m \pi}{a} }
q_n q_m
}
\\&=
\sum_{n = 1}^\infty
\lr{ \frac{\sigma}{2} \lr{\dot{q}_n}^2
- \frac{T}{2}
\lr{ \frac{n \pi}{a} }^2 q_n^2
}.
\end{aligned}
\end{equation}
\makeSubAnswer{}{qft:LukeProblemSet1:3b}
We have an Euler-Lagrange equation for each \( q_n \).  The conjugate momenta are
\begin{equation}\label{eqn:qftProblemSet1Problem3:140}
\PD{\dot{q}_n}{L} = \sigma \dot{q}_n.
\end{equation}

We also have
\begin{equation}\label{eqn:qftProblemSet1Problem3:160}
\PD{q_n}{L} = - T \lr{ \frac{n \pi}{a} }^2 q_n,
\end{equation}
%
so we have
\begin{equation}\label{eqn:qftProblemSet1Problem3:180}
\ddot{q}_n = - \frac{T}{\sigma} \lr{ \frac{n \pi}{a} }^2 q_n.
\end{equation}

These have solutions
\begin{equation}\label{eqn:qftProblemSet1Problem3:200}
q_n(t) = A_{\pm} \exp\lr{ \pm i \sqrt{ \frac{T}{\sigma} } \frac{n \pi}{a} t }.
\end{equation}

The angular frequencies are
\begin{equation}\label{eqn:qftProblemSet1Problem3:220}
\omega_n = 2 \pi \nu_n = \sqrt{ \frac{T}{\sigma} } \frac{n \pi}{a},
\end{equation}
%
so the frequencies are
\begin{equation}\label{eqn:qftProblemSet1Problem3:240}
\nu_n = \sqrt{ \frac{T}{\sigma} } \frac{n }{2 a}.
\end{equation}
}
}
