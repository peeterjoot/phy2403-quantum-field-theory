%
% Copyright © 2017 Peeter Joot.  All Rights Reserved.
% Licenced as described in the file LICENSE under the root directory of this GIT repository.
%
\section{Pole shifting.}

Recall that the four dimensional form of the Green's function was
\begin{dmath}\label{eqn:qftLecture13:400}
D_F = i \int \frac{d^4 p}{(2 \pi)^4} e^{-i p \cdot x} \inv{ p^2 - m^2 }.
\end{dmath}
For the Feynman case, the contour that we were taking around the poles can also be accomplished by shifting the poles strategically, as sketched in \cref{fig:feynmanDeformationTwoWays:feynmanDeformationTwoWaysFig1}.

\imageFigure{../figures/phy2403-quantum-field-theory/feynmanDeformationTwoWaysFig1}{Feynman deformation or equivalent shift of the poles.}{fig:feynmanDeformationTwoWays:feynmanDeformationTwoWaysFig1}{0.3}

This shift can be expressed explicit algebraically by introducing an offset
\begin{dmath}\label{eqn:qftLecture13:420}
D_F = i \int \frac{d^4 p}{(2 \pi)^4} e^{-i p \cdot x} \inv{ p^2 - m^2 + i \epsilon }
\end{dmath}
which puts the poles at

\begin{dmath}\label{eqn:qftLecture13:440}
p^0
= \pm \sqrt{ \omega_\Bp - i \epsilon }
= \pm \omega_\Bp \lr{ 1 - \frac{i \epsilon}{\omega_\Bp^2} }^{1/2}
= \pm \omega_\Bp \lr{ 1 - \inv{2} \frac{i \epsilon}{\omega_\Bp^2} }
=
\left\{
\begin{array}{l}
+\omega_\Bp - \inv{2} i \frac{\epsilon}{\omega_\Bp} \\
-\omega_\Bp + \inv{2} i \frac{\epsilon}{\omega_\Bp} \\
\end{array}
\right.
\end{dmath}

