%
% Copyright � 2018 Peeter Joot.  All Rights Reserved.
% Licenced as described in the file LICENSE under the root directory of this GIT repository.
%
%{
%%\input{../latex/blogpost.tex}
%%\renewcommand{\basename}{qftLecture2}
%%%\renewcommand{\dirname}{notes/phy1520/}
%%\renewcommand{\dirname}{notes/ece1228-electromagnetic-theory/}
%%%\newcommand{\dateintitle}{}
%%%\newcommand{\keywords}{}
%%
%%\input{../latex/peeter_prologue_print2.tex}
%%
%%\usepackage{peeters_layout_exercise}
%%\usepackage{peeters_braket}
%%\usepackage{peeters_figures}
%%\usepackage{siunitx}
%%\usepackage{verbatim}
%%%\usepackage{mhchem} % \ce{}
%%%\usepackage{macros_bm} % \bcM
%%%\usepackage{macros_qed} % \qedmarker
%%%\usepackage{txfonts} % \ointclockwise
%%
%%%\newcommand{m_\txte}[0]{m_\txte}
%%
%%\beginArtNoToc
%%
% Wednesday Sept 12, 2018.
%\generatetitle{UofT QFT Fall 2018 Lecture 2. Units, scales, and Lorentz transformations.  Taught by Prof. Erich Poppitz}
%\chapter{Units, scales, and Lorentz transformations.}
\index{units}
\index{scales}
\index{Lorentz transformations}
%\chapter{UofT QFT Fall 2018 Lecture 2, taught by Prof. Erich Poppitz}

%\paragraph{DISCLAIMER: Very rough notes from class.  Some additional side notes, but otherwise barely edited.}

%At Compton wavelength, multiple particle pair production is possible
%%%\Delta p \Delta x \sim \Hbar
%%%
%%%Suppose
%%%
%%%\Delta x \sim \frac{\Hbar}{m_\txte c}
%%%
%%%[paper]
%%%
\section{Natural units.}
\index{natural units}

\begin{dmath}\label{eqn:qftLecture2:20}
\begin{aligned}
[\Hbar] &= [\text{action}] = M \frac{L^2}{T^2} T = \frac{M L^2}{T} \\
[c]    &= [\text{velocity}] = \frac{L}{T} \\
       &  [\text{energy}] = M \frac{L^2}{T^2}.
\end{aligned}
\end{dmath}

Setting \( c = 1 \) means

\begin{dmath}\label{eqn:qftLecture2:240}
\frac{L}{T} = 1
\end{dmath}

and setting \( \Hbar = 1 \) means

\begin{dmath}\label{eqn:qftLecture2:260}
[\Hbar] = [\text{action}] = M L \cancel{\frac{L}{T}} = M L
\end{dmath}

therefore

\begin{dmath}\label{eqn:qftLecture2:280}
[L] = \inv{\text{mass}}
\end{dmath}

and

\begin{dmath}\label{eqn:qftLecture2:300}
[\text{energy}] = M \cancel{\frac{L^2}{T^2}} = \text{mass}\, \si{eV}
\end{dmath}

Summary


\begin{itemize}
\item
\( \text{energy} \sim \si{eV} \)
\item
\( \text{distance} \sim \inv{M} \)
\item
\( \text{time} \sim \inv{M} \)
\end{itemize}

%e (dimensionless)
From:
\begin{dmath}\label{eqn:qftLecture2:320}
\alpha = \frac{e^2}{4 \pi \cancel{\Hbar c}}
\end{dmath}
which is dimensionless (\(1/137\)), so electric charge is dimensionless.

Some useful numbers in natural units

\begin{dmath}\label{eqn:qftLecture2:40}
\begin{aligned}
   m_\txte &\sim 10^{-27} \si{g} \sim 0.5 \si{MeV} \\
   m_\txtp &\sim 2000 m_\txte \sim 1 \si{GeV} \\
   m_\pi &\sim 140 \si{MeV} \\
   m_\mu &\sim 105 \si{MeV} \\
   \Hbar c &\sim 200 \si{MeV} \,\si{fm} = 1
\end{aligned}
\end{dmath}

%distance \inv{\si{MeV}} \sim 10^{-11} \si{cm} 10^{-15} m, or 10^{-13} cm.

\section{Gravity.}
\index{gravity}

Interaction energy of two particles

\begin{dmath}\label{eqn:qftLecture2:60}
G_\txtN \frac{m_1 m_2}{r}
\end{dmath}

\begin{dmath}\label{eqn:qftLecture2:80}
[\text{energy}] \sim [G_\txtN] \frac{M^2}{L}
\end{dmath}

\begin{dmath}\label{eqn:qftLecture2:100}
[G_\txtN]
\sim
[\text{energy}] \frac{L}{M^2}
\end{dmath}

but energy x distance is dimensionless (action) in our units

\begin{dmath}\label{eqn:qftLecture2:120}
[G_\txtN]
\sim
{\text{dimensionless}}{M^2}
\end{dmath}

\begin{dmath}\label{eqn:qftLecture2:140}
\frac{G_\txtN}{\Hbar c} \sim \inv{M^2} \sim \frac{1}{10^{20} \si{GeV}}
\end{dmath}

Planck mass

\begin{dmath}\label{eqn:qftLecture2:160}
M_{\text{Planck}} \sim \sqrt{\frac{\Hbar c}{G_\txtN}}
\sim 10^{-4} g \sim \inv{\lr{10^{20} \si{GeV}}^2}
\end{dmath}

We can revisit the scale diagram from last lecture in terms of MeV mass/energy values, as sketched in
\cref{fig:Lecture2:Lecture2Fig1}.
\imageFigure{../figures/phy2403-quantum-field-theory/Lecture2Fig1}{Scales, take II.}{fig:Lecture2:Lecture2Fig1}{0.3}

At the classical electron radius scale, we consider phenomena such as back reaction of radiation, the self energy of electrons.  At the Compton wavelength we have to allow for production of multiple particle pairs.  At Bohr radius scales we must start using QM instead of classical mechanics.

\section{Cross section.}
\index{cross section}
(Verbal discussion of cross section, not captured in these notes).  Roughly, the cross section sounds like the number of events per unit time, related to the flux of some source through an area.

We'll compute the cross section of a number of different systems in this course.  The cross section is relevant in scattering such as the electron-electron scattering sketched in \cref{fig:Lecture2:Lecture2Fig2}.

\imageFigure{../figures/phy2403-quantum-field-theory/Lecture2Fig2}{Electron electron scattering.}{fig:Lecture2:Lecture2Fig2}{0.3}

We assume that QED is highly relativistic.  In natural units, our scale factor is basically the square of the electric charge
\begin{dmath}\label{eqn:qftLecture2:180}
\alpha \sim e^2,
\end{dmath}
so the cross section has the form
\begin{dmath}\label{eqn:qftLecture2:200}
\sigma \sim \frac{\alpha^2}{E^2} \lr{ 1 + O(\alpha) + O(\alpha^2) + \cdots }
\end{dmath}

In gravity we could consider scattering of electrons, where \( G_\txtN \) takes the place of \( \alpha \).  However, \( G_\txtN \) has dimensions.

For electron-electron scattering due to gravitons

\begin{dmath}\label{eqn:qftLecture2:220}
\sigma \sim \frac{G_\txtN^2 E^2}{1 + G_\txtN E^2 + \cdots }
\end{dmath}

Now the cross section grows with energy.  This will cause some problems (violating unitarity: probabilities greater than 1!) when \( O(G_\txtN E^2) = 1 \).

In any quantum field theories when the coupling constant is not-dimensionless we have the same sort of problems at some scale.

The point is that we can get far considering just dimensional analysis.

If the coupling constant has a dimension \((1/\text{mass})^N\,, N > 0\), then unitarity will be violated at high energy.  One such theory is the Fermi theory of beta decay (electro-weak theory), which had a coupling constant with dimensions inverse-mass-squared.  The relevant scale for beta decay was 4 Fermi, or \( G_\txtF \sim (1/{100 \si{GeV}})^2 \).  This was the motivation for introducing the Higgs theory, which was motivated by restoring unitarity.


%}
%%\EndNoBibArticle
