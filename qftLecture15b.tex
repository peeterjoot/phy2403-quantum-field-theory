%
% Copyright � 2017 Peeter Joot.  All Rights Reserved.
% Licenced as described in the file LICENSE under the root directory of this GIT repository.
%
%{
\input{../latex/blogpost.tex}
\renewcommand{\basename}{qftLecture15b}
\renewcommand{\dirname}{notes/phy2403/}
\newcommand{\keywords}{PHY2403H}
\input{../latex/peeter_prologue_print2.tex}

%\usepackage{phy2403}
\usepackage{peeters_braket}
%\usepackage{peeters_layout_exercise}
\usepackage{peeters_figures}
\usepackage{mathtools}
\usepackage{siunitx}
\usepackage{macros_cal} % LL

\newcommand{\ultensor}[3]{{{#1}^{#2}}_{#3}}

\beginArtNoToc
\generatetitle{PHY2403H Quantum Field Theory.  Lecture 15b: XXX.  Taught by Prof.\ Erich Poppitz}
%\chapter{XXX}
\label{chap:qftLecture15b}

%%Peeter's lecture notes from class.  These may be incoherent and rough.
%%
%%These are notes for the UofT course PHY2403H, Quantum Field Theory, taught by Prof. Erich Poppitz, covering \textchapref{{1}} \citep{peskin1995introduction} content.

\paragraph{DISCLAIMER: Very rough notes from class, with some additional side notes.}

These are notes for the UofT course PHY2403H, Quantum Field Theory, taught by Prof. Erich Poppitz, fall 2018.
%, covering \textchapref{{1}} \citep{peskin1995introduction} content.

\section{YYY}

%}
\EndArticle
%\EndNoBibArticle
