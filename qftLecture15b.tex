%
% Copyright � 2018 Peeter Joot.  All Rights Reserved.
% Licenced as described in the file LICENSE under the root directory of this GIT repository.
%
%{
%%%\input{../latex/blogpost.tex}
%%%\renewcommand{\basename}{qftLecture15b}
%%%\renewcommand{\dirname}{notes/phy2403/}
%%%\newcommand{\keywords}{PHY2403H}
%%%\input{../latex/peeter_prologue_print2.tex}
%%%
%%%%\usepackage{phy2403}
%%%\usepackage{peeters_braket}
%%%\usepackage{peeters_layout_exercise}
%%%\usepackage{peeters_figures}
%%%\usepackage{mathtools}
%%%\usepackage{siunitx}
%%%\usepackage{macros_cal} % LL
%%%\usepackage{simplewick}
%%%
%%%\newcommand{\ultensor}[3]{{{#1}^{#2}}_{#3}}
%%%\newcommand{\normalorder}[1]{\text{:\({#1}\):}}
%%%
%%%\beginArtNoToc
%%%\generatetitle{PHY2403H Quantum Field Theory.  Lecture 15b: Wick's theorem, vacuum expectation, Feynman diagrams, \(\phi^4\) interaction, tree level diagrams, scattering, cross section, differential cross section.  Taught by Prof.\ Erich Poppitz}
\index{Wick's theorem}
\index{vacuum expectation}
\index{Feynman diagram}
\index{tree level diagram}
\index{scattering}
\index{cross section}
\index{scattering cross section}
\index{differential cross section}
\label{chap:qftLecture15b}

%%Peeter's lecture notes from class.  These may be incoherent and rough.
%%
%%These are notes for the UofT course PHY2403H, Quantum Field Theory, taught by Prof. Erich Poppitz, covering \textchapref{{1}} \citep{peskin1995introduction} content.

%\paragraph{DISCLAIMER: Very rough notes from class, with some additional side notes.}
%
%These are notes for the UofT course PHY2403H, Quantum Field Theory, taught by Prof. Erich Poppitz, fall 2018.
%%, covering \textchapref{{1}} \citep{peskin1995introduction} content.
%
\section{Wick contractions.}

Here's a double dose of short hand, first an abbreviation for the Feynman propagator
\begin{dmath}\label{eqn:qftLecture15b:20}
D_F(1-2) \equiv D_F(x_1, x_2),
\end{dmath}
and second
\begin{dmath}\label{eqn:qftLecture15b:40}
\contraction{}{\phi}{{}_i}{\phi} \phi_i \phi_j = D_F(i - j),
\end{dmath}
which is called a contraction.

Contractions allow time ordered products to be written in a compact form

\maketheorem{Wick's theorem (stub).}{thm:qftLecture15b:80}{
The rough idea (from the example below) is that the time ordering of the fields has all the combinations of the pairwise contractions and normal ordered fields.

See \cref{qft:problemSet4:1} (Hw4) for full details.
} % theorem

Illustrating by example for the time ordering of \( n = 4 \) fields, we have
\begin{dmath}\label{eqn:qftLecture15b:60}
T( \phi_1 \phi_2 \phi_3 \phi_4)
=
\normalorder{ \phi_1 \phi_2 \phi_3 \phi_4 }
+
\contraction{}{\phi}{{}_1}{\phi} \phi_1 \phi_2 \normalorder{ \phi_3 \phi_4 }
+
\contraction{}{\phi}{{}_1}{\phi} \phi_1 \phi_3 \normalorder{ \phi_2 \phi_4 }
+
\contraction{}{\phi}{{}_1}{\phi} \phi_1 \phi_4 \normalorder{ \phi_2 \phi_3 }
+
\contraction{}{\phi}{{}_2}{\phi} \phi_2 \phi_3 \normalorder{ \phi_1 \phi_4 }
+
\contraction{}{\phi}{{}_2}{\phi} \phi_2 \phi_4 \normalorder{ \phi_1 \phi_3 }
+
\contraction{}{\phi}{{}_3}{\phi} \phi_3 \phi_4 \normalorder{ \phi_1 \phi_2 }
+
\contraction{}{\phi}{{}_1}{\phi} \phi_1 \phi_2
\contraction{}{\phi}{{}_3}{\phi} \phi_3 \phi_4
+
\contraction{}{\phi}{{}_1}{\phi} \phi_1 \phi_3
\contraction{}{\phi}{{}_2}{\phi} \phi_2 \phi_4
+
\contraction{}{\phi}{{}_1}{\phi} \phi_1 \phi_4
\contraction{}{\phi}{{}_2}{\phi} \phi_2 \phi_3.
\end{dmath}

\maketheorem{Corollary: Wick's, Vacuum expectation.}{thm:qftLecture15b:100}{
For \( n \) even
\begin{equation*}
\bra{0} T(\phi_1 \phi_2 \cdots \phi_n) \ket{0}
=
\contraction{}{\phi}{{}_1}{\phi} \phi_1 \phi_2
\contraction{}{\phi}{{}_3}{\phi} \phi_3 \phi_4
\contraction{}{\phi}{{}_5}{\phi} \phi_5 \phi_6
\cdots
\contraction{}{\phi}{{}_{n-1}}{\phi} \phi_{n-1} \phi_n
+ \text{all other terms}.
\end{equation*}
For \( n \) odd, this vanishes.
} % theorem

\section{Simplest Feynman diagrams.}
For \( n = 4 \) we have
\begin{dmath}\label{eqn:qftLecture15b:120}
\bra{0} T(\phi_1 \phi_2 \phi_3 \phi_4) \ket{0}
=
\contraction{}{\phi}{{}_1}{\phi} \phi_1 \phi_2
\contraction{}{\phi}{{}_3}{\phi} \phi_3 \phi_4
+
\contraction{}{\phi}{{}_1}{\phi} \phi_1 \phi_3
\contraction{}{\phi}{{}_2}{\phi} \phi_2 \phi_4
+
\contraction{}{\phi}{{}_1}{\phi} \phi_1 \phi_4
\contraction{}{\phi}{{}_2}{\phi} \phi_2 \phi_3,
\end{dmath}
the set of Wick contractions can be written pictorially \cref{fig:qftLecture15b:qftLecture15bFig1}, and are called Feynman diagrams
\imageFigure{../figures/phy2403-quantum-field-theory/qftLecture15bFig1}{Simplest Feynman diagrams.}{fig:qftLecture15b:qftLecture15bFig1}{0.15}

These are the very simplest Feynman diagrams.

\section{\texorpdfstring{\( \phi^4 \)}{Phi fourth} interaction.}
Introducing another shorthand, we will use an expectation like notation to designate the matrix element for the vacuum state
\begin{dmath}\label{eqn:qftLecture15b:140}
\expectation{\text{blah}} = \bra{0}\text{blah} \ket{0}.
\end{dmath}
For the \( \phi^4 \) theory, this allows us to write the numerator of the perturbed ground state interaction as
\begin{dmath}\label{eqn:qftLecture15b:340}
\bra{\Omega} \phi(x) \phi(y) \ket{\Omega}
\sim
   \bra{0}
   T\lr{
      \phi_I(x)
      \phi_I(y)
      e^{-i \int_{-T}^T H_{\text{I,int}}(t') dt'}
   }
\ket{0}
=
\expectation{
      \phi_I(x)
      \phi_I(y)
      e^{-i \int d^4 z \phi^4(z)}
}.
\end{dmath}
To first order, this is
\begin{dmath}\label{eqn:qftLecture15b:360}
\expectation{ T \phi_x \phi_y } - i \frac{\lambda}{4} \int d^4 z \expectation{
T \phi_x \phi_y \phi_z \phi_z \phi_z \phi_z
},
\end{dmath}
The first braket has the pictorial representation sketched in \cref{fig:qftLecture15b:qftLecture15bFigF2a}.
\imageFigure{../figures/phy2403-quantum-field-theory/qftLecture15bFigF2a}{First integral diagram.}{fig:qftLecture15b:qftLecture15bFigF2a}{0.1}
whereas the second has the diagrams sketched in
\cref{fig:qftLecture15b:qftLecture15bFigF2b}.
%\imageTwoFigures{path1}{path2}{fancy plots}{fig:blah}{scale=0.3}
\imageTwoFigures
{../figures/phy2403-quantum-field-theory/qftLecture15bFigF2b}
{../figures/phy2403-quantum-field-theory/qftLecture15bFigF2c}
{Second integral diagrams.}{fig:qftLecture15b:qftLecture15bFigF2b}{scale=0.3}

We can depict the entire second integral in diagrams as sketched in \cref{fig:qftLecture15b:qftLecture15bFig3}.
\imageFigure{../figures/phy2403-quantum-field-theory/qftLecture15bFig3}{Integrals as diagrams.}{fig:qftLecture15b:qftLecture15bFig3}{0.3}

Solving for the perturbed ground state can now be thought of as reduced to drawing pictures.  Each line from \( x \rightarrow x' \) represents a propagator \( D_F( x - x' ) \), and each vertex \( -i \lambda \int d^4 z \times \text{symmetry coefficients} \).\footnote{Symmetry coefficients weren't discussed until the next lecture.  This means making combinatorial arguments to count the number of equivalent diagrams.}

We may also translate back from the diagrams to an algebraic representation.  For the first order \( \phi^4 \) interaction, that is
\begin{dmath}\label{eqn:qftLecture15b:380}
\expectation{ T \phi_x \phi_y }
- \frac{i \lambda }{4} \int d^4 z D_F(x - y) D_F^2( z - z )
+ D_F(x - z) D_F( y - z).
\end{dmath}

Other diagrams can be similarly translated.  For example
\cref{fig:qftLecture15b:qftLecture15bFig5}.
\imageFigure{../figures/phy2403-quantum-field-theory/qftLecture15bFig5}{Figure eight diagram.}{fig:qftLecture15b:qftLecture15bFig5}{0.2}
represents
\begin{dmath}\label{eqn:qftLecture15b:400}
\int d^4 z D_F^2(z - z)
=
V_3 T \lr{ \int \frac{d^4 p}{(2\pi)^4} \inv{p^2 - m^2 + i \epsilon } }^2.
\end{dmath}
Clearly, additional interpretation will be required, since this diverges.  The resolution of this unfortunately has to be deferred to QFT II, where renormalization is covered.

\section{Tree level diagrams.}
We would like to only discuss tree level diagrams, which exclude diagrams like
\cref{fig:qftLecture15b:qftLecture15bFig6}
\footnote{I think this is what is referred to as connected, amputated graphs in the next lecture.  Such diagrams are the ones of interest for scattering and decay problems.}.
\imageFigure{../figures/phy2403-quantum-field-theory/qftLecture15bFig6}{Not a tree level diagram.}{fig:qftLecture15b:qftLecture15bFig6}{0.2}

For the braket \footnote{I'd written: \( \expectation{ \int \phi_1 \phi_2 \phi_3 \phi_4 \lambda \int d^4 z \phi_z \phi_z \phi_z \phi_z } \).  Is this two fold integral what was intended, or my correction in \cref{eqn:qftLecture15b:420}?}
\begin{dmath}\label{eqn:qftLecture15b:420}
\expectation{ \int d^4 z \phi_1 \phi_2 \phi_3 \phi_4 \phi_z \phi_z \phi_z \phi_z
}
\end{dmath}
we draw diagrams like those of \cref{fig:qftLecture15b:qftLecture15bFig7},
the first of which is a tree level diagram.
\imageFigure{../figures/phy2403-quantum-field-theory/qftLecture15bFig7}{First order interaction diagrams.}{fig:qftLecture15b:qftLecture15bFig7}{0.2}

%}
