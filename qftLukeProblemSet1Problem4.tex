%
% Copyright � 2015 Peeter Joot.  All Rights Reserved.
% Licenced as described in the file LICENSE under the root directory of this GIT repository.
%
\makeoproblem{Dimensional analysis.}
{qft:LukeProblemSet1:4}
{2015 ps1.4}
{

Even though we have set \( \Hbar = c = 1 \), we can still do dimensional analysis because we still have one unit left, mass (or 1/length). In \( d \) space-time dimensions (1 time and \( d-1 \) space), what is the dimension in mass units of a canonical free scalar field, \( \phi \)? (Work it out from the equal-time commutation relations.) Still in \( d \) dimensions, the Lagrange density for a scalar field with self-interactions might be of the form

\begin{dmath}\label{eqn:LukeProblemSet1Problem4:20}
\LL = \inv{2} \lr{ \partial_\mu \phi}^2 - \sum_{n \ge 2} a_n \phi^n.
\end{dmath}

\makesubproblem{}{qft:LukeProblemSet1:4a}
What is the dimension (again in mass units) of the Lagrange density?
\makesubproblem{}{qft:LukeProblemSet1:4b}
The action?
\makesubproblem{}{qft:LukeProblemSet1:4c}
The coefficients \( a_n \)? (as a check, whatever the value of \( d \), \(a_2\) had better have the dimensions of \(\textrm{mass}^2\) ).
} % makeproblem

\makeanswer{qft:LukeProblemSet1:4}{
\withproblemsetsParagraph{
\makeSubAnswer{}{qft:LukeProblemSet1:4a}

With \( \antisymmetric{\phi(\Bx)}{\pi(\By)} = i \deltathree(\Bx - \By) \), which is dimensionless, we have

\begin{dmath}\label{eqn:LukeProblemSet1Problem4:40}
1
= [ \phi \pi ]
= [ \phi^2 ] /L,
\end{dmath}

so

\begin{dmath}\label{eqn:LukeProblemSet1Problem4:60}
[\phi] = L^{1/2}.
\end{dmath}

This means that the dimensions of the Lagrangian are
\begin{dmath}\label{eqn:LukeProblemSet1Problem4:80}
[\LL]
= [(\partial_\mu \phi)^2]
= \inv{L^2} L
= \inv{L}.
\end{dmath}

\makeSubAnswer{}{qft:LukeProblemSet1:4b}

The dimensions of the action are

\begin{dmath}\label{eqn:LukeProblemSet1Problem4:100}
[S]
= [ \int d^d x \LL ]
= L^d \inv{L}
= L^{d-1}
\end{dmath}

\makeSubAnswer{}{qft:LukeProblemSet1:4c}

The dimensions of the coefficients are found from

\begin{dmath}\label{eqn:LukeProblemSet1Problem4:120}
\inv{L}
 =
[a_n \phi^n]
=
[a_n] L^{n/2},
\end{dmath}

or

\begin{dmath}\label{eqn:LukeProblemSet1Problem4:140}
[a_n] = L^{-1 - n/2}.
\end{dmath}

For \( n = 2 \) that is \( [a_n] = L^{-1 - 2/2} = L^{-2} \).  Provided \( [L] = 1/[M] \) this is what is expected.  To see that is the case consider the dimensions of the ratio

\begin{dmath}\label{eqn:qftProblemSet1Problem4:160}
[\Hbar/c]
= [ (M L^2/T)/(L/T) ]
= [ M L ].
\end{dmath}

If both \( \Hbar \) and \( c \) are dimensionless then the dimensions of length must be inverse mass.
}
}
