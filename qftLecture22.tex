%
% Copyright � 2017 Peeter Joot.  All Rights Reserved.
% Licenced as described in the file LICENSE under the root directory of this GIT repository.
%
%{
\input{../latex/blogpost.tex}
\renewcommand{\basename}{qftLecture22}
\renewcommand{\dirname}{notes/phy2403/}
\newcommand{\keywords}{PHY2403H}
\input{../latex/peeter_prologue_print2.tex}

%\usepackage{phy2403}
\usepackage{peeters_braket}
%\usepackage{peeters_layout_exercise}
\usepackage{peeters_figures}
\usepackage{mathtools}
\usepackage{siunitx}
\usepackage{macros_cal} % LL

\newcommand{\ultensor}[3]{{{#1}^{#2}}_{#3}}
\newcommand{\oPsi}[0]{\overbar{\Psi}}
\newcommand{\osigma}[0]{\overbar{\sigma}}
\newcommand{\ubar}[0]{\overbar{u}}
\newcommand{\vbar}[0]{\overbar{v}}
\newcommand{\deltathree}[0]{\delta^{(3)}}
\newcommand{\deltafour}[0]{\delta^{(4)}}
\newcommand{\ITwo}[0]{{\begin{bmatrix} 1 & 0 \\ 0 & 1 \end{bmatrix}}}
\newcommand{\DiracGammaZero}[0]{{\begin{bmatrix} 0 & 1 \\ 1 & 0 \end{bmatrix}}}
\newcommand{\DiracGammaK}[1]{{\begin{bmatrix} 0 & \sigma^{#1} \\ -\sigma^{#1} & 0 \end{bmatrix}}}

\beginArtNoToc
\generatetitle{PHY2403H Quantum Field Theory.  Lecture 22: XXX.  Taught by Prof.\ Erich Poppitz}
%\chapter{XXX}
\label{chap:qftLecture22}

%%Peeter's lecture notes from class.  These may be incoherent and rough.
%%
%%These are notes for the UofT course PHY2403H, Quantum Field Theory, taught by Prof. Erich Poppitz, covering \textchapref{{1}} \citep{peskin1995introduction} content.

\paragraph{DISCLAIMER: Very rough notes from class, with some additional side notes.}

These are notes for the UofT course PHY2403H, Quantum Field Theory, taught by Prof. Erich Poppitz, fall 2018.
%, covering \textchapref{{1}} \citep{peskin1995introduction} content.

\section{Recall:}

From the Dirac Lagrangian density
\begin{dmath}\label{eqn:qftLecture22:20}
\LL = \oPsi \lr{ i \gamma^\mu \partial_\mu -m } \Psi,
\end{dmath}
we found the Hamiltonian
\begin{dmath}\label{eqn:qftLecture22:40}
H = \int \frac{d^3 p}{(2\pi)^3} \sum_{s = 1}^2 \omega_\Bp
\lr{
   a_\Bp^{s \dagger} a_\Bp^s
   -
   b_\Bp^{s \dagger} b_\Bp^s
}.
\end{dmath}
This appears to be an energy with no bottom.  Dirac prescribes: assume Pauli exclusion for \( b \) and fill all the negative energy levels.

\paragraph{Fermi statistics:}
Grassman numbers\footnote{Is it a coincidence that these look like lightlike four-vectors \( x^2 = x^\mu x^\mu = 0 \)?}
\begin{dmath}\label{eqn:qftLecture22:60}
\lr{ b^\dagger }^2 = 0 = b^2 = a^2 = \lr{ a^\dagger }^2.
\end{dmath}

Creating a particle of negative energy \( b^\dagger \) is like \underline{destroying} a hole.

Our creation and anhillation operators are presumed to have non-trivial \underline{anti}-commutation relations (unlike the scalar theory where we had the same sort of commutation relations)
\begin{dmath}\label{eqn:qftLecture22:80}
\symmetric{a_\Bp^s}{a_\Bq^{r\dagger}} = (2 \pi)^3 \delta^{sr} \deltathree( \Bp - \Bq),
\end{dmath}
and those relations were used to cast the Hamiltonian in a more familiar form
\begin{dmath}\label{eqn:qftLecture22:100}
H = \int \frac{d^3 p}{(2\pi)^3} \sum_{s = 1}^2 \omega_\Bp
\lr{
   a_\Bp^{s \dagger} a_\Bp^s
   +
   \tilde{b}_\Bp^{s \dagger}
   \tilde{b}_\Bp^s
}
- V_3
\int \frac{d^3 p}{(2\pi)^3} 2 \omega_\Bp.
\end{dmath}
Fermions have negative zero-point energy \( -4 \times \) that of real massive scalar.
\makedigression{
Supersymmetry transforms these into one another, and was thought to solve the cosmic constant problem.
} % digression

\section{Moving on.}

We now switch notations, and drop the tildes, also defining the Fock vacuum by
\begin{dmath}\label{eqn:qftLecture22:120}
\begin{aligned}
a_\Bp^s \ket{0} &= 0 \\
b_\Bp^s \ket{0} &= 0.
\end{aligned}
\end{dmath}
Then
\begin{dmath}\label{eqn:qftLecture22:140}
H a_\Bp^{s\dagger} \ket{0} = \omega_\Bp (a_\Bp^{s\dagger} \ket{0}).
\end{dmath}
FIXME: show.
Similarly
\begin{dmath}\label{eqn:qftLecture22:160}
\BP =
\int \frac{d^3 p}{(2\pi)^3} \sum_{s = 1}^2 \Bp
\lr{
   a_\Bp^{s \dagger} a_\Bp^s
   +
   b_\Bp^{s \dagger} b_\Bp^s
},
\end{dmath}
gives
\begin{dmath}\label{eqn:qftLecture22:180}
\BP a_\Bq^{s\dagger} \ket{0} = \omega_\Bq (a_\Bq^{s\dagger} \ket{0}).
\end{dmath}
FIXME: show.

Under Lorentz transformation, including rotations:
\begin{dmath}\label{eqn:qftLecture22:200}
\begin{aligned}
\Psi(x) &\rightarrow \Psi'(x') = \Lambda_{1/2} \Psi(x) \\
\delta \Psi(x) &= \Psi'(x) - \Psi(x) \\
\Lambda_{1/2} &= 1 - \frac{i \omega_{\mu\nu}}{2} S^{\mu\nu} \\
\Psi'(x) &= \Lambda_{1/2} \Psi(\Lambda^{-1} x).
\end{aligned}
\end{dmath}
For rotation around \( \zcap \) only \( \omega_{12} \) is non-zero and
\begin{dmath}\label{eqn:qftLecture22:220}
\Lambda_{1/2} = 1 - \frac{i \alpha}{2}
\begin{bmatrix}
\sigma^3 & 0 \\
0 & \sigma^3
\end{bmatrix}
\end{dmath}
From \( \delta \Psi = \Lambda_{1/2} \Psi(\Lambda^{-1} x) - \Psi(x) \) the Noether current can be found to be
\begin{dmath}\label{eqn:qftLecture22:240}
\BJ = \int d^3 x \Psi^\dagger(x) \lr{ \Bx \cross \underbrace{(-i \spacegrad)}_{orbital} + \inv{2} \underbrace{\BOne \otimes \Bsigma}_{spin} },
\end{dmath}
where
\begin{dmath}\label{eqn:qftLecture22:260}
\BOne \otimes \Bsigma =
\begin{bmatrix}
\Bsigma & 0 \\
0 & \Bsigma
\end{bmatrix}.
\end{dmath}

For the rest frame of a particle (zero momentum), it can be shown that
\begin{dmath}\label{eqn:qftLecture22:280}
J^3 a_\Bp^{s\dagger} \ket{0} = \pm \inv{2} \evalbar{ a_\Bp^{s\dagger} \ket{0}}{\Bp = 0},
\end{dmath}
where the \( + \) is for \( s = 1 \) and the \( - \) is for \( s = 2 \).
FIXME: check these.

It can also be shown that
\begin{dmath}\label{eqn:qftLecture22:300}
J^3 b_\Bp^{s\dagger} \ket{0} = \mp \inv{2} \evalbar{ b_\Bp^{s\dagger} \ket{0}}{\Bp = 0}.
\end{dmath}
Roughly speaking, we can characterize these as opposite signed projections of the 3rd component of angular momentum.
FIXME: check these too.

We also have a \( U(1) \) global symmetry which implies charge.  If we let
\begin{dmath}\label{eqn:qftLecture22:320}
\begin{aligned}
\Psi &\rightarrow e^{i \alpha} \Psi \\
\oPsi &\rightarrow e^{-i \alpha} \oPsi,
\end{aligned}
\end{dmath}
then
\begin{dmath}\label{eqn:qftLecture22:340}
j^\mu
= \PD{(\partial_\mu \Psi}}{\LL} \delta \Psi
= \PD{(\partial_\mu \Psi}}{\LL} i \alpha \Psi
= \oPsi i \gamma^\mu i \alpha \Psi
= -\oPsi \gamma^\mu \alpha \Psi
\equiv - \alpha J^\mu,
\end{dmath}
that is
\begin{dmath}\label{eqn:qftLecture22:360}
J^\mu = \oPsi \gamma^\mu \Psi
\end{dmath}

Define the charge as
\begin{dmath}\label{eqn:qftLecture22:380}
Q
= \int d^3x J^0
= \int d^3x \oPsi \gamma^0 \Psi
= \int d^3x \Psi^\dagger \Psi
= \int \frac{d^3 q}{(2\pi)^3} \sum_{s = 1}^2
\lr{
   a_\Bp^{s \dagger} a_\Bp^s
   -
   b_\Bp^{s \dagger}
   b_\Bp^s
},
\end{dmath}
where we dropped the charge associated with the Dirac sea.

%}
\EndArticle
%\EndNoBibArticle
