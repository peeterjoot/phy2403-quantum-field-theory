%
% Copyright © 2018 Peeter Joot.  All Rights Reserved.
% Licenced as described in the file LICENSE under the root directory of this GIT repository.
%
%{
%\section{Problems:}

\makeproblem{Vary the Dirac action \cref{dfn:qftLecture19:31}.}{problem:qftLecture20:10}{
\index{Dirac action!variation}
} % problem

\makeanswer{problem:qftLecture20:10}{
From the action
%\begin{equation}\label{eqn:qftLecture20:760}
%S = \int d^4 x \overbar{\Psi} \lr{ i \gamma^\mu \partial_\mu - m } \Psi,
%\end{equation}
we find
\begin{equation}\label{eqn:qftLecture20:780}
\delta S =
\int d^4 x \delta \overbar{\Psi} \lr{ i \gamma^\mu \partial_\mu - m } \Psi
+
\int d^4 x \overbar{\Psi} \lr{ i \gamma^\mu \partial_\mu - m } \delta \Psi.
\end{equation}
There are two ways to deal with this.  One (somewhat unsatisfactory seeming to me) is to treat both \( \delta \overbar{\Psi} \) and \( \delta \Psi \) as independent variations, requiring that \( \delta S = 0 \) for any such variations.  In that case we find that \( \lr{ i \gamma^\mu \partial_\mu - m } \Psi = 0 \) if the total variation of the action is zero.  That leaves the somewhat awkward question of what to do with the \( 0 = \int d^4 x \overbar{\Psi} \lr{ i \gamma^\mu \partial_\mu - m } \delta \Psi \) constraint.  However, that question can  be resolved by observing that these two contributions to the variation are not independent.  In particular
\begin{equation}\label{eqn:qftLecture20:800}
\begin{aligned}
&\lr{ \int d^4 x \overbar{\Psi} \lr{ i \gamma^\mu \partial_\mu - m } \delta \Psi }^\dagger \\
&=
\int d^4 x
\delta \Psi^\dagger
\lr{ -i (\gamma^\mu)^\dagger \partial_\mu - m }
\gamma^0
\Psi \\
&=
\int d^4 x
\delta \Psi^\dagger \gamma^0
\lr{ -i \gamma^0 (\gamma^\mu)^\dagger \gamma^0 \lpartial_\mu - m }
\Psi \\
&=
\int d^4 x
\delta \overbar{\Psi}
\lr{ -i \gamma^\mu \lpartial_\mu - m }
\Psi \\
&=
\int d^4 x
\lr{
   \partial_\mu \lr{
      -i \delta \overbar{\Psi}
      \gamma^\mu
      \Psi
   }
   -
   \lr{
      -i \delta \overbar{\Psi}
      \gamma^\mu
      \partial_\mu \Psi
   }
   -m \delta \overbar{\Psi} \Psi
} \\
&=
\int d^4 x \delta \overbar{\Psi} \lr{ i \gamma^\mu \partial_\mu - m } \Psi,
\end{aligned}
\end{equation}
where the boundary integral has been assumed to be zero.  This shows that the total variation is
\begin{equation}\label{eqn:qftLecture20:820}
\delta S
=
\int d^4 x
\lr{
\delta
   \overbar{\Psi} D \Psi
+
   \lr{
\delta
\overbar{\Psi} D \Psi }^\dagger
},
\end{equation}
where
\begin{equation}\label{eqn:qftLecture20:840}
D = i \gamma^\mu \partial_\mu - m,
\end{equation}
represents the Dirac operator.  Requiring that the action variation \( \delta S = 0 \) is zero for all \( \delta \overbar{\Psi} \), means that \( D \Psi = 0 \), which proves \cref{eqn:qftLecture20:80}.

Given that the action itself is real, it makes sense for it's variation to be real, as demonstrated above.  A nice side effect of demonstrating this is the removal of the redundant variation variable.
} % answer
%}
