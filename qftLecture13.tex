%
% Copyright � 2017 Peeter Joot.  All Rights Reserved.
% Licenced as described in the file LICENSE under the root directory of this GIT repository.
%
%{
%%%\input{../latex/blogpost.tex}
%%%\renewcommand{\basename}{qftLecture13}
%%%\renewcommand{\dirname}{notes/phy2403/}
%%%\newcommand{\keywords}{PHY2403H}
%%%\input{../latex/peeter_prologue_print2.tex}
%%%
%%%%\usepackage{phy2403}
%%%\usepackage{peeters_braket}
%%%\usepackage{peeters_layout_exercise}
%%%\usepackage{peeters_figures}
%%%\usepackage{mathtools}
%%%\usepackage{siunitx}
%%%\usepackage{macros_cal} % LL
%%%
%%%\newcommand{\ultensor}[3]{{{#1}^{#2}}_{#3}}
%%%
%%%\beginArtNoToc
%%%\generatetitle{PHY2403H Quantum Field Theory.  Lecture 13: Forced Klein-Gordon equation, coherent states, number density, time ordered product, pole shifting, perturbation theory, Heisenberg picture, interaction picture, Dyson's formula.  Taught by Prof.\ Erich Poppitz}
\chapter{Forced Klein-Gordon equation, coherent states, number density, time ordered product, perturbation theory, Heisenberg picture, interaction picture, Dyson's formula}
\label{chap:qftLecture13}
\index{Klein-Gordon!forced}
\index{number density}
\index{perturbation theory}
\index{Heisenberg picture}
\index{Dyson's formula}

%%Peeter's lecture notes from class.  These may be incoherent and rough.
%%
%%These are notes for the UofT course PHY2403H, Quantum Field Theory, taught by Prof. Erich Poppitz, covering \textchapref{{1}} \citep{peskin1995introduction} content.

%%\paragraph{DISCLAIMER: Very rough notes from class, with some additional side notes.}
%%
%%These are notes for the UofT course PHY2403H, Quantum Field Theory, taught by Prof. Erich Poppitz, fall 2018.
%%%, covering \textchapref{{1}} \citep{peskin1995introduction} content.
%%
\section{Review: ``particle creation problem''.}

We imagined that we have a windowed source function \( j(y^0, \By) \), as sketched in \cref{fig:windowedImpulse:windowedImpulseFig5}, which is acting as a forcing source for the non-homogeneous Klein-Gordon equation
%\imageFigure{../figures/phy2403-quantum-field-theory/windowedImpulseFig5}{Finite window impulse response.}{fig:windowedImpulse:windowedImpulseFig5}{0.2}
\begin{dmath}\label{eqn:qftLecture13:20}
\lr{ \partial_\mu \partial^\mu + m^2 } \phi = j.
\end{dmath}
Our solution was
\begin{dmath}\label{eqn:qftLecture13:40}
\phi(x) = \phi(x_0) + i \int d^4 y D_R( x - y) j(y),
\end{dmath}
where \( \phi(x_0) \) obeys the homogeneous equation, and
\begin{dmath}\label{eqn:qftLecture13:60}
D_R(x - y) = \Theta(x^0 - y^0) \lr{ D(x - y) - D(y - x) },
\end{dmath}
and \( D(x) = \int \frac{d^3 p}{(2\pi)^3 2 \omega_\Bp } \evalbar{ e^{-i p \cdot x} }{p^0 = \omega_\Bp} \) is the Wightman function.

For \( x^0 > t_{\text{after}} \)
\begin{dmath}\label{eqn:qftLecture13:80}
\phi(x)
=
\int \frac{d^3 p}{(2\pi)^3 \sqrt{ 2 \omega_\Bp }}
\evalbar{
   \lr{ e^{-i p \cdot x} a_\Bp + e^{i p \cdot x } a_\Bp^\dagger }
}{
   p^0 = \omega_\Bp
}
+ i
\int \frac{d^3 p}{(2\pi)^3 2 \omega_\Bp }
\evalbar{
   \lr{ e^{-i p \cdot x} \tilde{j}(p) + e^{i p \cdot x} \tilde{j}(p_0, -\Bp) }
}{
   p^0 = \omega_\Bp
}
\end{dmath}
where we have used \( \tilde{j}^\conj(p_0, \Bp) = \tilde{j}(p_0, -\Bp) \).  This gives
\begin{dmath}\label{eqn:qftLecture13:100}
\phi(x) =
\int \frac{d^3 p}{(2\pi)^3 \sqrt{ 2 \omega_\Bp } }
\evalbar{
   \lr{
      e^{-i p \cdot x}
      \lr{ a_\Bp + i \frac{\tilde{j}(p)}{\sqrt{2 \omega_\Bp}} }
      + e^{i p \cdot x }
      \lr{ a_\Bp^\dagger - i \frac{\tilde{j}^\conj(p)}{\sqrt{2 \omega_\Bp}} }
   }
}{
p^0 = \omega_\Bp
}
\end{dmath}

It was left as an exercise to show that given
\begin{dmath}\label{eqn:qftLecture13:120}
H = \int d^3 p \lr{ \inv{2} \pi^2 + \inv{2} \lr{ \spacegrad \phi}^2 + \frac{m^2}{2} \phi^2 },
\end{dmath}
we obtain
\begin{dmath}\label{eqn:qftLecture13:140}
H_{\text{after}} =
\int d^3 x \omega_\Bp
\lr{ a_\Bp^\dagger - i \frac{\tilde{j}^\conj(p)}{\sqrt{2 \omega_\Bp}} }
\lr{ a_\Bp + i \frac{\tilde{j}(p)}{\sqrt{2 \omega_\Bp}} }.
\end{dmath}

System in ground state
\begin{equation}\label{eqn:qftLecture13:160}
\bra{0} \hatH_{\text{before}} \ket{0} = \expectation{E}_{\text{before}} = 0.
\end{equation}
\begin{dmath}\label{eqn:qftLecture13:180}
\bra{0} \hatH_{\text{after}} \ket{0} = \expectation{E}_{\text{after}}
=
\int d^3 x \omega_\Bp
\frac{ \tilde{j}^\conj(p) \tilde{j}(p)}{2 \omega_\Bp}
=
\inv{2} \int d^3 x
\Abs{\tilde{j}(p)}^2.
\end{dmath}
We can identify
\begin{dmath}\label{eqn:qftLecture13:200}
N(\Bp) =
\frac{\Abs{\tilde{j}(p)}^2}{2 \omega_\Bp},
\end{dmath}
as the number density of particles with momentum \( \Bp \).

\section{Digression: coherent states.}
\index{coherent state}
\makedefinition{Coherent state.}{dfn:qftLecture13:220}{
A coherent state is an eigenstate of the destruction operator
\begin{equation*}
a \ket{\alpha} = \alpha \ket{\alpha}.
\end{equation*}
} % definition
For the SHO, if we solve for such a coherent state, we find
\begin{dmath}\label{eqn:qftLecture13:240}
\ket{\alpha} = \text{constant} \times \sum_{n = 0}^\infty \frac{\alpha^n}{n!} \lr{ a^\dagger }^n \ket{0}.
\end{dmath}
If we assume the existence of a coherent state
\begin{dmath}\label{eqn:qftLecture13:260}
a_\Bp \ket{
\frac{\tilde{j}(p)}{\sqrt{2 \omega_\Bp}}
}
=
\frac{\tilde{j}(p)}{\sqrt{2 \omega_\Bp}}
\ket{
\frac{\tilde{j}(p)}{\sqrt{2 \omega_\Bp}}
},
\end{dmath}
then the expectation value of the number operator with respect to this state is the number density identified in \cref{eqn:qftLecture13:200}
\begin{equation}\label{eqn:qftLecture13:1200}
\bra{
\frac{\tilde{j}(p)}{\sqrt{2 \omega_\Bp}}
}
a_\Bp^\dagger a_\Bp
\ket{
\frac{\tilde{j}(p)}{\sqrt{2 \omega_\Bp}}
} = \frac{\Abs{\tilde{j}(p)}^2}{2 \omega_\Bp} = N(\Bp).
\end{equation}

\section{Feynman's Green's function}
\index{Feynman propagator}

\begin{dmath}\label{eqn:qftLecture13:280}
D_F(x)
=
\Theta(x^0) D(x) +
\Theta(-x^0) D(-x)
=
\Theta(x^0) \bra{0} \phi(x) \phi(0) \ket{0}
+\Theta(x^0) \bra{0} \phi(-x) \phi(0) \ket{0}.
\end{dmath}
Utilizing a translation operation \( U(a) = e^{i a_\mu P^\mu } \), where \( U(a) \phi(y) U^\dagger(a) = \phi(y + a) \), this second operation can be written as
\begin{dmath}\label{eqn:qftLecture13:300}
\bra{0} \phi(-x) \phi(0) \ket{0}
=
\bra{0} U^\dagger(a) U(a) \phi(-x) U^\dagger(a) U(a) \phi(0) U^\dagger(a) U(a) \ket{0}
=
\bra{0} U(a) \phi(-x) U^\dagger(a) U(a) \phi(0) U^\dagger(a) \ket{0}
=
\bra{0} \phi(-x + a) \phi(a) \ket{0},
\end{dmath}
In particular, with \( a = x \)
\begin{dmath}\label{eqn:qftLecture13:320}
\bra{0} \phi(-x) \phi(0) \ket{0}
=
\bra{0} \phi(0) \phi(x) \ket{0},
\end{dmath}
so the Feynman's Green function can be written
\begin{dmath}\label{eqn:qftLecture13:340}
D_F(x) =
\Theta(x^0) \bra{0} \phi(x) \phi(0) \ket{0}
+\Theta(x^0) \bra{0} \phi(x) \phi(x) \ket{0}
=
\bra{0}
\lr{
\Theta(x^0)
\phi(x) \phi(0)
+
\Theta(-x^0)
\phi(0) \phi(x)
}
\ket{0}.
\end{dmath}
We define
\index{time ordered product}
\makedefinition{Time ordered product.}{dfn:qftLecture13:360}{
The time ordered product of two operators is defined as
\begin{equation*}
T(\phi(x) \phi(y)) =
\left\{
\begin{array}{l l}
\phi(x)\phi(y) & \quad \mbox{\( x^0 > y^0 \)} \\
\phi(y)\phi(x) & \quad \mbox{\( x^0 < y^0 \)} \\
\end{array}
\right.,
\end{equation*}
or
\begin{equation*}
T(\phi(x) \phi(y)) =
\phi(x)\phi(y) \Theta(x^0 - y^0)
+
\phi(y)\phi(x) \Theta(y^0 - x^0).
\end{equation*}
} % definition

Using this helpful construct, the Feynman's Green function can now be written in a very simple fashion
\boxedEquation{eqn:qftLecture13:380}{
D_F(x) = \bra{0} T(\phi(x) \phi(0)) \ket{0}.
}

\section{Interacting field theory: perturbation theory in QFT.}
\index{perturbation theory}

We perturb the Hamiltonian
\begin{dmath}\label{eqn:qftLecture13:500}
H = H_0 + H_{\text{int}},
\end{dmath}
where \( H_0 \) is the free Hamiltonian and \( H_{\text{int}} \) is the interaction term (the perturbation).

\paragraph{Example:}

\begin{equation}\label{eqn:qftLecture13:460}
\begin{aligned}
H_0 &= SHO = \frac{p^2}{2} + \frac{\omega^2 q^2}{2} \\
H_{\text{int}} &= \lambda q^4.
\end{aligned}
\end{equation}
i.e.  the anharmonic oscillator.

In QFT
\begin{dmath}\label{eqn:qftLecture13:480}
\begin{aligned}
H_0 &=
\int d^3 x \lr{ \inv{2} \pi^2 + \inv{2} \lr{ \spacegrad \phi}^2 + \frac{m^2}{2} \phi^2 } \\
H_{\text{int}} &=
\lambda \int d^3 x \phi^4.
\end{aligned}
\end{dmath}

We will expand the interaction in small \( \lambda \).  Perturbation theory is the expansion in a small dimensionless coupling constant, such as
\begin{itemize}
\item \( \lambda \) in \( \lambda \phi^4 \) theory,
\item \( \alpha = e^2/4 \pi \sim \inv{137} \) in QED, and
\item \( \alpha_s \) in QCD.
\end{itemize}

\section{Perturbation theory, interaction representation and Dyson formula}
\index{interaction representation}
\index{Dyson formula}

\begin{dmath}\label{eqn:qftLecture13:520}
H = H_0 + H_{\text{int}}
\end{dmath}
Example interaction
\begin{dmath}\label{eqn:qftLecture13:540}
H_{\text{int}} = \lambda \int d^3 x \phi^4.
\end{dmath}

We know all there is to know about \( H_0 \) (decoupled SHOs, ...)
\begin{dmath}\label{eqn:qftLecture13:560}
H_0 \ket{0} = \ket{0} E^0_{\text{vac}}
\end{dmath}
where \( E^0_{\text{vac}} = 0 \).  Assume
\begin{dmath}\label{eqn:qftLecture13:580}
\lr{ H_0 + H_{\text{int}} } \ket{\Omega} = \ket{\Omega} E_{\text{vac}},
\end{dmath}
where the ground state energy of the perturbed system is zero when \( \lambda = 0 \).  That is \( E_{\text{vac}}(\lambda = 0 ) = 0 \).

So for
\begin{dmath}\label{eqn:qftLecture13:600}
\evalbar{\phi(x) }{x^0 = t_0, \text{some fixed value}}
=
\int \frac{d^3}{(2 \pi)^3 \sqrt{ 2 \omega_\Bp } }
\evalbar{
   \lr{
   e^{-i p \cdot x} a_\Bp
   + e^{i p \cdot x} a_\Bp^\dagger }
   }
{
p^0 = \omega_\Bp
}.
\end{dmath}
Let's call \( \phi(\Bx, t_0) \) the free Schr\"{o}dinger operator, where
\( \phi(\Bx, t_0) \) is evaluated at a fixed value of \( t_0 \).  At such a point, the Schr\"{o}dinger and Heisenberg pictures coincide.
\begin{dmath}\label{eqn:qftLecture13:620}
\antisymmetric{\phi(\Bx, t_0)}{\pi(\By, t_0)} = i \delta^3(\Bx - \By).
\end{dmath}

Normally (QM) one defines the Heisenberg operator as
\begin{dmath}\label{eqn:qftLecture13:640}
O_H = e^{i H(t - t_0)} O_S e^{-i H(t - t_0)},
\end{dmath}
where \( O_H \) depends on time, and \( O_S \) is defined at a fixed time \( t_0 \), usually 0.
From \cref{eqn:qftLecture13:640} we find
\begin{dmath}\label{eqn:qftLecture13:660}
\ddt{O_H} = i \antisymmetric{H}{O_H}.
\end{dmath}
The equivalent of \cref{eqn:qftLecture13:640} in QFT is very complicated.  We'd like to develop an intermediate picture.

We will define an intermediate picture, called the ``interaction representation'', which is equivalent to the Heisenberg picture with respect to \( H_0 \).
\index{interaction picture}
\makedefinition{Interaction picture operator.}{dfn:qftLecture13:680}{
\begin{equation*}
\phi_I(t, \Bx) =
e^{i H_0(t - t_0) }
\phi(t_0, \Bx)
e^{-i H_0(t - t_0) }.
\end{equation*}
} % definition

This is familiar, and is the Heisenberg picture operator that we had in free QFT
\begin{dmath}\label{eqn:qftLecture13:700}
\phi_I(t, \Bx) =
\int \frac{d^3}{(2 \pi)^3 \sqrt{ 2 \omega_\Bp } }
\evalbar{
   \lr{
   e^{-i p \cdot x} a_\Bp
   + e^{i p \cdot x} a_\Bp^\dagger }
   }
{
p^0 = \omega_\Bp
},
\end{dmath}
where \( x_0 = t \).

The Heisenberg picture operator is
\begin{dmath}\label{eqn:qftLecture13:720}
\phi_H(t, \Bx)
=
\phi(t, \Bx)
=
e^{i H(t - t_0) }
e^{-i H_0(t - t_0) }
\lr{
   e^{i H_0(t - t_0) }
   \phi_S(t_0, \Bx)
   e^{-i H_0(t - t_0) }
}
e^{i H_0(t - t_0) }
e^{-i H(t - t_0) }
=
e^{i H(t - t_0) }
e^{-i H_0(t - t_0) }
\phi_I(t, \Bx)
e^{-i H_0(t - t_0) }
e^{i H(t - t_0) }
\end{dmath}
or
\begin{dmath}\label{eqn:qftLecture13:760}
\phi_H(t, \Bx)
=
U^\dagger(t, t_0)
\phi_I(t_0, \Bx)
U(t, t_0),
\end{dmath}
where
\begin{dmath}\label{eqn:qftLecture13:740}
U(t, t_0) =
e^{i H_0(t - t_0) }
e^{-i H(t - t_0) }.
\end{dmath}

We want to apply perturbation techniques to find \( U(t, t_0) \) which is complicated.

\begin{dmath}\label{eqn:qftLecture13:780}
i \PD{t}{} U(t, t_0)
=
i e^{i H_0(t - t_0) } i H_0
e^{-i H(t - t_0) }
+
i e^{i H_0(t - t_0) }
e^{-i H(t - t_0) } (-i H)
=
e^{i H_0(t - t_0) }
\lr{ -H_0 + H }
e^{-i H(t - t_0) }
=
e^{i H_0(t - t_0) }
H_{\text{int}}
e^{-i H_0(t - t_0) }
e^{i H_0(t - t_0) }
e^{-i H(t - t_0) },
\end{dmath}
so we have
\boxedEquation{eqn:qftLecture13:800}{
i \PD{t}{} U(t, t_0)
=
H_{\text{int}, I}(t) U(t, t_0).
}
For the (Schr\"{o}dinger) interaction \( H_{\text{int}} = \
\lambda \int d^3 x \phi^4(\Bx, t_0)  \), what we really mean by
\( H_{\text{int}, I}(t) \) is
\begin{dmath}\label{eqn:qftLecture13:820}
H_{\text{int}, I}(t) = \lambda \int d^3 x \phi_I^4(\Bx, t).
\end{dmath}

It will be more convenient to remove the explicit \( \lambda \) factor from the interaction Hamiltonian, and write instead
\begin{dmath}\label{eqn:qftLecture13:880}
H_{\text{int}, I}(t) = \int d^3 x \phi_I^4(\Bx, t),
\end{dmath}
so the equation to solve is
\begin{dmath}\label{eqn:qftLecture13:1220}
i \PD{t}{} U(t, t_0)
=
\lambda H_{\text{int}, I}(t) U(t, t_0).
\end{dmath}

We assume that
\begin{dmath}\label{eqn:qftLecture13:900}
U(t, t_0)
=
U_0(t, t_0)
+ \lambda U_1(t, t_0)
+ \lambda^2 U_2(t, t_0)
+ \cdots
+ \lambda^n U_n(t, t_0).
\end{dmath}

Plugging into \cref{eqn:qftLecture13:880} we have
\begin{dmath}\label{eqn:qftLecture13:1160}
\begin{aligned}
  i &\lambda^0 \PD{t}{}U_0(t, t_0)
+ i \lambda^1 \PD{t}{}U_1(t, t_0)
+ i \lambda^2 \PD{t}{}U_2(t, t_0)
+ \cdots
+ i \lambda^n \PD{t}{}U_n(t, t_0) \\
&=
\lambda H_{\text{int}, I}(t)
\lr{
1
+ \lambda U_1(t, t_0)
+ \lambda^2 U_2(t, t_0)
+ \cdots
+ \lambda^n U_n(t, t_0)
},
\end{aligned},
\end{dmath}
so
equating equal powers of \( \lambda \) on each side gives a recurrence relation for each \( U_k, k > 0 \)
\begin{dmath}\label{eqn:qftLecture13:1180}
\PD{t}{}U_k(t, t_0) = -i H_{\text{int}, I}(t) U_{k-1}(t, t_0).
\end{dmath}

Let's consider each power in turn.
\paragraph{\(O(\lambda^0)\):}

Solving \cref{eqn:qftLecture13:800} to \( O(\lambda^0) \) gives
\begin{dmath}\label{eqn:qftLecture13:840}
i \PD{t}{} U_0(t, t_0) = 0,
\end{dmath}
or
\begin{dmath}\label{eqn:qftLecture13:860}
U(t, t_0) = 1 + O(\lambda).
\end{dmath}

\paragraph{\(O(\lambda^1)\):}
%%\begin{dmath}\label{eqn:qftLecture13:920}
%%\lambda i \PD{t}{U_1(t, t_0)} = \lambda
%%H_{\text{int}, I}(t) U_0(t, t_0)
%%\end{dmath}
%%or
\begin{dmath}\label{eqn:qftLecture13:940}
\PD{t}{U_1(t, t_0)} = -i H_{\text{int}, I}(t),
\end{dmath}
which has solution
\begin{dmath}\label{eqn:qftLecture13:960}
U_1(t, t_0) = -i \int_{t_0}^t H_{\text{int}, I}(t') dt'.
\end{dmath}

\paragraph{\(O(\lambda^2)\):}
%%\begin{dmath}\label{eqn:qftLecture13:980}
%%\lambda^2 i \PD{t}{U_2(t, t_0)} = \lambda^2
%%H_{\text{int}, I}(t) U_1(t, t_0),
%%\end{dmath}
%%or
\begin{dmath}\label{eqn:qftLecture13:1000}
\PD{t}{U_2(t, t_0)}
= -i H_{\text{int}, I}(t) U_1(t, t_0)
= (-i)^2 H_{\text{int}, I}(t)
\int_{t_0}^t H_{\text{int}, I}(t') dt',
\end{dmath}
which has solution
\begin{dmath}\label{eqn:qftLecture13:1020}
U_2(t, t_0)
= (-i )^2
\int_{t_0}^t H_{\text{int}, I}(t'') dt''
\int_{t_0}^{t''} H_{\text{int}, I}(t') dt'
= (-i )^2
\int_{t_0}^t dt''
\int_{t_0}^{t''}
dt'
H_{\text{int}, I}(t'')
H_{\text{int}, I}(t').
\end{dmath}
\paragraph{\(O(\lambda^3)\):}
\begin{dmath}\label{eqn:qftLecture13:1060}
\PD{t}{U_3(t, t_0)}
=
-i
H_{\text{int}, I}(t) U_2(t, t_0),
\end{dmath}
so
\begin{dmath}\label{eqn:qftLecture13:1240}
U_3(t, t_0)
=
-i
\int_{t_0}^t dt'''
H_{\text{int}, I}(t''') U_2(t''', t_0)
=
(-i )^3
\int_{t_0}^t dt'''
H_{\text{int}, I}(t''')
\int_{t_0}^{t'''} dt''
\int_{t_0}^{t''}
dt'
H_{\text{int}, I}(t'')
H_{\text{int}, I}(t')
=
(-i)^3
\int_{t_0}^t dt'''
\int_{t_0}^{t'''} dt''
\int_{t_0}^{t''} dt'
H_{\text{int}, I}(t''')
H_{\text{int}, I}(t'')
H_{\text{int}, I}(t').
\end{dmath}

\paragraph{Simplifying the integration region.}

For the two fold integral, the integration range is the upper triangular region sketched in \cref{fig:upperTriangleIntegrationRegion:upperTriangleIntegrationRegionFig2}.
\imageFigure{../figures/phy2403-quantum-field-theory/upperTriangleIntegrationRegionFig2}{Upper triangular integration region.}{fig:upperTriangleIntegrationRegion:upperTriangleIntegrationRegionFig2}{0.3}

\paragraph{Claim:} We can integrate over the entire square, and divide by two, provided we keep the time ordering
\begin{dmath}\label{eqn:qftLecture13:1040}
U_2(t, t_0)
= \frac{(-i )^2}{2}
\int_{t_0}^t dt''
\int_{t_0}^{t''}
dt'
T(H_{\text{int}, I}(t'') H_{\text{int}, I}(t') )
\end{dmath}

Demonstration:
\begin{dmath}\label{eqn:qftLecture13:1100}
\begin{aligned}
\frac{(-i)^2}{2}
&\int_{t_0}^t dt''
\int_{t_0}^t dt'
T( H_I(t'') H_I(t') ) \\
&=
\frac{(-i)^2}{2}
\int_{t_0}^t dt''
\int_{t_0}^t dt'
\Theta(t''- t')
H_I(t'') H_I(t')
+
\frac{(-i)^2}{2}
\int_{t_0}^t dt''
\int_{t_0}^t dt'
\Theta(t'- t'')
H_I(t') H_I(t''),
\end{aligned}
\end{dmath}
but the \( \Theta(t'' - t') \) function is non-zero only for \( t'' - t' > 0 \), or \( t' < t'' \), and the
\( \Theta(t' - t'') \) function is non-zero only for \( t' - t'' > 0 \), or \( t'' < t' \), so we can adjust the integration ranges for
\begin{dmath}\label{eqn:qftLecture13:1260}
\begin{aligned}
\frac{(-i)^2}{2}
&\int_{t_0}^t dt''
\int_{t_0}^t dt'
T( H_I(t'') H_I(t') ) \\
&=
\frac{(-i)^2}{2}
\int_{t_0}^t dt''
\int_{t_0}^{t''} dt'
H_I(t'') H_I(t')
+
\frac{(-i)^2}{2}
\int_{t_0}^{t'} dt''
\int_{t_0}^t dt'
H_I(t') H_I(t'') \\
&=
\frac{(-i)^2}{2}
\int_{t_0}^t dt''
\int_{t_0}^{t''} dt'
H_I(t'') H_I(t')
+
\frac{(-i)^2}{2}
\int_{t_0}^t dt''
\int_{t_0}^{t''} dt'
H_I(t'') H_I(t') \\
&=
U_2(t, t_0),
\end{aligned}
\end{dmath}
where we swapped integration variables in second integral.  We can clearly do the same thing for the higher order repeated integrals, but instead of a \(1/2 = 1/2!\) adjustment for the number of orderings, we will require a \( 1/n! \) adjustment for an \( n \)-fold integral.

\paragraph{Summary:}
\begin{dmath}\label{eqn:qftLecture13:1120}
\begin{aligned}
U_0 &= 1 \\
U_1 &= -i \int_{t_0}^t dt_1 H_I(t_1) \\
U_2 &= \frac{(-i)^2}{2}
\int_{t_0}^t dt_1
\int_{t_0}^t dt_2
T( H_I(t_1)
H_I(t_2) ) \\
U_3 &= \frac{(-i)^3}{3!}
\int_{t_0}^t dt_1
\int_{t_0}^t dt_2
\int_{t_0}^t dt_3
T( H_I(t_1)
H_I(t_2)
H_I(t_3)
) \\
U_n &= \frac{(-i)^n}{n!}
\int_{t_0}^t dt_1
\int_{t_0}^t dt_2
\int_{t_0}^t dt_3
\cdots
\int_{t_0}^t dt_n
T( H_I(t_1)
H_I(t_2)
\cdots
H_I(t_n)
) \\
\end{aligned}
\end{dmath}

Summing we find
\begin{dmath}\label{eqn:qftLecture13:1140}
U(t, t_0) = T \exp\lr{-i
\int_{t_0}^t dt_1 H_I(t')
}
=
\sum_{n = 0}^\infty
\frac{(-i)^n}{n!} \int_{t_0}^t dt_1 \cdots dt_n T( H_I(t_1) \cdots H_I(t_n) ).
\end{dmath}

This is called Dyson's formula.

\section{Next time.}

Our goal is to compute: \( \bra{\Omega} T(\phi(x_1) \cdots \phi(x_n)) \ket{\Omega} \).

%}
%\EndArticle
%\EndNoBibArticle
