%
% Copyright � 2018 Peeter Joot.  All Rights Reserved.
% Licenced as described in the file LICENSE under the root directory of this GIT repository.
%
%{
%%%\input{../latex/blogpost.tex}
%%%\renewcommand{\basename}{qftLecture13}
%%%\renewcommand{\dirname}{notes/phy2403/}
%%%\newcommand{\keywords}{PHY2403H}
%%%\input{../latex/peeter_prologue_print2.tex}
%%%
%%%%\usepackage{phy2403}
%%%\usepackage{peeters_braket}
%%%\usepackage{peeters_layout_exercise}
%%%\usepackage{peeters_figures}
%%%\usepackage{mathtools}
%%%\usepackage{siunitx}
%%%\usepackage{macros_cal} % LL
%%%
%%%\newcommand{\ultensor}[3]{{{#1}^{#2}}_{#3}}
%%%
%%%\beginArtNoToc
%%%\generatetitle{PHY2403H Quantum Field Theory.  Lecture 13: Forced Klein-Gordon equation, coherent states, number density, time ordered product, pole shifting, perturbation theory, Heisenberg picture, interaction picture, Dyson's formula.  Taught by Prof.\ Erich Poppitz}
%\chapter{Forced Klein-Gordon equation, coherent states, number density, time ordered product, perturbation theory, Heisenberg picture, interaction picture, Dyson's formula}
\label{chap:qftLecture13}
\index{Klein-Gordon!forced}
\index{number density}
\index{perturbation theory}
\index{Heisenberg picture}
%\index{Dyson's formula}

%%Peeter's lecture notes from class.  These may be incoherent and rough.
%%
%%These are notes for the UofT course PHY2403H, Quantum Field Theory, taught by Prof. Erich Poppitz, covering \textchapref{{1}} \citep{peskin1995introduction} content.

%%\paragraph{DISCLAIMER: Very rough notes from class, with some additional side notes.}
%%
%%These are notes for the UofT course PHY2403H, Quantum Field Theory, taught by Prof. Erich Poppitz, fall 2018.
%%%, covering \textchapref{{1}} \citep{peskin1995introduction} content.
%%
\section{Review: ``particle creation problem''.}

We imagined that we have a windowed source function \( j(y^0, \By) \), as sketched in \cref{fig:windowedImpulse:windowedImpulseFig5}, which is acting as a forcing source for the non-homogeneous Klein-Gordon equation
%\imageFigure{../figures/phy2403-quantum-field-theory/windowedImpulseFig5}{Finite window impulse response.}{fig:windowedImpulse:windowedImpulseFig5}{0.2}
\begin{equation}\label{eqn:qftLecture13:20}
\lr{ \partial_\mu \partial^\mu + m^2 } \phi = j.
\end{equation}
Our solution was
\begin{equation}\label{eqn:qftLecture13:40}
\phi(x) = \phi(x_0) + i \int d^4 y D_R( x - y) j(y),
\end{equation}
where \( \phi(x_0) \) obeys the homogeneous equation, and
\begin{equation}\label{eqn:qftLecture13:60}
D_R(x - y) = \Theta(x^0 - y^0) \lr{ D(x - y) - D(y - x) },
\end{equation}
and \( D(x) = \int \frac{d^3 p}{(2\pi)^3 2 \omega_\Bp } \evalbar{ e^{-i p \cdot x} }{p^0 = \omega_\Bp} \) is the Wightman function.

For \( x^0 > t_{\text{after}} \)
\begin{equation}\label{eqn:qftLecture13:80}
\phi(x)
=
\int \frac{d^3 p}{(2\pi)^3 \sqrt{ 2 \omega_\Bp }}
\evalbar{
   \lr{ e^{-i p \cdot x} a_\Bp + e^{i p \cdot x } a_\Bp^\dagger }
}{
   p^0 = \omega_\Bp
}
+ i
\int \frac{d^3 p}{(2\pi)^3 2 \omega_\Bp }
\evalbar{
   \lr{ e^{-i p \cdot x} \tilde{j}(p) + e^{i p \cdot x} \tilde{j}(p_0, -\Bp) }
}{
   p^0 = \omega_\Bp
}
\end{equation}
where we have used \( \tilde{j}^\conj(p_0, \Bp) = \tilde{j}(p_0, -\Bp) \).  This gives
\begin{equation}\label{eqn:qftLecture13:100}
\phi(x) =
\int \frac{d^3 p}{(2\pi)^3 \sqrt{ 2 \omega_\Bp } }
\evalbar{
   \lr{
      e^{-i p \cdot x}
      \lr{ a_\Bp + i \frac{\tilde{j}(p)}{\sqrt{2 \omega_\Bp}} }
      + e^{i p \cdot x }
      \lr{ a_\Bp^\dagger - i \frac{\tilde{j}^\conj(p)}{\sqrt{2 \omega_\Bp}} }
   }
}{
p^0 = \omega_\Bp
}
\end{equation}

It was left as an exercise to show that given
\begin{equation}\label{eqn:qftLecture13:120}
H = \int d^3 p \lr{ \inv{2} \pi^2 + \inv{2} \lr{ \spacegrad \phi}^2 + \frac{m^2}{2} \phi^2 },
\end{equation}
we obtain
\begin{equation}\label{eqn:qftLecture13:140}
H_{\text{after}} =
\int d^3 p \omega_\Bp
\lr{ a_\Bp^\dagger - i \frac{\tilde{j}^\conj(p)}{\sqrt{2 \omega_\Bp}} }
\lr{ a_\Bp + i \frac{\tilde{j}(p)}{\sqrt{2 \omega_\Bp}} }.
\end{equation}

System in ground state
\begin{equation}\label{eqn:qftLecture13:160}
\bra{0} \hatH_{\text{before}} \ket{0} = \expectation{E}_{\text{before}} = 0.
\end{equation}
\begin{equation}\label{eqn:qftLecture13:180}
\begin{aligned}
\bra{0} \hatH_{\text{after}} \ket{0} = \expectation{E}_{\text{after}}
&=
\int d^3 p \omega_\Bp
\frac{ \tilde{j}^\conj(p) \tilde{j}(p)}{2 \omega_\Bp}
\\&=
\inv{2} \int d^3 p
\Abs{\tilde{j}(p)}^2.
\end{aligned}
\end{equation}
We can identify
\begin{equation}\label{eqn:qftLecture13:200}
N(\Bp) =
\frac{\Abs{\tilde{j}(p)}^2}{2 \omega_\Bp},
\end{equation}
as the number density of particles with momentum \( \Bp \).

