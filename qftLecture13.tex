%
% Copyright � 2017 Peeter Joot.  All Rights Reserved.
% Licenced as described in the file LICENSE under the root directory of this GIT repository.
%
%{
\input{../latex/blogpost.tex}
\renewcommand{\basename}{qftLecture13}
\renewcommand{\dirname}{notes/phy2403/}
\newcommand{\keywords}{PHY2403H}
\input{../latex/peeter_prologue_print2.tex}

%\usepackage{phy2403}
\usepackage{peeters_braket}
\usepackage{peeters_layout_exercise}
\usepackage{peeters_figures}
\usepackage{mathtools}
\usepackage{siunitx}
\usepackage{macros_cal} % LL

\newcommand{\ultensor}[3]{{{#1}^{#2}}_{#3}}

\beginArtNoToc
\generatetitle{PHY2403H Quantum Field Theory.  Lecture 13: XXX.  Taught by Prof.\ Erich Poppitz}
%\chapter{XXX}
\label{chap:qftLecture13}

%%Peeter's lecture notes from class.  These may be incoherent and rough.
%%
%%These are notes for the UofT course PHY2403H, Quantum Field Theory, taught by Prof. Erich Poppitz, covering \textchapref{{1}} \citep{peskin1995introduction} content.

\paragraph{DISCLAIMER: Very rough notes from class, with some additional side notes.}

These are notes for the UofT course PHY2403H, Quantum Field Theory, taught by Prof. Erich Poppitz, fall 2018.
%, covering \textchapref{{1}} \citep{peskin1995introduction} content.

\section{Review: ``particle creation problem''.}

F1: windowed function from last lecture.

\begin{dmath}\label{eqn:qftLecture13:20}
\lr{ \partial_\mu \partial^\mu + m^2 } \phi = j
\end{dmath}

Our solution was
\begin{dmath}\label{eqn:qftLecture13:40}
\phi(x) = \phi(x_0) + i \int d^4 y D_R( x - y) j(y),
\end{dmath}
where \( \phi(x_0) \) obeys the homogeneous equation, and
\begin{dmath}\label{eqn:qftLecture13:60}
D_r(x - y) = \Theta(x^0 - y^0) \lr{ D(x - y) - D(y - x) },
\end{dmath}
and \( D(x) = \int \frac{d^3 p}{(2\pi)^3 2 \omega_\Bp } \evalbar{ e^{-i p \cdot x} }{p^0 = \omega_\Bp} \) is the Weightmann function.

For \( x^0 > t_{\text{after}} \)
\begin{dmath}\label{eqn:qftLecture13:80}
\phi(x)
=
\int \frac{d^3 p}{(2\pi)^3 \sqrt{ 2 \omega_\Bp }}
\evalbar{
   \lr{ e^{-i p \cdot x} a_\Bp + e^{i p \cdot x } a_\Bp^\dagger }
}{
   p^0 = \omega_\Bp
}
+ i
\int \frac{d^3 p}{(2\pi)^3 2 \omega_\Bp }
\evalbar{
   \lr{ e^{-i p \cdot x} \tilde{j}(p) + e^{i p \cdot x} \tilde{j}(p_0, -\Bp) }
}{
   p^0 = \omega_\Bp
}
\end{dmath}
where we have used \( \tilde{j}^\conj(p_0, \Bp) = \tilde{j}(p_0, -\Bp) \).  This gives

\begin{dmath}\label{eqn:qftLecture13:100}
\phi(x) =
\int \frac{d^3 p}{(2\pi)^3 \sqrt{ 2 \omega_\Bp } }
\evalbar{
e^{-i p \cdot x}
\lr{ a_\Bp + i \frac{\tilde{j}(p)}{\sqrt{2 \omega_\Bp}} }
+ e^{i p \cdot x }
\lr{ a_\Bp^\dagger - i \frac{\tilde{j}^\conj(p)}{\sqrt{2 \omega_\Bp}} }
}{
p^0 = \omega_\Bp
}
\end{dmath}

DIY: Given
\begin{dmath}\label{eqn:qftLecture13:120}
H = \int d^3 x \lr{ \inv{2} \pi^2 + \inv{2} \lr{ \spacegrad \phi}^2 + \frac{m^2}{2} \phi^2 },
\end{dmath}
we should get
\begin{dmath}\label{eqn:qftLecture13:140}
H_{\text{after}} =
\int d^3 x \omega_\Bp
\lr{ a_\Bp^\dagger - i \frac{\tilde{j}^\conj(p)}{\sqrt{2 \omega_\Bp}} }
\lr{ a_\Bp + i \frac{\tilde{j}(p)}{\sqrt{2 \omega_\Bp}} }
\end{dmath}

System in ground state
\begin{equation}\label{eqn:qftLecture13:160}
\bra{0} \hatH_{\text{before}} \ket{0} = \expectation{E}_{\text{before}} = 0.
\end{equation}
\begin{dmath}\label{eqn:qftLecture13:180}
\bra{0} \hatH_{\text{after}} \ket{0} = \expectation{E}_{\text{after}}
=
\int d^3 x \omega_\Bp
\frac{ \tilde{j}^\conj(p) \tilde{j}(p)}{2 \omega_\Bp}
=
\inv{2} \int d^3 x
\Abs{j(p)}^2.
\end{dmath}
We can identify
\begin{dmath}\label{eqn:qftLecture13:200}
N(\Bp) =
\frac{\Abs{j(p)}^2}{2 \omega_\Bp},
\end{dmath}
as the number density of particles with momentum \( \Bp \).

\makedefinition{Coherent state.}{dfn:qftLecture13:220}{
A coherent state is a state
\begin{equation*}
a \ket{\alpha} = \alpha \ket{\alpha}.
\end{equation*}
} % definition
For the SHO, if we solve for such a coherent state, we find
\begin{dmath}\label{eqn:qftLecture13:240}
\alpha = \text{const} \sum_{n = 0}^\infty \frac{\alpha^n}{n!} \lr{ a^\dagger }^n \ket{0}.
\end{dmath}
For this case, we have a coherent state
\begin{dmath}\label{eqn:qftLecture13:260}
a_p \ket{
\frac{j(p)}{2 \omega_\Bp}
}
=
\frac{j(p)}{2 \omega_\Bp}
\ket{
\frac{j(p)}{2 \omega_\Bp}
}.
\end{dmath}

\section{Feynman's Green's function}

\begin{dmath}\label{eqn:qftLecture13:280}
D_F(x)
=
\Theta(x^0) D(x) +
\Theta(-x^0) D(-x)
=
\Theta(x^0) \bra{0} \phi(x) \phi(0) \ket{0}
+\Theta(x^0) \bra{0} \phi(-x) \phi(0) \ket{0}
\end{dmath}
Utilizing a translation operation \( U(a) = e^{i a_\mu P^\mu } \), where \( U(a) \phi(y) U^\dagger(a) = \phi(y + a) \), this second operation can be written as
\begin{dmath}\label{eqn:qftLecture13:300}
\bra{0} \phi(-x) \phi(0) \ket{0}
=
\bra{0} U^\dagger(a) U(a) \phi(-x) U^\dagger(a) U(a) \phi(0) U^\dagger(a) U(a) \ket{0}
=
\bra{0} U(a) \phi(-x) U^\dagger(a) U(a) \phi(0) U^\dagger(a) \ket{0}
=
\bra{0} \phi(-x + a) \phi(a) \ket{0},
\end{dmath}
In particular, with \( a = x \)
\begin{dmath}\label{eqn:qftLecture13:320}
\bra{0} \phi(-x) \phi(0) \ket{0}
=
\bra{0} \phi(0) \phi(x) \ket{0},
\end{dmath}
so the Feynman's Green function can be written
\begin{dmath}\label{eqn:qftLecture13:340}
D_F(x) =
\Theta(x^0) \bra{0} \phi(x) \phi(0) \ket{0}
+\Theta(x^0) \bra{0} \phi(x) \phi(x) \ket{0}
=
\bra{0}
\lr{
\Theta(x^0)
\phi(x) \phi(0)
+
\Theta(-x^0)
\phi(0) \phi(x)
}
\ket{0}.
\end{dmath}
We define
\makedefinition{Time ordered product.}{dfn:qftLecture13:360}{
\begin{equation*}
T(\phi(x) \phi(y)) =
\left\{
\begin{array}{l l}
\phi(x)\phi(y) & \quad \mbox{\( x^0 > y^0 \)} \\
\phi(y)\phi(x) & \quad \mbox{\( x^0 < y^0 \)} \\
\end{array}
\right.
\end{equation*}
} % definition
so the Feynman's Green function can now be written in a very simple fashion
\boxedEquation{eqn:qftLecture13:380}{
D_F(x) = \bra{0} T(\phi(x) \phi(0)) \ket{0}.
}

\paragraph{Remark}

Recall that the four dimensional form of the Green's function was
\begin{dmath}\label{eqn:qftLecture13:400}
D_F = i \int \frac{d^4 p}{(2 \pi)^4} e^{-i p \cdot x} \inv{ p^2 - m^2 }
\end{dmath}

where we were going around the poles like
F2

We can make this explicit by introducing an offset
\begin{dmath}\label{eqn:qftLecture13:420}
D_F = i \int \frac{d^4 p}{(2 \pi)^4} e^{-i p \cdot x} \inv{ p^2 - m^2 + i \epsilon }
\end{dmath}
which puts the poles at

\begin{dmath}\label{eqn:qftLecture13:440}
(p^0)^2 = \pm \sqrt{ \omega_\Bp - i \epsilon }
= \pm \omega_\Bp \lr{ 1 - \frac{i \epsilon}{\omega_\Bp^2} }^{1/2}
= \pm \omega_\Bp \lr{ 1 - \inv{2} \frac{i \epsilon}{\omega_\Bp^2} }
=
\left\{
\begin{array}{l}
+\omega_\Bp - \inv{2} i \frac{\epsilon}{\omega_\Bp} \\
-\omega_\Bp + \inv{2} i \frac{\epsilon}{\omega_\Bp} \\
\end{array}
\right.
\end{dmath}

\section{Interacting field theory: pertubation theory in QFT.}

We perturb the Hamiltonian
\begin{dmath}\label{eqn:qftLecture13:500}
H = H_0 + H_{\text{int}}
\end{dmath}
where \( H_0 \) is the free Hamiltonian and \( H_{\text{int}} \) is the interaction term (the perturbation).

\paragraph{Example:}

\begin{equation}\label{eqn:qftLecture13:460}
\begin{aligned}
H_0 &= SHO = \frac{p^2}{2} + \frac{\omega^2 q^2}{2} \\
H_{\text{int}} &= \lambda q^4,
\end{aligned}
\end{equation}
i.e.  the anharmonic oscillator.

In QFT
\begin{dmath}\label{eqn:qftLecture13:480}
\begin{aligned}
H_0 &=
\int d^3 x \lr{ \inv{2} \pi^2 + \inv{2} \lr{ \spacegrad \phi}^2 + \frac{m^2}{2} \phi^2 } \\
H_{\text{int}} &=
\lambda \int d^3 x \phi^4.
\end{aligned}
\end{dmath}

We will expand the interaction in small \( \lambda \).  Pertubation theory is the expansion in a small dimensionless coupling constant, such as
\begin{itemize}
\item \( \lambda \) in \( \lambda \phi^4 \) theory,
\item \( \alpha = e^2/4 \pi \sim \inv{137} \) in QED, and
\item \( \alpha_s \) in QCD.
\end{itemize}

\section{Pertubation theory, interaction representation and Dyson formula}

\begin{dmath}\label{eqn:qftLecture13:520}
H = H_0 + H_{\text{int}}
\end{dmath}
Example interaction
\begin{dmath}\label{eqn:qftLecture13:540}
H_{\text{int}} = \lambda \int d^3 x \phi^4
\end{dmath}

We know all there is to know about \( H_0 \) (decoupled SHOs, ...)
\begin{dmath}\label{eqn:qftLecture13:560}
H_0 \ket{0} = \ket{0} E^0_vac
\end{dmath}
where \( E^0_vac = 0 \).  Assume
\begin{dmath}\label{eqn:qftLecture13:580}
\lr{ H_0 + H_{\text{int}} } \ket{\Omega} = \ket{\Omega} E_vac,
\end{dmath}
where the ground state energy of the perturbed system is zero when \( \lambda = 0 \).  That is \( E_vac(\lambda = 0 ) = 0 \).

So for
\begin{dmath}\label{eqn:qftLecture13:600}
\evalbar{\phi(x) }{x^0 = t_0, \text{some fixed value}}
=
\int \frac{d^3}{(2 \pi)^3 \sqrt{ 2 \omega_\Bp } }
\evalbar{
   \lr{
   e^{-i p \cdot x} a_\Bp
   + e^{i p \cdot x} a_\Bp^\dagger }
   }
{
p^0 = \omega_\Bp
}.
\end{dmath}
Let's call \( \phi(\Bx, t_0) \) the free Schr\"{o}dinger operator, where
\( \phi(\Bx, t_0) \) is evaluated at a fixed value of \( t_0 \).  At such a point, the Schr\"{o}dinger and Heisenberg pictures concide.
\begin{dmath}\label{eqn:qftLecture13:620}
\antisymmetric{\phi(\Bx, t_0)}{\pi(\By, t_0)} = i \delta^3(\Bx - \By).
\end{dmath}

Normally (QM) one defines the Heisenberg oeprator as
\begin{dmath}\label{eqn:qftLecture13:640}
O_H = e^{i H(t - t_0)} O_S e^{-i H(t - t_0)},
\end{dmath}
where \( O_H \) depends on time, and \( O_S \) is defined at a fixed time \( t_0 \), usually 0.
From \cref{eqn:qftLecture13:640} we find
\begin{dmath}\label{eqn:qftLecture13:660}
\ddt{O_H} = i \antisymmetric{H}{O_H}.
\end{dmath}
The equivalent of \cref{eqn:qftLecture13:640} in QFT is very complicated.  We'd like to develop an intermediate picture.

We will define an intermediate picture, called the ``interaction representation'', which is equivalent to the Heisenberg picture with respect to \( H_0 \).
\makedefinition{Intermediate picture operator.}{dfn:qftLecture13:680}{
\begin{equation*}
\phi_I(t, \Bx) =
e^{i H_0(t - t_0) }
\phi(t_0, \Bx)
e^{-i H_0(t - t_0) }.
\end{equation*}
} % definition

This is familiar, and is the Heisenberg picture operator that we had in free QFT
\begin{dmath}\label{eqn:qftLecture13:700}
\phi_I(t, \Bx) =
\int \frac{d^3}{(2 \pi)^3 \sqrt{ 2 \omega_\Bp } }
\evalbar{
   \lr{
   e^{-i p \cdot x} a_\Bp
   + e^{i p \cdot x} a_\Bp^\dagger }
   }
{
p^0 = \omega_\Bp
},
\end{dmath}
where \( x_0 = t \).

The Hiesenberg picture operator is
\begin{dmath}\label{eqn:qftLecture13:720}
\phi_H(t, \Bx)
=
\phi(t, \Bx)
=
e^{i H(t - t_0) }
e^{-i H_0(t - t_0) }
\lr{
   e^{i H_0(t - t_0) }
   \phi_S(t_0, \Bx)
   e^{-i H_0(t - t_0) }
}
e^{i H_0(t - t_0) }
e^{-i H(t - t_0) }
=
e^{i H(t - t_0) }
e^{-i H_0(t - t_0) }
\phi_I(t, \Bx)
e^{-i H_0(t - t_0) }
e^{i H(t - t_0) }
\end{dmath}
or
\begin{dmath}\label{eqn:qftLecture13:760}
\phi_H(t, \Bx)
=
U^\dagger(t, t_0)
\phi_I(t_0, \Bx)
U(t, t_0),
\end{dmath}
where
\begin{dmath}\label{eqn:qftLecture13:740}
U(t, t_0) =
e^{i H_0(t - t_0) }
e^{-i H(t - t_0) }.
\end{dmath}

We want to apply perturbutation techniques to find \( U(t, t_0) \) which is complicated.

\begin{dmath}\label{eqn:qftLecture13:780}
i \PD{t}{} U(t, t_0)
=
i e^{i H_0(t - t_0) } i H_0
e^{-i H(t - t_0) }
+
i e^{i H_0(t - t_0) }
e^{-i H(t - t_0) } (-i H)
=
e^{i H_0(t - t_0) }
\lr{ -H_0 + H }
e^{-i H(t - t_0) }
=
e^{i H_0(t - t_0) }
H_{\text{int}}
e^{-i H_0(t - t_0) }
e^{i H_0(t - t_0) }
e^{-i H(t - t_0) }
\end{dmath}
so we have
\boxedEquation{eqn:qftLecture13:800}{
i \PD{t}{} U(t, t_0)
=
H_{\text{int}, I}(t) U(t, t_0).
}
For the (Schr\"{o}dinger) interaction \( H_{\text{int}} = \
\lambda \int d^3 x \phi^4(\Bx, t_0)  \), what we really mean by
\( H_{\text{int}, I}(t) \) is
\begin{dmath}\label{eqn:qftLecture13:820}
H_{\text{int}, I}(t) = \lambda \int d^3 x \phi_I^4(\Bx, t)
\end{dmath}

Solving \cref{eqn:qftLecture13:800} to \( O(\lambda^0) \) gives
\begin{dmath}\label{eqn:qftLecture13:840}
i \PD{t}{} U(t, t_0) = 0,
\end{dmath}
or
\begin{dmath}\label{eqn:qftLecture13:860}
U(t, t_0) = 1 + O(\lambda).
\end{dmath}
Assume
Let's rejig things by taking \( \lambda \) out of \( H_{\text{int}, I}(t) \) as
\begin{dmath}\label{eqn:qftLecture13:880}
\begin{aligned}
H_{\text{int}, I}(t) &= \int d^3 x \phi_I^4(\Bx, t) \\
i \PD{t}{} U(t, t_0)
&=
\lambda H_{\text{int}, I}(t) U(t, t_0).
\end{aligned}
\end{dmath}
and assume that
\begin{dmath}\label{eqn:qftLecture13:900}
U(t, t_0)
=
1
+ \lambda U_1(t, t_0)
+ \lambda^2 U_2(t, t_0)
+ \cdots
+ \lambda^n U_n(t, t_0)
\end{dmath}

Plugging into \cref{eqn:qftLecture13:880} and equate equal powers of \( \lambda \) on two sides
\begin{dmath}\label{eqn:qftLecture13:1160}
i \lambda \PD{t}{}U_1(t, t_0)
+ i \lambda^2 \PD{t}{}U_2(t, t_0)
+ \cdots
+ i \lambda^n \PD{t}{}U_n(t, t_0)
=
\lambda H_{\text{int}, I}(t)
\lr{
1
+ \lambda U_1(t, t_0)
+ \lambda^2 U_2(t, t_0)
+ \cdots
+ \lambda^n U_n(t, t_0)
},
\end{dmath}
so
\begin{dmath}\label{eqn:qftLecture13:1180}
\PD{t}{}U_k(t, t_0) = -i H_{\text{int}, I}(t) U_{k-1}(t, t_0).
\end{dmath}

Let's consider each power in turn
\paragraph{\(O(\lambda^1)\):}
\begin{dmath}\label{eqn:qftLecture13:920}
\lambda i \PD{t}{U_1(t, t_0)} = \lambda
H_{\text{int}, I}(t) U_0(t, t_0)
\end{dmath}
or
\begin{dmath}\label{eqn:qftLecture13:940}
i \PD{t}{U_1(t, t_0)} = H_{\text{int}, I}(t),
\end{dmath}
which has solution
\begin{dmath}\label{eqn:qftLecture13:960}
U_1(t, t_0) = -i \int_{t_0}^t H_{\text{int}, I}(t') dt'.
\end{dmath}

\paragraph{\(O(\lambda^2)\):}
\begin{dmath}\label{eqn:qftLecture13:980}
\lambda^2 i \PD{t}{U_2(t, t_0)} = \lambda^2
H_{\text{int}, I}(t) U_1(t, t_0),
\end{dmath}
or
\begin{dmath}\label{eqn:qftLecture13:1000}
i \PD{t}{U_2(t, t_0)} = H_{\text{int}, I}(t) U_1(t, t_0),
\end{dmath}
FIXME: expand this out explicitly (as jotted and erased on the board quickly)
which has solution
\begin{dmath}\label{eqn:qftLecture13:1020}
U_2(t, t_0)
= (-i )^2
\int_{t_0}^t H_{\text{int}, I}(t'') dt''.
\int_{t_0}^{t''} H_{\text{int}, I}(t') dt'
= (-i )^2
\int_{t_0}^t dt''
\int_{t_0}^{t''}
dt'
H_{\text{int}, I}(t'')
H_{\text{int}, I}(t').
\end{dmath}
The integration range is sketched in
F3 (the upper half of the triangle).

Claim: We can integrate over the entire square, and divide by two, provided we keep the time ordering
\begin{dmath}\label{eqn:qftLecture13:1040}
U_2(t, t_0)
= \frac{(-i )^2}{2}
\int_{t_0}^t dt''
\int_{t_0}^{t''}
dt'
T(H_{\text{int}, I}(t'') H_{\text{int}, I}(t') )
\end{dmath}

\paragraph{\(O(\lambda^3)\):}
\begin{dmath}\label{eqn:qftLecture13:1060}
\lambda^3 i \PD{t}{U_3(t, t_0)} = \lambda^3 H_{\text{int}, I}(t) U_2(t, t_0),
\end{dmath}
so
\begin{dmath}\label{eqn:qftLecture13:1080}
U_3(t, t_0)
=
-i
\int_{t_0}^t dt'''
 H_{\text{int}, I}(t''') dt'' U_2(t''', t_0)
=
(-i)^3
\int_{t_0}^t dt'''
\int_{t_0}^{t'''} dt''
\int_{t_0}^{t''} dt'
H_{\text{int}, I}(t''')
H_{\text{int}, I}(t'')
H_{\text{int}, I}(t')
\end{dmath}

Let's return to the claim.

\begin{dmath}\label{eqn:qftLecture13:1100}
\frac{(-i)^2}{2}
\int_{t_0}^t dt''
\int_{t_0}^t dt'
T( H_I(t'') H_I(t') )
=
\frac{(-i)^2}{2}
\int_{t_0}^t dt''
\int_{t_0}^t dt'
\Theta(t''- t')
H_I(t'') H_I(t')
+
\frac{(-i)^2}{2}
\int_{t_0}^t dt''
\int_{t_0}^t dt'
\Theta(t'- t'')
H_I(t') H_I(t'')
=
\frac{(-i)^2}{2}
\int_{t_0}^t dt''
\int_{t_0}^{t''} dt'
%\Theta(t''- t')
H_I(t'') H_I(t')
+
\frac{(-i)^2}{2}
\int_{t_0}^{t'} dt''
\int_{t_0}^t dt'
%\Theta(t'- t'')
H_I(t') H_I(t'')
=
\frac{(-i)^2}{2}
\int_{t_0}^t dt''
\int_{t_0}^{t''} dt'
%\Theta(t''- t')
H_I(t'') H_I(t')
+
\frac{(-i)^2}{2}
\int_{t_0}^{t''} dt'
\int_{t_0}^{t} dt''
H_I(t') H_I(t'')
=
U_2(t, t_0),
\end{dmath}
where we changed integration variables in second integral.

Summarizing
\begin{dmath}\label{eqn:qftLecture13:1120}
\begin{aligned}
U_0 &= 1 \\
U_1 &= -i \int_{t_0}^t dt_1 H_I(t_1) \\
U_2 &= \frac{(-i)^2}{2}
\int_{t_0}^t dt_1
\int_{t_0}^t dt_2
T( H_I(t_1)
H_I(t_2) ) \\
U_3 &= \frac{(-i)^3}{3!}
\int_{t_0}^t dt_1
\int_{t_0}^t dt_2
\int_{t_0}^t dt_3
T( H_I(t_1)
H_I(t_2)
H_I(t_3)
) \\
U_n &= \frac{(-i)^n}{n!}
\int_{t_0}^t dt_1
\int_{t_0}^t dt_2
\int_{t_0}^t dt_3
\cdots
\int_{t_0}^t dt_n
T( H_I(t_1)
H_I(t_2)
\cdots
H_I(t_n)
) \\
\end{aligned}
\end{dmath}

Summing we find
\begin{dmath}\label{eqn:qftLecture13:1140}
U(t, t_0) = T \exp\lr{-i
\int_{t_0}^t dt_1 H_I(t')
}
=
\sum_{n = 0}^\infty
\frac{(-i)^n}{n!} \int_{t_0}^t dt_1 \cdots dt_n T( H_I(t_1) \cdots H_I(t_n) ).
\end{dmath}

Goal: \( \bra{\Omega} T(\phi(x_1) \cdots \phi(x_n)) \ket{\Omega} \).

%}
%\EndArticle
\EndNoBibArticle
