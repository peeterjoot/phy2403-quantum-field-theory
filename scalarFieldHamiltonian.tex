%
% Copyright � 2016 Peeter Joot.  All Rights Reserved.
% Licenced as described in the file LICENSE under the root directory of this GIT repository.
%
%{
%\input{../latex/blogpost.tex}
%\renewcommand{\basename}{scalarFieldHamiltonian}
%\renewcommand{\dirname}{notes/phy2403/}
%%\newcommand{\dateintitle}{}
%%\newcommand{\keywords}{}
%
%\input{../latex/peeter_prologue_print2.tex}
%
%\usepackage{peeters_layout_exercise}
%\usepackage{peeters_braket}
%\usepackage{peeters_figures}
%
%\beginArtNoToc
%
%\generatetitle{Hamiltonian for a scalar field}
%\chapter{Hamiltonian for a scalar field}

\makeproblem{}{problem:scalarFieldHamiltonian:1}{
In \citep{qftLectureNotes} it is left as an exercise to expand the scalar field Hamiltonian in terms of the raising and lowering operators.  Let's do that.
} % problem

\makeanswer{problem:scalarFieldHamiltonian:1}{
The field operator expanded in terms of the raising and lowering operators is

\begin{dmath}\label{eqn:scalarFieldHamiltonian:20}
\phi(x)
=
\int \frac{ d^3 k}{ (2 \pi)^{3/2} \sqrt{ 2 \omega_k } } \lr{
a_\Bk e^{-i k \cdot x}
+ a_\Bk^\dagger e^{i k \cdot x}
}
=
\int \frac{ d^3 k}{ (2 \pi)^{3/2} \sqrt{ 2 \omega_k } } \lr{
a_\Bk e^{-i \omega_k t + i \Bk \cdot \Bx}
+a_\Bk^\dagger e^{i \omega_k t - i \Bk \cdot \Bx}
}
=
\int \frac{ d^3 k}{ (2 \pi)^{3/2} \sqrt{ 2 \omega_k } } \lr{
a_\Bk e^{-i \omega_k t + i \Bk \cdot \Bx}
+a_{-\Bk}^\dagger e^{i \omega_k t + i \Bk \cdot \Bx}
}
=
\int \frac{ d^3 k}{ (2 \pi)^{3/2} \sqrt{ 2 \omega_k } } \lr{
a_\Bk e^{-i \omega_k t } +a_{-\Bk}^\dagger e^{i \omega_k t}
}
e^{ i \Bk \cdot \Bx}
.
\end{dmath}

Note that \( x \) and \( k \) here are both four-vectors, so this field is dependent on a spacetime point, but the integration is over a spatial volume.  This is discussed in the class notes but also justified nicely in \citep{peskin1995introduction} using the structure of the raising and lower operators.  The trick of reversing the sign above is also from that text.

The Hamiltonian in terms of the fields was
\begin{dmath}\label{eqn:scalarFieldHamiltonian:40}
H = \inv{2} \int d^3 x \lr{ \Pi^2 + \lr{ \spacegrad \phi }^2 + \mu^2 \phi^2 }.
\end{dmath}

The field derivatives are

\begin{dmath}\label{eqn:scalarFieldHamiltonian:60}
\Pi
= \partial_0 \phi
\int \frac{ d^3 k}{ (2 \pi)^{3/2} \sqrt{ 2 \omega_k } } \lr{
a_\Bk e^{-i \omega_k t } +a_{-\Bk}^\dagger e^{i \omega_k t}
}
e^{ i \Bk \cdot \Bx}
=
i
\int \frac{ d^3 k}{ (2 \pi)^{3/2}} \sqrt{ \frac{\omega_k}{2} } \lr{
-a_\Bk e^{-i \omega_k t } +a_{-\Bk}^\dagger e^{i \omega_k t}
}
e^{ i \Bk \cdot \Bx}
,
\end{dmath}

and

\begin{dmath}\label{eqn:scalarFieldHamiltonian:80}
\partial_n \phi
= \partial_n
\int \frac{ d^3 k}{ (2 \pi)^{3/2} \sqrt{ 2 \omega_k } } \lr{
a_\Bk e^{-i \omega_k t } +a_{-\Bk}^\dagger e^{i \omega_k t}
}
e^{ i \Bk \cdot \Bx}
=
i
\int \frac{ d^3 k}{ (2 \pi)^{3/2}} \frac{k^n}{ \sqrt{ 2 \omega_k } } \lr{
a_\Bk e^{-i \omega_k t } +a_{-\Bk}^\dagger e^{i \omega_k t}
}
e^{ i \Bk \cdot \Bx}
.
\end{dmath}

Introducing a second set of momentum variables \( \Bj \), the momentum portion of the Hamiltonian is

\begin{dmath}\label{eqn:scalarFieldHamiltonian:100}
\inv{2} \int d^3 x \Pi^2
=
-\inv{2}
\inv{(2 \pi)^{3}}
\int d^3 x
\int
d^3 j
d^3 k
\frac{
\omega_j
\omega_k
}{2}
\lr{
-a_\Bj e^{-i \omega_j t }
+a_{-\Bj}^\dagger e^{i \omega_j t }
}
 \lr{
-a_\Bk e^{-i \omega_k t }
+a_{-\Bk}^\dagger e^{i \omega_k t }
}
e^{ i \Bk \cdot \Bx}
e^{ i \Bj \cdot \Bx}
=
-\inv{2}
\int
d^3 j
d^3 k
\frac{
\omega_j
\omega_k
}{2}
\lr{
-a_\Bj e^{-i \omega_j t }
+a_{-\Bj}^\dagger e^{i \omega_j t }
}
 \lr{
-a_\Bk e^{-i \omega_k t }
+a_{-\Bk}^\dagger e^{i \omega_k t }
}
\delta^3( \Bk + \Bj )
=
-\inv{2}
\int
d^3 k
\frac{
\omega_k^2
}{2}
\lr{
-a_{-\Bk} e^{-i \omega_k t }
+a_{\Bk}^\dagger e^{i \omega_k t }
}
 \lr{
-a_\Bk e^{-i \omega_k t }
+a_{-\Bk}^\dagger e^{i \omega_k t }
}
%=
%-\inv{4}
%\inv{(2 \pi)^{3}}
%\int d^3 x
%\int
%d^3 j
%d^3 k
%\sqrt{
%\omega_j
%\omega_k}
%\biglr{
%  a_\Bj^\dagger a_\Bk^\dagger e^{i (\omega_k + \omega_j) t - i (\Bk + \Bj) \cdot \Bx}
%+ a_\Bj a_\Bk e^{-i (\omega_j + \omega_k) t + i (\Bj + \Bk) \cdot \Bx}
% - a_\Bj^\dagger a_\Bk e^{-i (\omega_k -\omega_j) t - i (\Bj - \Bk) \cdot \Bx}
%- a_\Bj a_\Bk^\dagger e^{-i (\omega_j - \omega_k) t - i (\Bk - \Bj) \cdot \Bx}
%}
%=
%-\inv{4}
%\int
%d^3 j
%d^3 k
%\sqrt{
%\omega_j
%\omega_k}
%\biglr{
%  a_\Bj^\dagger a_\Bk^\dagger e^{i (\omega_k + \omega_j) t } \delta^3(\Bk + \Bj)
%+ a_\Bj a_\Bk e^{-i (\omega_j + \omega_k) t } \delta^3(-\Bj - \Bk)
% - a_\Bj^\dagger a_\Bk e^{-i (\omega_k -\omega_j) t } \delta^3(\Bj - \Bk)
%- a_\Bj a_\Bk^\dagger e^{-i (\omega_j - \omega_k) t } \delta^3(\Bk - \Bj)
%}
=
-\inv{4}
\int
d^3 k
\omega_k
\lr{
  a_{-\Bk}^\dagger a_\Bk^\dagger e^{2 i \omega_k t }
+ a_{-\Bk} a_\Bk e^{- 2 i \omega_k t }
- a_\Bk^\dagger a_\Bk
- a_{-\Bk} a_{-\Bk}^\dagger
}.
\end{dmath}

For the gradient portion of the Hamiltonian we have

\begin{dmath}\label{eqn:scalarFieldHamiltonian:120}
\inv{2} \int d^3 x \lr{ \spacegrad \phi }^2
=
-\inv{2}
\inv{(2 \pi)^{3}}
\int d^3 x
\int
d^3 j
d^3 k
\inv{ \sqrt{ 4 \omega_j \omega_k } }
\lr{ \sum_{n=1}^3 j^n k^n }
\lr{
 a_\Bj e^{-i \omega_j t }
+a_{-\Bj}^\dagger e^{i \omega_j t }
}
 \lr{
 a_\Bk e^{-i \omega_k t }
+a_{-\Bk}^\dagger e^{i \omega_k t }
}
e^{ i \Bj \cdot \Bx}
e^{ i \Bk \cdot \Bx}
=
-\inv{2}
\int
d^3 j
d^3 k
\inv{ \sqrt{ 4 \omega_j \omega_k } }
\Bj \cdot \Bk
\lr{
 a_\Bj e^{-i \omega_j t }
+a_{-\Bj}^\dagger e^{i \omega_j t }
}
 \lr{
 a_\Bk e^{-i \omega_k t }
+a_{-\Bk}^\dagger e^{i \omega_k t }
}
\delta^3(\Bj + \Bk)
=
\inv{2}
\int
d^3 k
\inv{ \sqrt{ 4 \omega_k \omega_k } }
\Bk^2
\lr{
 a_{-\Bk} e^{-i \omega_k t }
+a_{\Bk}^\dagger e^{i \omega_k t }
}
 \lr{
 a_\Bk e^{-i \omega_k t }
+a_{-\Bk}^\dagger e^{i \omega_k t }
}
=
\inv{4}
\int
d^3 k
\frac{\Bk^2}{ \omega_k }
\biglr{
   a_{-\Bk}^\dagger a_\Bk^\dagger e^{2 i \omega_k t }
 + a_{-\Bk} a_\Bk e^{- 2 i \omega_k t }
 + a_\Bk^\dagger a_\Bk
 + a_{-\Bk} a_{-\Bk}^\dagger
}.
\end{dmath}

Finally, for the mass term, we have

\begin{dmath}\label{eqn:scalarFieldHamiltonian:140}
\inv{2} \int d^3 x \mu^2 \phi^2
=
\frac{\mu^2}{2}
\inv{(2 \pi)^{3}}
\int d^3 x
\int
d^3 j
d^3 k
\inv{ \sqrt{ 4 \omega_j \omega_k } }
\biglr{
 a_\Bj e^{-i \omega_j t }
+a_{-\Bj}^\dagger e^{i \omega_j t }
}
 \lr{
 a_\Bk e^{-i \omega_k t }
+a_{-\Bk}^\dagger e^{i \omega_k t }
}
e^{ i \Bj \cdot \Bx}
e^{ i \Bk \cdot \Bx}
=
\frac{\mu^2}{2}
\int
d^3 j
d^3 k
\inv{ \sqrt{ 4 \omega_j \omega_k } }
\biglr{
 a_\Bj e^{-i \omega_j t }
+a_{-\Bj}^\dagger e^{i \omega_j t }
}
 \lr{
 a_\Bk e^{-i \omega_k t }
+a_{-\Bk}^\dagger e^{i \omega_k t }
}
\delta^3(\Bj + \Bk)
=
\frac{\mu^2}{2}
\int
d^3 k
\inv{ 2 \omega_k }
\biglr{
 a_{-\Bk} e^{-i \omega_k t }
+a_{\Bk}^\dagger e^{i \omega_k t }
}
 \lr{
 a_\Bk e^{-i \omega_k t }
+a_{-\Bk}^\dagger e^{i \omega_k t }
}
%=
%\frac{\mu^2}{2}
%\inv{(2 \pi)^{3}}
%\int d^3 x
%\int
%d^3 j
%d^3 k
%\inv{ \sqrt{ 4 \omega_j \omega_k } }
%\biglr{
% a_\Bj a_\Bk e^{-i (\omega_k + \omega_j) t + i (\Bk + \Bj) \cdot \Bx}
%+a_\Bj^\dagger a_\Bk^\dagger e^{i (\omega_j + \omega_k) t - i (\Bk + \Bj) \cdot \Bx}
%+a_\Bj a_\Bk^\dagger e^{i (\omega_k - \omega_j) t - i (\Bk - \Bj) \cdot \Bx}
%+a_\Bj^\dagger a_\Bk e^{-i (\omega_k + \omega_j) t - i (\Bj - \Bk) \cdot \Bx}
%}
%=
%\frac{\mu^2}{2}
%\int
%d^3 j
%d^3 k
%\inv{ \sqrt{ 4 \omega_j \omega_k } }
%\biglr{
% a_\Bj a_\Bk e^{-i (\omega_k + \omega_j) t } \delta^3(- \Bk - \Bj)
%+a_\Bj^\dagger a_\Bk^\dagger e^{i (\omega_j + \omega_k) t } \delta^3( \Bk + \Bj)
%+a_\Bj a_\Bk^\dagger e^{i (\omega_k - \omega_j) t } \delta^3 (\Bk - \Bj)
%+a_\Bj^\dagger a_\Bk e^{-i (\omega_k + \omega_j) t } \delta^3 (\Bj - \Bk)
%}
=
\frac{\mu^2}{4}
\int
d^3 k
\inv{ \omega_k }
\lr{
 a_{-\Bk} a_\Bk e^{- 2 i \omega_k t }
+a_{-\Bk}^\dagger a_\Bk^\dagger e^{2 i \omega_k t }
+a_{-\Bk} a_{-\Bk}^\dagger
+a_\Bk^\dagger a_\Bk
}.
\end{dmath}

Now all the pieces can be put back together again

\begin{dmath}\label{eqn:scalarFieldHamiltonian:160}
\begin{aligned}
H
&=
\inv{4}
\int d^3 k
\inv{\omega_k}
\biglr{ \\
&\qquad -\omega_k^2
\lr{
  a_{-\Bk}^\dagger a_\Bk^\dagger e^{2 i \omega_k t }
+ a_{-\Bk} a_\Bk e^{- 2 i \omega_k t }
- a_\Bk^\dagger a_\Bk
- a_{-\Bk} a_{-\Bk}^\dagger
} \\
&\qquad +
\Bk^2
\lr{
  a_{-\Bk}^\dagger a_\Bk^\dagger e^{2 i \omega_k t }
+ a_{-\Bk} a_\Bk e^{- 2 i \omega_k t }
+ a_\Bk^\dagger a_\Bk
+ a_{-\Bk} a_{-\Bk}^\dagger
} \\
&\qquad +
\mu^2
\lr{
 a_{-\Bk} a_\Bk e^{- 2 i \omega_k t }
+a_{-\Bk}^\dagger a_\Bk^\dagger e^{2 i \omega_k t }
+a_{-\Bk} a_{-\Bk}^\dagger
+a_\Bk^\dagger a_\Bk
}
} \\
&=
\inv{4}
\int d^3 k
\inv{\omega_k}
\biglr{
a_{-\Bk}^\dagger a_\Bk^\dagger e^{2 i \omega_k t }
\lr{
-\omega_k^2
+ \Bk^2
+
\mu^2
} \\
&\qquad + a_{-\Bk} a_\Bk e^{- 2 i \omega_k t }
\lr{
-\omega_k^2
+ \Bk^2
+
\mu^2
} \\
&\qquad + a_\Bk a_\Bk^\dagger
\lr{
 \omega_k^2
+ \Bk^2
+
\mu^2
} \\
&\qquad + a_\Bk^\dagger a_\Bk
\lr{
 \omega_k^2
+ \Bk^2
+
\mu^2
}
}.
\end{aligned}
\end{dmath}

With \( \omega_k^2 = \Bk^2 + \mu^2 \), the time dependent terms are killed leaving
\begin{dmath}\label{eqn:scalarFieldHamiltonian:180}
H
=
\inv{2}
\int d^3 k
\omega_k
\lr{
  a_\Bk a_\Bk^\dagger
+ a_\Bk^\dagger a_\Bk
}.
\end{dmath}
} % answer

%}
%\EndArticle
