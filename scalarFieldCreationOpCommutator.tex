%
% Copyright � 2016 Peeter Joot.  All Rights Reserved.
% Licenced as described in the file LICENSE under the root directory of this GIT repository.
%
%{
%\input{../latex/blogpost.tex}
%\renewcommand{\basename}{scalarFieldCreationOpCommutator}
%\renewcommand{\dirname}{notes/phy2403/}
%%\newcommand{\dateintitle}{}
%%\newcommand{\keywords}{}
%
%\input{../latex/peeter_prologue_print2.tex}
%
%\usepackage{peeters_layout_exercise}
%\usepackage{peeters_braket}
%\usepackage{peeters_figures}
%
%\beginArtNoToc
%
%\generatetitle{Scalar field creation operator commutator}
%\chapter{Scalar field creation operator commutator}
\label{chap:scalarFieldCreationOpCommutator}

\makeproblem{Scalar field creation operator commutator.}{problem:scalarFieldCreationOpCommutator:1}{
In \citep{LukeQFT} it is stated that the creation operators of eq. 2.78

\begin{dmath}\label{eqn:scalarFieldCreationOpCommutator:20}
\alpha_k = \inv{2} \int \frac{d^3k}{(2\pi)^3} \lr{
\phi(x,0) + \frac{i}{\omega_k} \partial_0 \phi(x,0)
}
e^{-i \Bk \cdot \Bx }
\end{dmath}

associated with field operator \( \phi \) commute.  Verify that.

} % problem

\makeanswer{problem:scalarFieldCreationOpCommutator:1}{

\begin{dmath}\label{eqn:scalarFieldCreationOpCommutator:40}
\antisymmetric{\alpha_k}{\alpha_m}
=
\inv{4}
\frac{1}{(2\pi)^6}
\int d^3 x d^3 y
e^{-i \Bk \cdot \Bx }
e^{-i \Bm \cdot \By }
\antisymmetric
{
\phi(x,0) + \frac{i}{\omega_k} \partial_0 \phi(x,0)
}
{
\phi(y,0) + \frac{i}{\omega_m} \partial_0 \phi(y,0)
}
=
\frac{i}{4}
\frac{1}{(2\pi)^6}
\int d^3 x d^3 y
e^{-i \Bk \cdot \Bx }
e^{-i \Bm \cdot \By }
\lr{
\antisymmetric{\phi(x,0)}{\inv{\omega_m} \partial_0 \phi(y,0)}
+
\antisymmetric{\inv{\omega_k} \partial_0 \phi(x,0)}{\phi(y,0)}
}
=
\frac{i}{4}
\frac{1}{(2\pi)^6}
\int d^3 x d^3 y
e^{-i \Bk \cdot \Bx }
e^{-i \Bm \cdot \By }
\lr{
\frac{i}{\omega_m} \deltathree(\Bx - \By)
-
\frac{i}{\omega_k} \deltathree(\Bx - \By)
}
=
-\frac{1}{4}
\frac{1}{(2\pi)^6}
\int d^3 x
e^{ -i (\Bk + \Bm) \cdot \Bx }
\lr{
\frac{1}{\omega_m}
-
\frac{1}{\omega_k}
}
=
-\frac{1}{4}
\frac{1}{(2\pi)^3}
\lr{
\frac{1}{\omega_m}
-
\frac{1}{\omega_k}
}
\deltathree(\Bk + \Bm)
=
-\frac{1}{4}
\frac{1}{(2\pi)^3}
\lr{
\frac{1}{\omega_{\Abs{-\Bk}}}
-
\frac{1}{\omega_{\Abs{\Bk}}}
}
\deltathree(\Bk + \Bm)
=
0.
\end{dmath}
} % answer

%}
%\EndArticle
