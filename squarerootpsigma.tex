%
% Copyright � 2018 Peeter Joot.  All Rights Reserved.
% Licenced as described in the file LICENSE under the root directory of this GIT repository.
%
%{
%\input{../latex/blogpost.tex}
%\renewcommand{\basename}{squarerootpsigma}
%%\renewcommand{\dirname}{notes/phy1520/}
%\renewcommand{\dirname}{notes/ece1228-electromagnetic-theory/}
%%\newcommand{\dateintitle}{}
%%\newcommand{\keywords}{}
%
%\input{../latex/peeter_prologue_print2.tex}
%
%\usepackage{peeters_layout_exercise}
%\usepackage{peeters_braket}
%\usepackage{peeters_figures}
%\usepackage{siunitx}
%\usepackage{verbatim}
%%\usepackage{mhchem} % \ce{}
%%\usepackage{macros_bm} % \bcM
%%\usepackage{macros_qed} % \qedmarker
%%\usepackage{txfonts} % \ointclockwise
%
%\newcommand{\osigma}[0]{\overbar{\sigma}}
%
%\beginArtNoToc
%
%\generatetitle{Dirac spinor relations after rest frame boost}
%%\chapter{Dirac spinor relations after rest frame boost}
\index{Dirac spinor!rest frame}
%%\label{chap:squarerootpsigma}
%
\makeproblem{Exponential form of \( \sqrt{p \cdot \sigma}, \sqrt{p \cdot \osigma} \).}{problem:squarerootpsigma:1}{
In \citep{tobiasQFTL15DiracSquarerootPdotSigma},
Prof Osmond explicitly boosts a \( u^s(p_0) \) Dirac spinor from the rest frame with rest frame energy \( p_0 \), and claims
\begin{equation}\label{eqn:squarerootpsigma:20}
\begin{aligned}
   \sqrt{m} e^{-\inv{2} \eta \sigma^3} &= \sqrt{ p \cdot \sigma } \\
   \sqrt{m} e^{\inv{2} \eta \sigma^3} &= \sqrt{ p \cdot \osigma },
\end{aligned}
\end{equation}
for the components of \( u^s(\Lambda p_0) \).

Validate these identities by squaring both sides.
} % problem

\makeanswer{problem:squarerootpsigma:1}{
First
\begin{equation}\label{eqn:squarerootpsigma:40}
   e^{\pm \inv{2} \eta \sigma^3 }
   =
   \cosh\lr{ \inv{2} \eta \sigma^3 }
   \pm
   \sinh\lr{ \inv{2} \eta \sigma^3 } \sigma^3,
\end{equation}
which squares to (\nbref{uvspinors.nb})
\begin{equation}\label{eqn:squarerootpsigma:60}
   \lr{ e^{\pm \inv{2} \eta \sigma^3 } }^2
=
\begin{bmatrix}
   e^{\pm \eta} & 0 \\
   0 & e^{\mp \eta}
\end{bmatrix}.
\end{equation}

Explicitly boosting the rest energy \( p_0 \) gives
\begin{equation}\label{eqn:squarerootpsigma:80}
\begin{bmatrix}
   p_0 \\
   0
\end{bmatrix}
\rightarrow
\begin{bmatrix}
   \cosh\eta & \sinh\eta \\
   \sinh\eta & \cosh\eta \\
\end{bmatrix}
\begin{bmatrix}
   p_0 \\
   0
\end{bmatrix}
=
p_0
\begin{bmatrix}
   \cosh\eta \\
   \sinh\eta
\end{bmatrix},
\end{equation}
so after the boost
\begin{equation}\label{eqn:squarerootpsigma:100}
\begin{aligned}
p \cdot \sigma
&\rightarrow
p_0 \lr{ \cosh \eta - \sinh \eta \sigma^3 } \\
&= p_0
\begin{bmatrix}
   \cosh\eta - \sinh\eta & 0 \\
   0 & \cosh\eta + \sinh\eta
\end{bmatrix} \\
&=
p_0
\begin{bmatrix}
   e^{-\eta} & 0 \\
   0 & e^{\eta}
\end{bmatrix},
\end{aligned}
\end{equation}
where \( p_0 = m \) is still the rest frame energy.  According to
\cref{eqn:squarerootpsigma:60} this is exactly
\begin{equation}\label{eqn:squarerootpsigma:120}
   \lr{\sqrt{m} e^{-\inv{2} \eta \sigma^3 }}^2.
\end{equation}

Since \( p \cdot \osigma \) flips the signs of the spatial momentum, we have shown that
\begin{equation}\label{eqn:squarerootpsigma:140}
\begin{aligned}
   \lr{\sqrt{m} e^{-\inv{2} \eta \sigma^3 }}^2 &= p \cdot \sigma \\
   \lr{\sqrt{m} e^{\inv{2} \eta \sigma^3 }}^2 &= p \cdot \osigma,
\end{aligned}
\end{equation}
which isn't a full proof of the claimed result (i.e. the most general orientation isn't considered), but at least validates the claim.
} % answer

%}
%\EndArticle
