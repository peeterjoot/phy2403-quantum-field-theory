%
% Copyright � 2018 Peeter Joot.  All Rights Reserved.
% Licenced as described in the file LICENSE under the root directory of this GIT repository.
%
%{
%%%\input{../latex/blogpost.tex}
%%%\renewcommand{\basename}{momentum}
%%%%\renewcommand{\dirname}{notes/phy1520/}
%%%\renewcommand{\dirname}{notes/ece1228-electromagnetic-theory/}
%%%%\newcommand{\dateintitle}{}
%%%%\newcommand{\keywords}{}
%%%
%%%\input{../latex/peeter_prologue_print2.tex}
%%%
%%%\usepackage{peeters_layout_exercise}
%%%\usepackage{peeters_braket}
%%%\usepackage{peeters_figures}
%%%\usepackage{siunitx}
%%%\usepackage{verbatim}
%%%\usepackage{macros_cal} % LL
%%%
%%%\newcommand{\normalorder}[1]{\text{:\({#1}\):}}
%%%
%%%\beginArtNoToc
%%%
%%%\generatetitle{Momentum of scalar field}
\label{chap:momentum}
% \citep{sakurai2014modern} pr X.Y
% \citep{pozar2009microwave}
% \citep{qftLectureNotes}
% \citep{doran2003gap}
% \citep{jackson1975cew}
% \citep{griffiths1999introduction}

\section{Expansion of the field momentum.}

In \cref{eqn:qftLecture8:500} it was claimed that
\begin{equation}\label{eqn:momentum:20}
P^k = \int d^3 x \pihat \partial^k \phihat = \int \frac{d^3 p}{(2\pi)^3} p^k a_\Bp^\dagger a_\Bp.
\end{equation}
If I compute this, I get a normal ordered variation of this operator, but also get some time dependent terms.  Here's the computation (dropping hats)
\begin{dmath}\label{eqn:momentum:40}
P^k
= \int d^3 x \pihat \partial^k \phi
= \int d^3 x \partial_0 \phi \partial^k \phi
= \int d^3 x \frac{d^3 p d^3 q}{(2 \pi)^6} \inv{\sqrt{2 \omega_p 2 \omega_q} }
\partial_0
\lr{
   a_\Bp e^{-i p \cdot x}
   +
   a_\Bp^\dagger e^{i p \cdot x}
}
\partial^k
\lr{
   a_\Bq e^{-i q \cdot x}
   +
   a_\Bq^\dagger e^{i q \cdot x}
}.
\end{dmath}
The exponential derivatives are
\begin{dmath}\label{eqn:momentum:60}
\partial_0 e^{\pm i p \cdot x}
=
\partial_0 e^{\pm i p_\mu x^\mu}
=
\pm i p_0
\partial_0 e^{\pm i p \cdot x},
\end{dmath}
and
\begin{dmath}\label{eqn:momentum:80}
\partial^k e^{\pm i p \cdot x}
=
\partial^k e^{\pm i p^\mu x_\mu}
=
\pm i p^k e^{\pm i p \cdot x},
\end{dmath}
so
\begin{dmath}\label{eqn:momentum:100}
P^k
=
-\int d^3 x \frac{d^3 p d^3 q}{(2 \pi)^6} \inv{\sqrt{2 \omega_p 2 \omega_q} }
p_0 q^k
\lr{
   -a_\Bp e^{-i p \cdot x}
   +
   a_\Bp^\dagger e^{i p \cdot x}
}
\lr{
   -a_\Bq e^{-i q \cdot x}
   +
   a_\Bq^\dagger e^{i q \cdot x}
}
=
-\inv{2} \int d^3 x \frac{d^3 p d^3 q}{(2 \pi)^6} \sqrt{\frac{\omega_p}{\omega_q}} q^k
\lr{
     a_\Bp a_\Bq e^{-i (p + q) \cdot x}
   + a_\Bp^\dagger a_\Bq^\dagger e^{i (p + q) \cdot x}
   - a_\Bp a_\Bq^\dagger e^{i (q - p) \cdot x}
   - a_\Bp^\dagger a_\Bq e^{i (p - q) \cdot x}
}
=
\inv{2} \int \frac{d^3 p d^3 q}{(2 \pi)^3} \sqrt{\frac{\omega_p}{\omega_q}} q^k
\lr{
   - a_\Bp a_\Bq e^{- i(\omega_\Bp + \omega_\Bq) t} \deltathree(\Bp + \Bq)
   - a_\Bp^\dagger a_\Bq^\dagger e^{i(\omega_\Bp + \omega_\Bq) t} \deltathree(-\Bp - \Bq)
   + a_\Bp a_\Bq^\dagger e^{i(\omega_\Bq - \omega_\Bp) t} \deltathree(\Bp - \Bq)
   + a_\Bp^\dagger a_\Bq e^{i(\omega_\Bp - \omega_\Bq) t} \deltathree(\Bq - \Bp)
}
=
\inv{2} \int \frac{d^3 p }{(2 \pi)^3} p^k
\lr{
     a_\Bp^\dagger a_\Bp
   + a_\Bp a_\Bp^\dagger
   - a_\Bp a_{-\Bp} e^{- 2 i \omega_\Bp t}
   - a_\Bp^\dagger a_{-\Bp}^\dagger e^{2 i \omega_\Bp t}
}.
\end{dmath}

What is the rationale for ignoring those time dependent terms?  Does normal ordering also implicitly drop any non-paired creation/annihilation operators?  If so, why?

\section{Conservation of the field momentum.}

This follows up on unanswered questions related to the apparent time dependent terms in the previous expansion of \( P^i \) for a scalar field.

It turns out that examining the reasons that we can say that the field momentum is conserved also sheds some light on the question.  \( P^i \) is not an a-priori conserved quantity, but we may use the charge conservation argument to justify this despite it not having a four-vector nature (i.e. with zero four divergence.)

The momentum \( P^i \) that we have defined is related to the conserved quantity \( T^{0\mu} \), the energy-momentum tensor, which satisfies \( 0 = \partial_\mu T^{0\mu} \) by Noether's theorem (this was the conserved quantity associated with a spacetime translation.)

That tensor was
\begin{dmath}\label{eqn:momentum:120}
T^{\mu\nu} = \partial^\mu \phi \partial^\nu \phi - g^{\mu\nu} \LL,
\end{dmath}
and can be used to define the momenta
\begin{dmath}\label{eqn:momentum:140}
\int d^3 x T^{0k}
= \int d^3 x \partial^0 \phi \partial^k \phi
= \int d^3 x \pi \partial^k \phi.
\end{dmath}
Charge \( Q^i = \int d^3 x j^0 \) was conserved with respect to a limiting surface argument, and we can make a similar ``beer can integral'' argument for \( P^i \), integrating over a large time interval \( t \in [-T, T] \) as sketched in \cref{fig:spacetimeCylinder:spacetimeCylinderFig1}.  That is
\begin{dmath}\label{eqn:momentum:160}
0
=
\partial_\mu \int d^4 x T^{0\mu}
=
\partial_0 \int d^4 x T^{00}
+
\partial_k \int d^4 x T^{0k}
=
\partial_0 \int_{-T}^T dt \int d^3 x T^{00}
+
\partial_k \int_{-T}^T dt \int d^3 x T^{0k}
=
\partial_0 \int_{-T}^T dt \int d^3 x T^{00}
+
\partial_k \int_{-T}^T dt
\inv{2} \int \frac{d^3 p }{(2 \pi)^3} p^k
\lr{
     a_\Bp^\dagger a_\Bp
   + a_\Bp a_\Bp^\dagger
   - a_\Bp a_{-\Bp} e^{- 2 i \omega_\Bp t}
   - a_\Bp^\dagger a_{-\Bp}^\dagger e^{2 i \omega_\Bp t}
}
=
\int d^3 x \evalrange{T^{00}}{-T}{T}
+
T \partial_k
\int \frac{d^3 p }{(2 \pi)^3} p^k
\lr{
     a_\Bp^\dagger a_\Bp
   + a_\Bp a_\Bp^\dagger
}
-\inv{2}
\partial_k \int_{-T}^T dt
\int \frac{d^3 p }{(2 \pi)^3} p^k
\lr{
     a_\Bp a_{-\Bp} e^{- 2 i \omega_\Bp t}
   + a_\Bp^\dagger a_{-\Bp}^\dagger e^{2 i \omega_\Bp t}
}.
\end{dmath}
%\imageFigure{../figures/phy2403-quantum-field-theory/spacetimeCylinderFig1}{Cylindrical spacetime boundary.}{fig:spacetimeCylinder:spacetimeCylinderFig1}{0.3}

The first integral can be said to vanish if the field energy goes to zero at the time boundaries, and the last integral reduces to
\begin{dmath}\label{eqn:momentum:180}
-\inv{2}
\partial_k \int_{-T}^T dt
\int \frac{d^3 p }{(2 \pi)^3} p^k
\lr{
     a_\Bp a_{-\Bp} e^{- 2 i \omega_\Bp t}
   + a_\Bp^\dagger a_{-\Bp}^\dagger e^{2 i \omega_\Bp t}
}
=
-\int \frac{d^3 p }{2 (2 \pi)^3} p^k
\lr{
     a_\Bp a_{-\Bp} \frac{\sin( -2 \omega_\Bp T )}{-2 \omega_\Bp}
   + a_\Bp^\dagger a_{-\Bp}^\dagger \frac{\sin( 2 \omega_\Bp T )}{2 \omega_\Bp}
}
=
-\int \frac{d^3 p }{2 (2 \pi)^3} p^k
\lr{
     a_\Bp a_{-\Bp} + a_\Bp^\dagger a_{-\Bp}^\dagger
}
\frac{\sin( 2 \omega_\Bp T )}{2 \omega_\Bp}
.
\end{dmath}
The \( \sin \) term can be interpretted as a \( \sinc \) like function of \( \omega_\Bp \) which vanishes for large \( \Bp \).  It's not entirely \( \sinc \) like for a massive field as \( \omega_\Bp = \sqrt{ \Bp^2 + m^2 } \), which never hits zero, as shown in \cref{fig:sortOfSincFunction:sortOfSincFunctionFig1}.
%\imageFigure{../figures/phy2403-quantum-field-theory/sortOfSincFunctionFig1}{\texorpdfstring{\(\sin(2 \omega_\Bp T)/\omega_\Bp\)}{(sin 2 omega T)/omega}}{fig:sortOfSincFunction:sortOfSincFunctionFig1}{0.3}
\imageFigure{../figures/phy2403-quantum-field-theory/sortOfSincFunctionFig1}{Angular frequency dependent sinc.}{fig:sortOfSincFunction:sortOfSincFunctionFig1}{0.3}
Vanishing for large \( \Bp \) doesn't help the whole integral vanish, but we can resort to the Riemann-Lebesque lemma \citep{wiki:RiemannLebesgue} instead and interpret this integral as one with a plain old high frequency oscillation that is presumed to vanish (i.e. the rest is well behaved enough that it can be labelled as \( L_1 \) integrable.)

We see that only the non-time dependent portion of \( \mathbf{P} \) matters from a conserved quantity point of view, and having killed off all the time dependent terms, we are left with a conservation relationship for the momenta \( \spacegrad \cdot \BP = 0 \), where \( \BP \) in normal order is just
\begin{dmath}\label{eqn:momentum:200}
\normalorder{\BP} = \int \frac{d^3 p}{(2 \pi)^3} \Bp a_\Bp^\dagger a_\Bp.
\end{dmath}

%}
%\EndArticle
