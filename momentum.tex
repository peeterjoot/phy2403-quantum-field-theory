%
% Copyright � 2018 Peeter Joot.  All Rights Reserved.
% Licenced as described in the file LICENSE under the root directory of this GIT repository.
%
%{
\input{../latex/blogpost.tex}
\renewcommand{\basename}{momentum}
%\renewcommand{\dirname}{notes/phy1520/}
\renewcommand{\dirname}{notes/ece1228-electromagnetic-theory/}
%\newcommand{\dateintitle}{}
%\newcommand{\keywords}{}

\input{../latex/peeter_prologue_print2.tex}

\usepackage{peeters_layout_exercise}
\usepackage{peeters_braket}
\usepackage{peeters_figures}
\usepackage{siunitx}
\usepackage{verbatim}
%\usepackage{mhchem} % \ce{}
%\usepackage{macros_bm} % \bcM
%\usepackage{macros_qed} % \qedmarker
%\usepackage{txfonts} % \ointclockwise

\beginArtNoToc

\generatetitle{Momentum of scalar field}
%\chapter{Momentum of scalar field}
%\label{chap:momentum}
% \citep{sakurai2014modern} pr X.Y
% \citep{pozar2009microwave}
% \citep{qftLectureNotes}
% \citep{doran2003gap}
% \citep{jackson1975cew}
% \citep{griffiths1999introduction}

Way back in lecture 8, it was claimed that
\begin{equation}\label{eqn:momentum:20}
P^k = \int d^3 x \pihat \partial^k \phihat = \int \frac{d^3 p}{(2\pi)^3} p^k a_\Bp^\dagger a_\Bp.
\end{equation}
If I compute this, I get a normal ordered variation of this operator, but also get some time dependent terms.  Here's the computation (dropping hats)
\begin{dmath}\label{eqn:momentum:40}
P^k
= \int d^3 x \pihat \partial^k \phi
= \int d^3 x \partial_0 \phi \partial^k \phi
= \int d^3 x \frac{d^3 p d^3 q}{(2 \pi)^6} \inv{\sqrt{2 \omega_p 2 \omega_q} }
\partial_0
\lr{
   a_\Bp e^{-i p \cdot x}
   +
   a_\Bp^\dagger e^{i p \cdot x}
}
\partial^k
\lr{
   a_\Bq e^{-i q \cdot x}
   +
   a_\Bq^\dagger e^{i q \cdot x}
}.
\end{dmath}
The exponential derivatives are
\begin{dmath}\label{eqn:momentum:60}
\partial_0 e^{\pm i p \cdot x}
=
\partial_0 e^{\pm i p_\mu x^\mu}
=
\pm i p_0
\partial_0 e^{\pm i p \cdot x},
\end{dmath}
and
\begin{dmath}\label{eqn:momentum:80}
\partial^k e^{\pm i p \cdot x}
=
\partial^k e^{\pm i p^\mu x_\mu}
=
\pm i p^k e^{\pm i p \cdot x},
\end{dmath}
so
\begin{dmath}\label{eqn:momentum:100}
P^k
=
-\int d^3 x \frac{d^3 p d^3 q}{(2 \pi)^6} \inv{\sqrt{2 \omega_p 2 \omega_q} }
p_0 q^k
\lr{
   -a_\Bp e^{-i p \cdot x}
   +
   a_\Bp^\dagger e^{i p \cdot x}
}
\lr{
   -a_\Bq e^{-i q \cdot x}
   +
   a_\Bq^\dagger e^{i q \cdot x}
}
=
-\inv{2} \int d^3 x \frac{d^3 p d^3 q}{(2 \pi)^6} \sqrt{\frac{\omega_p}{\omega_q}} q^k
\lr{
     a_\Bp a_\Bq e^{-i (p + q) \cdot x}
   + a_\Bp^\dagger a_\Bq^\dagger e^{i (p + q) \cdot x}
   - a_\Bp a_\Bq^\dagger e^{i (q - p) \cdot x}
   - a_\Bp^\dagger a_\Bq e^{i (p - q) \cdot x}
}
=
\inv{2} \int \frac{d^3 p d^3 q}{(2 \pi)^3} \sqrt{\frac{\omega_p}{\omega_q}} q^k
\lr{
   - a_\Bp a_\Bq e^{- i(\omega_\Bp + \omega_\Bq) t} \delta^3(\Bp + \Bq)
   - a_\Bp^\dagger a_\Bq^\dagger e^{i(\omega_\Bp + \omega_\Bq) t} \delta^3(-\Bp - \Bq)
   + a_\Bp a_\Bq^\dagger e^{i(\omega_\Bq - \omega_\Bp) t} \delta^3(\Bp - \Bq)
   + a_\Bp^\dagger a_\Bq e^{i(\omega_\Bp - \omega_\Bq) t} \delta^3(\Bq - \Bp)
}
=
\inv{2} \int \frac{d^3 p }{(2 \pi)^3} p^k
\lr{
     a_\Bp^\dagger a_\Bp
   + a_\Bp a_\Bp^\dagger
   - a_\Bp a_{-\Bp} e^{- 2 i \omega_\Bp t}
   - a_\Bp^\dagger a_{-\Bp}^\dagger e^{2 i \omega_\Bp t}
}.
\end{dmath}

What is the rationale for ignoring those time dependent terms?  Does normal ordering also implicitly drop any non-paired creation/annihilation operators?  If so, why?

%}
%\EndArticle
\EndNoBibArticle
