%
% Copyright � 2015 Peeter Joot.  All Rights Reserved.
% Licenced as described in the file LICENSE under the root directory of this GIT repository.
%
\makeproblem{Lorentz transformation.}{qft:problemSet1:1}{

A Lorentz transformation \( x^\mu \rightarrow {x'}^\mu = {\wedge^\mu}_\nu x^\nu \) is such that it preserves the Minkowski metric \( \eta_{\mu\nu} \) meaning that \( \eta_{\mu\nu} x^\mu x^\nu = \eta_{\mu\nu} {x'}^\mu {x'}^\nu \) for all \( x \).

\makesubproblem{}{qft:problemSet1:1a}
Show that this implies that
\begin{dmath}\label{eqn:qftproblemSet1Problem1:20}
\eta_{\mu\nu}
=
\eta_{\sigma\tau} {\wedge^\sigma}_\mu {\wedge^\tau}_\nu.
\end{dmath}
\makesubproblem{}{qft:problemSet1:1b}
Use this result to show that an infinitesimal transformation of the form
\begin{dmath}\label{eqn:qftproblemSet1Problem1:40}
{\wedge^\mu}_\nu = {\delta^\mu}_\nu + {\omega^\mu}_\nu
\end{dmath}
is a Lorentz transformation when \( \omega^{\mu\nu} \) is antisymmetric i.e. \( \omega^{\mu\nu} = -\omega^{\nu\mu} \). (Note that there an antisymmetric \( 4 \times 4 \) matrix has six parameters, as does a Lorentz transformation - 3 rotations and 3 boosts - so the counting works out).
\makesubproblem{}{qft:problemSet1:1c}
Write down the matrix form for \( {\omega^\mu}_\nu \) that corresponds to a rotation through an infinitesimal angle \( \theta \) about the \( x^3\)-axis.
\makesubproblem{}{qft:problemSet1:1d}
Do the same for a boost along the \( x^1 \)-axis by an infinitesimal velocity \(v\).
} % makeproblem

\makeanswer{qft:problemSet1:1}{
\makeSubAnswer{}{qft:problemSet1:1a}

%{x'}^\mu &= \lr{ {\delta^\mu}_\alpha + {\omega^\mu}_\alpha } x^\alpha \\
%{x'}_\mu &= \lr{ {\delta_\mu}^\beta + {\omega_\mu}^\beta } x_\beta
%{x'}^\mu &= {\wedge^\mu}_\alpha x^\alpha \\
%{x'}_\mu &= {\wedge_\mu}^\beta x_\beta,
%
The dot product of the transformed coordinates is

\begin{dmath}\label{eqn:qftProblemSet1Problem1:60}
\eta_{\mu\nu} {x'}^\mu {x'}^\nu
=
\eta_{\mu\nu}
{\wedge^\mu}_\alpha x^\alpha
{\wedge^\nu}_\beta x^\beta
=
\eta_{\sigma\tau}
{\wedge^\sigma}_\mu
{\wedge^\tau}_\nu
x^\mu
x^\nu,
\end{dmath}

where the last step is just a change of indexes \( \mu \rightarrow \sigma, \nu \rightarrow \tau, \alpha \rightarrow \mu, \beta \rightarrow \nu \).  The identity \cref{eqn:qftproblemSet1Problem1:20} can be read off directly.

\makeSubAnswer{}{qft:problemSet1:1b}

\begin{dmath}\label{eqn:qftProblemSet1Problem1:80}
\eta_{\sigma\tau} {\wedge^\sigma}_\mu {\wedge^\tau}_\nu
=
\eta_{\sigma\tau}
\lr{ {\delta^\sigma}_\mu + {\omega^\sigma}_\mu }
\lr{ {\delta^\tau}_\nu + {\omega^\tau}_\nu }
=
\lr{ \eta_{\mu\tau} + \omega_{\tau\mu} }
\lr{ {\delta^\tau}_\nu + {\omega^\tau}_\nu }
=
  \eta_{\mu\tau} {\delta^\tau}_\nu
+ \eta_{\mu\tau} {\omega^\tau}_\nu
+ \omega_{\tau\mu} {\delta^\tau}_\nu
+ \omega_{\tau\mu} {\omega^\tau}_\nu
=
  \eta_{\mu\nu}
+ \omega_{\mu\nu}
+ \omega_{\nu\mu}
+ \omega_{\tau\mu} {\omega^\tau}_\nu
=
  \eta_{\mu\nu}
+ \omega_{\mu\nu}
- \omega_{\mu\nu}
+ O(\omega^2)
=
\eta_{\mu\nu}.
\end{dmath}

\makeSubAnswer{}{qft:problemSet1:1c}

With a \( \gamma_0^2 = 1, \gamma_k^2 = -1 \) metric, a rotation in the x-y plane around the z-axis can be written as

\begin{dmath}\label{eqn:qftProblemSet1Problem1:100}
\gamma_1 x^1 + \gamma_2 x^2
\rightarrow
\lr{ \gamma_1 x^1 + \gamma_2 x^2 } e^{\gamma_2 \gamma_1 \theta }
=
\lr{ \gamma_1 x^1 + \gamma_2 x^2 } \lr{ \cos \theta + \gamma_2 \gamma_1 \sin\theta }
=
\gamma_1 x^1 \cos\theta + \gamma_2 x^2 \cos\theta
+
\gamma_2 x^1 \sin\theta - \gamma_1 x^2 \sin\theta,
\end{dmath}

or
\begin{equation}\label{eqn:qftProblemSet1Problem1:120}
{
\begin{bmatrix}
x^1 \\
x^2 \\
\end{bmatrix}
}'
=
\begin{bmatrix}
\cos\theta & -\sin\theta \\
\sin\theta & \cos\theta
\end{bmatrix}
\begin{bmatrix}
x^1 \\
x^2 \\
\end{bmatrix},
\end{equation}

so in the small angle approximation, with a \( \gamma_0, \gamma_, \gamma_2, \gamma_3 \) basis, we have

\begin{dmath}\label{eqn:qftProblemSet1Problem1:140}
{\omega^\nu}_\mu
=
\begin{bmatrix}
0 & 0 & 0 & 0 \\
0 & 0 & -\theta & 0 \\
0 & \theta & 0 & 0 \\
0 & 0 & 0 & 0 \\
\end{bmatrix}.
\end{dmath}

\makeSubAnswer{}{qft:problemSet1:1d}

For the boost the rotation is also an exponential

\begin{dmath}\label{eqn:qftProblemSet1Problem1:160}
\gamma_1 x^1 + \gamma_0 x^0
\rightarrow
\lr{ \gamma_1 x^1 + \gamma_0 x^0 } e^{\gamma_0 \gamma_1 \alpha }
=
\lr{ \gamma_1 x^1 + \gamma_0 x^0 } \lr{ \cosh \alpha + \gamma_0 \gamma_1 \sinh\alpha }
=
\gamma_1 x^1 \cosh \alpha + \gamma_0 x^0 \cosh \alpha
+ \gamma_0 x^1 \sinh\alpha + \gamma_1 x^0 \sinh\alpha,
\end{dmath}

or
\begin{dmath}\label{eqn:qftProblemSet1Problem1:180}
{\begin{bmatrix}
x^0 \\
x^1 \\
\end{bmatrix} }'
=
\begin{bmatrix}
\cosh\alpha & \sinh\alpha \\
\sinh\alpha & \cosh\alpha \\
\end{bmatrix}
\begin{bmatrix}
x^0 \\
x^1 \\
\end{bmatrix}
\end{dmath}

The rapidity angle \( \alpha \) can be related to velocity by considering a spacetime difference in position

\begin{dmath}\label{eqn:qftProblemSet1Problem1:200}
\Delta
{\begin{bmatrix}
x^0 \\
x^1 \\
\end{bmatrix} }'
=
\begin{bmatrix}
\cosh\alpha \Delta x^0 + \sinh\alpha \Delta x^1 \\
\sinh\alpha \Delta x^0 + \cosh\alpha \Delta x^1 \\
\end{bmatrix},
\end{dmath}

For a particle fixed at the origin in the unprimed frame (i.e. \( \Delta x^1 = 0\, \forall t \)), we have
\begin{dmath}\label{eqn:qftProblemSet1Problem1:220}
\Delta
{\begin{bmatrix}
x^0 \\
x^1 \\
\end{bmatrix} }'
=
\begin{bmatrix}
\cosh\alpha \Delta x^0 \\
\sinh\alpha \Delta x^0
\end{bmatrix}.
\end{dmath}

In particular
\begin{dmath}\label{eqn:qftProblemSet1Problem1:240}
\frac{\Delta {x'}^1}{
\Delta {x'}^0}
=
\tanh \alpha.
\end{dmath}

If the unprimed frame is moving at velocity \( v \) along the x-axis, then the primed frame is moving at \( -v \), or

\begin{dmath}\label{eqn:qftProblemSet1Problem1:260}
-v = \tanh \alpha.
\end{dmath}

Noting that \( \cosh^2 \alpha - \sinh^2 \alpha = 1 \),

\begin{dmath}\label{eqn:qftProblemSet1Problem1:280}
v^2 = \frac{ \sinh^2 \alpha }{ 1 + \sinh^2 \alpha },
\end{dmath}

so
\begin{dmath}\label{eqn:qftProblemSet1Problem1:300}
\sinh^2 \alpha \lr{ v^2 - 1 } = -v^2,
\end{dmath}

or
\begin{dmath}\label{eqn:qftProblemSet1Problem1:320}
\sinh\alpha = \pm \frac{v}{\sqrt{1 - v^2}}.
\end{dmath}

We also have

\begin{dmath}\label{eqn:qftProblemSet1Problem1:340}
\cosh^2\alpha = \frac{v^2}{1 - v^2} + 1 = \frac{1}{1 - v^2}.
\end{dmath}

Picking the negative sign in \cref{eqn:qftProblemSet1Problem1:320} to match \cref{eqn:qftProblemSet1Problem1:260}, we have

\begin{dmath}\label{eqn:qftProblemSet1Problem1:181}
{\begin{bmatrix}
x^0 \\
x^1 \\
\end{bmatrix} }'
=
\inv{\sqrt{1 - v^2}}
\begin{bmatrix}
1           & -v \\
- v         & 1
\end{bmatrix}
\begin{bmatrix}
x^0 \\
x^1 \\
\end{bmatrix}.
\end{dmath}

In the small velocity limit, this gives
\begin{dmath}\label{eqn:qftProblemSet1Problem1:360}
{\omega^\mu}_\nu =
\begin{bmatrix}
0 & -v & 0 & 0 \\
-v & 0 & 0 & 0 \\
0 & 0 & 0 & 0 \\
0 & 0 & 0 & 0 \\
\end{bmatrix}.
\end{dmath}

}
