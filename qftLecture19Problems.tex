%
% Copyright © 2018 Peeter Joot.  All Rights Reserved.
% Licenced as described in the file LICENSE under the root directory of this GIT repository.
%
%{
\section{Problems:}

\makeproblem{Show that \( \overbar{\Psi} \Psi \) is a Lorentz scalar.}{problem:qftLecture19Problems:1}{
} % problem

\makeanswer{problem:qftLecture19Problems:1}{
The Lorentz property follows from \cref{eqn:qftLecture19:1614}
\begin{dmath}\label{eqn:qftLecture19Problems:1654}
\overbar{\Psi} \Psi
\rightarrow
\lr{ \overbar{\Psi} \Lambda^{-1}_{1/2}} \lr{ \Lambda_{1/2} \Psi }
=
\overbar{\Psi} \Psi.
\end{dmath}
The scalar nature of this product can be seen easily by expansion.
\begin{dmath}\label{eqn:qftLecture19Problems:1674}
\overbar{\Psi} \Psi
=
\Psi^\dagger
\gamma^0
\Psi
=
\begin{bmatrix}
\Psi_1^\conj &
\Psi_2^\conj &
\Psi_3^\conj &
\Psi_4^\conj
\end{bmatrix}
\begin{bmatrix}
0 & 0 & 1 & 0 \\
0 & 0 & 0 & 1 \\
1 & 0 & 0 & 0 \\
0 & 1 & 0 & 0 \\
\end{bmatrix}
\begin{bmatrix}
\Psi_1 \\
\Psi_2 \\
\Psi_3 \\
\Psi_4 \\
\end{bmatrix}
=
\begin{bmatrix}
\Psi_1^\conj &
\Psi_2^\conj &
\Psi_3^\conj &
\Psi_4^\conj
\end{bmatrix}
\begin{bmatrix}
\Psi_3 \\
\Psi_4 \\
\Psi_1 \\
\Psi_2 \\
\end{bmatrix}
=
\Psi_1^\conj
\Psi_3
+
\Psi_2^\conj
\Psi_4
+
\Psi_3^\conj
\Psi_1
+
\Psi_4^\conj
\Psi_2
=
2 \Real \lr{
\Psi_1^\conj
\Psi_3
+
\Psi_2^\conj
\Psi_4
}.
\end{dmath}
Clearly any individual \( \Psi^\dagger \gamma^\mu \Psi \) product will also be a scalar.
} % answer

\makeproblem{Show that \( \overbar{\Psi} \gamma^\mu \Psi \) transforms as a four vector.}{problem:qftLecture19Problems:2}{
} % problem

\makeanswer{problem:qftLecture19Problems:2}{
\begin{dmath}\label{eqn:qftLecture19Problems:1754}
\overbar{\Psi} \gamma^\mu \Psi
\rightarrow
\lr{
   \overbar{\Psi} \Lambda^{-1}_{1/2}
}
\gamma^\mu
\lr{
   \Lambda_{1/2}
   \Psi
}
=
\overbar{\Psi}
\lr{
   \Lambda^{-1}_{1/2}
   \gamma^\mu
   \Lambda_{1/2}
}
   \Psi
=
\overbar{\Psi}
\lr{
\ultensor{\Lambda}{\mu}{\nu}
   \gamma^\nu
}
   \Psi
=
\ultensor{\Lambda}{\mu}{\nu}
\overbar{\Psi}
   \gamma^\nu
   \Psi.
\end{dmath}
} % answer

%}
