%
% Copyright � 2017 Peeter Joot.  All Rights Reserved.
% Licenced as described in the file LICENSE under the root directory of this GIT repository.
%
%{
\input{../latex/blogpost.tex}
\renewcommand{\basename}{qftLecture15}
\renewcommand{\dirname}{notes/phy2403/}
\newcommand{\keywords}{PHY2403H}
\input{../latex/peeter_prologue_print2.tex}

%\usepackage{phy2403}
\usepackage{peeters_braket}
%\usepackage{peeters_layout_exercise}
\usepackage{peeters_figures}
\usepackage{mathtools}
\usepackage{siunitx}
\usepackage{macros_cal} % LL

\newcommand{\ultensor}[3]{{{#1}^{#2}}_{#3}}

\beginArtNoToc
\generatetitle{PHY2403H Quantum Field Theory.  Lecture 15: XXX.  Taught by Prof.\ Erich Poppitz}
%\chapter{XXX}
\label{chap:qftLecture15}

\paragraph{Disclaimer}

%%Peeter's lecture notes from class.  These may be incoherent and rough.
%%
%%These are notes for the UofT course PHY2403H, Quantum Field Theory, taught by Prof. Erich Poppitz, covering \textchapref{{1}} \citep{peskin1995introduction} content.

\paragraph{DISCLAIMER: Very rough notes from class, with some additional side notes.}

These are notes for the UofT course PHY2403H, Quantum Field Theory, taught by Prof. Erich Poppitz, fall 2018.
%, covering \textchapref{{1}} \citep{peskin1995introduction} content.

\section{Review}

We developed the interaction picture repreresentation, which is really the Heisenberg picture with respect to \( H_0 \).

Recall that we found
\begin{equation}\label{eqn:qftLecture15:20}
U(t, t') = e^{i H_0(t - t_0)} e^{-i H(t - t')} e^{-i H_0(t' - t_0)},
\end{equation}
with solution
\begin{dmath}\label{eqn:qftLecture15:200}
U(t, t')
=
T \exp{\lr{ -i \int_{t'}^t H_{\text{I,int}}(t'') dt''}},
\end{dmath}
\begin{dmath}\label{eqn:qftLecture15:220}
U(t, t')^\dagger
=
T \exp{\lr{ i \int_{t'}^{t} H_{\text{I,int}}(t'') dt''}}
=
T \exp{\lr{ -i \int_{t}^{t'} H_{\text{I,int}}(t'') dt''}}
= U(t', t),
\end{dmath}
and can use this to calculate the time evolution of a field
\begin{dmath}\label{eqn:qftLecture15:40}
\phi(\Bx, t)
=
U^\dagger(t, t_0)
\phi_I(\Bx, t)
U(t, t_0)
\end{dmath}
and found the ground state ket for \( H \) was
\begin{dmath}\label{eqn:qftLecture15:60}
\ket{\Omega}
=
\evalbar{
\frac{ U(t_0, -T) \ket{0} }
{
e^{-i E_0(T - t_0)} \braket{\Omega}{0}
}
}{T \rightarrow \infty(1 - i \epsilon)}.
\end{dmath}
\paragraph{Question:} What's the point of this, since it is self referential?
\paragraph{Answer:} We will see, and also see that it goes away.  Alternatively, you can write it as
\begin{equation*}
\ket{\Omega} \braket{\Omega}{0}
=
\evalbar{
\frac{ U(t_0, -T) \ket{0} }
{
e^{-i E_0(T - t_0)}
}
}{T \rightarrow \infty(1 - i \epsilon)}.
\end{equation*}

We can also show that
\begin{dmath}\label{eqn:qftLecture15:80}
\bra{\Omega}
=
\evalbar{
\frac{ \bra{0} U(T, t_0) }
{
e^{-i E_0(T - t_0)} \braket{0}{\Omega}
}
}{T \rightarrow \infty(1 - i \epsilon)}.
\end{dmath}

Our goal is still toe calculate
\begin{dmath}\label{eqn:qftLecture15:100}
\bra{\Omega} T \phi(x) \phi(y) \ket{\Omega}.
\end{dmath}
Claim: the ``LSZ'' theorem (a neat way of writing this) relates this to S matrix elements.

Assuming \( x^0 > y^0 \)

\begin{dmath}\label{eqn:qftLecture15:120}
\bra{\Omega} \phi(x) \phi(y) \ket{\Omega}
=
\frac{
\bra{0}
U(T, t_0)
U^\dagger(x^0, t^0)
\phi_I(x)
U(x^0, t^0)
%
U^\dagger(y^0, t^0)
\phi_I(y)
U(y^0, t^0)
U(t_0, -T)
\ket{0}
}
{
e^{-i 2 E_0 T} \Abs{\braket{0}{\Omega}}^2
}
\end{dmath}

Normalize \( \braket{\Omega}{\Omega} = 1 \), gives

\begin{dmath}\label{eqn:qftLecture15:140}
1
=
\frac{\bra{0} U(T, t_0) U(t_0, -T) \ket{0}}
{
e^{-i 2 E_0 T} \Abs{\braket{0}{\Omega}}^2
}
=
\frac{\bra{0} U(T, -T) \ket{0}}
{
e^{-i 2 E_0 T} \Abs{\braket{0}{\Omega}}^2
},
\end{dmath}
so that
\begin{dmath}\label{eqn:qftLecture15:240}
\bra{\Omega} \phi(x) \phi(y) \ket{\Omega}
=
\frac{
\bra{0}
U(T, t_0)
U^\dagger(x^0, t^0)
\phi_I(x)
U(x^0, t^0)
%
U^\dagger(y^0, t^0)
\phi_I(y)
U(y^0, t^0)
U(t_0, -T)
\ket{0}
}
{
   \bra{0} U(T, -T) \ket{0}
}
\end{dmath}

For \( t_1 > t_2 > t_3 \)
\begin{dmath}\label{eqn:qftLecture15:280}
U(t_1, t_2) U(t_2, t_3)
=
T e^{-i \int_{t_2}^{t_1} H_I}
T e^{-i \int_{t_3}^{t_2} H_I}
=
T \lr{
e^{-i \int_{t_2}^{t_1} H_I}
e^{-i \int_{t_3}^{t_2} H_I}
}
=
T(
e^{-i \int_{t_3}^{t_1} H_I}
),
\end{dmath}
with an end result of
\begin{dmath}\label{eqn:qftLecture15:320}
U(t_1, t_2) U(t_2, t_3) = U(t_1, t_3).
\end{dmath}
(DIY: work through the details -- this is a problem in \citep{peskin1995introduction})

This gives
\begin{dmath}\label{eqn:qftLecture15:300}
\bra{\Omega} \phi(x) \phi(y) \ket{\Omega}
=
\frac{
\bra{0}
U(T, x^0)
\phi_I(x)
U(x^0, y^0)
\phi_I(y)
U(y^0, -T)
\ket{0}
}
{
   \bra{0} U(T, -T) \ket{0}
}.
\end{dmath}

If \( y^0 > x^0 \) we have the same result, but the \( y \)'s will come first.

\paragraph{Claim:}
\begin{dmath}\label{eqn:qftLecture15:340}
\bra{\Omega} \phi(x) \phi(y) \ket{\Omega}
=
\frac{
   \bra{0}
   T\lr{
      \phi_I(x)
      \phi_I(y)
      e^{-i \int_{-T}^T H_{\text{I,int}}(t') dt'}
   }
   \ket{0}
}
{
   \bra{0}
      T ( e^{-i \int_{-T}^T H_{\text{I,int}}(t') dt'} )
   \ket{0}
}.
\end{dmath}

More generally
\boxedEquation{eqn:qftLecture15:360}{
\bra{\Omega}
\phi_I(x_1) \cdots
\phi_I(x_n)
\ket{\Omega}
=
\frac{
   \bra{0}
   T\lr{
\phi_I(x_1) \cdots
\phi_I(x_n)
      e^{-i \int_{-T}^T H_{\text{I,int}}(t') dt'}
   }
   \ket{0}
}
{
   \bra{0}
      T ( e^{-i \int_{-T}^T H_{\text{I,int}}(t') dt'} )
   \ket{0}
}.
}
This is the holy grail of pertubation theory.

In QFT II you will see this written in a path integral representation
\begin{dmath}\label{eqn:qftLecture15:380}
\bra{\Omega}
\phi_I(x_1) \cdots
\phi_I(x_n)
\ket{\Omega}
=
\frac
{
   \int [\calD \phi] \phi(x_1) \phi(x_2) \cdots \phi(x_n) e^{-i S[\phi]}
}
{
   \int [\calD \phi] e^{-i S[\phi]}
}.
\end{dmath}

\section{Unpacking it.}

\begin{dmath}\label{eqn:qftLecture15:400}
\int_{-T}^T H_{\text{I,int}}(t)
=
\int_{-T}^T
\int d^3 \Bx \frac{\lambda}{4} \lr{ \phi_I(\Bx, t) }^4
=
\int d^4 x
\frac{\lambda}{4} \lr{ \phi_I }^4
\end{dmath}

so we have
\begin{dmath}\label{eqn:qftLecture15:420}
\frac{
   \bra{0}
   T\lr{
\phi_I(x_1) \cdots
\phi_I(x_n)
      e^{-i \frac{\lambda}{4} \int d^4 x \phi_I^4(x) }
   }
   \ket{0}
}
{
   \bra{0}
      T
      e^{-i \frac{\lambda}{4} \int d^4 x \phi_I^4(x) }
   \ket{0}
}.
\end{dmath}

The numerator expands as
\begin{dmath}\label{eqn:qftLecture15:440}
   \bra{0} T\lr{ \phi_I(x_1) \cdots \phi_I(x_n) } \ket{0}
-i \frac{\lambda}{4} \int d^4 x
   \bra{0} T\lr{ \phi_I(x_1) \cdots \phi_I(x_n) \phi_I^4(x) }
+
\inv{2}
\lr{-i \frac{\lambda}{4}}^2 \int d^4 x d^4 y
   \bra{0} T\lr{ \phi_I(x_1) \cdots \phi_I(x_n)
      \phi_I^4(x)
      \phi_I^4(y)
} \ket{0}
+ \cdots
\end{dmath}
so we see that the problem ends up being the calculation of time ordered products.

%XX
\section{Problems}
\makeproblem{\( U(T, t_0) U(t_0, -T) \) }{problem:qftLecture15:1}{
Show that
\begin{equation*}
U(T, t_0) U(t_0, -T)
=
U(T, -T)
\end{equation*}
} % problem

\makeanswer{problem:qftLecture15:1}{
We can see that from
\begin{dmath}\label{eqn:qftLecture15:160}
\begin{aligned}
U(T, t_0) &= e^{i H_0(T - t_0)} e^{-i H(T - t_0)} \cancel{e^{-i H_0(t_0 - t_0)}}  \\
U(t_0, -T) &= \cancel{e^{i H_0(t_0 - t_0)}} e^{-i H(t_0 - -T)} e^{-i H_0(-T - t_0)},
\end{aligned}
\end{dmath}
so

\begin{dmath}\label{eqn:qftLecture15:180}
U(T, t_0)
U(t_0, -T)
=
e^{i H_0(T - t_0)} e^{-i H(T - t_0)} e^{-i H(t_0 + T)} e^{-i H_0(-T - t_0)}
=
e^{i H_0(T - t_0)} e^{-i H 2 T } e^{-i H_0(-T - t_0)},
\end{dmath}
whereas
\begin{dmath}\label{eqn:qftLecture15:260}
U(T, -T)
= e^{i H_0(T - t_0)} e^{-i H(T - -T)} e^{-i H_0(-T - t_0)}
= e^{i H_0(T - t_0)} e^{-i H 2 T} e^{-i H_0(-T - t_0)}
\end{dmath}
} % answer

%}
\EndArticle
%\EndNoBibArticle
