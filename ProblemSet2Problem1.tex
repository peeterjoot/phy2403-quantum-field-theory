%
% Copyright � 2018 Peeter Joot.  All Rights Reserved.
% Licenced as described in the file LICENSE under the root directory of this GIT repository.
%
\makeoproblem{Green's functions, spacelike domain.}{qft:problemSet2:1}{2018 Hw2.I}{
\index{Green's function!spacelike separation}
\index{principle value integration}
\index{half delta function}
\index{light cone}
\index{Compton wavelength}
Here, you'll study some properties of
\begin{equation}\label{eqn:ProblemSet2Problem1:20}
D(x) \equiv \antisymmetric{\phihat_{-}(x)}{\phihat_{+}(x)} = \int \frac{d^3 p}{(2\pi)^3 2 \omega_p} e^{-i \omega_p t + i \Bp \cdot \Bx}.
\end{equation}
\makesubproblem{}{qft:problemSet2:1a}
For m = 0 (``photon''), show that:
\begin{dmath}\label{eqn:ProblemSet2Problem1:40}
D(x) = -\inv{2 \pi^2} \calP \inv{t^2 - r^2} - \frac{i}{8 \pi} \lr{
\frac{\delta(t - r)}{r}
-\frac{\delta(t + r)}{r}
},
\end{dmath}
where \( r = \Norm{\Bx} \). Notice that \( D(x) \) is singular on the light cone \( t = r\). Does it vanish for spacelike separations?

Hint: Please recall that (and why!)
\begin{dmath}\label{eqn:ProblemSet2Problem1:60}
\inv{a \pm i \epsilon} = \calP \inv{a} \mp i \pi \delta(a)
\end{dmath}
(here \( \calP \) denotes ``principal value integration'',
as this relation is to be understood in terms of generalized functions, i.e. in the back of your mind it always needs to be integrated over a with suitable smooth and integrable ``test functions''). Note
also that what looks like a ``half-delta-function integral'' \( \int_0^\infty dy e^{i x y} \)
should really be understood as
\(
\lim_{\epsilon\rightarrow 0} \int_0^\infty dy e^{-\epsilon y + i x y}
\)
\makesubproblem{}{qft:problemSet2:1b}
For \( m^2 > 0 \), study the behaviour of \( D(x) \) for spacelike \( x \) and find the asymptotic behaviour for
\( -x^2 \gg 1/m^2 \) (i.e., at spacelike separations larger than the particle's Compton wavelength).
} % makeproblem

\makeanswer{qft:problemSet2:1}{
\withproblemsetsParagraph{
\makeSubAnswer{}{qft:problemSet2:1a}
Let's evaluate the integral in spherical polar coordinates
\begin{dmath}\label{eqn:ProblemSet2Problem1:80}
\begin{aligned}
\Bp &= p (\sin\theta \cos\phi, \sin\theta \sin\phi, \cos\theta) \\
\Bx &= r(0, 0, 1) \\
d^3 p &= p^2 dp \sin\theta d\theta d\phi \\
\omega_\Bp &= \sqrt{\Bp^2 + \cancel{m^2}} = \Norm{\Bp} = p.
\end{aligned},
\end{dmath}
which gives
\begin{dmath}\label{eqn:ProblemSet2Problem1:100}
D(x)
=
\inv{(2\pi)^2} \int_{p = 0}^\infty dp\, p^2 \int_{\theta = 0}^\pi -d(\cos\theta) \inv{2 p} e^{-i p t + i p r \cos\theta}
=
-\inv{8 \pi^2} \int_{p = 0}^\infty dp\, p
e^{-i p t}
\evalrange{\frac{e^{i p r u}}{i p r}}{u = \cos\theta = 1}{-1}
=
\frac{i}{8 \pi^2 r} \int_{p = 0}^\infty dp \,
e^{-i p t}
\lr{
   e^{- i p r}
   -e^{i p r}
}
=
\frac{i}{8 \pi^2 r} \int_{p = 0}^\infty dp \,
\lr{
   e^{- i p (r+t)}
   -e^{i p (r-t)}
}
\end{dmath}
As hinted, this half delta function should be interpreted offset slightly
\begin{dmath}\label{eqn:ProblemSet2Problem1:120}
D(x)
=
\frac{i}{8 \pi^2 r} \int_{p = 0}^\infty dp \,
\lr{
   e^{- i p (r+t) + p \epsilon}
   -e^{i p (r-t) - p \epsilon}
}
=
\frac{i}{8 \pi^2 r}
\evalrange{
   \lr{
      \frac{e^{- i p (r+t) - p \epsilon}}{-i(r+t) - \epsilon}
      - \frac{e^{i p (r-t) - p \epsilon}}{i(r-t) - \epsilon}
   }
}
{0}{\infty}
=
\frac{i}{8 \pi^2 r}
   \lr{
        \frac{1}{ i(r+t) + \epsilon }
      + \frac{1}{ i(r-t) - \epsilon }
   }
=
\frac{1}{8 \pi^2 r}
   \lr{
        \frac{1}{ r + t - \epsilon }
      + \frac{1}{ r - t + \epsilon }
   }
\end{dmath}
Employing the hint
\cref{eqn:ProblemSet2Problem1:60}\footnote{A nice explanation of this second hint can be found in \citep{wiki:Sokhotski} under ``Proof of the real version''.},
\cref{eqn:ProblemSet2Problem1:120}
can be cast into delta function form
\begin{dmath}\label{eqn:ProblemSet2Problem1:140}
D(x)
=
\frac{1}{8 \pi^2 r}
   \lr{
        \calP \frac{1}{ r+t } + i \pi \delta( r + t )
      +
        \calP \frac{1}{ r-t } - i \pi \delta( r - t )
   }
=
\frac{1}{8 \pi^2 r}
   \lr{
        \calP \frac{2 r }{ r^2 - t^2 } + i \pi \lr{ \delta( r + t ) - \delta( r - t ) }
   },
\end{dmath}
which, after cosmetic rearrangement, is \cref{eqn:ProblemSet2Problem1:40}.

\makeSubAnswer{}{qft:problemSet2:1b}
Let's evaluate \( D(x) \) function at spacelike point \( x = (0, r\zcap) \), and switch to polar momentum space coordinates
\begin{dmath}\label{eqn:ProblemSet2Problem1:99}
\Bp = p (\sin\theta \cos\phi, \sin\theta \sin\phi, \cos\theta).
\end{dmath}
This gives
\begin{dmath}\label{eqn:ProblemSet2Problem1:160}
D(0, r \zcap)
=
\inv{(2\pi)^3}
\int_0^\infty dp \int_0^{2\pi} d\phi \int_0^\pi d\theta p^2 \sin\theta \frac{e^{i p r \cos\theta}}{2 \sqrt{p^2 + m^2}}
=
\inv{2 (2\pi)^2}
\int_0^\infty dp \frac{p^2}{\sqrt{p^2 + m^2}} \int_{-1}^{1} du e^{i p r u}
=
\inv{2 (2\pi)^2}
\int_0^\infty dp \frac{p^2}{\sqrt{p^2 + m^2}} \frac{e^{i p r} - e^{-i p r}}{i p r}
=
\frac{-i}{2 (2\pi)^2 r}
\int_0^\infty dp \frac{p}{\sqrt{p^2 + m^2}} e^{i p r}
-
\frac{-i}{2 (2\pi)^2 r}
\int_0^{-\infty} (-dp') \frac{(-p')}{\sqrt{(-p')^2 + m^2}} e^{i p' r}
=
\frac{-i}{2 (2\pi)^2 r}
\int_{-\infty}^\infty dp \frac{p}{\sqrt{p^2 + m^2}} e^{i p r},
\end{dmath}
where a \( u = \cos\theta \) substitution was made, followed by \( p = -p' \) in the second integral.

This integral can be evaluated with the half dogbone contour sketched in \cref{fig:halfDogBoneContour:halfDogBoneContourFig1}.  The exponential \( e^{i p r} \) vanishes on the infinite radial contours \( D, E \) since the real part of \( i p r \) is negative in the upper half plane.  We are left with
\begin{dmath}\label{eqn:ProblemSet2Problem1:180}
\int_A = -\int_B - \int_C.
\end{dmath}
\imageFigure{../figures/phy2403-quantum-field-theory/halfDogBoneContourFig1}{Contour for branch at \( p = i m \).}{fig:halfDogBoneContour:halfDogBoneContourFig1}{0.3}

That doesn't look like a particularly helpful transformation at first, but because we are integrating around the branch cut running from \( [i m, i\infty] \), the square root differs by an \( 2 \pi i \) argument on each side of the cut.  Along contour \( B \) that square root is
\begin{dmath}\label{eqn:ProblemSet2Problem1:200}
(p^2 + m^2)^{1/2}
=
\lr{ e^{i \pi} \lr{ -p^2 - m^2} }^{1/2}
=
e^{i \pi/2} \sqrt{ -p^2 - m^2 }
=
i \sqrt{ -p^2 - m^2 }
\end{dmath}
and along contour \( D \)
\begin{dmath}\label{eqn:ProblemSet2Problem1:220}
(p^2 + m^2)^{1/2}
=
\lr{ e^{3 i \pi} \lr{ -p^2 - m^2} }^{1/2}
=
e^{3 i \pi/2} \sqrt{ -p^2 - m^2 }
=
-i \sqrt{ -p^2 - m^2 }.
\end{dmath}
Specifically
\begin{dmath}\label{eqn:ProblemSet2Problem1:240}
\int_A
= -\int_B - \int_C
=
\frac{i}{2 (2\pi)^2 r}
\int_{i \infty}^{i m} dp \frac{p}{i \sqrt{-p^2 - m^2}} e^{i p r}
+
\frac{i}{2 (2\pi)^2 r}
\int_{i m}^{i \infty} dp \frac{p}{-i \sqrt{-p^2 - m^2}} e^{i p r}
=
-\frac{1}{(2\pi)^2 r}
\int_{i m}^{i \infty} dp \frac{p}{\sqrt{-p^2 - m^2}} e^{i p r}.
\end{dmath}
Changing the integration variable to \( q \in [m, \infty] \) (i.e. \( p = i q \)), we have
\begin{dmath}\label{eqn:ProblemSet2Problem1:260}
D(0, r \zcap)
=
\frac{1}{(2\pi)^2 r}
\int_{m}^{\infty} dq \frac{q}{\sqrt{q^2 - m^2}} e^{-q r}
=
\frac{1}{(2\pi)^2 r^2}
\int_{m}^{\infty} r dq \frac{r q}{r \sqrt{q^2 - m^2}} e^{-q r}
=
\frac{1}{(2\pi)^2 r^2}
\int_{r m}^{\infty} dx \frac{x}{\sqrt{x^2 - (r m)^2}} e^{-x}
=
\frac{1}{(2\pi)^2 r} K_1(r m)
\sim
r^{-3/2}
e^{-r m }.
\end{dmath}
where the integral was evaluated with Mathematica, and the asymptotic approximation is from \citep{abramowitz1964handbook} \S 9.7.2 (\( K_\nu(z) \sim \sqrt{\pi/2z} e^{-z} \).)  For \( r \gg 1/m \), this goes to zero quickly.
}}
