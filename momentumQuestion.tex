%
% Copyright � 2018 Peeter Joot.  All Rights Reserved.
% Licenced as described in the file LICENSE under the root directory of this GIT repository.
%
%{
\input{../latex/blogpost.tex}
\renewcommand{\basename}{momentumQuestion}
%\renewcommand{\dirname}{notes/phy1520/}
\renewcommand{\dirname}{notes/ece1228-electromagnetic-theory/}
%\newcommand{\dateintitle}{}
%\newcommand{\keywords}{}

\input{../latex/peeter_prologue_print2.tex}

\usepackage{peeters_layout_exercise}
\usepackage{peeters_braket}
\usepackage{peeters_figures}
\usepackage{siunitx}
\usepackage{verbatim}
%\usepackage{mhchem} % \ce{}
%\usepackage{macros_bm} % \bcM
%\usepackage{macros_qed} % \qedmarker
%\usepackage{txfonts} % \ointclockwise

\beginArtNoToc

\generatetitle{XXX}
%\chapter{XXX}
%\label{chap:momentumQuestion}

I've been going through notes from Stefan and Emily for the last Monday class that I missed.  I tried deriving the following claimed result for the
creation and anhillation form of the momentum operator:
\begin{equation*}
\BP = \sum_{s = 1}^2
\int \frac{d^3 q}{(2\pi)^3} \Bp \lr{
   a_\Bp^{s\dagger}
   a_\Bp^{s}
   +
   b_\Bp^{s\dagger}
   b_\Bp^{s}
}.
\end{equation*}

I started with the To show this, we can plug in the field representations (\citep{peskin1995introduction} eq. 3.99, 3.100)
\begin{equation*}
\begin{aligned}
\Psi(x) &= \int \frac{d^3 p}{(2 \pi)^3 \sqrt{2 \omega_\Bp}} \sum_{s = 1}^2
\lr{ a_\Bp^s u^s(p) e^{-i p \cdot x} + b_\Bp^{s \dagger} v^s(p) e^{i p \cdot x} } \\
\overbar{\Psi}(x) &= \int \frac{d^3 q}{(2 \pi)^3 \sqrt{2 \omega_\Bq}} \sum_{r = 1}^2
\lr{ b_\Bq^r \overbar{v}^r(q) e^{-i q \cdot x} + a_\Bq^{r \dagger} \overbar{u}^r(q) e^{i q \cdot x} },
\end{aligned}
\end{equation*}
to find
\begin{equation*}
\begin{aligned}
P^k
&=
-i \int \Psi^\dagger \partial^k \Psi \\
&=
-i \int \frac{d^3 x d^3 p d^3 q}{(2 \pi)^6 \sqrt{ 2 \omega_\Bp 2 \omega_\Bq }} \sum_{r,s = 1}^2
\lr{
   b_\Bq^r
   \overbar{v}^r(q)
   e^{-i q \cdot x}
+
   a_\Bq^{r \dagger}
   \overbar{u}^r(q)
   e^{i q \cdot x}
}
\gamma^0
(i p^k)
\lr{
-
   a_\Bp^s
   u^s(p)
   e^{-i p \cdot x}
+
   b_\Bp^{s \dagger}
   v^s(p)
   e^{i p \cdot x}
} \\
&=
\int \frac{d^3 p}{ (2 \pi)^3 2 \omega_\Bp }
p^k
\sum_{r,s = 1}^2
\biglr{
-  b_{-\Bq}^r
   a_\Bp^s
   \overbar{v}^r(-\Bp)
   \gamma^0
   u^s(\Bp)
   e^{-2 i \omega_\Bp t}
+
   a_{-\Bq}^{r \dagger}
   b_\Bp^{s \dagger}
   \overbar{u}^r(-\Bp)
   \gamma^0
   v^s(\Bp)
   e^{2 i \omega_\Bp t} \\
&\qquad
+
   b_\Bp^r
   b_\Bp^{s \dagger}
   \overbar{v}^r(\Bp)
   \gamma^0
   v^s(\Bp)
-
   a_\Bp^{r \dagger}
   a_\Bp^s
   \overbar{u}^r(\Bp)
   \gamma^0
   u^s(\Bp)
}.
\end{aligned}
\end{equation*}
The frequency dependent cross terms are killed by the orthogonality conditions
%FIXME: \cref{thm:qftLecture21:1261}
\( v^{r \dagger}(-\Bp) u^s(\Bp) = u^{r\dagger}(\Bp) v^s(-\Bp) = 0\)
%\footnote{For the KG field we had to work much harder to argue those cross terms away in the momentum operator.}
, and the rest simplify using
%FIXME: \cref{eqn:qftLecture21:1200}
\( u^{r\dagger}(\Bp) u^s(\Bp) = v^{r\dagger}(\Bp) v^s(\Bp) = 2 \omega_\Bp \delta^{sr} \), leaving
\begin{equation*}
P^k
=
\int \frac{d^3 p}{ (2 \pi)^3 }
p^k
\sum_{r = 1}^2
\lr{
\lr{
-
   b_\Bp^{r \dagger}
   b_\Bp^r
+
(2\pi)^3 \delta^{(3)}(0)
}
-
   a_\Bp^{r \dagger}
   a_\Bp^r
}.
\end{equation*}

After discarding the vacuum energy term, I have the right functional form, but am off by a sign that I'm having trouble finding.

%}
\EndArticle
