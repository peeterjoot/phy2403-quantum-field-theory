%
% Copyright � 2018 Peeter Joot.  All Rights Reserved.
% Licenced as described in the file LICENSE under the root directory of this GIT repository.
%
\makeoproblem{The Goldstone boson scattering cross-section, its growth with $E_{c.m.}$, and the Higgs}{qft:problemSet4:3}{2018 HW4.III}{

{\flushleft{This}} problem has:
\begin{itemize}
\item A great historical significance, for giving an argument in favor of the  existence of a Higgs particle. The strongest argument for the Higgs particle's existence was that it was required---within the  weakly coupled scenario of electroweak symmetry breaking---to tame the growth of the $WW$ scattering amplitude and restore unitarity of the electroweak theory.  Unitarity is a sacred thing and we don't want to easily give it up.
\item A great future significance: measurements of $WW$ scattering at the LHC (and future colliders) will test the Higgs model precisely, in particular the hypothesis that the Higgs particle that was found last year {\it completely} restores unitarity and there is no other state required. Current measurements of $WW$ scattering at the LHC are not just not complete, they are nonexistent (and are very difficult, I am told), hence the  question of whether ``the Higgs is {\it the} Higgs" is still open.
\end{itemize}

{\flushleft{N}}ow, to the concrete stuff:

\makesubproblem{}{qft:problemSet4:3a}
You will calculate the scattering amplitude of Goldstone boson quanta via Higgs exchange, due to the coupling you found in eq. (1) of Problem 2. To be definite, study the amplitude ${M}(\phi^1 \phi^1 \rightarrow \phi^3 \phi^3)$ (I am being very nice here, as I let you only look at the $s$-channel process!).

For energies of the $\phi^a$ quanta greater than the mass of the $W$ and $Z$ bosons (roughly $100$ GeV), this scattering amplitude via $h$-exchange can be shown [you got to believe me here] to be the same as the scattering of {\it longitudinal} $W, Z$-bosons.

Show that
\begin{equation}
\label{g2}
{M}(\phi^1 \phi^1 \rightarrow \phi^3 \phi^3)\big\vert_{ h-exchange} = const. {s^2 \over v^2 (s - m_h^2)}~,
\end{equation}
where $s$ is the appropriate Mandelstam variable (the square of the c.m. energy), $m_h$ is the mass of $h$, $v= |m|/\sqrt\lambda$, and you will determine the constant.
What you found is that the scattering amplitude (\ref{g2}) grows with the c.m. energy, without bound. It should intuitively clear that this may violate unitarity by leading to probabilities greater then unity at sufficiently high energies.\footnote{Showing this more precisely---and putting bounds on the mass on the Higgs from unitarity---requires study of partial wave decomposition (which is also widely used in quantum mechanics; while the idea is the same, it gets technically a bit more messy in QFT), which is left for future studies.}
\makesubproblem{}{qft:problemSet4:3b}
Now, the interesting thing about the Higgs model is that the growth of (\ref{g2}) with center of mass energy is actually cancelled by the same amplitude, but now due to the direct coupling between $\phi^a$ quanta. To find these interactions, go to eq.~(9) of Homework 2 and study the coupling of $\phi^a$: substitute $H(x)$ of eq. (9) into eq. (5) and find the coupling between four $\phi$-quanta that gives the leading  contribution to the ${M}(\phi^1 \phi^1 \rightarrow \phi^3 \phi^3)\big\vert_{local \; \phi-interaction}$ scattering amplitude. Show that it has the form:
\begin{equation}
\label{g3}
const \;  \phi^c \phi^d  \partial_\mu \phi^a \partial^\mu \phi^b \; {\rm Tr}\left(\sigma^c \sigma^d \sigma^a \sigma^b\right)~,
\end{equation}
and determine the constant.
\makesubproblem{}{qft:problemSet4:3c}
Finally, use (\ref{g3}) to calculate ${M}(\phi^1 \phi^1 \rightarrow \phi^3 \phi^3)\big\vert_{local \; \phi-interaction}$ and show that, when added to ${M}(\phi^1 \phi^1 \rightarrow \phi^3 \phi^3)\big\vert_{ h-exchange}$, the various constants combine such that the  amplitude ${M}(\phi^1 \phi^1 \rightarrow \phi^3 \phi^3)\big\vert_{ h-exchange +local \; \phi-interaction }$ does not grow with the center of mass energy. Hence, in the Higgs model of Homework 2  unitary (as expected) rules.

{\flushleft{T}}he discovery of the Higgs---expected from such theoretical arguments---is a strong evidence in favor of the recent statement:

\smallskip

{\tt "Quantum field theory is how the world works." -Ed Witten} (NYT, August 12 2013)
} % makeproblem

\makeanswer{qft:problemSet4:3}{
\withproblemsetsParagraph{
\makeSubAnswer{}{qft:problemSet4:3a}
TODO.
\makeSubAnswer{}{qft:problemSet4:3b}
TODO.
\makeSubAnswer{}{qft:problemSet4:3c}
TODO.
}
}
