%
% Copyright © 2018 Peeter Joot.  All Rights Reserved.
% Licenced as described in the file LICENSE under the root directory of this GIT repository.
%
%{
\makeproblem{\( U(T, t_0) U(t_0, -T) \) }{problem:qftLecture15Problems:1}{
Show that
\begin{equation*}
U(T, t_0) U(t_0, -T) = U(T, -T).
\end{equation*}
} % problem

\makeanswer{problem:qftLecture15Problems:1}{
We can see that from
\begin{equation}\label{eqn:qftLecture15Problems:160}
\begin{aligned}
U(T, t_0) &= e^{i H_0(T - t_0)} e^{-i H(T - t_0)} \cancel{e^{-i H_0(t_0 - t_0)}}  \\
U(t_0, -T) &= \cancel{e^{i H_0(t_0 - t_0)}} e^{-i H(t_0 - -T)} e^{-i H_0(-T - t_0)},
\end{aligned}
\end{equation}
so
\begin{equation}\label{eqn:qftLecture15Problems:180}
\begin{aligned}
U(T, t_0)
U(t_0, -T)
&=
e^{i H_0(T - t_0)} e^{-i H(T - t_0)} e^{-i H(t_0 + T)} e^{-i H_0(-T - t_0)}
\\&=
e^{i H_0(T - t_0)} e^{-i H 2 T } e^{-i H_0(-T - t_0)},
\end{aligned}
\end{equation}
whereas
\begin{equation}\label{eqn:qftLecture15Problems:260}
\begin{aligned}
U(T, -T)
  &= e^{i H_0(T - t_0)} e^{-i H(T - -T)} e^{-i H_0(-T - t_0)}
\\&= e^{i H_0(T - t_0)} e^{-i H 2 T} e^{-i H_0(-T - t_0)}.
\end{aligned}
\end{equation}
} % answer
%}
