%
% Copyright � 2018 Peeter Joot.  All Rights Reserved.
% Licenced as described in the file LICENSE under the root directory of this GIT repository.
%
%{
%%\input{../latex/blogpost.tex}
%%\renewcommand{\basename}{qftLecture17}
%%\renewcommand{\dirname}{notes/phy2403/}
%%\newcommand{\keywords}{PHY2403H}
%%\input{../latex/peeter_prologue_print2.tex}
%%
%%%\usepackage{phy2403}
%%\usepackage{peeters_braket}
%%%\usepackage{peeters_layout_exercise}
%%\usepackage{peeters_figures}
%%\usepackage{mathtools}
%%\usepackage{siunitx}
%%\usepackage{macros_cal} % LL
%%
%%\newcommand{\ultensor}[3]{{{#1}^{#2}}_{#3}}
%%\newcommand{\deltathree}[0]{\delta^{(3)}}
%%\newcommand{\deltafour}[0]{\delta^{(4)}}
%%
%%\beginArtNoToc
%%\generatetitle{PHY2403H Quantum Field Theory.  Lecture 17: Scattering, decay, cross sections in a scalar theory.  Taught by Prof.\ Erich Poppitz}
%\chapter{Scattering, decay, cross sections in a scalar theory.}
\index{scattering}
\index{decay}
\index{cross sections}
\label{chap:qftLecture17}
%%
%%%%Peeter's lecture notes from class.  These may be incoherent and rough.
%%%%
%%%%These are notes for the UofT course PHY2403H, Quantum Field Theory, taught by Prof. Erich Poppitz, covering \textchapref{{1}} \citep{peskin1995introduction} content.
%%
%%\paragraph{DISCLAIMER: Very rough notes from class, with some additional side notes.}
%%
%%These are notes for the UofT course PHY2403H, Quantum Field Theory, taught by Prof. Erich Poppitz, fall 2018.
%%%, covering \textchapref{{1}} \citep{peskin1995introduction} content.

\section{Review: S-matrix.}

We defined an \( S-\)matrix
\begin{equation}\label{eqn:qftLecture17:20}
\bra{f} S \ket{i} = S_{fi} = \lr{ 2 \pi }^4 \deltafour \lr{ \sum \lr{p_i - \sum_{p_f} } } i M_{fi},
\end{equation}
where
\begin{equation}\label{eqn:qftLecture17:40}
i M_{fi} = \sum \text{ all connected amputated Feynman diagrams }.
\end{equation}
The matrix element \( \bra{f} S \ket{i} \) is the amplitude of the transition from the initial to the final state.
In general this can get very complicated, as the number of terms grows factorially with the order.

We also talked about decays.
\section{Scattering in a scalar theory.}
\index{scattering!scalar theory}
Suppose that we have a scalar theory with a light field \( \Phi, M \) and a heavy field \( \varphi, m \), where \( m > 2 M \).  Perhaps we have an interaction with a \( z^2 \) symmetry so that the interaction potential is quadratic in \( \Phi \)
\begin{dmath}\label{eqn:qftLecture17:60}
V_{\text{int}} = \mu \varphi \Phi \Phi.
\end{dmath}
We may have \( \Phi \Phi \rightarrow \Phi \Phi \) scattering.

We will denote diagrams using a double line for \( \phi \) and a single line for \( \Phi \), as sketched in
\cref{fig:qftLecture17:qftLecture17Fig1}.
\imageFigure{../figures/phy2403-quantum-field-theory/qftLecture17Fig1}{Particle line convention.}{fig:qftLecture17:qftLecture17Fig1}{0.2}

There are three possible diagrams:
\imageThreeFiguresOneLine{../figures/phy2403-quantum-field-theory/qftLecture17Fig2a}{../figures/phy2403-quantum-field-theory/qftLecture17Fig2b}{../figures/phy2403-quantum-field-theory/qftLecture17Fig2c}{Possible diagrams.}{fig:qftLecture17:qftLecture17Fig2}{scale=0.2}
%\cref{fig:qftLecture17:qftLecture17Fig2a}.
%\imageFigure{../figures/phy2403-quantum-field-theory/qftLecture17Fig2a}{CAPTION: qftLecture17Fig2a}{fig:qftLecture17:qftLecture17Fig2a}{0.3}
%\cref{fig:qftLecture17:qftLecture17Fig2b}.
%\imageFigure{../figures/phy2403-quantum-field-theory/qftLecture17Fig2b}{CAPTION: qftLecture17Fig2b}{fig:qftLecture17:qftLecture17Fig2b}{0.3}
%\cref{fig:qftLecture17:qftLecture17Fig2c}.
%\imageFigure{../figures/phy2403-quantum-field-theory/qftLecture17Fig2c}{CAPTION: qftLecture17Fig2c}{fig:qftLecture17:qftLecture17Fig2c}{0.3}

The first we will call the s-channel, which has amplitude
\begin{dmath}\label{eqn:qftLecture17:80}
A(\text{s-channel}) \sim \frac{i}{p^2 - m^2 + i \epsilon} =
\frac{i}{s - m^2 + i \epsilon},
\end{dmath}
where we designate the total squared four-momentum as
\begin{dmath}\label{eqn:qftLecture17:100}
(p_1 + p_2)^2 = s.
\end{dmath}
In the centre of mass frame
\begin{dmath}\label{eqn:qftLecture17:120}
\Bp_1 = - \Bp_2,
\end{dmath}
so
\begin{equation}\label{eqn:qftLecture17:140}
s = \lr{ p_1^0 + p_2^0 }^2 = E_{\text{cm}}^2.
\end{equation}

To the next order we have a diagram like
\cref{fig:qftLecture17:qftLecture17Fig3}.
\imageFigure{../figures/phy2403-quantum-field-theory/qftLecture17Fig3}{Higher order.}{fig:qftLecture17:qftLecture17Fig3}{0.15}
and can have additional virtual particles created, with diagrams like \cref{fig:qftLecture17:qftLecture17Fig4}.
\imageFigure{../figures/phy2403-quantum-field-theory/qftLecture17Fig4}{More virtual particles.}{fig:qftLecture17:qftLecture17Fig4}{0.1}

We will see (QFT II) that this leads to an addition imaginary \( i \Gamma \) term in the propagator
\begin{dmath}\label{eqn:qftLecture17:160}
\frac{i}{s - m^2 + i \epsilon}
\rightarrow
\frac{i}{s - m^2 - i m \Gamma + i \epsilon}.
\end{dmath}
If we choose to zoom into the such a figure, as sketched in
\cref{fig:qftLecture17:qftLecture17Fig5},
we find that it contains the interaction of interest for our diagram, so we can
(looking forward to currently unknown material) know that our diagram also has such an imaginary \( i \Gamma \) term in its
propagator.
\imageFigure{../figures/phy2403-quantum-field-theory/qftLecture17Fig5}{Zooming into the diagram for a higher order virtual particle creation event.}{fig:qftLecture17:qftLecture17Fig5}{0.2}

Assuming such a term, the squared amplitude becomes
\begin{dmath}\label{eqn:qftLecture17:180}
\evalbar{\sigma}{\mbox{\(s\) near \(m^2\)}}
\sim
\Abs{A_s}^2 \sim \inv{(s - m^2)^2 + m^2 \Gamma^2}
\end{dmath}

This is called a resonance, and is sketched in
\cref{fig:qftLecture17:qftLecture17Fig6}.
\imageFigure{../figures/phy2403-quantum-field-theory/qftLecture17Fig6}{Resonance.}{fig:qftLecture17:qftLecture17Fig6}{0.2}

Where are the poles of the modified propagator?

\begin{dmath}\label{eqn:qftLecture17:220}
\frac{i}{s - m^2 - i m \Gamma + i \epsilon}
=
\frac{i}{p_0^2 - \Bp^2 - m^2 - i m \Gamma + i \epsilon}
\end{dmath}

The pole is found, neglecting \( i \epsilon \), is found at
\begin{dmath}\label{eqn:qftLecture17:200}
p_0 = \sqrt{ \omega_\Bp^2 + i m \Gamma }
= \omega_\Bp \sqrt{ 1 + \frac{i m \Gamma }{\omega_\Bp^2} }
\approx \omega_\Bp + \frac{i m \Gamma }{2 \omega_\Bp}
\end{dmath}

\section{Decay rates.}
\index{decay rate}

We have an initial state
\begin{dmath}\label{eqn:qftLecture17:240}
\ket{i} = \ket{k},
\end{dmath}
and final state
\begin{dmath}\label{eqn:qftLecture17:260}
\ket{f} = \ket{p_1^f, p_2^f \cdots p_n^f}.
\end{dmath}
We defined decay rate as the ratio of the number of initial particles to the number of final particles.

The probability is proportional to
\begin{dmath}\label{eqn:qftLecture17:280}
\rho \sim \Abs{\bra{f} S \ket{i}}^2
=
(2 \pi)^4 \deltafour( p_{\text{in}} - \sum p_f )
(2 \pi)^4 \deltafour( p_{\text{in}} - \sum p_f )
\times \Abs{ M_{fi} }^2,
\end{dmath}
where the proportionality is because we will have to divide by all the norms of the final states\footnote{Required for the probabillity to be no greater than one.}.

Saying that \( \delta(x) f(x) = \delta(x) f(0) \) we can set the argument of one of the delta functions to zero, which gives us a vacuum volume element factor
\begin{equation}\label{eqn:qftLecture17:300}
(2 \pi)^4
\deltafour( p_{\text{in}} - \sum p_f )  =
(2 \pi)^4
\deltafour( 0 )
= V_3 T,
\end{equation}
so
\begin{dmath}\label{eqn:qftLecture17:320}
\frac{\text{probability for \( i \rightarrow f\) }}{\text{unit time}}
\sim
(2 \pi)^4 \deltafour( p_{\text{in}} - \sum p_f )
V_3
\times \Abs{ M_{fi} }^2,
\end{dmath}
%
For the norms, we use the relativistic normalization
\begin{dmath}\label{eqn:qftLecture17:360}
\braket{k}{p} = (2 \pi)^3 2 \omega_\Bp \deltathree(\Bp - \Bk)
\end{dmath}
and our volume element interpretation of \( \deltathree(0) \)\footnote{Originally seen in \cref{eqn:qftLecture6:420}.}, which is
%\begin{dmath}\label{eqn:qftLecture17:340}
%\braket{\Bk}{\Bk} = 2 \omega_\Bk V_3
%\end{dmath}
%so
\begin{dmath}\label{eqn:qftLecture17:380}
\braket{p}{p}
= 2 \omega_\Bp \int d^3 x \evalbar{e^{i \Bp \cdot \Bx}}{\Bx = 0}
= 2 \omega_\Bp V_3.
\end{dmath}
We now have the full expression for the probability per unit time
\begin{dmath}\label{eqn:qftLecture17:400}
\frac{\text{probability for \(i \rightarrow f\)}}{\text{unit time}}
=
\frac{
(2 \pi)^4 \deltafour( p_{\text{in}} - \sum p_f )
\Abs{ M_{fi} }^2 V_3
}
{
2 \omega_\Bk V_3
2 \omega_{\Bp_1}
\cdots
2 \omega_{\Bp_n} V_3^n
}
\end{dmath}

\paragraph{In terms of number of states in a small momentum space volume.}
If we multiply the number of final states with \( p_i^f \in (p_i^f, p_i^f + dp_i^f) \) for a particle in a box
\begin{dmath}\label{eqn:qftLecture17:420}
p_x = \frac{ 2 \pi n_x}{L}
\end{dmath}

\begin{dmath}\label{eqn:qftLecture17:440}
\Delta p_x = \frac{ 2 \pi }{L} \Delta n_x
\end{dmath}

\begin{dmath}\label{eqn:qftLecture17:460}
\Delta n_x
=
\frac{L}{2 \pi} \Delta p_x
\end{dmath}

and

\begin{dmath}\label{eqn:qftLecture17:480}
\Delta n_x
\Delta n_y
\Delta n_z
= \frac{V_3}{(2 \pi)^3 }
\Delta p_x
\Delta p_y
\Delta p_z
\end{dmath}

\begin{dmath}\label{eqn:qftLecture17:500}
\Gamma
=
\frac{\text{number of events \( i \rightarrow f \)}}{\text{unit time}}
=
\prod_{f} \frac{ d^3 p}{(2 \pi)^3 2 \omega_{\Bp^f} }
 \frac{ (2 \pi)^4 \deltafour( k - \sum_f p^f ) \Abs{M_{fi}}^2 }
{
2 \omega_{\Bk}
}
\end{dmath}

Note that everything here is Lorentz invariant except for the denominator of the second term ( \(2 \omega_{\Bk}\)).  This is a well known result (the decay rate changes in different frames).

\section{Cross section.}
\index{cross section}

For \( 2 \rightarrow \text{many} \) transitions

\begin{dmath}\label{eqn:qftLecture17:520}
\frac{\text{probability \( i \rightarrow f \)}}{\text{unit time}}
\times \lr{
\text{ number of final states with \( p_f \in (p_f, p_f + dp_f) \)
}
}
=
 \frac{ (2 \pi)^4 \deltafour( \sum p_i - \sum_f p^f ) \Abs{M_{fi}}^2 \cancel{V_3} }
{
2 \omega_{\Bk_1} V_3
2 \omega_{\Bk_2} \cancel{V_3 }
}
\prod_{f} \frac{ d^3 p}{(2 \pi)^3 2 \omega_{\Bp^f} }
\end{dmath}

We need to divide by the flux to obtain the cross section.

In the CM frame, as sketched in
\cref{fig:qftLecture17:qftLecture17Fig7},
the current is
\begin{dmath}\label{eqn:qftLecture17:540}
\Bj = n \Bv_1 - n \Bv_2,
\end{dmath}
so if the density is
\begin{dmath}\label{eqn:qftLecture17:560}
n = \inv{V_3},
\end{dmath}
(one particle in \(V_3\)), then
\begin{dmath}\label{eqn:qftLecture17:580}
\Bj = \frac{\Bv_1 - \Bv_2}{V_3}.
\end{dmath}
\imageFigure{../figures/phy2403-quantum-field-theory/qftLecture17Fig7}{Centre of mass frame.}{fig:qftLecture17:qftLecture17Fig7}{0.1}

\begin{dmath}\label{eqn:qftLecture17:640}
\sigma
=
\frac
{
   (2 \pi)^4 \deltafour( \sum p_i - \sum_f p^f ) \Abs{M_{fi}}^2
}
{
   2 \omega_{\Bk_1}
   2 \omega_{\Bk_2}
   \Norm{\Bv_1 - \Bv_2}
}
\prod_{f} \frac{ d^3 p}{(2 \pi)^3 2 \omega_{\Bp^f} }
\end{dmath}

This is where \citep{peskin1995introduction} stops.

There is, however, a nice Lorentz invariant generalization
\begin{dmath}\label{eqn:qftLecture17:600}
j = \inv{ V_3 \omega_{k_A} \omega_{k_B}} \sqrt{ (k_A k_B)^2 - m_A^2 m_B^2 }
\end{dmath}

(Claim: DIY)
\begin{dmath}\label{eqn:qftLecture17:620}
\evalbar{j}{CM} =
\inv{V_3}
\lr{
   \frac{\Norm{\Bk_A}}{\omega_{k_A}}
   +
   \frac{\Norm{\Bk_B}}{\omega_{k_B}}
}
=
\inv{V_3} \lr{ \Norm{\Bv_A} + \Norm{\Bv_B} }
=
\inv{V_3} \Norm{\Bv_1 - \Bv_2 }
\end{dmath}

\begin{dmath}\label{eqn:qftLecture17:660}
\sigma
=
 \frac
{
   (2 \pi)^4 \deltafour( \sum p_i - \sum_f p^f ) \Abs{M_{fi}}^2
}
{
   4 \sqrt{ (k_A k_B)^2 - m_A^2 m_B^2 }
}
\prod_{f} \frac{ d^3 p}{(2 \pi)^3 2 \omega_{\Bp^f} }.
\end{dmath}

%}
%\EndArticle
