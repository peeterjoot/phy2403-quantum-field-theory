%
% Copyright � 2017 Peeter Joot.  All Rights Reserved.
% Licenced as described in the file LICENSE under the root directory of this GIT repository.
%
%{
\input{../latex/blogpost.tex}
\renewcommand{\basename}{qftLecture17}
\renewcommand{\dirname}{notes/phy2403/}
\newcommand{\keywords}{PHY2403H}
\input{../latex/peeter_prologue_print2.tex}

%\usepackage{phy2403}
\usepackage{peeters_braket}
%\usepackage{peeters_layout_exercise}
\usepackage{peeters_figures}
\usepackage{mathtools}
\usepackage{siunitx}
\usepackage{macros_cal} % LL

\newcommand{\ultensor}[3]{{{#1}^{#2}}_{#3}}
\newcommand{\deltathree}[0]{\delta^{(3)}}
\newcommand{\deltafour}[0]{\delta^{(4)}}

\beginArtNoToc
\generatetitle{PHY2403H Quantum Field Theory.  Lecture 17: XXX.  Taught by Prof.\ Erich Poppitz}
%\chapter{XXX}
\label{chap:qftLecture17}

\paragraph{Disclaimer}

%%Peeter's lecture notes from class.  These may be incoherent and rough.
%%
%%These are notes for the UofT course PHY2403H, Quantum Field Theory, taught by Prof. Erich Poppitz, covering \textchapref{{1}} \citep{peskin1995introduction} content.

\paragraph{DISCLAIMER: Very rough notes from class, with some additional side notes.}

These are notes for the UofT course PHY2403H, Quantum Field Theory, taught by Prof. Erich Poppitz, fall 2018.
%, covering \textchapref{{1}} \citep{peskin1995introduction} content.

\section{Review: S-matrix}

We defined an \( S-\)matrix
\begin{dmath}\label{eqn:qftLecture17:20}
\bra{f} S \ket{i} = S_{fi} = \lr{ 2 \pi }^4 \deltafour \lr{ \sum_{p_i - \sum_{p_f} } } i M_{fi},
\end{dmath}
where
\begin{dmath}\label{eqn:qftLecture17:40}
i M_{fi} = \sum \text{ all connected amputated Feynman diagrams }.
\end{dmath}
The matrix element \( \bra{f} S \ket{i} \) is the amplitude of the transition from the initial to the final state.
In general this can get very complicated, as the number of terms grows factorially with the order.

We also talked about decays.

Suppose that we have a scalar theory with a light field \( \Phi, M \) and a heavy field \( \varphi, m \), where \( m > 2 M \).  Perhaps we have an interaction with a \( z^2 \) symmetry so that the interaction potential is quadratic in \( \Phi \)
\begin{dmath}\label{eqn:qftLecture17:60}
V_{\text{int}} = \mu \varphi \Phi \Phi.
\end{dmath}
We may have \( \Phi \Phi \rightarrow \Phi \Phi \) scattering.

We will denote diagrams using a double line for \( \phi \) and a single line for \( \Phi \), as sketched in
F1.

There are three possible diagrams:
F2

The first we will call the s-channel, which has amplitude

\begin{dmath}\label{eqn:qftLecture17:80}
A(\text{s-channel}) \sim \frac{i}{p^2 - m^2 + i \epsilon} =
\frac{i}{s - m^2 + i \epsilon}
\end{dmath}

\begin{dmath}\label{eqn:qftLecture17:100}
(p_1 + p_2)^2 = s
\end{dmath}
In the center of mass frame
\begin{dmath}\label{eqn:qftLecture17:120}
\Bp_1 = - \Bp_2,
\end{dmath}
so
\begin{dmath}\label{eqn:qftLecture17:140}
s = \lr{ p_1^0 + p_2^0 }^2 = E_{\text{cm}}^2.
\end{dmath}

To the next order we have
F3
and can have additional virtual particles created, with diagrams like
F4

We will see (QFT II) that this leads to an addition term in the propagator
\begin{dmath}\label{eqn:qftLecture17:160}
\frac{i}{s - m^2 + i \epsilon}
\rightarrow
\frac{i}{s - m^2 - i m \Gamma + i \epsilon}
\end{dmath}
The squared amplitude becomes
\begin{dmath}\label{eqn:qftLecture17:180}
\evalbar{\sigma}{s \text{near} m^2}
\sim
\Abs{A_s}^2 \sim \inv{(s - m)^2 + m^2 \Gamma^2}
\end{dmath}

This is called a resonance (name?), and is sketched in
F6

Where are the poles of the modified propagator?

\begin{dmath}\label{eqn:qftLecture17:220}
\frac{i}{s - m^2 - i m \Gamma + i \epsilon}
=
\frac{i}{p_0^2 - \Bp^2 - m^2 - i m \Gamma + i \epsilon}
\end{dmath}

The pole is found, neglecting \( i \epsilon \), is found at
\begin{dmath}\label{eqn:qftLecture17:200}
p_0 = \sqrt{ \omega_\Bp^2 + i m \Gamma }
= \omega_\Bp \sqrt{ 1 + \frac{i m \Gamma }{\omega_\Bp^2} }
\approx \omega_\Bp + \frac{i m \Gamma }{2 \omega_\Bp}
\end{dmath}

\section{Decay rates.}

We have an initial state
\begin{dmath}\label{eqn:qftLecture17:240}
\ket{i} = \ket{k},
\end{dmath}
and final state
\begin{dmath}\label{eqn:qftLecture17:260}
\ket{f} = \ket{p_1^f, p_2^f \cdots p_n^f}.
\end{dmath}
We defined decay rate as the ratio of the number of initial particles to the number of final particles.

The probability is
\begin{dmath}\label{eqn:qftLecture17:280}
\rho \sim \Abs{\bra{f} S \ket{i}}^2
=
(2 \pi)^4 \deltafour( p_{\text{in}} - \sum p_f )
(2 \pi)^4 \deltafour( p_{\text{in}} - \sum p_f )
\times \Abs{ M_{fi} }^2
\end{dmath}

Saying that \( \delta(x) f(x) = \delta(x) f(0) \) we can set
\begin{dmath}\label{eqn:qftLecture17:300}
(2 \pi)^4
\deltafour( p_{\text{in}} - \sum p_f )  =
(2 \pi)^4
\deltafour( 0 )
= V_3 T,
\end{dmath}
so
\begin{dmath}\label{eqn:qftLecture17:320}
\frac{\text{probability for} i \rightarrow f}{\text{unit time}}
\sim
(2 \pi)^4 \deltafour( p_{\text{in}} - \sum p_f )
V_3
\times \Abs{ M_{fi} }^2
\end{dmath}

\begin{dmath}\label{eqn:qftLecture17:340}
\braket{\Bk}{\Bk} = 2 \omega_\Bk V_3
\end{dmath}

coming from

\begin{dmath}\label{eqn:qftLecture17:360}
\braket{k}{p} = (2 \pi)^3 2 \omega_\Bp \deltathree(\Bp - \Bk)
\end{dmath}
so
\begin{dmath}\label{eqn:qftLecture17:380}
\braket{k}{k} = 2 \omega_\Bp V_3
\end{dmath}

\begin{dmath}\label{eqn:qftLecture17:400}
\frac{\text{probability for} i \rightarrow f}{\text{unit time}}
\sim
\frac{
(2 \pi)^4 \deltafour( p_{\text{in}} - \sum p_f )
\Abs{ M_{fi} }^2 V_3
}
{
2 \omega_\Bk V_3
2 \omega_{\Bp_1}
\cdots
2 \omega_{\Bp_n} V_3^n
}
\end{dmath}

If we multiply the number of final states with \( p_i^f \in (p_i^f, p_i^f + dp_i^f) \) for a particle in a box
\begin{dmath}\label{eqn:qftLecture17:420}
p_x = \frac{ 2 \pi n_x}{L}
\end{dmath}

\begin{dmath}\label{eqn:qftLecture17:440}
\Delta p_x = \frac{ 2 \pi }{L} \Delta n_x
\end{dmath}

\begin{dmath}\label{eqn:qftLecture17:460}
\Delta n_x
=
\frac{L}{2 \pi} \Delta p_x
\end{dmath}

and

\begin{dmath}\label{eqn:qftLecture17:480}
\Delta n_x
\Delta n_y
\Delta n_z
= \frac{V_3}{(2 \pi)^3 }
\Delta p_x
\Delta p_y
\Delta p_z
\end{dmath}

\begin{dmath}\label{eqn:qftLecture17:500}
\Gamma
=
\frac{\text{number of events \( i \rightarrow f \)}}{\text{unit time}}
=
\prod_{f} \frac{ d^3 p}{(2 \pi)^3 2 \omega_{\Bp^f} }
 \frac{ (2 \pi)^4 \deltafour( k - \sum_f p^f ) \Abs{M_{fi}}^2 }
{
2 \omega_{\Bk}
}
\end{dmath}

Note that everything here is Lorentz invariant except for the denominator of the second term ( \(2 \omega_{\Bk}\)).  This is a well known result (the decay rate changes in different frames).

\section{Cross section.}

For \( 2 \rightarrow \text{many} \) transitions

\begin{dmath}\label{eqn:qftLecture17:520}
\frac{\text{probability \( i \rightarrow f \)}}{\text{unit time}}
\times \lr{
\text{ number of final states with \( p_f \in (p_f, p_f + dp_f) \)
}
}
=
 \frac{ (2 \pi)^4 \deltafour( \sum p_i - \sum_f p^f ) \Abs{M_{fi}}^2 \cancel{V_3} }
{
2 \omega_{\Bk_1} V_3
2 \omega_{\Bk_2} \cancel{V_3 }
}
\prod_{f} \frac{ d^3 p}{(2 \pi)^3 2 \omega_{\Bp^f} }
\end{dmath}

We need to divide by the flux

In the CM frame

F7

The current is
\begin{dmath}\label{eqn:qftLecture17:540}
\Bj = n \Bv_1 - n \Bv_2,
\end{dmath}
so if the density is
\begin{dmath}\label{eqn:qftLecture17:560}
n = \inv{V_3},
\end{dmath}
(one particle in \(V_3\)), then
\begin{dmath}\label{eqn:qftLecture17:580}
\Bj = \frac{\Bv_1 - \Bv_2}{V_3}
\end{dmath}

This is where \citep{peskin1995introduction} stop,

\begin{dmath}\label{eqn:qftLecture17:640}
\sigma
=
 \frac{ (2 \pi)^4 \deltafour( \sum p_i - \sum_f p^f ) \Abs{M_{fi}}^2 \cancel{V_3} }
{
2 \omega_{\Bk_1}
2 \omega_{\Bk_2}
\Abs{\Bv_1 - \Bv_2}
}
\prod_{f} \frac{ d^3 p}{(2 \pi)^3 2 \omega_{\Bp^f} }
\end{dmath}

There is, however, a nice Lorentz invariant generalization
\begin{dmath}\label{eqn:qftLecture17:600}
j = \inv{ V_3 \omega_{k_A} \omega_{k_B}} \sqrt{ (k_A - k_B)^2 - m_A^2 m_B^2 }
\end{dmath}

\begin{dmath}\label{eqn:qftLecture17:620}
\evalbar{j}{CM} =
\inv{V_3}
\lr{
   \frac{\Abs{\Bk}}{\omega_{k_A}}
   +
   \frac{\Abs{\Bk}}{\omega_{k_B}}
}
=
\inv{V_3} \lr{ \Abs{\Bv_A} + \Abs{\Bv_B} }
=
\inv{V_3} \Abs{\Bv_1 - \Bv_2 }
\end{dmath}

\begin{dmath}\label{eqn:qftLecture17:660}
\sigma
=
 \frac{ (2 \pi)^4 \deltafour( \sum p_i - \sum_f p^f ) \Abs{M_{fi}}^2 \cancel{V_3} }
{
4 \sqrt{ (k_A - k_B)^2 - m_A^2 m_B^2 }
}
\prod_{f} \frac{ d^3 p}{(2 \pi)^3 2 \omega_{\Bp^f} }.
\end{dmath}



%}
\EndArticle
%\EndNoBibArticle
